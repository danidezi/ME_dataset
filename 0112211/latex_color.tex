
\documentclass[a4paper,12pt]{article}
\usepackage{amssymb}
\topmargin=-0.3truein
\textheight=8.9truein
\textwidth=6.65truein
\oddsidemargin=-0.3truein
\evensidemargin=-0.3truein
\usepackage{useful_macros}
\begin{document}
\baselineskip=18pt
\pagenumbering{arabic}
\parskip1.5em
\thispagestyle{empty}
\vskip2.5em
\begin{center}
{\Large{\bf The Seiberg-Witten prepotential and 
the Euler \\ class 
of the reduced moduli space of instantons \myHighlight{$^*$}\coordHE{}}}\\\vskip1.5em
R. Flume\myHighlight{$^{1}$}\coordHE{}\hspace{3em}R. Poghossian\myHighlight{$^{1\,\flat}$}\coordHE{}\hspace{3em}
H. Storch\myHighlight{$^{1}$}\coordHE{} \\ \vskip3em
\myHighlight{$^{1}$}\coordHE{}{\sl Physikalisches Institut der Universit\"at Bonn}\\
{\sl Nu{\ss}allee 12, D--53115 Bonn, Germany}\\
\end{center}
\vskip2em
\begin{abstract}
\noindent
The \myHighlight{$n$}\coordHE{}-instanton contribution to the Seiberg-Witten prepotential 
of \myHighlight{${\bf N}=2$}\coordHE{} supersymmetric \myHighlight{$d=4$}\coordHE{} Yang Mills theory is represented 
as the integral of  the exponential of an 
equivariantly exact form. Integrating out an overall scale and 
a \myHighlight{$U(1)$}\coordHE{} angle the integral is rewritten as \myHighlight{$(4n-3)$}\coordHE{} fold 
product of a closed two form. This two form is, formally, a 
representative of the Euler class of the Instanton moduli space 
viewed as a principal \myHighlight{$U(1)$}\coordHE{} bundle, because its pullback under 
bundel projection is the exterior derivative of an angular 
one-form. 
\end{abstract} 
\vspace{2cm}
\myHighlight{$^{\flat}$}\coordHE{} {\small{on leave of absence from Yerevan Physics 
Institute, Armenia}}\\
{\small{
e-mail: flume@th.physik.uni-bonn.de\\
\hspace*{1.3cm}poghos@th.physik.uni-bonn.de\\
\hspace*{1.3cm}hst@mosaic-ag.com}}
\vfill\eject\setcounter{page}{1}

\section{Introduction}

The conjecture of Seiberg and Witten \cite{sw1} of the exact formula 
for the prepotential of \myHighlight{${\bf N}=2$}\coordHE{}, \myHighlight{$d=4$}\coordHE{} supersymmetric Yang-Mills 
theories has been verified in microscopical instanton calculations for 
contributions of topological charge one \cite{finnel} - \cite{yung} 
and two \cite{dorey1}. 
The propose of the present communication is to report on our 
findings concerning the representations of the general \myHighlight{$n$}\coordHE{}- instanton 
coefficient of the Seiberg - Witten prepotential in algebraic - geometric 
terms.

We do not achieve the determination of the coefficients, 
but we hope that our contribution will be a step into this direction. 
This letter communication contains results, which will be published 
in more extended form elsewhere \cite{fps}.  

We will constrain ourselves to a discussion of the simplest 
possible setting for the appearance of an \myHighlight{${\bf N}=2$}\coordHE{} prepotential 
\myHighlight{${\cal F}$}\coordHE{}. 
That is, we choose \myHighlight{$SU(2)$}\coordHE{} as underlying gauge group (spontaneously 
broken to \myHighlight{$U(1)$}\coordHE{}) and restrict the field content to an \myHighlight{${\bf N}=2$}\coordHE{} 
vector multiplet, denoted in the following by \myHighlight{$\Psi$}\coordHE{} \footnote{\myHighlight{$\Psi$}\coordHE{} 
is assumed to be composed of 
reduced \myHighlight{$U(1)$}\coordHE{}-degrees of freedom due to symmetry breaking. In \myHighlight{$\Psi$}\coordHE{} is 
incorporated the breaking parameter \myHighlight{$v$}\coordHE{} : \myHighlight{$\Psi=v+ \cdots$}\coordHE{}.}. 
The SW proposal for \myHighlight{${\cal F}(\Psi)$}\coordHE{} as a function of \myHighlight{$\Psi$}\coordHE{} amounts 
under this circumstances to the ansatz for \myHighlight{${\cal F}$}\coordHE{} as inverse of 
the  modular elliptic function \myHighlight{$J$}\coordHE{} \cite{erdelyi} with the following 
representation in terms of a power series expansion
\begin{equation}\coord{}\boxEquation{
{\cal F}(\Psi)=\frac{i}{2\pi} \log \Psi^2\log \frac{\Psi^2}{\Lambda^2} 
-\frac{i}{\pi}\sum_{n=1}^{\infty}{\cal F}_n \left(\frac{\Lambda}{\Psi}
\right)^{4n} \Psi^2, 
\label{Fexpansion}
}{
{\cal F}(\Psi)=\frac{i}{2\pi} \log \Psi^2\log \frac{\Psi^2}{\Lambda^2} 
-\frac{i}{\pi}\sum_{n=1}^{\infty}{\cal F}_n \left(\frac{\Lambda}{\Psi}
\right)^{4n} \Psi^2, 
}{ecuacion}\coordE{}\end{equation}   
where \myHighlight{$\Lambda$}\coordHE{} denotes a dynamical scale enforced by ultraviolet 
divergencies. The first (logarithmic) term on the r.h.s. of eq. 
(\ref{Fexpansion}) comprises the perturbative contributions which 
are the classical prepotential \myHighlight{$\sim$}\coordHE{}  \myHighlight{$\Psi^2$}\coordHE{} and the one-loop 
contribution \myHighlight{$\sim$}\coordHE{}  \myHighlight{$\Psi^2 \log \Psi^2$}\coordHE{}. A non-renormalization 
theorem \cite{s2} implies that there are no other perturbative 
(higher loop) contributions to \myHighlight{${\cal F}(\Psi)$}\coordHE{}. 
The second term in the r.h.s. of eq. (\ref{Fexpansion}) with 
numerical coefficients \myHighlight{${\cal F}_n$}\coordHE{} is of nonperturbative origin 
due to instanton configurations. The perturbative and non-perturbative 
pieces add up to the inverse elliptic modular function reflecting 
therewith a dynamically realized (S-) duality. For the arguments 
leading to the ansatz, (\ref{Fexpansion}), which might be called 
macroscopic - including in particular duality and a certain 
minimality assumption (which might be coined as the assumption of minimal 
analytical consistency \cite{flume}) - 
we refer to \cite{sw1}. 

The plan of the paper is as follows. In Sec.~\ref{sec:II} we recall 
part of the pioneering work by Dorey, 
Khoze and Mattis (DKM) \cite{dorey1}, \cite{dorey2}, \cite{dorey3}, 
leading to the determination of the one- and two-instanton 
coefficients \cite{dorey1}  and also providing a suggestive ansatz
for the measure in the \myHighlight{$n$}\coordHE{}-instanton moduli space\cite{dorey3}. It 
will be noted that the DKM measure has to be supplemented by the 
specification of a domain of integration, since the space of 
integration proposed by DKM containing redundant degrees of 
freedom is not orientable. In Sec.~\ref{sec:III} we show that \myHighlight{${\cal F}_n$}\coordHE{} 
can be represented as integral 
of the top form of the exponential of an 
equivariantly exact form.  Sec.~\ref{sec:IV} contains our main 
result, the representation of \myHighlight{${\cal F}_n$}\coordHE{} as integral of a 
\myHighlight{$(4n-3)$}\coordHE{} fold product of a closed two-form. The two form is formally 
a representative of the Euler class associated with the ADHM moduli 
space viewed as a (principal) \myHighlight{$U(1)$}\coordHE{} bundle. 
We conclude the paper in section 5 with 
some comments. 

\section{The DKM results and the necessity of \myHighlight{$SO(n)$}\coordHE{} gauge 
fixing}\label{sec:II}
The coefficients \myHighlight{${\cal F}_n$}\coordHE{} are given in terms of certain ``reduced'' 
matrix elements, that is, expressions in which the parameters 
corresponding to overall translation invariance are eliminated 
together with their supersymmetry partners. These matrix 
elements are represented by integrals over the supersymmetrized 
moduli space of \myHighlight{$n$}\coordHE{}-instanton configurations - the Atiyah, Drinfeld, 
Manin (ADHM) \cite{adhm} parameters - 
\begin{equation}\coord{}\boxEquation{
{\cal F}_n=\int d\mu^{(n)}e^{-S^{(n)}}.
\label{modspaceint}
}{
{\cal F}_n=\int d\mu^{(n)}e^{-S^{(n)}}.
}{ecuacion}\coordE{}\end{equation}
\myHighlight{$d\mu^{(n)}$}\coordHE{} denotes here a particular measure on the \myHighlight{${\cal N}=2$}\coordHE{} 
supersymmetrised ADHM moduli space and \myHighlight{$S^{(n)}$}\coordHE{} stands for an 
effective \myHighlight{${\cal N}=2$}\coordHE{} action. We will quote the results of 
DKM concerning \myHighlight{$d\mu^{(n)}$}\coordHE{} and \myHighlight{$S^{n}$}\coordHE{} referring for detailed 
explanations to the articles \cite{dorey1}, \cite{dorey2} and  
\cite{dorey3}. 

The supersymmetrized ADHM moduli space may be embedded into a 
larger space whose bosonic coordinates are the quaternionic 
entries \footnote{Our conventions concerning quaternions are the following: 
an arbitrary quaternion \myHighlight{$x_{\alpha \dot{\beta}}$}\coordHE{} can be 
decomposed in a basis of unit quaternions \myHighlight{$x_{\alpha \dot{\beta}}= 
\sum_{\mu =1}^4 x_{\mu}\left(e^{\mu}\right)_{\alpha \dot{\beta}}$}\coordHE{}, 
\myHighlight{$x_{\mu}$}\coordHE{} real, \myHighlight{$\left\{e^{\mu}\right\}=-i \sigma^{\mu}$}\coordHE{}, \myHighlight{$\mu =
1,2,3$}\coordHE{}; \myHighlight{$\sigma^4={\bf I}_2$}\coordHE{}; \myHighlight{$\left(\bar{x}\right)^{\dot{\beta}\alpha}
=\epsilon^{\dot{\beta}\dot{\gamma}}\epsilon^{\alpha \delta}x_{\delta 
\dot{\gamma}}$}\coordHE{}; \myHighlight{$\epsilon^{12}=\epsilon^{\dot{1}\dot{2}}=1$}\coordHE{}. 
The notions of imadginary (\myHighlight{$\Im m$}\coordHE{}) and real (\myHighlight{$\Re e$}\coordHE{}) parts, which 
we will use further on, should be understood with respect to this 
conjugation.} of a \myHighlight{$(n+1)\times n$}\coordHE{} matrix, 
\begin{eqnarray}\coord{}\boxAlignEqnarray{\leftCoord{}
a_{\alpha \dot{\beta}}=\left( \begin{array}{ccc}
w_1 & \ldots & w_n \rightCoord{}\\\leftCoord{}
a'_{11} & \ldots & a'_{1n} \rightCoord{}\\
&\leftCoord{}\ldots &  \rightCoord{}\\\leftCoord{}
a'_{n1} & \ldots & a'_{nn}\rightCoord{}
\end{array}\right)_{\alpha \dot{\beta}} , \rightCoord{}\, \rightCoord{}\, \rightCoord{}\, \rightCoord{}\, \rightCoord{}\, a'_{ij}=a'_{ji}, 
\leftCoord{}\rightCoord{}\, \rightCoord{}\, \rightCoord{}\, \rightCoord{}\, \rightCoord{}\, \rightCoord{}\, \rightCoord{}\, \rightCoord{}\, \alpha , \rightCoord{}\,\rightCoord{}\, \dot{\beta}=1,2 \rightCoord{}
\label{amatrix}
\rightCoord{}}{0mm}{5}{22}{
a_{\alpha \dot{\beta}}=\left( \begin{array}{ccc}
w_1 & \ldots & w_n \\
a'_{11} & \ldots & a'_{1n} \\
&\ldots &  \\
a'_{n1} & \ldots & a'_{nn}
\end{array}\right)_{\alpha \dot{\beta}} , \, \, \, \, \, a'_{ij}=a'_{ji}, 
\, \, \, \, \, \, \, \, \alpha , \,\, \dot{\beta}=1,2 
}{1}\coordE{}\end{eqnarray}
and fermionic partners encoded into spinor valued \myHighlight{$(n+1)\times n$}\coordHE{} 
matrices 
\begin{eqnarray}\coord{}\boxAlignEqnarray{\leftCoord{}
{\rightCoord{}\leftCoord{}\cal M}_{\alpha} &=& \left( \begin{array}{ccc}
\mu_1 & \ldots & \mu_n \rightCoord{}\\\leftCoord{}
{\rightCoord{}\leftCoord{}\cal M}'_{11} & \ldots & {\cal M}'_{1n} \rightCoord{}\\
&\leftCoord{}\ldots &  \rightCoord{}\\\leftCoord{}
{\rightCoord{}\leftCoord{}\cal M}'_{n1} & \ldots & {\cal M}'_{nn}\rightCoord{}
\end{array}\right)_{\alpha}, \rightCoord{}\, \rightCoord{}\, \rightCoord{}\, \rightCoord{}\, \rightCoord{}\, 
{\rightCoord{}\leftCoord{}\cal N}_{\alpha}=\left( \begin{array}{ccc}
\nu_1 & \ldots & \nu_n \rightCoord{}\\\leftCoord{}
{\rightCoord{}\leftCoord{}\cal N}'_{11} & \ldots & {\cal N}'_{1n} \rightCoord{}\\
&\leftCoord{}\ldots &  \rightCoord{}\\\leftCoord{}
{\rightCoord{}\leftCoord{}\cal N}'_{n1} & \ldots & {\cal N}'_{nn}\rightCoord{}
\end{array}\right)_{\alpha}, \rightCoord{}\, \rightCoord{}\, \rightCoord{}\, \rightCoord{}\, \rightCoord{}\, \rightCoord{}\, \rightCoord{}\, \rightCoord{}\, \alpha=1,2 \nonumber \rightCoord{}\\\leftCoord{}
{\rightCoord{}\leftCoord{}\cal M}'_{ij} &=& {\cal M}'_{ji}, \rightCoord{}\, \rightCoord{}\, \rightCoord{}\, \rightCoord{}\, \rightCoord{}\, \rightCoord{}\,  
{\rightCoord{}\leftCoord{}\cal N}'_{ij} = {\cal N}'_{ji}.\rightCoord{}
\label{mnmatrix}
\rightCoord{}}{0mm}{16}{39}{
{\cal M}_{\alpha} &=& \left( \begin{array}{ccc}
\mu_1 & \ldots & \mu_n \\
{\cal M}'_{11} & \ldots & {\cal M}'_{1n} \\
&\ldots &  \\
{\cal M}'_{n1} & \ldots & {\cal M}'_{nn}
\end{array}\right)_{\alpha}, \, \, \, \, \, 
{\cal N}_{\alpha}=\left( \begin{array}{ccc}
\nu_1 & \ldots & \nu_n \\
{\cal N}'_{11} & \ldots & {\cal N}'_{1n} \\
&\ldots &  \\
{\cal N}'_{n1} & \ldots & {\cal N}'_{nn}
\end{array}\right)_{\alpha}, \, \, \, \, \, \, \, \, \alpha=1,2 \\
{\cal M}'_{ij} &=& {\cal M}'_{ji}, \, \, \, \, \, \,  
{\cal N}'_{ij} = {\cal N}'_{ji}.
}{1}\coordE{}\end{eqnarray}
The action \myHighlight{$S^{(n)}$}\coordHE{} and the measure \myHighlight{$d\mu^{(n)}$}\coordHE{} have been found by 
DKM to be of the form  
\begin{eqnarray}\coord{}\boxAlignEqnarray{\leftCoord{}
\label{action}
\pi^{-2}S^{(n)}=\frac{\leftCoord{}8n}{\rightCoord{}g^2}+16|v|^2\sum_{\rightCoord{}k=1}^{\leftCoord{}n} |w_k|^2 
\leftCoord{}-8 tr_n \bar{\Lambda}{\bf L}^{-1}\left(\Lambda + \Lambda_{f} \right)  
\leftCoord{}+ 4\sqrt{2} \sum_{\rightCoord{}k=1}^{\leftCoord{}n} \bar{\mu}_k \bar{v} \nu_k ,
\rightCoord{}}{0mm}{6}{5}{
\pi^{-2}S^{(n)}=\frac{8n}{g^2}+16|v|^2\sum_{k=1}^{n} |w_k|^2 
-8 tr_n \bar{\Lambda}{\bf L}^{-1}\left(\Lambda + \Lambda_{f} \right)  
+ 4\sqrt{2} \sum_{k=1}^{n} \bar{\mu}_k \bar{v} \nu_k ,
}{1}\coordE{}\end{eqnarray}
\begin{eqnarray}\coord{}\boxAlignEqnarray{\leftCoord{} 
d\mu^{(n)} =\frac{\leftCoord{}C_n}{\rightCoord{}Vol(O(n))} \prod_{\rightCoord{}i=1}^n d^4 w_i 
d^2 \mu_i d^2 \nu_i \prod_{\rightCoord{}i\le j}d^4 a'_{ij} 
d^2 {\cal M}'_{ij} d^2 {\cal N}'_{ij} 
\prod_{\rightCoord{}i<j}d\left({\cal A}_{tot}\right)_{ij} \nonumber \rightCoord{}\\\leftCoord{} 
\times \prod_{\rightCoord{}i<j}
\delta \left(\left({\bf L} \cdot {\cal A}_{tot}
\leftCoord{}-\Lambda -\Lambda_f\right)_{ij}\right)
\delta^{(3)}\left(\frac{\leftCoord{}1}{\rightCoord{}2}\left( \rightCoord{}
\left(\bar{a}a\right)_{ij}-\left(\bar{a}a\right)_{ji}
\right)\right) \nonumber \rightCoord{}\\\leftCoord{} 
\times \delta^{(2)}\left(\left(\bar{a}{\cal M}\right)_{ij}-
\left(\bar{a}{\cal M}\right)_{ji}\right) 
\delta^{(2)}\left(\left(\bar{a}{\cal N}\right)_{ij}-
\left(\bar{a}{\cal N}\right)_{ji}\right).\rightCoord{}
\label{measure}  
\rightCoord{}}{0mm}{6}{12}{ 
d\mu^{(n)} =\frac{C_n}{Vol(O(n))} \prod_{i=1}^n d^4 w_i 
d^2 \mu_i d^2 \nu_i \prod_{i\le j}d^4 a'_{ij} 
d^2 {\cal M}'_{ij} d^2 {\cal N}'_{ij} 
\prod_{i<j}d\left({\cal A}_{tot}\right)_{ij} \\ 
\times \prod_{i<j}
\delta \left(\left({\bf L} \cdot {\cal A}_{tot}
-\Lambda -\Lambda_f\right)_{ij}\right)
\delta^{(3)}\left(\frac{1}{2}\left( 
\left(\bar{a}a\right)_{ij}-\left(\bar{a}a\right)_{ji}
\right)\right) \\ 
\times \delta^{(2)}\left(\left(\bar{a}{\cal M}\right)_{ij}-
\left(\bar{a}{\cal M}\right)_{ji}\right) 
\delta^{(2)}\left(\left(\bar{a}{\cal N}\right)_{ij}-
\left(\bar{a}{\cal N}\right)_{ji}\right).
}{1}\coordE{}\end{eqnarray}
In (\ref{action}) \myHighlight{$\Lambda$}\coordHE{} and \myHighlight{$\Lambda_f$}\coordHE{} are the \myHighlight{$n\times n$}\coordHE{} 
antisymmetric matrices 
\begin{eqnarray}\coord{}\boxAlignEqnarray{\leftCoord{}  
\Lambda_{kl}&=&\bar{w}_k v w_l - \bar{w}_l v w_k ;\rightCoord{}\\\leftCoord{}
\Lambda_f&=&\frac{\leftCoord{}1}{\rightCoord{}2\sqrt{2}}\left(\bar{{\cal M}}{\rightCoord{}\cal N}- 
\bar{{\cal N}}{\cal M}\right), \rightCoord{}
\rightCoord{}}{0mm}{3}{6}{  
\Lambda_{kl}&=&\bar{w}_k v w_l - \bar{w}_l v w_k ;\\
\Lambda_f&=&\frac{1}{2\sqrt{2}}\left(\bar{{\cal M}}{\cal N}- 
\bar{{\cal N}}{\cal M}\right), 
}{1}\coordE{}\end{eqnarray} 
and \myHighlight{${\bf L}$}\coordHE{} is a real operator acting on any skew symmetric 
\myHighlight{$n\times n$}\coordHE{} matrix \myHighlight{$X$}\coordHE{} by 
\begin{eqnarray}\coord{}\boxAlignEqnarray{\leftCoord{} 
{\rightCoord{}\leftCoord{}\bf L}\cdot X = \frac{\leftCoord{}1}{\rightCoord{}2}\left\{X,W\right\} + \rightCoord{} 
\bar{a}'\left[a',X\right]- \left[\bar{a}',X\right]a', \rightCoord{} 
\label{loperator}
\rightCoord{}}{0mm}{3}{6}{ 
{\bf L}\cdot X = \frac{1}{2}\left\{X,W\right\} +  
\bar{a}'\left[a',X\right]- \left[\bar{a}',X\right]a',  
}{1}\coordE{}\end{eqnarray} 
where \myHighlight{$W$}\coordHE{} is the real valued symmetric \myHighlight{$n\times n$}\coordHE{} matrix
\begin{eqnarray}\coord{}\boxAlignEqnarray{\leftCoord{}
W_{kl}=\bar{w}_k w_l +\bar{w}_l w_k.\rightCoord{}
\rightCoord{}}{0mm}{1}{3}{
W_{kl}=\bar{w}_k w_l +\bar{w}_l w_k.
}{1}\coordE{}\end{eqnarray} 
The measure \myHighlight{$d\mu^{(n)}$}\coordHE{} is according to DKM \cite{dorey3} uniquely determined 
by the requirements of supersymmetry and the cluster property. 
The factor \myHighlight{$1/Vol(O(n))$}\coordHE{} takes care of the fact that 
redundant \myHighlight{$O(n)$}\coordHE{} degrees of freedom have not been discarded 
from the integral. The normalization constant \myHighlight{$C_n$}\coordHE{} is fixed 
through the cluster condition and the normalization of the 
one-instanton coefficient. The last three products of 
\myHighlight{$\delta$}\coordHE{} - functions on the r.h.s. of (\ref{measure}) 
enforce the supersymmetrized ADHM constraints. It can be shown that 
\myHighlight{${\bf L}$}\coordHE{} is invertible for \myHighlight{$a'$}\coordHE{} and 
\myHighlight{$w$}\coordHE{}'s satisfying the ADHM constraints. The auxiliary variables 
\myHighlight{${\cal A}_{tot}$}\coordHE{} can be integrated 
out leaving behind a factor \myHighlight{$1/\det {\bf L}$}\coordHE{}.

The DKM representation , eq. (\ref{modspaceint}), of \myHighlight{${\cal F}_n$}\coordHE{} 
together with the specifications in eq.'s (\ref{action}), 
(\ref{measure}) has to be supplemented by an \myHighlight{$O(n)$}\coordHE{} gauge fixing  
condition since the larger 
space including redundant \myHighlight{$O(n)$}\coordHE{} degrees of freedom turns out to 
be non-orientable. To see the necessity of such a gauge fixing 
procedure we choose a gauge and verify a posteriori that the 
restriction is unavoidable to obtain a non-vanishing result 
for \myHighlight{${\cal F}_n$}\coordHE{}. We may consider for this purpose any representation 
built from the variables \myHighlight{$w$}\coordHE{} and \myHighlight{$a'$}\coordHE{} and impose on this gauge 
fixing conditions. To concretize the ideas let us consider the real 
symmetric matrix 
\[\coord{}\boxMath{
Y_{ik}=\Re e \left(\bar{a}_{ij}a_{jk}\right)
}{corchetes}{0pt}\coordE{}\]
which transforms under the adjoint representation of \myHighlight{$O(n)$}\coordHE{}. \myHighlight{$Y$}\coordHE{} 
can be brought through an \myHighlight{$O(n)$}\coordHE{} transformation into diagonal 
form with the diagonal elements arranged in increasing order of 
their absolute values \footnote{
This condition does not fix completely the gauge since some 
discrete transformations are still possible. Dropping
the factor \myHighlight{$1/Vol(O(n))$}\coordHE{} from (\ref{measure}) one has to take 
care of the latter leftover redundancies.}. There will not appear 
in this procedure a 
non-trivial Faddeev-Popov determinant because of supersymmetry. 
(The assertion will become evident from the deductions in the 
next section where we identify fermions with one-forms.)

The gauge fixing condition degenerates at places where two or more 
than two eigenvalues of \myHighlight{$X$}\coordHE{} coincide. There part of the \myHighlight{$O(n)$}\coordHE{} 
group is restored. In the generic case of two coinciding eigenvalues 
an \myHighlight{$O(2)$}\coordHE{} sub-group is revived. It means that we hit a Gribov 
horizon \cite{gribov}. Points at the two sides of the horizon 
of codimension one are 
related by a permutation (considered as an element of the group 
\myHighlight{$O(n)$}\coordHE{}). This implies that the corresponding volume elements have 
opposite orientations what confirms the above claim 
\footnote{DKM avoid in \cite{dorey1} 
the horizon problem  by integrating the {\it modulus} of the integrand 
and the measure over the lager space.}. 

\section{Simplifications}\label{sec:III} 

To evaluate for general \myHighlight{$n$}\coordHE{} the integral (\ref{modspaceint}) with 
the measure (\ref{measure}) and the induced action, eq. 
(\ref{action}), appears to be a formidable task. The evaluation 
of \myHighlight{${\cal F}_2$}\coordHE{}, achieved by DKM, \cite{dorey1}, already seems, at 
least at first 
sight, to be miraculous. We propose in the following two 
subsections simplifications of the integrals, which we hope, will 
give some insight into the algebraic-geometric nature of the 
problem. The results of these two subsections have also 
been derived by Bellisai, Bruzzo, Fucito, Tanzini and Travaglini 
\cite{tanzini1}, \cite{tanzini2}, who use an  approach technically 
different from our.  

\subsection{Fermions as differential forms}

It is a well known fact that fermion zero modes in selfdual Yang-Mills 
backgrounds are in correspondence with the fluctuation modes of the 
vector fields which makes it appearing natural to identify part or 
all of the cotangent space of the ADHM moduli with the Grassmann 
valued fermion zero modes. For 
the case of the \myHighlight{${\cal N}=2$}\coordHE{} supersymmetry the correspondence of 
fermion modes and cotangent space turns out to be one-to-one. 

To start with we combine the Grassmannian spinor valued matrices 
\myHighlight{${\cal M}$}\coordHE{}, \myHighlight{${\cal N}$}\coordHE{} into a quaternion valued matrix denoted by 
\myHighlight{${\cal P}$}\coordHE{}. Let \myHighlight{$A^{\dot{\alpha}}$}\coordHE{}, \myHighlight{$B^{\dot{\alpha}}$}\coordHE{} be some 
two component (\myHighlight{$\dot{\alpha}=1,2$}\coordHE{}) c-number spinors satisfying 
\begin{eqnarray}\coord{}\boxAlignEqnarray{\leftCoord{}
A\cdot B\equiv A^{\dot{\alpha}} B_{\dot{\alpha}} \equiv 
A^{\dot{\alpha}} B^{\dot{\beta}} \epsilon_{\dot{\alpha}\dot{\beta}} 
\leftCoord{}=1 \nonumber \rightCoord{}\\\leftCoord{}
B_{\dot{\alpha}}=\epsilon_{\dot{\alpha}\dot{\beta}}
B^{\dot{\beta}}, \epsilon_{\dot{1}\dot{2}}=-\epsilon_{\dot{2}\dot{1}} 
\leftCoord{}=1, \nonumber\rightCoord{}
\rightCoord{}}{0mm}{4}{4}{
A\cdot B\equiv A^{\dot{\alpha}} B_{\dot{\alpha}} \equiv 
A^{\dot{\alpha}} B^{\dot{\beta}} \epsilon_{\dot{\alpha}\dot{\beta}} 
=1 \\
B_{\dot{\alpha}}=\epsilon_{\dot{\alpha}\dot{\beta}}
B^{\dot{\beta}}, \epsilon_{\dot{1}\dot{2}}=-\epsilon_{\dot{2}\dot{1}} 
=1, }{1}\coordE{}\end{eqnarray}
and define 
\begin{equation}\coord{}\boxEquation{
\left({\cal P}\right)_{\alpha \dot{\alpha}}={\cal M}_{\alpha} 
B_{\dot{\alpha}}-{\cal N}_{\alpha}A_{\dot{\alpha}}
}{
\left({\cal P}\right)_{\alpha \dot{\alpha}}={\cal M}_{\alpha} 
B_{\dot{\alpha}}-{\cal N}_{\alpha}A_{\dot{\alpha}}
}{ecuacion}\coordE{}\end{equation}
with the inverse relation
\begin{equation}\coord{}\boxEquation{
{\cal M}_{\alpha}=\left({\cal P}\right)_{\alpha \dot{\alpha}} 
A^{\dot{\alpha}}; \, \, \, \,
{\cal N}_{\alpha}=\left({\cal P}\right)_{\alpha \dot{\alpha}} 
B^{\dot{\alpha}}.
}{
{\cal M}_{\alpha}=\left({\cal P}\right)_{\alpha \dot{\alpha}} 
A^{\dot{\alpha}}; \, \, \, \,
{\cal N}_{\alpha}=\left({\cal P}\right)_{\alpha \dot{\alpha}} 
B^{\dot{\alpha}}.
}{ecuacion}\coordE{}\end{equation}   
ADHM matrix labels are here suppressed. The fermionic constraints 
read in terms of the new variables as 
\begin{equation}\coord{}\boxEquation{
\bar{a}{\cal P}-\left(\bar{a}{\cal P}\right)^T=0,
\label{newfermconstraint}
}{
\bar{a}{\cal P}-\left(\bar{a}{\cal P}\right)^T=0,
}{ecuacion}\coordE{}\end{equation}
and the fermionic \myHighlight{$\delta$}\coordHE{}-functions appearing in the measure 
(\ref{measure}) read as 
\begin{equation}\coord{}\boxEquation{
\prod_{i<j}\delta^{(4)}\left(\left(\bar{a}{\cal P}\right)_{ij}-
\left(\bar{a}{\cal P}\right)_{ji}\right). 
\label{fermdelta}
}{
\prod_{i<j}\delta^{(4)}\left(\left(\bar{a}{\cal P}\right)_{ij}-
\left(\bar{a}{\cal P}\right)_{ji}\right). 
}{ecuacion}\coordE{}\end{equation}
It is easily seen that the imaginary parts of the 
fermionic constraints (\ref{newfermconstraint})
are automatically fulfilled if one substitutes for \myHighlight{${\cal P}$}\coordHE{} 
any \myHighlight{$O(n)$}\coordHE{}-covariant derivative of \myHighlight{$a$}\coordHE{}, 
\begin{eqnarray}\coord{}\boxAlignEqnarray{\leftCoord{}
{\rightCoord{}\leftCoord{}\cal P}=\left(d+X\right)a\equiv{\cal D}a; \nonumber \rightCoord{}\\\leftCoord{}
X\cdot a\equiv X\cdot 
\left( \rightCoord{} 
\begin{array}{c} \rightCoord{}
w \rightCoord{}\\\leftCoord{}
a'\rightCoord{}
\end{array} \rightCoord{}
\right)=\left( \rightCoord{}
\begin{array}{c} \rightCoord{}
\leftCoord{}-w\cdot X  \rightCoord{}\\\leftCoord{}
\left[ X,a' \right]\rightCoord{}
\end{array} \rightCoord{}
\right) \rightCoord{}       
\label{covariantd}
\rightCoord{}}{0mm}{6}{15}{
{\cal P}=\left(d+X\right)a\equiv{\cal D}a; \\
X\cdot a\equiv X\cdot 
\left(  
\begin{array}{c} 
w \\
a'
\end{array} 
\right)=\left( 
\begin{array}{c} 
-w\cdot X  \\
\left[ X,a' \right]
\end{array} 
\right)        
}{1}\coordE{}\end{eqnarray}
with \myHighlight{$X$}\coordHE{} being any real skewsymmetric \myHighlight{$n \times n$}\coordHE{} matrix 
of one-forms. Indeed, one has the identities:     
\begin{eqnarray}\coord{}\boxAlignEqnarray{\leftCoord{}
\Im m \left(\bar{a}{\cal D}a-\left(\bar{a}{\cal D}a\right)^T\right) \rightCoord{} 
\leftCoord{}=d \rightCoord{}\, \Im m \left(\bar{a}a\right), \nonumber \rightCoord{}\\\leftCoord{} 
\Im m\left(\bar{a}\left(X\cdot a\right)- \rightCoord{}
\left(\bar{a}\left(X\cdot a\right)\right)^T \right) = \rightCoord{}
\left[X,\Im m \left(\bar{a}a\right)\right], \rightCoord{}
\label{xproperty}
\rightCoord{}}{0mm}{3}{8}{
\Im m \left(\bar{a}{\cal D}a-\left(\bar{a}{\cal D}a\right)^T\right)  
=d \, \Im m \left(\bar{a}a\right), \\ 
\Im m\left(\bar{a}\left(X\cdot a\right)- 
\left(\bar{a}\left(X\cdot a\right)\right)^T \right) = 
\left[X,\Im m \left(\bar{a}a\right)\right], 
}{1}\coordE{}\end{eqnarray}
so that both, the ordinary exterior differential as well as 
the \myHighlight{$O(n)$}\coordHE{} connection part, lead to vanishing contributions 
inserted into the imaginary part of eq. 
(\ref{newfermconstraint}) as long as \myHighlight{$a$}\coordHE{} satisfies the ADHM 
constraints. For an arbitrary \myHighlight{$O(n)$}\coordHE{} Lie 
algebra valued one form \myHighlight{$X$}\coordHE{} holds 
\begin{equation}\coord{}\boxEquation{
\Re e \left(\bar{a}\left(X\cdot a\right)-
\left(\bar{a}\left(X\cdot a\right)\right)^T \right)=-{\bf L}\cdot X 
\nonumber 
}{
\Re e \left(\bar{a}\left(X\cdot a\right)-
\left(\bar{a}\left(X\cdot a\right)\right)^T \right)=-{\bf L}\cdot X 
\nonumber 
}{ecuacion}\coordE{}\end{equation}
with \myHighlight{${\bf L}$}\coordHE{} as introduced in eq. (\ref{loperator}). This allows 
to choose a connection s.t. also the remaining real part of 
(\ref{newfermconstraint}) vanishes which is found to be given by 
\begin{equation}\coord{}\boxEquation{
X={\bf L}^{-1}\cdot \Re e \left(\bar{a}da-
\left(\bar{a}da\right)^T\right).
\label{xchoice}
}{
X={\bf L}^{-1}\cdot \Re e \left(\bar{a}da-
\left(\bar{a}da\right)^T\right).
}{ecuacion}\coordE{}\end{equation}    
We assume now that, the bosonic ADHM moduli \myHighlight{$a$}\coordHE{} satisfy the ADHM
constraints  and the gauge 
fixing condition 
\begin{equation}\coord{}\boxEquation{
Y_{ik}\equiv \Re e \left(\bar{a}a\right)_{ik}= 0; \,\,\,\, \mbox{for} \, \, 
i\neq k. \nonumber 
}{
Y_{ik}\equiv \Re e \left(\bar{a}a\right)_{ik}= 0; \,\,\,\, \mbox{for} \, \, 
i\neq k. \nonumber 
}{ecuacion}\coordE{}\end{equation}
The fermionic variables are determined by eq.'s (\ref{covariantd}) and 
(\ref{xchoice}) and satisfy therefore the constraints 
(\ref{newfermconstraint}). The Jacobian factors arising from the 
integration of the bosonic \myHighlight{$\delta$}\coordHE{}-functions in eq. (\ref{measure}) 
and the imaginary projections of the fermionic \myHighlight{$\delta$}\coordHE{}-functions, eq. 
(\ref{fermdelta}), cancel each other as a consequence of the 
relations (\ref{xproperty}). From the real part of the fermionic 
\myHighlight{$\delta$}\coordHE{}-functions survives the \myHighlight{$SO(n)$}\coordHE{} Haar measure (
multiplied by \myHighlight{${\bf L}$}\coordHE{})
\[\coord{}\boxMath{
\int_{g\in SO(n)}\prod_{i<j}\left({\bf L}\left(g^{-1}dg\right)\right)_{ij} 
=Vol(SO(n)) \det {\bf L}.
}{corchetes}{0pt}\coordE{}\]  
The factor \myHighlight{$\det {\bf L}$}\coordHE{} from the last integration drops out 
together with the \myHighlight{$1/\det {\bf L}$}\coordHE{} factor of the integration 
of the auxiliary variables \myHighlight{${\cal A}_{tot}$}\coordHE{} in (\ref{measure}).
Viewing the induced action \myHighlight{$S^{(n)}$}\coordHE{} as a  mixed differential form 
\[\coord{}\boxMath{ S^{(n)}=S^{(n)}(a,{\cal D}a) }{corchetes}{0pt}\coordE{}\]
we are all in all lead to rewrite eq. (\ref{modspaceint}) as 
\begin{equation}\coord{}\boxEquation{
{\cal F}_n\simeq C_n \frac{Vol(SO(n))}{Vol(O(n))}
\int_{{\cal M}_n} e^{-S^{(n)(a,{\cal D}a)}}
\label{newmodspaceint1}
}{
{\cal F}_n\simeq C_n \frac{Vol(SO(n))}{Vol(O(n))}
\int_{{\cal M}_n} e^{-S^{(n)(a,{\cal D}a)}}
}{ecuacion}\coordE{}\end{equation}         
with \myHighlight{${\cal M}_n$}\coordHE{} denoting the \myHighlight{$n$}\coordHE{}-instanton moduli space. The 
exponential under the integral has to be expanded s.t. the top 
form on \myHighlight{${\cal M}_n$}\coordHE{} is reached. 
     
\subsection{\myHighlight{$S^{(n)}$}\coordHE{} as an equivariantly exact form}

The integrand on the r.h.s. of eq. (\ref{newmodspaceint1}) is also 
invariant under a \myHighlight{$U(1)$}\coordHE{} symmetry (besides its \myHighlight{$O(n)$}\coordHE{} invariance), 
the remainder of the original \myHighlight{$SU(2)$}\coordHE{} gauge group. We want to introduce 
an equivariant calculus with respect to this \myHighlight{$U(1)$}\coordHE{} symmetry, (for a 
detailed account of the equivariant differential calculus one may 
consult chapter 7 of \cite{berline}). 

Let \myHighlight{${\bf M}$}\coordHE{} denote a manifold with a continuous group \myHighlight{${\bf G}$}\coordHE{} 
acting on it. Vectors \myHighlight{$X$}\coordHE{} of the Lie algebra \myHighlight{${\bf g}$}\coordHE{} of \myHighlight{${\bf G}$}\coordHE{} 
are mapped to vector fields \myHighlight{$L_X$}\coordHE{} on \myHighlight{${\bf M}$}\coordHE{}. With \myHighlight{$i_X$}\coordHE{} we denote the 
nilpotent operation of contraction of the vector field \myHighlight{$L_X$}\coordHE{} with 
differential forms on  \myHighlight{${\bf M}$}\coordHE{}. Following Cartan \cite{cartan} 
one introduces the ``equivariant'' external differential 
\begin{equation}\coord{}\boxEquation{
d_X =d-i_X.
\label{equivdiff}
}{
d_X =d-i_X.
}{ecuacion}\coordE{}\end{equation}
\myHighlight{$d_X$}\coordHE{} is nilpotent on \myHighlight{${\bf G}$}\coordHE{}-invariant differential forms since 
one has 
\begin{equation}\coord{}\boxEquation{
d_X^2 =-\left(d \circ i_X + i_X \circ d \right)={\cal L}_X  
}{
d_X^2 =-\left(d \circ i_X + i_X \circ d \right)={\cal L}_X  
}{ecuacion}\coordE{}\end{equation}             
with \myHighlight{${\cal L}_X$}\coordHE{} being identified according to Cartan's homotopy 
formula with the Lie derivative along the vector field \myHighlight{$L_X$}\coordHE{}. The 
extension of the formalism to the setting of a vector bundle over 
\myHighlight{${\bf M}$}\coordHE{} is straightforward. 

Let \myHighlight{${\cal D}=d+A$}\coordHE{} denote a covariant derivative acting on 
sections of such a bundle (in the case under consideration a bundle 
associated to \myHighlight{$O(n)$}\coordHE{}). The equivariant operation \myHighlight{${\cal D}_X$}\coordHE{} is 
defined in analogy to eq. (\ref{equivdiff}) by 
\begin{equation}\coord{}\boxEquation{
{\cal D}_X ={\cal D}-i_X
\label{covequivdiff}.
}{
{\cal D}_X ={\cal D}-i_X
.
}{ecuacion}\coordE{}\end{equation} 
This gives rise to the notion of equivariant curvature 
\begin{equation}\coord{}\boxEquation{
{\cal F}_X\left(\cdot \right) =\left({\cal D}_X^2 +{\cal L}_X \right)
\left(\cdot \right)
\label{equivcurvature}
}{
{\cal F}_X\left(\cdot \right) =\left({\cal D}_X^2 +{\cal L}_X \right)
\left(\cdot \right)
}{ecuacion}\coordE{}\end{equation} 
satisfying the equivariant Bianchi identity
\begin{equation}\coord{}\boxEquation{
{\cal D}_X {\cal F}_X=0
\label{equivbianchi}. 
}{
{\cal D}_X {\cal F}_X=0
. 
}{ecuacion}\coordE{}\end{equation}
The \myHighlight{$U(1)$}\coordHE{} symmetry in question is represented infinitesimally 
on the bosonic ADHM parameters \myHighlight{$w$}\coordHE{}, \myHighlight{$a'$}\coordHE{} by   
\begin{equation}\coord{}\boxEquation{
\delta w\sim vw, \, \, \, \, \delta \bar{w}\sim -\bar{w} v , 
\, \, \, \, \delta a'=0, 
\label{u1action} 
}{
\delta w\sim vw, \, \, \, \, \delta \bar{w}\sim -\bar{w} v , 
\, \, \, \, \delta a'=0, 
}{ecuacion}\coordE{}\end{equation}  
\myHighlight{$v$}\coordHE{} being the breaking parameter introduced above. Let \myHighlight{$L_v$}\coordHE{} denote 
the vector field on the moduli space corresponding to the 
transformation laws (\ref{u1action}). The contraction operation 
associated to this vector field is given by
\begin{equation}\coord{}\boxEquation{
i_v \cdot d w = vw, \, \, \, \, i_v \cdot d \bar{w} = -\bar{w} v , 
\, \, \, \, i_v \cdot d a'=0. 
\label{contraction} 
}{
i_v \cdot d w = vw, \, \, \, \, i_v \cdot d \bar{w} = -\bar{w} v , 
\, \, \, \, i_v \cdot d a'=0. 
}{ecuacion}\coordE{}\end{equation}    
A simple calculation reveals that the induced action \myHighlight{$S^{(n)}$}\coordHE{} can 
be represented as equivariant external exterior derivative of 
an \myHighlight{$U(1)$}\coordHE{} invariant one form denoted by \myHighlight{$\omega$}\coordHE{} 
\begin{eqnarray}\coord{}\boxAlignEqnarray{\leftCoord{}
\label{eq:newaction} 
S^{(n)}={\cal D}_v \cdot \omega \equiv d_v \omega \rightCoord{}\\\leftCoord{}
{\rightCoord{}\leftCoord{}\cal D}_v =d+X-i_v, \nonumber \rightCoord{}\\\leftCoord{}
d_v =d-i_v , \nonumber \rightCoord{}\\\leftCoord{}
\label{eq:omega}
\omega = -\sum_{\rightCoord{}i=1}^{\leftCoord{}n} 2\Re e\left( \bar{w}_i \bar{v}{\cal D} w_i\right)
\rightCoord{}}{0mm}{6}{7}{
S^{(n)}={\cal D}_v \cdot \omega \equiv d_v \omega \\
{\cal D}_v =d+X-i_v, \\
d_v =d-i_v , \\
\omega = -\sum_{i=1}^{n} 2\Re e\left( \bar{w}_i \bar{v}{\cal D} w_i\right)
}{1}\coordE{}\end{eqnarray}  
The two alternative representations of \myHighlight{$S^{(n)}$}\coordHE{} \footnote{
\myHighlight{$S^{(n)}$}\coordHE{} in eq. (\ref{eq:newaction}) differs by an irrelevant 
overall normalization factor from the DKM action, eq. (\ref{action}).
} 
in 
(\ref{eq:newaction}) are a consequence of the fact that the 
one-form \myHighlight{$\omega$}\coordHE{} is \myHighlight{$O(n)$}\coordHE{} invariant.

So, \myHighlight{${\cal F}_n$}\coordHE{} can be represented as the integral of an 
exponential of an equivariantly exact form (superseding eq. 
(\ref{newmodspaceint1}) ) 
\begin{equation}\coord{}\boxEquation{
{\cal F}_n
\simeq \int_{{\cal M}_n} e^{-d_v\omega}.
\label{newmodspaceint2}
}{
{\cal F}_n
\simeq \int_{{\cal M}_n} e^{-d_v\omega}.
}{ecuacion}\coordE{}\end{equation}  
To demonstrate the naturalness of the equivariant calculus in 
the present context (apart from the concise appearance of the 
action (\ref{eq:newaction}), (\ref{eq:omega})) we quote the 
antichiral supersymmetry transformations induced from \myHighlight{${\bf N}=2$}\coordHE{} 
field theory  to the ADHM supermoduli (as shown in \cite{dorey2})
\footnote{we quote here a special supersymmetry transformation. 
The general transformations are of the form \myHighlight{$Q^i_{\dot{\alpha}}
a_{\alpha \dot{\beta}} =\epsilon_{\dot{\alpha}\dot{\beta}}
\left({\cal D}a\right)^i_{\alpha}
, \, \, \, i=1,2$}\coordHE{} etc.}: 
\begin{eqnarray}\coord{}\boxAlignEqnarray{\leftCoord{}
\delta a_{\alpha \dot{\beta}}=\left({\cal D}_v a\right)_{\alpha \dot{\beta}} 
\leftCoord{}\rightCoord{}\, \rightCoord{}\, \rightCoord{}\, \rightCoord{}\, \left(\equiv {\cal D} a_{\alpha \dot{\beta}}\right); 
\nonumber \rightCoord{}\\\leftCoord{}
\delta \left({\cal D}_v a\right)_{\alpha \dot{\beta}}= 
{\rightCoord{}\leftCoord{}\cal D}_v^2 a_{\alpha \dot{\beta}}= {\cal F}_v a_{\alpha \dot{\beta}} 
\equiv 
\left( \rightCoord{}
\begin{array}{c} \rightCoord{}
\leftCoord{}-\omega_{\alpha \dot{\beta}} {\cal F}_v \rightCoord{}\\\leftCoord{}
\left[{\cal F}_v,a'_{\alpha \dot{\beta}} \right] \rightCoord{}
\end{array} \rightCoord{}
\right). \rightCoord{}
\rightCoord{}}{0mm}{6}{14}{
\delta a_{\alpha \dot{\beta}}=\left({\cal D}_v a\right)_{\alpha \dot{\beta}} 
\, \, \, \, \left(\equiv {\cal D} a_{\alpha \dot{\beta}}\right); 
\\
\delta \left({\cal D}_v a\right)_{\alpha \dot{\beta}}= 
{\cal D}_v^2 a_{\alpha \dot{\beta}}= {\cal F}_v a_{\alpha \dot{\beta}} 
\equiv 
\left( 
\begin{array}{c} 
-\omega_{\alpha \dot{\beta}} {\cal F}_v \\
\left[{\cal F}_v,a'_{\alpha \dot{\beta}} \right] 
\end{array} 
\right). 
}{1}\coordE{}\end{eqnarray}
The closure of this representation of supersymmetry transformations - 
modulo \myHighlight{$O(n)$}\coordHE{} transformations - is a consequence of the equivariant 
Bianchi identity, eq (\ref{equivbianchi}).  
                                
\section{\myHighlight{${\cal F}_n$}\coordHE{} as a formal intersection number}\label{sec:IV}

Standard localization theory for exact equivariant forms 
\cite{berline}, \cite{schwarz} might suggest that the integral 
(\ref{newmodspaceint2}) can be localized at the set of critical 
points of the vector field \myHighlight{$L_v$}\coordHE{}, that is , at \myHighlight{$w_1=\cdots w_n=0$}\coordHE{}. 
This turns out not to be the case, as the residuum in question at 
the locus \myHighlight{$w=0$}\coordHE{} vanishes. It should also be noted that the standard 
theory, tailored for compact manifolds without boundaries, does 
not obviously apply to our problem since there are at least three 
potential obstacles \\
(i) The variables \myHighlight{$w_i$}\coordHE{} and \myHighlight{$a'_{ij}$}\coordHE{} reach out to infinity; \\
(ii) We cannot ignore a Gribov horizon, as noted above, which 
supplies one type of boundary. \\
(iii) The other type of boundary, the Donaldson - Uhlenbeck boundary 
\cite{donaldson}, \cite{uhlenbeck}, appears at places where the 
rank condition (see \cite{adhm}) is violated. \\
To deal with item (i) we introduce a scaling variable by setting 
\myHighlight{$a=R^{1/2}\hat{a}$}\coordHE{}, i.e.  
\begin{equation}\coord{}\boxEquation{
w=R^{1/2}\hat{w}, \, \, \, \, \, a'_{ij}=R^{1/2}\hat{a}'_{ij}
}{
w=R^{1/2}\hat{w}, \, \, \, \, \, a'_{ij}=R^{1/2}\hat{a}'_{ij}
}{ecuacion}\coordE{}\end{equation}
s.t. holds \footnote{The appearance of the quadratic form on 
the l.h.s. of eq. (\ref{scalefixing}) is irrelevant as long as 
it is non-degenerate.}
\begin{equation}\coord{}\boxEquation{
\sum_{i=1} \left| \hat{w}_i \right|^2 +\sum_{i,j=1}^n 
\left| \hat{a}'_{ij} \right|^2 = 1
\label{scalefixing} 
}{
\sum_{i=1} \left| \hat{w}_i \right|^2 +\sum_{i,j=1}^n 
\left| \hat{a}'_{ij} \right|^2 = 1
}{ecuacion}\coordE{}\end{equation}       
\myHighlight{$S^{(n)}$}\coordHE{} reads in terms of the variables \myHighlight{$R$}\coordHE{}, \myHighlight{$\hat{a}$}\coordHE{} as 
\begin{equation}\coord{}\boxEquation{
S^{(n)}=dR \, \hat{\omega}^{(n)}+Rd_v \hat{\omega}^{(n)}
}{
S^{(n)}=dR \, \hat{\omega}^{(n)}+Rd_v \hat{\omega}^{(n)}
}{ecuacion}\coordE{}\end{equation}
with the notation \myHighlight{$\hat{\omega} \equiv \omega \left(\hat{a}, 
d \hat{a} \right)$}\coordHE{}. The scaling variable \myHighlight{$R$}\coordHE{} can be 
integrated straightforwardly, 
\begin{eqnarray}\coord{}\boxAlignEqnarray{\leftCoord{}
{\rightCoord{}\leftCoord{}\cal F}_n
\simeq \int_{\leftCoord{}{\cal M}_n} e^{-dR \hat{\omega}-
Rd_v \hat{\omega}}=-\int_{\leftCoord{}\hat{{\cal M}}_n}\int_0^{\infty} dR 
\hat{\omega}e^{-Rd_v \hat{\omega}}=\int_{\leftCoord{}\hat{{\cal M}}_n} 
\frac{\leftCoord{}\hat{\omega}}{\rightCoord{}d_v \hat{\omega}}.  \rightCoord{}
\label{newmodspaceint3}
\rightCoord{}}{0mm}{6}{5}{
{\cal F}_n
\simeq \int_{{\cal M}_n} e^{-dR \hat{\omega}-
Rd_v \hat{\omega}}=-\int_{\hat{{\cal M}}_n}\int_0^{\infty} dR 
\hat{\omega}e^{-Rd_v \hat{\omega}}=\int_{\hat{{\cal M}}_n} 
\frac{\hat{\omega}}{d_v \hat{\omega}}.  
}{1}\coordE{}\end{eqnarray}
(\myHighlight{$\hat{{\cal M}}_n$}\coordHE{} is the manifold of the rescaled moduli \myHighlight{$\hat{a}$}\coordHE{} )

Using the notation \myHighlight{$\rho =\hat{\omega}/i_v \cdot \hat{\omega}$}\coordHE{} 
we rewrite the previous equation in the form 
\begin{equation}\coord{}\boxEquation{
{\cal F}_n \simeq \int_{\hat{{\cal M}}_n} \rho \left(d\rho \right)^{4n-3}.
\label{restmodspaceint1}
}{
{\cal F}_n \simeq \int_{\hat{{\cal M}}_n} \rho \left(d\rho \right)^{4n-3}.
}{ecuacion}\coordE{}\end{equation}  
To verify the equality of (\ref{newmodspaceint3}) and 
(\ref{restmodspaceint1}) one has to expand the denominator in the 
integrand of (\ref{newmodspaceint3}) and to take into account the 
identities 
\begin{equation}\coord{}\boxEquation{
\frac{\hat{\omega}}{i_v \cdot \hat{\omega}} 
\left(\frac{d\hat{\omega}}{i_v \cdot \hat{\omega}} \right)^k =
\frac{\hat{\omega}}{i_v \cdot \hat{\omega}} 
\left(d\, \frac{\hat{\omega}}{i_v \cdot \hat{\omega}} \right)^k \equiv 
\rho \left(d\rho \right)^k 
}{
\frac{\hat{\omega}}{i_v \cdot \hat{\omega}} 
\left(\frac{d\hat{\omega}}{i_v \cdot \hat{\omega}} \right)^k =
\frac{\hat{\omega}}{i_v \cdot \hat{\omega}} 
\left(d\, \frac{\hat{\omega}}{i_v \cdot \hat{\omega}} \right)^k \equiv 
\rho \left(d\rho \right)^k 
}{ecuacion}\coordE{}\end{equation} 
for any nonnegative integer \myHighlight{$k$}\coordHE{}. 
The most important properties of the forms \myHighlight{$\rho$}\coordHE{}, \myHighlight{$d\rho$}\coordHE{} are contained 
in the equations 
\begin{eqnarray}\coord{}\boxAlignEqnarray{\leftCoord{}
\label{eq:rhocontraction}
i_v \cdot \rho =1; \rightCoord{}\\\leftCoord{}
\label{eq:drhocontraction}
i_v \cdot d\rho =0,
\rightCoord{}}{0mm}{2}{3}{
i_v \cdot \rho =1; \\
i_v \cdot d\rho =0,
}{1}\coordE{}\end{eqnarray}
the latter equality being a consequence of the relations 
\begin{equation}\coord{}\boxEquation{
\rho=\frac{\hat{\omega}}{i_v \cdot \hat{\omega}}, \,\, 
{\cal L}_v \hat{\omega}\equiv \left(i_v \circ d + 
d \circ i_v \right) \hat{\omega}=0.
}{
\rho=\frac{\hat{\omega}}{i_v \cdot \hat{\omega}}, \,\, 
{\cal L}_v \hat{\omega}\equiv \left(i_v \circ d + 
d \circ i_v \right) \hat{\omega}=0.
}{ecuacion}\coordE{}\end{equation}
One may view \myHighlight{$\hat{{\cal M}}_n$}\coordHE{} as a \myHighlight{$S^1$}\coordHE{} bundle vis-\`{a}-vis 
the action of the \myHighlight{$U(1)$}\coordHE{} symmetry. Eq. (\ref{eq:drhocontraction}) means 
that \myHighlight{$\rho$}\coordHE{} contains with coefficient \myHighlight{$1/2|v|$}\coordHE{} the differential of an 
angle parameterizing the \myHighlight{$U(1)$}\coordHE{} group orbits. \myHighlight{$2|v|\rho$}\coordHE{} is, with other words, 
an angular one-form of the \myHighlight{$S^1$}\coordHE{}-bundle and \myHighlight{$2|v|d\rho$}\coordHE{} 
would have to be identified with the Euler class of that bundle 
(see e.g. \cite{bott}) were there 
no boundaries. The differential of the \myHighlight{$U(1)$}\coordHE{} angle, call it \myHighlight{$\varphi$}\coordHE{}, 
only shows up in \myHighlight{$\rho$}\coordHE{} but not, according to eq. 
(\ref{eq:drhocontraction}), in \myHighlight{$d\rho$}\coordHE{}. We may therefore substitute
\myHighlight{$\rho$}\coordHE{} in the integrand on the r.h.s. of eq. 
(\ref{restmodspaceint1}) by \myHighlight{$2|v|d\varphi$}\coordHE{} (other parts of \myHighlight{$\rho$}\coordHE{} lead to
vanishing contributions). Integrating out \myHighlight{$\varphi$}\coordHE{} we arrive at 
\begin{equation}\coord{}\boxEquation{
{\cal F}_n \simeq \int_{\hat{{\cal M}}'_n} \left(d\rho \right)^{4n-3}, 
\label{restmodspaceint2}
}{
{\cal F}_n \simeq \int_{\hat{{\cal M}}'_n} \left(d\rho \right)^{4n-3}, 
}{ecuacion}\coordE{}\end{equation}       
where \myHighlight{$\hat{{\cal M}}'_n$}\coordHE{} denotes the base 
of the \myHighlight{$S^1$}\coordHE{}-bundle of which \myHighlight{$\hat{{\cal M}}_n$}\coordHE{} is the total bundle 
space. \myHighlight{${\cal F}_n$}\coordHE{} as represented in eq. (\ref{restmodspaceint2}), 
appears as an intersection number, i.e. the integral of the 
\myHighlight{$(4n-3)$}\coordHE{}-fold product of \myHighlight{$d\rho$}\coordHE{}.

{\bf Remarks:} \\
1)The concrete coordinatization of the \myHighlight{$U(1)$}\coordHE{} group requires the choice 
of a concrete, complex scalar \myHighlight{$U(1)$}\coordHE{} representation. It seems 
advisable to choose an \myHighlight{$O(n)$}\coordHE{} invariant combination of variables 
forming such a representation. (Only in this case the \myHighlight{$U(1)$}\coordHE{} angle 
will not be changed along \myHighlight{$O(n)$}\coordHE{} orbits and, hence, can be naturally 
defined in the space \myHighlight{$\hat{{\cal M}}_n$}\coordHE{}.)
One may, for example, combine the quaternionic components of 
\begin{eqnarray}\coord{}\boxAlignEqnarray{\leftCoord{}
w=\left(w_1, \cdots ,w_n \right) \rightCoord{}\\\leftCoord{}
w_i= \sum_{\rightCoord{}\mu=0}^{\leftCoord{}3} e_{\mu} \cdot w_i^{\mu},
\rightCoord{}}{0mm}{3}{4}{
w=\left(w_1, \cdots ,w_n \right) \\
w_i= \sum_{\mu=0}^{3} e_{\mu} \cdot w_i^{\mu},
}{1}\coordE{}\end{eqnarray} 
 into complex eigenvectors \myHighlight{$\psi_i$}\coordHE{}, \myHighlight{$\chi_i$}\coordHE{} of the \myHighlight{$U(1)$}\coordHE{} 
transformations 
\begin{eqnarray}\coord{}\boxAlignEqnarray{\leftCoord{}
\psi_i=w_i^0+iw_i^3; \nonumber \rightCoord{}\\\leftCoord{}
\chi_i==w_i^2+iw_i^1 \nonumber\rightCoord{} 
\rightCoord{}}{0mm}{2}{4}{
\psi_i=w_i^0+iw_i^3; \\
\chi_i==w_i^2+iw_i^1 }{1}\coordE{}\end{eqnarray}
and build with them \myHighlight{$O(n)$}\coordHE{}- invariant variables 
\begin{eqnarray}\coord{}\boxAlignEqnarray{\leftCoord{}
Z=\sum_{\rightCoord{}i=1}^{\leftCoord{}n}  \psi_i^2 \nonumber \rightCoord{}\\\leftCoord{}
Z'=\sum_{\rightCoord{}i=1}^{\leftCoord{}n}  \chi_i^2 \nonumber\rightCoord{} .\rightCoord{}
\rightCoord{}}{0mm}{4}{7}{
Z=\sum_{i=1}^{n}  \psi_i^2 \\
Z'=\sum_{i=1}^{n}  \chi_i^2 .
}{1}\coordE{}\end{eqnarray}
The real and imaginary part of \myHighlight{$Z$}\coordHE{} (\myHighlight{$Z'$}\coordHE{}) are of the type of doublets 
we are looking for. Suppose we parameterize the \myHighlight{$Z$}\coordHE{}- plane with a 
radial and an angular variable. The latter may be be identified 
with the above variable \myHighlight{$\varphi$}\coordHE{}. One is now confronted with three 
kinds of boundaries, the Gribov horizon, the Donaldson-Uhlenbeck 
(DU) boundary and the submanifold given by \myHighlight{$Z=0$}\coordHE{}, where the 
chosen coordinatization of the \myHighlight{$U(1)$}\coordHE{} symmetry breaks down.  
The Gribov horizon does not give a contribution because there at 
least one \myHighlight{$O(2)$}\coordHE{} subgroup of \myHighlight{$O(n)$}\coordHE{} is restored as symmetry and 
this is not covered by the locally \myHighlight{$O(n)$}\coordHE{} invariant combinations 
of differential forms showing up in \myHighlight{$S^{(n)}$}\coordHE{}. The vanishing 
of the contributions at the DU boundary is the consequence of simple 
estimates. We conclude that the integral (\ref{restmodspaceint2}) 
is localized at \myHighlight{$Z=0$}\coordHE{} 
\begin{equation}\coord{}\boxEquation{
{\cal F}_n \simeq \int_{\hat{{\cal M}}'_n \cap \{Z=0\}} 
\rho \left(d\rho \right)^{4n-4}. 
\label{restmodspaceint3}
}{
{\cal F}_n \simeq \int_{\hat{{\cal M}}'_n \cap \{Z=0\}} 
\rho \left(d\rho \right)^{4n-4}. 
}{ecuacion}\coordE{}\end{equation}          
We can iterate this procedure choosing as new \myHighlight{$U(1)$}\coordHE{} variable, e.g., the 
argument of the complex variable \myHighlight{$Z'$}\coordHE{} and afterwards possibly 
other combinations of variables. Unfortunately we have not been 
able to execute the recursion efficiently enough to reach a 
determination of \myHighlight{${\cal F}_n$}\coordHE{} for general \myHighlight{$n$}\coordHE{}. \\ 
2) We want to emphasize that the representation (\ref{restmodspaceint2}) 
for \myHighlight{${\cal F}_n$}\coordHE{} is by no means unique because of the quasi-cohomological 
character of the problem. To illustrate the point we show that the 
result for \myHighlight{${\cal F}_n$}\coordHE{} is to a large extent independent of the choice 
of an \myHighlight{$O(n)$}\coordHE{}- connection. \\ 
Let \myHighlight{${\bf L}_(\alpha, \beta)$}\coordHE{} denote the operator 
\begin{eqnarray}\coord{}\boxAlignEqnarray{\leftCoord{} 
{\rightCoord{}\leftCoord{}\bf L}_(\alpha, \beta)\left(\cdot \right) = 
\alpha \left\{W, \cdot \right\} +\beta \left( \rightCoord{}
\bar{a}'\left[a',\cdot \right]- \left[\bar{a}',\cdot \right]a'\right) \rightCoord{} 
\label{deformedl}
\rightCoord{}}{0mm}{2}{5}{ 
{\bf L}_(\alpha, \beta)\left(\cdot \right) = 
\alpha \left\{W, \cdot \right\} +\beta \left( 
\bar{a}'\left[a',\cdot \right]- \left[\bar{a}',\cdot \right]a'\right)  
}{1}\coordE{}\end{eqnarray}
with \myHighlight{$\alpha$}\coordHE{} and \myHighlight{$\beta$}\coordHE{} being positive real numbers. 
\myHighlight{${\bf L}_(\alpha, \beta)$}\coordHE{} is invertible on \myHighlight{$\hat{{\cal M}}_n$}\coordHE{}. 
We may consider the modified equivariant connection
\begin{eqnarray}\coord{}\boxAlignEqnarray{\leftCoord{}
{\rightCoord{}\leftCoord{}\cal D}_v^{(\alpha, \beta)} &=&d+A^{(\alpha, \beta)}-i_v, \nonumber \rightCoord{}\\\leftCoord{}
A^{(\alpha, \beta)}&=&\frac{\leftCoord{}1}{\rightCoord{}{\rightCoord{}\bf L}_(\alpha, \beta)}\Re e \left\{
\alpha \bar{w}dw+\beta \bar{a}'da'-\left(\cdots \right)^T \rightCoord{}
\right\} \nonumber\rightCoord{}
\rightCoord{}}{0mm}{4}{8}{
{\cal D}_v^{(\alpha, \beta)} &=&d+A^{(\alpha, \beta)}-i_v, \\
A^{(\alpha, \beta)}&=&\frac{1}{{\bf L}_(\alpha, \beta)}\Re e \left\{
\alpha \bar{w}dw+\beta \bar{a}'da'-\left(\cdots \right)^T 
\right\} }{1}\coordE{}\end{eqnarray}
and the modified action 
\begin{equation}\coord{}\boxEquation{
S^{(n)(\alpha, \beta)}=-d_v \cdot \Re e\left( \bar{w}\bar{v}
{\cal D}_v^{(\alpha, \beta)}w\right)  \nonumber
}{
S^{(n)(\alpha, \beta)}=-d_v \cdot \Re e\left( \bar{w}\bar{v}
{\cal D}_v^{(\alpha, \beta)}w\right)  \nonumber
}{ecuacion}\coordE{}\end{equation}
and 
\begin{equation}\coord{}\boxEquation{
{\cal F}_n^{(\alpha, \beta)}\simeq \int_{{\cal M}_n}
e^{-S^{(n)(\alpha, \beta)}}.  \nonumber 
}{
{\cal F}_n^{(\alpha, \beta)}\simeq \int_{{\cal M}_n}
e^{-S^{(n)(\alpha, \beta)}}.  \nonumber 
}{ecuacion}\coordE{}\end{equation}
We want to show that \myHighlight{${\cal F}_n^{(\alpha, \beta)}={\cal F}_n$}\coordHE{}. 
In fact we have 
\begin{eqnarray}\coord{}\boxAlignEqnarray{\leftCoord{}
{\rightCoord{}\leftCoord{}\cal F}_n -{\cal F}_n^{(\alpha, \beta)}=\int_{\leftCoord{}{\cal M}_n} 
\left(e^{-S^{(n)}}-e^{-S^{(n)(\alpha, \beta)}}\right)= \nonumber \rightCoord{}\\\leftCoord{}
\int_{\leftCoord{}{\hat{{\cal M}}_n}} \left(\frac{\leftCoord{} \Re e \left(\bar{w}\bar{v}
{\rightCoord{}\leftCoord{}\cal D}_v w\right)}{d_v \cdot  \Re e\left(\bar{w}\bar{v}
{\rightCoord{}\leftCoord{}\cal D}_v w\right)}-\frac{\leftCoord{} \Re e \left(\bar{w}\bar{v}
{\rightCoord{}\leftCoord{}\cal D}_v^{(\alpha, \beta)}w\right)}{d_v \cdot \Re e 
\left(\bar{w}\bar{v}{\cal D}_v^{(\alpha, \beta)}w\right)}\right). \rightCoord{}
\rightCoord{}}{0mm}{10}{8}{
{\cal F}_n -{\cal F}_n^{(\alpha, \beta)}=\int_{{\cal M}_n} 
\left(e^{-S^{(n)}}-e^{-S^{(n)(\alpha, \beta)}}\right)= \\
\int_{{\hat{{\cal M}}_n}} \left(\frac{ \Re e \left(\bar{w}\bar{v}
{\cal D}_v w\right)}{d_v \cdot  \Re e\left(\bar{w}\bar{v}
{\cal D}_v w\right)}-\frac{ \Re e \left(\bar{w}\bar{v}
{\cal D}_v^{(\alpha, \beta)}w\right)}{d_v \cdot \Re e 
\left(\bar{w}\bar{v}{\cal D}_v^{(\alpha, \beta)}w\right)}\right). 
}{1}\coordE{}\end{eqnarray}
It is easily seen that the relevant topform (t.f.) part of the 
integrand is exact. 
One concludes from arguments as used above in remark (i) that there 
are no boundary contributions to the integral of this exact form 
and hence \myHighlight{${\cal F}_n ={\cal F}_n^{(\alpha, \beta)}$}\coordHE{}.


\section{Concluding remarks}

Matone \cite{matone3} has pointed out an interesting analogy between 
the recursive determination of the Weil - Peterson volume of the 
moduli space \myHighlight{${\cal M}_{0,n}$}\coordHE{} of punctured Riemann spheres and the 
calculation of the \myHighlight{${\bf N}=2$}\coordHE{} prepotential. There are several 
facets to this analogy of the which we only quote the one which 
might be called microscopic. It has been shown by Zograf \cite{zograf} 
that the Weil - Peterson volume of \myHighlight{${\cal M}_{0,n}$}\coordHE{} 
\begin{equation}\coord{}\boxEquation{
Vol_{WP}\left({\cal M}_{0,n}\right)=\frac{1}{(n-3)!}\int_{{\cal M}_{0,n}}
\omega_{WP}^{n-3}
\label{wpvolume}
}{
Vol_{WP}\left({\cal M}_{0,n}\right)=\frac{1}{(n-3)!}\int_{{\cal M}_{0,n}}
\omega_{WP}^{n-3}
}{ecuacion}\coordE{}\end{equation}   
with \myHighlight{$\omega_{WP}$}\coordHE{} denoting the (closed) Weil - Peterson two form, 
may be determined through a two - term recursion relation (the analogue 
in the instanton case is a three term relation) 
\begin{eqnarray}\coord{}\boxAlignEqnarray{\leftCoord{}
v_n=\sum_{\rightCoord{}j=1}^{\leftCoord{}{}\leftCoord{}n-3}\frac{\leftCoord{}j(n-j-2)}{\rightCoord{}n-1}
\left( \rightCoord{}
\begin{array}{c} \rightCoord{}
n-4 \rightCoord{}\\\leftCoord{}
j-1\rightCoord{}
\end{array} \rightCoord{}
\right) \rightCoord{}
\left( \rightCoord{}
\begin{array}{c} \rightCoord{}
n \rightCoord{}\\\leftCoord{}
j+1\rightCoord{}
\end{array} \rightCoord{}
\right) \rightCoord{}
v_{j+2}v_{n-j}; \rightCoord{}\, \rightCoord{}\, \rightCoord{}\, \rightCoord{}\, \rightCoord{}\, \rightCoord{}\, \rightCoord{}\, \rightCoord{}\, n\geq 4, \rightCoord{}\, \rightCoord{}\, \rightCoord{}\, \rightCoord{}\, v_3=1,
\label{zografrecurence}
\rightCoord{}}{0mm}{6}{28}{
v_n=\sum_{j=1}^{{}n-3}\frac{j(n-j-2)}{n-1}
\left( 
\begin{array}{c} 
n-4 \\
j-1
\end{array} 
\right) 
\left( 
\begin{array}{c} 
n \\
j+1
\end{array} 
\right) 
v_{j+2}v_{n-j}; \, \, \, \, \, \, \, \, n\geq 4, \, \, \, \, v_3=1,
}{1}\coordE{}\end{eqnarray} 
\[\coord{}\boxMath{
v_n =\frac{(n-3)!}{\pi^{2(n-3)}} \,Vol_{WP}\left({\cal M}_{0,n}\right).
}{corchetes}{0pt}\coordE{}\]
The origin of the two-term structure of (\ref{zografrecurence}) 
is easy to understand within Zograf's approach to the problem. He 
is executing the integral of one of the \myHighlight{$n-3$}\coordHE{} two forms in 
(\ref{wpvolume}). The homology cycles separates the punctures 
on \myHighlight{$\hat{C}$}\coordHE{} into two groups leading 
to a factorization into the product of the volumes of two spheres, 
each with a smaller number of punctures, as a result of the 
localization of the two-dimensional integration on the Deligne, 
Knudsen, Mumford boundary \cite{deligne} of \myHighlight{${\cal M}_{0,n}$}\coordHE{}. 
The missing link for a perfect analogy between the two problems 
relies on the fact that we are not able to localize the integrals 
over the instanton moduli space somewhere. 

A crucial 
point for any further progress in this matter will be - to our 
opinion - that one finds a (so far missing) idea as to which geometrical 
fact the three-term recursion could be attached. (In the case 
of punctured spheres it is the effectiveness of a lasso as 
a tool for catching whatever one wants to catch in two dimensions.)
For that purpose a detailed knowledge of second cohomology classes 
\myHighlight{$H^2$}\coordHE{} of the space \myHighlight{$\hat{{\cal M}}'_n$}\coordHE{} would be of prime importance. 
The complete analogy between punctured Riemann spheres and 
the SW prepotential in four dimensions, enthusiastically announced 
in \cite{matone1} has still to be found. We hope that our 
representation (\ref{restmodspaceint2}) of the coefficients 
\myHighlight{${\cal F}_n$}\coordHE{} as formal 
intersection numbers may be a useful step into this direction.   

\section*{Acknowledgments}
R.F. owes his initiation to the subject of the Seiberg-Witten 
potentials to the late Lochlainn O'Raifeartaigh and to Ivo Sachs. 
This is gratefully  acknowledged.
R.P. acknowledges the partial financial support of INTAS 00-561 
and of the Alexander von Humboldt and Folkswagen foundations of Germany.

\begin{thebibliography}{99}\addcontentsline{toc}{section}{References}
\bibitem{sw1}N. Seiberg, E. Witten, Nucl.Phys. B426 (1994) 19-52; 
Erratum-ibid. B430 (1994) 485-486, hep-th/9407087. 
\bibitem{erdelyi}H. Bateman, A. Erd\'{e}lyi, Higher Transcendental 
Functions, Vol. 3 (McGraw-Hill, New York, 1953).
\bibitem{s2}N. Seiberg, Phys.Lett. B318 (1993) 469-475,hep-ph/9309335  
\bibitem{flume} R. Flume, M. Magro, L. O'Raifeartaigh, I. Sachs, O. Schnetz, 
Nucl.Phys. B494 (1997) 331-345, hep-th/9611123 
\bibitem{dorey1}N. Dorey, V.V. Khoze, M.P. Mattis, 
Phys.Rev. D54 (1996) 2921-2943, hep-th/9603136.
\bibitem{fps}R. Flume, R. Poghossian, H. Storch, The coefficients of 
the Seiberg-Witten prepotential as intersection numbers (?), hep-th/0110240, 
To be published in the collection ``From Integrable Models to Gauge 
Theories'' (World Scientific, Singapore, 02) to honour Sergey Matinyan at 
the occasion of his 70'th birthday,
\bibitem{dorey2}N. Dorey, V.V. Khoze, M. Mattis, 
Phys.Rev. D54 (1996) 7832-7848, hep-th/9607202. 
\bibitem{dorey3}N. Dorey, V.V. Khoze, M.P. Mattis, 
Nucl.Phys. B513 (1998) 681-708, hep-th/9708036.
\bibitem{finnel}D. Finnell (SLAC), P. Pouliot, Nucl.Phys. B453 (1995) 
225-239, 
hep-th/9503115. 
\bibitem{ito}K. Ito, N. Sasakura, Phys.Lett. B382 (1996):95-103, 
hep-th/9602073.
\bibitem{yung} A. Yung, Nucl.Phys. B485 (1997) 38-62,  hep-th/9605096.
\bibitem{tanzini1} D. Bellisai, F. Fucito, A. Tanzini, G. Travaglini, 
JHEP 0007 (2000) 017, hep-th/0003272.
\bibitem{tanzini2} U. Bruzzo, F. Fucito, A. Tanzini, G. Travaglini, 
Nucl.Phys. B611 (2001) 205-226, hep-th/0008225.  
\bibitem{adhm}M. Atyah, V. Drinfeld, N. Hitchin, Yu. Manin, 
Phys. Lett. A65 (1978) 185. 
\bibitem{gribov}V. Gribov, Nucl. Phys. B139 (1978) 1.
\bibitem{berline}N. Berline, E. Getzler and M. Vergne, 
Heat Kernels and Dirac Operators, Springer, Berlin, 1996. 
\bibitem{cartan}H. Cartan, La transgression dans un groupe de Lie 
et dans un \'{e}space fibr\'{e} principal, in Colloque de Topologie, 
C. B. R. M., Bruxelles, 1950, pp 57-71.
\bibitem{schwarz} A. Schwarz, O. Zaboronsky, 
Commun.Math.Phys. 183 (1997) 463-476, hep-th/9511112.
\bibitem{donaldson}S. K. Donaldson, P. B. Kronheimer, The Geometry of 
Four-Manifolds, Oxford University Press, 1990.
\bibitem{uhlenbeck}D. S. Freed and K. K. Uhlenbeck, Instantons and 
Four-Manifolds, Springer, Berlin, 1984
\bibitem{bott}R. Bott and L. Tu, Differential Forms in Algebraic 
Geometry, Springer - verlag, New York, 1982.
\bibitem{matone1}M. Matone, JHEP 0104 (2001) 041, hep-th/0103246. 
\bibitem{zograf}P. Zograf, Contemporary Mathematics, v. 150 (1993) 367-372. 
\bibitem{deligne}P. Deligne and D. Mumford, J. H. E. S. Publ. Math. 
36(1969)75; \\
F. Knudsen and D. Mumford, Math. Scand. 39 (1976) 19.  
\end{thebibliography}
\vfill\eject
  


  



\end{document}

\bye
