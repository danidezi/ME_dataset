\documentclass[a4paper,12pt]{article}
\usepackage{a4}
\usepackage{amsmath}
\usepackage{amsfonts}
\usepackage{latexsym}
\usepackage{epsfig}
\setlength{\oddsidemargin}{-5mm}
\setlength{\evensidemargin}{-5mm}
\setlength{\topmargin}{-1cm}
\setlength{\textwidth}{17cm}

\begin{document}

%
% Title page
%
\begin{flushright}
hep-th/0211126
\end{flushright}
\vskip 2 cm
\begin{center}
{\Large {\bf
Anomalies on orbifolds with gauge symmetry breaking
}
}
\\[0pt]

\vskip 1.5cm
Hyun Min Lee\footnote{E-mail: minlee@th.physik.uni-bonn.de}
\\[0pt]
\vspace{0.23cm}
{\it  Physikalisches Institut, Universit\"at Bonn,}
{\it Nussallee 12, D-53115 Bonn, Germany.}\\
\bigskip
\vspace{3.4cm} Abstract\\[2ex]
\end{center}
We consider the breaking of $5D$ SUSY $G=SU(N+K)$ gauge symmetry
into $H=SU(N)\times SU(K)\times U(1)$ on an orbifold $S^1/(Z_2\times
Z'_2)$. There appear two independent fixed points: one respects
the full bulk gauge symmetry $G$ while the other contains only the
unbroken gauge symmetry $H$. In the model with one bulk $(N+K)$-plet,
giving a $K$-plet as the zero mode, we show that
localized non-abelian gauge anomalies appear at the fixed points: 
the $H^3$ gauge anomalies are equally distributed on
both fixed points and the $H-(G/H)-(G/H)$ gauge anomalies
contribute only at the fixed point with $G$ symmetry. 
We also find that when we add a brane $\bar K$-plet 
in the field theoretic limit of a bulk field, 
the theory with the unbroken
gauge group can be consistent up to the introduction of a bulk non-abelian 
Chern-Simons term. Moreover, for our set of bulk and brane fields, 
we find that  
a nonzero log Fayet-Iliopoulos term is radiatively generated 
at the fixed point with $H$, which gives rise to
the localization of the bulk zero mode at that fixed point and modifies 
the massive modes without changing the KK mass spectrum.

\vskip 1cm
\noindent Keywords: Gauge unified theories, Orbifolds, Gauge anomalies, 
Chern-Simons term, Fayet-Iliopoulos term.

\noindent PACS numbers: 11.10.Kk, 11.25.Mj, 12.10.-g.
\newpage

%%%%%%%%%%%%%%%%%%%%%%%%%%%%%%%%%%%%%%%%%%%%%%%%%%%%%%%%%%%%%%%%%%%%%%%%%%%%
\section{Introduction}

Recently the orbifold unification models in the existence of extra dimensions 
have 
drawn much attention due to their simplicity in performing the gauge symmetry 
breaking and the doublet-triplet splitting at the same time. 
The unwanted zero modes appearing in the unification models are projected out 
by boundary conditions in the extra dimension, i.e,  they get masses of order 
of the compactification scale. 
For instance,
the Minimal Supersymmetric Standard Model(MSSM) fields were obtained 
in the 5D SUSY SU(5) model where the extra dimension is compactified 
on a simple orbifold $S^1/(Z_2\times Z'_2)$\cite{kawamura,pointgroup,kkl0}. 
The idea was also taken in the model with the 5D $SU(3)$ electroweak 
unification with the TeV-sized extra dimension\cite{su3,kkl}, the possibility 
of which was first considered in the context of the string 
orbifolds\cite{anton}.
In the orbifold with gauge symmetry breaking, in general, 
in addition to the fixed point where the bulk gauge symmetry is operative, 
there exists a fixed point where only the unbroken gauge group is 
respected\cite{pointgroup}: for instance,
$G_{SM}=SU(3)\times SU(2)\times U(1)$ in the case with the 5D SUSY SU(5) model
on $S^1/(Z_2\times Z'_2)$.  
Therefore, 
we can put a multiplet(so called a brane field) at that fixed point 
allowed by the representation of
the unbroken gauge group. 
In the more
realisic model constructions, it has been proposed various possibilities
of having incomplete multiplets located at the orbifold fixed points
(or branes) with an unbroken gauge group\cite{pointgroup,kkl0,su3,kkl}.
For instance, in the 5D SU(5) GUT on $S^1/(Z_2\times Z'_2)$,
it has been shown that the $s-\mu$ puzzle can be understood
from the introduction of a split multiplet for $\bf 10$ of the second 
generation\cite{kkl0}
while the top-bottom mass hierarchy can be also explained with one Higgs in the
bulk and the other Higgs at the brane\cite{kkl0,kkl}. 
Moreover, introducing incomplete
multiplets for the quark setor is indispensible in the 5D SU(3) electroweak
unification on $S^1/(Z_2\times Z'_2)$\cite{su3,kkl}.

However, we should be careful about the anomaly problem in introducing
incomplete multiplets since localized gauge anomalies could appear on the
independent orbifold boundaries\cite{scrucca,pilo,barbieri,scrucca2,kkl,nilles}.
It has been shown\cite{ah} that the abelian anomalies coming from a bulk 
field in 5D
are equally distributed at the fixed points by the half of its 4D anomaly
while the bulk Chern-Simons term\cite{ch} plays a role
in conveying localized anomaly at one fixed point 
to the other fixed point.
So, the 4D anomaly cancellation for zero modes is sufficient
for consistency. This is also the case for the 5D non-abelian theory on the
orbifold with gauge symmetry breaking.
As far as the zero modes obtained after compactification are of anomaly-free
combination for the 4D gauge group as in the Minimal Supersymmetric Standard
Model(MSSM),
there was an examplary proof in 5D $SU(5)$ and $SU(3)$ orbifold unification 
models\cite{kkl}
that there is no localized anomalies
irrespective of the locations of the zero modes up to the introduction
of a bulk Chern-Simons term.

Besides the localized gauge anomalies,
there is a possibility of having the localized Fayet-Iliopoulos terms
at the fixed points for the case with a $U(1)$ factor
included in the remaining gauge group after orbifold 
compactification\cite{nilles2,barbieri,nilles,pomarol}.
In the 5D $U(1)$ gauge theory on $S^1/Z_2$,
it has been shown that non-vanishing
localized Fayet-Iliopoulos terms are radiatively induced from a general set of
charged brane and bulk fields\cite{nilles}. 
In that case, even if there is no effective 4D
FI term for the non-anomalous spectrum of zero modes,
the supersymmetric condition is satisfied
only if the real adjoint scalar in the vector multiplet develops
a vacuum expectation value to compensate 
localized FI terms\cite{peskin,ah2,barbieri,nilles,pomarol}.
Particularly, for one pair of brane and bulk fields with opposite charges,
{\it quadratically divergent} FI terms localized at fixed points
give rise to not only
the dynamical localization of the bulk zero mode 
but also make the bulk massive modes decoupled. 
Even in the case
where quadratically divergent FI terms are cancelled locally
by introducing a brane field with the half charge 
at each fixed point, there remain
{\it logarithmically divergent} FI terms,
which does not modify either
the 4D supersymmetry or the KK spectrum of the bulk field
but still gives rise to the localization of the zero mode\cite{nilles}.

In this paper, we do the explicit calculation of the gauge anomalies 
in the case with the gauge symmetry breaking on orbifolds. 
For our purpose of a general application, 
we consider the 5D SUSY $G=SU(N+K)$ gauge theory on $S^1/(Z_2\times Z'_2)$, 
which is reduced to 
the 4D SUSY $H=SU(N)\times SU(K)\times U(1)$ gauge theory at the zero mode 
level by orbifold boundary conditions. 
In this case, there are two fixed points with different gauge groups: the full
gauge symmetry $G$ at $y=0$ and the unbroken gauge symmetry $H$ at $y=\pi R/2$. 
In the existence of a bulk hypermultiplet
in the fundamental representation of the bulk gauge group, 
we only leave a $K$-plet among the GUT multiplet as the zero mode by 
the boundary conditions consistent with the gauge symmetry breaking. 
Then, we obtain the localized non-abelian anomalies from a bulk fermion 
by decomposing the 5D fermions and gauge fields in terms of the bulk eigenmodes 
and using the standard results of 4D anomalies. 
As a result, the $H^3$ gauge anomalies are equally distributed 
at the fixed points while the $H-(G/H)-(G/H)$ gauge anomalies 
are located only at $y=0$. 

In addition to the bulk field giving rise to a $K$-plet as the zero mode,
we add a $\bar K$-plet at the fixed point $y=\pi R/2$. 
The brane $\bar K$-plet can be realized from a bulk field of 
$\overline{(N+K)}$
with a negative infinite kink mass in a gauge invariant way. 
In the case with $N+K=5(3)$ and $K=2$, the matter zero modes in the model 
correspond to two Higgs fields in MSSM.
Then, 
the $H^3$ gauge anomalies remains with opposite signs at the fixed points
while the other gauge anomalies remain nonzero only at $y=0$.
Consequently, we show that all the remaining localized gauge anomalies 
are cancelled exactly by
introducing a Chern-Simons 5-form with a jumping coefficient in the 5D 
action, which can be interpreted as being induced from the bulk heavy 
modes\cite{ch,ah,pilo}. 
The variation of this Chern-Simons term gives rise to the 4D gauge 
anomalies on the boundaries, which is exactly what is needed to cancel the 
gauge anomalies coming from the asymmetric assigning of two 4D fermions 
in the opposite representations under $H$. 

Moreover, we consider the localized FI terms in our model.
We show that due to the orbifold boundary conditions breaking the bulk gauge
symmetry into $H$, one $(N+K)$-plet gives rise to both a non-vanishing
quadratically divergent FI term of one 4D $K$-plet and
a non-vanishing log divergent FI term only at the fixed point with $H$.
Then, the quadratically divergent FI term from the bulk field
is cancelled by contribution from one brane $\bar K$-plet,
which is consistent with the absence of gravitational
mixed anomalies in this model. However, the log divergent
FI term coming from the bulk field remains non-vanishing at the fixed point
with a $U(1)$ factor.
In the existence of the log divergent FI term,
we show that the supersymmetric condition is satisfied independently of
the log divergent FI term only if the $U(1)$ gauge component of the
bulk real adjoint scalar takes a singular vacuum expectation value(VEV).
Consequently, we find that this odd mode with a non-zero VEV dynamically
localizes the bulk zero mode at the fixed point with $H$, so the bulk zero mode
locally cancels the anomalies from the brane field.  
On the other hand, the wave functions of bulk massive modes are also modified 
without changing their tree-level mass spectrum.
Thus, we argue that
such a parity violation due to the odd mode 
is encoded in the wave functions of massive modes,
which make up a parity-violating bulk Chern-Simons term to cancel 
the remaining Chern-Simons term.
Even though our analysis of anomalies and FI terms in this paper 
is concentrated on the case 
with a particular set of fields in representations of the gauge group, 
the results are also applicable to the general set of brane and bulk fields 
with no 4D gauge anomalies.  

Our paper is organized as follows. In the next section, we give an introduction
to the gauge symmetry breaking on orbifolds by adopting a specific example, 
the 5D SUSY $SU(N+K)$ gauge theory on $S^1/(Z_2\times Z'_2)$.
Then, in the section 3, for this GUT orbifold, 
we derive the detailed expression for the localized non-abelian anomalies
coming from a bulk fermion in the fundamental representation of $SU(N+K)$. 
The section 4 is devoted to the localization problem of a bulk fermion and the
cancellation of the localized gauge anomalies coming from a 4D anomaly-free 
combination of bulk and brane fermions. In the next two sections, 
we work out with 
the localized Fayet-Iliopoulos terms in our model 
and explain their physical implication.
Then, we conclude the paper in the last section. 


\section{Orbifold breaking of gauge symmetry}

Let us consider the five-dimensional SUSY $G=SU(N+K)$ gauge theory 
compactified
on an $S^1/(Z_2\times Z'_2)$ orbifold.
The fifth dimensional coordinate $y$ is compactified to a circle
$2\pi R\equiv 0$. Furthermore, the point $y=-a$ is identified to
$y=a$ ($Z_2$ symmetry) and the point $y=(\pi R/2)+a$ is identified
to $y=(\pi R/2)-a$ ($Z_2'$ symmetry).
Then, the fundamental region of
the extra dimension becomes the interval $[0,\frac{\pi R}{2}]$
between two fixed points $y=0$ and $y=\frac{\pi R}{2}$.

For the two $Z_2$ symmetries, one can define their actions $P$ and $P'$
within the configuration space of any bulk field:
\begin{eqnarray}
\phi(x,y)&\rightarrow& \phi(x,-y)=P\phi(x,y), \\
\phi(x,y')&\rightarrow& \phi(x,-y')=P'\phi(x,y')
\end{eqnarray}
where $y'\equiv y+\pi R/2$.
The $(P,P')$ actions can involve all the symmetries
of the bulk theory, for instance, the gauge symmetry and the R-symmetry
in the supersymmetric case.
In general, then, any bulk field $\phi$
can take one of four different Fourier expansions depending on their
pair of two $Z_2$ parities, $(i,j)$ as
\begin{equation}
\phi_{++}=\sum^\infty_{n=0}\sqrt{\frac{1}{2^{\delta_{n,0}}\pi R}}
\phi^{(2n)}_{++}(x^\mu)\cos\frac{2ny}{R}\label{mode1}
\end{equation}
\begin{equation}
\phi_{+-}=\sum^\infty_{n=0}\sqrt{\frac{1}{\pi R}}
\phi^{(2n+1)}_{+-}(x^\mu)\cos\frac{(2n+1)y}{R}\label{mode2}
\end{equation}
\begin{equation}
 \phi_{-+}=\sum^\infty_{n=0}\sqrt{\frac{1}{\pi R}}
\phi^{(2n+1)}_{-+}(x^\mu)\sin\frac{(2n+1)y}{R}\label{mode3}
\end{equation}
\begin{equation}
\phi_{--}=\sum^\infty_{n=0}\sqrt{\frac{1}{\pi R}}
\phi^{(2n+2)}_{--}(x^\mu)\sin\frac{(2n+2)y}{R}\label{mode4}
\end{equation}
where $x^\mu$ is the 4D space-time coordinate.

The minimal supersymmetry in 5D
corresponds to N=2 supersymmetry(or 8 supercharges)
in the 4D N=1 language.
Thus, a 5D chiral multiplet corresponds to an N=2 hypermultiplet consisting
of two N=1 chiral multiplets with opposite charges. Two 4D Weyl spinors make
up one 5D spinor. On the other hand,
a 5D vector multiplet corresponds to an N=2 vector multiplet composed of one
N=1 vector multiplet($V=(A_\mu,\lambda_1, D)\equiv V^q T^q$)
and one N=1 chiral multiplet
($\Sigma=((\Phi+iA_5)/\sqrt{2}, \lambda_2,F_\Sigma)\equiv \Sigma^q T^q$), 
which transforms in the adjoint representation of the bulk gauge 
group\footnote{We note $D=X_3-\partial_5\Sigma$ and 
$F_\Sigma=(X_1+iX_2)/\sqrt{2}$ 
in terms of the $SU(2)_R$ triplet $\vec X$ in $N=2$ supersymmetry.}.
Upon compactification, we consider the case
where one $Z_2$ breaks N=2 supersymmetry to N=1
while the other $Z_2$ breaks the bulk $G=SU(N+K)$ gauge group
to its subgroup $H=SU(N)\times SU(K)\times U(1)$.

For instance, a bulk chiral multiplet
$N+K$, which is composed of two chiral multiplets with opposite charges,
$H=(h,\psi, F_H)\equiv (H_1,H_2)^T$
and $H^c=(h^c,\psi^c)\equiv (H^c_1, H^c_2)$,
transforms under $Z_2$ and $Z'_2$ identifications as
\begin{eqnarray}
H(x,-y)&=&\eta P H(x,y), \ \ \ H^c(x,-y)=- \eta {\tilde H}(x,y)P^{-1} \\
H(x,-y')&=&\eta' P' H(x,y'), \ \ \ H^c(x,-y')=-\eta'H^c(x,y') P^{\prime -1}
\end{eqnarray}
where both $\eta$ and $\eta'$ can take $+1$ or $-1$, and 
$P^2=P^{\prime 2}=I_{N+K}$ where $I_{N+K}$ is the $(N+K)\times (N+K)$ 
identity matrix. 
Then, choosing the parity matrices as
\begin{eqnarray}
P= I_{N+K}, \ \ \ P'={\rm diag}(I_{K},-I_{N}),\label{parity}
\label{parity}
\end{eqnarray}
and with $\eta=\eta'=1$,
the corresponding N=1 supermultiplets are split as follows
\begin{eqnarray}
&H_1^{(2n)}:&[(++);(1,K,\frac{1}{K})],\ {\rm mass}=2n/R \\
&H_2^{(2n+1)}:&[(+-);(N,1,-\frac{1}{N})],
\ {\rm mass}=(2n+1)/R \\
&H_2^{c(2n+1)}:& [(-+);(\overline {N},1,\frac{1}{N})],
\ {\rm mass}=(2n+1)/R \\
&H_1^{c(2n+2)}:&
[(--);(1,\overline{K},-\frac{1}{K})],\ {\rm mass}=(2n+2)/R
\end{eqnarray}
where the brackets [ ] contain the quantum numbers of $Z_2\times
Z_2'\times SU(N)\times SU(K)\times U(1)$. Consequently,
upon compactification, there appears
a zero mode only from the $K$-plet among the bulk field components while
other fields get massive.

On the other hand, the bulk gauge multiplet
is transformed under the two $Z_2$ transformations respectively as
\begin{eqnarray}
V(x,-y)&=&PV(x,y)P^{-1}, \\
\Sigma(x,-y)&=&-P\Sigma(x,y)P^{-1}, \\
V(x,-y')&=&P'V(x,y')P^{\prime -1}, \\
\Sigma(x,-y')&=&-P'\Sigma(x,y')P^{\prime -1}.
\end{eqnarray}
Therefore, with the choice for the parity matrices in the
fundamental representation as Eq.~(\ref{parity}), the $G=SU(N+K)$
gauge symmetry is broken down to $H=SU(N)\times SU(K)\times U(1)$
because $P'$ does not commute with all the gauge generators of
$SU(N+K)$: $P' T^a P^{\prime -1}=T^a$ and $P' T^{\hat a}P^{\prime
-1}=-T^{\hat a}$ where $q=(a,\hat a)$ denote unbroken and broken
generators, respectively. Actually, due to the orbifold boundary conditions
for the gauge fields, we get the $Z'_2$ grading of $SU(N+K)$ as
\begin{eqnarray}
[T^a,T^b]=if^{abc}T^c, \,\, [T^a,T^{\hat b}]=if^{a{\hat b}{\hat c}}T^{\hat c},
\,\, [T^{\hat a}, T^{\hat b}]=if^{{\hat a}{\hat b}c}T^c\label{z2grade}
\end{eqnarray}
where $f^{ab{\hat c}}$ and $f^{{\hat a}{\hat b}{\hat c}}$ is zero
for the $Z'_2$ invariance. As will be shown in the next section, 
the gauge anomalies in our orbifold model respect this group structure. 
It is interesting to see that this $Z'_2$ graded algebra also appears 
in the case with the spontaneous breaking of the $SU(N+K)$ global symmetry. 

Consequently, upon compactification, the gauge multiplets of
$SU(N+K)$ are
\begin{eqnarray}
&V^{a(n)}:&[(++);(N^2-1,1)+(1,K^2-1)
+(1,1)],\ {\rm mass}=2n/R \\
&V^{{\hat a}(2n+1)}:&[(+-);(N,K)
+(\overline{N},\overline{K})],
\ {\rm mass}=(2n+1)/R \\
&\Sigma^{{\hat a}(2n+1)}:&[(-+);(N,K)
+(\overline{N},\overline{K})],
\ {\rm mass}=(2n+1)/R \\
&\Sigma^{a(2n+2)}:&[(--);(N^2-1,1)+(1,K^2-1)
+(1,1)],
\ {\rm mass}=(2n+2)/R
\end{eqnarray}
where the brackets [ ] contain the quantum numbers of $Z_2\times
Z_2'\times SU(N)\times SU(K)$. Therefore, the orbifolding retains
only the $SU(N)\times SU(K)\times U(1)$ gauge multiplets as
massless modes $V^{a(0)}$ while the KK massive modes for unbroken
and broken gauge bosons are paired up separately. Here we make an
interesting observation from our parity assignments that the
$G=SU(N+K)$ gauge symmetry is fully conserved at $y=0$ while only
the unbroken gauge group $H=SU(N)\times SU(K)\times U(1)$ is
operative at $y=\pi R/2$. Therefore, upon the
orbifold compactification, it is possible to put some incomplete
multiplets transforming only under the local gauge group at $y=\pi R/2$.
Actually, since the parity conservation is assumed in the Lagrangian,
each component of a gauge parameter $\omega=\omega^q T^q$ has the same $Z_2$ 
parities as those of the corresponding gauge field.
Therefore, in the existence of the two $Z_2$ symmetries, 
the bulk gauge transformation does not respect the full $SU(N+K)$ 
gauge transformation but it is restricted as follows 
\begin{eqnarray}
\delta A^a_M&=&\partial_M \omega^a+if^{abc}A^b_M \omega^c
+if^{a{\hat b}{\hat c}} A^{\hat b}_M\omega^{\hat c}, \label{gtransf1}\\
\delta A^{\hat a}_M &=&\partial_M \omega^{\hat a}+if^{{\hat a}{\hat b}c}
A^{\hat b}_M\omega^{c}+if^{{\hat a}b{\hat c}}A^b_M\omega^{\hat c}
\label{gtransf2}
\end{eqnarray} 
which are consistent with the $Z'_2$ graded algebra, eq.~(\ref{z2grade}).  
Particularly, since $\omega^{\hat a}$ takes the same parities $(+,-)$ 
as $A^{\hat a}_\mu$, the gauge transformation at $y=\pi R/2$ becomes the one
of the unbroken gauge group $H$ from eq.~(\ref{gtransf1}). 



\section{Non-abelian anomalies on orbifolds with gauge symmetry breaking}

A 5D fermion is not chiral in the 4D language. However, after orbifold
compactification of the extra dimension, a chiral fermion can be
obtained as the zero mode of a bulk non-chiral fermion.
Then, the chiral fermion gives rise to the 4D gauge anomaly after
integrating out the extra dimension. For the case with the 5D $U(1)$ gauge
theory on $S^1/Z_2$\cite{ah} or $S^1/(Z_2\times Z'_2)$\cite{scrucca},
it was shown that the 4D gauge anomaly coming from a zero mode is equally
distributed at the fixed points.
In this section, we do the anomaly analysis in the case
with the 5D $SU(N+K)$ gauge theory compactified on our gauge symmetry
breaking orbifold, $S^1/(Z_2\times Z'_2)$.

Let us consider a four-component bulk fermion in the fundamental representation
of $SU(N+K)$. Then, the action is
\begin{eqnarray}
S=\int d^4 x \int_0^{2\pi R} dy\, \bar{\psi}(iD\hspace{-2.5mm}/
-\gamma_5 D_5-m(y))\psi \label{5daction}
\end{eqnarray}
where $D\hspace{-2.5mm}/=\gamma^\mu D_\mu $ and
$D_M=\partial_M+iA_M$. Here $m(y)$ is a mass term for the bulk fermion and 
$A_M=A^q_M T^q$ is a classical
 non-abelian gauge field.

With the assignments of $Z_2$ and $Z'_2$ parities to a $(N+K)$-plet
hypermultiplet in the previous section, the fermion field transforms as
\begin{eqnarray}
\psi(y)=\gamma_5 P\psi(-y), \ \ \ \psi(y')=\gamma_5 P'\psi(-y')
\end{eqnarray}
where $P$ and $P'$ are given by Eq.~(\ref{parity}), acting in the group space.
Invariance of the action under two $Z_2$'s gives rise to
the conditions for the mass function
\begin{eqnarray}
m(y)=-m(-y), \ \ \ m(y')=-m(-y').
\end{eqnarray}
And the gauge fields also transform under $Z_2$ as
\begin{eqnarray}
A_\mu(y)&=&P A_\mu(-y) P^{-1}, \ \ \ A_5(y)=-P A_5(-y)P^{-1},
\end{eqnarray}
and we replace ($y\rightarrow y'$, $P\rightarrow P'$) for $Z'_2$ action.

Then, with $\psi=\psi^1+\psi^2$, where $1$ and $2$ denotes
$K$-plet and $N$-plet components respectively, the fermion field
is decomposed into four independent chiral components
\begin{eqnarray}
\psi^1=\psi^1_L+\psi^1_R, \ \ \psi^2=\psi^2_L+\psi^2_R
\end{eqnarray}
where
\begin{eqnarray}
\gamma_5\psi^1_{L(R)}=\pm\psi^1_{L(R)}, \ \
\gamma_5\psi^2_{L(R)}=\pm\psi^2_{L(R)}.
\end{eqnarray}

Due to the parity assignments, i.e., $(\pm,\pm)$ for
$\psi^1_{L(R)}$ and $(\pm,\mp)$ for $\psi^2_{L(R)}$, we can expand
each Weyl fermion in terms of KK modes
\begin{eqnarray}
\psi^1_{L(R)}(x,y)&=&\sum_n \psi^1_{L(R)n}(x)\xi^{(\pm\pm)}_n(y), \\
\psi^2_{L(R)}(x,y)&=&\sum_n\psi^2_{L(R)n}(x)\xi^{(\pm\mp)}_n(y),
\end{eqnarray}
with 
\begin{eqnarray}
(-\partial_5+m(y))(\partial_5+m(y))\xi^{(+\pm)}_n(y)
&=&M^2_n\xi^{(+\pm)}_n(y),\\
(\partial_5+m(y))(-\partial_5+m(y))\xi^{(-\mp)}_n(y)
&=&M^2_n\xi^{(-\mp)}_n(y)
\end{eqnarray}
where $M_n$ is the $n$th KK mass.
Here we note that $\xi$'s make an orthonormal basis for the function
on $[0,2\pi R)$:
\begin{eqnarray}
\int_0^{2\pi R}dy\, \xi^{(\pm\pm)}_m(y)\xi^{(\pm\pm)}_n(y)
=\int_0^{2\pi R}dy\, \xi^{(\pm\mp)}_m(y)\xi^{(\pm\mp)}_n(y)=\delta_{mn},\\
\int_0^{2\pi R}dy\, \xi^{(++)}_m(y)\xi^{(\pm\mp)}_n(y)
=\int_0^{2\pi R}dy\, \xi^{(--)}_m(y)\xi^{(\pm\mp)}_n(y)=0.
\end{eqnarray}

Under the gauge $A_5=0$\footnote{The result will be not changed 
in the case without a gauge condition\cite{barbieri}}, 
inserting the mode sum of the fermion
into the 5D action, we obtain
\begin{eqnarray}
S&=&\int d^4 x\bigg[\sum_{n}\overline{\psi^1_n}
(i\partial\hspace{-2.5mm}/-M_{2n})
\psi^1_n+\sum_n\overline{\psi^2_n}(i\partial\hspace{-2.5mm}/-M_{2n-1})\psi^2_n
\nonumber \\
&-&\sum_{m,n}\bigg(V_{mn}(A^a)+V_{mn}(A^i)+V_{mn}(B)+V_{mn}(A^{\hat
a})\bigg)\bigg]
\end{eqnarray}
where $\psi^1_n=\psi^1_{Ln}+\psi^1_{Rn}$ for $n>0$
($\psi^1_0=\psi^1_{L0}$), $\psi^2_n=\psi^2_{Ln}+\psi^2_{Rn}$, and 
$V_{mn}$'s denote gauge vertex couplings. The $G=SU(N+K)$ gauge
fields($A=A^q T^q$) can be decomposed into 
\begin{eqnarray}
(N+K)^2-1\rightarrow (N^2-1,1)+(1,K^2-1)+(1,1)
+(N,K)+(\overline{N},\overline{K}),
\end{eqnarray}
that is, $A^aT^a(a=1,\cdots,
N^2-1)$, $A^iT^i(i=1,\cdots,K^2-1)$, 
$A^{(N+K)^2-1}T^{(N+K)^2-1}\equiv B T^B$ gauge fields 
for the $H=SU(N)\times SU(K)\times U(1)$ group, 
and $A^{\hat a }(t^{\hat a})_{\alpha r}\equiv X^{\alpha r}
(\alpha=1,\cdots,N;r=1,\cdots,K)$
gauge fields for the $G/H$ group, respectively. 
Here, broken group generators are related to $t^{\hat a}$ as 
\begin{eqnarray}
T^{\hat a}\equiv 
\left(\begin{array}{rr}
0 & t^{\hat a}\\ (t^{\hat a})^\dagger & 0
\end{array}\right).
\end{eqnarray}
Then, $V_{mn}$'s are given by the following:
\begin{eqnarray}
V_{mn}(A^a)&=&J^{\mu a }_{mn(+-)}{\cal A}^{a(+-)}_{mn\mu}
+J^{\mu a}_{mn(-+)}{\cal A}^{a(-+)}_{mn\mu} \nonumber \\
V_{mn}(A^i)&=&J^{\mu i}_{mn(++)}{\cal A}^{i(++)}_{mn\mu}
+J^{\mu i}_{mn(--)}{\cal A}^{a(--)}_{mn\mu} \nonumber \\
V_{mn}(B)&=&J^{\mu B}_{mn(++)}{\cal B}^{(++)}_{mn\mu}
+J^{\mu B}_{mn(--)}{\cal B}^{(--)}_{mn\mu}+J^{\mu B}_{mn(+-)}{\cal
B}^{(+-)}_{mn\mu} +J^{\mu B}_{mn(-+)}{\cal
B}^{(-+)}_{mn\mu}\nonumber\\
V_{mn}(A^{\hat a})&=&J^{\mu{\hat a}}_{mn(+)}{\cal A}^{{\hat a}(+)}_{mn\mu} 
+J^{\mu{\hat a}}_{mn(-)}{\cal A}^{{\hat a}(-)}_{mn\mu}
\end{eqnarray}
where
\begin{eqnarray}
{\cal A}^{a(\pm\mp)}_{mn\mu}&=&\int_0^{2\pi R}dy \,
\xi^{(\pm\mp)}_m(y)\xi^{(\pm\mp)}_n(y)A^{a}_\mu(x,y), \\
{\cal A}^{i(\pm\pm)}_{mn\mu}&=&\int_0^{2\pi R}dy \,
\xi^{(\pm\pm)}_m(y)\xi^{(\pm\pm)}_n(y)A^{i}_\mu(x,y), \\
{\cal B}^{(\pm\pm)}_{mn\mu}&=&\int_0^{2\pi R}dy \,
\xi^{(\pm\pm)}_m(y)\xi^{(\pm\pm)}_n(y)B_\mu(x,y), \\
{\cal B}^{(\pm\mp)}_{mn\mu}&=&\int_0^{2\pi R}dy \,
\xi^{(\pm\mp)}_m(y)\xi^{(\pm\mp)}_n(y)B_\mu(x,y), \\
{\cal A}^{{\hat a}(\pm)}_{mn\mu}&=&\int_0^{2\pi R}dy \,
\xi^{(\pm\mp)}_m(y)\xi^{(\pm\pm)}_n(y)A^{\hat a}_\mu(x,y)
\end{eqnarray}
and
\begin{eqnarray}
J^{\mu a}_{mn(\pm\mp)}=\overline{\psi^2_m}\gamma^\mu
P_{\pm}T^{a}\psi^2_n,& & J^{\mu
i}_{mn(\pm\pm)}=\overline{\psi^1_m}
\gamma^\mu P_{\pm}T^{i}\psi^1_n, \\
J^{\mu B}_{mn(\pm\pm)}=\overline{\psi^1_m}\gamma^\mu
P_{\pm}T^{B}_{K\times K}\psi^1_n,& &J^{\mu B}_{mn(\pm\mp)}=\overline{\psi^2_m}
\gamma^\mu P_{\pm}T^B_{N\times N}\psi^2_n,
\end{eqnarray}
\begin{eqnarray}
J^{\mu{\hat a}}_{mn(\pm)}=\overline{\psi^2_m}\gamma^\mu
P_{\pm}t^{\hat a}\psi^1_n+\overline{\psi^1_m}\gamma^\mu
P_{\pm}(t^{\hat a})^\dagger\psi^2_n
\end{eqnarray}
with $P_{\pm}=(1\pm\gamma_5)/2$. 
Here a decomposition of $T^B$ is understood 
such as $T^B={\rm diag.}(T^B_{N\times N},T^B_{K\times K})$. 
We note that the chiral
current for the $SU(N+M)$ gauge symmetry is split into chiral
currents coupled to the unbroken and broken gauge fields. 

Applying the classical equations of motion and the standard
results for the 4D chiral anomalies\cite{ah,pilo,barbieri}, 
we can derive the anomalies
for the chiral currents classified above. 
By making an inverse Fourier-transformation by the convolution of the bulk 
eigenmodes, the 5D gauge vector current 
$J^{Mq}={\overline\psi}\gamma^MT^q\psi$ is given by 
\begin{eqnarray}
J^{\mu a}(x,y)&=&\sum_{m,n}(\xi^{(+-)}_m\xi^{(+-)}_n J^{\mu a}_{mn(+-)}
+\xi^{(-+)}_m\xi^{(-+)}_n J^{\mu a}_{mn(-+)}),\\
J^{\mu i}(x,y)&=&\sum_{m,n}(\xi^{(++)}_m\xi^{(++)}_n J^{\mu i}_{mn(++)}
+\xi^{(--)}_m\xi^{(--)}_n J^{\mu i}_{mn(--)}), \\
J^{\mu B}(x,y)&=&\sum_{m,n}(\xi^{(++)}_m\xi^{(++)}_n J^{\mu B}_{mn(++)} 
+\xi^{(--)}_m\xi^{(--)}_n J^{\mu B}_{mn(--)} \nonumber \\
&+&\xi^{(+-)}_m\xi^{(+-)}_n J^{\mu B}_{mn(+-)}
+\xi^{(-+)}_m\xi^{(-+)}_n J^{\mu B}_{mn(-+)}),\\
J^{\mu{\hat a}}(x,y)&=&\sum_{m,n}(\xi^{(+-)}_m\xi^{(++)}_n 
J^{\mu{\hat a}}_{mn(+)}+\xi^{(-+)}_m\xi^{(--)}_n J^{\mu{\hat a}}_{mn(-)})
\end{eqnarray}
and we can construct $J^{5 q}$ similarly. 
Consequently, it turns out 
that the divergence of the 5D gauge vector current is given 
in terms of the 4D gauge anomalies as 
\begin{eqnarray}
(D_M J^{M})^a(x,y)&=&f_2(y)({\cal Q}^a(A) +{\cal Q}^a(X)), \\
(D_M J^{M})^i(x,y)&=&f_1(y)({\cal Q}^i(A) +{\cal Q}^i(X)), \\
(D_M J^{M})^B(x,y)&=&f_1(y)({\cal Q}^B_+(A)+{\cal Q}^B_+(X)) 
+f_2(y)({\cal Q}^B_-(A)+{\cal Q}^B_-(X)), \\
(D_M J^{M})^{\hat a}(x,y)&=&f_1(y)({\cal Q}^{\hat a}_1(X)
+{\cal Q}^{\hat a}_+(X))+f_2(y)({\cal Q}^{\hat a}_2(X)
+{\cal Q}^{\hat a}_-(X))
\end{eqnarray}
where
\begin{eqnarray}
f_1(y)&=&\sum_n\bigg[(\xi^{(++)}_n(y))^2 -(\xi^{(--)}_n(y))^2\bigg]
=\frac{1}{4}\sum_n \delta(y-\frac{n\pi R}{2}), \\
f_2(y)&=&\sum_n\bigg[(\xi^{(+-)}_n(y))^2 -(\xi^{(-+)}_n(y))^2\bigg]
=\frac{1}{4}\sum_n (-1)^n\delta(y-\frac{n\pi R}{2}).
\end{eqnarray}
The localized gauge anomalies ${\cal Q}$'s are composed of two large parts: 
anomalies for unbroken group components and broken group components of the 5D
vector current. The anomalies for unbroken group components involve 
not only unbroken gauge fields 
\begin{eqnarray}
{\cal Q}^a(A)&=&\frac{1}{32\pi^2}(D^{abc}F^b_{\mu\nu}{\tilde F}^{c\mu\nu}(x,y)
+D^{abB} F^b_{\mu\nu}{\tilde F}^{B\mu\nu}(x,y)), \\
{\cal Q}^i(A)&=&\frac{1}{32\pi^2}D^{ijB}F^j_{\mu\nu}
{\tilde F}^{B\mu\nu}(x,y)+\frac{1}{32\pi^2}D^{ijk}F^j_{\mu\nu}
{\tilde F}^{k\mu\nu}(x,y), \\
{\cal Q}^B_+(A)&=&\frac{1}{32\pi^2}{\rm Tr}(T^B_{K\times K})^3
F^B_{\mu\nu}{\tilde F}^{B\mu\nu}(x,y)\nonumber \\
&+&\frac{1}{64\pi^2}{\rm Tr}(\{T^B_{K\times K},T^i\}T^j)F^i_{\mu\nu}
{\tilde F}^{j\mu\nu}(x,y), \\ 
{\cal Q}^B_-(A)&=&\frac{1}{32\pi^2}{\rm Tr}(T^B_{N\times N})^3
F^B_{\mu\nu}{\tilde F}^{B\mu\nu}(x,y) \nonumber \\
&+&\frac{1}{64\pi^2}{\rm Tr}(\{T^B_{N\times N},T^a\}T^b)F^a_{\mu\nu}
{\tilde F}^{b\mu\nu}(x,y), \\
{\cal Q}^B_+(A)+{\cal Q}^B_-(A)&=&\frac{1}{32\pi^2}(D^{BBB}
F^B_{\mu\nu}{\tilde F}^{B\mu\nu}(x,y)
+D^{Bij}F^i_{\mu\nu}{\tilde F}^{j\mu\nu}(x,y) \nonumber \\
&+&D^{Bab}F^a_{\mu\nu}{\tilde F}^{b\mu\nu}(x,y))\equiv {\cal Q}^B(A), 
\end{eqnarray}
but also broken gauge fields 
\begin{eqnarray}
{\cal Q}^a(X)&=&\frac{1}{64\pi^2}{\rm Tr}(T^a t^{\hat b}(t^{\hat c})^\dagger)
F^{\hat b}_{\mu\nu}{\tilde F}^{{\hat c}\mu\nu}(x,y)
=\frac{1}{32\pi^2}D^{a{\hat b}{\hat c}}
F^{\hat b}_{\mu\nu}{\tilde F}^{{\hat c}\mu\nu}(x,y),\\
{\cal Q}^i(X)&=&\frac{1}{64\pi^2}{\rm Tr}(T^i (t^{\hat b})^\dagger t^{\hat c})
F^{\hat b}_{\mu\nu}{\tilde F}^{{\hat c}\mu\nu}(x,y)
=\frac{1}{32\pi^2}D^{i{\hat b}{\hat c}}
F^{\hat b}_{\mu\nu}{\tilde F}^{{\hat c}\mu\nu}(x,y), \\
{\cal Q}^B_+(X)&=&\frac{1}{64\pi^2}{\rm Tr}
(\{T^B,(t^{\hat b})^\dagger\}t^{\hat c})
F^{\hat b}_{\mu\nu}{\tilde F}^{{\hat c}\mu\nu}(x,y),  \\
{\cal Q}^B_-(X)&=&\frac{1}{64\pi^2}{\rm Tr}
(\{T^B,t^{\hat b}\}(t^{\hat c})^\dagger)
F^{\hat b}_{\mu\nu}{\tilde F}^{{\hat c}\mu\nu}(x,y), \\
{\cal Q}^B_+(X)+{\cal Q}^B_-(X)
&=&\frac{1}{32\pi^2}D^{B{\hat b}{\hat c}}
F^{\hat b}_{\mu\nu}{\tilde F}^{{\hat c}\mu\nu}(x,y)\equiv {\cal Q}^B(X).
\end{eqnarray}
On the other hand, the anomalies for broken group components of the 5D vector 
current become 
\begin{eqnarray}
{\cal Q}^{\hat a}_1(X)&=&\frac{1}{64\pi^2}{\rm Tr}
((\{t^{\hat a},(t^{\hat b })^\dagger\}
+\{(t^{\hat a})^\dagger,t^{\hat b}\})T^a)
F^{\hat b}_{\mu\nu}{\tilde F}^{a\mu\nu}(x,y) \nonumber \\
&=&\frac{1}{32\pi^2}D^{{\hat a}{\hat b}a}
F^{\hat b}_{\mu\nu}{\tilde F}^{a\mu\nu}(x,y),\\
{\cal Q}^{\hat a}_2(X)&=&\frac{1}{64\pi^2}{\rm Tr}
((\{t^{\hat a},(t^{\hat b})^\dagger\}
+\{(t^{\hat a})^\dagger,t^{\hat b}\})T^i)
F^{\hat b}_{\mu\nu}{\tilde F}^{i\mu\nu}(x,y)  \nonumber \\
&=&\frac{1}{32\pi^2}D^{{\hat a}{\hat b}i}
F^{\hat b}_{\mu\nu}{\tilde F}^{i\mu\nu}(x,y),\\
{\cal Q}^{\hat a}_+(X)&=&\frac{1}{64\pi^2}{\rm Tr}
(((t^{\hat a})^\dagger t^{\hat b}+(t^{\hat b })^\dagger t^{\hat a})
T^B_{K\times K})F^{\hat b}_{\mu\nu}{\tilde F}^{B\mu\nu}(x,y) \nonumber \\
&=&\frac{1}{64\pi^2}{\rm Tr} (\{T^{\hat a},T^{\hat b}\}T^B_{K\times K})
F^{\hat b}_{\mu\nu}{\tilde F}^{B\mu\nu}(x,y), \\
{\cal Q}^{\hat a}_-(X)&=&\frac{1}{64\pi^2}{\rm Tr}
((t^{\hat a} (t^{\hat b})^\dagger+t^{\hat b} (t^{\hat a})^\dagger)
T^B_{N\times N})F^{\hat b}_{\mu\nu}{\tilde F}^{B\mu\nu}(x,y) \nonumber\\ 
&=&\frac{1}{64\pi^2}{\rm Tr} (\{T^{\hat a},T^{\hat b}\}T^B_{N\times N})
F^{\hat b}_{\mu\nu}{\tilde F}^{B\mu\nu}(x,y), \\
{\cal Q}^{\hat a}_+(X)+{\cal Q}^{\hat a}_-(X)&=&
\frac{1}{32\pi^2}D^{{\hat a}{\hat b}B}F^{\hat b}_{\mu\nu}
{\tilde F}^{B\mu\nu}(x,y) \equiv {\cal Q}^{\hat a}_3(X)
\end{eqnarray}
In all the expressions for the anomalies above,  
we note that $D^{abc}$ denotes the symmetrized trace of group generators
\begin{eqnarray}
D^{abc}=\frac{1}{2}{\rm Tr}(\{T^a,T^b\}T^c)
\end{eqnarray} 
and other $D$ symbols with different group idices are similarly understood. 

As a result, we find that a bulk fermion gives rise to the localized gauge 
anomalies for all gauge components of the 5D vector current.  
Since the broken gauge fields vanish at $y=\pi R/2$ due to their
boundary conditions, the localized gauge anomalies 
at $y=\pi R/2$ are only ${\cal Q}(A)$'s, i.e., the $H^3$ gauge anomalies. 
However, at the other fixed point $y=0$, in addition to ${\cal Q}(A)$'s, 
there also appear the localized gauge 
anomalies ${\cal Q}(X)$'s, i.e., the $H-(G/H)-(G/H)$ gauge anomalies.  
We note that there is no anomalies of the type $AAX$ or $XXX$ 
since their anomaly coefficients automatically vanishes due to the group 
structure.(Here $A$ denotes the unbroken gauge field with $(+,+)$ 
while $X$ denotes the broken gauge field with $(+,-)$.)
With this in mind and restricting to the region $[0,2\pi R)$, 
we can rewrite the divergence of the 5D vector current as
\begin{eqnarray}
(D_M J^{M})^a(x,y)&=&\frac{1}{2}\bigg(\delta(y)-\delta(y-\frac{\pi R}{2})\bigg)
{\cal Q}^a(A) 
+\frac{1}{2}\delta(y){\cal Q}^a(X), \label{am1}\\
(D_M J^{M})^i(x,y)&=&\frac{1}{2}\bigg(\delta(y)+\delta(y-\frac{\pi R}{2})\bigg)
{\cal Q}^i(A) 
+\frac{1}{2}\delta(y){\cal Q}^i(X), \label{am2}\\
(D_M J^{M})^B(x,y)&=&\frac{1}{2}\bigg(\delta(y)+\delta(y-\frac{\pi R}{2})\bigg)
{\cal Q}^B_+(A)
+\frac{1}{2}\bigg(\delta(y)-\delta(y-\frac{\pi R}{2})\bigg){\cal Q}^B_-(A)
\nonumber \\
&+&\frac{1}{2}\delta(y){\cal Q}^B(X),\label{am3}\\
(D_M J^{M})^{\hat a}(x,y)&=&\frac{1}{2}\delta(y) {\cal Q}^{\hat a}(X)\label{am4}
\end{eqnarray} 
where ${\cal Q}^{\hat a}(X)\equiv 
{\cal Q}^{\hat a}_1(X)+{\cal Q}^{\hat a}_2(X)+{\cal Q}^{\hat a}_3(X)$.




\section{Localization of a bulk field and anomaly problem}

As shown in the section 2, we can freely put some brane fields  
consistently with the local gauge symmetries at the fixed points: 
a brane field at $y=0$ should be a representation of $SU(N+K)$ while
a brane field at $y=\pi R/2$ should be a representation 
of $SU(N)\times SU(K)\times U(1)$. Since we assume that a bulk fermion gives 
rise to a $K$-plet as the zero mode and we want to have 
the anomaly-free theory at least at the zero mode level, 
we can only put a brane field of $\bar K$-plet at $y=\pi R/2$. 
This introduction of an incomplete brane multiplet is
sufficient for the 4D anomaly-free theory at low energies but it could be 
inconsistent due to the existence of the localized gauge anomalies 
on the boundaries of the extra dimension. In this section, we consider
the localization of a bulk fermion with a kink mass and subsequently deal 
with the appearing anomaly problem by using the results in the previous 
section. 

It was shown in the literature that the localization of a bulk fermion 
can be realized by introducing a kink mass in the Lagrangian and 
even a brane fermion is possible in the limit of a kink mass being 
infinite\cite{ah}. 
In the 5D $U(1)$ gauge theory on $S^1/Z_2$ with a single bulk fermion, 
as a result of introducing an infinite kink mass,
the anomaly contribution from a bulk fermion on the boundaries of the extra
dimension was interpreted as the sum 
of contributions from a brane fermion and a parity-violating 
Chern-Simon term in 5D\cite{ah}. 
In other words, as a kink mass becomes infinite, 
heavy KK modes are decoupled but their effects remain as a local counterterm
such as the 5D Chern-Simon term. The similar observation has been made
for the non-abelian anomalies on orbifolds\cite{kkl}. 

In our case with gauge symmetry breaking on orbifolds, however,
we should be careful about the sign of a kink mass because an infinite kink 
mass could give rise to the localization of the unwanted bulk modes 
as massless modes\cite{barbieri,pomarol,hebecker}. For instance, 
a positive(negative) infinite kink mass for the even modes 
($(+,+)$ and $(-,-)$) 
gives rise to a localization of the massless mode for $(+,+)$ 
at $y=0(y=\pi R/2)$. 
On the other hand, 
a positive infinite kink mass for the odd modes ($(+,-)$ and $(-,+)$) 
could lead to new massless modes localized at $y=0$ and $y=\pi R/2$, 
respectively. 
Suppose that there are the universal(preserving the bulk gauge symmetry) 
kink masses for even and odd modes, i.e., 
$m(y)=M\epsilon(y)I_{(N+K)\times (N+K)}$ in eq.~(\ref{5daction}) 
where $\epsilon(y)$ is the sign function with periodicity $\pi R$. 
Then, in order to avoid unwanted massless modes in the limit of the kink mass 
being infinite, we only have to take the sign of $M$ to be negative. 
That is to say, when we introduce a bulk multiplet $\overline{(N+K)}$ 
with $M\rightarrow -\infty$, 
we obtain a massless $\bar K$-plet only from the $(+,+)$ mode, 
which is localized at $y=\pi R/2$, while other modes get decoupled 
from the theory. 
Thus, in this respect, a brane $\bar K$-plet is naturally realized from a bulk 
complete multiplet in the field theoretic limit. 
In this process of localization, we find that 
the consistency with the incomplete brane field can be guaranteed 
with introducing a 5D Chern-Simons term\cite{kkl}, 
which would be interpreted as the 
effects from the decoupled heavy modes\cite{ch,ah,pilo}. 

When we introduce a brane $\bar K$-plet at $y=\pi R/2$, it gives rise to 4D
gauge anomalies such as $-{\cal Q}^i(A)$ and $-{\cal Q}^B_+(A)$ at that fixed 
point. Therefore, with the addition of the brane $\bar K$-plet to a bulk 
$(N+K)$-plet, the divergence of the 5D vector current is changed to
\begin{eqnarray}
(D_M J^{M})^a(x,y)&=&\frac{1}{2}\bigg(\delta(y)-\delta(y-\frac{\pi R}{2})\bigg)
{\cal Q}^a(A)+\frac{1}{2}\delta(y){\cal Q}^a(X), \label{bbanomaly1}\\
(D_M J^{M})^i(x,y)&=&\frac{1}{2}\bigg(\delta(y)-\delta(y-\frac{\pi R}{2})\bigg)
{\cal Q}^i(A)+\frac{1}{2}\delta(y){\cal Q}^i(X), \label{bbanomaly2}\\
(D_M J^{M})^B(x,y)&=&\frac{1}{2}\bigg(\delta(y)-\delta(y-\frac{\pi R}{2})\bigg)
{\cal Q}^B(A)
+\frac{1}{2}\delta(y){\cal Q}^B(X), \label{bbanomaly3}\\
(D_M J^{M})^{\hat a}(x,y)&=&\frac{1}{2}\delta(y)
{\cal Q}^{\hat a}(X) \label{bbanomaly4}.
\end{eqnarray}
Here we observe that the total localized gauge anomalies only involving
the unbroken gauge group(${\cal Q}(A)$'s) appear 
in the combination of $(\delta(y)-\delta(y-\pi R/2))$, so their integrated
gauge anomalies vanish. 
On the other hand, the anomalies involving 
broken gauge fields(${\cal Q}(X)$'s) 
remain nonzero even after integration because ${\cal Q}(X)$'s 
are nonzero only at $y=0$. This asymmetric localization of  
${\cal Q}(X)$'s reflects the difference between two fixed point groups. 
The existence of the localized gauge anomalies could make the theory with
the unbroken gauge group anomalous.  
However, these localized gauge anomalies can be 
exactly cancelled with the introduction of a Chern-Simons(CS) 5-form 
$Q_5[A=A^q T^q]$ with a jumping coefficient in the action\cite{kkl} 
\begin{eqnarray}
{\cal L}_{CS}=-\frac{1}{96\pi^2}\epsilon(y)Q_5[A]
\end{eqnarray}  
where $\epsilon(y)$ is the sign function with periodicity $\pi R$ and 
\begin{eqnarray}
Q_5[A]={\rm Tr}\bigg(AdAdA+\frac{3}{2}A^3 dA+\frac{3}{5}A^5\bigg).
\end{eqnarray}
The parity-odd function $\epsilon(y)$ in front of $Q_5$ is necessary 
for the parity invariance because
$Q_5$ is a parity-odd quantity according to our parity assignments for bulk
gauge fields, eqs.~(19)-(22). 
Under the gauge transformation 
$\delta A=d\omega+[A,\omega]\equiv D\omega$, 
\begin{eqnarray}
\delta Q_5=Q^1_4[\delta A,A]={\rm str}
\bigg(D\omega \,d(AdA+\frac{1}{2}A^3)\bigg)
\end{eqnarray}
where str means the symmetrized trace and the restricted gauge transformation
in eqs.~(\ref{gtransf1}) and (\ref{gtransf2}) is understood. 
Then, due to the sign function in front of $Q_5$, the variation 
of the Chern-Simons action gives rise to the 4D consistent anomalies on the 
boundaries
\begin{eqnarray}
\delta {\cal L}_{CS}&=&\frac{1}{48\pi^2}(\delta(y)-\delta(y-\frac{\pi R}{2}))
\,\epsilon^{\mu\nu\rho\sigma}\sum_{q=a,i,B}\omega^q{\rm str}(T^q\partial_\mu
(A_\nu\partial_\rho A_\sigma+\frac{1}{2}A_\nu A_\rho A_\sigma)) 
\nonumber \\
&+&\frac{1}{48\pi^2}\delta(y)\,\epsilon^{\mu\nu\rho\sigma}
\omega^{\hat a}{\rm str}
(T^{\hat a}\partial_\mu
(A_\nu\partial_\rho A_\sigma+\frac{1}{2}A_\nu A_\rho A_\sigma)).
\end{eqnarray} 
The consistent anomalies we obtained here can be changed to the covariant 
anomalies\cite{bardeen} by regarding the covariant non-abelian gauge current 
$J^q_\mu$ as being redefined from a non-covariant gauge current 
${\tilde J}^q_\mu$ as
\begin{eqnarray}
J^q_\mu(x,y)={\tilde J}^q_\mu(x,y)+U^q_\mu(x,y)
\end{eqnarray}
where 
\begin{eqnarray}
U^{q=(a,i,B)}_\mu&=&-\frac{1}{96\pi^2}(\delta(y)-\delta(y-\frac{\pi R}{2}))
\,\epsilon^{\mu\nu\rho\sigma}{\rm str}(T^q(A_\nu F_{\rho\sigma}
+F_{\rho\sigma}A_\nu-A_\nu A_\rho A_\sigma)) \nonumber \\
U^{q={\hat a}}_\mu&=&-\frac{1}{96\pi^2}\delta(y)
\,\epsilon^{\mu\nu\rho\sigma}{\rm str}(T^q(A_\nu F_{\rho\sigma}
+F_{\rho\sigma}A_\nu-A_\nu A_\rho A_\sigma)).
\end{eqnarray}
Consequently, when we take into account the fact that the broken gauge fields
are vanishing at $y=\pi R/2$, the CS term contributes to the anomaly for the 
5D covariant gauge current as 
\begin{eqnarray}
(D_M J^M)^{q_1=(a,i,B)}
&=&-\frac{1}{64\pi^2}(\delta(y)-\delta(y-\frac{\pi R}{2}))
\sum_{q_2,q_3=(b,j,B)}{\rm str}(T^{q_1}T^{q_2}T^{q_3})
F^{q_2}_{\mu\nu}F^{q_3\mu\nu}\nonumber \\
&-&\frac{1}{64\pi^2}\delta(y)
\sum_{q_2,q_3={\hat a}}{\rm str}(T^{q_1}T^{q_2}T^{q_3})
F^{q_2}_{\mu\nu}F^{q_3\mu\nu}, \label{csanomaly1}\\
(D_M J^M)^{q_1={\hat a}}&=&-\frac{1}{64\pi^2}\delta(y)
\sum_{q_2q_3={\hat b}(a,i,B)}{\rm str}(T^{q_1}T^{q_2}T^{q_3})
F^{q_2}_{\mu\nu}F^{q_3\mu\nu}\label{csanomaly2}
\end{eqnarray}
where $q_{1,2,3}$ run the bulk group indices.
It turns out that the CS contributions to the anomalies exactly cancel the 
remaining localized covariant gauge anomalies on the boundaries, 
eq.~(\ref{bbanomaly1})-(\ref{bbanomaly4}). 

\section{Fayet-Iliopoulos terms}

In our model, the only place where the $U(1)$-graviton-graviton anomalies 
could appear is the fixed point $y=\pi R/2$ with
the local gauge group including a $U(1)$ gauge factor. As argued 
in the literature\cite{kkl}, 
there is no gravitational counterpart $A\wedge R\wedge R$ 
of the 5D Chern-Simons term since the non-abelian gauge fields propagate 
in the bulk. It has been shown that the gravitational anomalies at $y=\pi R/2$
indeed cancel between the bulk and brane contributions without the need of
a bulk Chern-Simons term\cite{kkl}. Then, since both gravitational anomalies
and FI terms are proportional to the common factor ${\rm Tr}(q)$, 
where $q$ is the $U(1)$ charge operator, it seems that
the absence of the gravitational anomalies should guarantee
the absence of the FI terms which could also exist at $y=\pi R/2$. This is
the requirement for the stability of the 4D supersymmetric theory. 

In the orbifold models with an unbroken $U(1)$, however, it has been shown that 
the localized FI terms can be induced from a bulk field without breaking the 4D
supersymmetry\cite{scrucca,barbieri,nilles,pomarol}.
In this section, we present the explicit computation of
the Fayet-Iliopoulos(FI) terms\cite{nilles2,nilles}
for our set of bulk and brane fields in our model. 

The relevant part of the action for bulk($h,h^c$)
and brane($h_b$) scalar fields
for performing the FI term calculation is given by
\begin{eqnarray}
S&=&\int d^4 x\int^{2\pi R}_0 dy
\bigg[|\partial_M h|^2+|\partial_M h^{c\dagger}|^2
+g D^B(h^\dagger T^B h-h^c T^B h^{c\dagger}) \nonumber \\
&+&\delta(y-\frac{\pi R}{2})\bigg(|\partial_\mu h_b|^2
+gD^B h^\dagger_b q_b h_b\bigg)\bigg]
\end{eqnarray}
where $D^B$ imply the auxiliary field for the unbroken $U(1)$.
Denoting the bulk scalar fields as $h=(h_{++},h_{+-})^T$ and
$h^c=(h_{--},h_{-+})$, let us expand those in terms of bulk eigenmodes as
\begin{eqnarray}
h_{\pm\pm}(x,y)&=&\sum_n h_{(\pm\pm) n}(x)\xi^{(\pm\pm)}_n(y),\\
h_{\pm\mp}(x,y)&=&\sum_n h_{(\pm\mp) n}(x)\xi^{(\pm\mp)}_n(y).
\end{eqnarray}
As in the anomaly computation, inserting the above mode expansions 
in the 5D action gives
\begin{eqnarray}
S&=&\int d^4x\bigg[\sum_{\alpha,\beta=\pm}\sum_{m,n}
-h^\dagger_{(\alpha\beta) m}(x)
\bigg((\Box_4+M_n)\delta_{mn}
-gq_{\alpha\beta}D^{B(\alpha\beta)}_{mn}(x)\bigg) h_{(\alpha\beta) n}(x)
\nonumber \\
&-&h^\dagger_b(x)\bigg(\Box_4-gq_{--}
D^B(x,y=\frac{\pi R}{2})\bigg)h_b(x)\bigg]
\end{eqnarray}
where
\begin{eqnarray}
D^{B(\alpha\beta)}_{mn}(x)=\int^{2\pi R}_0 dy\,
\xi^{(\alpha\beta)}_m(y)\xi^{(\alpha\beta)}_n(y)D^B(x,y)
\end{eqnarray}
and
${\rm Tr}(T^B)=Kq_{++}+Nq_{+-}=0$, $q_{-+}=-q_{+-}$, $q_{--}=-q_{++}$,
and the introduction of a brane $\bar K$-plet with $q_b=q_{--}$ is understood.

From the one-loop tadpole diagram for the KK modes of auxiliary field $D^B$,
we can get the bulk and brane field contributions to the FI term
with the cutoff $\Lambda$ regularization as follows
\begin{eqnarray}
F(x)=F_{bulk}(x)+F_{brane}(x)
\end{eqnarray}
where
\begin{eqnarray}
F_{bulk}(x)=\sum_n \sum_{\alpha\beta}q_{\alpha\beta}T_n
D^{B(\alpha\beta)}_{nn}(x)
\end{eqnarray}
with
\begin{eqnarray}
T_n=ig\int\frac{d^4p}{(2\pi)^4}\frac{1}{p^2-M^2_n}
=\frac{g}{16\pi^2}\bigg(\Lambda^2-M^2_n\ln\frac{\Lambda^2+M^2_n}{M^2_n}\bigg),
\end{eqnarray}
and
\begin{eqnarray}
F_{brane}(x)=igKq_{--}D^B(x,y=\frac{\pi R}{2})\int\frac{d^4p}{(2\pi)^4}
\frac{1}{p^2}=\frac{gKq_{--}}{16\pi^2}\Lambda^2 D^B(x,y=\frac{\pi R}{2}).
\end{eqnarray}
Then, when we write the FI term in terms of the 5D field $D^B(x,y)$ as
\begin{eqnarray}
F(x)=\int^{2\pi R}_0dy\,f(y)D^B(x,y),
\end{eqnarray}
we make an inverse Fourier-transformation for the auxiliary field to
obtain the bulk profile for the FI term as
\begin{eqnarray}
f(y)=f_{even}(y)+f_{odd}(y)+f_{brane}(y)
\end{eqnarray}
where
\begin{eqnarray}
f_{even}(y)&=&gKq_{++}\sum_n T_n [|\xi^{(++)}_n|^2-|\xi^{(--)}_n|^2]\nonumber
\\
&=&\frac{gKq_{++}}{16\pi^2}\bigg[\frac{1}{2}\Lambda^2(\delta(y)
+\delta(y-\frac{\pi R}{2}))
+\frac{1}{4}\ln\frac{\Lambda}{\mu}(\delta^{\prime\prime}(y)
+\delta^{\prime\prime}(y-\frac{\pi R}{2}))\bigg],\\
f_{odd}(y)&=&gNq_{+-}\sum_n T_n [|\xi^{(+-)}_n|^2-|\xi^{(-+)}_n|^2]\nonumber
\\
&=&\frac{gNq_{+-}}{16\pi^2}\bigg[\frac{1}{2}\Lambda^2(\delta(y)
-\delta(y-\frac{\pi R}{2}))
+\frac{1}{4}\ln\frac{\Lambda}{\mu}(\delta^{\prime\prime}(y)
-\delta^{\prime\prime}(y-\frac{\pi R}{2}))\bigg], \\
f_{brane}(y)&=&\frac{gKq_{--}}{16\pi^2}\Lambda^2\delta(y-\frac{\pi R}{2}).
\end{eqnarray}
Here, prime denotes the derivative
with respect to the extra dimension coordinate. Consequently, the resultant
FI term is given by
\begin{eqnarray}
f(y)&=&\frac{g{\rm Tr}(T^B)}{32\pi^2}\bigg[\Lambda^2(\delta(y)
+\delta(y-\frac{\pi R}{2}))+\frac{1}{2}\ln\frac{\Lambda}{\mu}
\delta^{\prime\prime}(y)\bigg]+\frac{gKq_{++}}{32\pi^2}
\ln\frac{\Lambda}{\mu}\delta^{\prime\prime}(y-\frac{\pi R}{2}) \nonumber \\
&\equiv&c\,\delta^{\prime\prime}(y-\frac{\pi R}{2})
\end{eqnarray}
where we used ${\rm Tr}(T^B)=0$ in the last line and
\begin{eqnarray}
c\equiv \frac{gKq_{++}}{32\pi^2}\ln\frac{\Lambda}{\mu}.
\end{eqnarray}
We note that there is no FI term at $y=0$ with the full bulk gauge group,
which is as expected because there
is no $U(1)$ factor at this fixed point. Moreover, we find that there is no
conventional FI term with quadratic divergence even at $y=\pi R/2$
with a $U(1)$ factor, which is
consistent with the absence of mixed gravitational anomalies as argued
in $\cite{kkl}$. However, there exists a non-vanishing FI term
with logarithmic divergence at $y=\pi R/2$.



\section{Localization of a bulk zero mode via the log FI term}

In the 5D $U(1)$ gauge theory on $S^1/Z_2$, it has been shown that 
in the presence of the localized FI terms, 
the supersymmetric condition is satisfied
only if the real adjoint scalar in the vector multiplet develops
a vacuum expectation value\cite{peskin,ah2,barbieri,nilles}. 
Then, the localized FI
terms give rise to  not only the dynamical localization of the bulk zero mode  
but also make the bulk massive modes decoupled\cite{nilles}. 

In our case, there remains only the log FI term
at the fixed point with the unbroken gauge group.
This is different from the case in the 5D $U(1)$ gauge theory 
where the log FI term are equally distributed at the fixed points\cite{nilles}.
In this section, we present the physical implication of the log FI term in our
model for the localization of a bulk field and the anomaly cancellation. 
  
The effective potential in our model with no gauge field background 
is written as 
\begin{eqnarray}
V&=&\int^{2\pi R}_0 dy\,\bigg[-\frac{1}{2}\bigg(D^q+\partial_5\Phi^q
+g(h^\dagger T^q h-h^c T^q h^{c\dagger})
+(f(y)+\delta(y-\frac{\pi R}{2})gh^\dagger_b q_b h_b)
\delta^{qB}\bigg)^2\nonumber \\
&+&\frac{1}{2}\bigg(\partial_5\Phi^q
+g(h^\dagger T^q h-h^c T^q h^{c\dagger})
+(f(y)+\delta(y-\frac{\pi R}{2})gh^\dagger_b q_b h_b)
\delta^{qB}\bigg)^2 \nonumber \\
&+&2g^2|h^cT^q h|^2-|F^q_\Sigma+\sqrt{2}gh^c T^q h|^2
+|(\partial_5+g\Phi)h|^2+|(\partial_5-g\Phi)h^{c\dagger}|^2\bigg].
\end{eqnarray}
Then, the equation of motion for $D^q$ and $F^q_\Sigma$ and 
the supersymmetric condition with a zero vacuum energy give
\begin{eqnarray}
0&=&D^q=-\partial_5\Phi^q
-g(h^\dagger T^q h-h^c T^q h^{c\dagger})
-(f(y)+\delta(y-\frac{\pi R}{2})gh^\dagger_b q_b h_b)\delta^{qB}, \\ 
0&=&F^q_\Sigma=-\sqrt{2}gh^cT^q h, \\  
0&=&(\partial_5+g\Phi)h=(\partial_5-g\Phi)h^{c\dagger}.
\end{eqnarray} 
Therefore, for a gauge invariant background   
$\langle h\rangle=\langle h^c\rangle=\langle h_b\rangle=0$, 
$\Phi^q$ develops a singular VEV for the 
supersymmetric vacuum as
\begin{eqnarray}
\langle\Phi^q\rangle(y)=-\int^y_0 
dy f(y)\delta^{qB}=-c\,\delta'(y-\frac{\pi R}{2})\delta^{qB}
\end{eqnarray}
which breaks the $Z_2\times Z'_2$ parities spontaneously. 
Then, for the nonzero $\langle\Phi^B\rangle$, 
the shape of the zero mode is modified as
\begin{eqnarray}
h^{(0)}_{++}(y)=h^{(0)}_{++}(0)
{\rm exp}\bigg(-gq_{++}\int^y_0 dy \langle\Phi^B\rangle\bigg).
\end{eqnarray} 
Here we note that the delta function can be regularized as 
$\delta(y)=\frac{1}{2\rho}(0)$ for
$|y|<\rho(|y|>\rho)$ and the normalization for the zero mode is 
\begin{eqnarray}
\int^{2\pi R}_0 dy\,|h^{(0)}_{++}(y)|^2=1.
\end{eqnarray}
Consequently, as $\rho\rightarrow 0^+$, 
we obtain the wave function of the zero mode independently of the 
cutoff $\Lambda$ as 
\begin{eqnarray}
|h^{(0)}_{++}(y)|^2=\delta(y-\frac{\pi R}{2})+\delta(y-\frac{3\pi R}{2}).
\end{eqnarray}
Thus, the bulk zero mode is localized exactly at the fixed point $y=\pi R/2$
due to the log FI term. 
On the other hand, with a nonzero $\langle\Phi^B\rangle$, 
the equations of motion for the massive modes are given by
\begin{eqnarray}
(\partial^2_5+gq_{\alpha\beta}\partial_5\langle\Phi^B\rangle
-g^2q_{\alpha\beta}^2\langle\Phi^B\rangle^2+\lambda)h_{\alpha\beta}=0 
\end{eqnarray}
where $\lambda$ is the mass eigenvalue.
From the similar analysis in the appendix B of Ref.~\cite{nilles}, we find 
that the wave functions of massive modes are also modified due
to the nonzero $\langle\Phi^B\rangle$ but their mass spectrum is not changed.  
Because our model is supersymmetric, we obtain the same results
for the fermionic zero and massive modes as in the case with the scalar modes. 

As for the anomaly cancellation, the dynamically
localized bulk zero mode cancels the anomalies from the brane fermion locally.
On the other hand, since the anomalies coming from a bulk field is given 
as in eqs.~(\ref{am1}) to (\ref{am4}) by
the sum of zero mode and massive mode contributions irrespective of the bulk 
basis, the modified massive modes corresponds to a bulk Chern-Simons term, 
which cancel the remaining Chern-Simons term, eqs.~(\ref{csanomaly1}) 
and (\ref{csanomaly2}).


\section{Conclusion}
We considered the breaking of the 5D non-abelian gauge symmetry 
on $S^1/(Z_2\times Z'_2)$ orbifold. 
Then, we presented the localized gauge anomalies coming from a bulk 
fundamental field through the explicit KK mode decomposition of the 5D fields. 
In the orbifold with gauge symmetry breaking, there are fixed points with their
own local gauge symmetries. Thus, there is the possibility of embedding
some incomplete multiplets at the fixed point with unbroken gauge group,
which can be sometimes phenomenologically preferred. The incomplete brane 
multiplet we considered can be realized from a bulk muliplet in the field 
theoretic limit. Therefore, we have shown that
the 4D anomaly combination of a brane field and a bulk zero mode  
does not have the localized gauge anomalies 
up to the addition of a Chern-Simon 5-form with some jumping coefficient, 
which could be regarded as the effects of the bulk heavy modes as in the 
abelian gauge theory on $S^1/Z_2$.  
Then, we found a nonzero log FI term at the fixed point with $H$, 
which dynamically localizes the bulk zero mode at that fixed point 
while the wave functions of massive modes are modified to make up a bulk 
Chern-Simons term. 

\vskip 1cm 
\noindent {\bf\large Acknowledgements} 
\vskip 0.5cm

\noindent I would like to thank H. D. Kim, J. E. Kim, H. P. Nilles, 
M. Olechowski and M. Walter for reading the draft and useful discussions.
This work is supported in part by the KOSEF Sundo Grant, and by the 
European Community's Human Potential Programme under contracts 
HPRN-CT-2000-00131 Quantum Spacetime, HPRN-CT-2000-00148 Physics Across the
Present Energy Frontier and HPRN-CT-2000-00152 Supersymmetry and the Early
Universe. I was supported by priority grant 1096 of the Deutsche 
Forschungsgemeinschaft.

%%%%%%%%%%%%%%%%%%%%%%%%%%%%%%%%%%%%%%%%%%%%%%%%%%%%%%%%%%%%%%%%%%%%%%%%%%%%

%
% References
%
\begin{thebibliography}{99}

\def\apj#1#2#3{Astrophys.\ J.\ {\bf #1} (#2) #3}
\def\ijmp#1#2#3{Int.\ J.\ Mod.\ Phys.\ {\bf #1} (#2) #3}
\def\mpl#1#2#3{Mod.\ Phys.\ Lett.\ {\bf #1} (#2) #3}
\def\nat#1#2#3{Nature\ {\bf #1} (#2) #3}
\def\npb#1#2#3{Nucl.\ Phys.\ {\bf B #1} (#2) #3}
\def\plb#1#2#3{Phys.\ Lett.\ {\bf B #1} (#2) #3}
\def\prd#1#2#3{Phys.\ Rev.\ {\bf D #1} (#2) #3}
\def\prl#1#2#3{Phys.\ Rev.\ Lett.\ {\bf #1} (#2) #3}
\def\prt#1#2#3{Phys.\ Rep.\ {\bf #1} (#2) #3}
\def\sjnp#1#2#3{Sov.\ J.\ Nucl.\ Phys.\ {\bf #1} (#2) #3}
\def\zp#1#2#3{Z.\ Phys.\ {\bf C #1} (#2) #3}
\def\jhep#1#2#3{JHEP \ {\bf #1} (#2) #3}
\def\epjc#1#2#3{Eur.\ Phys.\ J. \ {\bf C #1} (#2) #3}


\bibitem{kawamura} Y. Kawamura, Prog. Theor. Phys. {\bf 105} (2001) 691
[hep-ph/0012125];
G. Altarelli and F. Feruglio, \plb{511}{2001}{257}[hep-ph/0102301].

\bibitem{pointgroup}
L. J. Hall and Y. Nomura, \prd{64}{2001}{055003}[hep-ph/0103125];
A. Hebecker and J. March-Russell, \npb{613}{2001}{3}[hep-ph/0106166];
A. Hebecker and J. March-Russell, \npb{625}{2002}{128}[hep-ph/0107039];
L. J. Hall and Y. Nomura, \prd{65}{2002}{125012}[hep-ph/0111068];
L. J. Hall and Y. Nomura, hep-ph/0205067.   

\bibitem{kkl0}
H. D. Kim, J. E. Kim and H. M. Lee, \epjc{24}{2002}{159}[hep-ph/0112094].

\bibitem{su3}
S. Dimopoulos and D. E. Kaplan, \plb{531}{2002}{127}[hep-ph/0201148];
S. Dimopoulos, D. E. Kaplan and N. Weiner, 
\plb{534}{2002}{124}[hep-ph/0202136];
L. J. Hall and Y. Nomura, \plb{532}{2002}{111}[hep-ph/0202107];
T. -J Li and W. Liao, \plb{545}{2002}{147}[hep-ph/0202090].

\bibitem{kkl}
H. D. Kim, J. E. Kim and H. M. Lee, \jhep{06}{2002}{048}[hep-ph/0204132];
H. M. Lee, hep-th/0210177. 

\bibitem{anton} I. Antoniadis, \plb{246}{1990}{377};
I. Antoniadis and K. Benakli, \plb{326}{1994}{69}.

\bibitem{scrucca}
C. A. Scrucca, M. Serone, L. Silvestrini and F. Zwirner, 
\plb{525}{2002}{169}[hep-th/0110073].

\bibitem{pilo} 
L. Pilo and A. Riotto, \plb{546}{2002}{135}[hep-th/0202144].

\bibitem{barbieri}
R. Barbieri, R. Contino, P. Creminelli, R. Rattazzi and C. A. Scrucca,
\prd{66}{2002}{024025}[hep-th/0203039].

\bibitem{scrucca2}
C. A. Scrucca, M. Serone and T. Trapletti, \npb{635}{2002}{33} 
[hep-th/0203190].

\bibitem{nilles}
S. G. Nibbelink, H. P. Nilles and M. Olechowski, \npb{640}{2002}{171} 
[hep-th/0205012]. 

\bibitem{ah} N. Arkani-Hamed, A. G. Cohen and H. Georgi,
\plb{516}{2001}{395}[hep-th/0103135].

\bibitem{ch}
C. G. Callan and J. A. Harvey, \npb{250}{1985}{427}.

\bibitem{nilles2} D. M. Ghilencea, S. Groot Nibbelink and H. P. Nilles,
\npb{619}{2001}{385}[hep-th/0108184].

\bibitem{peskin}
E. A. Mirabelli and M. E. Peskin, \prd{58}{1998}{065002} [hep-th/9712214].

\bibitem{ah2}
N. Arkani-Hamed, T. Gregoire and J. Walker, \jhep{0203}{2002}{055} 
[hep-th/0101233]. 

\bibitem{pomarol} 
D. Marti and A. Pomarol, hep-ph/0205034.

\bibitem{hebecker} 
A. Hebecker and J. March-Russell, \plb{541}{2002}{338}
[hep-ph/0205143].

\bibitem{bardeen} W. Bardeen and B. Zumino, \npb{244}{1984}{421}.

\end{thebibliography}

\end{document}

