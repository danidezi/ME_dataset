
\documentclass[a4paper,12pt]{article}
\usepackage{amssymb}
\usepackage{useful_macros}
\begin{document}
\begin{center}


\textbf{CAN THE HIERARCHY PROBLEM BE SOLVED BY 
FINITE-TEMPERATURE  MASSIVE FERMIONS IN THE RANDALL-SUNDRUM MODEL?}

\bigskip

\bigskip

   I. Brevik\footnote{e-mail: iver.h.brevik@mtf.ntnu.no}  \\        

  \bigskip

\bigskip            

          Division of Applied Mechanics,
           Norwegian University of Science and Technology,
           N-7491 Trondheim, Norway \\           
\bigskip
\bigskip

 June 2001
\end{center} 

\bigskip        

\begin{abstract}

    Quantum effects of bulk matter, in the form of massive fermions, are considered in the Randall-Sundrum \myHighlight{$AdS_5$}\coordHE{} brane world at finite temperatures. The thermodynamic energy (modulus potential) is calculated in the limiting case when the temperature is low, and is shown to possess a minimum, thus suggesting a new dynamical mechanism for stabilizing the brane world. Moreover, these quantum effects may solve the hierarchy scale problem, at quite low temperatures. The present note reviews essentially the fermion-related part of the recent article by I. Brevik, K. A. Milton, S. Nojiri, and S. D. Odintsov, Nucl. Phys. {\bf B 599}, 305 (2001).
\end{abstract}


\bigskip
\bigskip
\bigskip
\bigskip

Submitted to Gravitation and Cosmology (G@C), special issue devoted to Quantum Gravity, Unified Models and Strings, to mark the 100th anniversary of Tomsk State Pedagogical University. Edited by Professor S. D. Odintsov.

         

\newpage
 

\section{Introduction}

Consider the Randall-Sundrum (RS) scenario \cite{randall99}, i.e., a non-factorizable geometry with one extra fifth dimension. This dimension, called \myHighlight{$y$}\coordHE{}, is compactified on an orbifold \myHighlight{$S^1/Z_2$}\coordHE{} of radius \myHighlight{$R$}\coordHE{} such that \myHighlight{$-\pi R \leq y \leq \pi R$}\coordHE{}. The orbifold fixed points at \myHighlight{$y=0$}\coordHE{} and \myHighlight{$y=\pi R$}\coordHE{} are the locations of the two three-branes. We will below, instead of \myHighlight{$y$}\coordHE{}, use the nondimensional coordinate \myHighlight{$\phi$}\coordHE{}, defined by \myHighlight{$\phi=y/R$}\coordHE{}, thus lying in the interval \myHighlight{$[-\pi, \pi]$}\coordHE{}. The RS  5D metric is
\begin{equation}\coord{}\boxEquation{
ds^2=e^{-2kR|\phi|}\eta_{\mu\nu} dx^\mu dx^\nu -R^2 d\phi^2,
}{
ds^2=e^{-2kR|\phi|}\eta_{\mu\nu} dx^\mu dx^\nu -R^2 d\phi^2,
}{ecuacion}\coordE{}\end{equation}
\label{1}
where \myHighlight{$\eta_{\mu\nu}$}\coordHE{}=diag(-1,1,1,1) with \myHighlight{$\mu=0,1,2,3$}\coordHE{}. The 5D metric is \myHighlight{$g_{MN}$}\coordHE{} with capital subscripts, \myHighlight{$M=(\mu, 5)$}\coordHE{}. The parameter \myHighlight{$k \sim 10^{19}$}\coordHE{} GeV is of Planck scale, related to the AdS radius of curvature, which is \myHighlight{$1/k$}\coordHE{}. The points \myHighlight{$(x^\mu,\phi)$}\coordHE{} and \myHighlight{$(x^\phi, -\phi)$}\coordHE{} are identified. The metric (1) is valid if the 5D cosmological constant \myHighlight{$\Lambda$}\coordHE{} and the 5D Planck mass \myHighlight{$M_5$}\coordHE{} are related through 
\begin{equation}\coord{}\boxEquation{
\Lambda=-6M_5^3k^2.
}{
\Lambda=-6M_5^3k^2.
}{ecuacion}\coordE{}\end{equation}
\label{2}
The cosmological constants at the boundaries have to fulfil the constraints \myHighlight{$\Lambda_{0}=-\Lambda_{\pi}=-\Lambda/k$}\coordHE{}. This is the fine-tuning problem in the RS model. The 4D Planck mass \myHighlight{$M_P$}\coordHE{} is related to \myHighlight{$M_5$}\coordHE{} through
\begin{equation}\coord{}\boxEquation{
M_P^2=\frac{M_5^3}{k}\left( 1-e^{-2\pi kR} \right).
}{
M_P^2=\frac{M_5^3}{k}\left( 1-e^{-2\pi kR} \right).
}{ecuacion}\coordE{}\end{equation}
\label{3}
The \myHighlight{$\phi=0$}\coordHE{} brane (Planck brane) is associated with the mass scale \myHighlight{$M_P$}\coordHE{}, whereas the \myHighlight{$\phi=\pi$}\coordHE{} brane (TeV-brane) is associated with the scale \myHighlight{$M_P e^{-\pi kR}$}\coordHE{} lying in the TeV region provided that \myHighlight{$kR \simeq 12$}\coordHE{}.

Assume now that there is a scalar field \myHighlight{$\Phi$}\coordHE{} in the bulk, with action
\begin{equation}\coord{}\boxEquation{
S_\Phi=\frac{1}{2}\int d^4 x\, \int_{-\pi}^{\pi} d\phi \sqrt{-G}\left\{ G^{AB}\partial_A \Phi \partial_B \Phi
-\left(m^2+\frac{2\alpha k}{R}(\delta(\phi)-\delta(\phi-\pi))\right) \Phi^2 \right\}.
}{
S_\Phi=\frac{1}{2}\int d^4 x\, \int_{-\pi}^{\pi} d\phi \sqrt{-G}\left\{ G^{AB}\partial_A \Phi \partial_B \Phi
-\left(m^2+\frac{2\alpha k}{R}(\delta(\phi)-\delta(\phi-\pi))\right) \Phi^2 \right\}.
}{ecuacion}\coordE{}\end{equation}
\label{4}
Here \myHighlight{$\alpha$}\coordHE{} is a nondimensional constant which parametrizes the mass on the boundaries. All fields in the 5D bulk can be regarded as Kaluza-Klein modes, which in turn can be considered as 4D fields on the brane with an infinite tower of masses. The mass spectrum \myHighlight{$m_n$}\coordHE{} of the Kaluza-Klein modes in \myHighlight{$\Phi$}\coordHE{} is given \cite{goldberger00, gherghetta00} by roots of
\begin{equation}\coord{}\boxEquation{
j_{\nu} (x_n)y_{\nu} (ax_n)-j_{\nu} (ax_n)y(x_n)=0.
}{
j_{\nu} (x_n)y_{\nu} (ax_n)-j_{\nu} (ax_n)y(x_n)=0.
}{ecuacion}\coordE{}\end{equation}
\label{5}
Here \myHighlight{$x_n=m_n/ak$}\coordHE{}, \myHighlight{$a=e^{-\pi kR}$}\coordHE{}, \myHighlight{$\nu=\sqrt{4+m^2/k^2}$}\coordHE{}, and the altered Bessel functions are
\begin{equation}\coord{}\boxEquation{
j_\nu(z)=(2-\alpha)J_\nu(z)+zJ_\nu'(z),~~~y_\nu(z)=(2-\alpha)Y_\nu(z)+zY_\nu'(z).
}{
j_\nu(z)=(2-\alpha)J_\nu(z)+zJ_\nu'(z),~~~y_\nu(z)=(2-\alpha)Y_\nu(z)+zY_\nu'(z).
}{ecuacion}\coordE{}\end{equation}
\label{6}

\section{Effective potential for fermions in 5D AdS space at finite temperature}

Following Ref. \cite{brevik01}, we consider the energy (effective potential) for a bulk quantum field on a 5D AdS background at finite temperature. For a fermion of momentum \myHighlight{${\bf p}$}\coordHE{} and mass m the partition function \myHighlight{$Z_p^f$}\coordHE{} is \myHighlight{$2\cosh(\beta E_p/2)$}\coordHE{}, where \myHighlight{$\beta=1/T$}\coordHE{} is the inverse temperature and \myHighlight{$E_p=\sqrt{{\bf p}^2+m^2}$}\coordHE{}. The total fermionic partition function \myHighlight{$Z^f$}\coordHE{} in a three-dimensional volume \myHighlight{$V$}\coordHE{} then becomes
\begin{equation}\coord{}\boxEquation{
\beta F^f=-\ln Z^f = -V\int \frac{d^3p}{(2\pi)^3}\,\ln \left[ 2\cosh \left( \frac{\beta E_p}{2} \right) \right],
}{
\beta F^f=-\ln Z^f = -V\int \frac{d^3p}{(2\pi)^3}\,\ln \left[ 2\cosh \left( \frac{\beta E_p}{2} \right) \right],
}{ecuacion}\coordE{}\end{equation}
\label{7}
\myHighlight{$F^f$}\coordHE{} being the free energy. The corresponding thermodynamic energy \myHighlight{$E^f$}\coordHE{} is
\begin{equation}\coord{}\boxEquation{
E^f=\frac{\partial}{\partial \beta}(\beta F^f)=-V\int \frac{d^3 p}{(2\pi)^3}\, \frac{E_p}{2} \tanh \left(\frac{\beta E_p}{2} \right).
}{
E^f=\frac{\partial}{\partial \beta}(\beta F^f)=-V\int \frac{d^3 p}{(2\pi)^3}\, \frac{E_p}{2} \tanh \left(\frac{\beta E_p}{2} \right).
}{ecuacion}\coordE{}\end{equation}
\label{8}
As already mentioned, the 5D Kaluza-Klein modes can be considered as 4D fields on the brane with an infinite tower of masses, so by summing up the KK modes given by Eq.~(5) we get the following expression for the total KK energy:
\begin{equation}\coord{}\boxEquation{
E^{fKK}=-{\cal{F}} \left[ \frac{\sqrt{{\bf p}^2+a^2k^2 x^2}}{2}\tanh \left( \frac{\beta}{2}\sqrt{{\bf p}^2+a^2k^2x^2}\right)  \right].
}{
E^{fKK}=-{\cal{F}} \left[ \frac{\sqrt{{\bf p}^2+a^2k^2 x^2}}{2}\tanh \left( \frac{\beta}{2}\sqrt{{\bf p}^2+a^2k^2x^2}\right)  \right].
}{ecuacion}\coordE{}\end{equation}
\label{9}
Here the functional \myHighlight{$\cal F$}\coordHE{} is defined by
\begin{equation}\coord{}\boxEquation{
{\cal F}[f(p,x)]=V\int \frac{d^3p}{(2\pi)^3}\,\frac{i}{2\pi}\int_C dx\frac{d}{dx}f(p,x)
\ln [j_\nu(x)y_\nu(ax)-j_\nu(ax)y_\nu(x)],
}{
{\cal F}[f(p,x)]=V\int \frac{d^3p}{(2\pi)^3}\,\frac{i}{2\pi}\int_C dx\frac{d}{dx}f(p,x)
\ln [j_\nu(x)y_\nu(ax)-j_\nu(ax)y_\nu(x)],
}{ecuacion}\coordE{}\end{equation}
\label{10}
the contour \myHighlight{$C$}\coordHE{} encircling the positive real axis.

Since the zero temperature contribution to the energy, \myHighlight{$E^f(\infty)$}\coordHE{}, has been calculated in Ref. \cite{goldberger00a}, we subtract this contribution from \myHighlight{$E^f(\beta)$}\coordHE{} in Eq.~(8) and consider henceforth
\begin{equation}\coord{}\boxEquation{
\tilde{E}^f(\beta)=E^f(\beta)-E^f(\infty).
}{
\tilde{E}^f(\beta)=E^f(\beta)-E^f(\infty).
}{ecuacion}\coordE{}\end{equation}
\label{11}
This quantity is finite. In terms of the variable \myHighlight{$q$}\coordHE{} defined via \myHighlight{$|{\bf p}|=q/\beta$}\coordHE{} we can write Eq.~(11) as
\begin{equation}\coord{}\boxEquation{
\tilde{E}^f(\beta)=\frac{V}{2\pi^2\beta^4}\int_0^\infty dq\,
 q^2\frac{\sqrt{q^2+\beta^2m^2}}{e^{\sqrt{q^2+\beta^2m^2}}+1}.
}{
\tilde{E}^f(\beta)=\frac{V}{2\pi^2\beta^4}\int_0^\infty dq\,
 q^2\frac{\sqrt{q^2+\beta^2m^2}}{e^{\sqrt{q^2+\beta^2m^2}}+1}.
}{ecuacion}\coordE{}\end{equation}
\label{12}
This expression holds for arbitrary temperatures. Assume now that the temperature is low, \myHighlight{$\beta \gg 1$}\coordHE{}. Then it is convenient to change the variable from \myHighlight{$q$}\coordHE{} to \myHighlight{$s$}\coordHE{} via \myHighlight{$q=\sqrt{s^2+2\beta ms}$}\coordHE{}, whereby
\begin{eqnarray}\coord{}\boxAlignEqnarray{\leftCoord{}
\tilde{E}^f(\beta)&=&\frac{\leftCoord{}V}{\rightCoord{}2\pi^2\beta^4}\int_0^\infty ds\rightCoord{}\,\frac{\leftCoord{}(s+\beta m)^2\sqrt{s^2+2\beta ms}}{\rightCoord{}e^{s+\beta m}+1}
                                                   \nonumber \rightCoord{}\\
&&\leftCoord{}\rightarrow \frac{\leftCoord{}Vm^{5/2}e^{-\beta m}}{\rightCoord{}(2\pi \beta)^{3/2}} \rightCoord{}
\rightCoord{}}{0mm}{5}{8}{
\tilde{E}^f(\beta)&=&\frac{V}{2\pi^2\beta^4}\int_0^\infty ds\,\frac{(s+\beta m)^2\sqrt{s^2+2\beta ms}}{e^{s+\beta m}+1}
                                                   \\
&&\rightarrow \frac{Vm^{5/2}e^{-\beta m}}{(2\pi \beta)^{3/2}} 
}{1}\coordE{}\end{eqnarray}
to leading order. We assume that \myHighlight{$\beta m \gg 1$}\coordHE{}; then, because of the factor \myHighlight{$e^{-\beta m}$}\coordHE{} we need to include only the lowest root \myHighlight{$x=x_1$}\coordHE{} in Eq.~(5), corresponding to \myHighlight{$n=1$}\coordHE{}. When \myHighlight{$a$}\coordHE{} is small, this yields \myHighlight{$j_\nu(x_1)=0$}\coordHE{} (we assume \myHighlight{$\alpha +\nu \neq 2$}\coordHE{}), so that \myHighlight{$x_1$}\coordHE{} is of order unity. Adding the contribution from the zero-point energy we find the following effective potential:
\begin{equation}\coord{}\boxEquation{
V^f(a)=k^4 B_2^f\left( a^{2\mu}+\frac{B_3^f}{B_2^f}(\beta k)^{-3/2}a^{5/2}e^{-\beta kax_1} \right),
}{
V^f(a)=k^4 B_2^f\left( a^{2\mu}+\frac{B_3^f}{B_2^f}(\beta k)^{-3/2}a^{5/2}e^{-\beta kax_1} \right),
}{ecuacion}\coordE{}\end{equation}
\label{14}
where \myHighlight{$\mu=\nu+2$}\coordHE{}. We have here defined
\begin{equation}\coord{}\boxEquation{
B_2^f=-\frac{V}{16\pi^2}\frac{2^{1-2\nu}}{\nu \Gamma(\nu)^2}\int_0^\infty dt\,t^{2\nu+3}\,\frac{K_\nu '(t)}{I_\nu '(t)} >0,
}{
B_2^f=-\frac{V}{16\pi^2}\frac{2^{1-2\nu}}{\nu \Gamma(\nu)^2}\int_0^\infty dt\,t^{2\nu+3}\,\frac{K_\nu '(t)}{I_\nu '(t)} >0,
}{ecuacion}\coordE{}\end{equation}
\label{15}
\begin{equation}\coord{}\boxEquation{
B_3^f=\frac{V}{(2\pi)^{3/2}}\,x_1^{5/2}.
}{
B_3^f=\frac{V}{(2\pi)^{3/2}}\,x_1^{5/2}.
}{ecuacion}\coordE{}\end{equation}
\label{16}
We have now come to the main point: when \myHighlight{$\beta k$}\coordHE{} is large, the effective potential (14) has a nontrivial {\it minimum}. The order of magnitude of \myHighlight{$a$}\coordHE{} at the minimum, \myHighlight{$a=a_m$}\coordHE{}, is given roughly by
\begin{equation}\coord{}\boxEquation{
a_m \sim \frac{1}{\beta k} \ln \beta k.
}{
a_m \sim \frac{1}{\beta k} \ln \beta k.
}{ecuacion}\coordE{}\end{equation}
\label{17}
As mentioned above, taking \myHighlight{$kR \simeq 12$}\coordHE{} means that the \myHighlight{$\phi=\pi$}\coordHE{} brane is associated with the TeV region. That means, \myHighlight{$e^{-\pi kR}\sim 10^{-17}$}\coordHE{}. With \myHighlight{$k\sim 10^{19}$}\coordHE{} GeV we thus see that with \myHighlight{$1/\beta \sim 10$}\coordHE{} GeV we get \myHighlight{$\beta k \sim 10^{18}$}\coordHE{}, thus \myHighlight{$a_m \sim 10^{-17}$}\coordHE{} according to Eq.~(17). The weak scale, \myHighlight{$a_m k \sim 10^2$}\coordHE{} GeV, can in this way be generated.

As a numerical example, assume that \myHighlight{$\alpha=2$}\coordHE{}. If \myHighlight{$\nu=5/2$}\coordHE{} or \myHighlight{$\mu=\nu+2=9/2$}\coordHE{} we get \myHighlight{$x_1=3.6328$}\coordHE{}, \myHighlight{$B_3^f/B_2^f=11.9408$}\coordHE{}, and the minimum occurs at \myHighlight{$a_m=5.15 \times 10^{-17}$}\coordHE{}. 

In conclusion, a bulk quantum fermion may generate a thermal (flat) 5D AdS brane-world with the necessary hierarchy scale. This example shows that quantum bulk effects in a brane-world \myHighlight{$AdS_5$}\coordHE{} at nonzero temperature may not only stabilize the brane-world but also provide the dynamical mechanism for the resolution of the hierarchy problem, with no fine-tuning.

Our example involving a single fermion is somewhat unsatisfactory because the minimum at \myHighlight{$a \neq 0$}\coordHE{} is only local. A somewhat more extended discussion, involving both bosons and fermions, is given in Ref.~\cite{brevik01}.

We mention that a very recent approach to stabilize the radius of the brane-world \myHighlight{$AdS_5$}\coordHE{} is the paper of Flachi {\it et al.} \cite{flachi01}. 

Finally, as a general remark, we mention that the RS scenario is of interest also for quantum gravity (for a general treatise on QG, see Ref.~\cite{buchbinder92}).


\newpage

\begin{thebibliography}{99}

\bibitem{randall99}
L. Randall and R. Sundrum, {\it Phys. Rev. Lett.} {\bf 83}, 3370 (1999); {\it ibid.} {\bf 83}, 4690 (1999).

\bibitem{goldberger00}
W. D. Goldberger and M. B. Wise, {\it Phys. Lett.} {\bf B 475}, 275 (2000).

\bibitem{gherghetta00}
T. Gherghetta and A. Pomarol, {\it Nucl. Phys.} {\bf B 586}, 141 (2000).

\bibitem{brevik01}
I. Brevik, K. A. Milton, S. Nojiri, and S. D. Odintsov, {\it Nucl. Phys.} {\bf B 599}, 305 (2001).

\bibitem{goldberger00a}
W. D. Goldberger and I. Z. Rothstein, {\it Phys. Lett.} {\bf B 491}, 339 (2000).

\bibitem{flachi01}
A. Flachi, I. G. Moss, and D. J. Toms, "Quantized Bulk Fermions in the Randall-Sundrum Brane Model". hep-th/0106076.

\bibitem{buchbinder92}
I. L. Buchbinder, S. D. Odintsov, and I. L. Shapiro, "Effective Action in Quantum Gravity", IOP Publishing, Bristol, 1992. 




\end{thebibliography}



\end{document}

\bye
