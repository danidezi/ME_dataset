
\documentclass[a4paper,12pt,titlepage]{article}
\begin{document}

\title{BOUND STATES BY A PSEUDOSCALAR COULOMB POTENTIAL IN ONE-PLUS-ONE DIMENSIONS}
\date{}
\author{Antonio S. de Castro \\
\\
UNESP - Campus de Guaratinguet\'{a}\\
Departamento de F\'{\i}sica e Qu\'{\i}mica\\
Caixa Postal 205\\
12516-410 Guaratinguet\'{a} SP - Brasil\\
\\
Electronic mail: castro@feg.unesp.br}
\maketitle

\begin{abstract}
The Dirac equation is solved for a pseudoscalar Coulomb potential in a
two-dimensional world. An infinite sequence of bounded solutions are
obtained. These results are in sharp contrast with those ones obtained in
3+1 dimensions where no bound-state solutions are found.
\end{abstract}

The Coulomb potential of a point electric charge in a 1+1 dimension,
considered as the time component of a Lorentz vector, is linear and so it
provides a constant electric field always pointing to, or from, the point
charge. This problem is related to the confinement of fermions in the
Schwinger and in the massive Schwinger models \cite{col1}-\cite{col2} and in
the Thirring-Schwinger model \cite{fro}. It is frustrating that, due to the
tunneling effect (Klein\'{}s paradox), there are no bound states for this
kind of potential regardless of the strength of the potential \cite{cap}-%
\cite{gal}. The linear potential, considered as a Lorentz scalar, is also
related to the quarkonium model in one-plus-one dimensions \cite{hoo}-\cite
{kog}. Recently it was incorrectly concluded that even in this case there is
solely one bound state \cite{bha}. Later, the proper solutions for this last
problem were found \cite{cas}-\cite{hil}. However, it is well known from the
quarkonium phenomenology in the real 3+1 dimensional world that the best fit
for meson spectroscopy is found for a convenient mixture of vector and
scalar potentials put by hand in the equations (see, \textit{e.g.}, \cite
{luc}). The mixed vector-scalar linear potential in 1+1 dimensions was
recently considered \cite{asc1}. There it was found that there are
analytical bound-state solutions on condition that the scalar component of
the potential is of sufficient strength compared to the vector component ($%
|V_{s}|\geq |V_{t}|$). As a by-product, that approach also showed that there
exist relativistic confining potentials providing no bound-state solutions
in the nonrelativistic limit. Although the discussion was confined to the
vector-scalar mixing, the inclusion of a pseudoscalar potential could also
be allowed.

In a recent paper, McKeon and Van Leeuwen \cite{mck} considered a
non\-con\-ser\-ving-parity pseu\-do\-sca\-lar Coulomb (NCPPC) potential ($%
V=\lambda /r$) in 3+1 dimensions and concluded that there are no bounded
solutions for the reason that the different parity eigenstates mix.
Furthermore, they asserted that \textit{the absence of bound states in this
system confuses the role of the }$\pi $\textit{-meson in the binding of
nucleons}. Such an intriguing conclusion sets the stage for the analyses of
the confinement of fermions by other sorts of pseudoscalar potentials. A
natural question to ask is if the absence of bounded solutions by a NCPPC
potential is a characteristic feature of the four-dimensional world. With
this in mind, we approach in the present paper the less realistic Dirac
equation in one-plus-one dimensions with a NCPPC potential ($V=\lambda |x|$%
). The confinement of fermions by a pure \-con\-ser\-ving-parity
pseudoscalar double-step potential \cite{asc2} and their scattering by a
pure non\-con\-ser\-ving-parity pseudoscalar step potential \cite{asc3} have
already been analyzed in the literature providing the opportunity to find
some quite interesting results.

The two-dimensional Dirac equation can be obtained from the
four-dimen\-sional one with the mixture of spherically symmetric scalar,
vector and anomalous magnetic interactions. If we limit the fermion to move
in the $x$-direction ($p_{y}=p_{z}=0$) the four-dimensional Dirac equation
decomposes into two equivalent two-dimensional equations with 2-component
spinors and 2$\times $2 matrices \cite{str}. Then, there results that the
scalar and vector interactions preserve their Lorentz structures whereas the
anomalous magnetic interaction turns out to be a pseudoscalar interaction.
Furthermore, in the 1+1 world there is no angular momentum so that the spin
is absent. Therefore, the 1+1 dimensional Dirac equation allow us to explore
the physical consequences of the negative-energy states in a mathematically
simpler and more physically transparent way.

Let us begin by presenting the Dirac equation in 1+1 dimensions. In the
presence of a time-independent potential the 1+1 dimensional
time-independent Dirac equation for a fermion of rest mass $m$ reads 
\begin{equation}
\mathcal{H}\Psi =E\Psi  \label{eq1}
\end{equation}

\begin{equation}
\mathcal{H}=c\alpha p+\beta mc^{2}+\mathcal{V}  \label{eq1a}
\end{equation}

\noindent where $E$ is the energy of the fermion, $c$ is the velocity of
light and $p$ is the momentum operator. $\alpha $ and $\beta $ are Hermitian
square matrices satisfying the relations $\alpha ^{2}=\beta ^{2}=1$, $%
\left\{ \alpha ,\beta \right\} =0$. From the last two relations it steams
that both $\alpha $ and $\beta $ are traceless and have eigenvalues equal to 
$-$1, so that one can conclude that $\alpha $ and $\beta $ are
even-dimensional matrices. One can choose the 2$\times $2 Pauli matrices
satisfying the same algebra as $\alpha $ and $\beta $, resulting in a
2-component spinor $\Psi $. The positive definite function $|\Psi |^{2}=\Psi
^{\dagger }\Psi $, satisfying a continuity equation, is interpreted as a
probability position density and its norm is a constant of motion. This
interpretation is completely satisfactory for single-particle states \cite
{tha}. We use $\alpha =\sigma _{1}$ and $\beta =\sigma _{3}$. For the
potential matrix we consider 
\begin{equation}
\mathcal{V}=1V_{t}+\beta V_{s}+\alpha V_{e}+\beta \gamma ^{5}V_{p}
\label{eq2}
\end{equation}

\noindent where $1$ stands for the 2$\times $2 identity matrix and $\beta
\gamma ^{5}=\sigma _{2}$. This is the most general combination of Lorentz
structures for the potential matrix because there are only four linearly
independent 2$\times $2 matrices. The subscripts for the terms of potential
denote their properties under a Lorentz transformation: $t$ and $e$ for the
time and space components of the 2-vector potential, $s$ and $p$ for the
scalar and pseudoscalar terms, respectively. It is worth to note that the
Dirac equation is covariant under $x\rightarrow -x$ if $V_{e}(x)$ and $%
V_{p}(x)$ change sign whereas $V_{t}(x)$ and $V_{s}(x)$ remain the same.
This is because the parity operator $P=\exp (i\theta )P_{0}\sigma _{3}$,
where $\theta $ is a constant phase and $P_{0}$ changes $x$ into $-x$,
changes sign of $\alpha $ and $\beta \gamma ^{5}$ but not of $1$ and $\beta $%
.

Defining the spinor $\psi $ as 
\begin{equation}
\psi =\exp \left( \frac{i}{\hbar }\Lambda \right) \Psi  \label{eq5}
\end{equation}

\noindent where 
\begin{equation}
\Lambda (x)=\int^{x}dx^{\prime }\frac{V_{e}(x^{\prime })}{c}  \label{eq6}
\end{equation}

\noindent the space component of the vector potential is gauged away

\begin{equation}
\left( p+\frac{V_{e}}{c}\right) \Psi =\exp \left( \frac{i}{\hbar }\Lambda
\right) p\psi  \label{eq7}
\end{equation}

\noindent so that the time-independent Dirac equation can be rewritten as
follows:

\begin{equation}
H\psi =E\psi  \label{eq7a}
\end{equation}

\begin{equation}
H=\sigma _{1}cp+\sigma _{2}V_{p}+\sigma _{3}\left( mc^{2}+V_{s}\right)
+1V_{t}  \label{eq8}
\end{equation}

\noindent showing that the space component of a vector potential only
contributes to change the spinors by a local phase factor.

Provided that the spinor is written in terms of the upper and the lower
components 
\begin{equation}
\psi =\left( 
\begin{array}{c}
\phi \\ 
\chi
\end{array}
\right)  \label{eq8a}
\end{equation}

\noindent the Dirac equation decomposes into :

\begin{eqnarray}
\left( V_{t}-E+V_{s}+mc^{2}\right) \phi (x) &=&i\hbar c\chi ^{\prime
}(x)+iV_{p}\chi (x)  \nonumber \\
&&  \label{eq8b} \\
\left( V_{t}-E-V_{s}-mc^{2}\right) \chi (x) &=&i\hbar c\phi ^{\prime
}(x)-iV_{p}\phi (x)  \nonumber
\end{eqnarray}

\noindent where the prime denotes differentiation with respect to $x$. In
terms of $\phi $ and $\chi $ the spinor is normalized as $\int_{-\infty
}^{+\infty }dx\left( |\phi |^{2}+|\chi |^{2}\right) =1$, so that $\phi $ and 
$\chi $ are square integrable functions. It is clear from the pair of
coupled first-order differential equations (\ref{eq8b}) that both $\phi (x)$
and $\chi (x)$ must be discontinuous wherever the potential undergoes an
infinite jump and have opposite parities if the Dirac equation is covariant
under $x\rightarrow -x$. In the nonrelativistic approximation (potential
energies small compared to the rest mass) Eq. (\ref{eq8b}) loses all the
matrix structure and becomes

\begin{equation}
\chi =\frac{p}{2mc}\phi  \label{eq8c}
\end{equation}

\begin{equation}
\left( -\frac{\hbar ^{2}}{2m}\frac{d^{2}}{dx^{2}}+V_{t}+V_{s}\right) \phi
=\left( E-mc^{2}\right) \phi  \label{eq8d}
\end{equation}

\noindent Eq. (\ref{eq8c}) shows that $\chi $ if of order $v/c<<1$ relative
to $\phi $ and Eq. (\ref{eq8d}) shows that $\phi $ obeys the Schr\"{o}dinger
equation without any contribution from the pseudoscalar potential.

Now, let us choose $V_{t}=V_{s}=0$ and the intrinsically relativistic NCPPC
potential $V_{p}=\lambda |x|$, with $\lambda >0$. Defining

\begin{eqnarray}
\eta &=&\sqrt{\frac{2\lambda }{\hbar c}}\,x  \label{eq8f1} \\
&&  \nonumber \\
\nu &=&-1+\frac{E^{2}-m^{2}c^{4}}{2\hbar c\lambda }  \label{eq8f}
\end{eqnarray}

\smallskip \noindent \noindent the Dirac equation (\ref{eq8b}) turns into
the Schr\"{o}dinger-like differential equations

\begin{eqnarray}
-\frac{d^{2}\phi }{d\eta ^{2}}+\frac{\eta ^{2}}{4}\phi &=&\left\{ 
\begin{array}{l}
\left( \nu +1/2\right) \phi ,\qquad x>0 \\ 
\\ 
\left( \nu +3/2\right) \phi ,\qquad x<0
\end{array}
\right.  \nonumber \\
&&  \label{8f} \\
-\frac{d^{2}\chi }{d\eta ^{2}}+\frac{\eta ^{2}}{4}\chi &=&\left\{ 
\begin{array}{l}
\left( \nu +3/2\right) \chi ,\qquad x>0 \\ 
\\ 
\left( \nu +1/2\right) \chi ,\qquad x<0
\end{array}
\right.  \nonumber
\end{eqnarray}
\noindent The second-order differential equations (\ref{8f}) have the form 
\begin{equation}
y^{\prime \prime }(z)-\left( \frac{z^{2}}{4}+a\right) y(z)=0,  \label{eq9}
\end{equation}
whose solution is a parabolic cylinder function \cite{abr}. The solutions $%
D_{-a-1/2}(z)$ and $D_{-a-1/2}(-z)$ are linearly independent unless $%
n=-a-1/2 $ is a nonnegative integer. In that special circumstance $D_{n}(z)$
has the peculiar property that $D_{n}(-z)=(-1)^{n}D_{n}(z)$, and it is
proportional to $\exp \left( -z^{2}/4\right) H_{n}(z/\sqrt{2})$, where $%
H_{n}(z)$ is a Hermite polynomial. The solutions of (\ref{8f}) do not
exhibit this parity property so that we should not expect nonnegative
integer values for $\nu $. The physically acceptable solutions for bound
states must vanish in the asymptotic region $|\eta |\rightarrow \infty $ and
are expressed as 
\begin{eqnarray}
\phi &=&\left\{ 
\begin{array}{c}
C^{\left( +\right) }D_{\nu }(\eta ),\qquad x>0 \\ 
\\ 
C^{\left( -\right) }D_{\nu +1}(\eta ),\qquad x<0
\end{array}
\right.  \nonumber \\
&&  \label{eq10} \\
\eta &=&\left\{ 
\begin{array}{c}
D^{\left( +\right) }D_{\nu +1}(\eta ),\qquad x>0 \\ 
\\ 
D^{\left( -\right) }D_{\nu }(\eta ),\qquad x<0
\end{array}
\right.  \nonumber
\end{eqnarray}

\noindent where $C^{\left( \pm \right) }$ and $D^{\left( \pm \right) }$ are
normalization constants. Substituting the solutions (\ref{eq10}) into the
Dirac equation (\ref{eq8b}) and making use of the recurrence formulas

\begin{eqnarray}
\frac{d}{dz}D_{\nu }(z)-\frac{z}{2}D_{\nu }(z)+D_{\nu +1}(z) &=&0  \nonumber
\\
&&  \label{eq22} \\
\frac{d}{dz}D_{\nu }(z)+\frac{z}{2}D_{\nu }(z)-\nu D_{\nu -1}(z) &=&0 
\nonumber
\end{eqnarray}

\noindent one has as a result

\begin{eqnarray}
\left[ \frac{C^{\left( +\right) }}{D^{\left( +\right) }}\right] ^{2}
&=&-\left( \nu +1\right) \frac{E+mc^{2}}{E-mc^{2}}  \nonumber \\
&&  \label{22a} \\
\left[ \frac{C^{\left( -\right) }}{D^{\left( -\right) }}\right] ^{2}
&=&-\left( \frac{1}{\nu +1}\right) \frac{E+mc^{2}}{E-mc^{2}}  \nonumber
\end{eqnarray}

\noindent The continuity of the wavefunctions (\ref{eq10}) at $x=0$ furnishes

\begin{equation}
\frac{C^{\left( +\right) }}{D^{\left( +\right) }}=\frac{C^{\left( -\right) }%
}{D^{\left( -\right) }}\left[ \frac{D_{\nu +1}\left( 0\right) }{D_{\nu
}\left( 0\right) }\right] ^{2}  \label{22b}
\end{equation}

\noindent Together, (\ref{22a}) and (\ref{22b}) lead to the quantization
condition

\begin{equation}
D_{\nu +1}^{2}\left( 0\right) =\pm \left( \nu +1\right) D_{\nu }^{2}\left(
0\right)  \label{eq23}
\end{equation}

\noindent This last result combined with (\ref{eq8f}) \noindent shows that
the use of the minus sign demands that $-\left( 1+m^{2}c^{4}/2\hbar c\lambda
\right) <\nu <-1$. If, on the other side, one uses the plus sign then $%
-1<\nu <+\infty $. Because the normalization of the spinor is not important
for the calculation of the spectrum, one can arbitrarily choose $D_{\nu
}\left( 0\right) =1$.

The numerical computation of (\ref{eq23}) is substantially simpler when $%
D_{\nu +1}$ is written in terms of the derivative of $D_{\nu }$:

\begin{equation}
\frac{d}{d\eta }D_{\nu }(\eta )\left| _{\eta =0}\right. =\pm \sqrt{\pm
\left( \nu +1\right) }D_{\nu }\left( 0\right)  \label{eq24}
\end{equation}

\noindent By solving the quantization condition (\ref{eq24}) for $\nu $ one
should impose that the wavefunctions (\ref{eq10}) vanish for $|\eta
|\rightarrow \infty $, as pointed above. By using a fourth-fifth order
Runge-Kutta method \cite{fm} no solutions are found for $\nu <-1$. On the
other side, for $\nu >-1$ an infinite sequence of allowed values of $\nu $
are found corresponding to each sign in front of the square root symbol in (%
\ref{eq24}). The ten lowest states are given in Table 1. By inspection of
Table 1 one sees that for $\nu \rightarrow \infty $ their values come
equally spaced ($\Delta \nu =2$) whatever the sign of $\nu $ is.

The energy levels are obtained by inserting those allowed values of $\nu $
in (\ref{eq8f}):

\begin{equation}
E=\pm \sqrt{m^{2}c^{4}+2\hbar c\lambda \left( \nu +1\right) }  \label{eq25}
\end{equation}

\noindent The spectrum has a dependence on $\nu $ in such a way that $\Delta
E=0$ as $\nu $ tends to $\pm \infty $. One should realize that the energy
levels are symmetrical about $E=0$. It means that the potential couples to
the positive-energy component of the spinor in the same way it couples to
the negative-energy component. In other words, this sort of potential
couples to the mass of the fermion instead of its charge so that there is no
atmosphere for the production of particle-antiparticle pairs. No matter the
intensity of the coupling parameter ($\lambda $), the positive- and the
negative-energy solutions never meet. There is always an energy gap greater
or equal to $2mc^{2}$, thus there is no room for transitions from positive-
to negative-energy solutions. This all means that Klein\'{}s paradox does
not come to the scenario.

It is worthwhile to note that the Dirac equation with a nonvector potential,
or a vector potential contaminated with some scalar or pseudoscalar
coupling, is not invariant under $V\rightarrow V+const.$, this is so because
only the vector potential couples to the positive-energies in the same way
it couples to the negative-ones, whereas nonvector contaminants couple to
the mass of the fermion. Therefore, if there is any nonvector coupling the
absolute values of the energy will have physical significance and the
freedom to choose a zero-energy will be lost.

As stated in the third paragraph of this work, the anomalous magnetic
interaction in the four-dimensional world turns into a pseudoscalar
interaction in the two-dimensional world. The anomalous magnetic interaction
has the form $-i\mu \beta \vec{\alpha}.\vec{\nabla}\phi (r)$, where $\mu $
is the anomalous magnetic moment in units of the Bohr magneton and $\phi $
is the electric potential, \textit{i.e.}, the time component of a vector
potential \cite{tha}. In one-plus-one dimensions the anomalous magnetic
interaction turns into $\sigma _{2}\mu \phi ^{\prime }$, then one might
suppose that the pseudoscalar Coulomb potential is due to an electric
potential proportional to $sign(x)x^{2}/2$, where $sign(x)$ stands for the
sign function. The oddness of the electric potential (under $x\rightarrow -x$%
) does not present problem to the confinement of a fermion because its
effective mass, an $x$-dependent mass which always increases as the fermion
goes away from the origin, depends on $\left( \phi ^{\prime \,}\right) ^{2}$%
, an result independent of the sign of $sign(x)$ \cite{asc2}-\cite{asc3}.

For short, in addition to their intrinsic importance, the above conclusions
renders a contrast to the result found in \cite{mck}. There are bound-state
solutions for fermions interacting by a pseudoscalar Coulomb potential in
1+1 dimensions notwithstanding the spinor is not an eigenfunction of the
parity operator. Therefore, the quadri-dimensional version of this problem
requires clarification for the absence of bounded solutions. One might
ponder that the underlying reason is the way the spinors are affected by the
behavior of the potentials at the origin as well as at infinity because this
is the radical difference between the potentials in those two dissimilar
worlds.

A word should be said about the role of the $\pi $-meson field. The
Lagrangian density describing the pion-nucleon interaction $\mathcal{L}%
=-i\lambda \overline{\psi }\gamma _{5}\psi \pi $ is parity-invariant because
the $\pi $-meson is a pseudoscalar field,\textit{\ i.e.}, $\pi (\vec{r}%
,t)\rightarrow -\pi (-\vec{r},t)$ under the parity transformation.
Nonetheless, the interaction matrix term present in the Dirac equation as
written by McKeon and Van Leeuwen \cite{mck}, $i\beta \gamma _{5}V(r)$,
supposed to be due to the $\pi $-meson field is not parity-invariant due to
the reason that the potential function, $V(r)$, is parity-invariant but the
potential matrix is not. Moreover, that matrix potential does not commute
with the total angular momentum. These arguments expose the inadequacy of
the interaction term in the Dirac equation as proposed in \cite{mck}.
Therefore, any conundrum about the role of the $\pi $-meson should be
consistently presented  by taking into account a quintessential
parity-invariant and spherically symmetric potential. Furthermore, in order
to correspond to the physical reality one should be aware that the massive $%
\pi $-meson field gives rise to a Yukawa potential instead of a Coulomb
potential.

\bigskip

\noindent \textbf{Acknowledgments}

This work was supported in part through funds provided by CNPq and FAPESP.

\smallskip 
\begin{table}[tbp]
\caption{The lowest solutions of Eq. (22) for $\nu >-1$ (plus sign inside
the radical).}
\label{t1}
\begin{center}
\begin{tabular}{|c|c|}
\hline\hline
$\nu$ \rm{for negative square root} & $\nu$ \rm {for positive square
root} \\ \hline
-0.654531 & 0.548571 \\ 
1.468582 & 2.522304 \\ 
3.482395 & 4.514353 \\ 
5.487785 & 6.510727 \\ 
7.490650 & 8.508354 \\ 
9.492433 & 10.506935 \\ 
11.493638 & 12.505887 \\ 
13.494521 & 14.505158 \\ 
15.495179 & 16.504544 \\ 
17.495703 & 18.504078 \\ \hline\hline
\end{tabular}
\end{center}
\end{table}

\begin{thebibliography}{99}
\bibitem{col1}  S. Coleman, R. Jackiw and L. Susskind, Ann. Phys. (N.Y.) 93
(1975) 267.

\bibitem{col2}  S. Coleman, Ann. Phys. (N.Y.) 101 (1976) 239.

\bibitem{fro}  J. Fr\"{o}hlich and E. Seiler, Helv. Phys. Acta 49 (1976) 889.

\bibitem{cap}  A. Z. Capri and R. Ferrari, Can. J. Phys. 63 (1985) 1029.

\bibitem{gal}  H. Gali\'{c}, Am. J. Phys. 56 (1988) 312.

\bibitem{hoo}  G.\'{}t Hooft, Nucl. Phys. B\textbf{\ }75 (1974) 461.

\bibitem{kog}  J. Kogut and L. Susskind, Phys. Rev. D 9 (1974) 3501.

\bibitem{bha}  R. S. Bhalerao and B. Ram, Am. J. Phys. 69 (2001) 817.

\bibitem{cas}  A. S. de Castro, Am. J. Phys. 70 (2002) 450.

\bibitem{cav}  R. M. Cavalcanti, Am. J. Phys. 70 (2002) 451.

\bibitem{hil}  J. R. Hiller, Am. J. Phys. 70 (2002) 522.

\bibitem{luc}  W. Lucha, F. F. Sch\"{o}berl and D. Gromes, Phys. Rep. 200
(1991) 127 and references therein.

\bibitem{asc1}  A.S. de Castro, Phys. Lett. A 305 (2002) 100.

\bibitem{asc2}  A.S. de Castro and W.G. Pereira, Phys. Lett. A 308 (2003)
131.

\bibitem{asc3}  A.S. de Castro, Phys. Lett. A, to be published.

\bibitem{mck}  D.G.C. McKeon and G. Van Leeuwen, Mod. Phys. Lett. A 17
(2002) 1961.

\bibitem{str}  P. Strange, Relativistic Quantum Mechanics, Cambridge
University Press, Cambridge, 1998.

\bibitem{tha}  B. Thaller, The Dirac Equation (Springer-Verlag, Berlin,
1992).

\bibitem{abr}  M. Abramowitz and I. A. Stegun, Handbook of Mathematical
Functions (Dover, Toronto, 1965).

\bibitem{fm}  G. E. Forsythe, M. A. Malcolm and C. B. Moler, Computer
Methods for Mathematical Computation (Prentice-Hall, New Jersey, 1977).
\end{thebibliography}

\end{document}
