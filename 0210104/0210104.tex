\documentclass[a4paper,12pt] {article}
\usepackage{psfrag}
\usepackage{graphicx}
\usepackage{latexsym,amsfonts}
\usepackage{amssymb}
\pagestyle{plain}
\textwidth=16truecm
\textheight=23.6truecm
\topmargin-1.5cm
\hoffset-1.2cm
\baselineskip=24pt

%
\begin{document}
\setcounter{page}{1}


\title{Bosonic vacuum wave functions from the BCS--type wave function
of the ground state of the massless Thirring model}

\author{M. Faber\thanks{E--mail: faber@kph.tuwien.ac.at, Tel.:
+43--1--58801--14261, Fax: +43--1--5864203} ~~and~
A. N. Ivanov\thanks{E--mail: ivanov@kph.tuwien.ac.at, Tel.:
+43--1--58801--14261, Fax: +43--1--5864203}~\thanks{Permanent Address:
State Polytechnic University, Department of Nuclear Physics, 195251
St. Petersburg, Russian Federation}}

\date{\today}

\maketitle

\vspace{-0.5in}
\begin{center}
{\it Atominstitut der \"Osterreichischen Universit\"aten,
Arbeitsbereich Kernphysik und Nukleare Astrophysik, Technische
Universit\"at Wien, \\ Wiedner Hauptstr. 8-10, A-1040 Wien,
\"Osterreich }
\end{center}

\begin{center}
\begin{abstract}
A BCS--type wave function describes the ground state of the massless
Thirring model in the chirally broken phase. The massless Thirring
model with fermion fields quantized in the chirally broken phase
bosonizes to the quantum field theory of the free massless
(pseudo)scalar field (Eur. Phys. J. C {\bf 20}, 723 (2001)). The wave
functions of the ground state of the free massless (pseudo)scalar
field are obtained from the BCS wave function by averaging over
quantum fluctuations of the Thirring fermion fields. We show that we
obtain wave functions, orthogonal, normalized and non--invariant under
shifts of the massless (pseudo)scalar field. This testifies the
spontaneous breaking of the field--shift symmetry in the quantum field
theory of a free massless (pseudo)scalar field.
\end{abstract}
\end{center}

\newpage

\hspace{0.2in} A recent analysis of the massless Thirring model [1] has
shown that the wave function $|\Omega\rangle$ of the ground state of
the massless Thirring model is of BCS form
%
\begin{eqnarray}\label{label1}
|\Omega\rangle = \prod_{\textstyle k^1}[u_{\textstyle k^1} +
 v_{\textstyle k^1}\, a^{\dagger}(k^1)b^{\dagger}(-k^1)]|0\rangle,
\end{eqnarray}
%
where $|0\rangle$ is the perturbative, chiral symmetric vacuum,
$a^{\dagger}(k^1)$ and $b^{\dagger}(-k^1)$ are creation operators of
fermions and anti--fermions with momentum $k^1$ and $-k^1$,
respectively, $u_{\textstyle k^1}$ and $ v_{\textstyle k^1}$ are
coefficient functions  [1]
%
\begin{eqnarray}\label{label2}
u_{\textstyle k^1} = \sqrt{\frac{1}{2}\Bigg( 1 +
\frac{|k^1|}{\sqrt{(k^1)^2 + M^2}}\Bigg)}\quad,\quad v_{\textstyle
k^1} = \varepsilon(k^1)\,\sqrt{\frac{1}{2}\Bigg( 1 -
\frac{|k^1|}{\sqrt{(k^1)^2 + M^2}}\Bigg)},
\end{eqnarray}
%
where $\varepsilon(k^1)$ is the sign function.

The wave function (\ref{label1}) describes the fermions in a finite
volume $L$. According to Yoshida [2] in the limit $L\to \infty$ the
wave function can be transcribed into the form
%
\begin{eqnarray}\label{label3}
|\Omega\rangle = \exp\Big\{
 \int^{\infty}_{-\infty}dk^1\,\tilde{\Phi}(k^1)\,
 [a^{\dagger}(k^1)b^{\dagger}(-k^1) - b(-k^1)a(k^1)]\Big\}\,|0\rangle,
\end{eqnarray}
%
where the phase $\Phi(k^1)$ is defined by
%
\begin{eqnarray}\label{label4}
 \tilde{\Phi}(k^1) = \frac{1}{2}\,\arctan\Big(\frac{M}{k^1}\Big).
\end{eqnarray}
%
Under chiral rotations
%
\begin{eqnarray}\label{label5}
&&\psi(x) \to {\psi\,}'(x) = e^{\textstyle
i\gamma^5\alpha_{\rm A}}\psi(x),\nonumber\\
&&\bar{\psi}(x) \to
{\bar{\psi}\,}'(x) = \bar{\psi}(x)\,e^{\textstyle
i\gamma^5\alpha_{\rm A}}, 
\end{eqnarray}
%
where $\psi(x)$ and $\bar{\psi}(x)$ are operators of the massless
Thirring fermion fields, the wave function (\ref{label3}) transforms
as follows [1]
%
\begin{eqnarray}\label{label6}
\hspace{-0.5in}&&|\Omega\rangle \to |\Omega; \alpha_{\rm A}\rangle
=\nonumber\\ \hspace{-0.5in}&&=\exp\Big\{
\int^{\infty}_{-\infty}dk^1\tilde{\Phi}(k^1)[a^{\dagger}(k^1)
b^{\dagger}(-k^1)e^{\textstyle -2i\alpha_{\rm A}\varepsilon(k^1)} -
b(-k^1) a(k^1)e^{\textstyle +2i\alpha_{\rm A}\varepsilon(k^1)}]\Big\}
|0\rangle.
\end{eqnarray}
%
As has been shown in [1] the wave functions $|\Omega; \alpha_{\rm
A}\rangle$ and $|\Omega; \alpha\,'_{\rm A}\rangle$ (\ref{label5}) are
orthogonal for $\alpha\,'_{\rm A} \neq \alpha_{\rm A}\,({\rm
mod}\,2\pi)$, i.e. $\langle \alpha\,'_{\rm A}; \Omega|\Omega;
\alpha_{\rm A}\rangle = \delta_{\alpha\,'_{\rm A} \alpha_{\rm A}}$.


In the following we prove that the bosonized version of the wave
functions (\ref{label6}) are also orthogonal. We express
(\ref{label6}) in terms of the fermion field operators $\psi(x)$ and
$\bar{\psi}(x)$ using
%
\begin{eqnarray}\label{label7}
a(k^1) &=& \frac{1}{\sqrt{2\pi}}\int^{\infty}_{-\infty}dx^1\,
\frac{u^{\dagger}(k^1)}{\sqrt{2k^0}}\,\psi(0,x^1)\,e^{\textstyle
-ik^1x^1},\nonumber\\ b(-k^1)
&=&\frac{1}{\sqrt{2\pi}}\int^{\infty}_{-\infty}dy^1\,
\psi^{\dagger}(0,y^1)\,\frac{v(- k^1)}{\sqrt{2k^0}}\,e^{\textstyle
+ik^1y^1}.
\end{eqnarray}
%
The wave functions $u(k^1)$ and $v(k^1)$ are defined by [1]
%
\begin{eqnarray}\label{label8}
u(k^1) = \sqrt{2k^0}{\displaystyle \left(\begin{array}{c}\theta(+k^1)
\\ \theta(-k^1)
\end{array}\right)}\quad,\quad v(k^1) = \sqrt{2k^0}{\displaystyle \left(\begin{array}{c} + \theta(+k^1)
\\ - \theta(-k^1)
\end{array}\right)}.
\end{eqnarray}
%
They are normalized to $u^{\dagger}(k^1)u(k^1) =
v^{\dagger}(k^1)v(k^1) = 2k^0$ and $\theta(\pm k^1)$ are Heaviside
functions. Substituting (\ref{label7}) in (\ref{label6}) we get
%
\begin{eqnarray}\label{label9}
\hspace{-0.5in}|\Omega; \alpha_{\rm A}\rangle = \exp\Big\{
\int^{\infty}_{-\infty}dx^1\int^{\infty}_{-\infty}dy^1\,\Phi(x^1 -
y^1)\,\bar{\psi}(0,x^1)\gamma^5e^{\textstyle +2i\gamma^5\alpha_{\rm
A}}\psi(0,y^1)\Big\}|0\rangle,
\end{eqnarray}
%
where $\Phi(x^1 - y^1)$ is the original of $\tilde{\Phi}(k^1)$
%
\begin{eqnarray}\label{label10}
\hspace{-0.5in}\Phi(x^1 - y^1) &=& \int^{\infty}_{-\infty}
\frac{dk^1}{4\pi} \arctan\Big(\frac{M}{k^1}\Big)e^{\textstyle
+ik^1(x^1 - y^1)} = \nonumber\\
\hspace{-0.5in}&=& \int^{\infty}_{-\infty}
\frac{dk^1}{4\pi}\Big[\frac{\pi}{2} -
\arctan\Big(\frac{k^1}{M}\Big)\Big]\,e^{\textstyle +ik^1(x^1 - y^1)}=
\nonumber\\
\hspace{-0.5in}&=& \frac{\pi}{4}\,\delta(x^1 - y^1) + \int^{\infty}_0
\frac{dk^1}{2\pi i}\arctan\Big(\frac{k^1}{M}\Big)\,\sin(k^1(x^1 -
y^1)) =\nonumber\\
\hspace{-0.5in}&=&\frac{\pi}{4}\,\delta(x^1 - y^1) +
\frac{M}{x^1 - y^1}\int^{\infty}_0 \frac{dk^1}{2\pi
i}\,\frac{\cos(k^1(x^1 - y^1))}{(k^1)^2 + M^2}=\nonumber\\
\hspace{-0.5in}&=& \frac{\pi}{4}\,\delta(x^1 - y^1)+
\frac{1}{4i}\,\frac{\displaystyle e^{\textstyle -M|x^1 - y^1|}}{x^1 -
y^1} = \frac{1}{4i}\,\frac{\displaystyle e^{\textstyle -M|x^1 -
y^1|}}{x^1 - y^1 - i\,0}.
\end{eqnarray}
%
The function $\Phi(x^1 - y^1)$ has the property $\Phi^*(x^1 - y^1) =
\Phi(y^1 - x^1)$, which is important in order to provide a
normalization of the wave function (\ref{label9}), $\langle
\alpha_{\rm A}; \Omega |\Omega; \alpha_{\rm A}\rangle = 1$.

Integrating over $y^1$ [1] we obtain
%
\begin{eqnarray}\label{label11}
\hspace{-0.5in}|\Omega; \alpha_{\rm A}\rangle =
\exp\Big\{\frac{\pi}{2}
\int^{\infty}_{-\infty}dx^1\bar{\psi}(0,x^1)\gamma^5e^{\textstyle
+2i\gamma^5\alpha_{\rm A}}\psi(0,x^1)\Big\}|0\rangle,
\end{eqnarray}
%
According to the Nambu--Jona--Lasinio condition [3] the operators of
the massless fermion fields $\psi(0,x^1)$ at time zero should be equal
to the operators of the massive fermion fields $\Psi(0,x^1)$ with
dynamical mass $M$, i.e. $\psi(0,x^1) = \Psi(0,x^1)$.  This yields
%
\begin{eqnarray}\label{label12}
\hspace{-0.5in}|\{\Psi\};\Omega; \alpha_{\rm A}\rangle &=& \exp\Big\{
\frac{\pi}{2}\int^{\infty}_{-\infty}dx^1\bar{\Psi}(0,x^1)
\gamma^5e^{\textstyle +2i\gamma^5\alpha_{\rm A}}\Psi(0,x^1)\Big\}
|0\rangle =\nonumber\\ &=& \exp\{i\,F[\Psi, \bar{\Psi}; \alpha_{\rm
A}]\} |0\rangle,
\end{eqnarray}
%
where the operator $\exp\{i\,F[\Psi, \bar{\Psi}; \alpha_{\rm
A}]\}$ is defined by
%
\begin{eqnarray}\label{label13}
\exp\{i\,F[\Psi, \bar{\Psi}; \alpha_{\rm A}]\} = \exp\Big\{
\frac{\pi}{2}\int^{\infty}_{-\infty}dx^1\bar{\Psi}(0,x^1)
\gamma^5e^{\textstyle +2i\gamma^5\alpha_{\rm A}}\Psi(0,x^1)\Big\}.
\end{eqnarray}
%
For convenience we have replaced $|\Omega; \alpha_{\rm A}\rangle$ by
$|\{\Psi\};\Omega; \alpha_{\rm A}\rangle$ in order to underscore that
the wave function of the ground state of the massless Thirring model
in the chirally broken phase is a functional of the dynamical fermion
fields.

As has been shown in [1] the massless Thirring model bosonizes to the
quantum field theory of the free massless (pseudo)scalar field
$\vartheta(x)$. In order to find the wave function of the ground state
of the massless (pseudo)scalar field $\vartheta(x)$ we have to average
the operator (\ref{label13}) over the dynamical fermion degrees of
freedom. This can be carried out within the path--integral approach.
We would like to accentuate that the integration over fermion degrees
of freedom of the operator (\ref{label13}) should be understood as the
bosonization of the operator $\exp\{\,i\,F[\Psi,
\bar{\Psi};\alpha_{\rm A}]\}$ that means the replacement of fermion
degrees of freedom by boson ones, $\exp\{\,i\,F[\Psi,
\bar{\Psi};\alpha_{\rm A}]\} \to \exp\{\,i\,B[\vartheta; \alpha_{\rm
A}]\}$. The integration over fermion degrees of freedom does not touch
the wave function $|0\rangle$, which can be taken away in the
functional integral.

In the massless Thirring model the generating functional of Green
functions we define by [1]
%
\begin{eqnarray}\label{label14}
\hspace{-0.3in}&&Z_{\rm Th}[J,\bar{J}] =\int {\cal D}\vartheta\,Z_{\rm
Th}[\vartheta; J,\bar{J}] = \nonumber\\
\hspace{-0.3in}&&= \int {\cal D}\vartheta\,{\cal D}\Psi\,{\cal
D}\bar{\Psi}\,\exp\Big\{i\int d^2x\,[\bar{\Psi}(x)(i\hat{\partial} -
M\,e^{\textstyle +i\gamma^5\vartheta(x)})\Psi(x) + \bar{J}(x)\Psi(x) +
\bar{\Psi}(x)J(x)]\Big\},\nonumber\\
\hspace{-0.3in}&& 
\end{eqnarray}
%
where $J(x)$ and $\bar{J}(x)$ are external sources of dynamical
fermions.

The bosonized version of the operator $\exp\{\,i\,F[\Psi,
\bar{\Psi};\alpha_{\rm A}]\}$ is defined by
%
\begin{eqnarray}\label{label15}
\hspace{-0.3in}&&\exp\{\,i\,B[\vartheta; \alpha_{\rm A}]\} =
\frac{1}{Z_{\rm Th}[\vartheta; 0,0]}\int {\cal D}\Psi {\cal
D}\bar{\Psi}\,\exp\{\,i\,F[\Psi, \bar{\Psi};\alpha_{\rm
A}]\}\nonumber\\
\hspace{-0.3in}&&\times\,\exp\Big\{i\int
d^2x\,[\bar{\Psi}(x)(i\hat{\partial} - M\,e^{\textstyle
+i\gamma^5\vartheta(x)})\Psi(x)+ \bar{J}(x)\Psi(x) +
\bar{\Psi}(x)J(x)\Big]\Big\}\Big|_{J = \bar{J} = 0}=\nonumber\\
\hspace{-0.3in}&& = \frac{1}{Z_{\rm Th}[\vartheta; 0,0]}\int {\cal
D}\Psi{\cal D}\bar{\Psi}\,\exp\{i\int
d^2x\,[\bar{\Psi}(x)\Big(i\hat{\partial} - M\,e^{\textstyle
+i\gamma^5\vartheta(x)}\,\Big)\Psi(x)\nonumber\\
\hspace{-0.3in}&& +
\frac{\pi}{2}\,\delta(x^0)\,\bar{\Psi}(x)\gamma^5\,e^{\textstyle
+2i\gamma^5\alpha_{\rm A}}\Psi(x)]\Big\},
\end{eqnarray}
%
By a chiral rotation $\Psi(x) \to e^{\textstyle
-i\gamma^5\vartheta(x)/2}\Psi(x)$ we obtain
%
\begin{eqnarray}\label{label16}
\hspace{-0.3in}&&\exp\{\,i\,B[\vartheta; \alpha_{\rm A}]\} =
\nonumber\\
\hspace{-0.3in}&& = \frac{1}{Z_{\rm Th}[\vartheta; 0,0]}\int {\cal
D}\Psi{\cal D}\bar{\Psi}\,\exp\Big\{i\int d^2x\,[\bar{\Psi}(x)
\Big(i\hat{\partial} + \frac{1}{2}\,\gamma^{\mu} \varepsilon_{\mu\nu}
\partial^{\nu}\vartheta(x) - M\Big)\Psi(x)\nonumber\\
\hspace{-0.3in}&& +
\frac{\pi}{2}\,\delta(x^0)\bar{\Psi}(x)\gamma^5\,e^{\textstyle
+i\gamma^5(2\alpha_{\rm A} - \vartheta(x))}\,\Psi(x)]\Big\}.
\end{eqnarray}
%
A possible chiral Jacobian, induced by this chiral rotation [1,4], is
canceled by the contribution of $Z_{\rm Th}[\vartheta; 0,0]$ in the
denominator. 

Below we are going to show that integrating over fermion fields and
keeping leading terms in the large $M$ expansion [1] we get the
following expression for the operator $\exp\{\,i\,B[\vartheta;
\alpha_{\rm A}]\}$
%
\begin{eqnarray}\label{label17}
\exp\{\,i\,B[\vartheta; \alpha_{\rm A}]\} =
\exp\Big\{i\,\frac{\pi}{2}\,\frac{M}{g}
\int^{\infty}_{-\infty}dx^1\,\sin\Big(\beta\vartheta(0,x^1) -
2\alpha_{\rm A}\Big)\Big\},
\end{eqnarray}
%
where we have also used the gap--equation for the dynamical mass $M$
(see Eq.(1.14) of Ref.[1]). The coupling constant $\beta$ is related
to the coupling constant $g$ of the massless Thirring model [1,5]
%
\begin{eqnarray}\label{label18}
\frac{8\pi}{\beta^2} = 1 - e^{\textstyle - 2\pi/g}.
\end{eqnarray}
%
Expression (\ref{label17}) has been obtained as follows.  Integration
over the fermion fields gives the expression [1]
%
\begin{eqnarray}\label{label19}
\hspace{-0.3in}&& \exp\{\,i\,B[\vartheta; \alpha_{\rm A}]\} =
\exp\Big\{i\int
d^2x\sum^{\infty}_{n=1}\frac{(-1)^{n-1}}{n}\frac{1}{4\pi}\int
\prod^{n-1}_{\ell}\frac{d^2x_{\ell}d^2k_{\ell}}{(2\pi)^2}\,e^{\textstyle
-ik_{\ell}\cdot(x_{\ell} - x)} \nonumber\\
\hspace{-0.3in}&&\times\int\frac{d^2k}{\pi i}\,{\rm
tr}\Big\{\frac{1}{M - \hat{k}}\Phi(x_1)\frac{1}{M - \hat{k} -
\hat{k}_1}\Phi(x_2)\ldots\Phi(x_{n-1})\frac{1}{M - \hat{k} -
\hat{k}_1 - \ldots - \hat{k}_{n-1}}\Phi(x)\Big\},\nonumber\\
\hspace{-0.3in}&&
\end{eqnarray}
%
where we have denoted
%
\begin{eqnarray}\label{label20}
\Phi(x_k) = -\frac{\pi}{2}\,\delta(x^0_k)i\gamma^5\, e^{\textstyle
i\gamma^5(2\alpha_{\rm A} - \vartheta(x_k))}
\end{eqnarray}
%
with $x_0 = x$ and $k = 0,1,2,\ldots, n-1$. For the subsequent
calculation of the momentum and space--time integrals we suggest to
use some kind of Pauli--Villars regularization. We smear the
$\delta$--functions $\delta(x^0_k)$ with the scales $M_j$
%
\begin{eqnarray}\label{label21}
\Phi(x_k; M_j) = -\frac{\pi}{2}\,f(M_jx^0_k)i\gamma^5\, e^{\textstyle
i\gamma^5(2\alpha_{\rm A} - \vartheta(x_k))}.
\end{eqnarray}
%
We require that in the limit $M_j \to \infty$ the function
$f(M_jx^0_k)$ converges to the $\delta$--function
$\delta(x^0_k)$. Introducing then the coefficients $C_j$, which
satisfy the constraints
%
\begin{eqnarray}\label{label22}
\sum^{N}_{j = 1} C_j = 1,\quad \sum^{N}_{j = 1} C_j M^n_j = 0\,,\, n =
1, 2, \ldots\,,
\end{eqnarray}
%
we rewrite (\ref{label19}) as follows
%
\begin{eqnarray}\label{label23}
\hspace{-0.3in}&&\exp\{\,i\,B[\vartheta; \alpha_{\rm A}]\}=
\exp\Big\{i\sum^N_{j = 1}C_j\int
d^2x\sum^{\infty}_{n=1}\frac{(-1)^{n-1}}{n}\frac{1}{4\pi}\int
\prod^{n-1}_{\ell}\frac{d^2x_{\ell}d^2k_{\ell}}{(2\pi)^2}\,e^{\textstyle
-ik_{\ell}\cdot(x_{\ell} - x)} \nonumber\\
\hspace{-0.3in}&&\times\int\frac{d^2k}{\pi i}\,{\rm
tr}\Big\{\frac{1}{M - \hat{k}}\Phi(x_1; M_j)\frac{1}{M - \hat{k} -
\hat{k}_1}\Phi(x_2; M_j)\ldots\Phi(x_{n-1}; M_j) \nonumber\\
\hspace{-0.3in}&&\times\frac{1}{M - \hat{k} - \hat{k}_1 - \ldots -
\hat{k}_{n-1}}\Phi(x; M_j)\Big\}.
\end{eqnarray}
%
Taking, first, the large $M$ expansion for the calculation of momentum
integrals [1] and using the constraints (\ref{label22}) we get
(\ref{label17}).

Expression (\ref{label17}) can be obtained directly from
(\ref{label13}) by using our bosonization rules (see Eq.(3.24) of
Ref.[1])
%
\begin{eqnarray}\label{label24}
\bar{\Psi}(0,x^1)\gamma^5\Psi(0,x^1) &=&
+\,i\,\frac{M}{g}\,\sin(\beta\,\vartheta(0,x^1)),\nonumber\\
\bar{\Psi}(0,x^1)\Psi(0,x^1) &=&
-\,\frac{M}{g}\,\cos(\beta\,\vartheta(0,x^1)).
\end{eqnarray}
%
Substituting (\ref{label24}) in (\ref{label13}) we arrive at
(\ref{label17}).

Using the operator (\ref{label17}) we define the bosonic wave function
%
\begin{eqnarray}\label{label25}
|\{\vartheta\};\Omega; \alpha_{\rm A}\rangle &=&
\exp\{\,i\,B[\vartheta; \alpha_{\rm A}]\}|0\rangle =\nonumber\\ &=&
\exp\Big\{i\,\frac{\pi}{2}\,\frac{M}{g}
\int^{\infty}_{-\infty}dx^1\,\sin\Big(\beta\vartheta(0,x^1) -
2\alpha_{\rm A}\Big)\Big\}|0\rangle,
\end{eqnarray}
%
which is normalized to unity. Now we have to show that such wave
functions are orthogonal for $\alpha\,'_{\rm A} \neq \alpha_{\rm
A}\,({\rm mod}\,2\pi)$
%
\begin{eqnarray}\label{label26}
\hspace{-0.3in}&&\langle \alpha\,'_{\rm A}; \Omega;
\{\vartheta\}|\{\vartheta\};\Omega; \alpha_{\rm A}\rangle =\Big\langle
0\Big|\exp\Big\{+i\,\pi\,\frac{M}{g}\,\sin(\alpha\,'_{\rm A} -
\alpha_{\rm A}) \nonumber\\
\hspace{-0.3in}&&\times\int^{\infty}_{-\infty}\!\!
dx^1\cos(\beta\,\vartheta(0,x^1) - \alpha\,'_{\rm A} - \alpha_{\rm
A})\Big\}\Big|0\Big\rangle = \lim_{L\to
\infty}\exp\Big\{+i\,\pi\,\frac{LM}{g}\,\sin(\alpha\,'_{\rm A} -
\alpha_{\rm A})\Big\}\nonumber\\
\hspace{-0.3in}&&\times \Big\langle 0\Big|\exp\Big\{- 2\pi
i\,\frac{M}{g}\,\sin(\alpha\,'_{\rm A} - \alpha_{\rm A})
\int^{\infty}_{-\infty}\!\!dx^1\sin^2\Big(\frac{\beta}{2}\,
\vartheta(0,x^1) - \frac{\alpha\,'_{\rm A} + \alpha_{\rm
A}}{2}\Big)\Big\}\Big|0\Big\rangle =\nonumber\\
\hspace{-0.3in}&&= \delta_{\alpha\,'_{\rm A}\alpha_{\rm A}}\Big\langle
0\Big|\exp\Big\{- 2\pi i\,\frac{M}{g}\,\sin(\alpha\,'_{\rm A} -
\alpha_{\rm A})
\int^{\infty}_{-\infty}\!\!dx^1\sin^2\Big(\frac{\beta}{2}\,
\vartheta(0,x^1) - \frac{\alpha\,'_{\rm A} + \alpha_{\rm
A}}{2}\Big)\Big\}\Big|0\Big\rangle=\nonumber\\
\hspace{-0.3in}&&= \delta_{\alpha\,'_{\rm A}\alpha_{\rm A}}.
\end{eqnarray}
%
As has been shown in [1] the massless Thirring model with fermion
fields quantized in the chirally broken phase bosonizes to the free
massless (pseudo)scalar field theory with the Lagrangian ${\cal L}(x)
= \frac{1}{2}\,\partial_{\mu}\vartheta(x)\partial^{\mu}\vartheta(x)$,
which is invariant under shifts of the field $\vartheta(x) \to
\vartheta\,'(x) = \vartheta(x) + \alpha$ with $\alpha \in
\mathbb{R}^{\,1}$ [6].

The wave function (\ref{label25}) describes the ground state of the
free massless (pseudo)scalar field $\vartheta(x)$. Since this wave
function is not invariant under the field--shifts, the symmetry is
spontaneously broken. The quantitative characteristic of the
spontaneously broken phase in the quantum field theory of the free
massless (pseudo)scalar field $\vartheta(x)$ is a non--vanishing
spontaneous magnetization ${\cal M} = 1$ [6]. This confirms fully the
existence of the chirally broken phase in the massless Thirring model
obtained in [1,4--6].


\begin{thebibliography}{9}
\bibitem{[1]} 
M. Faber and A. N. Ivanov, 
Eur. Phys. J. C {\bf 20}, 723
(2001), hep-th/0105057.
\bibitem{[2]}
K. Yoshida,
Nucl. Phys. B {\bf 187}, 103 (1981).
\bibitem{[3]}
Y. Nambu and G. Jona--Lasinio,
Phys. Rev. {\bf 122}, 345 (1961).
\bibitem{[4]}
M. Faber and A. N. Ivanov, hep--th/0112183.
\bibitem{[5]}
M. Faber and A. N. Ivanov, hep--th/0205249.
\bibitem{[6]} 
M. Faber and A. N. Ivanov, Eur. Phys. J. C {\bf 24}, 653 (2002), 
hep--th/0112184; hep--th/0204237; hep--th/0206244; hep--th/0206034.
\end{thebibliography}

\end{document}






