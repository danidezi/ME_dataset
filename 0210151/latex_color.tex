
\documentclass[a4paper,12pt]{article}
\usepackage[T1]{fontenc}
\usepackage[dvips]{graphicx}
\usepackage[tbtags]{amsmath}
\usepackage{amsopn}
\usepackage{amsfonts}
\title{Abelian duality in three dimensions\footnote{%
supported by KBN grant 5 P03B 072 21}}
\author{Bogus\l aw Broda\footnote{e-mail: bobroda@uni.lodz.pl}\, and    Grzegorz Duniec\footnote{e-mail: gduniec@merlin.fic.uni.lodz.pl}\\
Department of Theoretical Physics\\
                         University of \L \'od\'z\\
                           Pomorska 149/153\\
                          PL-90-236 \L \'od\'z\\
                                       Poland}
\usepackage{useful_macros}
\begin{document}
\baselineskip24pt
\maketitle

\begin{abstract}
Abelian duality on the closed three-dimensional Riemannian manifold
\myHighlight{${\cal M}^3$}\coordHE{} is discussed. Partition functions for the ordinary \myHighlight{$U(1)$}\coordHE{} gauge theory and circle-valued scalar field theory on \myHighlight{${\cal M}^3$}\coordHE{} are
explicitly calculated and compared.
\end{abstract}

\noindent
{\bf Keywords:} Abelian duality, three-dimensional Abelian gauge theory, circle-valued scalar free field theory, quantum field theory on Riemannian manifold

\noindent
{\bf PACS numbers:} 02.40.-k, 11.10.Kk, 11.15.-q, 11.90.+t

\section{Introduction}

At present, Abelian duality is a classical theme in (quantum)
field theory. In four-dimensional case, there are two well-known
papers on electric-magnetic (or \myHighlight{$S$}\coordHE{}-duality) in Abelian gauge
theory, viz.\ [\ref{B01},\ref{B02}]. In three dimensions, one could
cite the other two papers, [\ref{B03}] and [\ref{B04}], but
unfortunately, neither is fully comprehensive nor satisfactory.
The first ([\ref{B03}]) lacks explicit formulas for the partition
functions and, in a sense, is purely "classical", whereas the
second one ([\ref{B04}]) lacks sufficient generality and
topological aspects are there largely ignored. Moreover, the both
are seemingly contradictory: [\ref{B03}] explicitly shows duality
between Abelian gauge field and scalar field, whereas the
corresponding partition functions calculated in [\ref{B04}]
markedly differ.

The aim of this short work is to fill the gap and
clarify some points (the reader is advised to consult the above-mentioned papers for more
details which are not repeated here).
In particular, we will explicitly calculate the both partition functions and discuss their mutual duality.

\section{Partition function of the Abelian theory}

We consider the connected, orientable three-dimensional Riemannian (of
Euclidean signature) manifold \myHighlight{$\mathcal{M}^{3}$}\coordHE{} with torsionless
(co)homology throughout. The action of \myHighlight{$U(1)$}\coordHE{} gauge theory is
defined by
\begin{equation}\coord{}\boxEquation{\label{T1}
S[A]= \frac{1}{4 \pi e^{2}}\int_{\mathcal{M}^{3}}\; F_{A}\wedge \ast F_{A}
=\frac{1}{8\pi e^{2}} \int_{\mathcal{M}^3}d^{3}x \sqrt{g}\;
F_{ij}F^{ij},\;\;\;\;           i, j= 1, 2, 3,
}{S[A]= \frac{1}{4 \pi e^{2}}\int_{\mathcal{M}^{3}}\; F_{A}\wedge \ast F_{A}
=\frac{1}{8\pi e^{2}} \int_{\mathcal{M}^3}d^{3}x \sqrt{g}\;
F_{ij}F^{ij},\;\;\;\;           i, j= 1, 2, 3,
}{ecuacion}\coordE{}\end{equation}
where \myHighlight{$F_{ij}=\partial_{i}A_{j}-\partial_{j}A_{i} \; (F_{A}=dA)$}\coordHE{}, \myHighlight{$e$}\coordHE{} is the coupling constant \myHighlight{$(e>0)$}\coordHE{}, and \myHighlight{$g_{ij}$}\coordHE{} is the metric tensor.
A standard form of the partition function is
\begin{equation}\coord{}\boxEquation{\label{T2}
Z_{U(1)}=\int\mathcal{D}A\; e^{- S[A]},
}{Z_{U(1)}=\int\mathcal{D}A\; e^{- S[A]},
}{ecuacion}\coordE{}\end{equation}
where \myHighlight{$\mathcal{D}A$}\coordHE{} denotes a formal integration measure with respect to gauge non-equivalent field configurations, i.e.\ the
Faddeev--Popov procedure is applied.

The partition function consists of several factors:
(1) \myHighlight{$Z_{\rm det}$}\coordHE{}---product of determinants,
(2) \myHighlight{$Z_{\rm class}$}\coordHE{}---sum over the classical saddle points,
(3) \myHighlight{$Z_{\rm vol}$}\coordHE{}---the volume of the space of classical minima,
(4) \myHighlight{$Z_{0}$}\coordHE{}---the contribution from zero modes.

The form of \myHighlight{$Z_{\rm det}$}\coordHE{}, coming from Gaussian integration with respect to gauge fields and ghosts,
is rather standard, i.e.
\begin{equation}\coord{}\boxEquation{\label{T3}
Z_{\rm det}=\frac{\rm det^{'}\Delta_{0} }
{(\rm det^{'}\Delta_{1})^\frac{1}{2}},
}{Z_{\rm det}=\frac{\rm det^{'}\Delta_{0} }
{(\rm det^{'}\Delta_{1})^\frac{1}{2}},
}{ecuacion}\coordE{}\end{equation}
where the prime denotes removal of zero modes,
\myHighlight{$\Delta_{0}$}\coordHE{} is a Laplacian acting on zero-forms
(Faddeev--Popov ghosts), and \myHighlight{$\Delta_{1}$}\coordHE{} is a Laplacian acting on
one-forms (gauge fields). Obviously, \myHighlight{$Z_{\rm det}$}\coordHE{} is independent of the
coupling constant \myHighlight{$e^2$}\coordHE{} and it corresponds to "quantum fluctuations".

The classical part represents the sum over classical saddle
points,
\begin{equation}\coord{}\boxEquation{\label{T4}
Z_{\rm class}=\sum e^{-S[A_{\rm class}]},
}{Z_{\rm class}=\sum e^{-S[A_{\rm class}]},
}{ecuacion}\coordE{}\end{equation}
where \myHighlight{$A_{\rm class}$}\coordHE{} are local minima of the action (1)
corresponding to different line bundles. For non-trivial second
homology of \myHighlight{$\mathcal{M}^{3}$}\coordHE{} we have field configurations with
non-zero flux,
\begin{equation}\coord{}\boxEquation{\label{T5}
\int_{\Sigma_{I}}F = 2 \pi m^{I}, \;\;\; m^{I} \in \mathbb{Z},
}{\int_{\Sigma_{I}}F = 2 \pi m^{I}, \;\;\; m^{I} \in \mathbb{Z},
}{ecuacion}\coordE{}\end{equation}
with \myHighlight{$I= 1,...,b_{2}={\rm dim} H_{2}(\mathcal{M}^{3})$}\coordHE{}. Then the
solutions of the field equations can be represented by the sum
\begin{equation}\coord{}\boxEquation{\label{T6}
F=2 \pi \sum_{ I } m^{ I } \omega_{ I },
}{F=2 \pi \sum_{ I } m^{ I } \omega_{ I },
}{ecuacion}\coordE{}\end{equation}
where \myHighlight{$\omega_{I}$}\coordHE{} span an integral basis of harmonic two-forms
with normalization
\begin{equation}\coord{}\boxEquation{\label{T7}
\int_{\Sigma_{ I}} \omega_{J}=\delta_{J}^{I}.
}{\int_{\Sigma_{ I}} \omega_{J}=\delta_{J}^{I}.
}{ecuacion}\coordE{}\end{equation}
Inserting the expansion \eqref{T6} into the partition function \eqref{T4} we obtain
\begin{equation}\coord{}\boxEquation{\label{T8}
Z_{\rm class}=\sum_{m^{I}} e^{- S[m^{I}]},
}{Z_{\rm class}=\sum_{m^{I}} e^{- S[m^{I}]},
}{ecuacion}\coordE{}\end{equation}
where
\begin{equation}\coord{}\boxEquation{\label{T9}
S[m^{I}]=\frac{\pi}{e^{2}} \sum_{I,J} G_{IJ} m^{I} m^{J},
}{S[m^{I}]=\frac{\pi}{e^{2}} \sum_{I,J} G_{IJ} m^{I} m^{J},
}{ecuacion}\coordE{}\end{equation}
with
\begin{equation}\coord{}\boxEquation{\label{T10}
G_{IJ}=\frac{1}{2} \int_{\mathcal{M}^{3}} d^{3}x
\sqrt{g}\; \omega_{Iij}\omega_{J}^{ij}.
}{G_{IJ}=\frac{1}{2} \int_{\mathcal{M}^{3}} d^{3}x
\sqrt{g}\; \omega_{Iij}\omega_{J}^{ij}.
}{ecuacion}\coordE{}\end{equation}

The volume of the space of classical minima is a torus of dimension
\myHighlight{$b_{1}(\mathcal{M}^{3})$}\coordHE{} [\ref{B01}], that is,
in our case,
\begin{equation}\coord{}\boxEquation{\label{T11}
Z_{\rm vol}={\rm vol}(T^{b_{1}})=(2\pi)^{b_{1}}.
}{Z_{\rm vol}={\rm vol}(T^{b_{1}})=(2\pi)^{b_{1}}.
}{ecuacion}\coordE{}\end{equation}

Finally, the contribution from zero modes is
\begin{equation}\coord{}\boxEquation{
(2\pi e)^{b_{0}-b_{1}}=(2\pi e)^{1-b_{1}},
}{
(2\pi e)^{b_{0}-b_{1}}=(2\pi e)^{1-b_{1}},
}{ecuacion}\coordE{}\end{equation}
where \myHighlight{$b_{0}$}\coordHE{} corresponds to Faddeev--Popov ghosts, whereas \myHighlight{$b_{1}$}\coordHE{} is related to Abelian gauge fields.
Actually, we should divide this by \myHighlight{$2\pi$}\coordHE{} to obtain an agreement
with "a direct calculation" of the path integral (this is
explicitly demonstrated by considering a simple finite-dimensional
example in Appendix of [\ref{B05}]).
Thus,
\begin{equation}\coord{}\boxEquation{\label{T12}
Z_{0}=\frac{1}{2\pi}(2\pi e)^{1-b_{1}}.
}{Z_{0}=\frac{1}{2\pi}(2\pi e)^{1-b_{1}}.
}{ecuacion}\coordE{}\end{equation}

Collecting all the above terms, i.e. Eq.\eqref{T3}, Eq.\eqref{T8}, Eq.\eqref{T11} and Eq.\eqref{T12}, we obtain an
explicit form of the partition function for three-dimensional \myHighlight{$U(1)$}\coordHE{} gauge theory,
\begin{equation}\coord{}\boxEquation{\label{T13}
Z_{ U(1) }= e^{1 - b_1 } \frac{ \rm det^{'} \Delta_{0} }
{(\rm det^{'}\Delta_{1})^\frac{1}{2} } \sum_{m^ {I} } e^{-S[m^{ I }] },
}{Z_{ U(1) }= e^{1 - b_1 } \frac{ \rm det^{'} \Delta_{0} }
{(\rm det^{'}\Delta_{1})^\frac{1}{2} } \sum_{m^ {I} } e^{-S[m^{ I }] },
}{ecuacion}\coordE{}\end{equation}
with \myHighlight{$S[m^{I}]$}\coordHE{} defined by Eq.\eqref{T9}.

\section{Partition function of the scalar theory}

In this section we will calculate the partition function of the
three-dimensional (single-component) scalar field theory on
\myHighlight{$\mathcal{M}^{3}$}\coordHE{}. In principle, we have the two possibilities: we
could consider ordinary scalar field assuming values in
\myHighlight{$\mathbb{R}$}\coordHE{} or we could analyze circle-valued scalar field with values in \myHighlight{$S$}\coordHE{}.
A short reflection suggests the second possibility. First of all, for
ordinary scalar field we would not have a chance to reproduce the
reach topological sector of the form \eqref{T8}, \eqref{T9}. Besides, in [\ref{B03}]
Witten has shown duality between the Abelian theory and circle-valued
scalar theory. Anyway, real-valued theory can be considered as a
topologically trivial sector in circle-valued one.

The action of the scalar theory is of the form
\begin{equation}\coord{}\boxEquation{\label{T14}
S[\phi]= \frac{e^{2}}{4\pi}\int_{\mathcal{M}^{3}}\; E_{\phi}\wedge \ast E_{\phi}
=\frac{e^{2}}{4\pi} \int_{\mathcal{M}^{3}} d^{3}x \sqrt{g}\; E_{i} E^{i},
}{S[\phi]= \frac{e^{2}}{4\pi}\int_{\mathcal{M}^{3}}\; E_{\phi}\wedge \ast E_{\phi}
=\frac{e^{2}}{4\pi} \int_{\mathcal{M}^{3}} d^{3}x \sqrt{g}\; E_{i} E^{i},
}{ecuacion}\coordE{}\end{equation}
where \myHighlight{$E_{i}=\partial_{i} \phi \; (E_{\phi}=d \phi)$}\coordHE{}, with \myHighlight{$\phi \in [0,2\pi)$}\coordHE{}. The
rest of the notation is analogous to the one used in the previous
section. The partition function for the scalar theory,
\begin{equation}\coord{}\boxEquation{\label{T15}
Z_{\phi}=\int D\phi \; e^{-S[\phi]},
}{Z_{\phi}=\int D\phi \; e^{-S[\phi]},
}{ecuacion}\coordE{}\end{equation}
can be calculated according to the scheme proposed for \myHighlight{$Z_{U(1)}$}\coordHE{}.

The determinantal part consists of only one determinant coming from non-zero modes of scalar field,
\begin{equation}\coord{}\boxEquation{\label{T16}
Z_{\rm det}=(\rm det^{'}\Delta_{0})^{-\frac{1}{2}}.
}{Z_{\rm det}=(\rm det^{'}\Delta_{0})^{-\frac{1}{2}}.
}{ecuacion}\coordE{}\end{equation}

As far as the classical part is concerned we can rewrite Eq.\eqref{T4} for \myHighlight{$\phi$}\coordHE{} field,
\begin{equation}\coord{}\boxEquation{\label{T17}
Z_{\rm class}=\sum e^{-S[\phi_{\rm class}]},
}{Z_{\rm class}=\sum e^{-S[\phi_{\rm class}]},
}{ecuacion}\coordE{}\end{equation}
where, analogously, the sum is taken over the classical saddle points.
But this time, for non-trivial first homology of \myHighlight{$\mathcal{M}^{3}$}\coordHE{} we have field configurations
with non-zero circulation
\begin{equation}\coord{}\boxEquation{\label{T18}
\int_{C_{I}} E,
}{\int_{C_{I}} E,
}{ecuacion}\coordE{}\end{equation}
with \myHighlight{$I=1,...,b_{1}=$}\coordHE{}dim\myHighlight{$H_{1}(\mathcal{M}^{3})$}\coordHE{}. Since for
orientable three-dimensional manifold \myHighlight{$\mathcal{M}^{3}$}\coordHE{} we have
\myHighlight{$b_{1}=b_{2}$}\coordHE{}, we assume the same type of indices for \myHighlight{$\Sigma$}\coordHE{} and
\myHighlight{$C$}\coordHE{} (see Eq.\eqref{T5}). Then the solution of the field equations
can be expressed by the sum
\begin{equation}\coord{}\boxEquation{\label{T19}
E=2\pi\sum_{I} n^{I}\alpha_{I},
}{E=2\pi\sum_{I} n^{I}\alpha_{I},
}{ecuacion}\coordE{}\end{equation}
where \myHighlight{$\alpha_{I}$}\coordHE{}  span a basis of harmonic
one-forms. The issue of normalization of \myHighlight{$\alpha_{I}$}\coordHE{} is postponed to the next section.
Inserting the expansion \eqref{T19} into the partition function
\eqref{T17} we obtain
\begin{equation}\coord{}\boxEquation{\label{T21}
Z_{\rm class}=\sum_{n^{I}} e^{-\widetilde{S}[n^{ I }]},
}{Z_{\rm class}=\sum_{n^{I}} e^{-\widetilde{S}[n^{ I }]},
}{ecuacion}\coordE{}\end{equation}
where this time
\begin{equation}\coord{}\boxEquation{\label{T22}
\widetilde{S} [n^{I}] = \pi e^{2} \sum_{I,J} H_{IJ} n^{I}n^{J},
}{\widetilde{S} [n^{I}] = \pi e^{2} \sum_{I,J} H_{IJ} n^{I}n^{J},
}{ecuacion}\coordE{}\end{equation}
with
\begin{equation}\coord{}\boxEquation{\label{T23}
H_{IJ}=\int_{\mathcal{M}^{3}} d^{3}x
\sqrt{g} \; \alpha_{Ii}\alpha_{J}^{i},
}{H_{IJ}=\int_{\mathcal{M}^{3}} d^{3}x
\sqrt{g} \; \alpha_{Ii}\alpha_{J}^{i},
}{ecuacion}\coordE{}\end{equation}

The volume of the space of classical minima which is the target
space for this model (the circle \myHighlight{$S$}\coordHE{}) is
\begin{equation}\coord{}\boxEquation{\label{T24}
Z_{\rm vol}={\rm vol} (S) = 2\pi,
}{Z_{\rm vol}={\rm vol} (S) = 2\pi,
}{ecuacion}\coordE{}\end{equation}
and the zero-mode contribution (from a single zero mode of scalar field)
\begin{equation}\coord{}\boxEquation{\label{T25}
Z_{0}=\frac{e}{2\pi}.
}{Z_{0}=\frac{e}{2\pi}.
}{ecuacion}\coordE{}\end{equation}

The final shape of the partition function as the product of
Eq.\eqref{T16}, Eq.\eqref{T21}, Eq.\eqref{T24}, and Eq.\eqref{T25}
assumes the form
\begin{equation}\coord{}\boxEquation{\label{T26}
Z_{\phi}=e \; \frac{1}{ (\rm det^{'}\Delta_{0})^{ \frac{1}{2} }}
\sum_{n^{I} } e^{-\widetilde{S}[n^{I}]},
}{Z_{\phi}=e \; \frac{1}{ (\rm det^{'}\Delta_{0})^{ \frac{1}{2} }}
\sum_{n^{I} } e^{-\widetilde{S}[n^{I}]},
}{ecuacion}\coordE{}\end{equation}
with \myHighlight{$\widetilde{S}[n^{I}]$}\coordHE{} defined by Eq.\eqref{T22}.

\section{Duality and final discussion}

Since we hope for duality, at least in some limited sense, of \myHighlight{$U(1)$}\coordHE{} gauge
theory and circle-valued scalar field theory, we could expect
exact duality of classical parts of the partition functions,
Eq.\eqref{T8} and Eq.\eqref{T17}. The classical duality takes place if the metric
\myHighlight{$G_{IJ}$}\coordHE{} on the space of second cohomology and the metric \myHighlight{$H_{IJ}$}\coordHE{}
on the space of first cohomology are equal,
\begin{equation}\coord{}\boxEquation{\label{T27}
G_{IJ}=H_{IJ}.
}{G_{IJ}=H_{IJ}.
}{ecuacion}\coordE{}\end{equation}
The equality \eqref{T27} follows from the duality in de Rham cohomology, i.e.
\begin{equation}\coord{}\boxEquation{
\begin{split}
\label{T28} G_{IJ}&=(\omega_{I}, \omega_{J})=
\int_{\mathcal{M}^{3}} \omega_{I} \wedge \ast\; \omega_{J}=
\int_{\mathcal{M}^{3}}
\omega_{I} \wedge \alpha_{J}\\
&= \int_{\mathcal{M}^{3}} \ast \; \alpha_{I} \wedge \alpha_{J}=
(\alpha_{I}, \alpha_{J})=H_{IJ},
\end{split}
}{
\begin{split}
G_{IJ}&=(\omega_{I}, \omega_{J})=
\int_{\mathcal{M}^{3}} \omega_{I} \wedge \ast\; \omega_{J}=
\int_{\mathcal{M}^{3}}
\omega_{I} \wedge \alpha_{J}\\
&= \int_{\mathcal{M}^{3}} \ast \; \alpha_{I} \wedge \alpha_{J}=
(\alpha_{I}, \alpha_{J})=H_{IJ},
\end{split}
}{split}\coordE{}\end{equation}
where by definition
\begin{equation}\coord{}\boxEquation{\label{T29}
\alpha_{I} = \ast \; \omega_{I}.
}{\alpha_{I} = \ast \; \omega_{I}.
}{ecuacion}\coordE{}\end{equation}
Eq.\eqref{T29} imposes the lacking normalization condition for
one-forms \myHighlight{$\alpha_{I}$}\coordHE{}, announced after Eq.\eqref{T19}, and crucial
for Eq.\eqref{T27}. This way, the both "classical parts" are
exactly dual in the sense of "\myHighlight{$e^{2}$}\coordHE{} goes to \myHighlight{$1/{e^{2}}$}\coordHE{}"
equivalence.

The "quantum parts" are definitely different. As far as the
determinants are concerned, the possible equality of Eq.\eqref{T3}
and Eq.\eqref{T16} would imply
\begin{equation}\coord{}\boxEquation{\label{T30}
\frac{\rm det^{'} \Delta_{0} }
{(\rm det^{'}\Delta_{1})^\frac{1}{2} }\;
\frac{1}{ (\rm det^{'}\Delta_{0})^{- \frac{1}{2} }}=1.
}{\frac{\rm det^{'} \Delta_{0} }
{(\rm det^{'}\Delta_{1})^\frac{1}{2} }\;
\frac{1}{ (\rm det^{'}\Delta_{0})^{- \frac{1}{2} }}=1.
}{ecuacion}\coordE{}\end{equation}
But the LHS of Eq.\eqref{T30} is exactly the inverse of the
Ray--Singer analytic torsion [\ref{B06}] which is not equal to 1,
in general. On the other hand treating the LHS of Eq.\eqref{T30}
as a formal ratio of volumes of spaces of forms [\ref{B07}] gives
1, which is in  accordance with the RHS of Eq.\eqref{T30}, and is in favor of duality. Then
analytic torsion following from (zeta) regularization of
determinants can be considered as a "volume ratio anomaly".

Another kind of regularization yields different powers of the
coupling constant \myHighlight{$e$}\coordHE{} in front of the both partition functions.
More precisely, the regularized partition function \eqref{T2} of \myHighlight{$U(1)$}\coordHE{} gauge theory should read
\begin{equation}\coord{}\boxEquation{\label{T31}
Z_{U(1)}(e) = (2 \pi e)^{B_{0}-B_{1}}\; \int\mathcal{D}A\; e^{- S[A]},
}{Z_{U(1)}(e) = (2 \pi e)^{B_{0}-B_{1}}\; \int\mathcal{D}A\; e^{- S[A]},
}{ecuacion}\coordE{}\end{equation}
whereas the partition function \eqref{T15} of \myHighlight{$S$}\coordHE{}-valued scalar theory is of the form
\begin{equation}\coord{}\boxEquation{\label{T32}
Z_{\phi} (e) =\left(\frac{e}{2 \pi}\right)^{B_{0}}\; \int D\phi \; e^{-S[\phi]},
}{Z_{\phi} (e) =\left(\frac{e}{2 \pi}\right)^{B_{0}}\; \int D\phi \; e^{-S[\phi]},
}{ecuacion}\coordE{}\end{equation}
where \myHighlight{$B_{k}$}\coordHE{} is the (infinite) dimension of the space of \myHighlight{$k$}\coordHE{}-forms on \myHighlight{$\mathcal{M}^{3}$}\coordHE{}.
Upon procedure described earlier we obtain the desired explicit forms of the partition functions, Eq.\eqref{T12} and
Eq.\eqref{T25}.

One could easily generalize our considerations to the case of
Abelian \myHighlight{$U(1)^{n}$}\coordHE{} gauge theory. The dual scalar model would
assume values in \myHighlight{$n$}\coordHE{}-dimensional torus \myHighlight{$T^{n}$}\coordHE{}, or we could speak
on (linear) \myHighlight{$\sigma$}\coordHE{}-model with target space \myHighlight{$T^{n}$}\coordHE{}. Obviously,
the whole analysis would be analogous to the above one.
\begin{thebibliography}{99}

\bibitem {label1}\label{B01} E. Witten, 1995  {\it On S-Duality In Abelian Gauge Theory},
e-print archive: hep-th/9505186.

\bibitem{label2}\label{B02} E. Verlinde, 1995  {\it Global Aspects Of Electric-Magnetic Duality},
Nucl. Phys.  B {\bf 455}, 211--228.

\bibitem{label3}\label{B03} E. Witten, 1999 {\it Lecture II-7: Abelian Duality} in {\it Quantum
Fields and Strings: A Course For Mathematicians}, vol. 1,2.\, ed.
P. Deligne, P. Etingof, D.S. Freed, L.C. Jeffrey, D. Kazhdan, J.W. Morgan, D.R. Morrison, E. Witten, AMS, Providence,
http://www.math.ias.edu/QFT/spring/witten7.ps

\bibitem{label4}\label{B04} E.M. Prodanov, S. Sen, 2000
{\it Abelian Duality}, Phys. Rev. D {\bf 62}, 045009--045013.

\bibitem{label5}\label{B05} L. Rozansky, 1995 {\it A Large K
Asymptotics of Wiiten's Invariant of Seifert Manifolds}, Commun.
Math. Phys. {\bf 171} 279--322.

\bibitem{label6}\label{B06}
D. Birmingham, M. Blau, M. Rakowski, G. Thompson, 1991 {\it
Topological Field Theory}, Phys. Rept. {\bf 209} 129--340.

\bibitem{label7}\label{B07} D.H. Adams, E.M. Prodanov, 2000 {\it A Remark
on Schwarz's Topological Field Theory}, Lett. Math. Phys. {\bf 51}
249--255.

\end{thebibliography}
\end{document}

Title: Abelian duality in three dimensions
Authors: Boguslaw Broda and Grzegorz Duniec
Comments: 10 pages (more loosely formatted), minor changes


Abstract: Abelian duality on three-dimensional general Riemannian closed manifold M3 is considered. Partition functions for U(1) gauge theory and circle-valued scalar theory on M3 are explicitly calculated and compared.

\bye
