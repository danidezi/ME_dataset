
\documentclass[a4paper,aps, amssymb, preprint, 12pt]{revtex4}
\draft
\usepackage{useful_macros}
\begin{document}

\parbox{4cm}{\vspace*{2cm}}

\title{Fermionic zeromodes in heterotic fivebrane backgrounds}
\author{Yasutaka Katagiri}
\thanks{e-mail: kata@phys.metro-u.ac.jp}
\address{Department of Physics, Tokyo Metropolitan University, Hachioji, Tokyo 192-0397, Japan}
\date{\today}
\author{Noriaki Kitazawa}
\thanks{e-mail: kitazawa@phys.metro-u.ac.jp}
\address{Department of Physics, Tokyo Metropolitan University, Hachioji, Tokyo 192-0397, Japan}

\begin{abstract}
We show the existence of fermionic zeromodes in heterotic fivebrane backgrounds.
By solving fermionic field equations explicitly, the normalizable and physical zeromodes are obtained.
The zeromodes have a non-vanishing gravitino component, and a possible scenario of the gravitino condensation with the gaugino condensation is suggested.
\end{abstract}

\pacs{}
\maketitle

\newpage

\section{Introduction}
Supersymmetry is a very attractive symmetry.
This symmetry solves the naturalness problem in an elegant manner, and consistent superstring theory needs this symmetry.
But this symmetry must be broken spontaneously in some way, since we do not have any concrete experimental evidences which support supersymmetry at low energies.
It is desirable that this symmetry is broken dynamically, and many models with the hidden sector in which the supersymmetry is broken dynamically has been proposed (details can be found in Ref.\cite{treelevel_SUSY_breaking_1, treelevel_SUSY_breaking_2}).
But there are large degrees of freedom in the choice of the structure of the hidden sector, and the mechanism of the supersymmetry breaking should inevitably depends on each choice.

On the other hand, the local supersymmetry can be broken purely by the topological effect of the gravitation\cite{Witten}.
This scenario is very attractive because of its less model dependence.
As an explicit realization of this scenario, the gravitino condensation \myHighlight{$\langle \psi_{ab}\psi^{ab}\rangle\neq 0$}\coordHE{} in four-dimensional Eguchi-Hanson metric is well-known\cite{EH, Konishi}.
In this background, the non-zero gravitino condensation which means supersymmetry breaking through the Konishi anomaly relation is induced due to the zeromodes.

It is interesting question whether this scenario is possible in the string theory.
The heterotic fivebrane, or Callan-Harvey-Strominger (CHS) solitons are known as exact solutions of the heterotic string theory\cite{CHS}.
These solitons keep just a half of the supersymmetry compared with that the heterotic string theory originally has.
In Ref.\cite{Rey}, the gravitational instanton whose geometry is similar to the heterotic fivebrane has been studied and the existence of the zeromodes which can contribute to the gravitino condensation was explicitly shown.
Therefore, we can expect the existence of zeromodes also in the heterotic fivebrane backgrounds, though these solutions have differences in their field components.

Though the supersymmetry breaking in ten-dimensional spacetime has no direct relations to the supersymmetry breaking in the real world, it is interesting to study the dynamics of fermions in the string theory which could trigger some interesting phenomena like supersymmetry breaking.

In this paper, we find zeromodes on the heterotic fivebrane by solving fermionic field equations explicitly.
The normalizability of the solution is confirmed, and the solution can be regarded properly as the well-defined mode.
Since the zeromodes are derived from the supersymmetry variations essentially, the issue whether the zeromodes can be gauged away is discussed also.
We conclude that the zeromodes are physical and can contribute to the gravitino condensation.

This paper is organized as follows.
A short review of the heterotic fivebrane is given in the next section.
In sec.\ref{sec:three}, following Ref.\cite{Bellisai}, we explicitly derive the fermionic field equations and discuss some aspects of the zeromodes in the heterotic fivebrane backgrounds.
In sec.\ref{sec:four} we obtain the zeromodes by solving fermionic field equations explicitly.
Discussions about the zeromodes are given also in this section.
The conclusion is given in sec.\ref{sec:five}.


\section{short review of the heterotic fivebrane soliton}
\label{sec:two}
The soliton is the solution of the classical field equation, and it is distinguished from other vacuum solutions by being transitionally non-invariant.
In the string theory, many objects with a solitonic property have been discovered.
Especially, the heterotic fivebrane, or the CHS soliton, is one of the important solitons.
It is obtained as the exact solution of the heterotic supergravity, and moreover, the validity as a solution of the heterotic string theory can be confirmed using the conformal field theory in a certain limit.
In this section, we give a short review of the construction of this soliton.

The action of the heterotic supergravity is as follows\cite{BdR,Bellisai}.
\begin{eqnarray}\coord{}\boxAlignEqnarray{\leftCoord{}  %heterotic SUGRA action
S_{hetero} &=& \int d^{10}x\left\{{\cal L}_{B}+{\cal L}_{F} \right \}, \rightCoord{}\label{S_hetero}\rightCoord{}\\\leftCoord{}
{\rightCoord{}\leftCoord{}\cal L}_{B} &=&
\sqrt{g}e^{-2\Phi}\left\{-\frac{\leftCoord{}1}{\rightCoord{}2}R(\omega)-\frac{\leftCoord{}1}{\rightCoord{}6}H_{MNP}H^{MNP}+2\partial_\mu\Phi\partial^\mu\Phi -\frac{\leftCoord{}1}{\rightCoord{}4}F^{\alpha}_{MN}F^{\alpha MN}\right\}, \rightCoord{}\label{bosonic_Lagrangian} \rightCoord{}\\\leftCoord{}
{\rightCoord{}\leftCoord{}\cal L}_{F} &=& 
\sqrt{g}e^{-2\Phi}\left\{ \rightCoord{}
\leftCoord{}-\frac{\leftCoord{}1}{\rightCoord{}2}\overline{\chi}^\alpha\Gamma^M{\cal D}(\omega, A)_M\chi^\alpha
\leftCoord{}-\frac{\leftCoord{}1}{\rightCoord{}2}\overline{\psi}_M\Gamma^{MNP}D(\omega)_N\psi_P \right. \nonumber \rightCoord{}\\
&&\leftCoord{}{}\left.-\overline{\psi}_M\Gamma^M\partial^N\Phi\psi_N
\leftCoord{}-\overline{\lambda}\Gamma^M D(\omega)_M\lambda
\leftCoord{}-\frac{\leftCoord{}1}{\rightCoord{}2}\overline{\psi}_M\Gamma^N\partial_N\Phi\Gamma^M\lambda \right.\nonumber \rightCoord{}\\
&&\leftCoord{}{}\left.+\frac{\leftCoord{}1}{\rightCoord{}24}H_{RST}\left[
\overline{\psi}_M\Gamma^{MRSTN}\psi_N
\leftCoord{}+6\overline{\psi}^R\Gamma^S\psi^T \rightCoord{}
\leftCoord{}+2\overline{\psi}_M\left(\Gamma^{MRST}-3g^{MT}\Gamma^{RS}\right)\lambda
\right] \right. \nonumber \rightCoord{}\\
&&\leftCoord{}{}-\left.\frac{\leftCoord{}1}{\rightCoord{}4}\overline{\chi}^\alpha\left( \rightCoord{}
\Gamma^{MNP}F^\alpha_{NP}+2{F^{\alpha M}}_{P}\Gamma^P
\right)\psi_M
\leftCoord{}-\frac{\leftCoord{}1}{\rightCoord{}4}\overline{\chi}^\alpha\Gamma^{NP}F^\alpha_{NP}\lambda \right.\nonumber \rightCoord{}\\
&&\leftCoord{}{}\left.+\frac{\leftCoord{}1}{\rightCoord{}24}H_{MNP}\:tr\left(\overline{\chi}\Gamma^{MNP}\chi\right)\right\}, \rightCoord{}\label{fermionic_Lagrangian}
\rightCoord{}}{0mm}{25}{26}{theequation}{1}\coordE{}\end{eqnarray}
where \myHighlight{$M,N,\ldots$}\coordHE{} are the indices of ten dimensional spacetime coordinates, \myHighlight{$\alpha$}\coordHE{} stands for \myHighlight{$E_8\times E_8$}\coordHE{} adjoint index and \myHighlight{$\Gamma^{MNP\ldots}$}\coordHE{} means totally anti-symmetrized gamma matrix (for example, \myHighlight{$\Gamma^{MN}$}\coordHE{} is defined as \myHighlight{$\Gamma^{MN}=1/2!(\Gamma^M\Gamma^N-\Gamma^N\Gamma^M))$}\coordHE{}.
This action has \myHighlight{${\cal N}=1$}\coordHE{} local supersymmetry.
\myHighlight{$H$}\coordHE{} is the three-form tensor field, and which plays an important role in the construction of the heterotic fivebrane.
It is related to the two form potential \myHighlight{$B$}\coordHE{} as Ref.\cite{GS}
\begin{eqnarray}\coord{}\boxAlignEqnarray{\leftCoord{}
H=dB+\alpha'\left(\omega_{3}^{Lotentz}(\Omega_{+}) -\frac{\leftCoord{}1}{\rightCoord{}30}\omega_{3}^{YM}(A)\right)+\ldots, \rightCoord{}\label{H_definition}
\rightCoord{}}{0mm}{2}{4}{
H=dB+\alpha'\left(\omega_{3}^{Lotentz}(\Omega_{+}) -\frac{1}{30}\omega_{3}^{YM}(A)\right)+\ldots, }{1}\coordE{}\end{eqnarray}
where dots means higher order \myHighlight{$\alpha'$}\coordHE{} corrections and \myHighlight{$\omega_{3}^{Lotentz}(\Omega_{+})$}\coordHE{} and \myHighlight{$\omega_{3}^{YM}(A)$}\coordHE{} are the Chern-Simons three-forms whose exterior derivative give \myHighlight{$TrR(\Omega_{+})\wedge{R}(\Omega_{+})$}\coordHE{} and \myHighlight{$trF(A)\wedge{F}(A)$}\coordHE{} respectively.
Taking the exterior derivative of Eq.(\ref{H_definition}), the famous anomalous Bianchi identity is obtained.
\begin{eqnarray}\coord{}\boxAlignEqnarray{\leftCoord{}
dH= \alpha'\left(TrR(\Omega_{+})\wedge R(\Omega_{+})- \frac{\leftCoord{}1}{\rightCoord{}30} trF(A)\wedge F(A)\right)+ \ldots. \rightCoord{}\label{anomalous_Bid}
\rightCoord{}}{0mm}{2}{4}{
dH= \alpha'\left(TrR(\Omega_{+})\wedge R(\Omega_{+})- \frac{1}{30} trF(A)\wedge F(A)\right)+ \ldots. }{1}\coordE{}\end{eqnarray}
The generalized spin connection \myHighlight{$\Omega_\pm$}\coordHE{} is defined as
\begin{eqnarray}\coord{}\boxAlignEqnarray{\leftCoord{}
{\rightCoord{}\leftCoord{}\Omega_{\pm M}}^{AB}={\omega_{M}}^{AB}\pm{{H_{M}}}^{AB}, \rightCoord{}\label{definition_Omega}
\rightCoord{}}{0mm}{2}{4}{
{\Omega_{\pm M}}^{AB}={\omega_{M}}^{AB}\pm{{H_{M}}}^{AB}, }{1}\coordE{}\end{eqnarray}
where \myHighlight{${\omega_M}^{AB}$}\coordHE{} is the usual spin connection and \myHighlight{$H$}\coordHE{} plays a role of a tortion.
In order to obtain the solitonic solution, it is convenient to look for the bosonic backgrounds which are invariant under a part of \myHighlight{${\cal N}=1$}\coordHE{} supersymmetry transformations, since only the vacuum is invariant under all of the supersymmetry transformations.

In zero fermionic backgrounds, the supersymmetry transformations take especially simple form.
All the variations of bosonic fields vanish and those of fermionic fields are given as follows\cite{Bellisai}.
\begin{eqnarray}\coord{}\boxAlignEqnarray{\leftCoord{}  %SUGRA variation
\delta\psi_M &=& \left(\partial_M+\frac{\leftCoord{}1}{\rightCoord{}4}{\Omega_{-M}}^{AB}\Gamma_{AB}\right)\epsilon +\frac{\leftCoord{}1}{\rightCoord{}2}\Gamma_M\delta\lambda, \rightCoord{}\label{GrV}\rightCoord{}\\\leftCoord{}
\delta\chi^\alpha &=& -\frac{\leftCoord{}1}{\rightCoord{}4}F_{MN}^{\alpha}\Gamma^{MN}\epsilon, \rightCoord{}\label{GaV}\rightCoord{}\\\leftCoord{}
\delta\lambda  &=& -\frac{\leftCoord{}1}{\rightCoord{}4}\left(\Gamma^M\partial_M\Phi-\frac{\leftCoord{}1}{\rightCoord{}6}H_{MNP}\Gamma^{MNP}\right)\epsilon, \rightCoord{}\label{DlV}
\rightCoord{}}{0mm}{7}{12}{theequation}{1}\coordE{}\end{eqnarray}
where the infinitesimal gauge parameter \myHighlight{$\epsilon$}\coordHE{} belongs to the {\bf 16} dimensional Majorana-Weyl spinor representation with positive ten-dimensional chirality.

Now we specify the type of the soliton what we are looking for.
Here, we adopt the fivebrane as a soliton we construct.
The solitonic object which is extended over the p-dimensional space is called p-brane.
Moreover, we take all the fields independent of coordinates along the 5-brane.
As a consequence of these ansatz, the Lorentz symmetry of the solution is reduced to \myHighlight{$SO(9,1) \to SO(5,1)\times SO(4)$}\coordHE{}, and the supersymmetry transformation is parametrized by \myHighlight{$\epsilon_{+}\oplus\epsilon_{-}=(4,2_{+})\oplus(4^*,2_{-})$}\coordHE{}.
Taking into account that the tensor field components whose indices are along the 5-brane vanish, one can verify that the bosonic backgrounds 
\begin{eqnarray}\coord{}\boxAlignEqnarray{\leftCoord{}   %CHS soliton
F_{\mu\nu} &=& \tilde{F}_{\mu\nu}, \rightCoord{}\\\leftCoord{}
H_{\mu\nu\rho} &=& -\sqrt{g}\epsilon_{\mu\nu\rho\sigma}\partial^\sigma\Phi, \rightCoord{}\label{H} \rightCoord{}\\\leftCoord{}
g_{\mu\nu} &=& e^{2\Phi}\delta_{\mu\nu}, \rightCoord{}\label{metric}
\rightCoord{}}{0mm}{2}{6}{theequation}{1}\coordE{}\end{eqnarray}
are invariant under just a half of the \myHighlight{${\cal N}=1$}\coordHE{} supersymmetry transformations which is induced by the spinor \myHighlight{$\epsilon_+$}\coordHE{}.
Here, \myHighlight{$\mu,\nu,\ldots$}\coordHE{} stand for the four-dimensional Euclidean coordinates transverse to the 5-brane.

The explicit form of the dilaton field \myHighlight{$\Phi$}\coordHE{} is not determined yet.
It can be derived from the consistency condition of the above ansatz.
With our ansatz, the identity of Eq.(\ref{anomalous_Bid}) is rewritten as 
\begin{eqnarray}\coord{}\boxAlignEqnarray{\leftCoord{}
e^{-2\Phi}\square e^{2\Phi} = \alpha'\left(TrR(\Omega_{+})\wedge R(\Omega_{+})- \frac{\leftCoord{}1}{\rightCoord{}30}trF(A)\wedge F(A)\right) +\ldots,
\rightCoord{}}{0mm}{2}{3}{
e^{-2\Phi}\square e^{2\Phi} = \alpha'\left(TrR(\Omega_{+})\wedge R(\Omega_{+})- \frac{1}{30}trF(A)\wedge F(A)\right) +\ldots,
}{1}\coordE{}\end{eqnarray}
Here, we take the selfdual gauge instanton \myHighlight{$A_\mu^\alpha(x)$}\coordHE{} as \myHighlight{$SU(2)$}\coordHE{} part of \myHighlight{$SU(2)\times E_7 \subset E_8$}\coordHE{} and make following identification.
\begin{eqnarray}\coord{}\boxAlignEqnarray{\leftCoord{}
R_{\mu\nu}(\Omega_{+})^{ab}=\frac{\leftCoord{}1}{\rightCoord{}2}\overline{\eta}^{Iab}F^I_{\mu\nu},
\rightCoord{}}{0mm}{2}{3}{
R_{\mu\nu}(\Omega_{+})^{ab}=\frac{1}{2}\overline{\eta}^{Iab}F^I_{\mu\nu},
}{1}\coordE{}\end{eqnarray}
where \myHighlight{$I$}\coordHE{} is the \myHighlight{$SU(2)$}\coordHE{} adjoint index and \myHighlight{$\overline{\eta}$}\coordHE{} is the 't Hoot's unti-selfdual \myHighlight{$\eta$}\coordHE{}  symbol.
With the ansatz (\ref{H}) and (\ref{metric}), this identification is ensured under the condition of \myHighlight{$\square e^{2\Phi}=0$}\coordHE{}, where \myHighlight{$\square$}\coordHE{} is the usual four-dimensional Laplacian \myHighlight{$\delta^{ab}\partial_a\partial_b$}\coordHE{}.
This is the analogy of the familiar trick used in the Calabi-Yau compactification of the heterotic string theory (standard embedding)\cite{Calabi-Yau_1, Calabi-Yau_2}.
This identification annihilates all the anomalous \myHighlight{$\alpha'$}\coordHE{} correction to the identity Eq.(\ref{anomalous_Bid}) and makes it simple and exact one: \myHighlight{$e^{-2\Phi}\square e^{2\Phi}=0$}\coordHE{}.

As a solution of the equation of \myHighlight{$\square e^{2\Phi}=0$}\coordHE{}, we adopt the following dilaton field \myHighlight{$\Phi$}\coordHE{} which was given originally in Ref.\cite{CHS} (other solutions are found in Ref.\cite{Dilaton_sol_1, Dilaton_sol_2, Dilaton_sol_3}).
\begin{eqnarray}\coord{}\boxAlignEqnarray{\leftCoord{}
e^{2\Phi} = e^{2\Phi_0}+\sum_{\rightCoord{}I\rightCoord{}}\frac{\leftCoord{}{\cal Q}_I}{\rightCoord{}\left(x-x_{0I}\right)^2},
\qquad {\cal Q}_I \in {\bf Z}\rightCoord{}\label{Dilaton},
\rightCoord{}}{0mm}{2}{6}{
e^{2\Phi} = e^{2\Phi_0}+\sum_{I}\frac{{\cal Q}_I}{\left(x-x_{0I}\right)^2},
\qquad {\cal Q}_I \in {\bf Z},
}{1}\coordE{}\end{eqnarray}
where the constant term \myHighlight{$e^{2\Phi_0}$}\coordHE{} is determined by the value of the dilaton at the infinity and \myHighlight{$x_{0i}$}\coordHE{} is the position where the solitons are located.
\myHighlight{${\cal Q}_I$}\coordHE{} means the charge of the heterotic fivebrane, and is quantized so that the string world sheet theory is consistently defined in this background\cite{Charge_quantization}.
The global geometry of this soliton is regarded as half-cylinders glued into a flat Euclidean space, or a collections of semi-wormholes.
We will consider only the single heterotic fivebrane from now on.


\section{Fermionic field equations}
\label{sec:three}
In the previous section, the concrete structure of the heterotic fivebrane was introduced.
We are now ready to discuss about the fermionic zeromodes in this backgrounds.
Following Ref.\cite{Bellisai}, the fermionic field equations are explicitly derived from the action (\ref{S_hetero}) as follows.
\begin{eqnarray}\coord{}\boxAlignEqnarray{\leftCoord{}
e^{-2\Phi}&&\bigg[-\Gamma^{MNP}\left(D_N-\partial_N\Phi\right)\psi_P
\leftCoord{}+\frac{\leftCoord{}1}{\rightCoord{}12}H_{RST}\Gamma^{MRSTN}\psi_N \nonumber \rightCoord{}\\
&&\leftCoord{}{}-\frac{\leftCoord{}1}{\rightCoord{}2}\left(\Gamma^N\partial_N\Phi+\frac{\leftCoord{}1}{\rightCoord{}6}H_{RST}\Gamma^{RST}\right)\Gamma^M\lambda
\leftCoord{}-\partial^N\Phi\Gamma^M\psi_N+\partial^M\Phi\Gamma^N\psi_N \nonumber \rightCoord{}\\
&&\leftCoord{}{}+\frac{\leftCoord{}1}{\rightCoord{}2}H^{MNP}\Gamma_N\psi_P
\leftCoord{}-\frac{\leftCoord{}1}{\rightCoord{}4}\left(\Gamma^{MNP}F^\alpha_{NP}-2{F^{\alpha M}}_P\Gamma^P\right)\chi^\alpha\bigg]=0,\rightCoord{}\\\leftCoord{}
e^{-2\Phi}&&\bigg[-\Gamma^M\left({\cal D}_M-\partial_M\Phi\right)\chi^\alpha
\leftCoord{}+\frac{\leftCoord{}1}{\rightCoord{}12}H_{MNP}\Gamma^{MNP}\chi^\alpha \nonumber \rightCoord{}\\
&&\leftCoord{}{}-\frac{\leftCoord{}1}{\rightCoord{}4}\left(\Gamma^{MNP}F^\alpha_{NP}+2{F^{\alpha M}}_P\Gamma^P\right)\psi_M
\leftCoord{}-\frac{\leftCoord{}1}{\rightCoord{}4}\Gamma^{NP}F^{\alpha}_{NP}\lambda \bigg]=0,\rightCoord{}\\\leftCoord{}
e^{-2\Phi}&&\bigg[-2\Gamma^M\left(D_M-\partial_M\Phi\right)\lambda
\leftCoord{}+\frac{\leftCoord{}1}{\rightCoord{}4}\Gamma^{NP}F^{\alpha}_{NP}\chi^\alpha \nonumber \rightCoord{}\\
&&\leftCoord{}{}-\frac{\leftCoord{}1}{\rightCoord{}2}\Gamma^M\left(\Gamma^N\partial_N\Phi-\frac{\leftCoord{}1}{\rightCoord{}6}\Gamma^{RST}H_{RST}\right)\psi_M\bigg]=0.\rightCoord{}
\rightCoord{}}{0mm}{24}{20}{
e^{-2\Phi}&&\bigg[-\Gamma^{MNP}\left(D_N-\partial_N\Phi\right)\psi_P
+\frac{1}{12}H_{RST}\Gamma^{MRSTN}\psi_N \\
&&{}-\frac{1}{2}\left(\Gamma^N\partial_N\Phi+\frac{1}{6}H_{RST}\Gamma^{RST}\right)\Gamma^M\lambda
-\partial^N\Phi\Gamma^M\psi_N+\partial^M\Phi\Gamma^N\psi_N \\
&&{}+\frac{1}{2}H^{MNP}\Gamma_N\psi_P
-\frac{1}{4}\left(\Gamma^{MNP}F^\alpha_{NP}-2{F^{\alpha M}}_P\Gamma^P\right)\chi^\alpha\bigg]=0,\\
e^{-2\Phi}&&\bigg[-\Gamma^M\left({\cal D}_M-\partial_M\Phi\right)\chi^\alpha
+\frac{1}{12}H_{MNP}\Gamma^{MNP}\chi^\alpha \\
&&{}-\frac{1}{4}\left(\Gamma^{MNP}F^\alpha_{NP}+2{F^{\alpha M}}_P\Gamma^P\right)\psi_M
-\frac{1}{4}\Gamma^{NP}F^{\alpha}_{NP}\lambda \bigg]=0,\\
e^{-2\Phi}&&\bigg[-2\Gamma^M\left(D_M-\partial_M\Phi\right)\lambda
+\frac{1}{4}\Gamma^{NP}F^{\alpha}_{NP}\chi^\alpha \\
&&{}-\frac{1}{2}\Gamma^M\left(\Gamma^N\partial_N\Phi-\frac{1}{6}\Gamma^{RST}H_{RST}\right)\psi_M\bigg]=0.
}{1}\coordE{}\end{eqnarray}
In the heterotic fivebrane backgrounds, these field equations are decomposed into two parts.
The one in the Minkowskian spacetime along the fivebrane are trivial and neglected here.
The rest of the field equations referring to the four-dimensional space, which is transverse to the fivebrane, are given as follows.
\begin{eqnarray}\coord{}\boxAlignEqnarray{\leftCoord{}
\leftCoord{}-\gamma^{\mu\nu\rho}\left(D_\nu-\partial_\nu\Phi\right)\psi_\rho
\leftCoord{}-\partial^\rho\Phi\gamma^\mu\psi_\rho
\leftCoord{}+\partial^\mu\Phi\gamma^\rho\psi_\rho
\leftCoord{}+\frac{\leftCoord{}1}{\rightCoord{}2}H^{\mu\nu\rho}\gamma_\nu\psi_\rho=0, \rightCoord{}\label{gravitino_eq} \rightCoord{}\\\leftCoord{}
\leftCoord{}-\gamma^\mu\left({\cal D}_\mu-\partial_\mu\Phi\right)\chi^I
\leftCoord{}+\frac{\leftCoord{}1}{\rightCoord{}12}H_{\mu\nu\rho}\gamma^{\mu\nu\rho}\chi^I
\leftCoord{}-{F^{I\mu}}_\rho\gamma^\rho\psi_\mu=0, \rightCoord{}\label{gaugino_eq} \rightCoord{}\\\leftCoord{}
\leftCoord{}-2\gamma^\mu\left(D_\mu-\partial_\mu\Phi\right)\lambda
\leftCoord{}-\gamma^\mu\gamma^\nu\partial_\nu\Phi\psi_\mu
\leftCoord{}+\frac{\leftCoord{}1}{\rightCoord{}4}F^I_{\mu\nu}\gamma^{\mu\nu}\chi^I=0, \rightCoord{}\label{dilatino_eq}
\rightCoord{}}{0mm}{16}{10}{
-\gamma^{\mu\nu\rho}\left(D_\nu-\partial_\nu\Phi\right)\psi_\rho
-\partial^\rho\Phi\gamma^\mu\psi_\rho
+\partial^\mu\Phi\gamma^\rho\psi_\rho
+\frac{1}{2}H^{\mu\nu\rho}\gamma_\nu\psi_\rho=0, \\
-\gamma^\mu\left({\cal D}_\mu-\partial_\mu\Phi\right)\chi^I
+\frac{1}{12}H_{\mu\nu\rho}\gamma^{\mu\nu\rho}\chi^I
-{F^{I\mu}}_\rho\gamma^\rho\psi_\mu=0, \\
-2\gamma^\mu\left(D_\mu-\partial_\mu\Phi\right)\lambda
-\gamma^\mu\gamma^\nu\partial_\nu\Phi\psi_\mu
+\frac{1}{4}F^I_{\mu\nu}\gamma^{\mu\nu}\chi^I=0, }{1}\coordE{}\end{eqnarray}
where \myHighlight{$I$}\coordHE{} is the \myHighlight{$SU(2)$}\coordHE{} adjoint index, and ten-dimensional gamma matrix \myHighlight{$\Gamma^M$}\coordHE{} is replaced with four-dimensional \myHighlight{$\gamma^\mu$}\coordHE{}.

What we are interested in here is the well-defined solution of the above equations.
To discuss about this issue, it is convenient to define the generalized Dirac operator \myHighlight{${\cal D}$}\coordHE{} as a \myHighlight{$3\times 3$}\coordHE{} matrix like operator which acts on the generalized spinor \myHighlight{$\Psi=(\psi_\rho,\chi^I,\lambda)$}\coordHE{}.
With these \myHighlight{$\cal D$}\coordHE{} and \myHighlight{$\Psi$}\coordHE{}, above field equations (\ref{gravitino_eq}), (\ref{gaugino_eq}) and (\ref{dilatino_eq}) are represented simply as \myHighlight{${\cal D}\Psi=0$}\coordHE{}.
In order to know whether \myHighlight{$\cal D$}\coordHE{} has zeromodes or not, index theorem is available.
With this theorem, it is possible to know the number of zeromodes which the Dirac operator has.
But, while the index formula for the usual Dirac operator (which acts on a Dirac spinor) is well-known\cite{APS}, the one for the operator \myHighlight{${\cal D}$}\coordHE{} has not been known so far (the existence of the background \myHighlight{$H$}\coordHE{} and the Rarita-Shwinger operator in \myHighlight{${\cal D}$}\coordHE{} make the issue difficult).
Nevertheless, the number of the zeromode of the Rarita-Shwinger operator in the heterotic fivebrane  backgrounds has been counted in Ref.\cite{Bellisai}, and the result indicates no gravitino zeromodes.
But as the author of Ref.\cite{Bellisai} says, this result might not be mathematically correct in a strict sense.

Without relying on the index theorem, we can obtain zeromodes explicitly as the supersymmetry variation \myHighlight{$\delta_{\epsilon_-}\Psi$}\coordHE{} in general.
In the heterotic fivebrane backgrounds, \myHighlight{$\delta_{\epsilon}\Psi$}\coordHE{} is obtained as the supersymmetry variations (\ref{GrV}), (\ref{GaV}) and (\ref{DlV}) which is induced by the \myHighlight{$\epsilon_-$}\coordHE{} (recall that heterotic fivebrane holds half of the supersymmetry).
\begin{eqnarray}\coord{}\boxAlignEqnarray{\leftCoord{} 
\delta_{\epsilon_-}\Psi = \left(
\begin{array}{c} \rightCoord{}
\left(\partial_\mu+\frac{\leftCoord{}1}{\rightCoord{}4}{\rightCoord{}\Omega_{-\mu}}^{ab}\gamma_{ab}\rightCoord{}
\leftCoord{}-\frac{\leftCoord{}1}{\rightCoord{}4}\partial_\nu\Phi\gamma_\mu\gamma^\nu\right)\epsilon_- \rightCoord{}\\\leftCoord{}
\leftCoord{}-\frac{\leftCoord{}1}{\rightCoord{}4}F_{\mu\nu}^{\alpha}\gamma^{\mu\nu}\epsilon_-\rightCoord{}\\\leftCoord{}
\leftCoord{}-\frac{\leftCoord{}1}{\rightCoord{}2}\partial_\nu\Phi\gamma^\nu\epsilon_-\rightCoord{}\\\leftCoord{}
\end{array} \rightCoord{}
\right). \rightCoord{}
\rightCoord{}}{0mm}{11}{14}{ 
\delta_{\epsilon_-}\Psi = \left(
\begin{array}{c} 
\left(\partial_\mu+\frac{1}{4}{\Omega_{-\mu}}^{ab}\gamma_{ab}
-\frac{1}{4}\partial_\nu\Phi\gamma_\mu\gamma^\nu\right)\epsilon_- \\
-\frac{1}{4}F_{\mu\nu}^{\alpha}\gamma^{\mu\nu}\epsilon_-\\
-\frac{1}{2}\partial_\nu\Phi\gamma^\nu\epsilon_-\\
\end{array} 
\right). 
}{1}\coordE{}\end{eqnarray}
After some calculations, one can find that this \myHighlight{$\delta_{\epsilon_-}\Psi$}\coordHE{} does not satisfy the field equation, namely, \myHighlight{${\cal D}\delta_{\epsilon_-}\Psi \neq 0$}\coordHE{}.
Therefore, we cannot obtain the zeromodes as supersymmetry variations.
The reason why the supersymmetry variation cannot satisfy the field equation is unclear.
It may result from the geometry of the heterotic fivebrane, that is, the existence of the boundary at the throat of semi-wormhole.
In general, the introduction of the boundary changes bulk field equations, and to keep these equations unchanged the boundary terms are required\cite{Gibbons-Hawking}.
With the analogy of this fact, the action of heterotic supergravity in this backgrounds receives the boundary terms.
Such a corrections may alter the supersymmetry transformation rules, and the variation rules of Eqs.(\ref{GrV}), (\ref{GaV}) and (\ref{DlV}) cease to come from the symmetry transformation of the action.
As a result, they cannot satisfy the field equations.
This is just a possible scenario.
Further future investigations are required to clarify this problem.

Many approaches to find the zeromode are available.
But by the above reasons, we have no choice except for explicitly solving fermionic field equations.
We try it in the next section.


\section{Explicit calculations}
\label{sec:four}
To solve the field equations, we take the following approach.
As we have seen in the previous section, the supersymmetry variations are natural candidates for the zeromodes.
But, in the heterotic fivebrane backgrounds, they do not satisfy all of the field equations but part of them.
Therefore, by modifying the formulae from supersymmetry variations, the solutions of the field equations can be obtained.

First, we try to solve the dilatino field equation (\ref{dilatino_eq}).
Notice that the equation (\ref{dilatino_eq}) contains two terms \myHighlight{$\gamma^\mu\gamma^\nu\partial_\nu\Phi\psi_\mu$}\coordHE{} and  \myHighlight{$F_{\mu\nu}\gamma^{\mu\nu}\chi$}\coordHE{}, whose derivative dimensionalities do not coincide.
Therefore, when we expand the dilatino field \myHighlight{$\lambda$}\coordHE{} as \myHighlight{$\lambda=\lambda_{(0)}+\alpha'\lambda_{(1)}+\ldots $}\coordHE{}, the equation (\ref{dilatino_eq}) decomposes into the following two equations.
\begin{eqnarray}\coord{}\boxAlignEqnarray{\leftCoord{}
\leftCoord{}-2\gamma^\mu\left(D_\mu-\partial_\mu\Phi\right)\lambda_{(0)}-\gamma^\mu\gamma^\nu\partial_\nu\Phi\psi_\mu &=& 0, \rightCoord{}\label{leading_dilatino_eq} \rightCoord{}\\\leftCoord{}
\leftCoord{}-2\gamma^\mu\left(D_\mu-\partial_\mu\Phi\right)\lambda_{(1)}+\frac{\leftCoord{}1}{\rightCoord{}4}F^I_{\mu\nu}\gamma^{\mu\nu}\chi^I &=& 0. \rightCoord{}\label{next_dilatino_eq}
\rightCoord{}}{0mm}{5}{6}{
-2\gamma^\mu\left(D_\mu-\partial_\mu\Phi\right)\lambda_{(0)}-\gamma^\mu\gamma^\nu\partial_\nu\Phi\psi_\mu &=& 0, \\
-2\gamma^\mu\left(D_\mu-\partial_\mu\Phi\right)\lambda_{(1)}+\frac{1}{4}F^I_{\mu\nu}\gamma^{\mu\nu}\chi^I &=& 0. }{1}\coordE{}\end{eqnarray}
The equation (\ref{leading_dilatino_eq}) is easily solved, since the supersymmetry variations of the gravitino and the dilatino, Eqs.(\ref{GrV}) and (\ref{DlV}), satisfy this equation.
\begin{eqnarray}\coord{}\boxAlignEqnarray{\leftCoord{} 
\psi_\rho &=& 
\left(\partial_\rho+\frac{\leftCoord{}3}{\rightCoord{}4}\partial^\sigma\Phi\gamma_{\rho\sigma}- \frac{\leftCoord{}1}{\rightCoord{}4}\partial_\rho\Phi\right)\eta, \rightCoord{}\label{before_gravitino} \rightCoord{}\\\leftCoord{}
\lambda_{(0)} &=& 
\leftCoord{}-\frac{\leftCoord{}1}{\rightCoord{}2}\partial_\mu\Phi\gamma^\mu\eta, \rightCoord{}\label{dilatino_sol}
\rightCoord{}}{0mm}{6}{8}{ 
\psi_\rho &=& 
\left(\partial_\rho+\frac{3}{4}\partial^\sigma\Phi\gamma_{\rho\sigma}- \frac{1}{4}\partial_\rho\Phi\right)\eta, \\
\lambda_{(0)} &=& 
-\frac{1}{2}\partial_\mu\Phi\gamma^\mu\eta, }{1}\coordE{}\end{eqnarray}
where \myHighlight{$\eta$}\coordHE{} is an arbitrary Weyl spinor with negative four-dimensional chirality.

In order to completely solve the dilatino field equation (\ref{dilatino_eq}), we should solve the equation (\ref{next_dilatino_eq}) and obtain \myHighlight{$\lambda_{(1)}$}\coordHE{}.
But \myHighlight{$\lambda_{(1)}$}\coordHE{} is not required here since the action which we are treating is the leading in the \myHighlight{$\alpha'$}\coordHE{} expansion of the full heterotic supergravity.
However, the equation (\ref{next_dilatino_eq}) can be solved anyway, as we see later.

Next, we try to solve the gravitino field equation (\ref{gravitino_eq}).
As in the case of dilatino, the supersymmetry variation of the gravitino, namely Eq.(\myHighlight{$\ref{before_gravitino}$}\coordHE{}), is expected to satisfy this equation, but it is not.
Therefore, the modification of the formula from gravitino variation is needed.
This modification should satisfy not only the equation (\ref{gravitino_eq}), but also the dilatino field equation (\ref{leading_dilatino_eq}).
Using the identities of the four-dimensional \myHighlight{$\gamma$}\coordHE{} matrix, we can see that the following modification is sufficient.
\begin{eqnarray}\coord{}\boxAlignEqnarray{\leftCoord{}
\eta &\to& e^{p\Phi}\eta_0,\rightCoord{}\\\leftCoord{}
\psi_\rho 
&\leftCoord{}\to& \psi_\rho - \frac{\leftCoord{}3}{\rightCoord{}4}\left(\partial^\sigma\Phi\gamma_{\rho\sigma}+3\partial_\rho\Phi\right)\eta, \nonumber \rightCoord{}\\
&&\leftCoord{}{}=\left(p-\frac{\leftCoord{}5}{\rightCoord{}2}\right)\partial_\rho\Phi\eta, \rightCoord{}\label{gravitino_sol}
\rightCoord{}}{0mm}{6}{7}{
\eta &\to& e^{p\Phi}\eta_0,\\
\psi_\rho 
&\to& \psi_\rho - \frac{3}{4}\left(\partial^\sigma\Phi\gamma_{\rho\sigma}+3\partial_\rho\Phi\right)\eta, \\
&&{}=\left(p-\frac{5}{2}\right)\partial_\rho\Phi\eta, }{1}\coordE{}\end{eqnarray}
where \myHighlight{$p$}\coordHE{} is an arbitrary real number and \myHighlight{$\eta_0$}\coordHE{} is a constant Weyl spinor.
This new \myHighlight{$\psi$}\coordHE{} satisfies both Eqs.(\ref{gravitino_eq}) and (\ref{leading_dilatino_eq}).

The field equation (\ref{gaugino_eq}) for the gaugino \myHighlight{$\chi^I$}\coordHE{} can be solved easily.
Utilizing the Bianchi identity \myHighlight{${\cal D}_{[\mu}{F^I}_{\nu\rho]}=0$}\coordHE{}, the solution of this field equation are obtained also as the supersymmetry variation Eq.(\ref{GaV}).
\begin{eqnarray}\coord{}\boxAlignEqnarray{\leftCoord{}
\chi^I &=& \left(\frac{\leftCoord{}p-5/2}{\rightCoord{}p-1}\right)\delta_{\eta}\chi^I, \nonumber \rightCoord{}\\
&\leftCoord{}=& -\left(\frac{\leftCoord{}1}{\rightCoord{}4}\right)\frac{\leftCoord{}p-5/2}{\rightCoord{}p-1}{\rightCoord{}F^I}_{\rho\sigma}\gamma^{\rho\sigma}\eta,
\rightCoord{}}{0mm}{5}{7}{
\chi^I &=& \left(\frac{p-5/2}{p-1}\right)\delta_{\eta}\chi^I, \\
&=& -\left(\frac{1}{4}\right)\frac{p-5/2}{p-1}{F^I}_{\rho\sigma}\gamma^{\rho\sigma}\eta,
}{1}\coordE{}\end{eqnarray}
where \myHighlight{$\delta_{\eta}$}\coordHE{} means the supersymmetry variation of the gaugino with parameter \myHighlight{$\eta$}\coordHE{}.

We could solve fermionic field equations (\ref{gravitino_eq}), (\ref{gaugino_eq}) and  (\ref{dilatino_eq}) at least formally.
To recapitulate, using the generalized spinor \myHighlight{$\Psi$}\coordHE{}, we have found the following solution.
\begin{eqnarray}\coord{}\boxAlignEqnarray{\leftCoord{}
\Psi_0 =
\left(\begin{array}{c} \rightCoord{}
\left(p-\frac{\leftCoord{}5}{\rightCoord{}2}\right)\partial_\rho\Phi\eta \rightCoord{}\\\leftCoord{}
\leftCoord{}-\frac{\leftCoord{}1}{\rightCoord{}2}\partial_\mu\Phi\gamma^\mu\eta \rightCoord{}\\\leftCoord{}
\leftCoord{}-\left(\frac{\leftCoord{}1}{\rightCoord{}4}\right)\left[\left(p-\frac{\leftCoord{}5}{\rightCoord{}2}\right)/\left(p-1\right)\right]F_{\rho\sigma}\gamma^{\rho\sigma}\eta\rightCoord{}
\end{array} \rightCoord{}
\right), \rightCoord{} 
\qquad \eta = e^{p\Phi}\eta_0. \rightCoord{}\label{zeromode}
\rightCoord{}}{0mm}{9}{13}{
\Psi_0 =
\left(\begin{array}{c} 
\left(p-\frac{5}{2}\right)\partial_\rho\Phi\eta \\
-\frac{1}{2}\partial_\mu\Phi\gamma^\mu\eta \\
-\left(\frac{1}{4}\right)\left[\left(p-\frac{5}{2}\right)/\left(p-1\right)\right]F_{\rho\sigma}\gamma^{\rho\sigma}\eta
\end{array} 
\right),  
\qquad \eta = e^{p\Phi}\eta_0. }{1}\coordE{}\end{eqnarray}

When we adopt this \myHighlight{$\Psi_0$}\coordHE{} as the zeromode, its nomalizability should be confirmed.
Actually, \myHighlight{$\Psi_0$}\coordHE{} is normalizable, and it can be verified as follows.
The solution \myHighlight{$\Psi_0$}\coordHE{} vanishes at the infinity, since the dilaton \myHighlight{$\Phi$}\coordHE{} is asymptotic to the constant \myHighlight{$\Phi_0$}\coordHE{}.
While the dilaton has a singularity at the throat of the semi-wormhole, \myHighlight{$\Psi_0$}\coordHE{} is normalizable if \myHighlight{$p$}\coordHE{} is equal or less than zero.
Therefore, \myHighlight{$\Psi_0$}\coordHE{} is normalizable and is a well-defined zeromode.

Up to now, the gauge of the gravitino \myHighlight{$\psi$}\coordHE{} has not been fixed.
In general, the \myHighlight{$\gamma$}\coordHE{}-tracelessness \myHighlight{$\gamma^\mu\psi_\mu=0$}\coordHE{} is used as a gauge fixing condition of the gravitino, but the gravitino solution of Eq.(\ref{gravitino_sol}) is not consistent with this gauge.
Therefore, we have to use the other gauge in which the solution of Eq.(\ref{gravitino_sol}) can survive.
As such a gauge fixing condition, \myHighlight{$D^\mu\psi_\mu=0$}\coordHE{} is suitable.
In this gauge, the gravitino solution with \myHighlight{$p=0$}\coordHE{} can survive and \myHighlight{$\Psi_0$}\coordHE{} takes especially simple form,
\begin{eqnarray}\coord{}\boxAlignEqnarray{\leftCoord{}
\Psi_0 = 
\left(\begin{array}{c} \rightCoord{}
\leftCoord{}-\partial_\rho\Phi\eta_0 \rightCoord{}\\\leftCoord{}
\leftCoord{}-\frac{\leftCoord{}1}{\rightCoord{}5}\partial_\mu\Phi\gamma^\mu\eta_0 \rightCoord{}\\\leftCoord{}
\leftCoord{}-\frac{\leftCoord{}1}{\rightCoord{}4}F_{\rho\sigma}\gamma^{\rho\sigma}\eta_0\rightCoord{}
\end{array} \rightCoord{}
\right). \rightCoord{}
\rightCoord{}}{0mm}{8}{10}{
\Psi_0 = 
\left(\begin{array}{c} 
-\partial_\rho\Phi\eta_0 \\
-\frac{1}{5}\partial_\mu\Phi\gamma^\mu\eta_0 \\
-\frac{1}{4}F_{\rho\sigma}\gamma^{\rho\sigma}\eta_0
\end{array} 
\right). 
}{1}\coordE{}\end{eqnarray}
Here, we rescaled the constant spinor \myHighlight{$\eta_0$}\coordHE{}.
In the general value of \myHighlight{$p$}\coordHE{}, the gauge fixing condition \myHighlight{$D^\mu\psi_\mu - p\partial_\mu\Phi\partial^\mu\Phi = 0$}\coordHE{} is applicable.
It is convenient to use the gauge with \myHighlight{$p=-3/2$}\coordHE{} to solve the higher-order dilatino equation (\ref{next_dilatino_eq}).
Taking into account the relation
\begin{eqnarray}\coord{}\boxAlignEqnarray{\leftCoord{}
F_{\mu\nu}\gamma^{\mu\nu}F_{\rho\sigma}\gamma^{\rho\sigma} \propto F_{\mu\nu}\tilde{F}^{\mu\nu},
\rightCoord{}}{0mm}{1}{2}{
F_{\mu\nu}\gamma^{\mu\nu}F_{\rho\sigma}\gamma^{\rho\sigma} \propto F_{\mu\nu}\tilde{F}^{\mu\nu},
}{1}\coordE{}\end{eqnarray}
and the algebraic property of the 't Hooft's \myHighlight{$\overline{\eta}$}\coordHE{} symbol, we can obtain \myHighlight{$\lambda_{(1)}$}\coordHE{} as
\begin{eqnarray}\coord{}\boxAlignEqnarray{\leftCoord{}
\lambda_{(1)} \propto {\epsilon}^{\mu\nu\rho\sigma}\omega_{3\:\mu\nu\rho}^{YM}(A){\gamma}_{\sigma}\eta,
\rightCoord{}}{0mm}{1}{2}{
\lambda_{(1)} \propto {\epsilon}^{\mu\nu\rho\sigma}\omega_{3\:\mu\nu\rho}^{YM}(A){\gamma}_{\sigma}\eta,
}{1}\coordE{}\end{eqnarray}
where \myHighlight{$\omega_{3\:\mu\nu\rho}^{YM}(A)$}\coordHE{} is the Chern-Simons three-form.

Looking back the way we have passed, the derivation of the zeromode \myHighlight{$\Psi_0$}\coordHE{} is based on supersymmetry variations.
In fact, only the gravitino component of the supersymmetry transformation is modified.
Therefore, the natural question arises here, that is, whether zeromode \myHighlight{$\Psi_0$}\coordHE{} can be gauged away or not.
This is an important question since we are interested in the physical zeromodes which can contribute to the gravitino condensation.
This possibility whether \myHighlight{$\Psi_0$}\coordHE{} can be gauged away or not can be seen as follows.
The well-defined gauge transformation should be generated by the well-defined gauge parameter.
When \myHighlight{$\Psi_0$}\coordHE{} is regarded as a supersymmetry variation, \myHighlight{$\eta$}\coordHE{} corresponds to a gauge parameter, and is not normalizable since it is asymptotic to a constant \myHighlight{$e^{p\Phi_0}\eta_0$}\coordHE{} at the infinity.
Therefore, \myHighlight{$\Psi_0$}\coordHE{} cannot be regarded as any gauge variations.
Even if our zeromode is derived from the supersymmetry variations essentially, it will never be gauged away.
We can expect that \myHighlight{$\Psi_0$}\coordHE{} is the mode which can contribute to the gravitino condensation in the background of the heterotic fivebrane.


\section{Conclusion}
\label{sec:five}
We made an attempt to find the fermionic zeromodes in the heterotic fivebrane backgrounds by solving the fermionic field equations explicitly.
As a result, we obtained a zeromode \myHighlight{$\Psi_0$}\coordHE{}, and its nomalizability was confirmed.
The relationship between \myHighlight{$\Psi_0$}\coordHE{} and gauge transformation (local supersymmetry transformation) was discussed, since \myHighlight{$\Psi_0$}\coordHE{} looks like just a supersymmetry variation.
The zeromode is not gauged away, and it is possible to contribute to the gravitino condensation.

The existence of both gravitino and gaugino components in the zeromode \myHighlight{$\Psi_0$}\coordHE{} could be interpreted as follows.
The authors of Ref.\cite{GSW} pointed out that some four-fermion terms, which appear in the next order of the \myHighlight{$\alpha'$}\coordHE{} expansion in the action of the heterotic supergravity, can give a perfect square
\begin{eqnarray}\coord{}\boxAlignEqnarray{\leftCoord{}
\leftCoord{}(H_{MNP} - \frac{\leftCoord{}g^2\phi}{\rightCoord{}12}tr(\overline{\chi}{\rightCoord{}\Gamma}_{MNP}\chi))^2, \rightCoord{}\label{psq}
\rightCoord{}}{0mm}{3}{5}{
(H_{MNP} - \frac{g^2\phi}{12}tr(\overline{\chi}{\Gamma}_{MNP}\chi))^2, }{1}\coordE{}\end{eqnarray}
where \myHighlight{$\Gamma$}\coordHE{} is the ten-dimensional gamma matrix, and gaugino \myHighlight{$\chi$}\coordHE{} belongs to the \myHighlight{$\bf 16$}\coordHE{} dimensional Majorana-Weyl spinor representation.
If we demand that the cosmological constant vanishes, the gaugino bilinear \myHighlight{$\overline{\chi}{\Gamma}^{MNP}\chi$}\coordHE{} in Eq.(\ref{psq}) can have the vacuum expectation value dynamically since the field strength \myHighlight{$H_{MNP}$}\coordHE{} never vanishes in the heterotic fivebrane backgrounds (notice that this soliton is the exact solution of the heterotic supergravity).
On the other hand, the supersymmetry transformations of the gravitino and dilatino receive the following next order corrections in \myHighlight{$\alpha'$}\coordHE{}\cite{BdR}.
\begin{eqnarray}\coord{}\boxAlignEqnarray{\leftCoord{}
\delta_{SUSY}{\psi}_M \sim 
\frac{\leftCoord{}1}{\rightCoord{}192}\Gamma^{NPS}\Gamma_M\epsilon\left[tr(\overline{\chi}\Gamma_{NPS}\chi)+\overline{\psi}^{ab}\Gamma_{NPS}\psi_{ab}\right], \rightCoord{}\label{NGr}\rightCoord{}\\\leftCoord{}
\delta_{SUSY}\lambda \sim 
\frac{\leftCoord{}1}{\rightCoord{}384}\sqrt{2}\Gamma^{NPS}\epsilon\left[tr(\overline{\chi}\Gamma_{NPS}\chi)+\overline{\psi}^{ab}\Gamma_{NPS}\psi_{ab}\right]. \rightCoord{}\label{NDl}
\rightCoord{}}{0mm}{4}{7}{
\delta_{SUSY}{\psi}_M \sim 
\frac{1}{192}\Gamma^{NPS}\Gamma_M\epsilon\left[tr(\overline{\chi}\Gamma_{NPS}\chi)+\overline{\psi}^{ab}\Gamma_{NPS}\psi_{ab}\right], \\
\delta_{SUSY}\lambda \sim 
\frac{1}{384}\sqrt{2}\Gamma^{NPS}\epsilon\left[tr(\overline{\chi}\Gamma_{NPS}\chi)+\overline{\psi}^{ab}\Gamma_{NPS}\psi_{ab}\right]. }{1}\coordE{}\end{eqnarray}
From this transformation rule, it is clear that the gaugino condensation \myHighlight{$\langle\overline{\chi}{\Gamma}^{MNP}\chi\rangle\neq0$}\coordHE{} breaks all supersymmetry.
But the heterotic fivebrane should be invariant under the half of the supersymmetry which is induced by \myHighlight{$\epsilon_+\in(4,2_+)$}\coordHE{}.
Therefore, the gravitino should condensate so that the gaugino condensation are cancelled out in Eqs.(\ref{NGr}) and (\ref{NDl}).
The fact that our zeromode \myHighlight{$\Psi_0$}\coordHE{} has both gravitino and gaugino components could suggest such a phenomenon.


\section{acknowledgments}
The authors would like to thank S. Saito for fruitful discussions and comments.


\begin{thebibliography}{99}
\bibitem{treelevel_SUSY_breaking_1}
H. P. Nilles, Phys. Rep. 110, 1 (1984).

\bibitem{treelevel_SUSY_breaking_2}
H. E. Harber and G. L. Kane, Phys. Rep. 117, 75 (1985).

\bibitem{Witten}
E. Witten, Nucl. Phys. B188, 513 (1981).

\bibitem{EH}
T.Eguchi, P. B. Gilkey and A. J. Hanson, Phys. Rep. 66, 213 (1980).

\bibitem{Konishi}
K. Konishi, N. Magnoli and H. Panagopoulos, Nucl. Phys. B309, 201 (1988) ; B323, 441 (1989).

\bibitem{CHS}
C. G. Callan, J. A. Harvey and A. Strominger, Nucl. Phys. B359, 611 (1991) ; B367, 60 (1991).

\bibitem{Rey}
Soo-Jong Rey, Phys. Rev. D43, 526 (1991).

\bibitem{Bellisai}
D. Bellisai, Nucl. Phys. B467, 127 (1996).

\bibitem{BdR}
E. A. Bergshoeff and M. de Roo, Nucl. Phys. B328, 439 (1989).

\bibitem{GS}
M. Green and J. Schwarz, Phys. Lett. B151, 21 (1985).





\bibitem{Calabi-Yau_1}
A. Strominger, Nucl. Phys. B274, 253 (1986).

\bibitem{Calabi-Yau_2}
P. Candelas, G. T. Horowitz, A. Strominger and E. Witten, Nucl. Phys. B258, 46 (1985).

\bibitem{Charge_quantization}
E. Witten, Comm. Math. Phys. 92, 455 (1984).



\bibitem{Dilaton_sol_1}
R. R. Khuri, Nucl. Phys. B387, 315 (1992) ; Phys. Lett. B294, 325 (1992).

\bibitem{Dilaton_sol_2}
J. Gauntlett, J. Harvey and J. T. Liu, Nucl. Phys. B409, 363 (1993).

\bibitem{Dilaton_sol_3}
M. J. Duff and R. R. Khuri, Nucl. Phys. B411, 473 (1994).

\bibitem{APS}
M. F. Atiyah, V. K. Patodi and I. M. Singer, Proc. Camb. Phil. Soc. 77, 43 (1975) ; 78, 405  (1975) ; 79, 71 (1976).



\bibitem{Gibbons-Hawking}
G. W. Gibbons and S. W. Hawking, Phys. Rev. D15, 2752 (1977).

\bibitem{GSW}
M. M. Green, J. H. Schwarz and E. Witten, "Superstring theory," Vol. 2, Cambridge Univ. Press.



\end{thebibliography}

\end{document}
\bye
