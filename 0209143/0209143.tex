

\documentclass[a4paper,a4paper]{article}
\setlength{\oddsidemargin}{-0.1 in}
\addtolength{\topmargin}{-1 cm}
\newtheorem{theorem}{Theorem}
\usepackage{graphicx}
%--------------------------------------------------------------------------
%--------
\DeclareSymbolFont{AMSa}{U}{msa}{m}{n}
\DeclareMathDelimiter\ulcorner{\mathopen} {AMSa}{"70}{AMSa}{"70}
\DeclareMathDelimiter\urcorner{\mathclose}{AMSa}{"71}{AMSa}{"71}
\makeatletter
\def\uufill{$\m@th\mathopen\ulcorner\mkern-7mu%
%  \cleaders\hbox{$\mkern-2mu\smash-\mkern-2mu$}\hfill
%  \mkern-7mu\smash-\mathord\urcorner$}
%%% Tammy add:  adjust the height 6pt if needed
  \cleaders\hbox{\rule[6pt]{1dd}{1dd}}\hfill
  \mkern-7mu\mathclose\urcorner$}
\def\overbrack#1{\vbox{\m@th\ialign{##\crcr
      \uufill\crcr\noalign{\kern-\p@\nointerlineskip}%
      $\hfil\displaystyle{#1}\hfil$\crcr}}}
\makeatother
%--------

\textwidth 16 cm
\textheight 8.8 in

\title{Photon as the Magnetic Monopole}

\author{Sze Kui Ng
\\  Department of Mathematics,
Hong Kong Baptist University, Hong Kong
\\E-mail: skng@hkbu.edu.hk
}
\begin{document}
\date{}
\maketitle
\begin{abstract}

In this paper we reinvestigate photon. We show that photon can be identified with the Dirac magnetic monopole. We give a model of quantum electrodynamics from which we derive photon as a quantum loop of this model. This nonlinear loop model of photon is exactly solvable and thus  may be regarded as a quantum soliton. From the winding numbers of this loop model of photon we derive the quantization property of energy of Planck's formula of radiation and the quantization property of electric charge. We show that these two quantizations are just the same quantization when photon is identified with the magnetic monopole. From this nonlinear model of photon we also construct a model of electron which has a mass mechanism for generating mass to electron.

{\bf PACS codes: } 11.25.Hf, 02.10.Kn,  11.15.-q.

{\bf Keywords:} Nonlinear loop model of photon, magnetic monopole.

\end{abstract}

\section{Introduction}\label{sec00}

                                        
%In this paper we reinvestigate the concept of photon.  We shall show that  photon  can be identified as the Dirac magnetic monopole. We shall establish a mathematical model of quantum electrodynamics from which we derive photon as a quantum soliton of this model and we shall show that this quantum soliton has the properties of  the magnetic monopole.

It is well known that the quantum era of physics began with the quantization of energy of electromagnetic field from which Planck derived the radiation formula. Einstein then introduced the light-quantum to explain the photoelectric effects. This light-quantum was later regarded as a particle called  photon\cite{Pai}\cite{Pla}\cite{Ein}. Later quantum mechanics was developed and we came into the modern quantum era. Following the set up of quantum mechanics the quantization of the electromagnetic field and the theory of quantum electrodynamics(QED) was then also set up.

In this development of quantum theory of physics  the photon plays a special role. While it is as the beginning of quantum physics it is not as easy to be understood as the quantum mechanics of other particles described by the Schroedinger equation. Indeed Einstein was carefully to regard the light-quantum as a particle and it was much later that the light-quantum was finally accepted as a particle called photon \cite{Pai}. Then the quantum field theory of electromagnetic field was developed for the photon. However it is well known that this quantum field theory has difficulties such as the ultraviolet divergences. It is because of the difficulty of understanding the photon that Einstein once said ''What is the photon? ''\cite{Pai}. 
%Thus though we have a successful theory of quantum electrodynamic (QED) we still cannot say that it is a complete theory of the photon.

On the other hand based on the symmetry of the electric and magnetic field described by the Maxwell equation and based on the complex wave function of quantum mechanics  Dirac derived the concept of magnetic monopole which  is hypothetically considered as a particle with magnetic charge as analogous to the electron with electric charge \cite{Dir}\cite{Dir2}. An important feature of this magnetic monopole is that it gives the quantization of electric charge. Thus it is interesting and important to find such particles. However despite of many searches it seems that no such particles have been found. 

In this paper we shall establish a mathematical model of photon to show that magnetic monopole can be identified with the photon. Before giving the detailed model let us consider some thoughts for the identification of these two particles, as follows.

First we have that photon is regarded as the basic quantum particle of the electromagnetic field while the magnetic monopole is a hypothetical particle  derived from the electromagnetic field. Thus if the magnetic monopole is not the photon then we have two kind of quantum particles derived from the electromagnetic field and that the magnetic monopole is derived from the photon because photon is the basic quantum particle of the electromagnetic field.
 %that it can completely describe the quantum effects of the electromagnetic field. 
From this we then have the odd conclusion that an elementary quantum particle (i.e. the magnetic monopole) is derived from another elementary quantum particle (i.e. the photon). If we identify these two particles then this odd conclusion can be resolved.

For the identification of these two particles let us reinvestigate the quantum field theory of photon in the literature \cite{Zub}. It is well known that we have the quantum field theory of the free Maxwell equation which is as the basic quantum theory of photon. 
%Not to mention some of the difficulties of this theory 
Let us consider some points of this theory, as follows. 
First we have that this  free field theory is a linear theory and the model of the quantum particles obtained from this theory is linear. However from the thought of soliton we have that stable particle should be a soliton which is of nonlinear nature. 
%From this point of view we have that the free quantum field theory of the Maxwell equation would not be the complete theory of photon. 
Secondly we have that  the quantum particles of  this  quantum theory of Maxwell equation are collective quantum effects just like the phonons which are elementary exictation of a statistical model. These phonons are usually considered as quasi-particles and are not regarded as real particles. In regarding the Maxwell equation as a statistical wave equation of electromagnetic field  we have that the quantum particles of the quantum  theory of Maxwell equation are analogous to the phonons. Thus they should be regarded as quasi-photons which have properties of photons but are not a complete description of photons. 

%From these thoughts we thus try to develop a nonlinear model of the photon.  

%From this point of view we should develop a nonlinear model of photon such that the photon is as an exactly solvable solution of this nonlinear model and then the photon is regarded as a soliton.

%Now for the identification of the photon and the magnetic monopole we have the following thought. Though the quantum particle of  the  quantum Maxwell  equation is not look like  the magnetic monopole we have that the real photon could be with the properties of the magnetic monopole. 
%In the following let us give a mathematical model of the photon and we identify it with the magnetic monopole.

In this paper we shall set up a nonlinear model of photon. In this model we show that photon can be identified with the Dirac magnetic monopole. We give a model of quantum electrodynamics from which we derive photon as a quantum loop of this model. This nonlinear loop model of photon is exactly solvable and thus  may be regarded as a quantum soliton. From the winding numbers of this loop model of photon we derive the quantization property of energy of Planck's formula of radiation and the quantization property of charge. We show that these two quantizations are just the same quantization when photon is identified with the magnetic monopole. From this nonlinear model of photon we also construct a model of electron which has a mass mechanism for generating mass to electron.

This paper is organized as follows. In section 2
we give a brief description of a quantum gauge model
of electrodynamics. With this  model in section 3 we
introduce the Dirc-Wilson loop.  We show that this loop is a nonlinear 
exactly solvable model and thus can be regarded as a soliton. We identify this Dirc-Wilson loop as the photon when the gauge group is the $U(1)$ group. To investigate the properties of 
this Dirc-Wilson loop in section 4 we derive a chiral symmetry from
the gauge symmetry of this quantum  model.
From this chiral symmetry in section 5 we derive a conformal field theory which includes the affine Kac-Moody algebra
and the Knizhnik-Zamolodchikov (KZ) equation. A main point of our model on the KZ equation is that we can derive two KZ equations
which are inversely dual to each other. This duality is the main point for the Dirc-Wilson loop to be exactly solvable and to have a winding property which gives properties of photon. 
In section 6 from the KZ equations  we solve the  Dirac-Wilson loop in a form with a winding property.
%the properties of the photon and gives the properties of magnetic monopole.
 From this winding property of the Dirc-Wilson loop in section 7 we derive the quantization of energy and the quantization of electric charge which are properties of photon and magnetic monopole. We then show that these two quantizations are just the same quantization and we identify photon with the magnetic monopole.  From this nonlinear model of photon in section 8 we also give a model of electron. In this model of electron we give a mass mechanism for generating mass to electron. 

\section{A Quantum Gauge Model}\label{sec2}

Let us construct a quantum gauge model for photon which is a mathematical model similar to the Wiener measure of the Brownian motion, as follows.
In probability theory we have the Wiener measure $\nu$ which is
a measure on the space $C[t_0,t_1]$ of continuous functions \cite{Jaf}.
This probability measure (which may also be regarded as a path integral) is a well defined mathematical theory for the Brownian motion and it may be symbolically written in the following form:
\begin{equation}
d\nu =e^{-L_0}dx
\label{wiener}
\end{equation}
where $L_0 := \frac12\int_{t_0}^{t_1}\left(\frac{dx}{dt}\right)^2dt$
is the energy integral of the Brownian particle and
$dx = \frac{1}{N}\prod_{t}dx(t)$ is symbolically a product of Lebesgue
measures $dx(t)$ and $N$ is a normalized constant.
%Once the Wiener measure is  defined we can then define other measures on $C[t_0,t_1]$ which is absolutely continuous with respect to the Wiener measure as follows \cite{Jaf}. Let a potential term $\frac12\int_{t_0}^{t_1}Vdt$ be added to $L_0$. Then we have a measure $\nu_1$ on $C[t_0,t_1]$ defined by:
%\begin{equation}
%d\nu_1 =e^{-\frac12\int_{t_0}^{t_1}Vdt}d \nu
%\label{wiener2}
%\end{equation}
%Under some condition on $V$ we have that $\nu_1$ is well defined on $C[t_0,t_1]$ and is absolutely continuous with respect to the Wiener measure. Let us call this method of contructing measures as the method of Feymann-Kac formula \cite{Jaf}.

Let us then follow this method to construct a measure for a quantum  model of electrodynamics, as follows. Similar to the Wiener measure we construct a measure for a quantum  model of electrodynamics from the 
following energy integral:
\begin{equation}
 \frac12\int_{s_0}^{s_1}[
\frac12\left(\frac{dA_1}{ds}-\frac{dA_2}{ds}\right)^*
\left(\frac{dA_1}{ds}-\frac{dA_2}{ds}\right)
+\sum_{i=1}^2
\left(\frac{dZ}{ds}-ieA_iZ\right)^*\left(\frac{dZ}{ds}-ieA_iZ\right)]ds
\label{1.1}
\end{equation}
where $s$ denotes the proper time in relativity and $e$
denotes the electric charge. The complex variable $Z$ will 
represent an electron and the complex variables
$A_1$ and $A_2$ represent electromagnetic field from which we shall construct  photon.
Here we allow $A_1$ and $A_2$ to be complex valued. 


The integral (\ref{1.1}) is invariant under the following gauge transformation:
\begin{equation}
\begin{array}{rl}
Z^{\prime}(s) & := Z(s)e^{iea(s)} \\
Z^{\prime *}(s)  & := Z(s)^{*}e^{-iea(s)}\\
A_i^{'}(s) & := A_i(s)+\frac{da}{ds}\\
A_i^{'*}(s) & := A^{*}_i(s)+\frac{da}{ds}
\quad i=1,2
\end{array}
\label{1.2}
\end{equation}
where $a(s)$ is a complex valued function. In this gauge transformation the pair variables $Z^{*}$ and $Z$ are regarded as independent variables. After the gauge transformation they are set to be complex conjugate to each other.
The other pair variables $Z^{\prime *}$ and $Z^{\prime}$, $A_i^{*}$ and $A_i$, $A_i^{'*}$ and $ A_i^{'}$ are similarly treated.

We remark that a  feature of (\ref{1.1}) is that it is not formulated with the four-dimensional space-time but is formulated with the one dimensional proper time. This one dimensional nature let this measure avoid the usual utraviolet divergence difficulty of quantum fields. 
%We refer to \cite{Ng} for the physical motivation of this model.

We can generalize this gauge model of photon with $U(1)$ gauge symmetry
to nonabelian gauge models.
As an illustration let us consider $SU(2)\otimes U(1)$ gauge symmetry where 
$SU(2)\otimes U(1)$ denotes the  direct product of the groups $SU(2)$ and $U(1)$ 
%(We have $SU(2)\otimes U(1)= U(1)\otimes SU(2)$).
Similar to 
(\ref{1.1}) we consider the following energy integral:
\begin{equation}
L := \frac12\int_{s_0}^{s_1}
[\frac12 Tr (D_1A_2-D_2A_1)^{*}(D_1A_2-D_2A_1) +
(D_1Z)^{*}(D_1Z)+(D_2Z)^{*}(D_2Z)]ds
\label{n1}
\end{equation}
where 
$Z= (z_1, z_2)^{T}$ is a two dimensional complex vector;
$A_j =\sum_{k=0}^{3}A_j^k t^k $ $(j=1,2)$ where
$A_j^k$ denotes a component of a gauge field $A^k$;
$t^k=ieT^k$ denotes a generator of $SU(2)\otimes U(1)$ where $T^k$ denotes
a self-adjoint generator of $SU(2)\otimes U(1)$ and $e$ denotes the charge of interaction; and
$D_j=\frac{1}{r(s)}\frac{d}{ds}-A_j$, 
 $D_j^*=\frac{1}{r(s)^{*}}\frac{d}{ds}-A_j^*$, $(j=1,2)$
where $r(s)\neq 0$ is a continuous function on $[s_0,s_1]$.

From (\ref{n1}) we can develop a nonabelian gauge model as similar
to that for the above abelian gauge model.
We have that (\ref{n1}) is invariant under the following
gauge transformation:
\begin{equation}
\begin{array}{rl}
Z^{\prime}(s) & :=U(a(s))Z(s) \\
Z^{\prime *}(s) & :=Z(s)^{*}U^{-1}(a(s)) \\
A_j^{\prime}(s) &
:=
\frac{1}{r(s)}U(a(s))A_j(s)U^{-1}(a(s))+
 U(a(s))\frac{dU^{-1}}{ds}(a(s)),\\
A_j^{\prime *}(s) &
:=
\frac{1}{r(s)^{*}}U(a(s))A_j^{*}(s)U^{-1}(a(s))+
 U(a(s))\frac{dU^{-1}}{ds}(a(s)),
j =1,2
\end{array}
\label{n2}
\end{equation}
where $U(a(s))=e^{a(s)}$ and $a(s)=\sum_k a^k (s)t^k$.
We shall mainly consider the case that $a$ is a function
of the form $a(s)=\omega(z(s))$ where $\omega$ 
and $z$ are analytic functions.

We remark that since the above model is a gauge model with
a gauge invariance it will be degenerate and we need a  gauge fixing to let this model be nondegenerate \cite{Fad}. A gauge model may have various gauge fixing conditions. As an example we have that the  Maxwell equation is a
gauge model for electrodynamics. It has various gauge fixing conditions such as the Lorentz gauge condition, the Feynman gauge condition, etc.
We shall later adopt a gauge fixing condition for the above gauge model.

\section{Dirac-Wilson Loop } \label{sec4}

Similar to the Wilson loop in quantum field theory from our quantum model we can also introduce an analogue of Wilson loop, as follows.

{\bf Definition}.
 A Dirac-Wilson loop (or Wilson loop) $W(C)$ is defined by :
\begin{equation}
W(C):= W(z_0, z_1):= Pe^{\int_C A_jdx^j}
\label{n4}
\end{equation}
where $C$ denotes a continuous closed curve 
which is of the following form:
\begin{equation}
C(s) =(x^1(z(s)), x^2(z(s))), \qquad s^{'}\leq s \leq s^{''}
\label{n4a}
\end{equation}
where $s_0\leq s^{'} \leq s^{''} \leq s_1$ and
$z(\cdot)$ is a continuously differentiable curve in the complex plane such that 
$z_0 :=z(s^{'})=z(s^{''}) =: z_1$. This closed curve
$C$ is
in a two dimensional plane $(x^1, x^2)$ with complex coordinates $x^1$, $x^2$ which is 
dual to $(A_1, A_2)$.
As usual the notation
$P$ in the definition of $W(C)$ denotes a path-ordered product \cite{Wit}\cite{Kau}\cite{Baez}.

For the curve $C(s)=(x^1(z(s)),x^2(z(s)))$ to be nontrivial we suppose that 
\begin{equation}
\frac{dx^1(z(s))}{ds}+\frac{dx^2(z(s))}{ds}\neq 0,
\quad \quad s^{\prime}\leq s \leq s^{\prime\prime}
\label{curve}
\end{equation}
Then we define a continuous function $r\neq 0$ on 
$[s_0, s_1]$ by
\begin{equation}
 r(s) := \left\{\begin{array}{cc}
\frac{dx^1(z(s))}{ds}+\frac{dx^2(z(s))}{ds} &
 \mbox{for $s\in [s^{\prime}, s^{\prime\prime}]$} \\
\frac{dx^1(z(s^{\prime}))}{ds}+
\frac{dx^2(z(s^{\prime}))}{ds}  &
\mbox{for
$s\in [s_0, s^{\prime}]\cup [s^{\prime\prime}, s_1]$}
\end{array}
\right.
\label{curve2}
\end{equation}

Let us give some remarks on the above definition
of Dirac-Wilson loop, as follows.

1) We use the notation $W(z_0, z_1)$ to mean that
this Dirac-Wilson loop $W(C)$ is based on the closed curve $z(\cdot)$
in the complex plane which starts at $z_0$ and ends
at $z_1$ with $z_0=z_1$. Thus this notation $W(z_0, z_1)$ denotes the Dirac-Wilson loop $W(C)$ constructed from the whole curve $z(\cdot)$. Here for convenience we only use the end points $z_0$ and
$z_1$ of the curve $z(\cdot)$ to denote this Dirac-Wilson loop.

Then we  extend the definition of $W(C)$ to the case that $C$ is not a closed curve with $z_0\neq z_1$. When $C$ is not a closed loop we shall
called $W(z_0, z_1)$ as a Wilson line.

2) We use the above Dirac-Wilson loop $W(C)$ to represent the unknot (Also called the trivial knot). When the gauge group is the $U(1)$ group we shall use it to model the photon.

3) In constructing the Dirac-Wilson loop we need to choose a representation of the $SU(2)$ group. Let us here choose the tensor product of the usual two dimensional representation of the $SU(2)$ for constructing the Dirac-Wilson loop. 



\section{A Chiral Symmetry} \label{sec5}

For a given curve $C(s)=(x^1(z(s)), x^2(z(s))), s_2 \leq s\leq s_3$ which may not be a closed curve
we define $W(z_0, z_1)$ by (\ref{n4}) 
where $z_0$ may not equal to $z_1$.
By following the usual approach of deriving a chiral symmetry from a gauge transformation of a gauge field
we have the following chiral symmetry
which is derived by applying an analytic gauge transformation with an analytic function $\omega$
for the transformation:
\begin{equation}
W(z_0, z_1) \mapsto U(\omega(z_1))
W(z_0, z_1)U^{-1}(\omega(z_0))
\label{n5}
\end{equation}
This chiral symmetry is analogous to the chiral symmetry
of the usual guage theory where $U$ denotes an element of the gauge group \cite{Kau}.
%We have the following theorem.
%\begin{theorem}
%The chiral symmetry (\ref{n5}) holds.
%\end{theorem}
%{\bf Proof}.
Let us derive (\ref{n5}) as follows.
Let $U(s):= U(\omega(z(s)))$.
Following Kauffman \cite{Kau} we have
\begin{equation}
\begin{array}{rl}
& U(s+ ds)(1+ dx^{\mu}A_{\mu})U^{-1}(s)\\
=& U(s+ ds)U^{-1}(s)
+ dx^{\mu}U(s+\triangle s)A_{\mu}U^{-1}(s) \\
= & 1+ \frac{dU}{ds}(s)U^{-1}(s)ds
  + dx^{\mu}U(s+ ds)A_{\mu}U^{-1}(s) \\
\approx & 1+ \frac{dU}{ds}(s)U^{-1}(s)ds
+ dx^{\mu}U(s)A_{\mu}U^{-1}(s) \\
=: & 1+ \frac{1}{r(s)}\frac{dU}{ds}(s)U^{-1}(s)
[(\frac{dx^1}{ds}+\frac{dx^2}{ds})ds]
 + U(s)[\frac{dx^1}{ds}A_1+\frac{dx^2}{ds}A_2]U^{-1}(s)ds\\

=: & 1 + \sum_{\mu=1}^2\frac{dx^{\mu}}{ds}
    [\frac{1}{r(s)}\frac{dU}{ds}(s)U^{-1}(s)+
     U(s)A_{\mu}U^{-1}(s)]ds
    \\
=:& 1 + dx^{\mu}A_{\mu}^{\prime}
\end{array}
\label{n5b}
\end{equation}
where 
\begin{equation}
A_{\mu}^{\prime}
:=\frac{1}{r(s)}\frac{dU}{ds}(s)U^{-1}(s)+
  U(s)A_{\mu}U^{-1}(s)
\label{n5c}
\end{equation}
From (\ref{n5b}) we have that (\ref{n5}) holds.
%since (\ref{n5}) is the limit of ordered product in which the left-side factor $U(s_i+ ds_i)$ in (\ref{n5b}) with $s_i=s$ is canceled with the right-side factor $U^{-1}(s_{i+1})$ of (\ref{n5b}) where $ s_{i+1}=s_i+ ds_i$  with $s_{i+1}=s$. This gives a derivation of (\ref{n5}).
%This proves the theorem.


As analogous to the chiral symmetry of the WZW model in
conformal field theory \cite{Fra}
from the above chiral symmetry  we have the following formulas for the
variations $\delta_{\omega}W$ and $\delta_{\omega^{\prime}}W$ with
respect to the chiral symmetry:
\begin{equation}
\delta_{\omega}W(z,z')=W(z,z')\omega(z)
\label{k1}
\end{equation}
and
\begin{equation}
\delta_{\omega^{\prime}}W(z,z')=-\omega^{\prime}(z')W(z,z')
\label{k2}
\end{equation}
where $z$ and $z'$ are independent variables and
$\omega^{\prime}(z')=\omega(z)$ when
$z'=z$. In (\ref{k1}) the variation is with respect to the
$z$ variable while in (\ref{k2}) the variation is with
respect to the $z'$ variable. This two-side-variations 
when $z\neq z'$ can be derived as follows.
For the left variation we may let $\omega$ be analytic
in a neighborhood of $z$ and continuous differentiably extended to a neighborhood of $z'$
such that $\omega(z')=0$ in this neighborhood of $z'$. Then from (\ref{n5}) we have that
(\ref{k1}) holds. Similarly we may let
$\omega^{\prime}$ be analytic in a neighborhood of $z'$
and continuous differentiably extended to a neighborhood of $z$ such that $\omega^{\prime}(z)=0$ in this neighborhood of $z$. Then we have that
(\ref{k2}) holds.


\section{Affine Kac-Moody Algebra} \label{sec6}

Now let us derive a quantum loop
algebra (or the affine Kac-Moody algebra)
structure from the Wilson line $W(z,z')$. To this end
let us first consider the classical case that the Wilson line $W(z,z')$ is formed by classical gauge field. 
Since $W(z,z')$ is constructed from $ SU(2)\otimes U(1)$ we have that the mapping $z \to W(z,z')$ (We consider $W(z,z')$ as a function of $z$ with $z'$ being fixed) has a loop group structure \cite{Lus}\cite{Seg}.
For a loop group we have the following generators:
\begin{equation}
J_n^a = t^a z^n \qquad n=0, \pm 1, \pm 2, ...
\label{km1}
\end{equation}
These generators satisfy the following algebra:
\begin{equation}
[J_m^a, J_n^b] =
if_{abc}J_{m+n}^c 
\label{km2}
\end{equation}
This is  the so called loop algebra \cite{Lus}\cite{Seg}. 
Let us then introduce the following generating
function $J$:
\begin{equation}
J(w) = \sum_a J^a(w)=\sum_a j^a(w) t^a
\label{km3}
\end{equation}
where we define
\begin{equation}
J^a(w)= j^a(w) t^a :=
\sum_{n=-\infty}^{\infty}J_n^a(z) (w-z)^{-n-1}  
\label{km3a}
\end{equation}

From $J$ we have 
\begin{equation}
J_n^a=  \frac{1}{2\pi i}\oint_z dw (w-z)^{n}J^a(w)  
\label{km4}
\end{equation}
where $\oint_z$ denotes a closed contour integral  with center $z$. This formula can be interpreted as that
$J$ is the generator of the loop group and that
$J_n^a$ is the directional generator in the direction
$\omega^a(w)= (w-z)^n$. We may generalize $(\ref{km4})$
to the following  directional generator: 
\begin{equation}
  \frac{1}{2\pi i}\oint_z dw \omega(w)J(w)  
\label{km5}
\end{equation}
where the analytic function 
$\omega(w)=\sum_a \omega^a(w)t^a$ is regarded
as a direction and we define
\begin{equation}
 \omega(w)J(w):= \sum_a \omega^a(w)J^a 
\label{km5a}
\end{equation}

Then since $W(z,z')\in SU(2)\otimes U(1)$,
from the variational formula (\ref{km5}) for the
loop algebra of the loop group of $SU(2)\otimes U(1)$ we have that the variation
of $W(z,z')$ in the direction $\omega(w)$
is given by
\begin{equation}
W(z,z')
  \frac{1}{2\pi i}\oint_z dw \omega(w)J(w)  
\label{km6}
\end{equation}
 

Now let us consider the quantum case which is based
on the quantum gauge model in section 2. For this quantum case
we shall choose a quantum generator $J$ which is
analogous to the $J$ in (\ref{km3}).
Let us consider the following correlation
which is a functional integration:
\begin{equation}
\langle W(z,z')A(z) \rangle :=  
\int dA_1^{*}dA_1dA_2^{*}dA_2dZ^{*}dZ  e^{-L} W(z,z')A(z)
\label{n8a}
\end{equation}
where $A(z)$ denotes a field from the quantum gauge
model (We first let $z'$ be fixed as a parameter). 

Let us derive a Ward identity by applying a gauge transformation on (\ref{n8a}) as follows.
Let $(A_1,A_2,Z)$ be regarded as a coordinate system of the integral
(\ref{n8a}). 
Under a gauge transformation (regarded as
a change of coordinate) this coordinate
is changed to another coordinate denoted by
$(A_1^{\prime}, A_2^{\prime}, Z^{\prime})$ 
%As similar to the usual change of variable for integration we have that the integral  (\ref{n8a}) is unchanged under a change of variable 
and we have the following equality:
\begin{equation}
\begin{array}{rl}
& \int dA_1^{\prime *}dA_1^{\prime}
 dA_2^{\prime *}dA_2^{\prime}dZ^{\prime *}dZ^{\prime}
 e^{-L^{\prime}} W^{\prime}(z,z')A^{\prime}(z) \\
= &
\int dA_1^{*}dA_1dA_2^{*}dA_2dZ^{*}dZ  e^{-L} W(z,z')A(z)
\end{array}
\label{int}
\end{equation}
where $W^{\prime}(z,z')$ denotes the Wilson line
based on $A_1^{\prime}$ and $A_2^{\prime}$
and similarly $A^{\prime}(z)$ denotes  the
field obtained from $A(z)$ with $(A_1, A_2,Z)$ replaced
by $(A_1^{\prime}, A_2^{\prime},Z^{\prime})$.
 
Then by the gauge invariance property the differential
 \begin{equation}
e^{-L}dA_1^{*}dA_1dA_2^{*}dA_2dZ^{*}dZ
\label{w1a}
\end{equation}
is unchanged under a gauge transformation \cite{Fad}.
Thus  we have 
\begin{equation}
0 = \langle W^{\prime}(z,z')A^{\prime}(z)\rangle -
  \langle W(z,z')A(z)\rangle
\label{w1}
\end{equation}
where the correlation notation 
$\langle \rangle$ denotes the integral with
respect to the differential (\ref{w1a}).


We can now carry out a calculus of variation for the Ward identity.
From the gauge transformation we have the formula 
$W^{\prime}(z,z')=U(\omega(z)W(z,z')U^{-1}(\omega(z'))$
This gauge transformation gives a variation of 
$W(z,z')$ with
the function $\omega$
as the variational direction $\omega$ in the
variational formulas (\ref{km5}) and  (\ref{km6}).
Thus analogous to the variational formula (\ref{km6})
we have that an ansatz of the variation of $W(z,z')$ under 
this gauge transformation is given by
\begin{equation}
W(z,z')
  \frac{1}{2\pi i}\oint_z dw \omega(w)J(w)  
\label{int3}
\end{equation}
where the generator $J$ for this variation is to
be specified. This $J$ will be a quantum generator
which generalizes the classical generator $J$ in
(\ref{km6}).

Thus under a gauge transformation  
from (\ref{w1})
we have the following variational equation:
\begin{equation}
0= \langle W(z,z')[\delta_{\omega}A(z)+\frac{1}{2\pi i}\oint_z dw\omega(w)J(w)A(z)]\rangle
\label{w2}
\end{equation}
where $\delta_{\omega}A(z) $ denotes the variation of the field $A(z)$ in the direction $\omega$.
From this equation an ansatz of $J$ is that $J$
%and the definition of the correlation $\langle \rangle$ we have the following equation
satisfies the following equation:
\begin{equation}
W(z,z')[\delta_{\omega}A(z)+\frac{1}{2\pi i}\oint_z dw\omega(w)J(w)A(z)] =0
\label{n8bb}
\end{equation}
From this equation we have the following variational equation:
\begin{equation}
\delta_{\omega}A(z)=\frac{-1}{2\pi i}\oint_z dw\omega(w)J(w)A(z)
\label{n8b}
\end{equation}
This completes the calculus of variation.

Let us now determine the generator $J$ in (\ref{n8b}).
As analogous to the WZW model in conformal field theory
\cite{Kni} \cite{Fra} let us consider a $J$ given by
\begin{equation}
J(z) := -k W^{-1}(z, z')\partial_z W(z, z')
\label{n6}
\end{equation}
where we set $z'=z$ after the differentiation
with respect to $z$; $ k>0 $ is a constant which is fixed when the $J$ is determined to be of the form (\ref{n6}) and the
minus sign is chosen by convention.
In the WZW model \cite{Kni}\cite{Fra}
 the $J$ of the form (\ref{n6})
is the  generator  of
the chiral symmetry of the WZW model. 
%Since $W(z, z')$ is a holomorphic function of $z$ and is based the loop group of $SU(2)$ which is with generators $t^a$
We can write the $J$ in (\ref{n6}) in the following form:
\begin{equation}
 J(w) = \sum_a J^a(w) = 
\sum_a j^a(w) t^a =
\sum_a
\sum_{n=-\infty}^{\infty}J_n^a (w-z)^{-n-1} 
\label{km8}
\end{equation}
We see that the generators $t^a$ of the gauge group appear
in this form of $J$ and  this form is
completely analogous to the classical $J$ in
(\ref{km3}). This shows that 
 this $J$ is a possible candidate for the generator
$J$ in (\ref{n8b}).

Here let us consider a property of the gauge invariance
of gauge field.
Because of gauge invariance there is a freedom to
choose a gauge for a gauge field and it is needed
to fix a gauge for  computation and for the gauge
model to be well defined.
%Different gauge fixings will give equivalent physical interpretations 
\cite{Fad}. 

Let us then consider again the $J$ in (\ref{n6})
and the Wilson line $W(z,z')$. Since $W(z,z')$ is constructed
with a gauge field we need to have a gauge fixing
for the computations related to $W(z,z')$. Then since the
$J$ in (\ref{n6}) is constructed from $W(z,z')$
we have that in choosing this $J$ as the generator
$J$ in (\ref{n8b}) we have indirectly added a condition
for the gauge fixing. 

In this paper we shall always choose this gauge fixing condition.
With this gauge fixing condition the
quantum gauge model is then completed.


Now we want to show that this generator $J$ in (\ref{n6})
can be uniquely solved (This also means that the gauge fixing condition has already fixed the gauge that we can carry out computation).
From (\ref{n5}) and (\ref{n6}) we have that the variation $\delta_{\omega}J$
of the generator $J$ in (\ref{n6}) is given by
\cite{Fra}\cite{Kni}:
\begin{equation}
\delta_{\omega}J= \lbrack J, \omega\rbrack -k\partial_z \omega
\label{n8c}
\end{equation}

From (\ref{n8b}) and (\ref{n8c}) we have that $J$ satisfies
the following relation of current algebra \cite{Fra}\cite{Fuc}\cite{Kni}:
\begin{equation}
J^a(w)J^b(z)=\frac{k\delta_{ab}}{(w-z)^2}
+\sum_{c}if_{abc}\frac{J^c(z)}{(w-z)}
\label{n8d}
\end{equation}
where as a convention the regular term of the product $J^a(w)J^b(z)$
is omitted.
Then by following \cite{Fra}\cite{Fuc}\cite{Kni} from (\ref{n8d}) and (\ref{km8}) and  we can show that 
the $J_n^a$ in (\ref{km8}) satisfy the following  Kac-Moody algebra: 
\begin{equation}
[J_m^a, J_n^b] =
if_{abc}J_{m+n}^c + km\delta_{ab}\delta_{m+n, 0}
\label{n8}
\end{equation}
where $k$ is  usually called the central extension
or the level of the Kac-Moody algebra.

Let us then consider the other side of the chiral symmetry. 
Similar to the $J$ in (\ref{n6}) we define a
generator $J^{\prime}$ by:
\begin{equation}
J^{\prime}(z')= k\partial_{z'}W(z, z')W^{-1}(z, z')
\label{d1}
\end{equation}
where after differentiation with respect to $z'$
we set $z=z'$. 
%We shall show that $J^{\prime}=J$.
Let us then consider
 the following correlation:
\begin{equation}
\langle A(z')W(z,z') \rangle := 
\int dA_1^{*}dA_1dA_2^{*}dA_2dZ^{*}dZ
  A(z')W(z,z')e^{-L}
\label{n8aa}
\end{equation}
where $z$ is fixed.
By an approach similar to the above derivation of (\ref{n8b})
we have the following  variational equation:
\begin{equation}
\delta_{\omega^{\prime}}A(z')
=\frac{-1}{2\pi i}\oint_{z^{\prime}} dwA(z')J^{\prime}(w)
\omega^{\prime}(w)
\label{n8b1}
\end{equation}
where as a gauge fixing we choose the $J^{\prime}$
in (\ref{n8b1}) be the $J^{\prime}$ in (\ref{d1}).
Then similar to (\ref{n8c}) we also we have
\begin{equation}
\delta_{\omega^{\prime}}J^{\prime}= 
\lbrack  J^{\prime}, \omega^{\prime}\rbrack -k\partial_{z'} \omega^{\prime}
\label{n8c1}
\end{equation}
Then from (\ref{n8b1}) and (\ref{n8c1}) we can derive the current
algebra and the Kac-Moody algebra for $J^{\prime}$ which are of the
same form of (\ref{n8d}) and (\ref{n8}).
From this we  have $J^{\prime}=J$.

Now with the above current operator $J$ and the formula (\ref{n8b}) we can follow the usual approach
in conformal field theory to derive the
Knizhnik-Zamolodchikov equation for the product of 
primary fields in a conformal field theory \cite{Fra}\cite{Fuc}\cite{Ng}.
In our case here we  derive the 
Knizhnik-Zamolodchikov equation for the product of
$n$ Wilson lines
$W(z, z')$.
Here from the two sides of 
$W(z, z')$  we can derive two Knizhnik-Zamolodchikov equations which are inversely dual to each other. 

We have the following
Knizhnik-Zamolodchikov equation \cite{Fra} \cite{Fuc}\cite{Ng}:
\begin{equation}
\partial_{z_i}
 W(z_1, z_1^{\prime})\cdot\cdot\cdot 
W(z_n, z_n^{\prime})
=\frac{-1}{k+g}
\sum_{j\neq i}^{n}
\frac{\sum_a t_i^a \otimes t_j^a}{z_i-z_j}
 W(z_1, z_1^{\prime})\cdot\cdot\cdot 
W(z_n, z_n^{\prime})
\label{n9}
\end{equation}
for $i=1, ..., n$
where $g$ denotes the dual Coxeter number for $SU(2)$ and we have $g=2e^2$. 
%Here with our definition of the generators $t^k$ of $SU(2)$ we have $g=2e^2$.

 We also have the following Knizhnik-Zamolodchikov equation with repect to
the $z_i^{\prime}$ variables which is dual to (\ref{n9}):
\begin{equation}
\partial_{z_i^{\prime}}
 W(z_1,z_1^{\prime})\cdot\cdot\cdot W(z_n,z_n^{\prime})
= \frac{-1}{k+g}\sum_{j\neq i}^{n}
 W(z_1, z_1^{\prime})\cdot\cdot\cdot 
W(z_n, z_n^{\prime})
\frac{\sum_a t_i^a\otimes t_j^a}{z_j^{\prime}-z_i^{\prime}}
\label{d8}
\end{equation}
for $i=1, ..., n$.

\section{Winding Number of Dirac-Wilson Loop as Quantization}\label{sec8a}

In this section we solve the Dirac-Wilson Loop in a form with a winding property. We show that there are discrete winding numbers come out from the Dirac-Wilson loop. When modeling photon by the abelian Dirac-Wilson loop these discrete winding numbers will be regarded as the quantization of energy of the Planck's formula of radiation.

Let us consider the following product of two
Wilson lines:
\begin{equation}
G(z_1,z_2, z_3, z_4):=
 W(z_1, z_2)W(z_3, z_4)
\label{m1}
\end{equation}
where the two Wilson lines $W(z_1, z_2)$ and 
$W(z_3, z_4)$ represent two pieces
of curves starting at $z_1$ and $z_3$ and ending at
$z_2$ and $z_4$ respectively. 

We have that this product $G$ satisfies the KZ equation for the
variables $z_1$, $z_3$ and satisfies the dual KZ equation
for the variables $z_2$ and $z_4$.
Let us first consider the case that the gauge group is $SU(2)$.
Then by solving the two-variables-KZ equation in (\ref{n9}) we have that a form of $G$ is 
given by \cite{Chari}\cite{Koh}\cite{Dri}
\begin{equation}
e^{-t\log [\pm (z_1-z_3)]}C_1
\label{m2}
\end{equation}
where $t:=\frac{1}{k+g}\sum_a t^a \otimes t^a$ is a Casimir operator of $SU(2)$ and $C_1$ denotes a constant matrix which is independent
of the variable $z_1-z_3$.
We see that $G$ is a multivalued analytic function
where the determination of the $\pm$ sign depended on the choice of the
branch.

Similarly by solving the dual two-variable-KZ equation
 in (\ref{d8}) we have that
$G$ is of the form
\begin{equation}
C_2e^{t\log [\pm (z_4-z_2)]}
\label{m3}
\end{equation}
where $C_2$ denotes a constant matrix which is independent
of the variable $z_4-z_2$.

From (\ref{m2}), (\ref{m3}) and we let
$C_1=Ae^{t\log[\pm (z_4-z_2)]}$, 
$C_2= e^{-t\log[\pm (z_1-z_3)]}A$ where $A$ is a constant matrix as an initial condition we have that $G$ is given by
\begin{equation}
G(z_1, z_2, z_3, z_4)=
e^{-t\log [\pm (z_1-z_3)]}Ae^{t\log [\pm (z_4-z_2)]}
\label{m4}
\end{equation}

%Then we let $U({\bf a}\otimes {\bf a})$ denote the universal enveloping algebra of a tensor algebra ${\bf a}\otimes {\bf a}$ where ${\bf a}$ denotes an algebra which is formed by the Lie algebra $su(2)$ and the identity matrix.
 

Let us set $z_2=z_3$. In this case the degree of freedom of
$W(z_1, z_2)W(z_3, z_4)$
is reduced and we have 
$W(z_1, z_4)=W(z_1, z_2)W(z_2, z_4)$.
Then since $t$ is a Casimir operator for 
$SU(2)$ and $A$ is an initial operator for $W(z_1,z_4)$ we have that
$\Phi=e^{-t\log [\pm (z_1-z_3)]} $  and $\Psi=e^{t\log [\pm (z_4-z_2)]} $ as matrix acted on $A$ commute with $A$ since  $\Phi $  and $\Psi$ are exponentials of $t$. Thus we have
\begin{equation}
W(z_1, z_4)
=W(z_1, z_2)W(z_2, z_4)
=e^{-t\log [\pm (z_1-z_2)]}Ae^{t\log[\pm (z_4-z_2)]}
=e^{-t\log [\pm (z_1-z_2)]}e^{t\log[\pm (z_4-z_2)]}A
\label{closed1a}
\end{equation}

Now let $z_1=z_4$. In this case we have a closed loop.
Now in (\ref{closed1a})
we have that $e^{-t\log[\pm (z_1-z_2)]}$ and
$e^{t\log [\pm (z_1-z_2)]}$ cancel each other and from the multivalued
property of the
$\log$ function we have
\begin{equation}
W(z_1, z_1)
=RA 
\label{closed2}
\end{equation}
where $R :=e^{-in\pi t}$ for $n=0, \pm 1, \pm 2, ...$ is  the monodromy of the KZ equation for $SU(2)$ where the integer $n$ is as a winding number \cite{Chari}. In choosing an integer $n$ we have chosen a branch of $R$.

Similarly when the gauge group is $SU(2)\otimes U(1)$ we have
\begin{equation}
W(z_1, z_1)
=R_{U(1)}R_{SU(2)}A 
\label{closed21}
\end{equation}
where $R_{SU(2)}:=R =e^{-in\pi t}$ for $n=0, \pm 1, \pm 2, ...$ is  the monodromy of KZ equation for $SU(2)$ and $R_{U(1)}$ denotes the monodromy of the KZ equation for $U(1)$. We have $R_{U(1)}=e^{in\frac{e^2\pi}{k+g}}$
for $n=0, \pm 1, \pm 2, ...$ where the $e$ of $e^2$ denotes the electric charge. We shall show that the winding number $n$ give the quantization property of photon.

Now
we have that the Dirac-Wilson loop $W(z_1, z_1)$ corresponds to a closed
curve in the complex plane with starting and ending
point $z_1$. 
Let this Dirac-Wilson loop $W(z_1, z_1)$ represents the unknot. Then we let the gauge group be just the $U(1)$ group. In this case we show in the following section that the Dirac-Wilson loop $W(z_1, z_1)$ is a model of the photon.
%Then from (\ref{closed2}) we have the following invariant for the unknot:
%\begin{equation}
%Tr W(z_1, z_1)= Tr R_{U(1)}RA 
%\label{m6}
%\end{equation}

\section{Photon as the Magnetic Monopole}

We see that the Dirac-Wilson loop is an exactly solvable nonlinear observable. Thus we may regard it as a quantum soliton of the above gauge model. In particular for the abelian gauge model with $U(1)$ as gauge group we regard the Dirac-Wilson loop as a quantum soliton of the electromagnetic field. We now want to show that this soliton has all the properties of photon and thus we may identify it with the photon. First we see that it has discrete energy levels of light-quantum given by
\begin{equation} 
n h \nu :=n \frac{\pi e^2}{k+g}, \qquad n=0,1,2,3, . . .
\label{planck}
\end{equation}
 where $h$ is the Planck's constant; $\nu$ denotes a frequency and the constant $k$ is determined from this formula. This formula is from the monodromy $R_{U(1)}$ for the abelian gauge model. We see that the Planck's constant $h$ comes out from this winding property of the Dirac-Wilson loop. Then since this Dirac-Wilson loop is a loop we have that it has the polarization property of light by the right hand rule along the loop and this polarization can also be regarded as the spin of photon. Now since this loop is a quantum soliton which behaves as a particle we have that this loop is a basic particle of the above abelian gauge model where the abelian gauge property is considered as the fundamental property of electromagnetic field. This shows that the Dirac-Wilson loop has properties of photon. We shall later show that from this loop model of photon we can describle the absorption and emission of photon by an electron. This property of absorption and emission is considered as a basic principle of the light-quantum hypothesis of Einstein \cite{Pai}. From these properties of the Dirac-Wilson loop we may identify it with the photon.

On the other hand from Dirac's analysis of the magnetic monopole \cite{Dir} we have that the property of magnetic monopole comes from a closed line integral of vector potential of the electromagnetic field which is similar to the Dirac-Wilson loop. Now from this Dirac-Wilson loop we can define the magnetic charge $q$ which is given by
\begin{equation} 
e q:= e\frac{e\pi n}{k+g}, \qquad n=0,1,2,3, . . .
\label{dirac}
\end{equation} 
where $e$ denotes the electric charge. This shows that the Dirac-Wilson loop gives the property of magnetic monopole. Since this loop is a quantum soliton which behaves as a particle we have that this Dirac-Wilson loop may be identified with the magnetic monopole. Thus we have that photon may be identified with the magnetic monopole.

With this identification we have the following interesting conclusion: Both the energy quantization of electromagnetic field and the charge quantization property come from the same property of photon. Indeed we have
\begin{equation} 
nh\nu=n\frac{e^2\pi}{k+g}=e\frac{e\pi n}{k+g}=eq, \qquad n=0,1,2,3, . . .
\label{dirac2}
\end{equation}
This formula shows that the energy quantization and the charge quantization are just the same quantization and is a property of the photon when photon is modeled by the Dirac-Wilson loop and identified with the magnetic monopole.

\section{Nonlinear Model of Electron}

In this section let us also give a loop model to the electron. This loop model of electron is based on the above loop model of the photon. From the loop model of photon we also construct an observable which gives mass to the electron and is thus a mass machanism for the electron.

Let W(z,z) denote a Dirac-Wilson loop which represents a photon. Let
$Z$ denotes the complex variable for electron in (\ref{1.1}). We then consider the following observable:
\begin{equation}
 W(z,z)Z
\label{electron}
\end{equation}
Since $W(z,z)$ is solvable we have that this observable is also solvable where in solving $W(z,z)$ the variable $Z$ is fixed. We let this observable be identified with the electron. Then we consider the following observable:
\begin{equation}
 Z^* W(z,z)Z
\label{electron1}
\end{equation}
This observable is with a scalar factor $Z^* Z$ where $Z^*$ denotes the complex conjugate of $Z$ and we regard it as the mass mechanism of the electron (\ref{electron}). For this observable we model the energy levels of $W(z,z)$ as the mass levels of electron and the mass $m$ of electron is the lowest nonzero energy level $h\nu$ of $W(z,z)$ and is given by:
\begin{equation}
 mc^2=h\nu
\label{electron2}
\end{equation}
where $c$ denotes the constant of the speed of light. From this model of the mass mechanism of electron we have that electron is with mass $m$ while photon is with zero mass because there does not have such a mass mechanism $Z^* W(z,z)Z$ for the photon. From this definition of mass we have the following formula relating the charge $e$ of electron, the magnetic charge $q_{min}$ of magnetic monopole and the mass $m$ of electron:
\begin{equation}
 mc^2= e q_{min}=h\nu
\label{electron3}
\end{equation}

By using the nonlinear model $W(z,z)Z$ to represent an electron we can then describe the absorption and emission of a photon by an electron where photon is as a parcel of energy described by the loop $W(z,z)$, as follows. Let $W(z,z)Z$ represents an electron and let $W_1(z_1,z_1)$ represents a photon. Then we have the observable $(W(z,z)+W_1(z_1,z_1))Z$ which represents an electron having absorbed the photon $W_1(z_1,z_1)$. This property of absorption and emission is as a basic principle of the hypothesis of light-quantum stated by Einstein \cite{Pai}. Let us quote the following paragraph from \cite{Pai}:

..., First, the light-quantum was conceived of as a parcel of energy as far as the properties of pure radiation (no coupling to matter) are concerned. Second, Einstein made the assumption--he call it the heuristic principle--that also in its coupling to matter (that is, in emission and absorption), light is created or annihilated in similar discrete parcels of energy. That, I believe, was Einstein's one revolutionary contribution to physics. It upset all existing ideas about the interaction between light and matter. ....

\section{Photon Carrying the Electric and Magnetic Charge}

In this loop model of photon we have that the electric charge $e$ is carried by the photon since the generator $t^0$ of $U(1)$ in this loop model is of the form $t^0=ieT^0$ where $T^0$ denotes a self-adjoint generator of $U(1)$. From this generator $t^0$ we have shown in the above that the magnetic charge $q$ is also carried by the photon for the photon to be identified with the magnetic monopole. Let us here describe the physical effects from this property of photon that photon carries the electric and magnetic charge. First from the nonlinear model of electron we have that an electron $W(z,z)Z$ also carries the electric charge when a photon $W(z,z)$ is absorbed to form the electron $W(z,z)Z$. This means that the electric charge of an electron is from the electric charge carried by a photon. As we know this electric charge gives the electric force between two electrons. Then how about the effect of this electric charge to two photons? We have that the two electric charges carried by two photons will give a correlation effect between the two photons as a potential. This correlation effect will not directly give an electric force between two photons. However when these two photons are both coupled with matters to form two electrons then this  correlation effect will give an electric force between these two electrons (In the case that only one photon is coupled to matter to form an electron this correlation effect will still not give an electric force between a photon and an electron). Thus when the two photons are both coupled to matters to form two electrons they will feel the electric force since the two electrons feel the electric force.

Similarly the magnetic charges carried by two photons will give a correlation effect as a potential between two photons. This effect will not directly give a magnetic force between these two photons. It is only when these two photons are both coupled to matters that from this correlation effect a magnetic force will come out between the two matters coupled by the two photons and in this case the two photons will also feel this magnetic force since they are coupled with these two matters.

\section{Statistics of Photons and Electrons}

Let us consider more on the nonlinear model of electron $W(z,z)Z$ which gives a relation between photons and electrons where photons is modeled by $W(z,z)$. We want to show that from this nonlinear model we may also derive the required statistics of photons and electrons that photons obey the Bose-Einstein statistics and electrons obey the Fermi-Dirac statistics. We have that $W(z,z)$ is as an operator acting on $Z$ which represents an electron while the nonlinear model of electron $W(z,z)Z$ gives a more complete description of electron that it describes that a photon  $W(z,z)$ is absorbed by an electron and the two again form an electron. There may be many photons $W_n (z,z), n=1,2,3,...$ absorbed by an electron which gives an electron $\sum_n W_n (z,z)Z$. This formula shows that identical (but different) photons can appear identically and it shows that photons obey the Bose-Einstein statistics. From the polarization of the Dirac-Wilson loop $W(z,z)$ we may assign spin $1$ to a photon represented by $W(z,z)$. 

Let us then consider statisics of electrons. We have that the observable $Z^*W(z,z)Z$ gives mass to the electron $W(z,z)Z$ and thus this observable is as a scalar and thus is assigned with spin $0$. As the observable $W(z,z)Z$ is between $W(z,z)$ and $Z^*W(z,z)Z$ which are with spin $1$ and $0$ respectively we thus assign spin $\frac12$ to the observable $W(z,z)Z$ and thus electron represented by this observable $W(z,z)Z$ is with  spin $\frac12$.
Then let $Z_1$ and $Z_2$ be two independent complex variables for two electrons. Let  $W(z,z)$ represents a photon. Then the operator $W(z,z)(Z_1+Z_2)$ means that two electrons are in the same state. However this operator means that a photon $W(z,z)$ is absorbed by two distinct electrons and this is impossible. Thus this operator $W(z,z)(Z_1+Z_2)$ cannot exist and this means that electrons obey Fermi-Dirac statistics.

%Then this gives Fermi-Dirac statistics to electrons.
%Then from the relation of $Z$ and $W(z,z)$ which are related by $W(z,z)Z$
%On the other hand when a photon $ W(z,z)$ acts on several electrons $Z_i, i=1,...,n$ we have the formula $W(z,z) ( Z_1(z), . . . Z_n(z))^T$ where $T$ denotes transpose and $( Z_1(z), . . . Z_n(z))^T$ denotes a column vector. This formula shows that electrons form spinor and thus electrons obey the Fermi-Dirac statistics. 
Thus we have that this nonlinear loop model of photon and electron give the required statistics of photons and electrons.


\section{Conclusion}

In this paper a quantum loop model of photon is set up. We show that this loop model is exactly solvable and thus may be considered as a quantum soliton. We show that this nonlinear model of photon has properties of photon and magnetic monopole and thus photon may be identified with the magnetic monopole. From the discrete winding numbers of this loop model we can derive the quantization property of energy for the Planck's formula and the quantization property of electric charge. We show that energy quantization and charge quantization are just the same quantization. On the other hand from the nonlinear model of photon a nonlinear loop model of electron is set up. This model of electron has a mass machanism which generates mass to the electron. 
%From this construction we get a formula relating the charge $e$ of electron, the magnetic charge $q_{min}$ of magnetic monopole (or the photon) and the mass $m$ of electron which is given by $mc^2= e q_{min}$. 


\begin{thebibliography}{38}

\bibitem{Pai}
A. Pais, Subtle is the Lord, (Oxford University Press 1982).

\bibitem{Pla}
M. Planck, Annalen der Physik {\bf 4} 553 (1901).

\bibitem{Ein}
A. Einstein, Annalen der Physik {\bf 17} 891 (1905).

\bibitem{Dir}
P.A.M. Dirac, Phys. Rev. {\bf 74} 817 (1948).

\bibitem{Dir2}
P.A.M. Dirac, Directions in Physics, (John Wiley and Sons Inc. 1978).

\bibitem{Zub}
C. Itzykson and J.B. Zuber,
Quantum field thepry, (McGraw-Hill Inc. 1980).

\bibitem{Jaf}
J. Glimm and A. Jaffe,
Quantum Physics,
(Springer-Verlag, 1987).

\bibitem{Fad}
L.D. Faddev and V.N. Popov,
Phys. Lett. {\bf 25B} 29 (1967).

\bibitem{Wit}
E. Witten,
%"Quantum field theory and the Jones polynomial",
Comm. Math. Phys. 121 (1989)351.

\bibitem{Kau}
L. Kauffman,
Knots and Physics,
(World Scientific, 1993).


\bibitem{Baez}
J. Baez and J. Muniain, 
Gauge Fields, Knots and Gravity,
(World Scientic 1994).

\bibitem{Lus}
D. Lust and S. Theisen,
Lectures on String Theory,
(Springer-Verlag 1989).

\bibitem{Seg}
A. Pressley and G. Segal,
Loop Groups,
(Clarendon Press 1986).


\bibitem{Fra}
P. Di Francesco, P. Mathieu and D. Senechal,
Conformal Field Theory,
(Springer-Verlag 1997).

\bibitem{Fuc}
J. Fuchs, 
Affine Lie Algebras and Quantum Groups,
(Cambridge University Press 1992).

\bibitem{Kni}
V.G. Knizhnik and A.B. Zamolodchikov,
%"Current algebra and Wess-Zumino model in two dimensions",
Nucl. Phys. B {\bf 247} 83 (1984).

\bibitem{Ng}
 S.K. Ng,
%Knot Invariants and New Quantum Field Theory, 
math-QA/0004151, math-QA/0008103.

%\bibitem{Cor}
%J. F. Cornwell,
%Group Theory in Physics,
%(Academic Press 1984).

\bibitem{Chari}
V. Chari and A. Pressley,
A Guide to Quantum Groups,
(Cambridge University Press 1994).

\bibitem{Koh}
T. Kohno,
%"Monodromy representations of braid groups and Yang-Baxter equations",
Ann. Inst. Fourier (Grenoble) {\bf 37}  139-160 (1987).

\bibitem{Dri}
V. G. Drinfel'd.
%"Quasi-Hopf algebras",
Leningrad Math. J. {\bf 1}  1419-57 (1990).


\end{thebibliography}

\end{document}
