
%
%
%
%
%
%
%

\documentclass[a4paper,12pt]{article}

\usepackage{amsmath,amssymb}


\textheight 220 mm
\textwidth 160 mm

\oddsidemargin  - 5mm
\evensidemargin - 5mm


\providecommand{\nn}{\nonumber}

\providecommand{\son}{SO(n, \mathbb{R})}
\providecommand{\sot}{SO(3, \mathbb{R})}
\providecommand{\sof}{SO(4, \mathbb{R})}

\usepackage{useful_macros}
\begin{document}



\title{\bf
Unconstrained \myHighlight{$SU(2)$}\coordHE{} Yang-Mills Theory with Topological Term
in the Long-Wavelength Approximation}

\author{
A. M. Khvedelidze \myHighlight{$^{a,b}$}\coordHE{},  \,\,
D. M. Mladenov    \myHighlight{$^c$}\coordHE{},      \,\,
H.-P. Pavel       \myHighlight{$^{c,d}$}\coordHE{},      \, and  \,
G. R\"opke        \myHighlight{$^d$}\coordHE{} \\[4mm]
\myHighlight{$^a$}\coordHE{}{\it A. Razmadze Mathematical Institute, Tbilisi, 380093, Georgia}\\[2mm]
\myHighlight{$^b$}\coordHE{} {\it Laboratory of Information Technologies},\\
{\it Joint Institute for Nuclear Research, 141980 Dubna, Russia}\\[2mm]
\myHighlight{$^c$}\coordHE{} {\it Bogoliubov Laboratory of Theoretical Physics},\\
{\it Joint Institute for Nuclear Research, 141980 Dubna, Russia}\\[2mm]
\myHighlight{$^d$}\coordHE{} {\it Fachbereich Physik der Universit\"at Rostock, D-18051 Rostock, Germany}}

\date{November 30, 2002}
\maketitle

\begin{abstract}
The Hamiltonian reduction of \myHighlight{$SU(2)$}\coordHE{} Yang-Mills theory
for arbitrary \myHighlight{$\theta$}\coordHE{}-angle to an unconstrained nonlocal
theory of a self-interacting positive definite symmetric \myHighlight{$3\times3$}\coordHE{}
matrix field \myHighlight{$S(x)$}\coordHE{} is performed.
It is shown that, after exact projection to reduced phase space,
the density of the Pontryagin index remains a pure divergence,
proving the \myHighlight{$\theta$}\coordHE{}-independence of the obtained unconstrained theory.
Expansion of the nonlocal kinetic part of the Hamiltonian
in powers of inverse coupling constant and truncation to lowest order,
however, leads to violation of the \myHighlight{$\theta$}\coordHE{}-independence of the theory.
In order to maintain this property on the level of the local approximate theory,
a modified expansion in inverse coupling constant is suggested,
which for vanishing \myHighlight{$\theta$}\coordHE{}-angle coincides with the original expansion.
The corresponding approximate Lagrangian up to second order in derivatives
is derived and the explicit form of the unconstrained analog of the Chern-Simons
current linear in derivatives is given.
Finally, for the case of degenerate field configurations
\myHighlight{$S(x)$}\coordHE{} with \myHighlight{$\mbox{rank}\|S\| = 1$}\coordHE{}, a nonlinear \myHighlight{$\sigma$}\coordHE{}-type model is obtained,
with the Pontryagin topological term reducing to the Hopf invariant of the
mapping from the \myHighlight{$3$}\coordHE{}-sphere \myHighlight{$\mathbb{S}^3$}\coordHE{} to the unit \myHighlight{$2$}\coordHE{}-sphere \myHighlight{$\mathbb{S}^2$}\coordHE{}
in the Whitehead form.

\vspace{0.5cm}
PACS numbers:  11.15.-q, 11.10.Ef, 11.10.-z, 11.25.Me, 12.38.Aw
\end{abstract}


\newpage

\bigskip

%%%%%%%%%%%%%%%%%%%%%%%%%%%%%%%%%%%%% 1 %%%%%%%%%%%%%%%%%%%%%%%%%%%%%%%%%%%%%%%%%%%

\section{Introduction}

%%%%%%%%%%%%%%%%%%%%%%%%%%%%%%%%%%%%%%%%%%%%%%%%%%%%%%%%%%%%%%%%%%%%%%%%%%%%%%%%%%%

For a complete understanding of the low-energy quantum phenomena
of Yang-Mills theory, it is necessary to have a nonperturbative,
gauge invariant description of the underlying classical theory
including the \myHighlight{$\theta$}\coordHE{}-dependent Pontryagin term \cite{JackiwRebbi}-\cite{Jackiw}.
Several representations of Yang-Mills theory in terms of local gauge
invariant fields
have been proposed \cite{GoldJack}-\cite{Majumdar} during the last decades,
implementing Gauss law as a generator of small gauge transformations.
However, dealing with such local gauge invariant fields
special consideration is needed, when the topological term is included,
since it is the 4-divergence of a current
changing under large gauge transformations. In particular,
consistency of constrained and unconstrained formulations of gauge
theories with topological term requires to verify that, after projection
to the reduced phase space, the classical equations of motion for
the unconstrained variables remain \myHighlight{$\theta$}\coordHE{}-independent
\footnote{
The question of consistency of the elimination of redundant variables
in theories containing both constraints and pure divergencies,
the so-called "divergence problem", has for the first time been analyzed
in the context of the canonical reduction of General Relativity
by P. Dirac \cite{Dirac} and by R. Arnowitt, S. Deser, C.W. Misner \cite{ADM}.
}.
Furthermore the question, which trace the large gauge transformations
with nontrivial Pontryagin topological index leave on the local gauge
invariant fields, has to be addressed.

Having this in mind, we extend in the present paper our approach
\cite{KP,AHG,KMPR}, to construct the unconstrained form of
\myHighlight{$SU(2)$}\coordHE{} Yang-Mills theory, to the case when the topological term is included
in the classical action.
We generalize the Hamiltonian reduction of classical \myHighlight{$SU(2)$}\coordHE{} Yang-Mills
field theory to arbitrary \myHighlight{$\theta$}\coordHE{}-angle
by reformulating the original degenerate
Yang-Mills theory as a nonlocal theory of a self-interacting
positive definite symmetric \myHighlight{$3\times 3$}\coordHE{} matrix field.
The consistency of the Hamiltonian reduction in the presence of the
Pontryagin term is demonstrated by constructing
the canonical transformation, well-defined on the reduced phase space,
that eliminates the \myHighlight{$\theta$}\coordHE{}-dependence of the classical equations of motion
for the unconstrained variables.

With the aim to obtain a practical form of the nonlocal unconstrained
Hamiltonian, we perform an expansion in powers of the inverse coupling constant,
equivalent to an expansion in the number of spatial derivatives.
We find that a straightforward application of the
derivative expansion violates the principle of \myHighlight{$\theta$}\coordHE{}-independence of the
classical observables. To cure this problem,
we propose to exploit the property of chromoelectro-magnetic duality of
pure Yang-Mills theory,
symmetry under exchange of chromoelectric and -magnetic fields.
The electric and magnetic fields are subject to
dual constraints, the Gauss-law and Bianchi identity, and
only when both are fulfilled, the
classical equations of motion are \myHighlight{$\theta$}\coordHE{}-independent.
Thus any approximation in resolving the Gauss law constraints
should be consistent with the Bianchi identity.
We show how to use the Bianchi identity to
rearrange the derivative expansion in such a way,
that the \myHighlight{$\theta$}\coordHE{}-independence is restored to all orders on the classical level.

In order to have a representation of the gauge invariant degrees of freedom
suitable for the study of the low energy phase of Yang-Mills theory,
we perform a main-axis transformation of the symmetric tensor field
and obtain the unconstrained Hamiltonian
in terms of the main-axis variables in the lowest order in \myHighlight{$1/g$}\coordHE{}.
Carrying out an inverse Legendre transformation
to the corresponding unconstrained Lagrangian,
we find the explicit form of the unconstrained analog of the Chern-Simons
current, linear in the derivatives.

Finally, we consider the case of degenerate symmetric field
configurations \myHighlight{$S$}\coordHE{} with \myHighlight{$\mbox{rank}\|S(x)\|= 1$}\coordHE{}.
We find a non-linear classical theory of a three-dimensional unit-vector
\myHighlight{${\bf n}$}\coordHE{}-field interacting with a scalar field.
Using typical boundary conditions for the unit-vector field
at spatial infinity, the Pontryagin topological charge density
reduces to the Abelian Chern-Simons invariant density \cite{Jackiw}.
We discuss its relation to the Hopf number of the
mapping from the \myHighlight{$3$}\coordHE{}-sphere \myHighlight{$\mathbb{S}^3$}\coordHE{} to the unit \myHighlight{$2$}\coordHE{}-sphere \myHighlight{$\mathbb{S}^2$}\coordHE{}
in the Whitehead representation\cite{Whitehead}.
The Abelian Chern-Simons invariant is known from different areas in
Physics, in fluid mechanics as ``fluid helicity``,
in plasma physics and magnetohydrodynamics  as ``magnetic helicity''
\cite{Woltier}-\cite{Saffman}.
In the context of 4-dimensional Yang-Mills theory a connection between
non-Abelian vacuum configurations and certain Abelian fields with nonvanishing
helicity established already in \cite{JackiwPi,NairJackiw}.

The paper is organized as follows.
In Section II the \myHighlight{$\theta$}\coordHE{}-independence of classical Yang-Mills
theory in the framework of the constrained Hamiltonian formulation is revised.
Section III is devoted to the derivation of
unconstrained \myHighlight{$SU(2)$}\coordHE{} Yang-Mills theory for arbitrary \myHighlight{$\theta$}\coordHE{}-angle.
The consistency of our reduction procedure is demonstrated by
explicitly quoting the canonical transformation,
which removes the \myHighlight{$\theta$}\coordHE{}-dependence from the unconstrained
classical theory.
In Section IV the unconstrained Hamiltonian up to
order \myHighlight{$o(1/g)$}\coordHE{} is obtained.
Section V presents the long-wavelength classical
Hamiltonian in terms of main-axis variables.
Performing an inverse Legendre transformation to the corresponding
Lagrangian up to second order in derivatives, the unconstrained analog
of the Chern-Simons current, linear in the derivatives, is obtained.
In Section VI the unconstrained action for degenerate field configurations
is considered.
Section VII finally gives our conclusions.
Several more technical details are presented in the Appendices A, B, and C.
Appendix A summarizes our notations and definitions,
Appendix B is devoted to the question of the exsistence of the "symmetric
gauge", and in Appendix C the proof the \myHighlight{$\theta$}\coordHE{}-dependence of the
"naive" \myHighlight{$1/g$}\coordHE{} approximation is given.

%%%%%%%%%%%%%%%%%%%%%%%%%%%%%%%%%%%%%% 2 %%%%%%%%%%%%%%%%%%%%%%%%%%%%%%%%%%%%%%%

\section{Constrained Hamiltonian formulation}

%%%%%%%%%%%%%%%%%%%%%%%%%%%%%%%%%%%%%%%%%%%%%%%%%%%%%%%%%%%%%%%%%%%%%%%%%%%%%%%%

\label{sec:cedg}

Yang-Mills gauge fields are classified topologically by the
value of Pontryagin index
\footnote{Necessary notations and definitions for \myHighlight{$SU(2)$}\coordHE{} Yang-Mills
theory used in the text have been collected in Appendix A.}
\begin{equation}\coord{}\boxEquation{
p_1 = - \frac{1}{8 \, \pi^2} \, \int \, \mbox{tr} \, F \wedge F \,.
}{
p_1 = - \frac{1}{8 \, \pi^2} \, \int \, \mbox{tr} \, F \wedge F \,.
}{ecuacion}\coordE{}\end{equation}
Its density, the so-called topological charge density
\myHighlight{$
Q\, = -\, (1/ 8 \, \pi^2)\, \mbox{tr} \, F \wedge F\,,
$}\coordHE{}
being locally exact
\myHighlight{$ Q \, = \, d\,C\,, $}\coordHE{}
can be added to the conventional Yang-Mills Lagrangian
with arbitrary parameter \myHighlight{$\theta$}\coordHE{}
\begin{equation}\coord{}\boxEquation{
\label{eq:Lagr}
{\cal L}  =  - \frac{1}{g^2}  \, \mbox{tr} \, F \wedge {}^\ast\! F -
\, \frac{\theta}{8\pi^2\, g^2} \, \mbox{tr} \, F \wedge F \,,
}{
{\cal L}  =  - \frac{1}{g^2}  \, \mbox{tr} \, F \wedge {}^\ast\! F -
\, \frac{\theta}{8\pi^2\, g^2} \, \mbox{tr} \, F \wedge F \,,
}{ecuacion}\coordE{}\end{equation}
without changing the classical equations of motion.
In the Hamiltonian formulation, this
shifts the canonical momenta,
conjugated to the field variables \myHighlight{$A_{ai}$}\coordHE{},
\begin{eqnarray}\coord{}\boxAlignEqnarray{\leftCoord{} \rightCoord{}\label{eq:mom}
&&\leftCoord{}  \Pi_{ai} =
\frac{\leftCoord{} \partial{\cal L} }{\rightCoord{}\partial {\dot A}_{ai} } =
{\rightCoord{}\leftCoord{}\dot A}_{ai} - \left( D_i (A) \right)_{ac} A_{c0} +
\frac{\leftCoord{}\theta}{\rightCoord{}8 \rightCoord{}\, \pi^2} \rightCoord{}\, B_{ai}\rightCoord{}\,,
\rightCoord{}}{0mm}{5}{9}{ &&  \Pi_{ai} =
\frac{ \partial{\cal L} }{\partial {\dot A}_{ai} } =
{\dot A}_{ai} - \left( D_i (A) \right)_{ac} A_{c0} +
\frac{\theta}{8 \, \pi^2} \, B_{ai}\,,
}{1}\coordE{}\end{eqnarray}
by the magnetic field \myHighlight{$(\theta /8 \, \pi^2) \, B_{ai}\,$}\coordHE{}.
As a result, the total  Hamiltonian \cite{DiracL,HenTeit}
of Yang-Mills theory with \myHighlight{$\theta$}\coordHE{}-angle as a functional
of canonical variables \myHighlight{$(A_{a0}, \Pi_a)$}\coordHE{} and  \myHighlight{$(A_{ai}, \Pi_{ai})$}\coordHE{}
obeying the Poisson bracket relations
\begin{eqnarray}\coord{}\boxAlignEqnarray{\leftCoord{} \rightCoord{}\label{eq:pbo}
&&\leftCoord{} \rightCoord{}
\leftCoord{}\{ A_{ai}(t, \vec{x})\rightCoord{}\, ,\Pi_{bj}(t, \vec{y}) \} =
\delta_{ab}\rightCoord{}\, \delta_{ij}\rightCoord{}\, \delta^{(3)}(\vec{x} - \vec{y})\rightCoord{}\,,\rightCoord{}\\
&&\leftCoord{} \rightCoord{}
\leftCoord{}\{ A_{a0}(t, \vec{x})\rightCoord{}\,,\Pi_{b}(t, \vec{y}) \} =
\delta_{ab}\rightCoord{}\, \delta^{(3)}(\vec{x} - \vec{y})\rightCoord{}\,,
\rightCoord{}}{0mm}{5}{13}{ && 
\{ A_{ai}(t, \vec{x})\, ,\Pi_{bj}(t, \vec{y}) \} =
\delta_{ab}\, \delta_{ij}\, \delta^{(3)}(\vec{x} - \vec{y})\,,\\
&& 
\{ A_{a0}(t, \vec{x})\,,\Pi_{b}(t, \vec{y}) \} =
\delta_{ab}\, \delta^{(3)}(\vec{x} - \vec{y})\,,
}{1}\coordE{}\end{eqnarray}
takes the form
\begin{equation}\coord{}\boxEquation{ \label{eq:tothamn}
H_T = \int d^3 x \left[
\frac{1}{2}\left( \Pi_{ai} - \frac{\theta}{8\pi^2} B_{ai}\right)^2 +
\frac{1}{2}\, B_{ai}^2 - A_{a0} \left(D_i(A)\right)_{ac} \Pi_{ci}
+ \lambda_a \, \Pi_a
 \right]\,.
}{ H_T = \int d^3 x \left[
\frac{1}{2}\left( \Pi_{ai} - \frac{\theta}{8\pi^2} B_{ai}\right)^2 +
\frac{1}{2}\, B_{ai}^2 - A_{a0} \left(D_i(A)\right)_{ac} \Pi_{ci}
+ \lambda_a \, \Pi_a
 \right]\,.
}{ecuacion}\coordE{}\end{equation}
Here, the linear combination of three primary constraints
\begin{equation}\coord{}\boxEquation{
\label{primconst}
\Pi_a (x) = 0\,,
}{
\Pi_a (x) = 0\,,
}{ecuacion}\coordE{}\end{equation}
with arbitrary functions \myHighlight{$\lambda_a(x)$}\coordHE{}
and the secondary constraints, the non-Abelian Gauss law
\begin{equation}\coord{}\boxEquation{
 \label{eq:secconstr}
\left(D_i(A)\right)_{ac} \Pi_{ci} = 0 \,
}{
 \left(D_i(A)\right)_{ac} \Pi_{ci} = 0 \,
}{ecuacion}\coordE{}\end{equation}
reflect the gauge invariance of the theory.

Based on the representation (\ref{eq:tothamn}) for the total Hamiltonian,
one can immediately verify that classical theories with different
value of the \myHighlight{$\theta$}\coordHE{}-angle are equivalent.
Performing the canonical transformation
\begin{eqnarray}\coord{}\boxAlignEqnarray{\leftCoord{}
A_{ai}(x)   &\longmapsto & A_{ai}(x) \rightCoord{}\,, \nn\rightCoord{}\\\leftCoord{}
\Pi_{bj}(x) &\longmapsto & E_{bj}    :=
\Pi_{bj}(x) - \frac{\leftCoord{}\theta}{\rightCoord{}8 \pi^2}\rightCoord{}\, B_{bj}(x)\rightCoord{}\,,
\label{eq:cantrtheta}
\rightCoord{}}{0mm}{3}{7}{
A_{ai}(x)   &\longmapsto & A_{ai}(x) \,, \nn\\
\Pi_{bj}(x) &\longmapsto & E_{bj}    :=
\Pi_{bj}(x) - \frac{\theta}{8 \pi^2}\, B_{bj}(x)\,,
}{1}\coordE{}\end{eqnarray}
to the new variables \myHighlight{$A_{ai}$}\coordHE{} and \myHighlight{$E_{bj}$}\coordHE{},
and using the Bianchi identity
\begin{equation}\coord{}\boxEquation{
(D_i(A))_{ab} \, B_{bi}(A) = 0\,,
}{
(D_i(A))_{ab} \, B_{bi}(A) = 0\,,
}{ecuacion}\coordE{}\end{equation}
one can then see that the \myHighlight{$\theta$}\coordHE{}-dependence completely
disappears from the Hamiltonian (\ref{eq:tothamn}).
Note that the canonical transformation (\ref{eq:cantrtheta})
can be represented in the form
\begin{equation}\coord{}\boxEquation{ \label{clctr}
E_{ai}\,  =
\, \Pi_{ai} - \, \theta \,\frac{\delta}{\delta A_{ai}}\, W[A]\,,
}{ E_{ai}\,  =
\, \Pi_{ai} - \, \theta \,\frac{\delta}{\delta A_{ai}}\, W[A]\,,
}{ecuacion}\coordE{}\end{equation}
where \myHighlight{$W[A]$}\coordHE{} denotes the winding number functional,
\begin{equation}\coord{}\boxEquation{\label{clctr1}
W[A] \, = \, \int d^3 x \, K^0[A]\,,
}{W[A] \, = \, \int d^3 x \, K^0[A]\,,
}{ecuacion}\coordE{}\end{equation}
constructed from the zero component of the Chern-Simons current
\begin{equation}\coord{}\boxEquation{
\label{CSC}
K^\mu[A] \, = - \frac{1}{16 \, \pi^2}\varepsilon^{\mu\alpha\beta\gamma}\,
\mbox{tr}
\left(
F_{\alpha\beta} \, A_\gamma - \frac{2}{3}\, A_\alpha \, A_\beta \, A_\gamma
\right)\,.
}{
K^\mu[A] \, = - \frac{1}{16 \, \pi^2}\varepsilon^{\mu\alpha\beta\gamma}\,
\mbox{tr}
\left(
F_{\alpha\beta} \, A_\gamma - \frac{2}{3}\, A_\alpha \, A_\beta \, A_\gamma
\right)\,.
}{ecuacion}\coordE{}\end{equation}
The question now arises, whether, after reduction of Yang-Mills theory
including topological term to the unconstrained system,
a transformation analogous to (\ref{eq:cantrtheta}) can be found,
that correspondingly eliminates any \myHighlight{$\theta$}\coordHE{}-dependence on the
reduced level, proving the consistency of the Hamiltonian reduction.

%%%%%%%%%%%%%%%%%%%%%%%%%%%%%%%%%%%%%% 3 %%%%%%%%%%%%%%%%%%%%%%%%%%%%%%%%%%%%%%%

\section{Unconstrained Hamiltonian formulation}

%%%%%%%%%%%%%%%%%%%%%%%%%%%%%%%%%%%%%%%%%%%%%%%%%%%%%%%%%%%%%%%%%%%%%%%%%%%%%%%%
\label{sec:unhf}


\subsection{Hamiltonian reduction for arbitrary \myHighlight{$\theta$}\coordHE{}-angle}

\label{sec:unhf1}

In order to derive the unconstrained form of \myHighlight{$SU(2)$}\coordHE{} Yang Mills theory
with \myHighlight{$\theta$}\coordHE{}-angle we follow the method developed in \cite{KP} .
We perform the point transformation
\begin{equation}\coord{}\boxEquation{
\label{eq:gpottr}
A_{ai} \left(q, S \right) =
O_{ak}(q)\, S_{ki} +
\frac{1}{2 g} \,\varepsilon_{abc}\, \left(\partial_i O (q)\, O^T (q) \right)_{bc}
}{
A_{ai} \left(q, S \right) =
O_{ak}(q)\, S_{ki} +
\frac{1}{2 g} \,\varepsilon_{abc}\, \left(\partial_i O (q)\, O^T (q) \right)_{bc}
}{ecuacion}\coordE{}\end{equation}
from the gauge fields \myHighlight{$A_{ai}(x)$}\coordHE{} to the new set of three
fields \myHighlight{$q_j(x),\,j = 1,2,3,$}\coordHE{} parameterizing an orthogonal \myHighlight{$3 \times 3$}\coordHE{} matrix
\myHighlight{$O(q)$}\coordHE{} and the six fields \myHighlight{$S_{ik}(x) = S_{ki}(x),\, i,k = 1,2,3,$}\coordHE{}
collected in the positive definite symmetric \myHighlight{$3 \times 3$}\coordHE{} matrix \myHighlight{$S(x)$}\coordHE{}
\footnote{
It is necessary to note that a decomposition similar to (\ref{eq:gpottr})
was used in \cite{Simonov} as generalization of the well-known polar
decomposition valid for arbitrary quadratic matrices.}.
Eq. (\ref{eq:gpottr}) can be seen as a gauge transformation to
new field configuration \myHighlight{$S(x)$}\coordHE{} which satisfy the ``symmetric gauge''
condition
\begin{equation}\coord{}\boxEquation{
\label{symgauge}
\chi_a(S):= \varepsilon_{abc}\, S_{bc} = 0\,.
}{
\chi_a(S):= \varepsilon_{abc}\, S_{bc} = 0\,.
}{ecuacion}\coordE{}\end{equation}
The complete analysis of the existence and uniqueness of this gauge,
i.e. whether any gauge potential \myHighlight{$A_{ai}$}\coordHE{} can be made symmetric by a
unique gauge transformation, is complex mathematical problem.
Here we shall consider transformation (\ref{eq:gpottr}) in a region where
the uniqueness and regularity of the change of coordinates
can be guaranteed.
In Appendix B, we proof the existence and uniqueness of the symmetric
gauge for the case of a non-degenerate matrix \myHighlight{$A$}\coordHE{}
using the inverse coupling constant expansion.
Furthermore, as an illustration of the obstruction of the uniqueness of
the symmetric gauge-fixing (appearance of Gribov copies) for degenerate
matrices \myHighlight{$A$}\coordHE{}, the Wu-Yang monopole configuration, is considered.
Although it is antisymmetric in space and color indices,
it can be brought into the symmetric form, but there exist two
gauge transformations by which this can be achieved.
The case of degenerate matrix field \myHighlight{$S$}\coordHE{}, \myHighlight{$\det\|S\|=0$}\coordHE{}, will be discussed
for the special situation \myHighlight{$\mbox{rank}\|S\|=1$}\coordHE{} in Section VI.

The transformation (\ref{eq:gpottr}) induces a point canonical
transformation linear in the new momenta \myHighlight{$P_{ik}(x)$}\coordHE{} and \myHighlight{$p_i(x)$}\coordHE{},
conjugated to \myHighlight{$S_{ik}(x)$}\coordHE{} and \myHighlight{$q_i (x)$}\coordHE{}, respectively.
Their expressions in terms of the old variables
\myHighlight{$(A_{ai}(x)\,, \Pi_{ai}(x))$}\coordHE{} can be obtained from the requirement
of the canonical invariance of the symplectic 1-form
\begin{equation}\coord{}\boxEquation{
\label{eq:can1f}
\sum^3_{i, a = 1 }\, \Pi_{ai}\, \dot{A}_{ai}\, dt  =
\sum^3_{i, j = 1}\, P_{ij}\, \dot{S}_{ij}\, dt  +
\,\sum^3_{i = 1} \, p_i\, \dot{q}_i\, dt\,,
}{
\sum^3_{i, a = 1 }\, \Pi_{ai}\, \dot{A}_{ai}\, dt  =
\sum^3_{i, j = 1}\, P_{ij}\, \dot{S}_{ij}\, dt  +
\,\sum^3_{i = 1} \, p_i\, \dot{q}_i\, dt\,,
}{ecuacion}\coordE{}\end{equation}
with the fundamental brackets
\begin{eqnarray}\coord{}\boxAlignEqnarray{\leftCoord{}
\label{eq:Diracb}
&&\leftCoord{} \rightCoord{}
\leftCoord{}\{ S_{i j}(t, \vec{x}) \rightCoord{}\,, P_{k l}(t, \vec{y}) \} = \frac{\leftCoord{}1}{\rightCoord{}2}\rightCoord{}\,
\left(\delta_{i k}\rightCoord{}\, \delta_{j l} + \delta_{i l}\rightCoord{}\, \delta_{j k} \right)\rightCoord{}\,
\delta^{(3)}(\vec{x} - \vec{y})\rightCoord{}\,,\rightCoord{}\\
&&\leftCoord{} \rightCoord{}
\leftCoord{}\{ q_i(t, \vec{x}) \rightCoord{}\,,  p_j (t, \vec{y})\} =
\delta_{i j}\rightCoord{}\, \delta^{(3)}(\vec{x} - \vec{y})~,
\rightCoord{}}{0mm}{6}{14}{
&& 
\{ S_{i j}(t, \vec{x}) \,, P_{k l}(t, \vec{y}) \} = \frac{1}{2}\,
\left(\delta_{i k}\, \delta_{j l} + \delta_{i l}\, \delta_{j k} \right)\,
\delta^{(3)}(\vec{x} - \vec{y})\,,\\
&& 
\{ q_i(t, \vec{x}) \,,  p_j (t, \vec{y})\} =
\delta_{i j}\, \delta^{(3)}(\vec{x} - \vec{y})~,
}{1}\coordE{}\end{eqnarray}
for the new canonical
pairs \myHighlight{$\left(S_{ij}(x)\,, P_{ij}(x)\right)$}\coordHE{} and
\myHighlight{$\left(q_{i}(x)\,, p_{i}(x)\right)$}\coordHE{}.
The brackets (\ref{eq:Diracb}) account for the second class symmetry-constraints
\myHighlight{$S_{ij} = S_{ji}$}\coordHE{} and \myHighlight{$P_{ij} = P_{ji}$}\coordHE{} and therefore are Dirac brackets.
As result we obtain the expression
\begin{equation}\coord{}\boxEquation{
\label{eq:elpotn}
\Pi_{ai} = O_{ak}(q) \biggl[\,
P_{\ ki} + g \, \varepsilon _{kin}{}^\ast\! D^{-1}_{nm}(S)
\left( {\cal S}_m - \Omega^{-1}_{jm} p_j\right) \,
\biggr]\,,
}{
\Pi_{ai} = O_{ak}(q) \biggl[\,
P_{\ ki} + g \, \varepsilon _{kin}{}^\ast\! D^{-1}_{nm}(S)
\left( {\cal S}_m - \Omega^{-1}_{jm} p_j\right) \,
\biggr]\,,
}{ecuacion}\coordE{}\end{equation}
of the old momenta \myHighlight{$\Pi_{ai}$}\coordHE{} in terms of the new canonical variables,
(for a detailed derivation see \cite{KP}).
Here \myHighlight{${}^\ast\! D_{mn}^{-1}(S)$}\coordHE{} denotes the inverse of
the differential matrix operator
\footnote{Note that the operator \myHighlight{${}^\ast\! D_{mn}(S)$}\coordHE{}
 corresponds in the conventional gauge-fixing
method to the so-called Faddeev-Popov operator (FP),
the matrix of Poisson brackets between the Gauss law constraint
(\ref{eq:secconstr}) and  the symmetric gauge (\ref{symgauge}),
 \myHighlight{$\{ \left(D_i(S)\right)_{mc} \Pi_{ci}(x), \chi_n (y)\}=
 {}^\ast\! D_{mn}(S)\delta^3(x-y).$}\coordHE{} }
\begin{equation}\coord{}\boxEquation{
\label{DeltaQ}
{}^\ast\! D_{mn}(S)  =  \varepsilon_{njc}\left(D_j(S)\right)_{mc}\,,
}{
{}^\ast\! D_{mn}(S)  =  \varepsilon_{njc}\left(D_j(S)\right)_{mc}\,,
}{ecuacion}\coordE{}\end{equation}
the vector \myHighlight{${\cal S}$}\coordHE{} is defined as
\begin{equation}\coord{}\boxEquation{
{\cal S}_m   = \frac{1}{g} \, (D_j(S))_{mn}P_{nj}~,
}{
{\cal S}_m   = \frac{1}{g} \, (D_j(S))_{mn}P_{nj}~,
}{ecuacion}\coordE{}\end{equation}
and the matrix \myHighlight{$\Omega^{-1}$}\coordHE{} the inverse of
\begin{equation}\coord{}\boxEquation{
\label{Omega}
\Omega_{ni}(q) \, : =
\,-\frac{1}{2}\, \varepsilon_{nbc}\,
\left( O^T(q)\, \frac{\partial O(q)}{\partial q_i}\right)_{bc}.
}{
\Omega_{ni}(q) \, : =
\,-\frac{1}{2}\, \varepsilon_{nbc}\,
\left( O^T(q)\, \frac{\partial O(q)}{\partial q_i}\right)_{bc}.
}{ecuacion}\coordE{}\end{equation}

Here we would like to comment on the geometrical meaning of the above expressions.
The vector \myHighlight{${\cal S} $}\coordHE{} coincides up to divergence with the spin density part
of the Noetherian angular momentum after projection to the surface given by
the Gauss law constraints.
Furthermore, the matrix \myHighlight{$\Omega^{-1}$}\coordHE{} defines the main geometrical
structures on \myHighlight{$SO(3,R)$}\coordHE{} group manifold, namely the three
left-invariant Killing vectors fields
\myHighlight{$\eta_a :=\Omega^{-1}_{ja} \frac{\partial}{\partial q_j}$}\coordHE{} obeying the
\myHighlight{$so(3)$}\coordHE{} algebra \myHighlight{$[ \eta_a, \eta_b]= \epsilon_{abc}\eta_c$}\coordHE{},
and the invariant  metric \myHighlight{$g:=-\mbox{tr}\left(O^TdO O^TdO\right)=
(1/2)\left(\Omega^T\Omega\right)_{ij}dq_idq_j$}\coordHE{}
as the standard metric on \myHighlight{$S^3$}\coordHE{}.
Since \myHighlight{$\det\Omega$}\coordHE{} is proportional to the Haar measure
on \myHighlight{$SO(3, R)$}\coordHE{}  \myHighlight{$\sqrt{\det \|g\|} =|\det\|\Omega(q)\||$}\coordHE{},
and it is expected to vanish at certain coordinate singularities
(see also e.g. discussion in  ch. 8 of \cite{Creutz}).
In deriving the expression (\ref{eq:elpotn}) we shall here limit ourselves to
the region where the matrix \myHighlight{$\Omega$}\coordHE{} is invertible.

The main advantage of introducing the variables \myHighlight{$S_{ij}$}\coordHE{} and \myHighlight{$q_i$}\coordHE{}
is, that they Abelianise the non-Abelian Gauss law constraints
(\ref{eq:secconstr}).
In terms of the new variables the Gauss's law constraints
\begin{equation}\coord{}\boxEquation{
\label{Omega-1}
g \, O_{as}(q)\, \Omega^{-1}_{\ is}(q)\, p_i = 0\,,
}{
g \, O_{as}(q)\, \Omega^{-1}_{\ is}(q)\, p_i = 0\,,
}{ecuacion}\coordE{}\end{equation}
depend only on \myHighlight{$(q_i,p_i)$}\coordHE{}, showing
that the variables \myHighlight{$(S_{ij} \,, P_{ij})$}\coordHE{} are gauge-invariant, physical fields.
Hence, assuming \myHighlight{$\det\Omega(q)\ne 0$}\coordHE{}\
in (\ref{eq:elpotn}) and (\ref{Omega-1}),
the reduced Hamiltonian, defined as the projection of the
total Hamiltonian onto the constraint shell,
can be obtained from (\ref{eq:tothamn}) by
imposing the equivalent set of Abelian constraints
\begin{equation}\coord{}\boxEquation{
\label{p_a=0}
 p_i = 0\,.
}{
p_i = 0\,.
}{ecuacion}\coordE{}\end{equation}
Due to gauge invariance, the reduced Hamiltonian is
independent of the coordinates \myHighlight{$q_i$}\coordHE{}  canonically conjugated
to \myHighlight{$p_i$}\coordHE{} and is hence a function  of the unconstrained
gauge-invariant variables \myHighlight{$S_{ij}$}\coordHE{} and \myHighlight{$P_{ij}$}\coordHE{} only
\begin{equation}\coord{}\boxEquation{
\label{eq:uncYME}
H = \int d^3{x}\, \biggl[\,
\frac{1}{2}\, \left( P_{ai} - \frac{\theta}{8 \pi^2}\, B^{(+)}_{ai}(S)
\right)^2 +
\left( P_a - \frac{\theta}{8 \pi^2}\, B^{(-)}_{a}(S)\right)^2 +
\frac{1}{2}\, V(S)\,
\biggr]\,.
}{
H = \int d^3{x}\, \biggl[\,
\frac{1}{2}\, \left( P_{ai} - \frac{\theta}{8 \pi^2}\, B^{(+)}_{ai}(S)
\right)^2 +
\left( P_a - \frac{\theta}{8 \pi^2}\, B^{(-)}_{a}(S)\right)^2 +
\frac{1}{2}\, V(S)\,
\biggr]\,.
}{ecuacion}\coordE{}\end{equation}
Here the \myHighlight{$P_a$}\coordHE{} denotes the nonlocal functional,
according to (\ref{eq:elpotn}) defined as solution of the system of
differential equations
\begin{equation}\coord{}\boxEquation{
\label{vecE}
{}^\ast\! D_{ks}(S) P_s = (D_j(S))_{kn}P_{nj}\,.
}{
{}^\ast\! D_{ks}(S) P_s = (D_j(S))_{kn}P_{nj}\,.
}{ecuacion}\coordE{}\end{equation}
The nonlocal second term in the Hamiltonian (\ref{eq:uncYME})
therefore stems from the antisymmetric part of the
\myHighlight{$\Pi_{ai}$}\coordHE{}, which remains after implementing Gauss's law \myHighlight{$p_a = 0$}\coordHE{},
in terms of the physical \myHighlight{$P_{ai}$}\coordHE{}. Hence this term contains \myHighlight{$FP^{-2}$}\coordHE{},
see (26), and is the analogon of the
well-known non-local part of Hamiltonian in the Coulomb gauge,
see e.g. \cite{ChrLee}.

Furthermore,
\begin{equation}\coord{}\boxEquation{
\label{symasymB}
B^{(+)}_{ai}(S) := \frac{1}{2}\,(B_{ai}(S) + B_{ia}(S))\,, \qquad
B^{(-)}_a (S):= \frac{1}{2}\,\varepsilon_{abc} \, B_{bc}(S)~,
}{
B^{(+)}_{ai}(S) := \frac{1}{2}\,(B_{ai}(S) + B_{ia}(S))\,, \qquad
B^{(-)}_a (S):= \frac{1}{2}\,\varepsilon_{abc} \, B_{bc}(S)~,
}{ecuacion}\coordE{}\end{equation}
denote the symmetric and antisymmetric
parts of the reduced chromomagnetic field
\begin{equation}\coord{}\boxEquation{
\label{redB}
B_{ai}(S) = \varepsilon_{ijk}\,
\left(\partial_j S_{ak} + \frac{g}{2}\,
\varepsilon_{abc}\, S_{bj}\, S_{ck}\right)\,.
}{
B_{ai}(S) = \varepsilon_{ijk}\,
\left(\partial_j S_{ak} + \frac{g}{2}\,
\varepsilon_{abc}\, S_{bj}\, S_{ck}\right)\,.
}{ecuacion}\coordE{}\end{equation}
It is the same functional of the symmetric field \myHighlight{$S$}\coordHE{} as the original
\myHighlight{$B_{ai}(A)$}\coordHE{}, since the chromomagnetic field transforms homogeneously
under the change of coordinates (\ref{eq:gpottr}).
Finally the potential \myHighlight{$V(S)$}\coordHE{} is the square of the reduced magnetic field
(\ref{redB}),
\begin{equation}\coord{}\boxEquation{
\label{V(S)}
V(S)\ d^3x = \frac{1}{2} \ \mbox{tr}\,{}^\ast\! F^{(3)}\,\wedge \, F^{(3)}~,
}{
V(S)\ d^3x = \frac{1}{2} \ \mbox{tr}\,{}^\ast\! F^{(3)}\,\wedge \, F^{(3)}~,
}{ecuacion}\coordE{}\end{equation}
with the curvature 2-form in 3-dimensional Euclidean space
\begin{equation}\coord{}\boxEquation{
\label{3F}
 F^{(3)} = d S + S\, \wedge\, S\,,
}{
F^{(3)} = d S + S\, \wedge\, S\,,
}{ecuacion}\coordE{}\end{equation}
in terms of the symmetric 1-form
\begin{equation}\coord{}\boxEquation{
\label{3S}
S = g\tau_k\, S_{kl} \, dx_l, \qquad k, l = 1,2,3\,,
}{
S = g\tau_k\, S_{kl} \, dx_l, \qquad k, l = 1,2,3\,,
}{ecuacion}\coordE{}\end{equation}
whose 6 components depend on the time variable as an external parameter.
The reduced chromomagnetic field (\ref{redB}) is given in terms of the
dual field strength \myHighlight{${}^\ast\! F^{(3)}$}\coordHE{} as
\myHighlight{$B_{ai}(S) =\frac{1}{2}\, \varepsilon_{ijk}\, F^{(3)}_{\ ajk}\,$}\coordHE{}.


\subsection{Canonical equivalence of unconstrained theories with \\
 different \myHighlight{$\theta$}\coordHE{}-angles}

\label{subsecIIB}

For the original degenerate action in terms of the \myHighlight{$A_{\mu}$}\coordHE{} fields
the equivalence of classical theories
with arbitrary value of \myHighlight{$\theta$}\coordHE{}-angle has been reviewed in
Section \ref{sec:cedg}.
Let us now examine the same problem for the derived unconstrained
theory considering the analog of the canonical transformation
(\ref{eq:cantrtheta}) after projection onto the constraint surface,
\begin{eqnarray}\coord{}\boxAlignEqnarray{\leftCoord{}
S_{ai}(x) & \longmapsto & S_{ai}(x) \rightCoord{}\,, \nn\rightCoord{}\\\leftCoord{}
P_{bj}(x) & \longmapsto & {\cal E}_{bj}(x) :=
P_{bj}(x) - \frac{\leftCoord{}\theta}{\rightCoord{}8\rightCoord{}\,\pi^2}\rightCoord{}\, B^{(+)}_{bj}(x)\rightCoord{}\label{eq:uncantrtheta}\rightCoord{}\,.\rightCoord{}
\rightCoord{}}{0mm}{3}{10}{
S_{ai}(x) & \longmapsto & S_{ai}(x) \,, \nn\\
P_{bj}(x) & \longmapsto & {\cal E}_{bj}(x) :=
P_{bj}(x) - \frac{\theta}{8\,\pi^2}\, B^{(+)}_{bj}(x)\,.
}{1}\coordE{}\end{eqnarray}
One can easily check that this transformation
to new variables \myHighlight{$S_{ai}$}\coordHE{} and \myHighlight{${\cal E}_{bj}$}\coordHE{}
is canonical with respect to the Dirac brackets (\ref{eq:Diracb}).
In terms of the new variables \myHighlight{$S_{ai}$}\coordHE{} and \myHighlight{${\cal E}_{bj}$}\coordHE{} the
Hamiltonian (\ref{eq:uncYME}) can be written as
\begin{equation}\coord{}\boxEquation{
\label{eq:ht1}
H =
\int d^3 x
\biggl[\,
\frac{1}{2}\, {\cal E}_{ai}^2 + {\cal E}_{a}^2 + \frac{1}{2}\, V(S) \,
\biggr]\,,
}{
H =
\int d^3 x
\biggl[\,
\frac{1}{2}\, {\cal E}_{ai}^2 + {\cal E}_{a}^2 + \frac{1}{2}\, V(S) \,
\biggr]\,,
}{ecuacion}\coordE{}\end{equation}
with \myHighlight{${\cal E}_a$}\coordHE{}  defined as
\begin{equation}\coord{}\boxEquation{
{\cal E}_a := P_a - \frac{\theta}{8\, \pi^2} \, B^{(-)}_{a}\,.
}{
{\cal E}_a := P_a - \frac{\theta}{8\, \pi^2} \, B^{(-)}_{a}\,.
}{ecuacion}\coordE{}\end{equation}
Now, if \myHighlight{$P_a$}\coordHE{} is a solution of equation (\ref{vecE}), then
\myHighlight{${\cal E}_a$}\coordHE{} is a solution of the same equation
\begin{equation}\coord{}\boxEquation{
\label{vecE"}
{}^\ast\! D_{ks}(S){\cal E} _s = (D_j(S))_{kn}{\cal E}_{nj}\,,
}{
{}^\ast\! D_{ks}(S){\cal E} _s = (D_j(S))_{kn}{\cal E}_{nj}\,,
}{ecuacion}\coordE{}\end{equation}
with the replacement \myHighlight{$P_{ai} \longmapsto {\cal E}_{ai}$}\coordHE{},
since the reduced field \myHighlight{$B_{ai}$}\coordHE{} satisfies the Bianchi identity
\begin{equation}\coord{}\boxEquation{
\label{BI}
(D_i(S))_{ab} \, B_{bi}(S) = 0\,.
}{
(D_i(S))_{ab} \, B_{bi}(S) = 0\,.
}{ecuacion}\coordE{}\end{equation}
Hence we arrive at the same unconstrained Hamiltonian system
(\ref{eq:ht1}) and (\ref{vecE"}) with vanishing \myHighlight{$\theta$}\coordHE{}-angle.
Note that after the elimination of the three unphysical fields \myHighlight{$q_j(x)$}\coordHE{}
the projected canonical transformation (\ref{eq:uncantrtheta})
that removes the \myHighlight{$\theta$}\coordHE{}-dependence from the Hamiltonian can be written as
\begin{equation}\coord{}\boxEquation{
{\cal E}_{bj}(x) =
P_{bj}(x) - \theta \, \frac{\delta}{\delta S_{bj}}\, W[S]\,,
}{
{\cal E}_{bj}(x) =
P_{bj}(x) - \theta \, \frac{\delta}{\delta S_{bj}}\, W[S]\,,
}{ecuacion}\coordE{}\end{equation}
which is of the same form as (\ref{clctr}) with the nine gauge fields
\myHighlight{$A_{ik}(x)$}\coordHE{} replaced by the six unconstrained fields \myHighlight{$S_{ik}(x)$}\coordHE{}.

In summary, the exact projection to reduced phase space leads to an
unconstrained system,
whose equations of motion are consistent with the original degenerate theory
in the sense that they are \myHighlight{$\theta$}\coordHE{}-independent.
Thus if our consideration is restricted only to
the classical level of the exact nonlocal unconstrained theory, the
generalization to arbitrary \myHighlight{$\theta$}\coordHE{}-angle can be avoided\footnote{
The extension of the proof of \myHighlight{$\theta$}\coordHE{}-independence to
quantum theory requires to show the unitarity of
the operator corresponding to transformation
(\ref{eq:uncantrtheta}).}.
However, in order to work with such a complicated nonlocal
Hamiltonian it is necessary to make approximations, such as for example
expansion in the number of spatial derivatives, which we shall carry
out in the next section.
For these one has to check that this approximation
is free of the "divergence problem", that is all terms in the corresponding
truncated action containing the \myHighlight{$\theta$}\coordHE{}-angle can be collected into a
4-divergence and all dependence on \myHighlight{$\theta$}\coordHE{}
disappears from the classical equations of motion.

%%%%%%%%%%%%%%%%%%%%%%%%%%%%%%%%%%%%%% 4 %%%%%%%%%%%%%%%%%%%%%%%%%%%%%%%%%%%%%%%

\section{Expansion of the unconstrained Hamiltonian in \myHighlight{$1/g$}\coordHE{}}

%%%%%%%%%%%%%%%%%%%%%%%%%%%%%%%%%%%%%%%%%%%%%%%%%%%%%%%%%%%%%%%%%%%%%%%%%%%%%%%%

\label{secIV}

Let us now consider the regime when the unconstrained
fields are slowly varying in space-time and expand
the nonlocal part of the kinetic term in the unconstrained
Hamiltonian (\ref{eq:uncYME}) as a series of terms with increasing
powers of inverse coupling constant \myHighlight{$1/g$}\coordHE{},
equivalent to an expansion in the
number of spatial derivatives of field and momentum.
Our expansion is purely formal and we shall in this work not
study the question of its convergence.
We shall see, that for nonvanishing \myHighlight{$\theta$}\coordHE{}-angle,
a straightforward expansion in \myHighlight{$1/g$}\coordHE{} leads to the above mentioned
"divergence problem", and suggest an
improved form of the expansion in \myHighlight{$1/g$}\coordHE{} of the unconstrained Hamiltonian
exploiting the Bianchi identity.


\subsection{Divergence problem in lowest-order approximation}

\label{Sec:IV1}

According to \cite{KP}, the nonlocal funtional \myHighlight{$P_a$}\coordHE{} in the
unconstrained Hamiltonian (\ref{eq:ht1}), defined as solution of the system of
linear differential equations (\ref{vecE}), can formally be expanded
in powers of \myHighlight{$1/g$}\coordHE{}.
The vector \myHighlight{$P_a$}\coordHE{} is then given as a sum of terms containing an increasing
number of spatial derivatives of field and momentum
\begin{equation}\coord{}\boxEquation{
\label{vecEmexp}
P_s (S, P) = \sum_{n=0}^{\infty}(1/g)^n\, a_s^{(n)}(S, P)\,.
}{
P_s (S, P) = \sum_{n=0}^{\infty}(1/g)^n\, a_s^{(n)}(S, P)\,.
}{ecuacion}\coordE{}\end{equation}
The zeroth-order term is
\begin{equation}\coord{}\boxEquation{
\label{vecE1}
a^{(0)}_{s} =
\gamma^{-1}_{sk}\varepsilon_{klm}\left(PS\right)_{lm}\,,
}{
a^{(0)}_{s} =
\gamma^{-1}_{sk}\varepsilon_{klm}\left(PS\right)_{lm}\,,
}{ecuacion}\coordE{}\end{equation}
with \myHighlight{$\gamma_{ik}:= S_{ik} - \delta_{ik}\, \mbox{tr}\, S$}\coordHE{},
and the first-order term is determined as
\begin{equation}\coord{}\boxEquation{
\label{vecE2}
a^{(1)}_{s} = - \, \gamma^{-1}_{sl}\,
\left[
(\mbox{rot}\ \vec{a}^{(0)})_l + \partial_k P_{kl}
\right]
}{
a^{(1)}_{s} = - \, \gamma^{-1}_{sl}\,
\left[
(\mbox{rot}\ \vec{a}^{(0)})_l + \partial_k P_{kl}
\right]
}{ecuacion}\coordE{}\end{equation}
from the zeroth-order term.
The higher terms are then obtained by the simple recurrence relations
\begin{equation}\coord{}\boxEquation{ \label{vecE3}
a^{(n+1)}_{s} =
- \,\gamma^{-1}_{sl}(\mbox{rot}\, {\vec {a}}^{\ (n)})_l \,.
}{ a^{(n+1)}_{s} =
- \,\gamma^{-1}_{sl}(\mbox{rot}\, {\vec {a}}^{\ (n)})_l \,.
}{ecuacion}\coordE{}\end{equation}
Inserting these expressions into (\ref{eq:uncYME})
we obtain the corresponding expansion of unconstrained Hamiltonian as
a series in higher and higher numbers of derivatives.

Let us check, whether the truncation of the expansion
(\ref{vecEmexp}) to lowest order is consistent with \myHighlight{$\theta$}\coordHE{}-independence,
that is, whether all \myHighlight{$\theta$}\coordHE{}-dependent
terms can be collected into 4-divergence after Legendre
transformation to the corresponding Lagrangian.
In \myHighlight{$o(1/g)$}\coordHE{} approximation (\ref{vecE1}), the Hamiltonian reads\footnote{
When all spatial derivatives of the
fields and momenta are neglected, Yang-Mills theory reduces to the so-called
Yang-Mills mechanics and its \myHighlight{$\theta$}\coordHE{}-independence has been shown
in \cite{AHG}.
}
\begin{equation}\coord{}\boxEquation{
\label{eq:ham2}
H^{(2)} =
\int d^3{x}\,
\biggl[\,
\, \frac{1}{2}\mbox{tr}\,\left(P - \frac{\theta}{8 \pi^2}\, B^{(+)}\right)^2
+\left(a_s^{(0)}(S, P) - \frac{\theta}{8\, \pi^2}\, B^{(-)}_s
\right)^2 + \,\frac{1}{2} V(S) \ \biggr]\,,
}{
H^{(2)} =
\int d^3{x}\,
\biggl[\,
\, \frac{1}{2}\mbox{tr}\,\left(P - \frac{\theta}{8 \pi^2}\, B^{(+)}\right)^2
+\left(a_s^{(0)}(S, P) - \frac{\theta}{8\, \pi^2}\, B^{(-)}_s
\right)^2 + \,\frac{1}{2} V(S) \ \biggr]\,,
}{ecuacion}\coordE{}\end{equation}
where \myHighlight{$B^{(+)}$}\coordHE{} and \myHighlight{$B^{(-)}$}\coordHE{} denote the symmetric and antisymmetric
parts of the chromomagnetic field, defined in (\ref{symasymB}).

After inverse Legendre transformation of the Hamiltonian (\ref{eq:ham2}),
the \myHighlight{$\theta$}\coordHE{}-dependent terms in the corresponding Lagrangian
cannot be collected to a total 4-divergence, as is shown in Appendix C,
and therefore contribute to the unconstrained equations of motion.
Hence applying a straightforward derivative expansion to the
Yang-Mills theory with topological term after projection to reduced phase
space we face the "divergence problem" dicussed above.


\subsection{Improved \myHighlight{$1/g$}\coordHE{} expansion using the Bianchi identity}

\label{SECTIONIV2}

In order to avoid the ``divergence problem''  one can proceed as follows.
Let us consider additionally to the differential equation (\ref{vecE}),
which determines the nonlocal term \myHighlight{$P_a$}\coordHE{}, the Bianchi identity (\ref{BI})
as an equation for determination of the antisymmetric part \myHighlight{$B^{(-)}_s$}\coordHE{}
of the chromomagnetic field
\begin{equation}\coord{}\boxEquation{
\label{eq:abm}
^{\ast}\! D_{ks}(S)\, B^{(-)}_s = (D_i(S))_{kl}\, B^{(+)}_{li}\,,
}{
^{\ast}\! D_{ks}(S)\, B^{(-)}_s = (D_i(S))_{kl}\, B^{(+)}_{li}\,,
}{ecuacion}\coordE{}\end{equation}
in terms of its symmetric part \myHighlight{$B^{(+)}_{bc}$}\coordHE{}.
The complete analogy of this equation with (\ref{vecE}) expresses the
duality of chromoelectric and chromagnetic fields on the unconstrained level.
Hence one can write
\begin{equation}\coord{}\boxEquation{
\label{eq:BG}
^{\ast}\! D_{ks}(S)\,
\left[P_s - \frac{\theta}{8 \pi^2}\, B^{(-)}_s \right] =
(D_i(S))_{kl}\,
\left[ P_{li} - \frac{\theta}{8\, \pi^2}\, B^{(+)}_{li}\right]\,.
}{
^{\ast}\! D_{ks}(S)\,
\left[P_s - \frac{\theta}{8 \pi^2}\, B^{(-)}_s \right] =
(D_i(S))_{kl}\,
\left[ P_{li} - \frac{\theta}{8\, \pi^2}\, B^{(+)}_{li}\right]\,.
}{ecuacion}\coordE{}\end{equation}
Using the same type of the spatial derivative expansion as before in
(\ref{vecE1})-(\ref{vecE3}), we obtain
\begin{equation}\coord{}\boxEquation{
\label{P-B-}
P_s - \frac{\theta}{8\, \pi^2}\, B^{(-)}_s =
\sum_{n = 0}^{\infty}\,(1/g)^n\,
a^{(n)}_s(S, P - \frac{\theta}{8 \pi^2} \, B^{(+)})\,.
}{
P_s - \frac{\theta}{8\, \pi^2}\, B^{(-)}_s =
\sum_{n = 0}^{\infty}\,(1/g)^n\,
a^{(n)}_s(S, P - \frac{\theta}{8 \pi^2} \, B^{(+)})\,.
}{ecuacion}\coordE{}\end{equation}
In this way we achieve a form of the derivative expansion such that the
unconstrained Hamiltonian is a functional of
field combination \myHighlight{$P_{ai} - (\theta/8 \,\pi^2)\, B^{(+)}_{ai}$}\coordHE{}
\begin{equation}\coord{}\boxEquation{
\label{eq:uncYMEi}
H =
\int d^3{x}\, \biggl[\,
\frac{1}{2}\, \left(P_{ai} - \frac{\theta}{8\pi^2}\, B^{(+)}_{ai}\right)^2 +
\left(
\sum_{n=0}^{\infty}(1/g)^n\,
a^{(n)}_i(S, P - \frac{\theta}{8 \pi^2}\, B^{(+)})\right)^2
+ \frac{1}{2}\, V(S) \, \biggr]\,,
}{
H =
\int d^3{x}\, \biggl[\,
\frac{1}{2}\, \left(P_{ai} - \frac{\theta}{8\pi^2}\, B^{(+)}_{ai}\right)^2 +
\left(
\sum_{n=0}^{\infty}(1/g)^n\,
a^{(n)}_i(S, P - \frac{\theta}{8 \pi^2}\, B^{(+)})\right)^2
+ \frac{1}{2}\, V(S) \, \biggr]\,,
}{ecuacion}\coordE{}\end{equation}
explicitly showing the chromoelectro-magnetic duality on the reduced level
and hence free of the "divergence problem".
To obtain the unconstrained Hamiltonian up to leading order \myHighlight{$o(1/g)$}\coordHE{},
only the lowest term \myHighlight{$a^{(0)}_s(S, P - (\theta/8 \,\pi^2)\, B^{(+)})$}\coordHE{}
in the sum in (\ref{eq:uncYMEi}) has to be taken into account, so that
\begin{equation}\coord{}\boxEquation{
\label{eq:iham2}
H^{(2)} =
\frac{1}{2}\int d^3{x}\, \biggl[\,
\, \mbox{tr}\, \left( P - \frac{\theta}{8\pi^2}\, B^{(+)}\right)^2 -
\frac{1}{ \det^2 \gamma}\, \mbox{tr}\,
\left(\gamma\, [S, P - \frac{\theta}{8 \,\pi^2}\, B^{(+)}]\, \gamma\right)^2 +
\, V(S)
\biggr]\,.
}{
H^{(2)} =
\frac{1}{2}\int d^3{x}\, \biggl[\,
\, \mbox{tr}\, \left( P - \frac{\theta}{8\pi^2}\, B^{(+)}\right)^2 -
\frac{1}{ \det^2 \gamma}\, \mbox{tr}\,
\left(\gamma\, [S, P - \frac{\theta}{8 \,\pi^2}\, B^{(+)}]\, \gamma\right)^2 +
\, V(S)
\biggr]\,.
}{ecuacion}\coordE{}\end{equation}
The advantage of this Hamiltonian compared with (\ref{eq:ham2}), derived
before, is that the classical equations of motion following from
(\ref{eq:iham2}) are \myHighlight{$\theta$}\coordHE{}-independent.
In order to obtain a transparent
form of the corresponding surface term in the unconstrained action,
it is useful to perform a main-axis transformation of the symmetric
matrix field \myHighlight{$S(x)$}\coordHE{}.


%%%%%%%%%%%%%%%%%%%%%%%%%%%%%%%%%%%%%% 5 %%%%%%%%%%%%%%%%%%%%%%%%%%%%%%%%%%%%%%%

\section{Long-wavelength approximation to reduced theory}

%%%%%%%%%%%%%%%%%%%%%%%%%%%%%%%%%%%%%%%%%%%%%%%%%%%%%%%%%%%%%%%%%%%%%%%%%%%%%%%%

\label{sec:V}

In this section we shall at first rewrite the
unconstrained Hamiltonian (\ref{eq:iham2}) in terms of main-axis variables
of the symmetric tensor field \myHighlight{$S_{ij}$}\coordHE{}.
The corresponding second-order Lagrangian \myHighlight{$L^{(2)}$}\coordHE{} is then obtained
via Legendre transformation and the form of the corresponding
unconstrained total divergence derived in an explicit way.



\subsection{Hamiltonian in terms of main-axis variables}



\label{secIV2}

In \cite{KP} it was shown, that the field \myHighlight{$S_{ij}(x)$}\coordHE{}
transforms as a second-rank tensor under the spatial rotations.
This can be used to explicitly separate the rotational degrees of
freedom from the scalars in the Hamiltonian (\ref{eq:iham2}).
Following \cite{KP} we introduce  the main-axis representation
of the  symmetric \myHighlight{$3 \times 3$}\coordHE{} matrix field \myHighlight{$S(x)$}\coordHE{},
\begin{equation}\coord{}\boxEquation{
\label{eq:mainax}
S (x) =
R^T(\chi(x))
\left(
\begin{array}{ccc}
\phi_1(x)  &   0         &    0       \\
0          & \phi_2(x)   &    0        \\
0          &   0         &   \phi_3(x)
\end{array}
\right )
R(\chi(x))\,.
}{
S (x) =
R^T(\chi(x))
\left(
\begin{array}{ccc}
\phi_1(x)  &   0         &    0       \\
0          & \phi_2(x)   &    0        \\
0          &   0         &   \phi_3(x)
\end{array}
\right )
R(\chi(x))\,.
}{ecuacion}\coordE{}\end{equation}
The Jacobian of this transformation is
\begin{equation}\coord{}\boxEquation{
J\left(\frac{S_{ij}[\phi, \chi]}{\phi_{k}, \chi_{l}}\right) \propto
\prod_{i\neq j}\mid \phi_i(x) - \phi_j(x) \mid~,
}{
J\left(\frac{S_{ij}[\phi, \chi]}{\phi_{k}, \chi_{l}}\right) \propto
\prod_{i\neq j}\mid \phi_i(x) - \phi_j(x) \mid~,
}{ecuacion}\coordE{}\end{equation}
and thus (\ref{eq:mainax}) can be used as definition of
the new configuration variables,
the three diagonal fields \myHighlight{$\phi_1, \phi_2, \phi_3$}\coordHE{} and
the three angular fields \myHighlight{$\chi_1, \chi_2,\chi_3$}\coordHE{},
only if all eigenvalues of the matrix \myHighlight{$S$}\coordHE{} are different.
To have the uniqueness of the inverse transformation we assume here
that
\begin{equation}\coord{}\boxEquation{
\label{princorb}
0 < \phi_1(x) < \phi_2(x) < \phi_3(x)\,.
}{
0 < \phi_1(x) < \phi_2(x) < \phi_3(x)\,.
}{ecuacion}\coordE{}\end{equation}
The variables \myHighlight{$\phi_i$}\coordHE{} in the main-axis transformation (\ref{eq:mainax})
parameterize the orbits of the action of a group element \myHighlight{$g\in\sot$}\coordHE{} on symmetric
matrices, \myHighlight{$S\rightarrow S^\prime = g\, S\, g^{-1} $}\coordHE{}.
The configuration (\ref{princorb}) belongs to the so-called  principle orbit
class,
whereas all orbits with coinciding eigenvalues of the matrix \myHighlight{$S$}\coordHE{} are
singular orbits \cite{ORaf}.
In order to parameterize configurations belonging to a singular stratum
one should in principle use a decomposition of the \myHighlight{$S$}\coordHE{} field different from the
above main-axes transformation (\ref{eq:mainax}).
Alternatively, one can consider the singular orbits as the boundary of the
principle orbit-type stratum and study the corresponding dynamics using a
certain limiting procedure
\footnote{The relation between an explicit parameterization
of the singular strata and their description as a certain limit
of the principle orbit stratum has been studied recently in \cite{AD}
investigating the geodesic motion on the \myHighlight{$GL(n, R)$}\coordHE{} group manifold.}.
In this Section we shall limit ourselves to the consideration of the dynamics
on the principle orbits and leave the important case of the singular orbits
expected to contain interesting physics for future studies.

The momenta \myHighlight{$\pi_i$}\coordHE{} and \myHighlight{$p_{\chi_i}$}\coordHE{}, canonical conjugate
to the diagonal elements \myHighlight{$\phi_i$}\coordHE{} and  \myHighlight{$\chi_i$}\coordHE{}, can be found using
the condition of the canonical invariance of the symplectic 1-form
\begin{equation}\coord{}\boxEquation{
\sum^3_{i,j = 1}\, P_{ij}\, \dot{S}_{ij}\, dt  =
\sum^3_{i = 1}\, \pi_{i}\, \dot{\phi}_{i} dt  +
\sum^3_{i = 1}\, p_{\chi_i}\, \dot{\chi}_i  dt\,.
}{
\sum^3_{i,j = 1}\, P_{ij}\, \dot{S}_{ij}\, dt  =
\sum^3_{i = 1}\, \pi_{i}\, \dot{\phi}_{i} dt  +
\sum^3_{i = 1}\, p_{\chi_i}\, \dot{\chi}_i  dt\,.
}{ecuacion}\coordE{}\end{equation}
The original physical momenta \myHighlight{$P_{ik}$}\coordHE{}, expressed
in terms of the new canonical variables, read
\begin{eqnarray}\coord{}\boxAlignEqnarray{\leftCoord{} \rightCoord{}\label{eq:newmom}
P(x) = \rightCoord{}
R^T(x)\rightCoord{}\, \rightCoord{}
\sum_{\rightCoord{}s = 1}^{\leftCoord{}3}\left(
\pi_s(x) \rightCoord{}\, \overline{\alpha}_s + \frac{\leftCoord{}1}{\rightCoord{}2}{\rightCoord{}\cal P}_s(x) \rightCoord{}\,\alpha_s
\right)\rightCoord{}\, \rightCoord{}
R(x)\rightCoord{}\,. \rightCoord{}
\rightCoord{}}{0mm}{3}{15}{ P(x) = 
R^T(x)\, 
\sum_{s = 1}^{3}\left(
\pi_s(x) \, \overline{\alpha}_s + \frac{1}{2}{\cal P}_s(x) \,\alpha_s
\right)\, 
R(x)\,. 
}{1}\coordE{}\end{eqnarray}
Here \myHighlight{$\overline{\alpha}_i$}\coordHE{} and \myHighlight{$\alpha_i$}\coordHE{}
denote the diagonal and off-diagonal basis elements for symmetric matrices
with the orthogonality relations
\myHighlight{$\mbox{tr}\, (\overline{\alpha}_i \, \overline{\alpha}_j) = \delta_{ij},$}\coordHE{}  \myHighlight{$\mbox{tr}\, ({\alpha}_i \, {\alpha}_j) = 2\, \delta_{ij},$}\coordHE{}  \myHighlight{$\mbox{tr}\, (\overline{\alpha}_i \, {\alpha}_j) = 0,$}\coordHE{}
and
\begin{equation}\coord{}\boxEquation{
\label{P+approx}
{\cal P}_i (x) = - \, \frac{\xi_i(x)}{\phi_j(x) - \phi_k(x)}\,,
\qquad
(\mbox{cyclic permutations} \, i\not=j\not= k )\,.
}{
{\cal P}_i (x) = - \, \frac{\xi_i(x)}{\phi_j(x) - \phi_k(x)}\,,
\qquad
(\mbox{cyclic permutations} \, i\not=j\not= k )\,.
}{ecuacion}\coordE{}\end{equation}
The \myHighlight{$\xi_i$}\coordHE{} are the three \myHighlight{$\sot$}\coordHE{} right-invariant Killing vector fields
given in terms of the angles \myHighlight{$\chi_i$}\coordHE{} and their conjugated momenta
\myHighlight{$p_{\chi_i}$}\coordHE{} via
\begin{equation}\coord{}\boxEquation{
\xi_i = M^{-1}_{ji} p_{\chi_j}\,,
}{
\xi_i = M^{-1}_{ji} p_{\chi_j}\,,
}{ecuacion}\coordE{}\end{equation}
where the matrix \myHighlight{$M$}\coordHE{} is
\begin{equation}\coord{}\boxEquation{ \label{eq:MCr}
M_{ji} := - \frac{1}{2}\, \varepsilon_{jab}
\left(\frac{\partial R}{\partial \chi_i}\, R^T\right)_{ab}\,.
}{ M_{ji} := - \frac{1}{2}\, \varepsilon_{jab}
\left(\frac{\partial R}{\partial \chi_i}\, R^T\right)_{ab}\,.
}{ecuacion}\coordE{}\end{equation}


The physical chromomagnetic field \myHighlight{$B(S)$}\coordHE{} can be regarded
as the components of the curvature 2-form \myHighlight{$F^{(3)}$}\coordHE{}, defined
in terms of the symmetric 1-form \myHighlight{$S$}\coordHE{} in (\ref{3F}).
Starting from the coordinate basis expression of \myHighlight{$S$}\coordHE{} in (\ref{3S}),
we observe that the main-axis transformation (\ref{eq:mainax}) corresponds to
the representation
\begin{equation}\coord{}\boxEquation{
\label{eq:sof}
S =\sum_{a=1}^3 e_a\,\phi_{a} \,\omega_a \,,
}{
S =\sum_{a=1}^3 e_a\,\phi_{a} \,\omega_a \,,
}{ecuacion}\coordE{}\end{equation}
with the 1-forms
\begin{equation}\coord{}\boxEquation{
\omega_i := R_{ij}\, (\chi(x))dx_j \,, \qquad i = 1,2,3
}{
\omega_i := R_{ij}\, (\chi(x))dx_j \,, \qquad i = 1,2,3
}{ecuacion}\coordE{}\end{equation}
and the \myHighlight{$su(2)$}\coordHE{} Lie algebra basis
\begin{equation}\coord{}\boxEquation{
e_a := R_{ab}(\chi(x))\, \tau_b\,, \qquad a = 1,2,3~.
}{
e_a := R_{ab}(\chi(x))\, \tau_b\,, \qquad a = 1,2,3~.
}{ecuacion}\coordE{}\end{equation}
In this basis the components of the non-Abelian field strength \myHighlight{$F^{(3)}$}\coordHE{} read
\begin{equation}\coord{}\boxEquation{
\label{eq:chrm}
F^{(3)}_{aij} =
\delta_{aj}\, X_i\, \phi_j - \delta_{ai}\, X_j\, \phi_i +
\phi_i\, \Gamma_{aji} - \phi_j \, \Gamma_{aij} +
\Gamma_{a[ij]}\, \phi_a +
g\varepsilon_{aij}\, \phi_i \,\phi_j \,,
\quad \mbox{(no summation)}\,,
}{
F^{(3)}_{aij} =
\delta_{aj}\, X_i\, \phi_j - \delta_{ai}\, X_j\, \phi_i +
\phi_i\, \Gamma_{aji} - \phi_j \, \Gamma_{aij} +
\Gamma_{a[ij]}\, \phi_a +
g\varepsilon_{aij}\, \phi_i \,\phi_j \,,
\quad \mbox{(no summation)}\,,
}{ecuacion}\coordE{}\end{equation}
with the components of connection 1-form \myHighlight{$\Gamma$}\coordHE{} defined as
\begin{equation}\coord{}\boxEquation{
\label{Gammaspace}
\Gamma_{aib} := \left(X_i R\, R^T \right)_{ab}\,.
}{
\Gamma_{aib} := \left(X_i R\, R^T \right)_{ab}\,.
}{ecuacion}\coordE{}\end{equation}
The vector fields
\begin{equation}\coord{}\boxEquation{
X_i := R_{ij}\,\partial_j \,,
}{
X_i := R_{ij}\,\partial_j \,,
}{ecuacion}\coordE{}\end{equation}
are dual to the 1-forms \myHighlight{$\omega_j\,, \, \omega_i( X_j) = \delta_{ij}$}\coordHE{},
and act on the basis elements \myHighlight{$e_a$}\coordHE{} as
\begin{equation}\coord{}\boxEquation{
X_i \, e_a = - \Gamma_{bia}e_b\,.
}{
X_i \, e_a = - \Gamma_{bia}e_b\,.
}{ecuacion}\coordE{}\end{equation}
From the expressions (\ref{eq:chrm}) we obtain for the potential (\ref{V(S)})
(see \cite{KP} and Erratum \cite{ErrKP}),
\begin{equation}\coord{}\boxEquation{
\label{Vinhom1}
V(\phi,\chi) = \sum_{i\neq j}^{3}
\left(
\Gamma_{iij} (\phi_i-\phi_j) - X_j \phi_i\right)^2
+ \sum_{cyclic}^3\left(\Gamma_{ijk}(\phi_i-\phi_k) -
\Gamma_{ikj}(\phi_i - \phi_k) - g\phi_j \phi_k\right)^2\,.
}{
V(\phi,\chi) = \sum_{i\neq j}^{3}
\left(
\Gamma_{iij} (\phi_i-\phi_j) - X_j \phi_i\right)^2
+ \sum_{cyclic}^3\left(\Gamma_{ijk}(\phi_i-\phi_k) -
\Gamma_{ikj}(\phi_i - \phi_k) - g\phi_j \phi_k\right)^2\,.
}{ecuacion}\coordE{}\end{equation}

The explicit expressions for the diagonal components \myHighlight{$\beta_i$}\coordHE{} and
the off-diagonal components \myHighlight{$b_i$}\coordHE{} of the
the symmetric part of the chromomagnetic field
\begin{equation}\coord{}\boxEquation{
\label{B+}
B^{(+)} = R^T(\chi)\,
 \sum_{i=1}^3\,\left( \beta_i\overline{\alpha}_i  +
\frac{1}{2}b_i\alpha_i \right)\, R(\chi)\,,
}{
B^{(+)} = R^T(\chi)\,
 \sum_{i=1}^3\,\left( \beta_i\overline{\alpha}_i  +
\frac{1}{2}b_i\alpha_i \right)\, R(\chi)\,,
}{ecuacion}\coordE{}\end{equation}
are given in terms of the diagonal fields \myHighlight{$\phi_i$}\coordHE{} and
the angular fields \myHighlight{$\chi_i$}\coordHE{} in the cyclic form
\begin{eqnarray}\coord{}\boxAlignEqnarray{
&&\leftCoord{} \rightCoord{}
\beta_i =
g\phi_j\phi_k - (\phi_i-\phi_j)\Gamma_{ikj} +
\leftCoord{}(\phi_i-\phi_k)\Gamma_{ijk}\rightCoord{}\,,\rightCoord{}\\
&&\leftCoord{} \rightCoord{}
b_i=
X_i(\phi_j-\phi_k)-(\phi_i-\phi_j)\Gamma_{ijj} +
\leftCoord{}(\phi_i-\phi_k)\Gamma_{ikk} \rightCoord{}\,.\rightCoord{}
\label{b_i}
\rightCoord{}}{0mm}{4}{8}{
&& 
\beta_i =
g\phi_j\phi_k - (\phi_i-\phi_j)\Gamma_{ikj} +
(\phi_i-\phi_k)\Gamma_{ijk}\,,\\
&& 
b_i=
X_i(\phi_j-\phi_k)-(\phi_i-\phi_j)\Gamma_{ijj} +
(\phi_i-\phi_k)\Gamma_{ikk} \,.
}{1}\coordE{}\end{eqnarray}
and the antisymmetric part \myHighlight{$B^{(-)}_i$}\coordHE{} of the unconstrained magnetic field is
\begin{equation}\coord{}\boxEquation{
\label{B-}
B^{(-)}_i=\frac{1}{2}\sum_{cyclic}^3 R^T_{ia}\left[X_a(\phi_b+\phi_c)+
(\phi_b-\phi_a)\Gamma_{abb}+(\phi_c-\phi_a)\Gamma_{acc}\right] \,.
}{
B^{(-)}_i=\frac{1}{2}\sum_{cyclic}^3 R^T_{ia}\left[X_a(\phi_b+\phi_c)+
(\phi_b-\phi_a)\Gamma_{abb}+(\phi_c-\phi_a)\Gamma_{acc}\right] \,.
}{ecuacion}\coordE{}\end{equation}
The zeroth-order term of the expansion (\ref{P-B-}), finally, reads
\begin{equation}\coord{}\boxEquation{
\label{P-approx}
a^{(0)}_i =
-\frac{1}{2}\sum_{cyclic}^3 \frac{R^T_{ia}}{(\phi_b + \phi_c)}\left(
\xi_a + \frac{\theta}{8\pi^2}\, (\phi_b - \phi_c)\,b_{a}
\,\right)
}{
a^{(0)}_i =
-\frac{1}{2}\sum_{cyclic}^3 \frac{R^T_{ia}}{(\phi_b + \phi_c)}\left(
\xi_a + \frac{\theta}{8\pi^2}\, (\phi_b - \phi_c)\,b_{a}
\,\right)
}{ecuacion}\coordE{}\end{equation}
Altogether, the \myHighlight{$o(1/g)$}\coordHE{} Hamiltonian (\ref{eq:iham2}), as a functional
of main-axis variables, becomes
\begin{equation}\coord{}\boxEquation{
\label{eq:unch2}
H^{(2)}  =
\frac{1}{2}\, \int d^3x
\left[
\sum_{i=1}^3
\left(\pi_i - \frac{\theta}{8\pi^2}\, {\beta}_i\right)^2  +
\sum_{cyclic}\, k_i \left(
\xi_i + \frac{\theta}{8\pi^2}\,(\phi_j - \phi_k)\,b_{i}\,\right)^2
+  V(\phi,\chi)
\right]\,,
}{
H^{(2)}  =
\frac{1}{2}\, \int d^3x
\left[
\sum_{i=1}^3
\left(\pi_i - \frac{\theta}{8\pi^2}\, {\beta}_i\right)^2  +
\sum_{cyclic}\, k_i \left(
\xi_i + \frac{\theta}{8\pi^2}\,(\phi_j - \phi_k)\,b_{i}\,\right)^2
+  V(\phi,\chi)
\right]\,,
}{ecuacion}\coordE{}\end{equation}
with
\begin{equation}\coord{}\boxEquation{ \label{eq:km}
k_i := \frac{\phi_j^2 + \phi_k^2}{(\phi_j^2 - \phi_k^2)^2} \,, \quad
(\mbox{cyclic permutations}\,\,  i\not = j\not = k )\,.
}{ k_i := \frac{\phi_j^2 + \phi_k^2}{(\phi_j^2 - \phi_k^2)^2} \,, \quad
(\mbox{cyclic permutations}\,\,  i\not = j\not = k )\,.
}{ecuacion}\coordE{}\end{equation}

The transformation (\ref{eq:uncantrtheta}), rewritten
in terms of angular and scalar variables,
\begin{eqnarray}\coord{}\boxAlignEqnarray{\leftCoord{}
\label{eq:crt}
&&\leftCoord{} \rightCoord{}
\pi_i \longmapsto \pi_i + \frac{\leftCoord{}\theta}{\rightCoord{}8\pi^2}\rightCoord{}\,\beta_i \rightCoord{}\,, \qquad
\phi_i \longmapsto \phi_i \rightCoord{}\,, \nn\rightCoord{}\\
&&\leftCoord{} \rightCoord{}
\xi_i \longmapsto \xi_i - \frac{\leftCoord{}\theta}{\rightCoord{}8\pi^2}\rightCoord{}\,(\phi_j - \phi_k)\rightCoord{}\,b_i \rightCoord{}\,,
\rightCoord{}}{0mm}{5}{13}{
&& 
\pi_i \longmapsto \pi_i + \frac{\theta}{8\pi^2}\,\beta_i \,, \qquad
\phi_i \longmapsto \phi_i \,, \nn\\
&& 
\xi_i \longmapsto \xi_i - \frac{\theta}{8\pi^2}\,(\phi_j - \phi_k)\,b_i \,,
}{1}\coordE{}\end{eqnarray}
excludes the \myHighlight{$\theta$}\coordHE{}-dependence from the Hamiltonian (\ref{eq:unch2})
reducing it to the zero \myHighlight{$\theta$}\coordHE{}-angle expression \cite{KP}
\begin{equation}\coord{}\boxEquation{
\label{eq:uncz}
H^{(2)} \, = \,
\frac{1}{2}\, \int\, d^3x
\left[
\sum_{i=1}^3 \pi_i^2  +
\sum_{cyclic}\xi_i^2\frac{\phi_j^2+\phi_k^2}{(\phi_j^2-\phi_k^2)^2}
+V(\phi, \chi)
\right]\,.
}{
H^{(2)} \, = \,
\frac{1}{2}\, \int\, d^3x
\left[
\sum_{i=1}^3 \pi_i^2  +
\sum_{cyclic}\xi_i^2\frac{\phi_j^2+\phi_k^2}{(\phi_j^2-\phi_k^2)^2}
+V(\phi, \chi)
\right]\,.
}{ecuacion}\coordE{}\end{equation}


\subsection{Second-order unconstrained Lagrangian}


We are now ready to derive the Lagrangian up to second-order in derivatives
corresponding to the Hamiltonian (\ref{eq:unch2}).
Carrying out the inverse Legendre transformation,
\begin{eqnarray}\coord{}\boxAlignEqnarray{\leftCoord{}
\dot{\phi}_i &=& \pi_i - \frac{\leftCoord{}\theta}{\rightCoord{}8\pi^2}\beta_i\rightCoord{}\,,\rightCoord{}\\\leftCoord{}
\label{eq:ilt}
\dot \chi_a &=& G_{ab}\Big(p_{\chi_b} -
\frac{\leftCoord{}\theta}{\rightCoord{}8\pi^2}\sum_{\rightCoord{}cyclic}M^T_{bi}
\leftCoord{}(\phi_j -\phi_k)\rightCoord{}\,b_{i}\Big)\rightCoord{}\,,
\rightCoord{}}{0mm}{5}{9}{
\dot{\phi}_i &=& \pi_i - \frac{\theta}{8\pi^2}\beta_i\,,\\
\dot \chi_a &=& G_{ab}\Big(p_{\chi_b} -
\frac{\theta}{8\pi^2}\sum_{cyclic}M^T_{bi}
(\phi_j -\phi_k)\,b_{i}\Big)\,,
}{1}\coordE{}\end{eqnarray}
with the matrix \myHighlight{$M$}\coordHE{} given in (\ref{eq:MCr}),
and the \myHighlight{$3\times 3$}\coordHE{} matrix \myHighlight{$G$}\coordHE{}
\begin{equation}\coord{}\boxEquation{
G= M^{-1} k {M^{-1}}^T\,,
}{
G= M^{-1} k {M^{-1}}^T\,,
}{ecuacion}\coordE{}\end{equation}
similar to the diagonal matrix
\myHighlight{$k=\mbox{diag}\|{k_1, k_2, k_3}\|$}\coordHE{}
with entries \myHighlight{$k_i$}\coordHE{} of (\ref{eq:km}),
we arrive at the second-order Lagrangian
\begin{equation}\coord{}\boxEquation{
\label{Leffgen}
L^{(2)}(\phi, \chi)=\frac{1}{2}\int d^3x\left[\sum_{i=1}^{3}\dot{\phi}_i^2+
\sum_{i,j =1}^{3}\dot{\chi}_iG^{-1}_{ij}\dot{\chi}_j
  - V(\phi, \chi)\right] -
   \theta\int d^3x\,\,Q^{(2)}(\phi, \chi)\,,
}{
L^{(2)}(\phi, \chi)=\frac{1}{2}\int d^3x\left[\sum_{i=1}^{3}\dot{\phi}_i^2+
\sum_{i,j =1}^{3}\dot{\chi}_iG^{-1}_{ij}\dot{\chi}_j
  - V(\phi, \chi)\right] -
   \theta\int d^3x\,\,Q^{(2)}(\phi, \chi)\,,
}{ecuacion}\coordE{}\end{equation}
with all \myHighlight{$\theta$}\coordHE{}-dependence gathered in the reduced topological
charge density
\begin{equation}\coord{}\boxEquation{ \label{eq:q2}
{Q}^{(2)}\, =\frac{1}{8\pi^2}\sum_{a=1}^3\left( \dot{\phi}_a
\beta_a +\sum^{i,j,k}_{cyclic}\dot{\chi}_a
M^{T}_{ai}(\phi_j-\phi_k)\,b_i\right)\,.
}{ {Q}^{(2)}\, =\frac{1}{8\pi^2}\sum_{a=1}^3\left( \dot{\phi}_a
\beta_a +\sum^{i,j,k}_{cyclic}\dot{\chi}_a
M^{T}_{ai}(\phi_j-\phi_k)\,b_i\right)\,.
}{ecuacion}\coordE{}\end{equation}
Using the Maurer-Cartan structure equations for the 1-forms \myHighlight{$\omega_i$}\coordHE{}
\begin{equation}\coord{}\boxEquation{
\label{eg:streq}
d\omega_a  = \Gamma_{a0c} dt\wedge \omega_c +
\Gamma_{abc}\omega_b\wedge \omega_c~,
}{
d\omega_a  = \Gamma_{a0c} dt\wedge \omega_c +
\Gamma_{abc}\omega_b\wedge \omega_c~,
}{ecuacion}\coordE{}\end{equation}
with the space components of \myHighlight{$\Gamma$}\coordHE{} given in
(\ref{Gammaspace}), and the time components correspondingly defined  as
\begin{equation}\coord{}\boxEquation{
\Gamma_{a0b} = \left(\dot{ R} R^T \right)_{ab}\, ,
}{
\Gamma_{a0b} = \left(\dot{ R} R^T \right)_{ab}\, ,
}{ecuacion}\coordE{}\end{equation}
Eq. (\ref{eq:q2})
can be rewritten as
\begin{equation}\coord{}\boxEquation{
\label{dC}
Q^{(2)}= d C^{(2)}~,
}{
Q^{(2)}= d C^{(2)}~,
}{ecuacion}\coordE{}\end{equation}
with the 3-form
\begin{eqnarray}\coord{}\boxAlignEqnarray{\leftCoord{}
 C^{(2)}&=& \frac{\leftCoord{}1}{\rightCoord{}8\pi^2}\sum_{\rightCoord{}a<b}^{\leftCoord{}3} (\phi_a-\phi_b)^2\rightCoord{}\,\Gamma_{a0b}
dt\wedge\omega_a \wedge  \omega_b \rightCoord{}\,- \nn\rightCoord{}\\
&&\leftCoord{}\quad\quad\quad
\leftCoord{}-\frac{\leftCoord{}3}{\rightCoord{}8\pi^2}\sum_{\rightCoord{}cyclic}^{\leftCoord{}3} \left[(\phi_a-\phi_b)^2\rightCoord{}\,\Gamma_{acb}
\leftCoord{}-\frac{\leftCoord{}2}{\rightCoord{}3}\ \varepsilon_{abc}\phi_1\phi_2\phi_3\right]
\omega_a \wedge \omega_b \wedge \omega_c~.\rightCoord{}
\rightCoord{}}{0mm}{9}{12}{
 C^{(2)}&=& \frac{1}{8\pi^2}\sum_{a<b}^{3} (\phi_a-\phi_b)^2\,\Gamma_{a0b}
dt\wedge\omega_a \wedge  \omega_b \,- \nn\\
&&\quad\quad\quad
-\frac{3}{8\pi^2}\sum_{cyclic}^{3} \left[(\phi_a-\phi_b)^2\,\Gamma_{acb}
-\frac{2}{3}\ \varepsilon_{abc}\phi_1\phi_2\phi_3\right]
\omega_a \wedge \omega_b \wedge \omega_c~.
}{1}\coordE{}\end{eqnarray}
This completes our construction of the second-order
Lagrangian with all \myHighlight{$\theta$}\coordHE{}-contributions gathered in a total
differential (\ref{eq:q2}).
The \myHighlight{$Q^{(2)}$}\coordHE{} in the effective Lagrangian (\ref{Leffgen})
can be represented as the divergence
\begin{equation}\coord{}\boxEquation{
Q^{(2)} = \partial^\mu K^{(2)}_\mu\,,
}{
Q^{(2)} = \partial^\mu K^{(2)}_\mu\,,
}{ecuacion}\coordE{}\end{equation}
of the 4-vector \myHighlight{$K^{(2)}_\mu=(K^{(2)}_0,K^{(2)}_i)$}\coordHE{}, with the components
\begin{eqnarray}\coord{}\boxAlignEqnarray{\leftCoord{}
\label{K0un}
K^{(2)}_0 &=& \frac{\leftCoord{}1}{\rightCoord{}16\pi^2}
\sum_{\rightCoord{}cyclic}^{\leftCoord{}3} \left[(\phi_a-\phi_b)^2\Gamma_{acb}
\leftCoord{}-\frac{\leftCoord{}2}{\rightCoord{}3}\ g\ \phi_a\phi_b\phi_c\right]~,\rightCoord{}\\\leftCoord{}
K^{(2)}_i &=&\rightCoord{}\,\frac{\leftCoord{}1}{\rightCoord{}16\pi^2}\sum_{\rightCoord{}cyclic}^{\leftCoord{}3}
R^T_{ia}(\phi_b-\phi_c)^2\Gamma_{b0c}~.\rightCoord{}
\rightCoord{}}{0mm}{8}{10}{
K^{(2)}_0 &=& \frac{1}{16\pi^2}
\sum_{cyclic}^{3} \left[(\phi_a-\phi_b)^2\Gamma_{acb}
-\frac{2}{3}\ g\ \phi_a\phi_b\phi_c\right]~,\\
K^{(2)}_i &=&\,\frac{1}{16\pi^2}\sum_{cyclic}^{3}
R^T_{ia}(\phi_b-\phi_c)^2\Gamma_{b0c}~.
}{1}\coordE{}\end{eqnarray}
Thus we have found the unconstrained analog of the Chern-Simons current
\myHighlight{$K_\mu^{(2)}$}\coordHE{}, linear in the derivatives.
Under the assumption, that the vector part  \myHighlight{$K^{(2)}_i$}\coordHE{} vanishes
at spatial infinity, the unconstrained form of the Pontryagin index \myHighlight{$p_1$}\coordHE{}
can be represented as the difference of the two surface integrals
\begin{equation}\coord{}\boxEquation{
\label{eq:achsgen}
W_\pm = \int d^3x\,\, K_0^{(2)}\left( t \to  \pm\infty, \, {\vec x}\right)\,,
}{
W_\pm = \int d^3x\,\, K_0^{(2)}\left( t \to  \pm\infty, \, {\vec x}\right)\,,
}{ecuacion}\coordE{}\end{equation}
which are the winding number functional (\ref{clctr1}) for the physical
field \myHighlight{$S$}\coordHE{} in terms of main-axis variables (\ref{eq:sof}) at \myHighlight{$t \to\pm\infty $}\coordHE{}
respectively,
since \myHighlight{$K_0^{(2)}(\phi,\chi)$}\coordHE{} of (\ref{K0un}) coincides with the full
\myHighlight{$K_0[S[\phi,\chi]]$}\coordHE{} of (\ref{CSC}).
In the next Section we shall show, how for certain field configurations,
it reduces to the Hopf number of the
mapping from the \myHighlight{$3$}\coordHE{}-sphere \myHighlight{$\mathbb{S}^3$}\coordHE{} to the unit \myHighlight{$2$}\coordHE{}-sphere
\myHighlight{$\mathbb{S}^2$}\coordHE{}.


%%%%%%%%%%%%%%%%%%%%%%%%%%%%%%%%%%%%%% 6 %%%%%%%%%%%%%%%%%%%%%%%%%%%%%%%%%%%%%%

\section{Unconstrained theory for degenerate configurations  }

%%%%%%%%%%%%%%%%%%%%%%%%%%%%%%%%%%%%%%%%%%%%%%%%%%%%%%%%%%%%%%%%%%%%%%%%%%%%%%%%


The previous study  was restricted to the
consideration of the domain of configuration space with \myHighlight{$\det \|S\| \neq 0$}\coordHE{},
where the change of variables (\ref{eq:gpottr}) is well defined.
In this Section we would like to discuss the dynamics on
the special degenerate stratum (DS) with \myHighlight{$\mbox{rank}\|S\| = 1$}\coordHE{},
corresponding to the case of two eigenvalues of the matrix \myHighlight{$S$}\coordHE{} vanishing.
To investigate the dynamics on degenerate orbits it is in principle
necessary to use a decomposition of the gauge potential
different from our representation (\ref{eq:gpottr})
and the corresponding subsequent main-axis transformation (\ref{eq:mainax}).
Instead of this, we shall use here the fact, that the
degenerate orbits can be regarded as the boundary of the non-degenerate ones
and find the corresponding dynamics taking the corresponding limit from
the non-degenerate orbits.
Assuming the validity of such an approach
we shall analyze the limit when
two eigenvalues of the symmetric matrix \myHighlight{$S$}\coordHE{}  tend to zero
\footnote{
It can easily be checked that the degenerate stratum with
\myHighlight{$\mbox{rank}\|S\|=1$}\coordHE{} is dynamically invariant.
Furthermore, it is obvious from the representation
(\ref{eq:uncz}) of the unconstrained Hamiltonian, that it is necessary to have
\myHighlight{$\xi_k\rightarrow 0$}\coordHE{} for some fixed \myHighlight{$k$}\coordHE{},
in order to obtain a finite contribution of the
kinetic term to the Hamiltonian in the limit
\myHighlight{$\phi_i,\phi_j\rightarrow 0$}\coordHE{} for \myHighlight{$(i, j \neq k)$}\coordHE{}.}.
Due to the cyclic symmetry under permutation of diagonal fields
it is enough to choose one  singular configuration
\begin{equation}\coord{}\boxEquation{
\label{eq:conf}
\phi_1(x) = \phi_2(x) = 0\ \ \ {\rm and} \qquad \phi_3(x)
\ \quad {\rm arbitrary}\,.
}{
\phi_1(x) = \phi_2(x) = 0\ \ \ {\rm and} \qquad \phi_3(x)
\ \quad {\rm arbitrary}\,.
}{ecuacion}\coordE{}\end{equation}
Note that for the configuration (\ref{eq:conf}) the homogeneous part of the
square of the magnetic field, vanishes and the potential term in the
Lagrangian (\ref{Leffgen}) reduces to the expression
\begin{eqnarray}\coord{}\boxAlignEqnarray{\leftCoord{}
V = \rightCoord{}
&&\leftCoord{} \phi_3^2\big[(\Gamma_{2 1 3})^2+(\Gamma_{2 2 3})^2
           \leftCoord{}+(\Gamma_{2 3 3})^2
           \leftCoord{}+(\Gamma_{3 1 1})^2+(\Gamma_{3 2 1})^2
           \leftCoord{}+(\Gamma_{3 3 1})^2 + (\Gamma_{3 [12]})^2 \big]\nonumber\rightCoord{}\\
&&\leftCoord{} +\big[(X_1\phi_3)^2+(X_2\phi_3)^2\big]
   \leftCoord{}+2\phi_3\big[\Gamma_{3 3 1} X_1\phi_3
                   \leftCoord{}+\Gamma_{3 3 2} X_2\phi_3\big]~,
\label{V21}
\rightCoord{}}{0mm}{8}{4}{
V = 
&& \phi_3^2\big[(\Gamma_{2 1 3})^2+(\Gamma_{2 2 3})^2
           +(\Gamma_{2 3 3})^2
           +(\Gamma_{3 1 1})^2+(\Gamma_{3 2 1})^2
           +(\Gamma_{3 3 1})^2 + (\Gamma_{3 [12]})^2 \big]\\
&& +\big[(X_1\phi_3)^2+(X_2\phi_3)^2\big]
   +2\phi_3\big[\Gamma_{3 3 1} X_1\phi_3
                   +\Gamma_{3 3 2} X_2\phi_3\big]~,
}{1}\coordE{}\end{eqnarray}
which can be rewritten as \cite{KP,ErrKP}
\begin{equation}\coord{}\boxEquation{
\label{V22}
V= (\nabla \phi_3)^2
+ \phi_3^2\left[(\partial_i{\mathbf{n}})^2+ (\mathbf{n}\cdot
{\rm rot}{\ \mathbf{n}})^2\right]
 -(\mathbf{n} \cdot \nabla \phi_3)^2
+ ([\mathbf{n} \times \mbox{rot\ } \mathbf{n} ] \cdot \nabla \phi_3^2)~,
}{
V= (\nabla \phi_3)^2
+ \phi_3^2\left[(\partial_i{\mathbf{n}})^2+ (\mathbf{n}\cdot
{\rm rot}{\ \mathbf{n}})^2\right]
 -(\mathbf{n} \cdot \nabla \phi_3)^2
+ ([\mathbf{n} \times \mbox{rot\ } \mathbf{n} ] \cdot \nabla \phi_3^2)~,
}{ecuacion}\coordE{}\end{equation}
introducing the unit vector
\begin{equation}\coord{}\boxEquation{
n_i(x):=R_{3i}(\chi(x))\,.
}{
n_i(x):=R_{3i}(\chi(x))\,.
}{ecuacion}\coordE{}\end{equation}
Hence the unconstrained second-order Lagrangian
corresponding to the degenerate stratum   with \myHighlight{$\mbox{rank}\|S(x)\| = 1$}\coordHE{}
represents  the nonlinear \myHighlight{$\sigma$}\coordHE{}-model type Lagrangian
\begin{eqnarray}\coord{}\boxAlignEqnarray{\leftCoord{}
\label{eq:DSL}
L_{\rm DS} &=&
{\rightCoord{}\leftCoord{}1\leftCoord{}\over\rightCoord{} 2}\int d^3x \rightCoord{}
\Big[(\partial_\mu \phi_3)^2+  \phi_3^2(\partial_\mu \mathbf{n})^2-
 \phi_3^2(\mathbf{n}\cdot{\rm rot}{\ \mathbf{n}})^2
 \leftCoord{}+(\mathbf{n} \cdot \nabla \phi_3)^2\nn\rightCoord{}\\
&&\leftCoord{}\ \ \ \ \ \ \ \ \ \ \ \ \ \rightCoord{}
\leftCoord{}- ([\mathbf{n} \times \mbox{rot\ } \mathbf{n} ] \cdot \nabla \phi_3^2)\Big]
\leftCoord{}-\theta\int d^3x\rightCoord{}\,\rightCoord{}\, Q_{\rm DS}\rightCoord{}\,,
\rightCoord{}}{0mm}{7}{10}{
L_{\rm DS} &=&
{1\over 2}\int d^3x 
\Big[(\partial_\mu \phi_3)^2+  \phi_3^2(\partial_\mu \mathbf{n})^2-
 \phi_3^2(\mathbf{n}\cdot{\rm rot}{\ \mathbf{n}})^2
 +(\mathbf{n} \cdot \nabla \phi_3)^2\nn\\
&&\ \ \ \ \ \ \ \ \ \ \ \ \ 
- ([\mathbf{n} \times \mbox{rot\ } \mathbf{n} ] \cdot \nabla \phi_3^2)\Big]
-\theta\int d^3x\,\, Q_{\rm DS}\,,
}{1}\coordE{}\end{eqnarray}
for the unit vector \myHighlight{$\mathbf{n}(x)$}\coordHE{} - field  coupled to the field \myHighlight{$\phi_3(x)$}\coordHE{}.
The density of topological term \myHighlight{$Q_{\rm DS}$}\coordHE{} in the Lagrangian (\ref{eq:DSL})
can be represented as the divergence
\begin{equation}\coord{}\boxEquation{
Q_{\rm DS} = \partial_\mu K_{\rm DS}^\mu\,
}{
Q_{\rm DS} = \partial_\mu K_{\rm DS}^\mu\,
}{ecuacion}\coordE{}\end{equation}
of the 4-vector
\begin{equation}\coord{}\boxEquation{
K_{\rm DS}^\mu = \frac{1}{16\pi^2}\phi_3^2 \left(  (\mathbf{n}(x)
\cdot \mbox{rot}\
\mathbf{n}(x))\,, \
[\mathbf{n}(x) \times \dot{\mathbf{n}}(x)] \right)\,.
}{
K_{\rm DS}^\mu = \frac{1}{16\pi^2}\phi_3^2 \left(  (\mathbf{n}(x)
\cdot \mbox{rot}\
\mathbf{n}(x))\,, \
[\mathbf{n}(x) \times \dot{\mathbf{n}}(x)] \right)\,.
}{ecuacion}\coordE{}\end{equation}
If we impose the usual boundary condition that the field \myHighlight{${\mathbf{n}}$}\coordHE{}
becomes time-independent at spatial infinity,  the contribution from the
vector part  \myHighlight{$K_{\rm DS}^i$}\coordHE{} vanishes and
the unconstrained form of the Pontryagin topological index \myHighlight{$p_1$}\coordHE{}
for the degenerate stratum with \myHighlight{$\mbox{rank}\|S\|=1$}\coordHE{} can be represented
as the difference
\begin{equation}\coord{}\boxEquation{
p_1 = n_+  - n_-\,
}{
p_1 = n_+  - n_-\,
}{ecuacion}\coordE{}\end{equation}
of the surface integrals
\begin{equation}\coord{}\boxEquation{
\label{eq:achs}
n_{\pm} = \frac{1}{16\pi^2}\int d^3x
\left( {\mathbf{V}}_{\pm}(\vec{x})\cdot \mbox{rot}\
{\mathbf{V}}_{\pm} (\vec{x}) \right)\,
}{
n_{\pm} = \frac{1}{16\pi^2}\int d^3x
\left( {\mathbf{V}}_{\pm}(\vec{x})\cdot \mbox{rot}\
{\mathbf{V}}_{\pm} (\vec{x}) \right)\,
}{ecuacion}\coordE{}\end{equation}
of the fields
\begin{equation}\coord{}\boxEquation{
{\mathbf{V}}_{\pm}(\vec{x}) := \lim_{t \to \pm \infty}\ \phi_3(x){\mathbf{n}} ~.
}{
{\mathbf{V}}_{\pm}(\vec{x}) := \lim_{t \to \pm \infty}\ \phi_3(x){\mathbf{n}} ~.
}{ecuacion}\coordE{}\end{equation}
We shall show now that the surface integrals (\ref{eq:achs}) are
Hopf invariants in the representation of Whitehead \cite{Whitehead}.

Under the Hopf mapping of a 3-sphere to a 2-sphere having unit radius,
\myHighlight{$N\, : \mathbb{S}^3 \to \mathbb{S}^2$}\coordHE{}, the preimage of a point on
\myHighlight{$\mathbb{S}^2$}\coordHE{} is a closed loop.
The number \myHighlight{$Q_H$}\coordHE{} of times, the loops corresponding to two distinct points
on \myHighlight{$\mathbb{S}^2$}\coordHE{} are linked to each other, is the so-called Hopf invariant.
According to Whitehead \cite{Whitehead}, this linking number can be
represented by the integral
\begin{equation}\coord{}\boxEquation{ \label{eq:Hopf}
Q_H = \frac{1}{32\pi^2}\int_{S^3}  w^1 \wedge w^2\,,
}{ Q_H = \frac{1}{32\pi^2}\int_{S^3}  w^1 \wedge w^2\,,
}{ecuacion}\coordE{}\end{equation}
with the so-called Hopf 2-form curvature  \myHighlight{$w^2 = H_{ij} dx^i  \wedge dx^j $}\coordHE{}
given in terms of the map \myHighlight{$N$}\coordHE{} as
\begin{equation}\coord{}\boxEquation{
\label{Hij1}
H_{ij}  = \varepsilon_{abc}N_a
\left( \partial_i N_b\right) \left(\partial_j N_c \right)\,,
}{
H_{ij}  = \varepsilon_{abc}N_a
\left( \partial_i N_b\right) \left(\partial_j N_c \right)\,,
}{ecuacion}\coordE{}\end{equation}
and the 1-form \myHighlight{$w^1$}\coordHE{} related to it via \myHighlight{$w^2 =dw^1$}\coordHE{}.
Since the curvature \myHighlight{$H_{ij}$}\coordHE{} is divergence-free,
\begin{equation}\coord{}\boxEquation{
\varepsilon_{ijk}\partial_{i}H_{jk}=0~,
}{
\varepsilon_{ijk}\partial_{i}H_{jk}=0~,
}{ecuacion}\coordE{}\end{equation}
it can be represented as the rotation
\begin{equation}\coord{}\boxEquation{
\label{Hij2}
H_{ij} =\partial_i \mathcal{A}_j - \partial_j\mathcal{A}_i~,
}{
H_{ij} =\partial_i \mathcal{A}_j - \partial_j\mathcal{A}_i~,
}{ecuacion}\coordE{}\end{equation}
in terms of some vector field \myHighlight{$\mathcal{A}_i$}\coordHE{} (\myHighlight{$i=1,2,3$}\coordHE{})
defined over the whole of \myHighlight{$\mathbb{S}^3$}\coordHE{}.
Thus the Hopf invariant takes the form
\begin{equation}\coord{}\boxEquation{
\label{eq:rot}
Q_{H} = \frac{1}{16\pi^2}\int d^3x
\left( {\mathbf{\mathcal{A}}}\cdot \mbox{rot}\
{\mathbf{\mathcal{A}}} \right)\,.
}{
Q_{H} = \frac{1}{16\pi^2}\int d^3x
\left( {\mathbf{\mathcal{A}}}\cdot \mbox{rot}\
{\mathbf{\mathcal{A}}} \right)\,.
}{ecuacion}\coordE{}\end{equation}

Therefore, the surface integrals (\ref{eq:achs}) are just Hopf invariants
in the Whitehead representation (\ref{eq:rot})
and the unconstrained form of the topological term \myHighlight{$Q^{(2)}$}\coordHE{}
is an 3-dimensional Abelian Chern-Simons term \cite{Jackiw}
with ``potential'' \myHighlight{$V_i\ $}\coordHE{} and the corresponding ``magnetic
field" \myHighlight{$ \mbox{rot}\mathbf{V}$}\coordHE{}.
The topological term in the original \myHighlight{$SU(2)$}\coordHE{} Yang-Mills theory
reduces for rank-1 degenerate orbits not to a winding number,
but the linking number \myHighlight{$Q_H$}\coordHE{} of the field lines.

We ask at this place the question whether the obtained unconstrained theory for
degenerate field configurations can be treated as the classical counterpart of
an effective quantum model relevant to the low energy region of Yang-Mills theory.
According to the recent argumentation by
Faddeev and Niemi \cite{FaddeevNiemi} the so-called
\myHighlight{$O(3)$}\coordHE{} Faddeev-Skyrme model can be used as an quantum effective
theory for the infrared sector of Yang-Mills theory. The Hopf invariant
serves as a topological characteristic of the low energy gluon
field configurations \cite{Langmann}
\footnote{See also the discussion in
\cite{BaalWipf},  where the representation of gauge fields in terms of the
complex two-component \myHighlight{$\mathbb{C}\mathbb{P}^1$}\coordHE{} variables has been exploited.}.
The \myHighlight{$O(3)$}\coordHE{} Faddeev-Skyrme model represents the
theory of a three-dimensional unit vector with an action
that includes the standard kinetic part and the so-called
Skyrme term, providing the stability of solitonic type solutions
(see e.g. discussion in \cite{Battye:1998}).
The degenerate field configurations \myHighlight{$\phi_1=\phi_2=0$}\coordHE{}
with arbitrary \myHighlight{$\phi_3$}\coordHE{} of (\ref{eq:conf}) correspond to the zeros of the
homogeneous part of the square of the magnetic field, which is the leading
order \myHighlight{$g \rightarrow \infty $}\coordHE{} term in the Hamiltonian, and thus might
be interpreted as the classical vacua.
As an outlook to a possible quantum description we consider this diagonal field
\myHighlight{$\phi_3(x)$}\coordHE{} as slow varying with a constant non-zero vacuum expectation value
\begin{equation}\coord{}\boxEquation{\label{eq:nexp}
\langle \phi_3(x)\rangle = \mu~.
}{\langle \phi_3(x)\rangle = \mu~.
}{ecuacion}\coordE{}\end{equation}
Therefore, neglecting the fluctuations of the
field \myHighlight{$\phi_3(x)$}\coordHE{} around the expectation value (\ref{eq:nexp})
in the Lagrangian (\ref{eq:DSL}) we arrive at the corresponding effective action
\begin{equation}\coord{}\boxEquation{
\label{eq:effnl}
  S_{eff}({\mathbf n}) = \frac{\mu^2}{2}\int d^4 x
  \left[ (\partial_\mu {\mathbf n})^2 -
  \left({\mathbf n}(x)\cdot \mbox{rot}\ {\mathbf n}(x) \right)^2\right] -
\frac{\theta \mu^2}{16\pi^2}\int d^3 x
\left({\mathbf n}(x)\cdot \mbox{rot}\ {\mathbf n}(x) \right) \,.
}{
S_{eff}({\mathbf n}) = \frac{\mu^2}{2}\int d^4 x
  \left[ (\partial_\mu {\mathbf n})^2 -
  \left({\mathbf n}(x)\cdot \mbox{rot}\ {\mathbf n}(x) \right)^2\right] -
\frac{\theta \mu^2}{16\pi^2}\int d^3 x
\left({\mathbf n}(x)\cdot \mbox{rot}\ {\mathbf n}(x) \right) \,.
}{ecuacion}\coordE{}\end{equation}
The effective action (\ref{eq:effnl}) is a non-linear sigma-model-type
theory similar to that proposed by Faddeev and Niemi \cite{FaddeevNiemi}.
In difference to the Faddeev-Skyrme model, however,
where a standard Skyrme stabilizing term, fourth order in derivatives,
is used, in our result (\ref{eq:effnl}),
the square of the density of the Hopf invariant appears in the Whitehead
form (\ref{Hij2}), linear in derivatives.
Furthermore, in the Faddeev-Niemi effective action, the unit-vector
is a Lorentz scalar, while the Lorentz transformation
properties of the field \myHighlight{${\mathbf n}$}\coordHE{} in the action (\ref{eq:effnl})
are not standard due to the noncovariance of the symmetric gauge imposed.
Their investigation carefully taking into account surface contributions
to the unconstrained form of the generators of the Poincar\'{e} group is
under investigation.

%%%%%%%%%%%%%%%%%%%%%%%%%%%%%%%%%%%%%% 7 %%%%%%%%%%%%%%%%%%%%%%%%%%%%%%%%%%%%%%

\section{Conclusions and remarks}

%%%%%%%%%%%%%%%%%%%%%%%%%%%%%%%%%%%%%%%%%%%%%%%%%%%%%%%%%%%%%%%%%%%%%%%%%%%%%%%%

We have generalized the Hamiltonian reduction of
\myHighlight{$SU(2)$}\coordHE{} Yang-Mills gauge theory to the case of nonvanishing \myHighlight{$\theta$}\coordHE{}-angle,
and shown that there is agreement between reduced
and original constrained equations of motions.
We have employed an improved derivative expansion to the non-local
kinetic term in the obtained unconstrained Hamiltonian and
investigated it in long-wavelength approximation.
The corresponding second order Lagrangian has been constructed,
with all \myHighlight{$\theta$}\coordHE{}-dependence gathered in a 4-divergence of
a current, linear in the derivatives, which is the
unconstrained analog of the original Chern-Simons current.

For the degenerate gauge field configurations \myHighlight{$S$}\coordHE{} with \myHighlight{$\mbox{rank}\|S\| =1$}\coordHE{}
we have argued that the obtained long-wavelength Lagrangian reduces to a
classical theory with an Abelian Chern-Simons term originating from the
Pontryagin topological functional.
Therefore the topological characteristic of degenerate
configuration is given not by a winding number,
but the linking number of the field lines.

Finally let us comment on the Poincar\'{e} covariance of our
unconstrained version of Yang-Mills theory. It is well-known that the
Hamiltonian formulation of degenerate theories reduced with the help of
non-covariant gauges destroy the manifest Poincar\'{e} invariance.
Our ``symmetric'' gauge condition (\ref{symgauge})
is not covariant under standard Lorentz transformations.
This, however, does not necessarily violate
the Poincare invariance of our reduced theory.
Such a situation can be found in classical Electrodynamics.
After imposing the Coulomb gauge condition the vector potential
ceases to be an ordinary Lorentz vector and
transforms non-homogeneously under Lorentz transformations.
The standard Lorentz boosts are compensated by some additional
gauge-type transformation depending on the
boost parameters and the gauge potential itself
(see e.g. \cite{BjorkenDrell,HansonReggeTeitelboim,PavelPervushin}).
As for the case of the Coulomb gauge in
Electrodynamics, a thorough analysis of the Poincar\'{e} group representation
for our reduced theory obtained imposing the symmetric gauge condition
is required. This problem is technically highly difficult and demands
special consideration that is beyond the scope of present article.


%%%%%%%%%%%%%%%%%%%%%%%%%%%%%%%%%%%%%% 8 %%%%%%%%%%%%%%%%%%%%%%%%%%%%%%%%%

\section*{Acknowledgements}

%%%%%%%%%%%%%%%%%%%%%%%%%%%%%%%%%%%%%%%%%%%%%%%%%%%%%%%%%%%%%%%%%%%%%%%%%%%%


We are grateful for discussions with Z. Aouissat, D. ~Blaschke,  R.~Horan,
A.~Kovner, A.~N.~Kvinikhidze,  M.~Lavelle, M.D.~Mateev, D.~McMullan,
V.N.~Pervushin, P.~Schuck, D.V.~Shirkov, M.~Staudacher and  A.N.~ Tavkhelidze.
A.K. and D.M. would like to thank Professor G. R\"opke
for his kind hospitality in the group ''Theoretische Vielteilchenphysik''
at the Fachbereich Physik of Rostock University, where part of this work has
been done. A.K. thanks the Deutsche Forschungsgemeinschaft and
H.-P.P. the Bundesministerium fuer Forschung und Technologie for
financial support.

%%%%%%%%%%%%%%%%%%%%%%%%%%%%%%%%%%%%%% Appendix A %%%%%%%%%%%%%%%%%%%%%%%%%%%%%

\section*{Appendix A: Conventions and notations}

%%%%%%%%%%%%%%%%%%%%%%%%%%%%%%%%%%%%%%%%%%%%%%%%%%%%%%%%%%%%%%%%%%%%%%%%%%%%%%%%

\label{ap:A}

In this Appendix, we collect several notations and definitions for \myHighlight{$SU(2)$}\coordHE{}
Yang-Mills theory used in the text following \cite{Jackiw}.

The classical Yang-Mills action of the \myHighlight{$su(2)$}\coordHE{}-valued connection \myHighlight{$1$}\coordHE{}-form \myHighlight{$A$}\coordHE{}
in \myHighlight{$4$}\coordHE{}-dimensional Minkowski space-time with a metric \myHighlight{$\eta =
\mbox{diag}\|1,-1,-1,-1\|$}\coordHE{}
reads
\begin{equation}\coord{}\boxEquation{
\label{eq:action}
I  =  - \frac{1}{g^2} \, \int  \, \mbox{tr} \, F \wedge {}^\ast\! F -
\, \frac{\theta}{8\pi^2\,g^2} \, \int \, \mbox{tr} \, F \wedge F \,,
}{
I  =  - \frac{1}{g^2} \, \int  \, \mbox{tr} \, F \wedge {}^\ast\! F -
\, \frac{\theta}{8\pi^2\,g^2} \, \int \, \mbox{tr} \, F \wedge F \,,
}{ecuacion}\coordE{}\end{equation}
with the curvature 2-form
\begin{equation}\coord{}\boxEquation{
F = d A + A \, \wedge\, A\,
}{
F = d A + A \, \wedge\, A\,
}{ecuacion}\coordE{}\end{equation}
and its Hodge dual \myHighlight{${}^\ast\! F$}\coordHE{}.
The trace in (\ref{eq:action}) is calculated in the
antihermitian \myHighlight{$su(2)$}\coordHE{} algebra basis
\myHighlight{$\tau^a = \sigma^a/2\, i $}\coordHE{} with Pauli matrices \myHighlight{$\sigma^a, \, a=1,2,3$}\coordHE{},
satisfying \myHighlight{$[\tau_a,\, \tau_b] = \varepsilon_{abc}\,\tau_c$}\coordHE{}, and
\myHighlight{$\mbox{tr}\left(\tau_a\tau_b\right)= - \frac{1}{2}\, \delta_{ab}$}\coordHE{}.

In the coordinate basis the components of the connection 1-form \myHighlight{$A$}\coordHE{} are
\begin{equation}\coord{}\boxEquation{
A = g \, \tau^a \, A_\mu^a \, dx^\mu\,,
}{
A = g \, \tau^a \, A_\mu^a \, dx^\mu\,,
}{ecuacion}\coordE{}\end{equation}
and the components of the curvature 2-form \myHighlight{$F$}\coordHE{} are
\begin{eqnarray}\coord{}\boxAlignEqnarray{\leftCoord{}
F \rightCoord{}\,& = &\rightCoord{}\, \frac{\leftCoord{}1}{\rightCoord{}2}\rightCoord{}\, g\rightCoord{}\, \tau^a\rightCoord{}\, F^a_{\mu\nu}\rightCoord{}\, dx^\mu\rightCoord{}\,
\wedge\rightCoord{}\, dx^\nu\rightCoord{}\,,\rightCoord{}\\\leftCoord{}
F^a_{\mu\nu}& = &
\partial_\mu A^a_\nu - \partial_\nu A^a_\mu + g \rightCoord{}\,
\varepsilon^{abc} A^b_\mu A^c_\nu \rightCoord{}\,.\rightCoord{}
\rightCoord{}}{0mm}{3}{16}{
F \,& = &\, \frac{1}{2}\, g\, \tau^a\, F^a_{\mu\nu}\, dx^\mu\,
\wedge\, dx^\nu\,,\\
F^a_{\mu\nu}& = &
\partial_\mu A^a_\nu - \partial_\nu A^a_\mu + g \,
\varepsilon^{abc} A^b_\mu A^c_\nu \,.
}{1}\coordE{}\end{eqnarray}
Its dual \myHighlight{${}^\ast\! F$}\coordHE{} are given as
\begin{eqnarray}\coord{}\boxAlignEqnarray{\leftCoord{}
{\rightCoord{}\leftCoord{}}^\ast\! F\rightCoord{}\, & = &\rightCoord{}\, \rightCoord{}
\frac{\leftCoord{}1}{\rightCoord{}2}\rightCoord{}\, g \rightCoord{}\, \tau^a\rightCoord{}\, {}^\ast\! F^a_{\mu\nu}\rightCoord{}\, dx^\mu\wedge dx^\nu\rightCoord{}\,,\rightCoord{}\\\leftCoord{}
{\rightCoord{}\leftCoord{}}^\ast\! F^a_{\mu\nu}\rightCoord{}\, &=&\rightCoord{}\, \frac{\leftCoord{}1}{\rightCoord{}2}\rightCoord{}\, \varepsilon_{\mu\nu\rho\sigma}
F^{a\rightCoord{}\, \rho\sigma}\rightCoord{}\,, \rightCoord{}
\rightCoord{}}{0mm}{6}{21}{
{}^\ast\! F\, & = &\, 
\frac{1}{2}\, g \, \tau^a\, {}^\ast\! F^a_{\mu\nu}\, dx^\mu\wedge dx^\nu\,,\\
{}^\ast\! F^a_{\mu\nu}\, &=&\, \frac{1}{2}\, \varepsilon_{\mu\nu\rho\sigma}
F^{a\, \rho\sigma}\,, 
}{1}\coordE{}\end{eqnarray}
with totally antisymmetric Levi-Civita pseudotensor
\myHighlight{$\varepsilon_{\mu\nu\rho\sigma}$}\coordHE{} using the convention
\begin{equation}\coord{}\boxEquation{
\varepsilon^{0123} = -\, \varepsilon_{0123} = 1.
}{
\varepsilon^{0123} = -\, \varepsilon_{0123} = 1.
}{ecuacion}\coordE{}\end{equation}
The \myHighlight{$\theta$}\coordHE{}-angle enters the classical action as coefficient in front of
the Pontryagin index density
\begin{equation}\coord{}\boxEquation{
Q\, = -\, \frac{1}{8 \pi^2}\, \mbox{tr} \, F \wedge F\,.
}{
Q\, = -\, \frac{1}{8 \pi^2}\, \mbox{tr} \, F \wedge F\,.
}{ecuacion}\coordE{}\end{equation}
The Pontryagin index density is a closed form \myHighlight{$d\,Q=0$}\coordHE{} and thus
locally exact
\begin{equation}\coord{}\boxEquation{\label{eq:topcharge}
Q \, = \, d\,C\,,
}{Q \, = \, d\,C\,,
}{ecuacion}\coordE{}\end{equation}
with the Chern 3-form
\begin{equation}\coord{}\boxEquation{
C = - \frac{1}{8 \pi^2}\, {\rm tr}\left( A\wedge dA +
\frac{2}{3}\, A\wedge A\wedge A \right)\,.
}{
C = - \frac{1}{8 \pi^2}\, {\rm tr}\left( A\wedge dA +
\frac{2}{3}\, A\wedge A\wedge A \right)\,.
}{ecuacion}\coordE{}\end{equation}
The corresponding Chern-Simons current \myHighlight{$K^\mu$}\coordHE{}
is a dual of the 3-form \myHighlight{$C$}\coordHE{},
\begin{equation}\coord{}\boxEquation{
K^\mu =
(1/3!)\, \varepsilon^{\mu\nu\rho\sigma}\, C_{\nu\rho\sigma}\,=
-\,\frac{1}{16 \pi^2}\, \varepsilon^{\mu\alpha\beta\gamma}\mbox{tr}
\left(
F_{\alpha\beta}\, A_\gamma - \frac{2}{3}\, A_\alpha A_\beta A_\gamma
\right)\,.
}{
K^\mu =
(1/3!)\, \varepsilon^{\mu\nu\rho\sigma}\, C_{\nu\rho\sigma}\,=
-\,\frac{1}{16 \pi^2}\, \varepsilon^{\mu\alpha\beta\gamma}\mbox{tr}
\left(
F_{\alpha\beta}\, A_\gamma - \frac{2}{3}\, A_\alpha A_\beta A_\gamma
\right)\,.
}{ecuacion}\coordE{}\end{equation}
with the notations \myHighlight{$A_\mu :=g \, \tau^a \, A_\mu^a \,$}\coordHE{} and
\myHighlight{$ F_{\mu\nu} :=g \, \tau^a \, F_{\mu\nu}^a \,$}\coordHE{}.
The chromomagnetic field is given as
\begin{equation}\coord{}\boxEquation{
B^a_{i} = \frac{1}{2} \, \varepsilon_{ijk}\, F^a_{jk}\, =
\varepsilon_{ijk}\, \left(\partial_j A_{ak} + \frac{g}{2}\,
\varepsilon_{abc}\, A_{bj}\, A_{ck}\right)\,,
}{
B^a_{i} = \frac{1}{2} \, \varepsilon_{ijk}\, F^a_{jk}\, =
\varepsilon_{ijk}\, \left(\partial_j A_{ak} + \frac{g}{2}\,
\varepsilon_{abc}\, A_{bj}\, A_{ck}\right)\,,
}{ecuacion}\coordE{}\end{equation}
and the covariant derivative in the adjoint representation as
\begin{equation}\coord{}\boxEquation{
\left( D_i (A) \right)_{ac}  =
\delta_{ac} \partial_i +  g\, \varepsilon_{abc} A_{bi}~.
}{
\left( D_i (A) \right)_{ac}  =
\delta_{ac} \partial_i +  g\, \varepsilon_{abc} A_{bi}~.
}{ecuacion}\coordE{}\end{equation}
Finally, we frequently use the matrix notations
\begin{equation}\coord{}\boxEquation{
A_{ai} := A^a_{i}\,,\qquad
B_{ai} := B^a_{i}\,.
}{
A_{ai} := A^a_{i}\,,\qquad
B_{ai} := B^a_{i}\,.
}{ecuacion}\coordE{}\end{equation}

%%%%%%%%%%%%%%%%%%%%%%%%%%%%%%%%%%%%% Appendix B %%%%%%%%%%%%%%%%%%%%%%%%%%%%%%

\section*{Appendix B: On the existence of the ``symmetric gauge'' }

%%%%%%%%%%%%%%%%%%%%%%%%%%%%%%%%%%%%%%%%%%%%%%%%%%%%%%%%%%%%%%%%%%%%%%%%%%%%%%%%
\label{ap:B}

In this Appendix we discuss the condition under which the symmetric gauge
\begin{equation}\coord{}\boxEquation{
\label{eq:symmgauge}
\chi_a (A) = \varepsilon_{abi}\,A_{bi}(x) = 0\,,
}{
\chi_a (A) = \varepsilon_{abi}\,A_{bi}(x) = 0\,,
}{ecuacion}\coordE{}\end{equation}
exists.

According to the conventional  gauge-fixing method
(see e.g. \cite{FadSlav}), a gauge \myHighlight{$\chi_a (A) = 0$}\coordHE{} exists,
if the corresponding equation
\begin{equation}\coord{}\boxEquation{
\label{eq:gf}
\chi_a (A^\omega) =0
}{
\chi_a (A^\omega) =0
}{ecuacion}\coordE{}\end{equation}
in terms of the gauge transformed potential
\begin{equation}\coord{}\boxEquation{
\label{eq:gtr}
A^\omega_{ai}\tau_a =
U^+ (\omega)
         \left( A_{ai}\tau_a
         + \frac{1}{g}\frac{\partial}{\partial x_i} \right)U(\omega)
}{
A^\omega_{ai}\tau_a =
U^+ (\omega)
         \left( A_{ai}\tau_a
         + \frac{1}{g}\frac{\partial}{\partial x_i} \right)U(\omega)
}{ecuacion}\coordE{}\end{equation}
has a unique solution for the unknown function \myHighlight{$\omega(x)$}\coordHE{}
\footnote{
Here we assume that the second gauge condition \myHighlight{$A_{a0}=0$}\coordHE{} is fulfilled and
the function \myHighlight{$\omega(x)$}\coordHE{} therefore depends only on the space coordinates.}.

Hence the symmetric gauge (\ref{eq:symmgauge}) exists, if any gauge potential
\myHighlight{$A$}\coordHE{} can be made symmetric by a unique time-independent gauge transformation.
The equation that determines the gauge transformation \myHighlight{$\omega(x)$}\coordHE{} which converts
an arbitrary gauge potential \myHighlight{$A(x)$}\coordHE{} into its symmetric counterpart
can be written as a matrix equation
\begin{equation}\coord{}\boxEquation{
\label{eq:symgeq}
O^T(\omega)A-A^TO(\omega) = \frac{1}{g}\left(\Sigma(\omega)
                           -\Sigma^T(\omega)\right)\,,
}{
O^T(\omega)A-A^TO(\omega) = \frac{1}{g}\left(\Sigma(\omega)
                           -\Sigma^T(\omega)\right)\,,
}{ecuacion}\coordE{}\end{equation}
with the orthogonal \myHighlight{$3\times 3$}\coordHE{} matrix related to the \myHighlight{$SU(2)$}\coordHE{} group element
\begin{equation}\coord{}\boxEquation{\label{eq:ort}
O_{ab}(\omega) =
- 2\ \mbox{tr} \left(U^+(\omega)\tau_a U(\omega)\tau_b\right)
}{O_{ab}(\omega) =
- 2\ \mbox{tr} \left(U^+(\omega)\tau_a U(\omega)\tau_b\right)
}{ecuacion}\coordE{}\end{equation}
and the \myHighlight{$3\times 3$}\coordHE{} matrix \myHighlight{$\Sigma$}\coordHE{}
\begin{equation}\coord{}\boxEquation{
\label{eq:Omega}
\Sigma_{ai}(\omega):= - \frac{1}{4 i}
\ \varepsilon_{amn}\left(O^T(\omega)
\frac{\partial O(\omega)}{\partial x_i}\right)_{mn}\,.
}{
\Sigma_{ai}(\omega):= - \frac{1}{4 i}
\ \varepsilon_{amn}\left(O^T(\omega)
\frac{\partial O(\omega)}{\partial x_i}\right)_{mn}\,.
}{ecuacion}\coordE{}\end{equation}

We shall now prove the following

{\sf Theorem} :
\emph{ For any non-degenerate matrix \myHighlight{$A$}\coordHE{}
equation (\ref{eq:symgeq}) admits a unique solution
in form of a \myHighlight{$1/g$}\coordHE{} expansion}
\begin{equation}\coord{}\boxEquation{
\label{eq:1gsol}
O(\omega) = O^{(0)}\left[1\ +\sum_{n=1}^{\infty}
\left(\frac{1}{g}\right)^n X^{(n)}\right]\,.
}{
O(\omega) = O^{(0)}\left[1\ +\sum_{n=1}^{\infty}
\left(\frac{1}{g}\right)^n X^{(n)}\right]\,.
}{ecuacion}\coordE{}\end{equation}

{\sf Proof:}
In order to prove the statement, we first note that equating coefficients
of equal powers in \myHighlight{$1/g$}\coordHE{} in the orthogonality condition \myHighlight{$O^TO = OO^T=I$}\coordHE{} of
the matrix \myHighlight{$O$}\coordHE{}, imposes the condition of orthogonality of \myHighlight{$O^{(0)}$}\coordHE{},
\begin{equation}\coord{}\boxEquation{
\label{orth}
O^{(0)T}O^{(0)}=O^{(0)}O^{(0)T}=I~,
}{
O^{(0)T}O^{(0)}=O^{(0)}O^{(0)T}=I~,
}{ecuacion}\coordE{}\end{equation}
as well as the conditions
\begin{eqnarray}\coord{}\boxAlignEqnarray{
&&\leftCoord{} X^{(1)} +  {X^{(1)}}^T = 0\rightCoord{}\,, \nn \rightCoord{}\\
&&\leftCoord{} X^{(2)} +  {X^{(2)}}^T +  X^{(1)}{X^{(1)}}^T = 0\rightCoord{}\,, \nn \rightCoord{}\\
&&\leftCoord{} \cdots  \quad\qquad  \cdots \rightCoord{}\,, \nn \rightCoord{}\\
&&\leftCoord{} X^{(n)} + {X^{(n)}}^T + \sum_{\rightCoord{}i+j=n}X^{(i)}{X^{(j)}}^T = 0\rightCoord{}\,, \nn \rightCoord{}\\
&&\leftCoord{} \cdots  \quad \qquad \cdots
\label{eq:ofn}
\rightCoord{}}{0mm}{5}{11}{
&& X^{(1)} +  {X^{(1)}}^T = 0\,, \nn \\
&& X^{(2)} +  {X^{(2)}}^T +  X^{(1)}{X^{(1)}}^T = 0\,, \nn \\
&& \cdots  \quad\qquad  \cdots \,, \nn \\
&& X^{(n)} + {X^{(n)}}^T + \sum_{i+j=n}X^{(i)}{X^{(j)}}^T = 0\,, \nn \\
&& \cdots  \quad \qquad \cdots
}{1}\coordE{}\end{eqnarray}
for the unknown functions \myHighlight{$X^{(n)}$}\coordHE{}.
Furthermore, plugging expansion (\ref{eq:1gsol})
into equation (\ref{eq:symgeq}) and combining the terms of equal powers of \myHighlight{$1/g$}\coordHE{}
we find that the orthogonal matrix \myHighlight{$O^{(0)}$}\coordHE{} should satisfy equation
(\ref{eq:symgeq}) to leading order in \myHighlight{$1/g$}\coordHE{}
\begin{equation}\coord{}\boxEquation{
\label{eq:f0}
{O^{(0)}}^TA - A^T{O^{(0)}}=0\,,
}{
{O^{(0)}}^TA - A^T{O^{(0)}}=0\,,
}{ecuacion}\coordE{}\end{equation}
and the \myHighlight{$X^{(n)}$}\coordHE{} should satisfy the infinite set of equations
\begin{eqnarray}\coord{}\boxAlignEqnarray{
&&\leftCoord{} {X^{(1)}}^T{O^{(0)}}^TA - A^T{O^{(0)}}{X^{(1)}} \rightCoord{}
\leftCoord{}= \Sigma^{(0)}- {\Sigma^{(0)}}\rightCoord{}\,, \nn\rightCoord{}\\
&&\leftCoord{} \cdots  \quad\qquad  \cdots \rightCoord{}\,, \nn \rightCoord{}\\
&&\leftCoord{} {X^{(n)}}^T{O^{(0)}}^TA- A^T{O^{(0)}}^T{X^{(n)}} \rightCoord{}
\leftCoord{}= \Sigma^{(n-1)}- {\Sigma^{(n-1)}}^T\rightCoord{}\,, \nn \rightCoord{}\\
&&\leftCoord{} \cdots  \quad \qquad \cdots~, \rightCoord{}
\label{eq:fn}
\rightCoord{}}{0mm}{6}{11}{
&& {X^{(1)}}^T{O^{(0)}}^TA - A^T{O^{(0)}}{X^{(1)}} 
= \Sigma^{(0)}- {\Sigma^{(0)}}\,, \nn\\
&& \cdots  \quad\qquad  \cdots \,, \nn \\
&& {X^{(n)}}^T{O^{(0)}}^TA- A^T{O^{(0)}}^T{X^{(n)}} 
= \Sigma^{(n-1)}- {\Sigma^{(n-1)}}^T\,, \nn \\
&& \cdots  \quad \qquad \cdots~, 
}{1}\coordE{}\end{eqnarray}
where the corresponding \myHighlight{$1/g$}\coordHE{} expansion for the matrix \myHighlight{$\Sigma(\omega)$}\coordHE{}
\begin{equation}\coord{}\boxEquation{\label{eq:sigex}
\Sigma(\omega) = \sum_{n=0}^{\infty}
\left(\frac{1}{g}\right)^n \Sigma^{(n)}
}{\Sigma(\omega) = \sum_{n=0}^{\infty}
\left(\frac{1}{g}\right)^n \Sigma^{(n)}
}{ecuacion}\coordE{}\end{equation}
has been used.
Note that in expansion (\ref{eq:sigex})
the \myHighlight{$n$}\coordHE{}-th  order term \myHighlight{$\Sigma^{(n)}$}\coordHE{} is given in terms of
\myHighlight{$O^{(0)}$}\coordHE{} and \myHighlight{$X^{(a)}$}\coordHE{} with \myHighlight{$ a = 1,\cdots ,n-1$}\coordHE{}.

From the structure of equations (\ref{orth})-(\ref{eq:fn}) one can see that the
solution to (\ref{eq:symgeq}) reduces to an algebraic problem.
Indeed, the solution to the first, homogeneous equation (\ref{eq:f0})
is given by the polar decomposition for the arbitrary matrix \myHighlight{$A$}\coordHE{},
\begin{equation}\coord{}\boxEquation{
\label{eq:f0s}
O^{(0)}= A S^{(0)-1 }\,, \quad\quad S^{(0)}=\sqrt{A A^T}~.
}{
O^{(0)}= A S^{(0)-1 }\,, \quad\quad S^{(0)}=\sqrt{A A^T}~.
}{ecuacion}\coordE{}\end{equation}
This solution is unique only if \myHighlight{$\det \|A\| \neq 0$}\coordHE{}.
It follows from the well-known property that the polar decomposition is valid
for an arbitrary matrix \myHighlight{$A$}\coordHE{}, but the orthogonal matrix \myHighlight{$O^{(0)}$}\coordHE{}
is unique only for non-degenerate matrices \cite{Gantmacher}.

To proceed further we use this solution and equations
(\ref{eq:ofn}) for unknown \myHighlight{$X$}\coordHE{} to rewrite the remaining equations
(\ref{eq:fn}) as
\begin{eqnarray}\coord{}\boxAlignEqnarray{
&&\leftCoord{} {X^{(1)}}S^{(0)} + S^{(0)} {X^{(1)}}   = C^{(0)}\rightCoord{}\,, \nn\rightCoord{}\\
&&\leftCoord{} \cdots  \quad\qquad  \cdots \rightCoord{}\,, \nn \rightCoord{}\\
&&\leftCoord{} {X^{(n)}}S^{(0)} + S^{(0)} {X^{(n)}} = C^{(n-1)}\rightCoord{}\,, \nn \rightCoord{}\\
&&\leftCoord{} \cdots  \quad \qquad \cdots~, \rightCoord{}
\label{eq:fn2}
\rightCoord{}}{0mm}{4}{9}{
&& {X^{(1)}}S^{(0)} + S^{(0)} {X^{(1)}}   = C^{(0)}\,, \nn\\
&& \cdots  \quad\qquad  \cdots \,, \nn \\
&& {X^{(n)}}S^{(0)} + S^{(0)} {X^{(n)}} = C^{(n-1)}\,, \nn \\
&& \cdots  \quad \qquad \cdots~, 
}{1}\coordE{}\end{eqnarray}
where the \myHighlight{$n$}\coordHE{}-th  order coefficient  \myHighlight{$C^{(n)}$}\coordHE{} is given in terms of
\myHighlight{$O^{(0)}$}\coordHE{} and \myHighlight{$X^{(1)}, X^{(2)},\dots,  X^{(n-1)} $}\coordHE{}.

Thus starting from the zeroth-order term, the higher-order
terms \myHighlight{$X^{(n)}$}\coordHE{} are given recursively as the solutions
of matrix equations of the type \myHighlight{$X S^{(0)}+ S^{(0)} X = C$}\coordHE{} with known
symmetric positive definite matrix
\myHighlight{$S^{(0)} =\sqrt{A A^T}$}\coordHE{} and matrix \myHighlight{$C$}\coordHE{}, expressed in terms of
the preceding  \myHighlight{$X^{(a)}$}\coordHE{}, \myHighlight{$ a=1, \dots ,n-1 $}\coordHE{}.
The theory of such algebraic equations is well
elaborated, see e.g. \cite{Lancaster,Gantmacher}.
In particular Theorem 8.5.1 in \cite{Lancaster}
states that for matrix equations for unknown matrix \myHighlight{$X$}\coordHE{} of the type
\myHighlight{$XA + BX = C$}\coordHE{}, there is a unique solution
if and only if  the matrices
\myHighlight{$A$}\coordHE{} and -\myHighlight{$B$}\coordHE{} have no common eigenvalues.
Based on this theorem one can conclude, that the unique  solution
to (\ref{orth})-(\ref{eq:fn}) and hence to our original problem
(\ref{eq:symgeq}) exists always for any non-degenerate matrix \myHighlight{$A$}\coordHE{}.

It is necessary to emphasize that in order to prove
the existence and uniqueness of the representation (\ref{eq:gpottr})
it should be shown additionally to the above {\sf Theorem} that the
corresponding symmetric matrix field \myHighlight{$S$}\coordHE{},
\begin{equation}\coord{}\boxEquation{\label{eq:sex}
S(x) = \sum_{n=0}^{\infty}
\left(\frac{1}{g}\right)^n S^{(n)}(x)~,
}{S(x) = \sum_{n=0}^{\infty}
\left(\frac{1}{g}\right)^n S^{(n)}(x)~,
}{ecuacion}\coordE{}\end{equation}
is sign-definite. Above, the positive-definiteness has been shown
only for the zeroth-order term \myHighlight{$S^{(0)}=\sqrt{A A^T}$}\coordHE{}.
The study of this problem,
as well as an analogous investigation for the degenerate field configurations \myHighlight{$A$}\coordHE{}
with \myHighlight{$\det \|A \|= 0$}\coordHE{}, are
beyond the scope of this appendix and will be discussed in detail elsewhere.
Here we limit ourselves to the consideration of a specific example,
elucidating the generic picture.

In the case that the matrix \myHighlight{$A$}\coordHE{} is degenerate, we encounter the problem
of Gribov's copies.
As an illustration of the non-uniqueness of gauge transformation, which
turns a given field configuration \myHighlight{$A$}\coordHE{} into the corresponding symmetric form,
we consider the ``degenerate'' field
\begin{equation}\coord{}\boxEquation{
\label{eq:wu-yang}
A_{a0} = 0\,, \qquad
A_{ai} = - \frac{1}{g r}\  \varepsilon_{aic}\hat{r}_c~,
}{
A_{a0} = 0\,, \qquad
A_{ai} = - \frac{1}{g r}\  \varepsilon_{aic}\hat{r}_c~,
}{ecuacion}\coordE{}\end{equation}
known as the non-Abelian Wu-Yang monopole field, with the unit vector
\myHighlight{$\hat{r}_a = x_a/r\,$}\coordHE{} and \myHighlight{$\, r = \sqrt{x_1^2+x_2^2 + x_3^2}$}\coordHE{}\,.

Performing the gauge transformation
\begin{equation}\coord{}\boxEquation{
\label{eg:gtrw}
S_{ai}\tau_a= U^+ (\omega)
\left( A_{ai}\tau_a + \frac{1}{g}\frac{\partial}{\partial x_i} \right)U(\omega)~,
}{
S_{ai}\tau_a= U^+ (\omega)
\left( A_{ai}\tau_a + \frac{1}{g}\frac{\partial}{\partial x_i} \right)U(\omega)~,
}{ecuacion}\coordE{}\end{equation}
with \myHighlight{$U(\omega)= e^{\omega_a\tau_a}$}\coordHE{} parameterized by one time independent
spherical symmetric function
\begin{equation}\coord{}\boxEquation{\label{eq:fgtr}
\omega_a = f(r)\ \hat{r}_a\,,
}{\omega_a = f(r)\ \hat{r}_a\,,
}{ecuacion}\coordE{}\end{equation}
the Wu-Yang monopole configuration (\ref{eq:wu-yang}),
antisymmetric in space and color indices, can be brought into the
``symmetric form''
\begin{equation}\coord{}\boxEquation{
\label{eq:sinsy}
S_{ai}^{\pm} = \pm \frac{\sqrt{3}}{g r}
\left(\delta_{ai} - \hat{r}_a \hat{r}_i\right)~,
}{
S_{ai}^{\pm} = \pm \frac{\sqrt{3}}{g r}
\left(\delta_{ai} - \hat{r}_a \hat{r}_i\right)~,
}{ecuacion}\coordE{}\end{equation}
if the function \myHighlight{$f(r)$}\coordHE{} is constant and takes four values
\begin{equation}\coord{}\boxEquation{
\label{eq:wutr}
f(r) = \Bigg\{ \begin{array}{cc}\pi/3~,~ 7\pi/3\, &\quad {\rm for}\quad (+)~,
       \\ 5\pi/3~,~ 11\pi/3\, &\quad {\rm for}\quad (-)~. \end{array}
}{
f(r) = \Bigg\{ \begin{array}{cc}\pi/3~,~ 7\pi/3\, &\quad {\rm for}\quad (+)~,
       \\ 5\pi/3~,~ 11\pi/3\, &\quad {\rm for}\quad (-)~. \end{array}
}{ecuacion}\coordE{}\end{equation}
Here \myHighlight{$S^+$}\coordHE{} can be obtained from
Wu-Yang monopole configuration (\ref{eq:wu-yang}) applying two different
gauge transformation with \myHighlight{$\, f(r)=\pi /3\,, 7\pi/3$}\coordHE{}
\begin{equation}\coord{}\boxEquation{\label{eq:gtrv_}
U_{1,2} =\pm \Big(\frac{\sqrt{3}}{2} - \hat{r}\cdot \tau \Big)\,,
}{U_{1,2} =\pm \Big(\frac{\sqrt{3}}{2} - \hat{r}\cdot \tau \Big)\,,
}{ecuacion}\coordE{}\end{equation}
while the \myHighlight{$S^{-}$}\coordHE{} configuration
can be reached using \myHighlight{$\, f(r)= 5\pi/3\,, 11\pi/3$}\coordHE{}
\begin{equation}\coord{}\boxEquation{\label{eq:gtrv-}
U_{3, 4} =\mp \Big(\frac{\sqrt{3}}{2}+ \hat{r}\cdot \tau \Big)\,.
}{U_{3, 4} =\mp \Big(\frac{\sqrt{3}}{2}+ \hat{r}\cdot \tau \Big)\,.
}{ecuacion}\coordE{}\end{equation}

Here it is in order to make the following comments:
\begin{itemize}
\item
For the above gauge transformations we have \myHighlight{$\lim_{r\to \infty}U \neq \pm I$}\coordHE{}.
Thus they are neither small gauge transformations nor
large gauge transformations belonging to any integer \myHighlight{$n$}\coordHE{}-homotopy class
\cite{Jackiw}.
\item
The symmetric configurations (\ref{eq:sinsy}) corresponding to the
Wu-Yang monopole lie on the stratum of degenerate symmetric matrices
with one eigenvalue vanishing and two eigenvalues equal to each other.
\item
The symmetric configurations \myHighlight{$S^+$}\coordHE{} and \myHighlight{$S^-$}\coordHE{} in (\ref{eq:sinsy})
with two-fold Gribov degeneracy
are related to each other by parity conjugation.
\end{itemize}


%%%%%%%%%%%%%%%%%%%%%%%%%%%%%%%%%%%%%% Appendix C %%%%%%%%%%%%%%%%%%%%%%%%%%%%%%

\section*{Appendix C: Proof of
\myHighlight{$\theta$}\coordHE{}-dependence of the naive \myHighlight{$1/g$}\coordHE{} approximation }

%%%%%%%%%%%%%%%%%%%%%%%%%%%%%%%%%%%%%%%%%%%%%%%%%%%%%%%%%%%%%%%%%%%%%%%%%%%%%%%%
\label{ap:C}

In this Appendix it is shown that straightforward application of expansion
of the nonlocal part \myHighlight{$P_a$}\coordHE{} of the kinetic term in the unconstrained Hamiltonian
to zeroth-order discussed in Section \ref{Sec:IV1}, leads to the
appearance of \myHighlight{$\theta$}\coordHE{}-dependence of the reduced system on the classical level.
Expressing the Hamiltonian (\ref{eq:ham2}), in terms of the main-axis variables,
defined in Section \ref{sec:V}, and performing an
inverse Legendre transformation, one obtains the Lagrangian density
\begin{eqnarray}\coord{}\boxAlignEqnarray{\leftCoord{}
\label{Leffgent2}
{\rightCoord{}\leftCoord{}\cal L}^{(2)}(\phi, \chi)&=&\frac{\leftCoord{}1}{\rightCoord{}2}\left(\sum_{\rightCoord{}i=1}^{\leftCoord{}{}\leftCoord{}3}\dot{\phi}_i^2+
\sum_{\rightCoord{}i,j =1}^{\leftCoord{}{}\leftCoord{}3}\dot{\chi}_iG^{-1}_{ij}\dot{\chi}_j
  \leftCoord{}- V(\phi, \chi)\right)- \frac{\leftCoord{}1}{\rightCoord{}2}\left(\frac{\leftCoord{}\theta}{\rightCoord{}8\pi^2}\right)^2 \rightCoord{}
  \sum_{\rightCoord{}cyclic}\frac{\leftCoord{}\Delta^2_i}{\rightCoord{}\phi^2_j+\phi^2_k}\nn\\ &\leftCoord{}&-
\frac{\leftCoord{}\theta}{\rightCoord{}8\pi^2}\sum_{\rightCoord{}a=1}^{\leftCoord{}3}\left( \dot{\phi}_a \beta_a
\leftCoord{}+\sum_{\rightCoord{}cyclic}\dot{\chi}_a
M^{T}_{ai}(\phi_j-\phi_k)\left(b_i+
\frac{\leftCoord{}(\phi_j-\phi_k)}{\rightCoord{}\phi^2_j+\phi^2_k}\Delta_i\right)\right)\rightCoord{}\,,
\rightCoord{}}{0mm}{16}{16}{
{\cal L}^{(2)}(\phi, \chi)&=&\frac{1}{2}\left(\sum_{i=1}^{{}3}\dot{\phi}_i^2+
\sum_{i,j =1}^{{}3}\dot{\chi}_iG^{-1}_{ij}\dot{\chi}_j
  - V(\phi, \chi)\right)- \frac{1}{2}\left(\frac{\theta}{8\pi^2}\right)^2 
  \sum_{cyclic}\frac{\Delta^2_i}{\phi^2_j+\phi^2_k}\nn\\ &&-
\frac{\theta}{8\pi^2}\sum_{a=1}^{3}\left( \dot{\phi}_a \beta_a
+\sum_{cyclic}\dot{\chi}_a
M^{T}_{ai}(\phi_j-\phi_k)\left(b_i+
\frac{(\phi_j-\phi_k)}{\phi^2_j+\phi^2_k}\Delta_i\right)\right)\,,
}{1}\coordE{}\end{eqnarray}
denoting the difference
\begin{equation}\coord{}\boxEquation{
\Delta_i=\frac{1}{2}(\phi_j-\phi_k)b_i-(\phi_j+\phi_k)\,R_{is}B^{(-)}_s~,
}{
\Delta_i=\frac{1}{2}(\phi_j-\phi_k)b_i-(\phi_j+\phi_k)\,R_{is}B^{(-)}_s~,
}{ecuacion}\coordE{}\end{equation}
with \myHighlight{$b_i$}\coordHE{} of (\ref{b_i}) and \myHighlight{$B^{(-)}_i$}\coordHE{} of (\ref{B-}),
or explicitly,
\begin{equation}\coord{}\boxEquation{
\Delta_i=
-\big[X_i(\phi_j\phi_k)+(\Gamma_{ijj}+\Gamma_{ikk})\phi_j\phi_k
-\phi_i(\phi_j\Gamma_{ikk}+\phi_k\Gamma_{ijj})\big]~.
}{
\Delta_i=
-\big[X_i(\phi_j\phi_k)+(\Gamma_{ijj}+\Gamma_{ikk})\phi_j\phi_k
-\phi_i(\phi_j\Gamma_{ikk}+\phi_k\Gamma_{ijj})\big]~.
}{ecuacion}\coordE{}\end{equation}
It easy to convince ourselves that the term proportional to \myHighlight{$\theta^2$}\coordHE{}
is not a surface term. Indeed, considering for simplicity configurations of
spatially constant angular variables \myHighlight{$\chi_i$}\coordHE{} and
\myHighlight{$\phi_1=\phi_2=\phi_3=:\phi\,$}\coordHE{},it reduces to
\begin{equation}\coord{}\boxEquation{
-\left(\frac{\theta}{8\pi^2}\right)^2\sum_{i=1}^{3}\,\,
\partial_i\phi\,\, \partial_i\phi~,
}{
-\left(\frac{\theta}{8\pi^2}\right)^2\sum_{i=1}^{3}\,\,
\partial_i\phi\,\, \partial_i\phi~,
}{ecuacion}\coordE{}\end{equation}
which is not a 4-divergence.
For \myHighlight{$\Delta_i=0$}\coordHE{} the Lagrangian density
(\ref{Leffgent2}) reduces to (\ref{Leffgen}), obtained
from the improved  Hamiltonian (\ref{eq:iham2}),
free of the divergence problem.

\begin{thebibliography}{99}
%
\bibitem{JackiwRebbi}
R. Jackiw and C. Rebbi,
{\it Vacuum periodicity in a Yang-Mills quantum theory},
Phys. Rev. Lett. {\bf 37}, 172-175 (1976).
%
\bibitem{Callan}
C.G. Callan, R.F. Dashen, and D.J. Gross,
{\it The structure of the gauge theory vacuum},
Phys. Lett. {\bf B 63}, 334-340 (1976).
%
\bibitem{Deser}
S. Deser, R. Jackiw, and S.Templeton,
{\it Topologically massive gauge theories},
Annals Phys. {\bf 140}, 372-411 (1982).
%
\bibitem{Jackiw}
R. Jackiw,
{\it Topological Investigations of Quantized Gauge Theories}
in {\it Current Algebra and Anomalies},
(World Scientific Publishing, Singapore, 1985).
%
\bibitem{GoldJack}
J. Goldstone and R. Jackiw,
{\it Unconstrained temporal gauge for Yang-Mills theory},
Phys. Lett. {\bf B 74}, 81-84 (1978).
%
\bibitem{Faddeev79}
A.G. Izergin, V.F. Korepin, M.E. Semenov-Tyan-Shanskii, and L.D. Faddeev,
{\it On gauge fixing conditions in the Yang-Mills theory},
Theor. Math. Phys. {\bf 38}, 1-9 (1979);
[Teor. Mat. Fiz. {\bf 38 }, 3-14 (1979)].
%
\bibitem{DasKakuTown}
A. Das, M. Kaku, and P.K. Townsend,
{\it Gauge fixing ambiguities, flux strings, and the unconstrained Yang-Mills
theory}, Nucl. Phys. {\bf B 149}, 109-122 (1979).
%
\bibitem{Muz}
M. Creutz, I.J. Muzinich, and T.N. Tudron,
{\it Gauge fixing and canonical quantization},
Phys. Rev. {\bf 19 }, 531-539 (1979).
%
\bibitem{ChrLee}
N.H. Christ and T.D. Lee,
{\it Operator ordering and Feynman rules in gauge theories},
Phys. Rev. {\bf D 22}, 939-958 (1980).
%
\bibitem{Simon}
Yu. Simonov,
{\it QCD Hamiltonian in the polar representation},
Sov. J. Nucl. Phys. {\bf 41}, 835-841 (1985);
[Yad. Fiz. {\bf 41}, 1311-1323 (1985)].
%
\bibitem{Simonov}
Yu. Simonov,
{\it Gauge invariant formulation of \myHighlight{$SU(2)$}\coordHE{} gluodynamcs},
Sov. J. Nucl. Phys. {\bf 41}, 1014-1019 (1985)
[Yad. Fiz. {\bf 41}, 1601-1610, (1985)].
%
\bibitem{Tav}
V.V. Vlasov, V.A. Matveev, A.N. Tavkhelidze, S.Yu. Khlebnikov,
and M.E. Shaposhnikov,
{\it Canonical quantization of gauge theories with scalar condensate and
the problem of spontaneous symmetry breaking},
Phys. Elem. Part. Nucl. {\bf 18}, 5-38 (1987).
%
\bibitem{Haller}
K. Haller,
{\it Yang-Mills theory and quantum chromodynamics in the temporal gauge},
Phys. Rev. {\bf D 36}, 1839-1845 (1987).
%
\bibitem{Anishetty}
R. Anishetty,
{\it Local dynamics on gauge invariant basis of nonabelian gauge theories},
Phys. Rev. {\bf D 44}, 1895-1896 (1991).
%
\bibitem{Newman2}
E.T. Newman and C. Rovelli,
{\it Generalized lines of force as the gauge invariant degrees of freedom
for general relativity and Yang-Mills theory},
Phys. Rev. Lett {\bf 69}, 1300-1303 (1992).
%
\bibitem{Lunev}
F.A. Lunev,
{\it Four-dimensional Yang-Mills theory in local gauge invariant variables},
Mod. Phys. Lett. {\bf A 9}, 2281-2292 (1994).
%
\bibitem{KJohnson}
M. Bauer, D.Z. Freedman, and P.E. Haagensen,
{\it Spatial geometry of the electric field representation of nonabelian
gauge theories},
Nucl. Phys. {\bf B 428}, 147-168 (1994);\\
%
P.E. Haagensen and K. Johnson,
{\it Yang-Mills fields and Riemannian geometry},
Nucl. Phys.  {\bf B 439}, 597-616 (1995);\\
%
R. Schiappa,
{\it Supersymmetric Yang-Mills theory and Riemannian geometry},
Nucl. Phys. {\bf B 517}, 462-484 (1998).
%
\bibitem{Nachbe}
H. Nachbagauer,
{\it Finite temperature formalism for nonabelian gauge theories in the
physical phase space},
Phys. Rev. {\bf D 52}, 3672-3678, (1995).
%
\bibitem{Chechelashvili}
G. Chechelashvili, G. Jorjadze, and N. Kiknadze,
{\it Practical scheme of reduction to gauge invariant variables},
Theor. Math. Phys. {\bf 109}, 1316-1328 (1997);
[Teor. Mat. Fiz. {\bf 109} 90-106 (1996)].
%
\bibitem{Lavelle}
M. Lavelle and D. McMullan,
{\it Constituent quarks from QCD},
Phys. Rep. {\bf 279}, 1-65 (1997).
%
\bibitem{Horan}
R. Horan, M. Lavelle, and D. McMullan,
{\it Charges in gauge theories},
Pramana {\bf 51}, 317-355 (1998).
%
\bibitem{KP}
A.M. Khvedelidze and H.-P. Pavel,
{\it Unconstrained Hamiltonian formulation of \myHighlight{$SU(2)$}\coordHE{} gluodynamics},
Phys. Rev. {\bf D 59} 105017 (1999).
%
\bibitem{ErrKP}
Erratum to \cite{KP}: The last line of (3.42) should be replaced by
\myHighlight{$\left(\Gamma^1_{\ 23}\, \phi_3 + \Gamma^1_{\ 32}\, \phi_2 +
\Gamma^1_{\ [23]}\, \phi_1 - g\, \phi_2\, \phi_3\right)^2$}\coordHE{}.
Correspondingly, in formula (4.6) the term
\myHighlight{$\left(\Gamma^3_{\ [12]}\, \phi_3\right)^2$}\coordHE{}
should be added.
Finally, in formulae (4.8), (4.10) and (4.11) the term
\myHighlight{$\left({\mathbf n}\cdot \mbox{rot}\ {\mathbf n} \right)^2$}\coordHE{}
should be included.
The correct formulae are given in the present work in
(\ref{Vinhom1}),(\ref{V21}), and (\ref{V22}) correspondingly.
%
\bibitem{Majumdar}
P. Majumdar and H.S. Sharatchandra,
{\it Reformulating Yang-Mills theory in terms of local gauge invariant variables},
Nucl. Phys. Proc. Suppl. {\bf 94}, 715-717 (2001).
%
\bibitem{Dirac}
P.A.M. Dirac,
{\it The theory of gravitation in Hamiltonian form},
Proc. Roy. Soc. {\bf A 246}, 333-343 (1958);\\
%
{\it Fixation of coordinates in the Hamiltonian theory of gravitation},
Phys. Rev. {\bf 114}, 924-930 (1959);\\
%
{\it Energy of the gravitational field},
Phys. Rev. Lett. {\bf 2}, 368-371 (1959).
%
\bibitem{ADM}
R. Arnowitt, S. Deser, and C.W. Misner,
{\it Consistensy of the canonical reduction of general relativity},
J. Math. Phys. {\bf 1}, 434-439 (1960).
%
\bibitem{AHG}
A. Khvedelidze, H.-P. Pavel, and  G. R\"opke,
{\it Unconstrained \myHighlight{$SU(2)$}\coordHE{} Yang-Mills quantum mechanics with theta angle},
Phys. Rev. {\bf D 61}, 025017, (2000).
%
\bibitem{KMPR}
A.M. Khvedelidze, D.M. Mladenov, H.-P. Pavel, and G. R\"opke,
{ \it On unconstrained \myHighlight{$SU(2)$}\coordHE{} gluodynamics with theta angle},
Eur. Phys. J. {\bf C 24}, 137-141 (2002).
%
\bibitem{Whitehead}
J.H.C. Whitehead,
{\it An expression of Hopf's invariant as an integral},
Proc. Nat. Acad. Sci. USA {\bf 33}, 117-123 (1947).
%
\bibitem{Woltier}
L. Woltier,
Proc. Nat. Acad. Sci. USA {\bf 44}, 489 (1958).
%
\bibitem{Moffat}
H. Moffat, {\it The degree of knottedness of tangled vortex lines}
J. Fluid Mech. {\bf 35}, 117-129 (1969).
%
\bibitem{KuzMikh}
E.A. Kuznetsov and A.V. Mikhailov,
{\it On the topological meaning of canonical Clebsch variables},
Phys. Lett. {\bf A 77} 37-38 (1980).
%
\bibitem{Saffman}
P.G. Saffman,
{\it Vortex Dynamics},
(Cambridge University Press, Cambrigde, UK, 1992).
%
\bibitem{JackiwPi}
R. Jackiw and S.Y. Pi,
{\it Creation and evolution of magnetic helicity},
Phys. Rev. {\bf D 61}, 105015 (2000).
%
\bibitem{NairJackiw}
R. Jackiw, V.P. Nair, and S.Y. Pi,
{\it Chern-Simons reduction and non-Abelian fluid mechanics},
Phys. Rev. {\bf D 62}, 085018 (2000).
%
\bibitem{DiracL}
P.A.M. Dirac,
{\it Lectures on Quantum Mechanics},
(Belfer Graduate School of Science, Yeshiva University Press, New York, 1964).
%
\bibitem{HenTeit}
M. Henneaux and C. Teitelboim,
{\it Quantization of Gauge Systems},
(Princeton University Press, Princeton, NJ, 1992).
%
\bibitem{Creutz}
M. Creutz,
{\it Quarks, Gluons and Lattices},
(Cambridge Univ. Press, Cambridge, 1983).
%
\bibitem{ORaf}
L. O'Raifeartaigh,
{\it Group Structure of Gauge Theories},
(Cambridge University Press, Cambridge, UK, 1986).
%
\bibitem{AD}
A.M. Khvedelidze and D.M. Mladenov,
{\it
Generalized Calogero-Moser-Sutherland models from geodesic motion on
\myHighlight{$GL(n, {R})$}\coordHE{} group manifold},
Phys. Lett. {\bf A 299}, 522-530 (2002);
\bibitem{FaddeevNiemi}
L. Faddeev and A.J. Niemi,
{\it Partially dual variables in \myHighlight{$SU(2)$}\coordHE{} Yang-Mills theory},
Phys. Rev. Lett. {\bf 82}, 1624-1627 (1999).
%
\bibitem{Langmann}
E, Langmann and A.J. Niemi,
{\it Towards a string representation of infrared \myHighlight{$SU(2)$}\coordHE{} Yang-Mills theory},
Phys. Lett. {\bf B 463}, 252-256 (1999).
%
\bibitem{BaalWipf}
P. van Baal and A. Wipf,
{\it Classical gauge vacua as knots},
Phys. Lett. {\bf B 515}, 181-184 (2001).
%
\bibitem{Battye:1998}
R.A. Battye and P. Sutcliffe,
{\it Solitons, links and knots},
Proc. Roy. Soc. Lond. {\bf A 455}, 4305-4331 (1999).
%
%
\bibitem{FadSlav}
L.D. Faddeev, A.A. Slavnov,
{\it Gauge Fields: Introduction to Quantum Theory},
(Benjamin-Cummings, 1984).
%
\bibitem{Lancaster}
P. Lancaster,
{\it Theory of Matricies},
(Academic Press, New York-London, 1969).
%
\bibitem{Gantmacher}
F.R. Gantmacher,
{\it Matrix Theory},
(Chelsea, New York, 1959).
%
\bibitem{BjorkenDrell}
J.D. Bjorken and S.D. Drell,
{\it Relativistic Quantum Fields},
(New York, McGraw Hill, 1965).
%
\bibitem{HansonReggeTeitelboim}
A.J. Hanson, T. Regge, and C.Teitelboim,
{\it Constrained Hamiltonian Systems},
(Accademia Nazionale dei Lincei, Rome, 1976).
%
\bibitem{PavelPervushin}
H.-P. Pavel and V.N. Pervushin,
{\it Reduced phase space quantization of massive vector theory},
Int. J. Mod. Phys. {\bf A 14}, 2285-2308 (1999).
\end{thebibliography}
\end{document}

%%%%%%%%%%%%%%%%%%%%%%%%%%%%%%%%%% END %%%%%%%%%%%%%%%%%%%%%%%%%%%%%%%%%%%%%%%%%%%%%

\bye
