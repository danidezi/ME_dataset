
\documentclass[a4paper,aps,twocolumn,amsfonts]{revtex4}
   %\usepackage{epsfig}
   \usepackage{useful_macros}
\begin{document}

   \title{Statistical Quantum Mechanics of Many Universes}
   \author{J. Gamboa
   }\email {jgamboa@lauca.usach.cl}
   \affiliation{Departamento de F\'{\i}sica, Universidad de Santiago de Chile,
   Casilla 307, Santiago 2, Chile}
   \author{F. M\'endez}
   \email {fmendez@lauca.usach.cl}
   \affiliation{Departamento de F\'{\i}sica, Universidad de Santiago de Chile,
   Casilla 307, Santiago 2, Chile}
   \begin{abstract}
   The quantum statistical mechanics of generally covariant systems --particles, strings and membranes-- on noncommutative field spaces is studied.  We discuss how to introduce non-local communication among different systems via noncommutativity. This idea is applied to cosmology where we argue that due to the breaking of relativistic invariance  one can consider a privileged  reference system where many universes interact as a quantum gas in a reservoir. If roughly  speaking, we approximate the universes as tensionless membranes, then,  the interaction among universes 
   provided by noncommutativity is harmonic. The oscillation frequency for each universe is proportional to \myHighlight{$B/M$}\coordHE{}, where \myHighlight{$B$}\coordHE{} is the  noncommutativity parameter  --that we identify as the primordial magnetic field,  {\it i.e.}  \myHighlight{$\sim 10^{-16}\, \mbox{GeV}^2$}\coordHE{}- and \myHighlight{$M$}\coordHE{} is the mass of the 
   universe ( \myHighlight{$\sim 10^{77} \mbox{GeV}$}\coordHE{}) and, therefore each universe have the  pulsation frequency  \myHighlight{$\omega \sim 10^{-68} \,s^{-1}$}\coordHE{}.  
   \end{abstract}
   \pacs{ PACS numbers:03.65.-w, 03.65.Db}
   \maketitle

   \section{Introduction}
   Noncommutative fields might provide of important consequences in our explanations of the physical world. Although noncommutativity in the field space induces a violation of the microcausality principle, it also provides of an explicit mechanism for non-local communication at quantum field theory level.

   More precisely  causality and Lorentz invariance are implemented in a standard quantum field theory assuming that the commutator 
   \begin{equation}\coord{}\boxEquation{
   \left[ \phi ({\bf x}), \phi ({\bf x}^{'})\right], \label{1}
   }{
   \left[ \phi ({\bf x}), \phi ({\bf x}^{'})\right], }{ecuacion}\coordE{}\end{equation}
   vanishes  if 
   \begin{equation}\coord{}\boxEquation{
   (x - x^{'})^2 \leq 0,  \label{2}
   }{
   (x - x^{'})^2 \leq 0,  }{ecuacion}\coordE{}\end{equation}
otherwise such properties cannot be preserved \cite{kos} . 

   The deformation of the field's canonical algebra yield to very interesting phenomenological consequences such as; a) Non-trivial dispersion relation \cite{ccgm}  that could explain cosmic ray physics and the violation of the GZK cut off \cite{gzk}, b) a possible explanation of the matter-antimatter asymmetry \cite{rubi}  , c) a new approach to phenomenological relics of quantum gravity \cite{alot}.

   But this lack of causality bring another consequences that, {\it a priori}, could be seem very speculative. Indeed, noncommutativity in the field space implement non-local communication and, therefore, at cosmological level could allow interaction among universes.

Technically speaking, the violation of causality means that, instead of (\ref{1}) and (\ref{2}), one should have
   \begin{equation}\coord{}\boxEquation{
   \left[ \phi_i ({\bf x}), \phi_j ({\bf x}^{'})\right] \neq 0,
   \label{3}
   }{
   \left[ \phi_i ({\bf x}), \phi_j ({\bf x}^{'})\right] \neq 0,
   }{ecuacion}\coordE{}\end{equation}
   where \myHighlight{$\phi_i ({\bf x})$}\coordHE{} are the components of a  charged scalar field.

   Although the RHS of (\ref{3}) is unknown, one we can demand that the violation of causality is just a tiny departure of the standard case. Then,  the most simple choice for the RHS in (\ref{3}) is
   \begin{equation}\coord{}\boxEquation{
   \left[ \phi_i ({\bf x}), \phi_j ({\bf x}^{'})\right] = i \epsilon_{ij} ~\theta~
   \delta ( {\bf x} -{\bf x}^{'}),
   \label{4}
   }{
   \left[ \phi_i ({\bf x}), \phi_j ({\bf x}^{'})\right] = i \epsilon_{ij} ~\theta~
   \delta ( {\bf x} -{\bf x}^{'}),
   }{ecuacion}\coordE{}\end{equation}
   where \myHighlight{$\theta$}\coordHE{} is a constant parameter that measure the violation of causality.

   The equation (\ref{4}) is the definition of the {\it noncommutative fields} as an extension of
   \[\coord{}\boxMath{
   \left[ x,y\right] = i~ \theta,
   }{corchetes}{0pt}\coordE{}\]
   in noncommutative field theory.

   The equation (\ref{4}) mean that two regions that are causally disconnected, could interchange information if we admit a small noncommutativity  in the phase space of fields.  This fact means that, eventually,  non- interacting systems could interact via a noncommutative structure of the phase space of field. Thus, noncommutativity provide of a very natural way  for introducing interactions.

Although this last fact might be a problem  in a relativistic theory, in our case  the Lorentz invariance is explicitly broken once (\ref{4}) is assumed and, therefore, could be interesting to consider physical realizations of this  possibility.

A realization of this last fact could occur in cosmology. Indeed, as the gravity field is given by the metric, then the microcausality principle imply that
\begin{equation}\coord{}\boxEquation{
\left[ g_{ij}({\bf x}), g_{kl}({\bf x}^{'})\right] = 0. \label{ww}
}{
\left[ g_{ij}({\bf x}), g_{kl}({\bf x}^{'})\right] = 0. }{ecuacion}\coordE{}\end{equation}

Thus, from here, one concludes that our universe is isolated of other universes and, therefore, unless one break explicitly (\ref{ww}) the interaction among universes is forbidden by relativistic  invariance.

If we admit that relativistic invariance is broken in the sense of (\ref{4}), then our universe is only one of the many possible universes contained in a sort of reservoir, {\it i.e.} a gas of universes. In this reservoir one can define an evolution  parameter \myHighlight{$s$}\coordHE{} which may coincide with the conventional time and, therefore, would allow  an evolution operator definition such as in quantum mechanics.

   From this point of view, let us assume that the universe \myHighlight{$i$}\coordHE{} is described by a metric \myHighlight{$g$}\coordHE{} and has a field  \myHighlight{$\Psi_i(g)$}\coordHE{}, then, the condition for non-local communication  among universes is
   \begin{equation}\coord{}\boxEquation{
   \left[ \Psi_i (g), \Psi_j (g^{'}) \right] = i \theta_{ij} \delta (g,g^{'}),
   \label{5}
   }{
   \left[ \Psi_i (g), \Psi_j (g^{'}) \right] = i \theta_{ij} \delta (g,g^{'}),
   }{ecuacion}\coordE{}\end{equation}
   with \myHighlight{$i,j= 1, 2, 3, \dots$}\coordHE{}.

   The possibility \myHighlight{$\theta_{ij}\neq 0$}\coordHE{} could give information about the existence of other universes  and it would provides of an explicit evidence on the causality principle violation at very high energies.

   The purpose of this paper is, firstly,  to construct noncommutative versions of generally covariant systems and, secondly, to elaborate the approach sketched above for cosmology as a consequence of a weak causality principle violation.

   The paper is organized as follow: in section II we consider relativistic particles  in a noncommutative space and we give an explicit expression for the partition function. In section III, the non-commutative \myHighlight{$p$}\coordHE{}-dimensional membrane in the strong coupling limit is considered and we argue how this model can be applied to cosmology. Finally, in section 5 we give our conclusions and outlook.

   \section{Noncommutative relativistic quantum mechanics}

   In this section we will construct noncommutative versions of generally covariant systems. We will start considering, firstly, the relativistic particle on a \myHighlight{$D$}\coordHE{}-dimensional spacetime and later-- in the next section -- we will extend our results to tensionless strings and membranes.

   \subsection{Relativistic free particle and the proper-time gauge}

   There are many approaches to understand the relativistic quantum mechanics of the free particle. One of them is the so called proper-time method, which was used   in the early 50th in connection with quantum electrodynamics \cite{nambu}. One start considering a particle in a \myHighlight{$D+1$}\coordHE{}- dimensional Euclidean  spacetime.

   The diffusion equation for such system is
   \begin{equation}\coord{}\boxEquation{
   -\frac{1}{2} \Box \varphi (x, s) =  \frac{\partial \varphi}{\partial s},
   \label{6}
   }{
   -\frac{1}{2} \Box \varphi (x, s) =  \frac{\partial \varphi}{\partial s},
   }{ecuacion}\coordE{}\end{equation}
   where \myHighlight{$\Box$}\coordHE{} is the \myHighlight{$D$}\coordHE{}-dimensional Laplacian.

   Then, using the ansatz
   \begin{equation}\coord{}\boxEquation{
   \varphi (x,s) = e^{-\frac{m^2}{2} s} \phi (x), \label{7}
   }{
   \varphi (x,s) = e^{-\frac{m^2}{2} s} \phi (x), }{ecuacion}\coordE{}\end{equation}
one find the Klein-Gordon equation if \myHighlight{$m$}\coordHE{} is the mass of the particle.

In this approach, the propagation amplitude is  the Laplace transform
   \begin{equation}\coord{}\boxEquation{
   G[x,x^{'}; m^2] = \int_0^\infty d s e^{-s\frac{m^2}{2} } \, G[x,x^{'}; s],
   \label{8}
   }{
   G[x,x^{'}; m^2] = \int_0^\infty d s e^{-s\frac{m^2}{2} } \, G[x,x^{'}; s],
   }{ecuacion}\coordE{}\end{equation}
   where
   \begin{eqnarray}\coord{}\boxAlignEqnarray{\leftCoord{}
   G[x,x^{'}; s] &=& \int {\cal D} x \rightCoord{}\, e^{- \int_0^1 d \tau \frac{\leftCoord{}{\dot
   x}^2}{2 s}}, \rightCoord{}\label{9} \nonumber\rightCoord{}
   \rightCoord{}\\
&\leftCoord{}=& s^{-D/2} \rightCoord{}\, e^{- \frac{\leftCoord{}(\Delta x)^2}{\rightCoord{}2 s}}.\rightCoord{}\label{99}
\rightCoord{}}{0mm}{4}{9}{
   G[x,x^{'}; s] &=& \int {\cal D} x \, e^{- \int_0^1 d \tau \frac{{\dot
   x}^2}{2 s}}, \\
&=& s^{-D/2} \, e^{- \frac{(\Delta x)^2}{2 s}}.}{1}\coordE{}\end{eqnarray}

   From this one obtain the partition function for a gas of \myHighlight{$N$}\coordHE{} free relativistic particles \footnote{In this paper we will ignore  the Gibbs factor \myHighlight{$1/N!$}\coordHE{}. The reader should note also that we are assuming the Maxwell-Boltzmann statistics for the particles.}
   \begin{equation}\coord{}\boxEquation{
   Z_s=\left(\mbox{Tr}\left[e^{-\frac{m^2}{2} s} G[x,x';s]\right]\right)^N, 
   }{
   Z_s=\left(\mbox{Tr}\left[e^{-\frac{m^2}{2} s} G[x,x';s]\right]\right)^N, 
   }{ecuacion}\coordE{}\end{equation}
   or equivalently 
   \begin{equation}\coord{}\boxEquation{
   \ln Z = N \left[ - \frac{m^2}{2} s- \frac{D}{2} \ln s   + \ln {\cal V} \right],
   \label{10}
   }{
   \ln Z = N \left[ - \frac{m^2}{2} s- \frac{D}{2} \ln s   + \ln {\cal V} \right],
   }{ecuacion}\coordE{}\end{equation}
   where \myHighlight{${\cal V}=V \times \mbox{const.}$}\coordHE{} is the \myHighlight{$D$}\coordHE{}-dimensional spacetime, \myHighlight{$V$}\coordHE{} is the \myHighlight{$D-1$}\coordHE{}-dimensional ordinary spatial volume and \myHighlight{$s$}\coordHE{} play the role of \myHighlight{$\beta =1/kT$}\coordHE{}.

   \subsection{The relativistic particle in a noncommutative space}

   The equation (\ref{6}) suggest a simple way to extent the problem to a relativistic particles gas on a noncommutative space.

   Indeed,  from (\ref{6}) one see that the Hamiltonian for a relativistic particle is
   \begin{equation}\coord{}\boxEquation{
   {\hat H} =\frac{1}{2} p^2_\mu.  \label{ham}
   }{
   {\hat H} =\frac{1}{2} p^2_\mu.  }{ecuacion}\coordE{}\end{equation}

   Once (\ref{ham}) is given, one implement noncommutativity from the deformed algebra
   \begin{eqnarray}\coord{}\boxAlignEqnarray{\leftCoord{}
   \left[ x_\mu, x_\nu\right] &=& i \theta_{\mu \nu},
   \leftCoord{}\rightCoord{}\,\rightCoord{}\,\rightCoord{}\,\rightCoord{}\,\rightCoord{}\,\rightCoord{}\,\rightCoord{}\,\rightCoord{}\,\rightCoord{}\,\rightCoord{}\,\rightCoord{}\, \left[ p_\mu, p_\nu\right] = i B_{\mu \nu}, \rightCoord{}\label{11}
   \rightCoord{}\\\leftCoord{}
   \left[ x_\mu, p_\nu\right] &=& i \delta_{\mu \nu}, \rightCoord{}\label{12}
\rightCoord{}}{0mm}{3}{16}{
   \left[ x_\mu, x_\nu\right] &=& i \theta_{\mu \nu},
   \,\,\,\,\,\,\,\,\,\,\, \left[ p_\mu, p_\nu\right] = i B_{\mu \nu}, \\
   \left[ x_\mu, p_\nu\right] &=& i \delta_{\mu \nu}, }{1}\coordE{}\end{eqnarray}
   where \myHighlight{$\theta_{\mu \nu}$}\coordHE{} and  \myHighlight{$B_{\mu \nu}$}\coordHE{} are the deformation parameters in the phase space.

   By convenience  we choose the gauge
   \begin{eqnarray}\coord{}\boxAlignEqnarray{\leftCoord{}
   \theta_{i0} &=& 0\rightCoord{}\, \rightCoord{}\, \rightCoord{}\, \rightCoord{}\,\rightCoord{}\, \rightCoord{}\, \rightCoord{}\, \rightCoord{}\,  \theta_{ij} = \epsilon_{ij}
   \theta, \rightCoord{}\label{13}
   \rightCoord{}\\\leftCoord{}
   B_{i0} &=& 0, \rightCoord{}\, \rightCoord{}\, \rightCoord{}\, \rightCoord{}\,\rightCoord{}\, \rightCoord{}\, \rightCoord{}\, \rightCoord{}\, B_{ij} = \epsilon_{ij}  B.\rightCoord{}
   \label{14}
\rightCoord{}}{0mm}{2}{21}{
   \theta_{i0} &=& 0\, \, \, \,\, \, \, \,  \theta_{ij} = \epsilon_{ij}
   \theta, \\
   B_{i0} &=& 0, \, \, \, \,\, \, \, \, B_{ij} = \epsilon_{ij}  B.
   }{1}\coordE{}\end{eqnarray}

   Therefore, the equation of motion for this particle is
   \begin{eqnarray}\coord{}\boxAlignEqnarray{\leftCoord{}
   {\rightCoord{}\leftCoord{}\dot x}_\mu &=& p_\mu, \nonumber\rightCoord{}
   \rightCoord{}\\\leftCoord{}
   {\rightCoord{}\leftCoord{}\dot p}_i &=&  \epsilon_{ij} B p_j.  \rightCoord{}\label{15}
\rightCoord{}}{0mm}{4}{7}{
   {\dot x}_\mu &=& p_\mu, \\
   {\dot p}_i &=&  \epsilon_{ij} B p_j.  }{1}\coordE{}\end{eqnarray}

   This equations can be integrated directly by using (\ref{13}) and (\ref{14}). Indeed, one of the equations is trivial, namely, the energy conservation condition (\myHighlight{$\dot{p}_0=0$}\coordHE{}).

   The remaining equations have the solution
   \begin{eqnarray}\coord{}\boxAlignEqnarray{\leftCoord{}
   p_1 &=& \frac{\leftCoord{}1}{\rightCoord{}2} \left( \alpha~ e^{-i B t} + \alpha^\dag ~ e^{ i B t}
   \right), \nonumber\rightCoord{}
   \rightCoord{}\\\leftCoord{}
   p_2 &=& \frac{\leftCoord{}1}{\rightCoord{}2i} \left( \alpha ~e^{- i B t} - \alpha^\dag~ e^{ i B
   t} \right), \rightCoord{}\label{16}
\rightCoord{}}{0mm}{4}{7}{
   p_1 &=& \frac{1}{2} \left( \alpha~ e^{-i B t} + \alpha^\dag ~ e^{ i B t}
   \right), \\
   p_2 &=& \frac{1}{2i} \left( \alpha ~e^{- i B t} - \alpha^\dag~ e^{ i B
   t} \right), }{1}\coordE{}\end{eqnarray}
   where  \myHighlight{$\alpha$}\coordHE{}'s are constant operators.

   The coordinates \myHighlight{$x_ {1,2}$}\coordHE{} are obtained in a similar way using (\ref{15}), {\it i.e.}
   \begin{eqnarray}\coord{}\boxAlignEqnarray{\leftCoord{}
   x_1 &=& \frac{\leftCoord{}1}{\rightCoord{}2 i B} \left(\alpha^\dag~ e^{ i B t} -\alpha~ e^{- i B t}  \right) + x_{01}, \nonumber\rightCoord{}
   \rightCoord{}\\\leftCoord{}
   x_2 &=& \frac{\leftCoord{}1}{\rightCoord{}2  B} \left( \alpha~ e^{- i B t} + \alpha^\dag~ e^{ i B t} \right)+ x_{02}. \rightCoord{}\label{17}
\rightCoord{}}{0mm}{4}{7}{
   x_1 &=& \frac{1}{2 i B} \left(\alpha^\dag~ e^{ i B t} -\alpha~ e^{- i B t}  \right) + x_{01}, \\
   x_2 &=& \frac{1}{2  B} \left( \alpha~ e^{- i B t} + \alpha^\dag~ e^{ i B t} \right)+ x_{02}. }{1}\coordE{}\end{eqnarray}

   From the commutation relation of \myHighlight{$p$}\coordHE{}'s, one find that is possible to define  operators \myHighlight{$a$}\coordHE{} and \myHighlight{$a^{\dagger}$}\coordHE{} satisfying the algebra
   \begin{eqnarray}\coord{}\boxAlignEqnarray{\leftCoord{}
   \left[a, a \right] &=& 0=\left[a^\dag, a^\dag \right], \nonumber\rightCoord{}
   \rightCoord{}\\\leftCoord{}
   \left[a, a^\dag \right] &=& 1 \rightCoord{}\label{18}, \rightCoord{}
\rightCoord{}}{0mm}{2}{6}{
   \left[a, a \right] &=& 0=\left[a^\dag, a^\dag \right], \\
   \left[a, a^\dag \right] &=& 1 , 
}{1}\coordE{}\end{eqnarray}
   where
   \[\coord{}\boxMath{
   \alpha \rightarrow \sqrt{B} a,\,\,\,\,\,\,\,\,\,\,\,\,\,\,\,\,\,\, \alpha^{\dag}
   \rightarrow \sqrt{B}a^{\dag}.
   }{corchetes}{0pt}\coordE{}\]

   The equations of motion --as a second order equation system-- are
   \[\coord{}\boxMath{
   \ddot{x}_\mu=B_{\mu\nu}\dot{x}_\nu,
   }{corchetes}{0pt}\coordE{}\]
   which can be solved by the Ansatz \myHighlight{$x_\mu = a_\mu \,e^{i\omega s}$}\coordHE{}.

   The last equation is
   \[\coord{}\boxMath{
   (i\omega\delta_{\mu\nu} -B_{\mu\nu})a_\nu=0.
   }{corchetes}{0pt}\coordE{}\]

   Therefore, the dispersion relation for this system is
   \begin{equation}\coord{}\boxEquation{
   \omega_{\pm}=\left\{
   \begin{array}{c}
   \pm B \\
   0
   \end{array}
   \right. , \label{dis}
   }{
   \omega_{\pm}=\left\{
   \begin{array}{c}
   \pm B \\
   0
   \end{array}
   \right. , }{ecuacion}\coordE{}\end{equation}
   and --as one of the eigenvalues vanishes-- the Hamiltonian spectrum is degenerated

   Thus, the  hamiltonian for a relativistic particle living on a noncommutative space is
   \begin{equation}\coord{}\boxEquation{
   H=\frac{B}{2}\left( a^\dag a +\frac{1}{2}\right)
   +\frac{1}2\sum_{n=1}^{D-3}(p_\mu^2)_n.
   \label{23}
   }{
   H=\frac{B}{2}\left( a^\dag a +\frac{1}{2}\right)
   +\frac{1}2\sum_{n=1}^{D-3}(p_\mu^2)_n.
   }{ecuacion}\coordE{}\end{equation}

   Finally, the statistical mechanics for a gas of \myHighlight{$N$}\coordHE{} relativistic particles on a noncommutative space is obtained from the partition function
   \begin{eqnarray}\coord{}\boxAlignEqnarray{\leftCoord{}
   Z_s&=&\left( s^{-\frac{\leftCoord{}D-3}{\rightCoord{}2}}e^{-\frac{\leftCoord{}m^2}{\rightCoord{}2}s} \sum_{\rightCoord{}n=0}^{\leftCoord{}{}\leftCoord{}\infty}
   {\rightCoord{}\leftCoord{}\cal G}_0 e^{-s\frac{\leftCoord{}B}{\rightCoord{}2}(n+\frac{\leftCoord{}
   \leftCoord{}1}{2})} \right)^N,\nonumber\rightCoord{}
   \rightCoord{}\\
&\leftCoord{}=&\left[\frac{\leftCoord{}{\cal G}_0   e^{-\frac{\leftCoord{}m^2}{\rightCoord{}2}s}
   s^{-\frac{\leftCoord{}D-3}{\rightCoord{}2}}}{\sinh{(\frac{\leftCoord{}B}{\rightCoord{}2} s)}} \right]^N, \rightCoord{}
   \label{25}
\rightCoord{}}{0mm}{14}{13}{
   Z_s&=&\left( s^{-\frac{D-3}{2}}e^{-\frac{m^2}{2}s} \sum_{n=0}^{{}\infty}
   {\cal G}_0 e^{-s\frac{B}{2}(n+\frac{
   1}{2})} \right)^N,\\
&=&\left[\frac{{\cal G}_0   e^{-\frac{m^2}{2}s}
   s^{-\frac{D-3}{2}}}{\sinh{(\frac{B}{2} s)}} \right]^N, 
   }{1}\coordE{}\end{eqnarray}
   where \myHighlight{${\cal G}_0$}\coordHE{} is the degeneracy factor due to the zero eigenvalue of the Hamiltonian \footnote{Although this factor can be computed by using a regularization prescription, here this factor is absorved as a normalization constant.}.

   The thermodynamical properties of this system are directly computed from (\ref{25}).

   \section{The strong coupling regime for membranes in noncommutative spaces}

   In this section we will discuss the extension  of the previous problem to membranes moving on a noncommutative space in the strong coupling regime.

   A relativistic membrane is a \myHighlight{$p$}\coordHE{}-dimensional object embedded on a \myHighlight{$D$}\coordHE{}-dimensional spacetime.

   The hamiltonian constraints are
   \begin{eqnarray}\coord{}\boxAlignEqnarray{\leftCoord{}
   H_{\perp}&=&\frac{\leftCoord{}1}{\rightCoord{}2}(p^2 + T^2 g),\rightCoord{}\label{26}
   \rightCoord{}\\\leftCoord{}
   H_i&=&p_\mu\partial _i x^\mu.\rightCoord{}\label{27}
\rightCoord{}}{0mm}{3}{6}{
   H_{\perp}&=&\frac{1}{2}(p^2 + T^2 g),\\
   H_i&=&p_\mu\partial _i x^\mu.}{1}\coordE{}\end{eqnarray}

   The strong coupling regime correspond to \myHighlight{$T\rightarrow 0$}\coordHE{}. In this limit, the membrane becomes an infinite set of massless relativistic particles moving on a \myHighlight{$D$}\coordHE{}-spacetime satisfying the condition (\ref{27}).

   Using, this philosophy we will start constructing tensionless strings.

   \subsection{Tensionless strings from particles}

   Let us start by noticing that a tensionless string \cite{gamboa} is made up of infinite massless relativistic particles causally disconnected and, therefore, instead of (\ref{6}) one have
   \begin{eqnarray}\coord{}\boxAlignEqnarray{\leftCoord{}
   \leftCoord{}-\frac{\leftCoord{}1}{\rightCoord{}2}\Box \varphi_1 (x, s_1)&=& \frac{\leftCoord{}\partial
   \varphi_1}{\partial s_1},\nonumber\rightCoord{}
   \rightCoord{}\\\leftCoord{}
   \leftCoord{}-\frac{\leftCoord{}1}{\rightCoord{}2}\Box \varphi_2 (x, s_2) &=&  \frac{\leftCoord{}\partial
   \varphi_2}{\partial s_2},\nonumber\rightCoord{}
   \rightCoord{}\\
&\leftCoord{}\vdots& \nonumber\rightCoord{}
   \rightCoord{}\\\leftCoord{}
   \leftCoord{}-\frac{\leftCoord{}1}{\rightCoord{}2}\Box \varphi_k (x, s_k) &=&  \frac{\leftCoord{}\partial
   \varphi_k}{\partial s_k}.\rightCoord{}
\rightCoord{}}{0mm}{13}{12}{
   -\frac{1}{2}\Box \varphi_1 (x, s_1)&=& \frac{\partial
   \varphi_1}{\partial s_1},\\
   -\frac{1}{2}\Box \varphi_2 (x, s_2) &=&  \frac{\partial
   \varphi_2}{\partial s_2},\\
&\vdots& \\
   -\frac{1}{2}\Box \varphi_k (x, s_k) &=&  \frac{\partial
   \varphi_k}{\partial s_k}.
}{1}\coordE{}\end{eqnarray}

   These equations can be solved  by generalizing the Ansatz (\ref{7}), {\it i.e}
   \begin{equation}\coord{}\boxEquation{
   \varphi(x_1,\dots,x_k,\dots;s_1\dots,s_k,\dots)=\prod
   _{i=1}^{\infty}\text{\Large e}^{-\frac{m^2}{2}s_i}\phi(x_i),
   }{
   \varphi(x_1,\dots,x_k,\dots;s_1\dots,s_k,\dots)=\prod
   _{i=1}^{\infty}\text{\Large e}^{-\frac{m^2}{2}s_i}\phi(x_i),
   }{ecuacion}\coordE{}\end{equation}
where \myHighlight{$m^2$}\coordHE{} is an infrared regulator that we will be taken zero at the end of the calculation.

   In the infinite number of particles limit, the propagation amplitude is
   \begin{eqnarray}\coord{}\boxAlignEqnarray{
&\leftCoord{}G&[x(\sigma),x'(\sigma)]=\nonumber\rightCoord{}
   \rightCoord{}\\
&\leftCoord{}=& \int_0^{\infty} {\cal D} s(\sigma)\text{\Large e}^{-\frac{\leftCoord{}m^2}{\rightCoord{}2}\int  d\sigma
   s(\sigma)}G[x(\sigma),x'(\sigma);s(\sigma)],  \rightCoord{}\label{29}
\rightCoord{}}{0mm}{3}{6}{
&G&[x(\sigma),x'(\sigma)]=\\
&=& \int_0^{\infty} {\cal D} s(\sigma)\text{\Large e}^{-\frac{m^2}{2}\int  d\sigma
   s(\sigma)}G[x(\sigma),x'(\sigma);s(\sigma)],  }{1}\coordE{}\end{eqnarray}
   where \myHighlight{$G[x(\sigma),x'(\sigma);s(\sigma)]$}\coordHE{} is given by
   \begin{equation}\coord{}\boxEquation{
   G[x(\sigma),x'(\sigma);s(\sigma)]=s^{-D/2}(\sigma) \text{\Large e}^{-\int
   d\sigma \frac{[\Delta x(\sigma)]^2}{2s(\sigma)}}. \label{299}
   }{
   G[x(\sigma),x'(\sigma);s(\sigma)]=s^{-D/2}(\sigma) \text{\Large e}^{-\int
   d\sigma \frac{[\Delta x(\sigma)]^2}{2s(\sigma)}}. }{ecuacion}\coordE{}\end{equation}

   Using (\ref{29}) and (\ref{299}), the partition function of a tensionless string gas is
   \begin{equation}\coord{}\boxEquation{
   Z[s(\sigma)]=\left(s^{-D/2}\text{\Large e}^{-\int d\sigma
   \frac{m^2}{2}s(\sigma)}\right)^N.
   \label{32}
   }{
   Z[s(\sigma)]=\left(s^{-D/2}\text{\Large e}^{-\int d\sigma
   \frac{m^2}{2}s(\sigma)}\right)^N.
   }{ecuacion}\coordE{}\end{equation}

   This partition function reproduce correctly the results for the thermodynamics  of a tensionless string gas \cite{string}.

   Indeed, from (\ref{32}), the Helmholtz free energy is
   \[\coord{}\boxMath{
   F[s]=\frac{N}{s(\sigma)}\left[ \frac{D}{2} \ln(s(\sigma)) +\frac{m^2}{2}\int
   d\sigma s(\sigma) +\ln({\cal V})
   \right].
   }{corchetes}{0pt}\coordE{}\]

   As  \myHighlight{$1/s$}\coordHE{} is the temperature, then in the limit \myHighlight{$m^2\rightarrow 0 $}\coordHE{} we see that \myHighlight{$F/T \sim \ln(T)$}\coordHE{}, again in agreement with other null string calculations \cite{string,atick}.

   From the last equation one obtain that
   \begin{equation}\coord{}\boxEquation{
   P[s(\sigma)]{ V}=\frac{N}{s(\sigma)},
   }{
   P[s(\sigma)]{ V}=\frac{N}{s(\sigma)},
   }{ecuacion}\coordE{}\end{equation}
   is the state equation for a tensionless string gas.

   \subsection{Tensionless membranes from tensionless strings}

  In order to construct tensionless membranes, one start considering a membrane as an infinite collection of tensionless strings. Thus, if the membrane is p-dimensional with local coordinates \myHighlight{$(\sigma_1,\dots,\sigma_p)$}\coordHE{}, then the propagation amplitude, formally, correspond to  (\ref{29}), but with the substitution
   \[\coord{}\boxMath{
   \sigma\rightarrow (\sigma_1,\dots,\sigma_p).
   }{corchetes}{0pt}\coordE{}\]

   Therefore, the partition function for a gas of \myHighlight{$N$}\coordHE{} tensionless membranes is
   \begin{eqnarray}\coord{}\boxAlignEqnarray{\leftCoord{}
   Z[s(\sigma)]= \left[\lim_{n \to \infty}\left([s(\sigma)]^{-D/2}\text{\Large
   e}^{- \frac{\leftCoord{}m^2}{\rightCoord{}2}\int d^p \sigma s(\sigma)}{\rightCoord{}\cal V} \rightCoord{}
   \right)^n\right]^N, \nonumber \rightCoord{}\label{34}
   \rightCoord{}\\
&&\leftCoord{} \rightCoord{}
\rightCoord{}}{0mm}{3}{8}{
   Z[s(\sigma)]= \left[\lim_{n \to \infty}\left([s(\sigma)]^{-D/2}\text{\Large
   e}^{- \frac{m^2}{2}\int d^p \sigma s(\sigma)}{\cal V} 
   \right)^n\right]^N, \\
&& 
}{1}\coordE{}\end{eqnarray}
   where \myHighlight{$n$}\coordHE{} is the number of tensionless strings.

   As a matter of fact, one should note that (\ref{34}) is doubly regularized by \myHighlight{$m^2$}\coordHE{} and \myHighlight{$n$}\coordHE{}.

   From the last equation one can obtain the state equation
   \begin{equation}\coord{}\boxEquation{
   P[s(\sigma)]~{ V} = \lim_{n\to \infty}\frac{nN}{s(\sigma)}.
   }{
   P[s(\sigma)]~{ V} = \lim_{n\to \infty}\frac{nN}{s(\sigma)}.
   }{ecuacion}\coordE{}\end{equation}

   The Helmholtz free energy has a  similar behavior to the tensionless string case.

   \subsection{Including noncommutativity in membranes}

Using the tensionless strings results, we can generalize our arguments in order to include noncommutativity in membranes.

Thus, the partition function for a gas of \myHighlight{$N$}\coordHE{}- tensionless membranes is
   \begin{eqnarray}\coord{}\boxAlignEqnarray{\leftCoord{}
   Z[s]&=&\mbox{Tr}\left[G[x(\sigma),x'(\sigma);s(\sigma)] \rightCoord{}
   \right]\nonumber\rightCoord{}
   \rightCoord{}\\
&\leftCoord{}=&\biggl( [s(\sigma)]^{-\frac{\leftCoord{}D-3}{\rightCoord{}2}}e^{-\frac{\leftCoord{}m^2}{\rightCoord{}2}\int d^p\sigma
   s(\sigma) } \times \nonumber\rightCoord{}
   \rightCoord{}\\
&\leftCoord{}{}& \times \sum _{\rightCoord{}n=0}^{\leftCoord{}{}\leftCoord{}\infty}{\cal G}_0 e^{-
   \frac{\leftCoord{}B}2(n+\frac{\leftCoord{}1}{\rightCoord{}2}\int d^p\sigma s(\sigma))} \biggr)^{nN} \nonumber\rightCoord{}
   \rightCoord{}\\
&\leftCoord{}=&\lim_{n\to\infty}\left[ \frac{\leftCoord{}{\cal
   G}_0[s(\sigma)]^{-\frac{\leftCoord{}D-3}{\rightCoord{}2}}}{\sinh\left(\frac{\leftCoord{}B}{\rightCoord{}2}\int d^{p}\sigma s
   \leftCoord{}(\sigma)\right)}\right]^{nN}. \rightCoord{}
\rightCoord{}}{0mm}{14}{16}{
   Z[s]&=&\mbox{Tr}\left[G[x(\sigma),x'(\sigma);s(\sigma)] 
   \right]\\
&=&\biggl( [s(\sigma)]^{-\frac{D-3}{2}}e^{-\frac{m^2}{2}\int d^p\sigma
   s(\sigma) } \times \\
&{}& \times \sum _{n=0}^{{}\infty}{\cal G}_0 e^{-
   \frac{B}2(n+\frac{1}{2}\int d^p\sigma s(\sigma))} \biggr)^{nN} \\
&=&\lim_{n\to\infty}\left[ \frac{{\cal
   G}_0[s(\sigma)]^{-\frac{D-3}{2}}}{\sinh\left(\frac{B}{2}\int d^{p}\sigma s
   (\sigma)\right)}\right]^{nN}. 
}{1}\coordE{}\end{eqnarray}

   \section{Interacting membranes via noncommutativity}

   In the previous section we argued how to construct noncommutative  extended objects. In this section we would like to give an insight in a different physical context and to compare with our previous results.

   The main point of this section is to show how to introduce interactions via noncommutativity.

   In the previous examples noncommutativity was included in the space where an object is embedded. Here we will consider noncommutativity between different objects or,  in other words, noncommutativity also can considered as a nonlocal interaction.

   In order to explain the main idea, let us start  considering two particles in one dimension, labeled by coordinates \myHighlight{$x_1$}\coordHE{} and \myHighlight{$y_1$}\coordHE{} and canonical momenta \myHighlight{$p_1$}\coordHE{} and \myHighlight{$p_2$}\coordHE{} respectively.

   The Hamiltonian for this system is
   \begin{equation}\coord{}\boxEquation{
   H=\frac{1}{2}p_1^2 + \frac{1}{2}p_2^2.
   \label{38}
   }{
   H=\frac{1}{2}p_1^2 + \frac{1}{2}p_2^2.
   }{ecuacion}\coordE{}\end{equation}


   Although naively the particles in (\ref{38}) are free, it can interact if we posit the commutator
   \begin{equation}\coord{}\boxEquation{
   [p_1,p_2]=iB, 
   \label{39}
   }{
   [p_1,p_2]=iB, 
   }{ecuacion}\coordE{}\end{equation}
   where  \myHighlight{$B$}\coordHE{} measures the strength of this interaction and it is play a role of a magnetic field.

   Therefore, if (\ref{39}) is fulfilled, then the two-particle system (\ref{38}) is equivalent to a {\bf one particle} system in a two-dimensional noncommutative space \cite{gamboa1}.

   The above example can be generalized for more particles; for instance, let us considering two free particles moving in a commutative plane.

   The Hamiltonian  is
   \begin{equation}\coord{}\boxEquation{
   H=\frac{1}{2}(p_{1x}^2+p_{1y}^2)+\frac{1}{2}(p_{2x}^2+p_{2y}^2).
   }{
   H=\frac{1}{2}(p_{1x}^2+p_{1y}^2)+\frac{1}{2}(p_{2x}^2+p_{2y}^2).
   }{ecuacion}\coordE{}\end{equation}

   Then, let us assume that the interaction is given by
    \begin{equation}\coord{}\boxEquation{
    [p_{1x},p_{2x}]=iB,\,\,\,\,\,\,\,\,\,\,\,\,\,\,\,[p_{1y},p_{2y}]=iB,
   \label{40}
    }{
    [p_{1x},p_{2x}]=iB,\,\,\,\,\,\,\,\,\,\,\,\,\,\,\,[p_{1y},p_{2y}]=iB,
   }{ecuacion}\coordE{}\end{equation}
    then, such as in the previous case,  (\ref{40}) is
   \begin{equation}\coord{}\boxEquation{
   H=\frac{1}{2}(p_{1x}^2+ p_{2x}^2)+\frac{1}{2}(p_{1y}^2+ p_{2y}^2),
   \label{41}
   }{
   H=\frac{1}{2}(p_{1x}^2+ p_{2x}^2)+\frac{1}{2}(p_{1y}^2+ p_{2y}^2),
   }{ecuacion}\coordE{}\end{equation} 
   in other words, the previous system can be understood as a system of {\bf two particles} living on a noncommutative plane.

   For \myHighlight{$N$}\coordHE{} particles moving on a \myHighlight{$D$}\coordHE{} dimensional  commutative space, the generalization is straightforward. Indeed,  the Hamiltonian is 
   \begin{equation}\coord{}\boxEquation{
   H =  \frac{1}{2} (p^2_{1x} +p^2_{1y}+ ...) + \frac{1}{2} (p^2_{2x} +p^2_{2y}+ ...) + ... ,  \label{l6}
   }{
   H =  \frac{1}{2} (p^2_{1x} +p^2_{1y}+ ...) + \frac{1}{2} (p^2_{2x} +p^2_{2y}+ ...) + ... ,  }{ecuacion}\coordE{}\end{equation}
    then the interaction can be written as 
   \begin{equation}\coord{}\boxEquation{
   [ p_i^a, p_j^b] = i \delta_{ij} \epsilon^{ab} B, \label{l7}
   }{
   [ p_i^a, p_j^b] = i \delta_{ij} \epsilon^{ab} B, }{ecuacion}\coordE{}\end{equation}
   where \myHighlight{$a,b$}\coordHE{} run on  \myHighlight{$1, ...,D$}\coordHE{} labeling the different species of particles and the indices \myHighlight{$i,j, ...$}\coordHE{} select the vectorial component of 
   \myHighlight{${\bf x}$}\coordHE{}  \footnote{The component of the antisymmetric density tensor \myHighlight{$\epsilon^{ab}$}\coordHE{} are  defined as \myHighlight{$+1$}\coordHE{} if \myHighlight{$a>b$}\coordHE{}.}.

   If we rewrite the Hamiltonian as
   \begin{equation}\coord{}\boxEquation{
   H =  \frac{1}{2} (p^2_{1x} +p^2_{1x}+ ...) + \frac{1}{2} (p^2_{1y} +p^2_{1y}+ ...) + ... ,  \label{l8}
   }{
   H =  \frac{1}{2} (p^2_{1x} +p^2_{1x}+ ...) + \frac{1}{2} (p^2_{1y} +p^2_{1y}+ ...) + ... ,  }{ecuacion}\coordE{}\end{equation}
   then, (\ref{l8}) is equivalent to {\bf \myHighlight{${\bf D}$}\coordHE{} particles} moving on a \myHighlight{$N$}\coordHE{}-dimensional  noncommutative space.

   Thus in this context, we conclude that if two particles interact non-locally, then the space is necessarily noncommutative.

   \section{Cosmological implications}

   The goal of this section is to reinterpret the results discussed in the last section as a cosmological problem.

   Following the argument given in the introduction, the possibility of many universes keeping  relativistic invariance imply that the universes necessarily are causally disconnected.

   Therefore, the interaction among universes would be possible only if relativistic  invariance is explicitly violated. Taking this possibility, one conjecture can consider that each tensionless membrane is an universe and, therefore,  many universes could be considered as a gas of tensionless membranes in a reservoir.

   In order to produce interaction among different membranes, one break relativistic invariance assuming nontrivial commutators like 
   (\ref{l7})  for the infinite-dimensional case. 
   
   Thus, using the results obtained at the end of section IV, one find that the partition function for a gas of \myHighlight{$N$}\coordHE{} universes is
   \begin{equation}\coord{}\boxEquation{
   Z=  \lim_{n \to \infty}  \left[\frac{{\cal G}_0[s (\sigma)]^{-\frac{N }{2}}}{\sinh\left(\frac{B}{2}~\int
   d^p\,\sigma\, s (\sigma)\right)}\right]^{nD},  \label{311}
   }{
   Z=  \lim_{n \to \infty}  \left[\frac{{\cal G}_0[s (\sigma)]^{-\frac{N }{2}}}{\sinh\left(\frac{B}{2}~\int
   d^p\,\sigma\, s (\sigma)\right)}\right]^{nD},  }{ecuacion}\coordE{}\end{equation}
   and, therefore, each pair of membranes of the gas  interact harmonically.

   This last fact could have interesting observational consequences.  Indeed, the harmonic interaction among many universes involves infared-ultraviolet shifts. This interaction should  produce a periodic deformation of each universe with period -- according to (\ref{311}) --  proportional to \myHighlight{$B^{-1}$}\coordHE{}.  
   
   However, the question is, what is the physical meaning of the \myHighlight{$B$}\coordHE{} parameter?. According with the interaction procedure sketched above, \myHighlight{$B$}\coordHE{} correpond   to a tiny magnetic field and that, one might guess  that correspond to the primordial magnetic field. Phenomenologically, this primordial magnetic field is  \myHighlight{$\sim 10^{-16} \mbox{GeV}^2$}\coordHE{} and, therefore, the oscillation frequency of the universe is 
   \begin{equation}\coord{}\boxEquation{
   \omega \sim 10^{-93} \, \mbox{GeV}, \label{eff}
   }{
   \omega \sim 10^{-93} \, \mbox{GeV}, }{ecuacion}\coordE{}\end{equation}
   if the mass of the universe \myHighlight{$\sim 10^{77} \,\mbox{GeV}$}\coordHE{}. 

   Although this frequency  is extremely small --and possibly undetectable with presently available technology-- the measurement of (\ref{eff}) could reveal the interaction among universes. 

   This  harmonic pulsation effect among universes could be of a new source of gravitational radiation and the relativistic invariance violation could be an explanation for the seed field puzzle \cite{ru}. 

   \section{conclusions}

   In conclusion, we have constructed the statistical mechanics of generally covariant systems moving on a noncommutative space and, from these results, we have studied the quantum statistical mechanics of tensionless membranes gas.

   We have also shown that one can  introduce no-local interactions by means noncommutativity  implying measurable cosmological consequences if many universes are present.

   Our main results are summarized as follows:
   \begin{itemize}
   \item{}Each null membrane is considered as a universe that satisfy the cosmological principle during its evolution.  If the RHS of (\ref{5}) is zero, then the universes are causally disconnected.

   \item{}If (\ref{5}) is different from zero, then the universes interact harmonically implying infrared-ultraviolet shifts. The periodic fluctuations of the universe could be a source of anisotropy, maybe, it could explain the presently observed anisotropy.

   \item{}The periodic motion among universes is a source of gravitational waves with frequencies extremely tiny. 
   
    \item{} The noncommutativity is an effect that could be attributed to a primordial magnetic field. 
   \end{itemize}
   
   Although the effects discussed in this paper are very small, in our opinion are qualitatively interesting and it furnishes of a different point of view to the standard cosmology discussions.

Finally, the extended noncommutative objects discussed in this paper are a natural generalization of noncommutative relativistic particles. However, there is an important conceptual difference between the commutative and noncommutative particles, namely, in the noncommutative case there are no antiparticles solution ({\it c.f.} eq. (\ref{dis})). 

Therefore --in this context-- the annihilation universes processes would be forbidden in the strong coupling limit. 

\acknowledgments 
We would like to thank  J. L. Cort\'es and A. P. Polychronakos by discussions. This work has been partially supported by the grants 1010596 and 3000005 from Fondecyt-Chile.




 \begin{references}
 \bibitem{kos} The problem of Lorentz  invariance violation has been  extensively studied in the last years by Kosteleck\'y and collaborators. Some references are:   D. Colladay and V.A. Kosteleck\'y,  {\it Phys. Lett.} {\bf B511}, 209 (2001);  V.A. Kosteleck\'y, R. Lehnert, {\it Phys. Rev.}
 {\bf 63 D}, 065008 (2001);  R. Bluhm and  V.A. Kosteleck\'y, {\it Phys. Rev. Lett.} {\bf 84}, 1381 (2000); V.A. Kosteleck\'y and  Charles D. Lane, 
 {\it Phys. Rev.} {\bf 60D}, 116010 (1999); R. Jackiw and V.A. Kosteleck\'y, {\it Phys. Rev. Lett.} {\bf 82}, 3572 (1999);  D. Colladay, V.A.  Kosteleck\'y, {\it  Phys. Rev.}   {\bf 58D}, 116002 (1998).
\bibitem{ccgm} J. Carmona, J. L. Cort\'es, J. Gamboa and F. M\'endez, {\it Dispersion Relations, Noncommutativity and Lorentz Violation Invariance},  hep-th/0207158.
\bibitem{gzk}K. Greisen, {\it Phys. Rev. Lett.} {\bf 16}, 748 (1966); G.T. Zatsepin and V.A. Kuzmin, {\it  Zh. Eksp. Teor. Fiz. Pis'ma Red}. {\bf 4}, 414 (1966) [{\it JETP Lett.} {\bf 4}, 78 (1966)].
\bibitem{rubi} J. Carmona, J. L. Cort\'es, J. Gamboa and F. M\'endez, {Quantum Theory of Noncommutative Fields} hep-th/0301248 (to appear in JHEP).  
\bibitem{alot} G. Amellino-Camelia, J. Ellis, N. E. Mavromatos, D. V.  Nanopoulos and S. Sarkhar, {\it Nature} {\bf393}, 763 (1998); L. Smolin and J. Maguejo, {\it Phys. Rev. Lett.} {\bf 88}, 190403 (2002).
\bibitem{nambu} See {\it e.g.}, R. P. Feyman, {\it Phys. Rev.}  {\bf 80}, 440 (1950).
\bibitem{gamboa} See {\it e.g.} T. Eguchi, {\it Phys. Rev. Lett.}  {\bf 44}, 126 (1980); F. Lizzi, B. Rai, G. Sparano and A. Srivastava, 
{\it Phys. Lett. } {\bf 182}, 326 (1986);  J. Gamboa, C. Ram\'{\i}rez and M. Ruiz-Altaba, {\it Phys. Lett.} {\bf 225}, 335 (1989);  {\it Nucl. Phys.} 
{\bf 338}, 143 (1990);  {\it Phys. Lett.} {\bf 231}, 57 (1989).
\bibitem{string}F. Lizzi and G. Sparano, {\it Phys. Lett} {\bf B232}, 311 (1989).
\bibitem{atick} J.J. Atick and E. Witten, {\it Nucl. Phys.} {\bf B310}, 291 (1989).
\bibitem{gamboa1} For discussions on noncommutative quantum mechanics in the context of this paper see {\it e.g.} ;   V. P. Nair and A. P. Polychronakos, {\it Phys. Lett.} {\bf B505}, 267 (2001); J. Gamboa, M. Loewe and J. C. Rojas, {\it Phys. Rev.} {\bf D66}, 045018 (2002);  J. Gamboa, M. Loewe, F. M\'endez  and J. C. Rojas, {\it Int. J. Mod. Phys.} {\bf A17}, 
2555 (2002);  {\it Mod. Lett. Phys.} {\bf A16}, 2075 (2001); S. Bellucci, A.  Nersessian , C. Sochichiu, {\it Phys.Lett.}  {\bf B522}, 345 (2001).; C. Duval, P.A. Horvathy,  {\it J. Phys. } {\bf A34}, 10097 (2001);  C. Acatrinei, {\it JHEP} {\bf 0109}, 007~ (2001).
\bibitem{ru}  For a recent review see {e.g.}, D. Grasso and H. R. Rubisntein, {\it Phys. Rep.} {\bf 348}, 163 (2001).

   
   \end{references}
   \end{document}

































\bye
