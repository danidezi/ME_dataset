%% This document created by Scientific Word (R) Version 2.0
%% Starting shell: mathart1


\documentclass[a4paper,12pt,thmsa]{article}

%%%%%%


%TCIDATA{TCIstyle=Article/art1.lat,elba,article}


\begin{document}

\title{Fractional spin through quantum affine algebras }
\author{*M. Mansour and $\bullet $E.H. Zakkari \and Emails: *mansour70@mailcity.com,
$\bullet $hzakkari@hotmail.com \\
%EndAName
Laboratory of Theoretical Physics\\
University Mohamed V\\
PO BOX 1014\\
Rabat, Morocco.}
\date{}
\maketitle

\begin{abstract}
In this paper, we study the fractional decomposition of the quantum
enveloping affine algebras $U_Q(\hat A(n))$ and $U_Q(\widehat{C}(n))$ in the
$Q\rightarrow q=e\frac{2i\pi }k$ limit. This decomposition is based on the
bosonic representation and can be related to the fractional supersymmetry
and $k$-fermionic spin. The equivalence between the quantum affine algebras
and the classical ones in the fermionic realization is also established.
\end{abstract}

\newpage\

\section{Introduction}

The concept of quantum group and algebras $[1,2]$, have enriched the arena
of mathematics and theoretical physics. Quantum groups were appeared in
studying Yang-Baxter equations $[3]$ as well as scattering method $[4]$. In $%
[5,6]$ the quantum analogous of Lie superalgebras was constructed. The
quantized enveloping algebras associated to affine algebras and
superalgebras are given in $[1,7]$. It is well known that the boson
realization is a very powerful and elegant method for studying quantum
algebras representations. Based on this method, the representation theory of
quantum affine algebras has been an object of intensive studies, namely, the
results for the oscillator representations of affine algebras. There are
obtained $[8-10]$ through consistent realizations involving deformed Bose
and Fermi operators $[11,12]$.

To make a connection with the quantum group theory, a new geometric
interpretation of fractional supersymmetry has been introduced in $[13-17]$.
In these latters, the authors show that the one-dimensional superspace is
isomorphic to the braided line when the deformation parameter goes to a root
of unity. The similar technics are used, in $[18]$, to show how internal
spin arises naturally in certain limit of the $Q$-deformed momentum algebras
$U_Q(sl(2))$.

Indeed, using $Q$-Schwinger realization, it is proved that the decomposition
of the $U_Q(sl(2))$ into a direct product $U(sl(2))$ and the deformed $%
U_q(sl(2))$ $($ note that $U_Q(sl(2))=$ $U_q(sl(2)$ at $Q=q)$ . The property
of splitting quantum algebras $A_n,$ $B_n,$ $C_n$ and $D_n$ and quantum
superalgebras $C(n),$ $B(n,m),$ $C(n+1)$ and $D(n,m)$ in the limit $%
Q\rightarrow q$ limit is investigated in $[19]$.

We also notice that the case of deformed Virasoro algebras and some other
particular quantum (Super) algebras is given in $[20]$.

The aim of this paper is to investigate the decomposition property of the
quantum affine algebras $U_Q(\hat A(n))$ and $U_Q(\hat C(n))$ in the $%
Q\rightarrow q$ limit. We start in section $2$ by defining $k$-fermionic
algebra. In section $3$, we discuss the decomposition property of $Q$-boson
oscillator in the $Q\rightarrow q$ limit. We introduce the way in which one
obtains two independent objects, an ordinary boson and a $k$-fermion, from
one $Q-$deformed boson when $Q\rightarrow q$. We establish also the
equivalence between a $Q$-deformed fermion and conventional (ordinary) one.
Using these results, we analyze the $Q\rightarrow q$ limit of the quantum
affine algebra $U_Q(\hat A(n))$ (section 4) and the quantum affine algebra $%
U_Q(\widehat{C}(n))$ (section 5). Some concluding remarks are given in
section $6.$

\section{Preliminaries about k-fermionic algebra.}

The q-deformed bosonic algebra $\Sigma _q$ generated by $A^{+}$, $A^{-}$ and
number operator $N$ is given by:

\begin{equation}
A^{-}A^{+}-qA^{+}A^{-}=q^{-N}
\end{equation}

\begin{equation}
A^{-}A^{+}-q^{-1}A^{+}A^{-}=q^N
\end{equation}

\begin{equation}
q^NA^{\pm }q^{-N}=q^{\pm 1}A^{\pm }
\end{equation}

\begin{equation}
q^Nq^{-N}=q^{-N}q^N=1,
\end{equation}
where the deformation parameter:

\begin{equation}
q=e^{\frac{2i\pi }l},\; l\in N-\{0,1\},
\end{equation}

is a root of unity.

The annihilation operator $A^{-}$ is hermetic conjugated to creation
operator $A^{+}$ and $N$ is hermetic also. From $\left[ 1-4\right] $, it is
easy to have the following relations:

\begin{equation}
A^{-}(A^{+})^n=[[n]]q^{-N}(A^{+})^{n-1}+q^n(A^{+})^nA^{-}
\end{equation}

\begin{equation}
(A^{-})^nA^{+}=[[n]](A^{-})^{n-1}q^{-N}+q^nA^{+}(A^{-})^n,
\end{equation}
where the notation $[[$ $]]$ is defined by:

\begin{equation}
\lbrack [n]]=\frac{1-q^{2n}}{1-q^2}
\end{equation}

We introduce a new variable $k$ defined by:
\begin{equation}
k=l\;\; \hbox{for  odd  values  of}\;\; l,
\end{equation}

\begin{equation}
k\, =\, \frac l2\;\; \hbox{for even values of}\;\; l,
\end{equation}
such that for odd $l$ (resp. even $l$ ), we have $q^k=1$ $($resp. $q^k=-1).$
In the particular case $n=k$, $eqs(6,$ $7)$ permit us to have:

\begin{equation}
A^{-}(A^{+})^k=\pm (A^{+})^kA^{-}
\end{equation}

\begin{equation}
(A^{-})^kA^{+}=\pm A^{+}(A^{-})^k,
\end{equation}
and the eqs $(1-5)$ yield to:

\begin{equation}
q^N(A^{+})^k=(A^{+})^kq^N
\end{equation}

\begin{equation}
q^N(A^{-})^k=(A^{-})^kq^N
\end{equation}
One can show that the elements $(A^{-})^k$ and $(A^{+})^k$ are the elements
of the centre of $\sum_q$ algebra (odd values for $l$); and the irreducible
representations are $k$-dimensional. These two properties allows to:
\begin{equation}
(A^{+})^k=\alpha I
\end{equation}

\begin{equation}
(A^{-})^k=\beta I
\end{equation}

The extra possibilities parameterized by:

\[
(1) \;\; \alpha \, =\, 0, \;\; \beta \neq 0
\]

\[
(2) \;\; \alpha \neq 0, \;\; \beta =0
\]

\[
(3) \;\; \alpha \neq 0, \;\;\beta \neq 0,
\]
are not relevant for the considerations of this paper. In the two cases $(1)$
and $(2)$ we have the so-called semi-periodic (semi-cyclic) representation
and the case $(3)$ correspond to the periodic one. In what follows, we are
interested to a representation of the algebra $\sum_q$ such that the
following:

\[
(A^{\mp })^k=0,
\]
is satisfied. We note that the algebra $\sum_{-1}$ obtained for $k=2$,
correspond to ordinary fermion operators with $(A^{+})^2=0$ and $(A^{-})^2=0$
which reflects the exclusion's Pauli principle. In the limit case where $%
k\rightarrow \infty $, the algebra $\sum_1$ correspond to the ordinary
bosons. For other values of $k$, the $k$-fermions operators interpolate
between fermions and bosons, these are also called anyons with fractional
spin in the sense of Majid $[21,22]$.

\section{Fractional spin through Q-boson.}

In the previous section, we have worked with $q$ at root of unity. In this
case, quantum oscillator $(k$-fermionic$)$ algebra exhibit a rich
representation with very special properties different from the case where $q$
is generic. So, in the first case the Hilbert space is finite dimensional.
In contrast, where $q$ is generic, the Fock space is infinite dimensional.
In order to investigate the decomposition of $Q$-deformed boson in the limit
$Q\rightarrow e\frac{2i\pi }k$ we start by recalling the $Q$-deformed
algebra $\Delta _Q$.

The algebra $\Delta _Q$ generated by an annihilation operator $B^{-}$, a
creation operator $B^{+}$ and a number operator $N_B$:

\begin{equation}
B^{-}B^{+}-QB^{+}B^{-}=Q^{-N_B}
\end{equation}

\begin{equation}
B^{-}B^{+}-Q^{-1}B^{+}B^{-}=Q^{N_B}
\end{equation}

\begin{equation}
Q^{N_B}B^{+}Q^{-N_B}=QB^{+}
\end{equation}

\begin{equation}
Q^{N_B}B^{-}Q^{-N_B}=Q^{-1}B^{-}
\end{equation}

\begin{equation}
Q^{N_B}Q^{-N_B}=Q^{-N_B}Q^{^{+}N_B}=1.
\end{equation}

From the above equations, we obtain:
\begin{equation}
\lbrack
Q^{-N_B}B^{-},[Q^{-N_B}B^{-},[....[Q^{-N_B}B^{-},(B^{+})^k]_{Q^{2k}}...]_{Q^4}]_{Q^2}]=Q^{%
\frac{k(k-1)}2}[k]!
\end{equation}
where the $Q$-deformed factorial is given by:

\begin{equation}
\lbrack k]!=[k][k-1][k-2]...............[1],
\end{equation}
and:

\[
\lbrack 0]!=1
\]

\[
\lbrack k]=\frac{Q^k-Q^{-k}}{Q-Q^{-1}}
\]

The $Q$-commutator, in $eq(22)$, of two operators $A$ and $B$ is defined by:

\[
\lbrack A,B]_Q=AB-QBA
\]

The aim of this section is to determine the limit of $\Delta _Q$ algebra
when $Q$ goes to the root of unity $q$. The starting point is the limit $%
Q\rightarrow q$ of the $eq(22),$

\[
lim\, _{Q\rightarrow q}\frac
1kQ^{-N_B}[Q^{-N_B}B^{-},[Q^{-N_B}B^{-},[....[Q^{-N_B}B^{-},(B^{+})^k]_{Q^{2k}}...]_{Q^4}]_{Q^2}]
\]

\begin{equation}
=lim\, _{Q\rightarrow q}\frac{Q^{\frac{k(k-1)}2}}{[k]!}[%
Q^{-N_B}(B^{-})^k,(B^{+})^k]=q^{\frac{k(k-1)}2}
\end{equation}

This equation can be reduced to:

\begin{equation}
lim\, _{Q\rightarrow
q}[\frac{Q^{\frac{kN_B}2}(B^{-})^k}{([k]!)^{\frac
12}},\frac{(B^{+})^kQ^{\frac{kN_B}2}}{([k]!)^{\frac 12}}]=1.
\end{equation}

Since $q$ is a root of unity, it is possible to change the sign on the
exponent of $q^{\frac{kN_B}2}$ terms in the above equation.

We define the operators as in $[18]$:

\begin{equation}
b^{-}=lim\, _{Q\rightarrow q}\frac{Q^{\pm
\frac{kN_B}2}}{([k]!)^{\frac
12}}(B^{-})^k,\, b^{+}=lim_{Q\rightarrow q}\frac{(B^{+})^kQ^{^{\pm }%
\frac{kN_B}2}}{([k]!)^{\frac 12}},
\end{equation}
which lead to an ordinary boson algebra noted $\Delta _0$, generated by:

\begin{equation}
\lbrack b^{-},b^{+}]=1.
\end{equation}

The number operator of this new bosonic algebra defined as the usual case, $%
N_b=b^{+}b^{-}$. At this stage we are in position to discuss the splitting
of $Q$-deformed boson in the $Q\rightarrow q$ limit. Let us introduce the
new set of generators given by:

\begin{equation}
A^{-}=B^{-}q^{-\frac{kN_b}2}
\end{equation}

\begin{equation}
A^{+}=B^{+}q^{-\frac{kN_b}2}
\end{equation}

\begin{equation}
N_A=N_B-kN_b,
\end{equation}
which define a $k$-fermionic algebra:

\begin{equation}
\lbrack A^{+},A^{-}]_{q^{-1}}=q^{N_A}
\end{equation}

\begin{equation}
\lbrack A^{-},A^{+}]_q=q^{-N_A}
\end{equation}

\begin{equation}
\lbrack N_A,A^{\pm }]=\pm A^{\pm }.
\end{equation}
It is easy to verify that the two algebras generated by the set of operators
$\{b^{+},b^{-},N_b\}$ and $\{A^{+},A^{-},N_A\}$ are mutually commutative. We
conclude that in the $Q\rightarrow q$ limit, the $Q$-deformed bosonic
algebra oscillator decomposes into two independent oscillators, an ordinary
boson and $k$-fermion; formally one can write:

\[
\lim_{Q\rightarrow q}\Delta _Q\equiv \Delta _0\otimes \Sigma _q,
\]
where $\Delta _0$ is the classical bosonic algebra generated by the
operators $\{b^{+},b^{-},$ $N_b\}.$

Similarly, we want to study the $Q$-fermion algebra at root of unity. To do
this, we start by considering the $Q$- deformed fermionic algebra, noted $%
\Xi _Q$:

\begin{equation}
F^{-}F^{+}+QF^{+}F^{-}=Q^{N_{F}}
\end{equation}

\begin{equation}
F^{-}F^{+}+Q^{-1}F^{+}F^{-}=Q^{-N_F}
\end{equation}

\begin{equation}
Q^{N_F}F^{+}Q^{-N_F}=QF^{+}
\end{equation}

\begin{equation}
Q^{N_F}F^{-}Q^{-N_F}=Q^{-1}F^{-}
\end{equation}

\begin{equation}
Q^{N_F}Q^{-N_F}=Q^{-N_F}Q^{N_F}=1
\end{equation}

\begin{equation}
(F^{+})^2=0, \; \; (F^{-})^2=0
\end{equation}

In the case $n=2$ $(q=-1),$ we define the new fermionic operators as follow:

\begin{equation}
f^{+}=\lim \, _{Q\rightarrow 1}F^{+}Q^{^{-}\frac{N_F}2}
\end{equation}

\begin{equation}
f^{-}=\lim \, _{Q\rightarrow 1}Q^{^{-}\frac{N_F}2}F^{-}.
\end{equation}

By a direct calculus, we obtain the following anti-commutation relation:

\begin{equation}
\{f^{-},\, f^{+}\}=1.
\end{equation}

Moreover, we have the nilpotency condition:

\begin{equation}
(f^{-})^2=0, \;\;(f^{+})^2=0.
\end{equation}

Thus, we see that the $Q$-deformed fermion reproduce the conventional
(ordinary) fermion. The same convention notation permits us to write:

\[
lim_{_{Q\rightarrow q}}\Xi _Q\equiv \Sigma _{-1}
\]

\section{Quantum affine algebra $U_Q(\hat A(n))$ at $Q$ a root of unity}

We apply the above results to derive the property of decomposition of
quantum affine algebra $U_Q(\hat A(n))$ in the $Q\rightarrow q$ limit.
Recalling that the $U_Q(\hat A(n))$ algebra is generated by the set of
generators $\{e_i,$ $f_i,$ $k_i^{\pm }=Q^{\pm d_ih_i},$ $0\leq i$ $\langle $
$n\}$ satisfying the following relations:

\begin{equation}
\lbrack e_i,f_j]=\delta _{ij}\frac{k_i-k_i^{-1}}{Q_i-Q_i^{-1}}
\end{equation}

\begin{equation}
k_{i}e_{j}k_{i}^{-1}=Q_{i}^{a_{ij}}e_{j},\, %
k_{i}f_{j}k_{i}^{-1}=Q_{i}^{a_{ij}}f_{j}
\end{equation}

\begin{equation}
k_ik_i^{-1}=k_i^{-1}k_i=1,\, k_ik_j=k_jk_i.
\end{equation}

The quantum affine algebra $U_Q(\hat A_n)$ admits two $Q$-oscillators
representations: bosonic and fermionic ones; in the bosonic realization, the
generators of $U_Q(\hat A_n)$ can be constructed by introducing $(n+1)$ $Q$%
-deformed bosons as follows:

\[
e_{i}=B_{i}^{-}B_{i+1}^{+},1\leq i\leq n
\]

\[
f_{i}=B_{i}^{+}B_{i+1}^{-},1\leq i\leq n
\]

\[
k_{i}=Q^{-N_{i}+N_{i+1}},1\leq i\leq n
\]

\[
e_{0}=B_{n+1}^{-}B_{1}^{+}
\]

\[
f_{0}=B_{1}^{-}B_{n+1}^{+}
\]

\[
k_0=Q^{N_1-N_{n+1}}.
\]

The fermionic realization of $U_Q(\hat A(n))$ is given by:

\[
e_i=F_i^{+}F_{i+1}^{-},1\leq i\leq n
\]

\[
f_i=F_i^{-}F_{i+1}^{+},1\leq i\leq n
\]

\[
k_i=Q^{N_i-N_{i+1}},1\leq i\leq n
\]

\[
e_0=F_{n+1}^{+}F_1^{-}
\]

\[
f_0=F_1^{+}F_{n+1}^{-}
\]

\[
k_0=Q^{-N_1+N_{n+1}}.
\]

At this stage, our aim is to investigate the limit $Q\rightarrow q$ of the
affine algebra $U_Q(\hat A_n)$. As it is already mentioned in the
introduction, our analysis is based on the $Q$-oscillator representation
based on $Q$-Schwinger realization. In the $Q\rightarrow q$, the splitting
of $Q$-deformed bosons leads to classical bosons $\{b_i^{+},b_i^{-},N_{b_i},$
$1\leq i\leq n\}$ given by the $eqs(26,$ $27)$ and $k$-fermionic algebra $%
\{A_i^{+},A_i^{-},N_{A_i},$ $1\leq i\leq n\}$given by $eqs(31-33)$. From the
classical bosons, we define for $i=1,...,n$ the operators:

\begin{equation}
e_i=b_i^{-}b_{i+1}^{+}
\end{equation}

\begin{equation}
f_i=b_i^{+}b_{i+1}^{-}
\end{equation}

\begin{equation}
h_i=-N_{b_i}+N_{b_{i+1}}
\end{equation}

\begin{equation}
e_0=b_1^{-}b_{n+1}^{+}
\end{equation}

\begin{equation}
f_0=b_1^{+}b_{n+1}^{-}
\end{equation}

\begin{equation}
h_0=-N_{b_1}+N_{b_{n+1}},
\end{equation}
the set $\{e_i,f_i,h_i,0\leq i\leq n\}$ generate the classical algebra $%
U(\hat A(n)).$ From the remaining generators $\{A_i^{+},A_i^{-},N_{A_i},$ $%
1\leq i\leq n+1\}$, we can realize $U_q(\hat A(n))$, generated by $E_i,$ $%
F_i,$ $K_i,E_0,F_0$ and $K_0$ where:

\begin{equation}
E_i=A_i^{-}A_{i+1}^{+},1\leq i\leq n
\end{equation}

\begin{equation}
F_i=A_i^{+}A_{i+1}^{-},1\leq i\leq n
\end{equation}

\begin{equation}
K_i=q^{-N_{A_i}+N_{A_{i+1}}},1\leq i\leq n
\end{equation}

\begin{equation}
E_{0}=A_{1}^{+}A_{n+1}^{-}
\end{equation}

\begin{equation}
F_{0}=A_{1}^{-}A_{n+1}^{+}
\end{equation}

\begin{equation}
K_0=q^{N_{A_1}-N_{A_{n+1}}}.
\end{equation}
The algebra $U_q(\hat A(n))$ is the same version of $U_Q(\hat
A_n)$ obtained by simply taking $Q=q$ and $B_i\sim A_i.$ Due to
the commutativity of elements of $U_q(\hat A(n))$ and $U(\hat
A_n),$ we obtain the following decomposition of the quantum affine
algebra $U_Q(\hat A_n)$ in the bosonic realization

\[
lim\, _{Q\rightarrow q}U_Q(\hat A_n)\equiv U_q(\hat A(n))\otimes
U(\hat A(n)).
\]
We discuss now the equivalence between $U_Q(\hat A_n)$ and $U(\hat A(n))$
algebras in the fermionic realization. Indeed, we have discussed in section $%
2$, how one can identify the conventional fermions with $Q$-deformed
fermions. Consequently, due to this equivalence, it is possible to construct
$Q$-deformed affine algebras $U_Q(\hat A_n)$ using ordinary fermions. It is
also possible to construct the affine algebra $U(\hat A_n)$ by considering $Q
$-deformed fermions. So, in the fermionic realization we have equivalence
between $U(\hat A_n)$ and $U_Q(\hat A_n).$ To be more clear, we consider the
$U_Q(\hat A_n)$ in the $Q$-fermionic representation. Where the generators
are given by:

\begin{equation}
e_i=F_i^{-}F_{i+1}^{+},1\leq i\leq n
\end{equation}

\begin{equation}
f_i=F_i^{+}F_{i+1}^{-},1\leq i\leq n
\end{equation}

\begin{equation}
k_i=Q^{N_{F_i}-N_{F_{i+1}}},1\leq i\leq n
\end{equation}

\begin{equation}
e_0=F_{n+1}^{+}F_1^{-}
\end{equation}

\begin{equation}
f_{0}=F_{1}^{+}F_{n+1}^{-}
\end{equation}

\begin{equation}
k_0=Q^{^{-}N_{F_1}+N_{F_{n+1}}}.
\end{equation}

Due to the equivalence fermion $Q$-fermion, the operators $f_i^{-}$, $f_i^{+}
$ are defined as a constant multiple of conventional fermion operators:

\begin{equation}
f_i^{+}=F_i^{+}Q^{\frac{-N_{Fi}}2}
\end{equation}

\begin{equation}
f_i^{-}=Q^{\frac{-N_{F_i}}2}F_i^{-},
\end{equation}
from which we can realize the generators:

\begin{equation}
E_i=f_i^{-}f_{i+1}^{+},\, 1\leq i\leq n
\end{equation}

\begin{equation}
F_i=f_i^{+}f_{i+1}^{-},\, 1\leq i\leq n
\end{equation}

\begin{equation}
H_i=N_{f_i}-N_{f_{i+1}},\, 1\leq i\leq n
\end{equation}

\begin{equation}
E_0=f_{n+1}^{+}f_1^{-}
\end{equation}

\begin{equation}
F_{0}=f_{1}^{+}f_{n+1}^{-}
\end{equation}

\begin{equation}
H_0=-N_{f_1}+N_{f_{n+1}}.
\end{equation}

The set $\{E_i,F_i,H_i$ $;$ $0\leq i\leq n\}$ generate the classical affine
algebra $U(\hat A_n)$ in the fermionic representation and we have

\[
U_q(\hat A(n))\equiv U(\hat A(n)).
\]

\section{Quantum affine algebra $U_Q(\widehat{C}(n))$ at a root of unity.}

Let $Q\in C-\{0\}$ be the deformation parameter. We shall use also $%
Q_i=Q^{d_i}$ with $d_i$ are numbers that symmetries the Cartan matrix $%
(a_{ij})$. The quantum affine algebra $U_Q(\widehat{C}(n))$ is described in
the Serre-Chevalley basis in terms of the simple root $e_i$, $f_i$ and
Cartan generators $h_i$, where $i=0,...n$, satisfy the following commutation
relations:

\begin{equation}
\lbrack e_i,f_j]=\delta _{ij}\frac{Q^{d_ih_i}-Q^{-d_ih_i}}{Q_i-Q_i^{-1}}
\end{equation}

\begin{equation}
\lbrack h_i,h_j]=0
\end{equation}
\begin{equation}
\lbrack h_i,e_j]=a_{ij}e_j,\, [h_i,f_j]=-a_{ij}f_j.
\end{equation}

Introducing the quantities $k_i=Q^{d_ih_i}$ which permit to rewrite the $%
eqs(73-75)$ as follows:

\begin{equation}
\lbrack e_{i},f_{j}]=\delta _{ij}\frac{k_{i}-k_{i}^{-1}}{Q_{i}-Q_{i}^{-1}}
\end{equation}

\begin{eqnarray}
k_ie_jk_i^{-1} &=&Q_i^{a_{ij}}e_j \\ \, k_if_jk_i^{-1}
&=&Q_i^{a_{ij}}f_j \\ k_ik_i^{-1} &=&k_i^{-1}k_i=1 \\
k_ik_j&=&k_jk_i.
\end{eqnarray}

Explicitly the generators of the quantum algebra $U_Q(\widehat{C}_{\, %
}(n))$ are given in the bosonic case by:

\begin{equation}
e_{i}=B_{i}^{+}B_{i+1}^{-}+B_{2n-i}^{+}B_{2n-i+1}^{-},1\leq i\leq n-1
\end{equation}

\begin{equation}
f_{i}=B_{i}^{-}B_{i+1}^{+}+B_{2n-i}^{-}B_{2n-i+1}^{+},1\leq i\leq n-1
\end{equation}

\begin{equation}
h_{i}=N_{B_{i}}-N_{B_{i+1}}+N_{B_{2n-i}}-N_{B_{2n-i+1}},1\leq i\leq n-1
\end{equation}

\begin{equation}
e_{n}=B_{n+1}^{-}B_{n}^{+}
\end{equation}

\begin{equation}
f_{n}=B_{n+1}^{+}B_{n}^{-}
\end{equation}

\begin{equation}
h_{n}=N_{B_{n}}-N_{B_{n+1}}
\end{equation}

\begin{equation}
e_{0}=B_{2n}^{+}B_{1}^{-}
\end{equation}

\begin{equation}
f_{0}=B_{1}^{+}B_{2n}^{-}
\end{equation}

\begin{equation}
h_0=N_{B_{2n}}-N_{B_1}.
\end{equation}

Due to the property of $Q$-boson decomposition in the $Q\rightarrow q$
limit, each $Q$-boson $\{B_i^{-},B_i^{+},N_{B_i}\}$ reproduce an ordinary
bosonic algebra $\{b_i^{-},b_i^{+},N_{b_i}\}$ and $k$-fermion operators $%
\left\{ A_i^{-},A_i^{+},N_{A_i}\right\} .$

From the set $\left\{ b_i^{+},\, b_i^{-},\, N_{b_i},\, %
i=0....n\right\} $ we can construct the classical affine algebra $U(\hat
C(n))$ as follow:

\begin{equation}
E_i=b_i^{+}b_{i+1}^{-}+b_{2n-i}^{+}b_{2n-i+1}^{-},\, 1\leq i\leq
n-1
\end{equation}

\begin{equation}
F_i=b_i^{-}b_{i+1}^{+}+b_{2n-i}^{-}b_{2n-i+1}^{+},\, 1\leq i\leq
n-1
\end{equation}

\begin{equation}
H_i=N_{b_i}-N_{b_{i+1}}+N_{b_{2n-i}}-N_{b_{2n-i+1}},\, 1\leq i\leq
n-1
\end{equation}

\begin{equation}
E_{n}=b_{n+1}^{-}b_{n}^{+}
\end{equation}

\begin{equation}
F_{n}=b_{n+1}^{+}b_{n}^{-}
\end{equation}

\begin{equation}
H_{n}=N_{b_{n}}-N_{b_{n+1}}
\end{equation}

\begin{equation}
E_{0}=b_{2n}^{+}b_{1}^{-}
\end{equation}

\begin{equation}
F_{0}=b_{1}^{+}b_{2n}^{-}
\end{equation}

\begin{equation}
H_0=N_{b_{2n}}-N_{b_1}.
\end{equation}

From the $k$-fermionic operators $\left\{ A_i^{-},\, A_i^{+},\, %
N_{A_i},1\leq i\leq n+1\right\} $, one can construct as in $eqs(81-89)$ the $%
q$-deformed affine algebra $U_q(\hat C(n))$. It is easy to verify that $U_q(%
\widehat{C}(n))$ and $U(\widehat{C}(n))$ are mutually commutative. As a
result, we have the following decomposition of quantum algebra $U_Q(\widehat{%
C}(n))$ in the $Q\rightarrow q$ limit:

\[
lim\, _{Q\rightarrow q}U_Q(\widehat{C}(n))\equiv U(\widehat{C}%
(n))\otimes U_q(\widehat{C}(n)).
\]
The equivalence between $U_Q(\widehat{C}(n))$ and $U(\widehat{C}(n))$
algebras in the fermionic representation can be easily deduced; in fact we
can construct the affine deformed algebra $U_Q(\widehat{C}(n))$ using the
ordinary fermions and conversely, the classical affine algebra $U(\widehat{C}%
(n))$ can be realized in terms of deformed fermions. Indeed, we consider the
$U_Q(\widehat{C}(n))$ in the $Q$-fermionic representation, where the
generators are given by:

\begin{equation}
E_i=F_i^{+}F_{i+1}^{-}+F_{2n-i}^{+}F_{2n-i+1}^{-},\, 1\leq i\leq
n-1
\end{equation}

\begin{equation}
F_i=F_i^{-}F_{i+1}^{+}+F_{2n-i}^{-}F_{2n-i+1}^{+},\, 1\leq i\leq
n-1
\end{equation}

\begin{equation}
K_i=Q_i^{d_i(N_{B_{i+1}}-N_{B_i}+N_{B_{_{2n-i+1}}}-N_{B_{2n-i}})},\, %
1\leq i\leq n-1
\end{equation}

\begin{equation}
E_{n}=F_{n+1}^{-}F_{n}^{+}
\end{equation}

\begin{equation}
F_{n}=F_{n+1}^{+}F_{n}^{-}
\end{equation}

\begin{equation}
K_{n}=Q_{n}^{d_{n}(N_{B_{n+1}}-N_{B_{n}})}
\end{equation}

\begin{equation}
E_{0}=F_{2n}^{+}F_{1}^{-}
\end{equation}

\begin{equation}
F_{0}=F_{1}^{+}F_{2n}^{-}
\end{equation}

\begin{equation}
K_0=Q^{N_1-N_{B_{2n}}}.
\end{equation}

The elements $d_{i}$ are the non zero integers such that $%
d_{i}a_{ij}=a_{ij}d_{i}$ and $a_{ij}$ is the $ij$-elements of $n\times n$ \
generalized Cartan matrix.

As in the case of $U_Q(\hat A_n)$, the $Q-$deformed fermions \ can be
identified to classical ones.

So, we can deduced that in the fermionic representation the $Q-$deformed
algebra $U_Q(\widehat{C}(n))$ is equivalent to the classical affine algebra $%
U(\widehat{C}(n))$, one can write:

\[
lim_{Q\rightarrow q}U_Q(\widehat{C}(n))\equiv U(\widehat{C}(n)).
\]

\section{Conclusion}

We have presented the general method leading to the investigation of the $%
Q\rightarrow q=e^{\frac{2i\pi }k}$limit of the quantum affine algebras $%
U_Q(\hat A_n)$ and $U_Q(\widehat{C}(n)).$ We note that the $Q$-oscillator
representation is crucial in this manner of splitting in this paper. The
technics and formulae used in this paper, will be useful to extend this
study to the infinite deformed algebras $[26]$, and quantum affine
superalgebras $[27].$

\newpage\

\section*{References}

$[1]$ V.G Drinfeld, \textit{Quantum groups}, Proc. Int. Cong. Math.
(Berkley,1986), Vol 1. p. 798.

$[2]$ M. Jimbo, Lett. Math. Phys. 11 (1986) 247.

$[3]$ P. Kulish and E. Sklyanin, Lecture Notes in Physics, VoL 151
(Springer,1981), p. 61.

$[4]$ L. D. Fadeev, \textit{Integrable Models i (1+1) dimensional Quantum
field theory}, les Houches Lectures, 1982(Elsivier, 1982), p. 563.

$[5]$ R.Floreanini, P. Spridinov and L. Vinet, Phys. Lett. B 242 (1990).

$[6]$ R. Floreanini, P. Spiridinov and L.Vinet, Commun. Math. 137 (1991) 149.

$[7]$ H. Yamane,\textit{\ Quantized enveloping algebras associated to simple
Lie} \textit{superalgebras}. Proceedings of the XX th IGGTMP, Toyonaka
(1994) ; World Scientific Singapore (1995).

$[8]$ L. Frappat, A. Sciarrino, S. Sciuto and Sorba, Phys. Lett. B 369
(1996).

$[9]$ L. Frappat, A. Sciarrino, S. Sciuto and Sorba$,$ q\_algeb/9609033.

$[10]$ A. J. Feingold, I. B. frenkel. Ad. Math. 56 (1985) 117.

$[11]$ L. C. Biedenharn, J. Phys. A 22 (1998) L 873.

$[12]$ A. J. Macfarlane, J. Phys. A 22 (1988) 4581.

$[13]$ R. S. Dunne, A. J. Macfarlane, J. A. de Azcarraga, and J.C. Perez
Bueno, Phys. Lett B 387 (1996) 294.

$[14$] R. S. Dunne, A. J. Macfarlane, J. A. de Azcarraga, and J.C. Perez
Bueno, hep-th/960087.

$[15]$ R. S. Dunne, A. J. Macfarlane, J. A. de Azcarraga, and J.C. Perez
Bueno, Czech. J. P. Phys. 46, (1996) 1145.

$[16]$ J. A. Azcarraga R. S. Dunne, A. J. Macfarlane and J.C. Perez Bueno,
Czech. J. P. Phys. 46, (1996) 1235.

$[17]$ R. S. Dunne, hep-th/9703111.

$[18]$ R. S. Dunne, hep-th/9703137.

$[19]$ M. Mansour, M. Daoud and Y. Hassouni, Phys. Lett. B 454 (1999).

$[20]$ M. Mansour, M. Daoud and Y. Hassouni, Rep. Math. Phys. Vol. 44
(1999), 435.

$[21]$ S. Majid, \textit{Anyonic Quantum groups in spinors}, \textit{%
Twistors, Clifford algebras and} \textit{quantum deformations} (proc. Of 2nd
Max Born Symposium, Wroclam, Poland, 1992), z.Oziewicz et al. Khluwwer.

$[22]$ S. Majid, hep-th/9410241.

$[23]$ M. Daoud and Y. Hassouni M.Kibler, \textit{The k-fermions as objects
interpolating} \textit{between fermions and bosons}, Symmetries in Science
X, eds B Gruber and M. Ramek (1998, New York: plenum press )

$[24]$ V. G. Kac, Commun. Phys 53 (1977).

$[25]$ M. Rosso, Commun. Math. Phys 124 (1989) 307.

$[26]$ M. Mansour, E. H. Zakkari, \textit{Q-Fractional spin through some
infinite deformed} \textit{algebra}, to be submitted.

$[27]$ M. Mansour, work in progress.

\end{document}
