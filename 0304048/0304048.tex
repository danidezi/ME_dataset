\documentclass[a4paper,12pt]{article}
\usepackage{epsfig}
\setlength{\oddsidemargin}{-0.15cm}
\setlength{\textwidth}{16.3cm}
\setlength{\topmargin}{-1.25 cm}
\setlength{\textheight}{22cm}
\parskip=4pt



\newcommand{\sect}[1]{\section{#1}\setcounter{equation}{0}}
\renewcommand{\baselinestretch}{1.3}
\newcommand{\vs}[1]{\vspace{#1 mm}}
\newcommand{\hs}[1]{\hspace{#1 mm}}
\renewcommand{\thefootnote}{\fnsymbol{footnote}}

\def\thesection{\arabic{section}}

\begin{document}
\baselineskip=16pt
\begin{titlepage}
\setcounter{page}{0}
\begin{center}

\vspace{0.5cm} {\Large \bf Cosmological Scaling Solutions and
Multiple Exponential Potentials }\\
\vspace{10mm}
Zong-Kuan Guo\footnote{e-mail address: guozk@itp.ac.cn}$^{a}$,
Yun-Song Piao\footnote{e-mail address: yspiao@itp.ac.cn}$^{a}$
and
Yuan-Zhong Zhang$^{b,a}$ \\
\vspace{6mm} {\footnotesize{\it
  $^a$Institute of Theoretical Physics, Chinese Academy of Sciences,
      P.O. Box 2735, Beijing 100080, China\\
  $^b$CCAST (World Lab.), P.O. Box 8730, Beijing 100080\\}}

\vspace*{5mm} \normalsize
\smallskip
\medskip
\smallskip
\end{center}
\vskip0.6in \centerline{\large\bf Abstract} {We present a
phase-space analysis of cosmology containing multiple scalar
fields with positive and negative exponential potentials.
In addition to the well-known assisted inflationary solutions,
there exits power-law multi-kinetic-potential scaling solutions
for sufficiently flat positive potentials or steep negative
potentials, which are the unique late-time attractor whenever they
exist. We briefly discuss the physical consequences of these
results.}

\vspace*{2mm}
\end{titlepage}



Scalar field cosmological models are of great importance in modern
cosmology. The dark energy is attributed to the dynamics of a
scalar field, which convincingly realizes the goal of explaining
present-day cosmic acceleration generically using only attractor
solutions~\cite{CD}. A scalar field can drive an accelerated
expansion and thus provides possible models for cosmological
inflation in the early universe~\cite{GU}. In particular, there
have been a number of studies of spatially homogeneous scalar
field cosmological models with an exponential potential. There are
already known to have interesting properties; for example, if one
has a universe containing a perfect fluid and such a scalar field,
then for a wide range of parameters the scalar field mimics the
perfect fluid, adopting its equation of state~\cite{CL}. These
scaling solutions are attractors at late times~\cite{ST}. The
inflation~\cite{LM, PCZ} and other cosmological effect~\cite{HL}
of multiple scalar fields have also been considered.


The scale-invariant form makes the exponential potential
particularly simple to study analytically. There are well-known
exact solutions corresponding to power-law solutions for the
cosmological scale factor $a\propto t^p$ in a spatially flat
Friedmann-Robertson-Walker (FRW) model~\cite{FLM},
but more generally the coupled
Einstein-Klein-Gordon equations for a single field can be reduced
to a one-dimensional system which makes it particularly suited to
a qualitative analysis~\cite{JJH,CLW}. Recently, adopting a system
of dimensionless dynamical variables~\cite{EW},
%Heard and Wands have investigated
the cosmological scaling solutions with positive and negative
exponentials has been studied~\cite{HW}. Usually there are many
scalar fields with exponential potentials in supergravity,
superstring and the generalized Einstein theories, thus multi
potentials may be more important.
%are too steep to yield successful inflation.
In this paper, we will consider multiple scalar fields with
positive and negative exponential potentials. We have assumed that
there is no direct coupling between the exponential potentials.
The only interaction is gravitational. A phase-space analysis of
the spatially flat FRW models shows that there exist cosmological
scaling solutions which are the unique late-time attractors, and
successful inflationary solutions which are driven by multiple
scalar fields with a wide range of each potential slope parameter
$\lambda$.




We start with more general model with $m$ scalar
fields $\phi _i$, in which each has an identical-slope potential
\begin{equation}
V_i(\phi _i)=V_{0i}\exp{(-\lambda \kappa \phi _i)}
\end{equation}
where $\kappa ^2\equiv 8 \pi G_N$ is the gravitational coupling
and $\lambda$ is a dimensionless constant characterising the slope
of the potential.
Note that there is no direct coupling of the fields, which
influence each other only via their effect on the expansion.
The evolution equation of each scalar field for a
spatially flat FRW model with Hubble parameter $H$ is
\begin{equation}
\label{2E}
\ddot{\phi _i}+3H\dot{\phi _i^2}+\frac{dV_i(\phi _i)}{d\phi _i}=0
\end{equation}
subject to the Friedmann constraint
\begin{equation}
\label{3E}
H^2=\frac{\kappa ^2}{3}\sum_{i=1}^{m}[V_i(\phi
_i)+\frac{1}{2}\dot{\phi}_i^2]
\end{equation}
Defining $2m$ dimensionless variables
\begin{equation}
x_i=\frac{\kappa \dot{\phi _i}}{\sqrt{6}H}, \qquad
y_i=\frac{\kappa \sqrt{|V_i|}}{\sqrt{3}H}
\end{equation}
the evolution equations (\ref{2E}) can be written
as an autonomous system:
\begin{eqnarray}
\label{5E}
x'_i&=&-3x_i(1-\sum_{i=1}^{m}x_i^2)\pm \lambda \sqrt{\frac{3}{2}}y_i^2 \\
y'_i&=&y_i(3\sum_{i=1}^{m}x_i^2-\lambda \sqrt{\frac{3}{2}}x_i)
\label{6E}
\end{eqnarray}
where a prime denotes a deriative with respect to the logarithm of
the scalar factor, $N\equiv \ln a$, and the constraint equation
(\ref{3E}) becomes
\begin{equation}
\sum_{i=1}^{m}(x_i^2\pm y_i^2)=1
\end{equation}
where upper/lower signs denote the two distinct cases of $V_i>0$/$V_i<0$.
$x_i^2$ measures the contribution to the expansion due to the field's
kinetic energy density, while $\pm y_i^2$ represents the contribution
of the potential energy.
Critical points correspond to fixed points where $x'_i=0$,
$y'_i=0$, and there are self-similar solutions with
\begin{equation}
\frac{\dot{H}}{H^2}=-3\sum_{i=1}^{m}x_i^2
\end{equation}
This corresponds to a power-law solution for the scalar factor
\begin{equation}
a\propto t^p, \qquad \textrm{where} \quad
p=\frac{1}{3\sum_{i=1}^{m}x_i^2}
\end{equation}
The system (\ref{5E}) and (\ref{6E}) has at most one $m$-dimensional
sphere embedded in $2m$-dimensional phase-space corresponding to
kinetic-dominated solutions, and $(2^m-1)$ fixed points, one of
which is a $m$-kinetic-potential scaling solution.

In order to analysis the stability of the critical points, we only
consider the cosomlogies containing two scalar fields. There are
one unit circle and three fixed points as follows:

\textbf{S:} $x_1^2+x_2^2=1$, $y_1=y_2=0$ \\
These kinetic-dominated solutions exist for any form of the
potential, which are equivalent to stiff-fluid dominated evolution
with $a\propto t^{1/3}$, irrespective of the nature of the
potential. In $x_1^2+x_2^2=1$ and $y_1=y_2=0$ sub-space, each
fixed point is marginally stable. Linear perturbations $y_1\to
y_1+\delta y_1$ and $y_2\to y_2+\delta y_2$ yield two eigenmodes
\begin{eqnarray*}
\delta y'_1=(3-\lambda \sqrt{\frac{3}{2}}x_1)\delta y_1 \\
\delta y'_2=(3-\lambda \sqrt{\frac{3}{2}}x_2)\delta y_2
\end{eqnarray*}
Thus the solutions are stable to potential energy perturbations
for $\lambda >\frac{\sqrt{6}}{x_1}$ and $\lambda
>\frac{\sqrt{6}}{x_2}$, where $x_1,x_2>0$. This result shows that
there maybe exit stable points only for sufficiently steep
$(\lambda >2\sqrt{3})$ potential.

\textbf{A:} $x_1=\frac{\lambda}{\sqrt{6}},y_1=\sqrt{\pm (1-\frac{\lambda
^2}{6})},x_2=y_2=0$,
or
$x_1=y_1=0,x_2=\frac{\lambda}{\sqrt{6}},y_2=\sqrt{\pm (1-\frac{\lambda
^2}{6})})$ \\
The two single-potential-kinetic solutions exist for sufficiently
flat $(\lambda ^2<6)$ positive potentials or steep $(\lambda
^2>6)$ negative potentials. The power-law exponent, $p=2/\lambda
^2$, depends on the slope of the potential. In order to study the
stability of the critical points we perturb lineatly about the
critical points and get three eignmodes
\begin{eqnarray*}
\delta x'_1&=&(\lambda ^2-6)/2\delta x_1 \\
\delta x'_2&=&(\lambda ^2-6)/2\delta x_2 \\
\delta y'_2&=&\frac{\lambda ^2}{2}\delta y_2
\end{eqnarray*}
Thus the single-potential-kinetic solutions are unatable for the
potive and negative potentials. This indicate that the stability
is destroyed by the potential energy perturbations of another
scalar field.

\textbf{B:} $x_1=x_2=\frac{\lambda}{2\sqrt{6}}$,
$y_1=y_2=\sqrt{\pm
(\frac{1}{2}-\frac{\lambda ^2}{24})}$ \\
The double-potential-kinetic scaling solution exist for flat
$(\lambda ^2<12)$ positive potentials, or steep $(\lambda ^2>12)$
negative potentials. This corresponds to a power-law solution with
$a\propto t^{4/\lambda ^2}$. Linear perturbations yield three
eigenmodes
\begin{eqnarray*}
\delta x'_1&=&-3\delta x_1 \\
\delta x'_2&=&(\lambda ^2-6)/2\delta x_2 \\
\delta y'_2&=&0
\end{eqnarray*}
For positive potentials, the scaling solution is marginally stable
for $\lambda ^2<6$, while it is unstable for $6<\lambda ^2<12$.
For negative potentials the scaling solution is never stable.

\vs{4}
\begin{tabular}{||p{4cm}|p{2.8cm}|p{2.5cm}|p{4cm}||} \hline \hline
Critical points & Existence & Eigenvalues & Stability \\ \hline
$x_1^2+x_2^2=1$, $y_1=y_2=0$ & all $\lambda$ & $(3-\lambda
\sqrt{\frac{3}{2}}x_1)$; $(3-\lambda \sqrt{\frac{3}{2}}x_2)$ &
stable $(\lambda ^2>12)\qquad $ unstable $(\lambda ^2<12)$ \\
\hline $(\frac{\lambda}{\sqrt{6}},\sqrt{\pm (1-\frac{\lambda
^2}{6})},0,0)$, $(0,0,\frac{\lambda}{\sqrt{6}},\sqrt{\pm
(1-\frac{\lambda ^2}{6})})$ & $\lambda ^2<6 (V>0)$ $\lambda ^2>6
(V<0)$ & $(\lambda ^2-6)/2;\quad $ $(\lambda ^2-6)/2$;
$\frac{\lambda ^2}{2}$ & unstable \\ \hline
$x_1=x_2=\frac{\lambda}{2\sqrt{6}}$, $y_1=y_2=\sqrt{\pm
(\frac{1}{2}-\frac{\lambda ^2}{24})}$ & $\lambda ^2<12 (V>0)$
$\lambda ^2>12 (V<0)$ & $(\lambda ^2-6)/2;\quad $ $-3$; 0 & stable
$(V>0,\lambda ^2<6)$ unstable $(V<0)$ \\ \hline \hline
\end{tabular}
\begin{center}
TABLE 2. The properties of the critical points
\end{center}




In summary, we have presented a phase-space analysis of the
evolution of a spatially flat FRW unvierse containing two scalar
fields with positive and negative exponential potentials. The
regions of $(\lambda )$ parameter space lead to different
qualitative evolution.
\begin{itemize}
\item For steep positive potentials $(V>0, \lambda ^2>12)$,
 only a circle $S$ exists, some kinetic-dominated
 scaling solutions of which are the late-time attrators.
\item For intermediate positive potentials $(V>0, 6<\lambda ^2<12)$,
 a circle $S$ and a fixed point $B$ exist. There exist no stable
 points.
\item For flat positive potentials $(V>0, \lambda ^2<6)$, all
 critical points exist. The double-kinetic -potential scaling
 solution is the unique late-time attractor.
\item For steep
 negative potentials $(V<0, \lambda ^2>12)$, all critical points
 exist. some kinetic-dominated scaling solutions of which are the
 late-time attrators.
\item For intermediate negative potentials $(V<0, 6<\lambda ^2<12)$,
 a circle $S$ and two fixed point $A$ exist.
 There exist no stable points.
\item For flat negative potentials $(V<0, \lambda ^2<6)$,
 only a circle $S$ exists, which are never stable.
\end{itemize}


Generalizing above discussion to $m$ scalar fields is straight.
The multi-kinetic-potential scaling solution exists for positive
potentials $(\lambda ^2<6m)$, or negative potentials $(\lambda
^2<6m)$. As long as each potential satisfies $\lambda ^2<2m$, this
power-law solution is inflationary. For the case $m=1$, the
dimensionless constant $\lambda$ must be smaller than $\sqrt{2}$
to guarantee power-law inflation~\cite{HW}. However, presently
known theories yield expotential potentials with $\lambda
>\sqrt{2}$. In such cases multiple scalar fields may proceed
inflation. The reason for this behavior is that while each field
experiences the `downhill' force from its own potential, it feels
the friction from all the scalar fields via their contribution to
the expansion~\cite{LM}.


We emphasize that we have assumed that there is no direct coupling
between these exponential potentials and each scalar field has an
identical potential. It is worth studying further the case their
potentials have different slopes.




\begin{thebibliography}{99}

\bibitem{CD}K.Coble, S.Dodelson and J.A.Frieman, Phys.Rev.
{\bf D55} (1997) 1851;\\
R.R.Caldwell and P.J.Steinhardt, Phys.Rev. {\bf D57} (1998) 6057;\\
I.Zlatev, L.M.Wang and P.J.Steinhardt, Phys.Rev.Lett.
{\bf 82} (1999) 896;\\
P.J.Steinhardt, L.M.Wang and I.Zlatev, Phys.Rev. {\bf D59} (1999)
123504.
\bibitem{GU}A.H.Guth, Phys.Rev. {\bf D23} (1981) 347;\\
A.D.Linde, Phys.Lett. {\bf B108} (1982) 389;\\
A.D.Linde, Phys.Lett. {\bf B129} (1983) 177.
\bibitem{CL}E.J.Copland, A.R.Liddle, D.H.Lyth, E.D.Stewart and D.Wands,
Phys.Rev. {\bf D49} (1994) 6410;\\
M.Dine, L.Randall and S.Thomas, Phys.Rev.Lett. {\bf 75} (1995)
398.
\bibitem{ST}E.D.Stewart, Phys.Rev. {\bf D51} (1995) 6847.
\bibitem{LM}A.R.Liddle, A.Mazumdar and F.E.Schunck,
astro-ph/9804177.
\bibitem{PCZ}Yun-Song Piao, Wenbin Lin,
Xinmin Zhang and Yuan-Zhong Zhang, Phys. Lett. {\bf B528}
(2002) 188, hep-ph/0109076; \\
Y.S. Piao, R.G. Cai, X.M. Zhang and
Y.Z. Zhang, Phys. Rev. {\bf D66} (2002) 121301, hep-ph/0207143.
\bibitem{HL}Q.G. Huang and M. Li, hep-ph/0302208.
\bibitem{FLM}F.Lucchin and S.Matarrese, Phys.Rev. {\bf D32} (1985) 1316;\\
Y.Kitada and K.I.Maeda, Class.Quant.Grav. {\bf 10} (1993) 703.
\bibitem{JJH}J.J.Halliwell, Phys.Lett. {\bf B185} (1987) 341;\\
A.B.Burd and J.D.Barrow, Nucl.Phys. {\bf B308} (1988) 929;\\
A.A.Coley, J.Ibanez and R.J.van den Hoogen, J.Math.Phys. {\bf 38}
(1997) 5256.
\bibitem{CLW}E.J.Copeland, A.R.Liddle and D.Wands, Phys.Rev. {\bf D57}
(1998) 4686;\\
A.P.Billyard, A.A.Coley and R.J.van den Hoogen, Phys.Rev. {\bf
D58} (1998)
123501;\\
R.J.van den Hoogen, A.A.Coley and D.Wands, Class.Quant.Grav. {\bf
16} (1999) 1843.
\bibitem{EW}G.F.R.Ellis and J.Wainwright, {\it Dynamical systems in
cosmology} (Cambridge UP, 1997).
\bibitem{HW}I.P.C.Heard and D.Wands, gr-qc/0206085.


\end{thebibliography}

\end{document}
