
\documentclass[a4paper,12pt]{article}

\begin{document}

\author{C. Bizdadea\thanks{%
e-mail address: bizdadea@central.ucv.ro}, E. M. Cioroianu\thanks{%
e-mail address: manache@central.ucv.ro}, I. Negru\thanks{%
e-mail address: inegru@central.ucv.ro}, S. O. Saliu\thanks{%
e-mail address: osaliu@central.ucv.ro} \\
Faculty of Physics, University of Craiova\\
13 A. I. Cuza Str., Craiova RO-1100, Romania}
\title{Lagrangian interactions within a special class of covariant mixed-symmetry
type tensor gauge fields}
\maketitle

\begin{abstract}
Consistent nontrivial interactions within a special class of covariant
mixed-symmetry type tensor gauge fields of degree three are constructed from
the deformation of the solution to the master equation combined with
specific cohomological techniques. In spacetime dimensions strictly greater
than four, the only consistent interaction terms are those gauge invariant
under the original symmetry. Only in four spacetime dimensions the gauge
symmetry is found deformed.

PACS number: 11.10.Ef
\end{abstract}

\section{Introduction}

The cohomological development of the Becchi-Rouet-Stora-Tyutin symmetry,
allowed, among others, the determination of the general solution to the
Wess-Zumino consistency condition \cite{1and2}--\cite{10and23}, of the
general form of the counterterms involved with the renormalization of gauge
invariant operators (Kluberg-Stern and Zuber conjecture) \cite{6and23}--\cite
{12and3}, the cohomological approach to global symmetries and conservation
laws in classical gauge theories \cite{13and4}--\cite{16and4}, the
reformulation of the problem of constructing consistent interactions that
can be introduced in gauge invariant theories as a problem of deformation of
the solution to the classical master equation \cite{17and5}--\cite{21and5},
etc. In particular, the cohomological reformulation of consistent
interactions in theories with gauge symmetries has led to important results,
such as the impossibility of cross-interactions in multigraviton theories 
\cite{22and6}--\cite{24and6}, or the existence of the Seiberg-Witten map in
noncommutative field theories whose commutative versions allow rigid (that
cannot be deformed) gauge symmetries \cite{25and7}--\cite{27and7}. The
fundamental algebraic structure on which the BRST symmetry is based is the
graded differential complex endowed with a second-order nilpotent
differential, together with the (local) cohomology of this differential.
Lately, the usual differential tools have been developed to cover
generalized differential $N$-complexes \cite{28and8}--\cite{29and810} for
irreducible tensor gauge fields of mixed Young symmetry type, endowed with
higher-order nilpotent operators. In this context, the generalized
Poincar\'{e} lemma, that governs the cohomology of $N$-complexes related to
tensor fields of mixed symmetry type, has been formulated and proved. This
modern differential algebraic setting helped at solving some nice problems,
like, for instance, the interpretation of the construction of the
Pauli-Fierz theory \cite{30and9}, the dual formulation of linearized gravity 
\cite{31and9}--\cite{31and11}, or the impossibility of consistent
cross-interactions in the dual formulation of linearized gravity \cite
{31and11}.

In this paper we investigate all consistent Lagrangian interactions that can
be added to a special class of covariant mixed-symmetry type tensor gauge
fields of degree three that transform according to a reducible
representation of the Lorentz group. Our main results are: (i) in spacetime
dimensions strictly greater than four, the only consistent interactions are
gauge invariant terms, that do not deform the gauge symmetry; (ii) in four
dimensions, there appear nontrivial consistent deformations that modify the
gauge transformations and their reducibility, but not the gauge algebra.

Our strategy goes as follows. Initially, we generate the
antibracket-antifield BRST symmetry of the uncoupled model with covariant
mixed-symmetry type tensor fields of degree three in an arbitrary spacetime
dimension. Next, we apply the deformation procedure, and compute the
first-order deformation of the solution to the master equation, whose
integrand belongs to the zeroth order local cohomology of the free BRST
differential $H^{0}\left( s|d\right) $, where $s$ and $d$ mean the BRST,
respectively, the exterior spacetime differentials. Further, we investigate
the higher-order deformations. Finally, we analyse the Lagrangian
description of the interacting theory, namely, the Lagrangian action, the
deformed gauge transformations, and their tensor structure. In $D>4$
spacetime dimensions, the first-order deformation reduces to its antighost
number zero component, while all higher-order deformations can be taken to
vanish. Thus, the original gauge symmetry is not deformed in this case. For $%
D=4$, the first-order deformation is parametrized in terms of two arbitrary
constants. Its consistency requires that one of the constants must vanish.
Under these circumstances, the deformation of the solution to the master
equation that is consistent at all orders in the coupling constant concludes
at the first order. It is local, manifestly Lorentz covariant, and, most
important, nontrivial. It follows that the gauge transformations of the
mixed-symmetry type tensor fields are truly deformed, although their gauge
algebra remains abelian. The interacting model is also first-order
reducible, but the reducibility relations hold on-shell, in contrast to the
starting uncoupled theory.

The paper is organized into nine sections. Section 2 introduces the
Lagrangian model in an arbitrary spacetime dimension. Section 3 is devoted
to the BRST symmetry of the uncoupled theory. In Section 4 we briefly review
the Lagrangian mechanism for constructing consistent interactions from the
point of view of the antibracket-antifield BRST method. Section 5 focuses on
the deformation procedure in more that four spacetime dimensions and argues
that the only consistent interactions simply add gauge invariant terms to
the starting action. In Section 6 we restrict ourselves to the
four-dimensional case, and reconsider the uncoupled model in first-order
form. In this context, in Section 7 we solve the main equations of the
antibracket-antifield deformation procedure, and show that there are
consistent deformations of the solution to the classical master equation at
order one in the coupling constant. We investigate the conditions that must
be fulfilled such that these are second- and higher-order consistent. In
Section 8 we identify the Lagrangian version of the interacting model in
four dimensions, and analyse the tensor structure of the new gauge theory.
Section 9 ends the paper with some conclusions.

\section{Introducing the free model}

We start from the Lagrangian action for a special class of covariant
mixed-symmetry type tensor gauge fields of degree three 
\begin{equation}
S_{0}\left[ A_{\alpha \beta }^{\;\;\;(\sigma )}\right] =-\frac{1}{12}\int
d^{D}x\,F_{\alpha \beta \gamma }^{\;\;\;\;(\sigma )}F_{\;\;\;\;(\sigma
)}^{\alpha \beta \gamma },  \label{fcol1}
\end{equation}
where the fields are only antisymmetric in their first two indices $%
A_{\alpha \beta }^{\;\;\;(\sigma )}=-A_{\alpha \beta }^{\;\;\;(\sigma )}$.
The tensor fields $A_{\alpha \beta }^{\;\;\;(\sigma )}$ can be regarded as
being described by a Young diagram with three cells and two columns.
However, they \textit{do not satisfy} the identity $A_{\left[ \alpha \beta
(\sigma )\right] }=0$, associated with the Young symmetrizer of the
corresponding diagram. Actually, they transform according to a reducible
representation of the Lorentz group. This is the reason why we use the
notation $A_{\alpha \beta (\sigma )}$ instead of the standard one $A_{\alpha
\beta |\sigma }$. Spacetime indices are raised and lowered with the flat
Minkowskian metric of ``mostly plus'' signature in $D$ dimensions ($D\geq 4$%
): $-+++\cdots $. We define the field strength components in the usual
manner 
\begin{equation}
F_{\alpha \beta \gamma }^{\;\;\;\;(\sigma )}=\partial _{\alpha }A_{\beta
\gamma }^{\;\;\;(\sigma )}+\partial _{\beta }A_{\gamma \alpha
}^{\;\;\;(\sigma )}+\partial _{\gamma }A_{\alpha \beta }^{\;\;\;(\sigma
)}\equiv \partial _{\left[ \alpha \right. }A_{\left. \beta \gamma \right]
}^{\;\;\;\;(\sigma )},  \label{fcol2}
\end{equation}
where we make the convention that $\left[ \alpha \beta \cdots \right] $
signifies plain antisymmetry with respect to the indices between brackets,
without additional numerical factors. The tensor fields $A_{\alpha \beta
}^{\;\;\;(\sigma )}$ can also be viewed in some sense as a collection of
two-forms, where the collection index is spacetime-like. This motivates the
form (\ref{fcol2}) of their field strengths, as well as the expression (\ref
{fcol1}) of the Lagrangian action. It is widely known that gauge theories
involving antisymmetric tensor fields are important due to their connection
with string theory and supergravity models \cite{string1}--\cite{string6}.
In particular, interacting two-forms (described by the Freedman-Townsend
model) play a special role due on the one hand to its link with Witten's
string theory \cite{string7} and, on the other hand, to its equivalence to
the nonlinear sigma model \cite{string8}. The interacting theory to be
derived by us in Section 8 resembles in a way to the Freedman-Townsend
model, but clearly exhibits new features. The gauge invariances of action (%
\ref{fcol1}) are given by 
\begin{equation}
\delta _{\epsilon }A_{\alpha \beta }^{\;\;\;(\sigma )}=\partial _{\left[
\alpha \right. }\epsilon _{\left. \beta \right] }^{\;\;\;(\sigma )}\equiv 
\stackrel{(0)}{R}_{\alpha \beta \;\;\;\;(\rho )}^{\;\;\;(\sigma )\gamma
}\epsilon _{\gamma }^{\;(\rho )},  \label{fcol3}
\end{equation}
with $\epsilon _{\gamma }^{\;(\rho )}$ bosonic gauge parameters, that are
tensors of degree two with no symmetry, such that the gauge generators in
condensed De Witt notations are expressed through 
\begin{equation}
\stackrel{(0)}{R}_{\alpha \beta \;\;\;\;(\rho )}^{\;\;\;(\sigma )\gamma
}=\delta _{\rho }^{\sigma }\partial _{\left[ \alpha \right. }\delta _{\left.
\beta \right] }^{\gamma }.  \label{fcol4}
\end{equation}
The above gauge transformations are abelian and exhibit off-shell
first-stage reducible, the accompanying reducibility relations and functions
respectively reading as 
\begin{equation}
\stackrel{(0)}{R}_{\alpha \beta \;\;\;\;(\rho )}^{\;\;\;(\sigma )\gamma }%
\stackrel{(0)}{Z}_{\gamma \;\;\;(\tau )}^{\;(\rho )}=0,  \label{fcol5}
\end{equation}
\begin{equation}
\stackrel{(0)}{Z}_{\gamma \;\;\;(\tau )}^{\;(\rho )}=\delta _{\tau }^{\rho
}\partial _{\gamma }.  \label{fcol6}
\end{equation}

In consequence, we deal with a usual free linear gauge theory of Cauchy
order three (local, polynomial in the fields and their derivatives, that
satisfies the standard regularity conditions \cite{32and12}, whose gauge
algebra is not open, and with a finite reducibility order). As the number of
physical degrees of freedom carried by a tensor field $A_{\alpha \beta
}^{\;\;\;(\sigma )}$ is equal to $D\left( D-2\right) \left( D-3\right) /2$,
the subsequent analysis is meaningful only for $D\geq 4$.

\section{BRST symmetry of the free theory}

We are interested in constructing the consistent Lagrangian
(self)interactions that can be added to action (\ref{fcol1}) without
changing the content of the field spectrum or the number of independent
gauge symmetries. In view of this, we apply the general rules of the
antibracket-antifield deformation procedure based on (co)homological
techniques \cite{17and5}, \cite{21and5}.

The first step in the development of the antibracket-antifield deformation
approach consists in generating the Lagrangian BRST symmetry of the free
model under study. The Lagrangian BRST complex is organized into the
field/ghost, respectively, antifield spectrum 
\begin{equation}
\Phi ^{\Delta }=\left( A_{\alpha \beta }^{\;\;\;(\sigma )},\eta _{\alpha
}^{\;(\sigma )},C^{(\sigma )}\right) ;\;\Phi _{\Delta }^{*}=\left(
A_{\;\;\;\;(\sigma )}^{*\alpha \beta },\eta _{\;\;\;(\sigma )}^{*\alpha
},C_{(\sigma )}^{*}\right) ,  \label{fcol12}
\end{equation}
with the Grassmann parities 
\begin{equation}
\varepsilon \left( A_{\alpha \beta }^{\;\;\;(\sigma )}\right) =0=\varepsilon
\left( C^{(\sigma )}\right) ,\;\varepsilon \left( \eta _{\alpha }^{\;(\sigma
)}\right) =1,  \label{fcol13}
\end{equation}
\begin{equation}
\varepsilon \left( \Phi _{\Delta }^{*}\right) =\left( \varepsilon \left(
\Phi ^{\Delta }\right) +1\right) \;\mathrm{mod}\;2.  \label{fcol15}
\end{equation}
While the ghosts $\eta _{\alpha }^{\;(\sigma )}$ are due to the gauge
symmetry, the ghosts for ghosts $C^{(\sigma )}$ are required by the
first-order reducibility relations. The Lagrangian BRST symmetry acts like a
differential $s$ ($s^{2}=0$), which we assume to behave like a right
derivation. Since the gauge algebra is abelian and the reducibility
functions are field-independent, it follows that $s$ reduces to the sum
between the Koszul-Tate differential $\delta $ and the exterior longitudinal
derivative $\gamma $ only 
\begin{equation}
s=\delta +\gamma ,  \label{fcol16}
\end{equation}
that are respectively graded in terms of the antighost number ($\mathrm{agh}$%
) and the pure ghost number ($\mathrm{pgh}$).\footnote{%
For more general gauge theories, $s$ has a richer structure than in (\ref
{fcol16}), being necessary the addition of supplementary operators, $%
\stackrel{(k)}{s}$, with $\mathrm{agh}\left( \stackrel{(k)}{s}\right) =k>0$,
in order to ensure the second-order nilpotency of $s$.} While the
Koszul-Tate differential ($\delta ^{2}=0$, $\mathrm{agh}\left( \delta
\right) =-1$, $\mathrm{pgh}\left( \delta \right) =0$) realizes a resolution
of smooth functions defined on the stationary surface of field equations,
the exterior longitudinal derivative ($\mathrm{pgh}\left( \gamma \right) =1$%
, $\mathrm{agh}\left( \gamma \right) =0$) anticommutes with $\delta $ and
turns out to be a true differential in the particular case of the model
under study $\left( \gamma ^{2}=0\right) $. Its cohomological space at pure
ghost number zero computed in the homology of $\delta $, $H^{0}\left( \gamma
|H_{*}\left( \delta \right) \right) $, is given by the algebra of Lagrangian
physical observables, and is in the meantime isomorphic to the zeroth order
cohomological space of $s$, $H^{0}\left( s\right) $, that contains the
so-called BRST observables \cite{33and13}--\cite{35and13}. The degrees ($%
\mathrm{agh}$) and ($\mathrm{pgh}$) of the BRST generators (\ref{fcol12})
are given by 
\begin{equation}
\mathrm{agh}\left( \Phi ^{\Delta }\right) =0,\;\mathrm{agh}\left(
A_{\;\;\;\;(\sigma )}^{*\alpha \beta }\right) =1,\;\mathrm{agh}\left( \eta
_{\;\;\;(\sigma )}^{*\alpha }\right) =2,\;\mathrm{agh}\left( C_{(\sigma
)}^{*}\right) =3,  \label{fcol17}
\end{equation}
\begin{equation}
\mathrm{pgh}\left( \Phi _{\Delta }^{*}\right) =0,\;\mathrm{pgh}\left(
A_{\alpha \beta }^{\;\;\;(\sigma )}\right) =0,\;\mathrm{pgh}\left( \eta
_{\alpha }^{\;(\sigma )}\right) =1,\;\mathrm{pgh}\left( C^{(\sigma )}\right)
=2,  \label{fcol18}
\end{equation}
while the actions of $\delta $ and $\gamma $ read as 
\begin{equation}
\delta \Phi ^{\Delta }=0,\;\delta A_{\;\;\;\;(\sigma )}^{*\alpha \beta }=-%
\frac{\delta S_{0}^{L}}{\delta A_{\alpha \beta }^{\;\;\;(\sigma )}}\equiv -%
\frac{1}{2}\partial _{\gamma }F_{\;\;\;\;(\sigma )}^{\gamma \alpha \beta },
\label{fcol19}
\end{equation}
\begin{equation}
\delta \eta _{\;\;\;(\sigma )}^{*\alpha }=-2\partial _{\beta
}A_{\;\;\;\;(\sigma )}^{*\beta \alpha },\;\delta C_{(\sigma )}^{*}=\partial
_{\alpha }\eta _{\;\;\;(\sigma )}^{*\alpha },  \label{fcol20}
\end{equation}
\begin{equation}
\gamma \Phi _{\Delta }^{*}=0,\;\gamma A_{\alpha \beta }^{\;\;\;(\sigma
)}=\partial _{\left[ \alpha \right. }\eta _{\left. \beta \right]
}^{\;\;\;(\sigma )},\;\gamma \eta _{\alpha }^{\;(\sigma )}=\partial _{\alpha
}C^{(\sigma )},\;\gamma C^{(\sigma )}=0.  \label{fcol21}
\end{equation}
The overall degree of the BRST complex is named ghost number ($\mathrm{gh}$)
and is defined like $\mathrm{gh}=\mathrm{pgh}-\mathrm{agh}$, such that $%
\mathrm{gh}\left( s\right) =1$. An important property of the Lagrangian BRST
symmetry is that it is canonically generated in a structure named
antibracket and usually denoted by $\left( ,\right) $ \cite{rusi1}--\cite
{35and13} via a generator $\stackrel{(0)}{S}$%
\begin{equation}
s\cdot =\left( \cdot ,\stackrel{(0)}{S}\right) .  \label{fcol21a}
\end{equation}
The fields/ghosts are decreed conjugated in the antibracket with the
associated antifields 
\begin{equation}
\left( \Phi ^{\Delta },\Phi _{\Delta ^{\prime }}^{*}\right) =\delta _{\Delta
^{\prime }}^{\Delta }.  \label{fcol22}
\end{equation}
The second-order nilpotency of $s$ is equivalent to the fact that $\stackrel{%
(0)}{S}$ is solution to the classical master equation 
\begin{equation}
\left( \stackrel{(0)}{S},\stackrel{(0)}{S}\right) =0,\;\varepsilon \left( 
\stackrel{(0)}{S}\right) =0,\;\mathrm{gh}\left( \stackrel{(0)}{S}\right) =0,
\label{fcol23}
\end{equation}
with some boundary conditions. In the case of the model investigated here,
with the help of the relations (\ref{fcol16}) and (\ref{fcol19}--\ref
{fcol21a}), one finds that the complete solution to the master equation can
be written in the form 
\begin{equation}
\stackrel{(0)}{S}=S_{0}\left[ A_{\alpha \beta }^{\;\;\;(\sigma )}\right]
+\int d^{D}x\left( A_{\;\;\;\;(\sigma )}^{*\alpha \beta }\partial _{\left[
\alpha \right. }\eta _{\left. \beta \right] }^{\;\;\;(\sigma )}+\eta
_{\;\;\;(\sigma )}^{*\alpha }\partial _{\alpha }C^{(\sigma )}\right) ,
\label{fcol24}
\end{equation}
and we observe that it contains pieces of antighost number equal to zero,
one, respectively, two.

\section{Brief review of antibracket-antifield deformation procedure}

In order to develop the general approach to the problem of consistent
interactions in gauge theories from the perspective of the
antibracket-antifield deformation procedure, we briefly recall some basic
results \cite{17and5}, \cite{21and5}. Assume that 
\begin{equation}
\bar{S}_{0}\left[ A_{\alpha \beta }^{\;\;\;(\sigma )}\right] =S_{0}\left[
A_{\alpha \beta }^{\;\;\;(\sigma )}\right] +g\int d^{n}x\,a_{0}+O\left(
g^{2}\right) ,  \label{fcol26}
\end{equation}
defines a consistent deformation of action (\ref{fcol1}), with deformed
gauge symmetry 
\begin{equation}
\bar{\delta}_{\epsilon }A_{\alpha \beta }^{\;\;\;(\sigma )}=\bar{R}_{\alpha
\beta \;\;\;\;(\rho )}^{\;\;\;(\sigma )\gamma }\epsilon _{\gamma }^{\;(\rho
)},  \label{defgsym}
\end{equation}
where 
\begin{equation}
\bar{R}_{\alpha \beta \;\;\;\;(\rho )}^{\;\;\;(\sigma )\gamma }=\stackrel{(0)%
}{R}_{\alpha \beta \;\;\;\;(\rho )}^{\;\;\;(\sigma )\gamma }+g\stackrel{(1)}{%
R}_{\alpha \beta \;\;\;\;(\rho )}^{\;\;\;(\sigma )\gamma }+O\left(
g^{2}\right) .  \label{fcol27}
\end{equation}
Consistent means that $\bar{S}_{0}$ is fully invariant under the gauge
symmetry $\bar{\delta}_{\epsilon }$ (at any order in the coupling constant $%
g $), $\bar{\delta}_{\epsilon }\bar{S}_{0}=0$. Moreover, we add the demand
that the number of independent deformed gauge symmetries should remain
unchanged with respect to the free theory, which implies the existence of
some first-order reducibility functions for the interacting theory 
\begin{equation}
\bar{Z}_{\gamma \;\;\;(\tau )}^{\;(\rho )}=\stackrel{(0)}{Z}_{\gamma
\;\;\;(\tau )}^{\;(\rho )}+g\stackrel{(1)}{Z}_{\gamma \;\;\;(\tau
)}^{\;(\rho )}+O\left( g^{2}\right) ,  \label{fcol28}
\end{equation}
such that the new reducibility relations may now hold on-shell, i.e., on the
stationary surface of field equations for $\bar{S}_{0}$%
\begin{equation}
\bar{R}_{\alpha \beta \;\;\;\;(\rho )}^{\;\;\;(\sigma )\gamma }\bar{Z}%
_{\gamma \;\;\;(\tau )}^{\;(\rho )}\approx 0.  \label{fcol29}
\end{equation}
It is possible to reformulate more economically this problem in terms of the
solution to the master equation. The key observation on which this approach
relies is that a consistent deformation of the free action (\ref{fcol1}) and
of its gauge symmetries (\ref{fcol3}) defines a deformation of the solution (%
\ref{fcol24}) of the master equation that preserves both the master equation
and the field/ghost-antifield spectrum. Indeed, if the interactions can be
consistently constructed, then the solution (\ref{fcol24}) of the master
equation for the free theory can be deformed into the solution $\bar{S}$ of
the master equation for the interacting theory 
\begin{equation}
\stackrel{(0)}{S}\rightarrow \bar{S}=\stackrel{(0)}{S}+g\stackrel{(1)}{S}%
+g^{2}\stackrel{(2)}{S}+\cdots ,  \label{fcol25}
\end{equation}
\begin{equation}
\left( \stackrel{(0)}{S},\stackrel{(0)}{S}\right) =0\rightarrow \left( \bar{S%
},\bar{S}\right) =0.  \label{fcol7}
\end{equation}
The master equation $\left( \bar{S},\bar{S}\right) =0$ guarantees that the
consistency requirements on $\bar{S}_{0}$, $\bar{R}_{\alpha \beta
\;\;\;\;(\rho )}^{\;\;\;(\sigma )\gamma }$ and $\bar{Z}_{\gamma \;\;\;(\tau
)}^{\;(\rho )}$ are fulfilled. The main advantage in reformulating the
problem of consistent interactions as the problem of deforming the master
equation is that we can make use of the cohomological techniques of the
deformation theory. The master equation for $\bar{S}$ splits according to
the deformation parameter $g$ as 
\begin{equation}
g^{0}:\left( \stackrel{(0)}{S},\stackrel{(0)}{S}\right) =0,  \label{fcol9a}
\end{equation}
\begin{equation}
g^{1}:2\left( \stackrel{(1)}{S},\stackrel{(0)}{S}\right) =0,  \label{fcol10}
\end{equation}
\begin{equation}
g^{2}:2\left( \stackrel{(2)}{S},\stackrel{(0)}{S}\right) +\left( \stackrel{%
(1)}{S},\stackrel{(1)}{S}\right) =0,  \label{fcol11}
\end{equation}
\[
\vdots 
\]
The equation (\ref{fcol9a}) is checked by assumption, while the equation (%
\ref{fcol10}) stipulates that the first-order deformation is a cocycle of
the free BRST differential (\ref{fcol21a}) 
\begin{equation}
s\stackrel{(1)}{S}=0.  \label{fcol30}
\end{equation}
However, only cohomologically nontrivial solutions to (\ref{fcol10}) should
be taken into account, since trivial (BRST-exact) ones can be eliminated by
a (in general non-linear) field redefinition \cite{21and5}. In this way, we
conclude that the nontrivial deformations $\stackrel{(1)}{S}$ are determined
by the zeroth order cohomological space $H^{0}\left( s\right) $ of the
undeformed theory, which is generically nonempty due to its isomorphism to
the space of physical observables of the free theory. The next equation, (%
\ref{fcol11}), implies that $\left( \stackrel{(1)}{S},\stackrel{(1)}{S}%
\right) $ must be trivial (BRST-exact) in the cohomology of $s$ at ghost
number one, $H^{1}\left( s\right) $. There are no obstructions in finding
solutions to the remaining equations (\ref{fcol11}), etc. (for instance, see 
\cite{21and5})) as long as no restrictions on the interactions, such as
spacetime locality or manifest Lorentz covariance, are imposed. In general,
the resulting interactions may be nonlocal, and there might even appear
obstructions if one insists on their locality. However, as it will be seen
below, in the case of the model under considerations we obtain only local
and manifestly covariant interactions.

\section{Deformations in $D>4$}

There are three different types of deformations, namely: (i) that modify
only the Lagrangian action, but not the gauge symmetry; (ii) that change the
gauge symmetries, but not their algebra; and (iii) that deform also the
gauge algebra. In spacetime dimensions strictly greater than four ($D>4$),
the only consistent nontrivial deformations for the model under study are
shown to be only of type (i), and hence they merely add to the free action (%
\ref{fcol1}) interaction terms that are invariant under the original gauge
transformations (\ref{fcol3}). This can be seen by developing the
first-order deformation of the solution to the master equation according to
the antighost number 
\begin{equation}
\stackrel{(1)}{S}=\sum_{k=0}^{n}\int d^{D}x\,a_{k},\;\mathrm{gh}\left(
a_{k}\right) =0,\;\varepsilon \left( a_{k}\right) =0,\;\mathrm{agh}\left(
a_{k}\right) =k.  \label{fcoli1}
\end{equation}
The equation for $\stackrel{(1)}{S}$, namely, (\ref{fcol30}) takes the local
form 
\begin{equation}
s\left( \sum_{k=0}^{n}a_{k}\right) =\partial _{\mu }h^{\mu }.  \label{fcoli2}
\end{equation}
The number of terms from the expansion of $\stackrel{(1)}{S}$ is finite and
it can be shown that one can take the piece of highest antighost number to
be $\gamma $-closed, $\gamma a_{n}=0$, and so $a_{n}\in H\left( \gamma
\right) $. This can be done by using the arguments from \cite{7and23}, \cite
{23and6}, \cite{32and12}, \cite{36and14}. From the definitions (\ref{fcol19}%
--\ref{fcol21}), it is simple to see that $H\left( \gamma \right) $ is
generated by the field strength components $F_{\alpha \beta \gamma
}^{\;\;\;\;(\sigma )}$ and their derivatives, by the antifields $\Phi
_{\Delta }^{*}$ together with their derivatives, as well as by the
undifferentiated ghosts for ghosts $C^{(\sigma )}$. Since $\mathrm{gh}\left(
C^{(\sigma )}\right) =2$, it follows that $n=2m$ and 
\begin{equation}
a_{n}\equiv a_{2m}=\mu ^{\rho _{1}\cdots \rho _{m}}\left( \left[ \Phi
_{\Delta }^{*}\right] ,\left[ F_{\alpha \beta \gamma }^{\;\;\;\;(\sigma
)}\right] \right) C_{(\rho _{1})}\cdots C_{(\rho _{m})},  \label{fcoli3}
\end{equation}
where the notation $f\left( \left[ q\right] \right) $ signifies that $f$
depends on $q$ and its derivatives up to a finite order. The spacetime
derivatives of the ghosts for ghosts are exact in $H\left( \gamma \right) $
according to the third relation in (\ref{fcol21}), and hence trivial, such
that they are discarded from $a_{2m}$. By projecting the equation (\ref
{fcoli2}) on antighost number $\left( 2m-1\right) $, we infer the equation $%
\delta a_{2m}+\gamma a_{2m-1}=\partial _{\mu }h^{\prime \mu }$. Replacing
the expression (\ref{fcoli3}) in the last equation, it results that a
necessary condition for the existence of $a_{2m-1}$ is that the functions $%
\mu ^{\rho _{1}\cdots \rho _{m}}$ belong to the local homology of the
Koszul-Tate differential at antighost number $2m$, $\mu ^{\rho _{1}\cdots
\rho _{m}}\in H_{2m}\left( \delta |d\right) $%
\begin{equation}
\delta \mu ^{\rho _{1}\cdots \rho _{m}}=\partial _{\mu }h^{\prime \mu \rho
_{1}\cdots \rho _{m}},\;\mathrm{agh}\left( \mu ^{\rho _{1}\cdots \rho
_{m}}\right) =2m,\;\mathrm{pgh}\left( \mu ^{\rho _{1}\cdots \rho
_{m}}\right) =0.  \label{fcoli4}
\end{equation}
But, as the model under study is linear and of Cauchy order equal to three,
we have that the local homology of $\delta $ vanishes for antighost numbers
greater that three, $H_{k}\left( \delta |d\right) =0$, $k>3$. On the other
hand, the last representative, $a_{n}$, is constructed from some functions
that pertain to an even-order space $H_{2m}\left( \delta |d\right) $, such
that the first admitted value $2m\leq 3$ is $m=1$ ($n=2$). Then, we can
assume that $\stackrel{(1)}{S}=\sum_{k=0}^{2}\int d^{D}x\,a_{k}$, where $%
a_{2}=\mu ^{\rho }C_{(\rho )}$, with $\mu ^{\rho }\in H_{2}\left( \delta
|d\right) $.

At this stage, we remark that the general representative of $H_{2}\left(
\delta |d\right) $ is of the form $\alpha =\lambda _{\alpha }^{\;(\sigma
)}\eta _{\;\;\;(\sigma )}^{*\alpha }$, for some constants $\lambda _{\alpha
}^{\;(\sigma )}$. Consequently, it follows that $\mu ^{\rho }=\lambda
^{\sigma }\eta _{\;\;\;(\sigma )}^{*\rho }+\tilde{\lambda}_{\alpha }\eta
^{*\alpha (\rho )}$, where $\lambda ^{\sigma }$ and $\tilde{\lambda}_{\alpha
}$ are also constant. On the other hand, the only covariant choice of these
constants is $\lambda ^{\sigma }=k_{1}\partial ^{\sigma }$ and $\tilde{%
\lambda}_{\alpha }=k_{2}\partial _{\alpha }$, which further yields 
\begin{equation}
\mu ^{\rho }=k_{1}\partial ^{\sigma }\eta _{\;\;\;(\sigma )}^{*\rho
}+k_{2}\partial _{\alpha }\eta ^{*\alpha (\rho )},  \label{lala1}
\end{equation}
with $k_{1,2}$ numerical constants. The second term in the right hand-side
of (\ref{lala1}) is $\delta $-exact (see the latter relations in (\ref
{fcol20})). As one can always add a $\delta $-exact term to the solution of
the equation $\delta \mu ^{\rho }=\partial _{\alpha }h^{\prime \alpha \rho }$%
, it results that the second term from the right hand-side of (\ref{lala1})
can be removed. Then, we find that 
\begin{equation}
a_{2}=\partial ^{\sigma }\left( k_{1}\eta _{\;\;\;(\sigma )}^{*\rho
}C_{(\rho )}\right) -\gamma \left( k_{1}\eta _{\;\;\;(\sigma )}^{*\rho }\eta
_{\;(\rho )}^{\sigma }\right) .  \label{lala2}
\end{equation}
Since we are free to add a $\gamma $-exact term and a $\gamma $-invariant
divergence to the last representative, we get that $a_{2}$ can be chosen to
vanish, $a_{2}=0$. Further, $a_{1}$ is not eligible as last representative
in $\stackrel{(1)}{S}$ because it does not display an even pure ghost
number. As a consequence, $\stackrel{(1)}{S}$ simply reduces to the
component that is ghost- and antifield-independent, $\stackrel{(1)}{S}=\int
d^{D}x\,a_{0}$, where $a_{0}$ is a $\gamma $-cocycle modulo $d$, $\gamma
a_{0}=\partial _{\mu }h^{\prime \prime \mu }$. The nontrivial solutions to
this equations are of two kinds. The first one corresponds to $h^{\prime
\prime \mu }=0$ and is given by local functions that are invariant under the
original gauge transformations, which are polynomials in the field strength
components $F_{\alpha \beta \gamma }^{\;\;\;\;(\sigma )}$ and their
spacetime derivatives, while the second kind is associated with $h^{\prime
\prime \mu }\neq 0$ and is spanned by generalized Chern-Simons terms.

In conclusion, the only nontrivial first-order deformation of the solution
to the master equation for a special class of mixed-symmetry type tensor
gauge fields of degree three in spacetime dimensions strictly greater than
four is, according to the classification from the beginning of this section,
of type (i). Beside the Lagrangian action, neither the gauge algebra, nor
the gauge transformations, nor the reducibility relations are modified. The
deformed solution $\bar{S}=\stackrel{(0)}{S}+g\int d^{D}x\,a_{0}$ is already
consistent to all orders in the deformation parameter, so we can take $%
\stackrel{(k)}{S}=0$, $k>1$.

\section{Four-dimensional model in first-order form}

Since there are no deformations of the free solution (\ref{fcol24}) to the
master equation in more that four spacetime dimensions that truly deform the
gauge transformations and eventually their algebra, in the sequel we focus
on the four dimensional case. It is convenient to rewrite the Lagrangian
action (\ref{fcol1}) for $D=4$ in first-order form. For subsequent purpose,
we replace the tensor fields $A_{\alpha \beta }^{\;\;\;(\sigma )}$ by their
duals with respect to the antisymmetry indices, $\frac{1}{2}\varepsilon
^{\alpha \beta \gamma \delta }A_{\gamma \delta (\sigma )}$, which we will
still denote by $A_{\;\;\;(\sigma )}^{\alpha \beta }$, so the Lagrangian
action in first-order form becomes 
\begin{equation}
S_{0}^{\prime }\left[ A_{\;\;\;(\sigma )}^{\alpha \beta },B_{\alpha
}^{\;(\sigma )}\right] =\frac{1}{2}\int d^{4}x\left( -B_{\alpha }^{\;(\sigma
)}B_{\;\;(\sigma )}^{\alpha }+A_{\;\;\;(\sigma )}^{\alpha \beta }H_{\alpha
\beta }^{\;\;\;(\sigma )}\right) ,  \label{fcola1}
\end{equation}
by means of adding an auxiliary tensor field of degree two $B_{\alpha
}^{\;(\sigma )}$ with no symmetry, endowed with the field strength 
\begin{equation}
H_{\alpha \beta }^{\;\;\;(\sigma )}=\partial _{\alpha }B_{\beta }^{\;(\sigma
)}-\partial _{\beta }B_{\alpha }^{\;(\sigma )}\equiv \partial _{\left[
\alpha \right. }B_{\left. \beta \right] }^{\;\;(\sigma )}.  \label{fcola2}
\end{equation}
It is known that auxiliary fields do not change the dynamics, and,
essentially, they do not modify the local cohomological group $H^{0}\left(
s|d\right) $ either \cite{32and12}. The gauge invariances of action (\ref
{fcola1}) are given by 
\begin{equation}
\delta _{\epsilon }A_{\;\;\;(\sigma )}^{\alpha \beta }=\varepsilon ^{\alpha
\beta \gamma \delta }\partial _{\gamma }\epsilon _{\delta (\sigma )}\equiv 
\stackrel{(0)}{R}_{\;\;\;(\sigma )}^{\prime \alpha \beta \;\;\;\delta (\rho
)}\epsilon _{\delta (\rho )},\;\delta _{\epsilon }B_{\alpha }^{\;(\sigma
)}=0,  \label{fcola3}
\end{equation}
with 
\begin{equation}
\stackrel{(0)}{R}_{\;\;\;(\sigma )}^{\prime \alpha \beta \;\;\;\delta (\rho
)}=\delta _{\sigma }^{\rho }\varepsilon ^{\alpha \beta \gamma \delta
}\partial _{\gamma },  \label{fcola4}
\end{equation}
and $\varepsilon ^{\alpha \beta \gamma \delta }$ the completely
antisymmetric symbol in four spacetime dimensions, $\varepsilon ^{0123}=+1$.
They are off-shell first-stage reducible, where the reducibility relations
and functions read as 
\begin{equation}
\stackrel{(0)}{R}_{\;\;\;(\sigma )}^{\prime \alpha \beta \;\;\;\delta (\rho
)}\stackrel{(0)}{Z}_{\delta (\rho )}^{\prime \;\;\;\;(\tau )}=0,
\label{fcola5}
\end{equation}
\begin{equation}
\stackrel{(0)}{Z}_{\delta (\rho )}^{\prime \;\;\;\;(\tau )}=\delta _{\rho
}^{\tau }\partial _{\delta }.  \label{fcola6}
\end{equation}

The generators of the BRST complex and their degrees are listed below 
\begin{equation}
\Phi ^{\Delta }=\left( A_{\;\;\;(\sigma )}^{\alpha \beta },B_{\alpha
}^{\;(\sigma )},\eta _{\alpha (\sigma )},C_{(\sigma )}\right) ,
\label{fcola7}
\end{equation}
\begin{equation}
\Phi _{\Delta }^{*}=\left( A_{\alpha \beta }^{*\;\;(\sigma
)},B_{\;\;\;(\sigma )}^{*\alpha },\eta ^{*\alpha (\sigma )},C^{*(\sigma
)}\right) ,  \label{fcola7a}
\end{equation}
\begin{equation}
\varepsilon \left( A_{\;\;\;(\sigma )}^{\alpha \beta }\right) =\varepsilon
\left( B_{\alpha }^{\;(\sigma )}\right) =\varepsilon \left( C_{(\sigma
)}\right) =0,\;\varepsilon \left( \eta _{\alpha (\sigma )}\right) =1,
\label{fcola8}
\end{equation}
\begin{equation}
\mathrm{agh}\left( \Phi ^{\Delta }\right) =0,\;\mathrm{agh}\left( A_{\alpha
\beta }^{*\;\;(\sigma )}\right) =1=\mathrm{agh}\left( B_{\;\;\;(\sigma
)}^{*\alpha }\right) ,  \label{fcola9}
\end{equation}
\begin{equation}
\mathrm{agh}\left( \eta ^{*\alpha (\sigma )}\right) =2,\;\mathrm{agh}\left(
C^{*(\sigma )}\right) =3,  \label{fcola10}
\end{equation}
\begin{equation}
\mathrm{pgh}\left( \Phi _{\Delta }^{*}\right) =0,\;\mathrm{pgh}\left(
A_{\;\;\;(\sigma )}^{\alpha \beta }\right) =0=\mathrm{pgh}\left( B_{\alpha
}^{\;(\sigma )}\right) ,  \label{fcola11}
\end{equation}
\begin{equation}
\mathrm{pgh}\left( \eta _{\alpha (\sigma )}\right) =1,\;\mathrm{pgh}\left(
C_{(\sigma )}\right) =2.  \label{fcola12}
\end{equation}
In terms of the new variables, the BRST differential $s=\delta +\gamma $ of
the free theory has the form 
\begin{equation}
\delta \Phi ^{\Delta }=0,\;\delta A_{\alpha \beta }^{*\;\;(\sigma )}=-\frac{1%
}{2}H_{\alpha \beta }^{\;\;\;(\sigma )},\;\delta B_{\;\;\;(\sigma
)}^{*\alpha }=B_{\;\;(\sigma )}^{\alpha }+\partial _{\beta }A_{\;\;\;(\sigma
)}^{\beta \alpha },  \label{fcola13}
\end{equation}
\begin{equation}
\delta \eta ^{*\alpha (\sigma )}=\varepsilon ^{\alpha \beta \gamma \delta
}\partial _{\beta }A_{\gamma \delta }^{*\;\;(\sigma )},\;\delta C^{*(\sigma
)}=\partial _{\alpha }\eta ^{*\alpha (\sigma )},  \label{fcola14}
\end{equation}
\begin{equation}
\gamma \Phi _{\Delta }^{*}=0,\;\gamma A_{\;\;\;(\sigma )}^{\alpha \beta
}=\varepsilon ^{\alpha \beta \gamma \delta }\partial _{\gamma }\eta _{\delta
(\sigma )},\;\gamma \left( B_{\alpha }^{\;(\sigma )}\right) =0,
\label{fcola15}
\end{equation}
\begin{equation}
\gamma \eta _{\alpha (\sigma )}=\partial _{\alpha }C_{(\sigma )},\;\gamma
C_{(\sigma )}=0,  \label{fcola16}
\end{equation}
whereas the solution to the classical master equation for the free model
with auxiliary fields is 
\begin{equation}
\stackrel{(0)}{S}=S_{0}^{\prime }\left[ A_{\;\;\;(\sigma )}^{\alpha \beta
},B_{\alpha }^{\;(\sigma )}\right] +\int d^{4}x\left( \varepsilon ^{\alpha
\beta \gamma \delta }A_{\alpha \beta }^{*\;\;(\sigma )}\partial _{\gamma
}\eta _{\delta (\sigma )}+\eta ^{*\alpha (\sigma )}\partial _{\alpha
}C_{(\sigma )}\right) ,  \label{fcola17}
\end{equation}
being understood that $s\cdot =\left( \cdot ,\stackrel{(0)}{S}\right) $.

\section{Lagrangian BRST deformation in four dimensions}

We have seen that the equation (\ref{fcol10}) requires that the first-order
deformation of the solution to the master equation is an $s$-cocycle modulo $%
d$ at ghost number zero 
\begin{equation}
\stackrel{(1)}{S}=\int d^{4}x\,a,\;s\stackrel{(1)}{S}=0\Leftrightarrow
sa=\partial _{\mu }u^{\mu }.  \label{fcola19}
\end{equation}
In four dimensions, on the one hand the local homology of $\delta $ vanishes
again for $k>3$ \cite{20and5} 
\begin{equation}
H_{k}\left( \delta |d\right) =0,\;k>3,  \label{fcola19a}
\end{equation}
and, on the other hand, the cohomology of $\gamma $ is generated by $\left(
\Phi _{\Delta }^{*},\;B_{\alpha }^{\;(\sigma )}\right) $ and their spacetime
derivatives up to some finite orders, together with the undifferentiated
ghosts for ghosts $C_{(\sigma )}$. (We did not include the field strength
components (\ref{fcol2}) or their derivatives in $H\left( \gamma \right) $
because the auxiliary fields and their derivatives take over this role via
the equations of motion for $B_{\alpha }^{\;(\sigma )}$.) As discussed in
the previous section, the cocycle $a$ can be assumed to decompose in a
finite number of terms via the antighost number 
\begin{equation}
a=\sum_{k=0}^{n}a_{k},\;\mathrm{gh}\left( a_{k}\right) =0,\;\mathrm{agh}%
\left( a_{k}\right) =k,  \label{fcola20}
\end{equation}
where the component of highest antighost number belongs to $H\left( \gamma
\right) $ 
\begin{equation}
\gamma a_{n}=0.  \label{fcola21}
\end{equation}
Following a reasoning similar to that from Section 5, we infer that $n=2m$
and 
\begin{equation}
a_{2m}=\mu ^{\rho _{1}\cdots \rho _{m}}\left( \left[ \Phi _{\Delta
}^{*}\right] ,\left[ B_{\alpha }^{\;(\sigma )}\right] \right) C_{(\rho
_{1})}\cdots C_{(\rho _{m})},  \label{fcola23}
\end{equation}
with $\mathrm{agh}\left( \mu ^{\rho _{1}\cdots \rho _{m}}\right) =2m$, $%
\mathrm{pgh}\left( \mu ^{\rho _{1}\cdots \rho _{m}}\right) =0$. In the
meantime, the local form of the equation (\ref{fcola19}) projected on
antighost number $\left( 2m-1\right) $%
\begin{equation}
\delta a_{2m}+\gamma a_{2m-1}=\partial _{\mu }v^{\mu },  \label{fcola22}
\end{equation}
induces that a necessary condition for the existence of $a_{2m-1}$ is that 
\begin{equation}
\mu ^{\rho _{1}\cdots \rho _{m}}\in H_{2m}\left( \delta |d\right) .
\label{fcola24}
\end{equation}
Combining (\ref{fcola19a}) with (\ref{fcola23}), it is legitimate to presume
that the expansion (\ref{fcola20}) stops after the first three terms ($%
n=2m=2 $) 
\begin{equation}
a=a_{0}+a_{1}+a_{2},  \label{fcola25}
\end{equation}
where $a_{2}$ is of the form $\mu ^{\rho }C_{(\rho )}$, with $\mu ^{\rho }$
an element of $H_{2}\left( \delta |d\right) $, whose dependence on the
fields and antifields is like in (\ref{fcola23}). After some computation, we
find that $H_{2}\left( \delta |d\right) $ is two-dimensional, where a
possible choice of its two independent and nontrivial elements is 
\begin{equation}
\mu ^{\prime \rho }=-\left( \partial _{\sigma }\eta ^{*\alpha (\rho
)}\right) B_{\alpha }^{\;(\sigma )}+\eta ^{*\alpha (\sigma )}\partial
_{\sigma }B_{\alpha }^{\;(\rho )}+\varepsilon ^{\alpha \beta \gamma \delta
}A_{\alpha \beta }^{*\;\;(\sigma )}\partial _{\sigma }A_{\gamma \delta
}^{*\;\;(\rho )},  \label{fcola26}
\end{equation}
\begin{equation}
\mu ^{\prime \prime \rho }=-\left( \partial ^{\rho }\eta ^{*\alpha (\sigma
)}\right) B_{\alpha (\sigma )}+\eta ^{*\alpha (\sigma )}\partial ^{\rho
}B_{\alpha (\sigma )}+\varepsilon ^{\alpha \beta \gamma \delta }A_{\alpha
\beta }^{*\;\;(\sigma )}\partial ^{\rho }A_{\gamma \delta (\sigma )}^{*}.
\label{fcola27}
\end{equation}
Indeed, we have that 
\begin{equation}
\delta \mu ^{\prime \rho }=\partial _{\mu }\left( \varepsilon ^{\mu \alpha
\beta \gamma }\left( B_{\gamma }^{\;(\sigma )}\partial _{\sigma }A_{\alpha
\beta }^{*\;\;(\rho )}-A_{\alpha \beta }^{*\;\;(\sigma )}\partial _{\sigma
}B_{\gamma }^{\;(\rho )}\right) \right) ,  \label{fcola26a}
\end{equation}
\begin{equation}
\delta \mu ^{\prime \prime \rho }=\partial _{\mu }\left( \varepsilon ^{\mu
\alpha \beta \gamma }\left( B_{\gamma }^{\;(\sigma )}\partial ^{\rho
}A_{\alpha \beta (\sigma )}^{*}-A_{\alpha \beta }^{*\;\;(\sigma )}\partial
^{\rho }B_{\gamma (\sigma )}\right) \right) .  \label{fcola27a}
\end{equation}
Consequently, the last component in the first-order deformation (\ref
{fcola25}) takes the concrete form 
\begin{eqnarray}
&&a_{2}=\left( -\left( c_{1}\partial _{\sigma }\eta ^{*\alpha (\rho
)}+c_{2}\partial ^{\rho }\eta _{\;\;\;(\sigma )}^{*\alpha }\right) B_{\alpha
}^{\;(\sigma )}+\right.  \nonumber \\
&&\eta ^{*\alpha (\sigma )}\left( c_{1}\partial _{\sigma }B_{\alpha
}^{\;(\rho )}+c_{2}\partial ^{\rho }B_{\alpha (\sigma )}\right) +  \nonumber
\\
&&\left. \varepsilon ^{\alpha \beta \gamma \delta }A_{\alpha \beta
}^{*\;\;(\sigma )}\left( c_{1}\partial _{\sigma }A_{\gamma \delta
}^{*\;\;(\rho )}+c_{2}\partial ^{\rho }A_{\gamma \delta (\sigma
)}^{*}\right) \right) C_{(\rho )},  \label{fcola28}
\end{eqnarray}
with $c_{1}$ and $c_{2}$ two real constants, otherwise arbitrary. From the
action of $\delta $ on $a_{2}$%
\begin{eqnarray}
&&\delta a_{2}=-\gamma \left( \varepsilon ^{\alpha \beta \gamma \delta
}\left( A_{\alpha \beta }^{*\;\;(\sigma )}\left( c_{1}\partial _{\sigma
}B_{\gamma }^{\;(\rho )}+c_{2}\partial ^{\rho }B_{\gamma (\sigma )}\right)
-\right. \right.  \nonumber \\
&&\left. \left. \left( c_{1}\partial _{\sigma }A_{\alpha \beta }^{*\;\;(\rho
)}+c_{2}\partial ^{\rho }A_{\alpha \beta (\sigma )}^{*}\right) B_{\gamma
}^{(\sigma )}\right) \eta _{\delta (\rho )}\right) +  \nonumber \\
&&\partial _{\mu }\left( \varepsilon ^{\mu \alpha \beta \gamma }\left(
B_{\gamma }^{\;(\sigma )}\left( c_{1}\partial _{\sigma }A_{\alpha \beta
}^{*\;\;(\rho )}+c_{2}\partial ^{\rho }A_{\alpha \beta (\sigma )}^{*}\right)
-\right. \right.  \nonumber \\
&&\left. \left. -A_{\alpha \beta }^{*\;\;(\sigma )}\left( c_{1}\partial
_{\sigma }B_{\gamma }^{\;(\rho )}+c_{2}\partial ^{\rho }B_{\gamma (\sigma
)}\right) \right) C_{(\rho )}\right) ,  \label{fcola28a}
\end{eqnarray}
the equation (\ref{fcola22}) for $m=1$ yields $a_{1}$ 
\begin{eqnarray}
&&a_{1}=\varepsilon ^{\alpha \beta \gamma \delta }\left( A_{\alpha \beta
}^{*\;\;(\sigma )}\left( c_{1}\partial _{\sigma }B_{\gamma }^{\;(\rho
)}+c_{2}\partial ^{\rho }B_{\gamma (\sigma )}\right) -\right.  \nonumber \\
&&\left. \left( c_{1}\partial _{\sigma }A_{\alpha \beta }^{*\;\;(\rho
)}+c_{2}\partial ^{\rho }A_{\alpha \beta (\sigma )}^{*}\right) B_{\gamma
}^{(\sigma )}\right) \eta _{\delta (\rho )},  \label{fcola29}
\end{eqnarray}
up to a solution $a_{1}^{\prime }$ of the `homogeneous' equation $\gamma
a_{1}^{\prime }=\partial _{\mu }k^{\mu }$ and, certainly, up to trivial
irrelevant terms. The solutions of the homogeneous equation do modify the
gauge transformations, but not their algebra, since they correspond to a
vanishing $a_{2}$. Using again the arguments from \cite{7and23}, \cite
{23and6}, \cite{32and12}, \cite{36and14}, it can be shown that we can
redefine $a_{1}^{\prime }$ such that $a_{1}^{\prime }$ is a $\gamma $%
-cocycle, $\gamma a_{1}^{\prime }=0$. As $\mathrm{pgh}\left( a_{1}^{\prime
}\right) =1$, it follows that it must be linear in the ghosts $\eta _{\delta
(\rho )}$ and their spacetime derivatives. However, on account of the
relations (\ref{fcola15}--\ref{fcola16}) we observe that neither the ghosts,
nor their derivatives, are nontrivial in $H\left( \gamma \right) $, and
hence we can take $a_{1}^{\prime }=0$.

At this point we are able to determine the term $a_{0}$, which is precisely
the deformed Lagrangian at order one in $g$. The equation for $a_{0}$
follows from the equation (\ref{fcola19}) (with $a$ in the form (\ref
{fcola25})) projected on antighost number zero 
\begin{equation}
\delta a_{1}+\gamma a_{0}=\partial _{\mu }w^{\mu }.  \label{fcola30}
\end{equation}
By means of (\ref{fcola29}), we find that 
\begin{eqnarray}
&&\delta a_{1}=\gamma \left( \left( c_{1}\partial _{\sigma }B_{\alpha
}^{\;(\rho )}+c_{2}\partial ^{\rho }B_{\alpha (\sigma )}\right) B_{\beta
}^{\;(\sigma )}A_{\;\;\;(\rho )}^{\alpha \beta }\right) -  \nonumber \\
&&\partial _{\mu }\left( \varepsilon ^{\mu \alpha \beta \gamma }\left(
c_{1}\partial _{\sigma }B_{\alpha }^{\;(\rho )}+c_{2}\partial ^{\rho
}B_{\alpha (\sigma )}\right) B_{\beta }^{\;(\sigma )}\eta _{\gamma (\rho
)}\right) ,  \label{fcola31}
\end{eqnarray}
which further leads to the expression 
\begin{equation}
a_{0}=-\left( c_{1}\partial _{\sigma }B_{\alpha }^{\;(\rho )}+c_{2}\partial
^{\rho }B_{\alpha (\sigma )}\right) B_{\beta }^{\;(\sigma )}A_{\;\;\;(\rho
)}^{\alpha \beta },  \label{fcola32}
\end{equation}
also up to a solution $a_{0}^{\prime }$ of the `homogeneous' equation $%
\gamma a_{0}^{\prime }=\partial _{\mu }l^{\mu }$ corresponding to a
vanishing $a_{1}$. These solutions only modify the free action (\ref{fcola1}%
) by adding to it terms that are invariant under the gauge transformations (%
\ref{fcola3}) of the free model, $\gamma \left( \int d^{4}x\,a_{0}^{\prime
}\right) =0$. The most general form of such gauge-invariant terms is
expressed by arbitrary polynomials in the auxiliary fields $B_{\alpha
}^{\;(\sigma )}$ and their derivatives. However, these interactions are less
interesting as they merely change the field equations, but do not modify the
gauge symmetry.

Now, the first-order deformation of the solution to the master equation 
\begin{equation}
\stackrel{(1)}{S}=\int d^{4}x\,a=\int d^{4}x\left( a_{0}+a_{1}+a_{2}\right) ,
\label{fcola33}
\end{equation}
where $a_{0,1,2}$ are indicated in (\ref{fcola28}), (\ref{fcola29}) and (\ref
{fcola32}) is by construction an $s$-cocycle of ghost number zero, such that 
$\stackrel{(0)}{S}+g\stackrel{(1)}{S}$ is solution to the master equation up
to order $g$. It is essential to remark that the interactions are not
trivial, and that they truly deform the gauge symmetry and its reducibility,
since the antifields cannot be eliminated from $a$ even by redefinitions.

Next, let us investigate the consistency of the first-order deformation at
order $g^{2}$. In this light, we invoke the equation (\ref{fcol11}). If we
employ the notations $\stackrel{(2)}{S}=\int d^{4}x\,b$ and $\frac{1}{2}%
\left( \stackrel{(1)}{S},\stackrel{(1)}{S}\right) =\int d^{4}x\,\Delta $,
then the equation (\ref{fcol11}) takes the local form 
\begin{equation}
\Delta =-s\,b+\partial _{\mu }f^{\mu }.  \label{fcola36}
\end{equation}
By direct computation, we get that 
\begin{eqnarray}
&&\Delta =\left( c_{1}\right) ^{2}\varepsilon ^{\alpha \beta \gamma \delta
}\partial _{\mu }\left( B_{\alpha }^{\;(\mu )}\left( B_{\beta }^{\;(\sigma
)}\left( \partial _{\sigma }B_{\gamma }^{\;(\rho )}\right) \eta _{\delta
(\rho )}-A_{\beta \gamma }^{*\;\;(\sigma )}\left( \partial _{\sigma
}B_{\delta }^{\;(\rho )}\right) C_{(\rho )}\right) +\right.  \nonumber \\
&&\left. \left( B_{\alpha }^{\;(\mu )}\left( \partial _{\rho }A_{\beta
\gamma }^{*\;\;(\sigma )}\right) B_{\delta }^{\;(\rho )}-A_{\alpha \beta
}^{*\;\;(\mu )}\left( \partial _{\rho }B_{\gamma }^{\;(\sigma )}\right)
B_{\delta }^{\;(\rho )}\right) C_{(\sigma )}\right) +  \nonumber \\
&&c_{2}\varepsilon ^{\alpha \beta \gamma \delta }\left( \partial _{\sigma
}B_{\gamma }^{\;(\rho )}\right) \left( c_{1}A_{\alpha \beta }^{*\;\;(\sigma
)}\left( 2\left( \partial ^{\tau }B_{\delta (\rho )}\right) C_{(\tau
)}+B_{\delta (\rho )}\partial ^{\tau }C_{(\tau )}\right) -\right.  \nonumber
\\
&&\left. c_{2}\left( 2\left( \partial ^{\tau }A_{\alpha \beta
}^{*\;\;(\sigma )}\right) C_{(\tau )}+A_{\alpha \beta }^{*\;\;(\sigma
)}\partial ^{\tau }C_{(\tau )}\right) B_{\delta (\rho )}\right) +  \nonumber
\\
&&c_{1}c_{2}\varepsilon ^{\alpha \beta \gamma \delta }\left( -B_{\delta
(\rho )}\left( \partial ^{\tau }B_{\gamma }^{\;(\rho )}\right) \partial
_{\sigma }\left( A_{\alpha \beta }^{*\;\;(\sigma )}C_{(\tau )}\right)
+\right.  \nonumber \\
&&\left( \left( \partial ^{\rho }A_{\alpha \beta }^{*\;\;(\tau )}\right)
\left( \partial _{\rho }B_{\delta (\sigma )}\right) -\left( \partial ^{\rho
}A_{\alpha \beta (\sigma )}^{*}\right) \left( \partial _{\rho }B_{\delta
}^{\;(\tau )}\right) \right) B_{\gamma }^{\;(\sigma )}C_{(\tau )}+  \nonumber
\\
&&\left( A_{\alpha \beta (\sigma )}^{*}\partial ^{\tau }B_{\gamma
}^{\;(\sigma )}-\left( \partial ^{\tau }A_{\alpha \beta (\sigma
)}^{*}\right) B_{\gamma }^{\;(\sigma )}\right) \partial ^{\rho }\left(
B_{\delta (\rho )}C_{(\tau )}\right) +  \nonumber \\
&&+\left( \partial ^{\rho }A_{\alpha \beta (\sigma )}^{*}\right) \left(
2\left( \partial ^{\tau }B_{\gamma }^{\;(\sigma )}\right) C_{(\tau
)}+B_{\gamma }^{\;(\sigma )}\partial ^{\tau }C_{(\tau )}\right) B_{\delta
(\rho )}-  \nonumber \\
&&\left( 2\left( \partial ^{\tau }A_{\alpha \beta (\sigma )}^{*}\right)
C_{(\tau )}+A_{\alpha \beta (\sigma )}^{*}\partial ^{\tau }C_{(\tau
)}\right) \left( \partial ^{\rho }B_{\gamma }^{\;(\sigma )}\right) B_{\delta
(\rho )}+  \nonumber \\
&&A_{\alpha \beta (\sigma )}^{*}\left( \partial ^{\rho }B_{\gamma
}^{\;(\sigma )}\right) \left( \partial _{\rho }B_{\delta }^{\;(\tau
)}\right) C_{(\tau )}-B_{\gamma (\rho )}\left( \partial ^{\rho }B_{\beta
(\sigma )}\right) \partial ^{\tau }\left( B_{\alpha }^{\;(\sigma )}\eta
_{\delta (\tau )}\right) +  \nonumber \\
&&\left. B_{\alpha }^{\;(\sigma )}\left( \left( \partial _{\sigma }B_{\beta
}^{\;(\rho )}+\partial ^{\rho }B_{\beta (\sigma )}\right) \left( \partial
_{\rho }B_{\gamma }^{\;(\tau )}\right) \eta _{\delta (\tau )}+\left(
\partial ^{\tau }B_{\beta (\sigma )}\right) \partial ^{\rho }\left(
B_{\gamma (\rho )}\eta _{\delta (\tau )}\right) \right) \right) +  \nonumber
\\
&&\left( c_{2}\right) ^{2}\varepsilon ^{\alpha \beta \gamma \delta }\left(
B_{\alpha }^{\;(\sigma )}\left( \partial ^{\rho }B_{\beta (\sigma )}\right)
\left( 2\left( \partial ^{\tau }B_{\gamma (\rho )}\right) \eta _{\delta
(\tau )}+B_{\gamma (\rho )}\partial ^{\tau }\eta _{\delta (\tau )}\right)
+\right.  \nonumber \\
&&\left. \left( A_{\alpha \beta (\sigma )}^{*}\partial ^{\rho }B_{\gamma
}^{\;(\sigma )}-\left( \partial ^{\rho }A_{\alpha \beta (\sigma
)}^{*}\right) B_{\gamma }^{\;(\sigma )}\right) \left( 2\left( \partial
^{\tau }B_{\delta (\rho )}\right) C_{(\tau )}+B_{\delta (\rho )}\partial
^{\tau }C_{(\tau )}\right) \right) .  \label{fcola35}
\end{eqnarray}
By inspecting the right hand-side of (\ref{fcola35}), we observe that all
the terms proportional with $\left( c_{1}\right) ^{2}$ simply reduce to a
four-dimensional divergence, while those with $\left( c_{2}\right) ^{2}$ and 
$c_{1}c_{2}$ cannot be written as in (\ref{fcola36}). Then, the consistency
of $\stackrel{(1)}{S}$ requires that 
\begin{equation}
c_{2}=0.  \label{fcola34}
\end{equation}
Under these considerations, the equation (\ref{fcol11}) is satisfied for $%
\stackrel{(2)}{S}=0$. The remaining equations involved with the higher-order
deformations hold if we set $\stackrel{(k)}{S}=0,\;k>2$. In addition, the
constant $c_{1}$ takes any nonzero real value, and, for definiteness, will
be fixed to unity 
\begin{equation}
c_{1}=1.  \label{fcola37}
\end{equation}

\section{Identification of the interacting theory}

Now, we are in the position to identify the interacting theory. The complete
deformed solution to the master equation for the model under study, which is
consistent to all orders in the deformation parameter, reads as 
\begin{eqnarray}
&&\bar{S}=\int d^{4}x\left( \frac{1}{2}\left( -B_{\alpha }^{\;(\sigma
)}B_{\;\;(\sigma )}^{\alpha }+A_{\;\;\;(\rho )}^{\alpha \beta }\bar{H}%
_{\alpha \beta }^{\;\;\;(\rho )}\right) +\right.  \nonumber \\
&&\varepsilon ^{\alpha \beta \gamma \delta }\left( A_{\alpha \beta
}^{*\;\;(\sigma )}\left( D_{\gamma }\right) _{\;\;\sigma }^{\rho }-g\partial
_{\sigma }A_{\alpha \beta }^{*\;\;(\rho )}B_{\gamma }^{(\sigma )}\right)
\eta _{\delta (\rho )}+  \nonumber \\
&&\left( \eta ^{*\alpha (\sigma )}\left( D_{\alpha }\right) _{\;\;\sigma
}^{\rho }-g\left( \partial _{\sigma }\eta ^{*\alpha (\rho )}\right)
B_{\alpha }^{\;(\sigma )}\right) C_{(\rho )}-  \nonumber \\
&&\left. g\varepsilon ^{\alpha \beta \gamma \delta }A_{\alpha \beta
}^{*\;\;(\sigma )}\left( \partial _{\sigma }A_{\gamma \delta }^{*\;\;(\rho
)}\right) C_{(\rho )}\right) ,  \label{fcola42}
\end{eqnarray}
where the field strength components of $B_{\alpha }^{\;(\rho )}$ are
deformed like 
\begin{equation}
\bar{H}_{\alpha \beta }^{\;\;\;(\rho )}=H_{\alpha \beta }^{\;\;\;(\rho
)}-g\left( \partial _{\sigma }B_{\left[ \alpha \right. }^{\;(\rho )}\right)
B_{\left. \beta \right] }^{\;(\sigma )},  \label{fcola43}
\end{equation}
and the `covariant derivative' is defined via 
\begin{equation}
\left( D_{\gamma }\right) _{\;\;\sigma }^{\rho }=\delta _{\sigma }^{\rho
}\partial _{\gamma }+g\partial _{\sigma }B_{\gamma }^{\;(\rho )}.
\label{fcola44}
\end{equation}

The deformed solution (\ref{fcola42}) contains all the information on the
gauge structure of the resulting interacting theory. More precisely, the
terms of antighost number zero induce the Lagrangian action of the coupled
model 
\begin{equation}
\bar{S}_{0}\left[ A_{\;\;\;(\sigma )}^{\alpha \beta },B_{\alpha }^{\;(\sigma
)}\right] =\frac{1}{2}\int d^{4}x\left( -B_{\alpha }^{\;(\sigma
)}B_{\;\;(\sigma )}^{\alpha }+A_{\;\;\;(\rho )}^{\alpha \beta }\bar{H}%
_{\alpha \beta }^{\;\;\;(\rho )}\right) ,  \label{fcola45}
\end{equation}
while the pieces of antighost number one furnish its gauge transformations 
\begin{eqnarray}
&&\bar{\delta}_{\epsilon }A_{\;\;\;(\sigma )}^{\alpha \beta }=\varepsilon
^{\alpha \beta \gamma \delta }\left( \left( D_{\gamma }\right) _{\;\;\sigma
}^{\rho }+g\delta _{\sigma }^{\rho }\left( \partial _{\tau }B_{\gamma
}^{\;(\tau )}+B_{\gamma }^{\;(\tau )}\partial _{\tau }\right) \right)
\epsilon _{\delta (\rho )}\equiv  \nonumber \\
&&\bar{R}_{\;\;\;(\sigma )}^{\alpha \beta \;\;\;\delta (\rho )}\epsilon
_{\delta (\rho )},\;\bar{\delta}_{\epsilon }B_{\alpha }^{\;(\sigma )}=0,
\label{fcola46}
\end{eqnarray}
where the new gauge generators of the tensor fields of degree three are
expressed by 
\begin{equation}
\bar{R}_{\;\;\;(\sigma )}^{\alpha \beta \;\;\;\delta (\rho )}=\varepsilon
^{\alpha \beta \gamma \delta }\left( \left( D_{\gamma }\right) _{\;\;\sigma
}^{\rho }+g\delta _{\sigma }^{\rho }\left( \partial _{\tau }B_{\gamma
}^{\;(\tau )}+B_{\gamma }^{\;(\tau )}\partial _{\tau }\right) \right) .
\label{fcola46a}
\end{equation}
The elements with antighost number two that are simultaneously linear in the
ghosts for ghosts and in the antifields of the ghosts determine the
first-order reducibility functions 
\begin{equation}
\bar{Z}_{\alpha (\sigma )}^{\;\;\;\;(\rho )}=\left( D_{\alpha }\right)
_{\;\;\sigma }^{\rho }+g\delta _{\sigma }^{\rho }\left( \partial _{\tau
}B_{\alpha }^{\;(\tau )}+B_{\alpha }^{\;(\tau )}\partial _{\tau }\right) .
\label{fcola47}
\end{equation}
The appearance of the terms quadratic in the antifields of the fields $%
A_{\;\;\;(\sigma )}^{\alpha \beta }$ and linear in the ghosts for ghosts
signifies that the first-order reducibility relations 
\begin{equation}
\bar{R}_{\;\;\;(\sigma )}^{\alpha \beta \;\;\;\delta (\rho )}\bar{Z}_{\delta
(\rho )}^{\;\;\;\;(\tau )}\approx 0,  \label{fcola48}
\end{equation}
hold only on-shell (i.e., on the stationary surface of the field equations
deriving from action (\ref{fcola45})), in contrast to the free model, for
which the reducibility takes place off-shell. The absence of the antifields $%
B_{\;\;\;(\sigma )}^{*\alpha }$ emphasises that the deformation procedure
does not endow the auxiliary fields with gauge invariances. Meanwhile, the
absence of pieces with antighost number two that are both quadratic in the
ghosts $\eta _{\delta (\rho )}$ and linear in their antifields enables us to
state that the gauge algebra of the interacting model remains abelian. Of
course, we can always add to action (\ref{fcola45}) any polynomial in the
auxiliary fields, without further deforming the gauge symmetry.

The added interactions do not spoil either the locality, or the manifest
Lorentz covariance, and, essentially, are nontrivial as the terms involving
antifields cannot be removed from the deformed solution to the maser
equation by adding to it trivial ($s$-exact modulo $d$) terms.

\section{Conclusion}

To conclude with, in this paper we have investigated the consistent
Lagrangian interactions for a special class of covariant reducible
mixed-symmetry type tensor gauge fields of degree three. In spacetime
dimensions strictly greater than four the couplings do not modify the gauge
symmetry of the initial free model, and are merely given by strictly gauge
invariant quantities or generalized Chern-Simons terms. A privileged
situation is encountered in four spacetime dimensions, where there appear
nontrivial consistent interactions that truly deform the gauge symmetry and
the behaviour of the reducibility relations, but not the gauge algebra. In
this sense, both situations reveal the rigidity of the original abelian
gauge algebra against the deformation procedure.

The analysis developed in this paper can be useful at the study of
introducing general interactions among covariant mixed-symmetry type tensor
gauge fields, such as those involved with integer higher spin gauge theories
(in direct or dual formulations).

\section*{Acknowledgment}

This work has been supported by MEC-CNCSIS-Romania (type-A grant 944/2002).

\begin{thebibliography}{99}
\bibitem{1and2}  A. Barkallil, G. Barnich and C. Schomblond, \textit{Results
on the Wess-Zumino consistency condition for arbitrary Lie algebras}
[math-ph/0205047]

\bibitem{2and2}  G. Barnich, \textit{Refining the anomaly consistency
condition, Phys. Rev.} \textbf{D62} (2000) 045007 [hep-th/0003135]

\bibitem{3and2}  G. Barnich, F. Brandt and M. Henneaux, \textit{Conserved
currents and gauge invariance in Yang-Mills theory, Phys. Lett.} \textbf{B346%
} (1995) 81-86 [hep-th/9411202]

\bibitem{4and2}  G. Barnich, F. Brandt and M. Henneaux, \textit{General
solution of the Wess-Zumino consistency condition for Einstein gravity,
Phys. Rev.}\textbf{\ D51} (1995) 1435-1439 [hep-th/9409104]

\bibitem{5and2}  M. Dubois-Violette, M. Henneaux, M. Talon and C. M.
Viallet, \textit{General Solution of the Consistency Equation, Phys. Lett.} 
\textbf{B289} (1992) 361-367 [hep-th/9206106]

\bibitem{6and23}  G. Barnich, \textit{A note on the BRST cohomology of the
extended antifield formalism, Proceedings of the ``Spring School in QFT,
Supersymmetries and Superstrings'' (Calimanesti, Romania, 24-30 April 1998),
Phys. Ann. Univ. of Craiova} \textbf{9} (1999) 92-106 [hep-th/9912247]

\bibitem{7and23}  G. Barnich, F. Brandt and M. Henneaux, \textit{Local BRST
cohomology in gauge theories, Phys. Rept.} \textbf{338} (2000) 439-569
[hep-th/0002245]

\bibitem{8and23}  G. Barnich, \textit{Higher order cohomological
restrictions on anomalies and counterterms, Phys. Lett.} \textbf{B419}
(1998) 211-216 [hep-th/9710162]

\bibitem{9and23}  M. Henneaux, \textit{Anomalies and Renormalization of BFYM
Theory, Phys. Lett.} \textbf{B406} (1997) 66-69 [hep-th/9704023]

\bibitem{10and23}  G. Barnich and M. Henneaux, \textit{Renormalization of
gauge invariant operators and anomalies in Yang-Mills theory, Phys. Rev.
Lett.} \textbf{72} (1994) 1588-1591 [hep-th/9312206]

\bibitem{11and3}  G. Barnich, \textit{A general non renormalization theorem
in the extended antifield formalism, JHEP} \textbf{9903} (1999) 010
[hep-th/9805030]

\bibitem{12and3}  M. Henneaux, \textit{Remarks on the renormalization of
gauge invariant operators in Yang-Mills theory, Phys. Lett.} \textbf{B313}
(1993) 35-40 [hep-th/9306101]; Erratum-ibid. B316 (1993) 633

\bibitem{13and4}  G. Barnich and F. Brandt, \textit{Covariant theory of
asymptotic symmetries, conservation laws and central charges, Nucl. Phys.} 
\textbf{B633} (2002) 3-82 [hep-th/0111246]

\bibitem{14and4}  G. Barnich, \textit{Classical and quantum aspects of the
extended antifield formalism, These d'agregation, Universite Libre de
Bruxelles} (June 2000) [hep-th/0011120]

\bibitem{15and4}  F. Brandt, M. Henneaux and A. Wilch, \textit{Extended
antifield-formalism, Nucl. Phys.} \textbf{B510} (1998) 640-656
[hep-th/9705007]

\bibitem{16and4}  F. Brandt, M. Henneaux and A. Wilch, \textit{Global
Symmetries in the Antifield-Formalism, Phys. Lett.} \textbf{B387} (1996)
320-326 [hep-th/9606172]

\bibitem{17and5}  M. Henneaux, \textit{Consistent Interactions Between Gauge
Fields: The Cohomological Approach, Contemp. Math.} \textbf{219} (1998) 93
[hep-th/9712226]

\bibitem{18and5}  M. Henneaux and B. Knaepen, \textit{All consistent
interactions for exterior form gauge fields, Phys. Rev.} \textbf{D56} (1997)
6076-6080 [hep-th/9706119]

\bibitem{19and5}  M. Henneaux, B. Knaepen and C. Schomblond, \textit{%
Characteristic cohomology of p-form gauge theories, Commun. Math. Phys.} 
\textbf{186} (1997) 137-165, [hep-th/9606181]

\bibitem{20and5}  M. Henneaux, \textit{Uniqueness of the Freedman-Townsend
Interaction Vertex For Two-Form Gauge Fields, Phys. Lett.} \textbf{B368}
(1996) 83-88 [hep-th/9511145]

\bibitem{21and5}  G. Barnich and M. Henneaux, \textit{Consistent couplings
between fields with a gauge freedom and deformations of the master equation,
Phys. Lett.} \textbf{B311} (1993) 123-129 [hep-th/9304057]

\bibitem{22and6}  N. Boulanger, \textit{Multi-graviton theories : yes-go and
no-go results, Fortsch. Phys.} \textbf{50} (2002) 858-863 [hep-th/0111216]

\bibitem{23and6}  N. Boulanger, T. Damour, L. Gualtieri and M. Henneaux, 
\textit{Inconsistency of interacting, multi-graviton theories, Nucl. Phys.} 
\textbf{B597} (2001) 127-171 [hep-th/0007220]

\bibitem{24and6}  N. Boulanger, T. Damour, L. Gualtieri and M. Henneaux, 
\textit{No consistent cross-interactions for a collection of massless spin-2
fields, Proceedings of the ``Spring School in QFT and Hamiltonian Systems''
(Calimanesti, Romania, 2-7 May 2000), Phys. Ann. Univ. of Craiova} \textbf{10%
} (2000) 94-106 [hep-th/0009109]

\bibitem{25and7}  G. Barnich, F. Brandt and M. Grigoriev, \textit{%
Seiberg-Witten maps and noncommutative Yang-Mills theories for arbitrary
gauge groups, JHEP} \textbf{0208} (2002) 023 [hep-th/0206003]

\bibitem{26and7}  G. Barnich, F. Brandt and M. Grigoriev, \textit{%
Seiberg-Witten maps in the context of the antifield formalism, Fortsch. Phys.%
} \textbf{50} (2002) 825-830 [hep-th/0201139]

\bibitem{27and7}  G. Barnich, M. Grigoriev and M. Henneaux, \textit{%
Seiberg-Witten maps from the point of view of consistent deformations of
gauge theories, JHEP} \textbf{0110} (2001) 004 [hep-th/0106188]

\bibitem{28and8}  M. Dubois-Violette and M. Heneaux, \textit{Tensor fields
of mixed Young symmetry type and N-complexes, Commun. Math. Phys.} \textbf{%
226} (2002) 393-418 [math.QA/0110088]

\bibitem{29and810}  M. Dubois-Violette and M. Heneaux, \textit{Generalized
cohomology for irreducible tensor fields of mixed Young symmetry type, Lett.
Math. Phys.} \textbf{49} (1999) 245-252 [math.QA/9907135]

\bibitem{30and9}  X. Bekaert and N. Boulanger, \textit{Tensor gauge fields
in arbitrary representations of GL(D,R): duality \& Poincare lemma}
[hep-th/0208058]

\bibitem{31and9}  P. de Medeiros and C. Hull, \textit{Exotic tensor gauge
theory and duality} [hep-th/0208155]

\bibitem{31and11}  X. Bekaert N. Boulanger and M. Henneaux, \textit{%
Consistent deformations of dual formulations of linearized gravity : a no-go
result} [hep-th/0210278]

\bibitem{string1}  E. Cremmer and J. Scherk, \textit{Spontaneous dynamical
breaking of gauge symmetry in dual models, Nucl. Phys.} \textbf{B72} (1974)
117-124

\bibitem{string2}  M. Kalb and P. Ramond, \textit{Classical direct
interstring action, Phys. Rev.} \textbf{D9} (1974) 2273-2284

\bibitem{string3}  C. Teitelboim, \textit{Monopoles of higher rank, Phys.
Lett.} \textbf{B167} (1986) 69

\bibitem{string4}  C. Teitelboim, \textit{Gauge invariance for extended
objects, Phys. Lett.} \textbf{B167} (1986) 63

\bibitem{string5}  A. Lahiri, \textit{Constrained dynamics of decoupled
abelian two-form, Mod. Phys. Lett.} \textbf{A8} (1993) 2403-2412
[hep-th/9302046]

\bibitem{string6}  M. B. Green, J. H. Schwarz and E. Witten, \textit{%
Superstring theory, Cambridge University Press, Cambridge} (1987)

\bibitem{string7}  E. Witten, \textit{Noncommutative geometry and string
field theory, Nucl. Phys.} \textbf{B268} (1986) 253

\bibitem{string8}  D. Z. Freedman and P. K. Townsend, \textit{Antisymmetric
tensor gauge theories and non-linear }$\sigma $\textit{-models, Nucl. Phys.} 
\textbf{B177} (1981) 282-296

\bibitem{32and12}  G. Barnich, F. Brandt and M. Henneaux, \textit{Local BRST
cohomology in the antifield formalism: I. General theorems, Commun. Math.
Phys.} \textbf{174} (1995) 57-92 [hep-th/9405109]

\bibitem{rusi1}  I. A. Batalin and G. A. Vilkovisky, \textit{Gauge algebra
and quantization, Phys. Lett.} \textbf{B102} (1981) 27

\bibitem{rusi2}  I. A. Batalin and G. A. Vilkovisky, \textit{Feynman rules
for reducible gauge theories, Phys. Lett.} \textbf{B120} (1983) 166

\bibitem{rusi3}  I. A. Batalin and G. A. Vilkovisky, \textit{Quantization of
gauge theories with linearly dependent generators, Phys. Rev.} \textbf{D28}
(1983) 2567; Erratum-ibid. \textbf{D30} (1984) 508

\bibitem{rusi4}  I. A. Batalin and G. A. Vilkovisky, \textit{Closure of the
gauge algebra, generalized Lie algebra equations and Feynman rules, Nucl.
Phys.} \textbf{B134} (1984) 106

\bibitem{rusi5}  I. A. Batalin and G. A. Vilkovisky, \textit{Existence
theorem for gauge algebra, J. Math. Phys.} \textbf{26} (1985) 172

\bibitem{33and13}  M. Henneaux, \textit{Lectures on the antifield-BRST
formalism for gauge theories, Nucl. Phys.} \textbf{B} \textit{(Proc. Suppl.)}
\textbf{18A} (1990) 47-106

\bibitem{34and13}  M. Henneaux and C. Teitelboim, \textit{Quantization of
Gauge Systems, Princeton University Press, Princeton, New Jersey} (1992)

\bibitem{35and13}  J. Gomis, J. Paris and S. Samuel, \textit{Antibracket,
Antifields and Gauge-Theory Quantization, Phys.Rept.} \textbf{259} (1995)
1-145 [hep-th/9412228]

\bibitem{36and14}  G. Barnich, F. Brandt and M. Henneaux, \textit{Local BRST
cohomology in the antifield formalism: II. Application to Yang-Mills theory,
Commun. Math. Phys.} \textbf{174} (1995) 93-116 [hep-th/9405194]
\end{thebibliography}

\end{document}

