
\documentclass[a4paper,12pt]{article}
\usepackage{amssymb}
\usepackage{amsmath}
\usepackage{useful_macros}
\begin{document}
\setlength{\textheight}{9.0in}
\setlength{\topmargin}{-0.6in}
\setlength{\oddsidemargin}{0.15in}
\setlength{\evensidemargin}{0.4in}
\renewcommand{\thefootnote}{\fnsymbol{footnote}}
\providecommand{\e}{{\bf E}}
\providecommand{\m}{{\bf B}}
\providecommand{\h}{{\bf H}}
\renewcommand{\d}{{\bf D}}

\vspace{1cm}

\begin{center}
{\Large {\bf Shrinking D-Strings Expanding onto a Hypertube}}
\vspace{1truecm}

{\large S. Tamaryan}\\
\smallskip
{\it  Theory Department, Yerevan Physics Institute,\\ 
Alikhanian Br. St.-2,\\
Yerevan, 375036, Armenia.}\\
\smallskip
sayat@moon.yerphi.am\\
\vspace{1truecm}
{\large D. H. Tchrakian}\\
\smallskip
{\it Department of Mathematical Physics,\\
National University of Ireland Maynooth,\\
Maynooth, Ireland.}\\
\smallskip
tigran@thphys.may.ie

\end{center}

\vspace{1cm}

{\centerline {\bf Abstract}}

\vspace{0.4cm}

A solution to the BI action describing an expanded D3-brane is considered. The configuration is not supersymmetric but is supported against collapse by the Poynting vector. In the uniform magnetic field limit, the brane magnetic charge grows without limit while the energy and the radius of the brane remain finite. In this limit the brane consists of a large number of small D-strings forming a hypertube and kept in equilibrium by the electric flux along the hypertube.


\newpage
\section{Introduction}


A new brane stabilization mechanism was suggested in \cite{tub}, where brane collapse is prevented by the Poynting vector of the BI electromagnetic field. These D2-branes have a cylindrical shape and preserve 1/4 of the supersymmetry, whence they were called supertubes. The observation that the Hamiltonian density has a lower bound played an important role. When this bound is reached, the D2-brane becomes tensionless and carries no D2-brane charge. As a result the energy of the brane becomes the sum of the energies of the electric and magnetic fields, which after charge quantisation can be interpreted as dissolved Fundemental strings and D0-branes, respectively. This is an expanded F-string-D0-brane system, supported against collapse by the centrifugal force. The expansion of strings and D0-branes into a D2-brane in the presence of a nontrivial RR field was originally discovered in \cite{tunl,my}. It is also observed that in models describing D-branes on group mainfolds, a similar decondensation of D0-branes will take place in the presence of a NS field \cite{bach,tay,my2}.

The supertube model was generalised and developed in several directions. Supergravity solutions with the same properties as the supertube were obtained in \cite{emt}, where in addition, multi-tube solutions representing a number of parrallel supertubes with arbitrary locations and radii were presented. This is possible only when attractive and repulsive forces balance each other precisely such that the no-force condition, typical of BPS states, is satisfied. 

Another interesting development was achieved in \cite{bak1}, where cylindrical D2-branes with arbitrary cross-section were considered. It turned out that the D2-brane with the topology of \myHighlight{$\mathbb{R}\times{\cal C}$}\coordHE{}, \myHighlight{${\cal C}$}\coordHE{} being an arbitrary curve, preserves the same 1/4 of supersymmetry provided that the electric field has unit magnitude and the magnetic field does not change sign on the entire curve.

The existence of the supertube and the BIon spike solutions in curved target space is investigated in \cite{cts}. The criterion is explicitly derived, and it requires that the time-time component of the metric tensor be equal to that of the flat target space. This result naturally connects with the fact that in supersymmetric states, the relevant physical quntities must be time independent if any part of the supersymmetry is to survive.

The rotating tubular D2-brane as a time dependent supersymmetric solution is realised in \cite{rot}, where the Poynting angular momentum is replaced by the mechanical angular momentum without disturbing the 8 supersymmetries.
 
The generalisation of the supersymmetric brane model for D3-branes was investigated in \cite{dual, bak2, hel}. While supersymmetric D3-branes with \myHighlight{$\mathbb{R}^2\times\mathbb{S}^1$}\coordHE{} topology were constructed explicitly, the supersymmetry or stability of the \myHighlight{$\mathbb{R}^1\times\mathbb{S}^2$}\coordHE{} type branes are not yet completely clarified. Such D3-branes cannot be obtained from the tubular D2-branes via T-duality, and their explicit construction is also problematic\cite{hel, flux}. In this work we consider the problem of a tubular D3-brane with spherical basesurface, in an interseting example, and show that it is not supersymmetric. However, it is stable and is supported against collapse by the Poynting vector despite the loss of supersymmetry. We did not find any supersymmetric realisations, but, were unable to prove in general that this type brane has no supersymmetry.

D3-brane models have also been considered in a different context. The D3-brane wrapped on a 3-sphere with both NS and RR background fluxes, and, instantonic D3-branes carrying \myHighlight{$(p,q)$}\coordHE{} string chrages, were investigated in \cite{evs2,evs}.

In the D3-brane constructed here, the absence of supersymmetry does not prevent the brane from being stable. The mechanism in force leading to stability, is the exact cancellation of the attractive force of tension by the repulsive centrifugal force. This is due to the saturation of a BPS bound even in the absence of supersymmetry. Similar situations hold in other examples, two of  which we note: In one, configurations with topology \myHighlight{$\mathbb{S}^1\times\mathbb{S}^1\times\mathbb{R}$}\coordHE{} (with the orthogonal circles both supported by separate angular momenta) were considered in \cite{per}, which were shown to be absolutely stable under flux preserving small variations. In the other, the dynamics of RR radiating rotating ellipsoidal D2-branes were studied in \cite{har}.

In section {\bf 2}, following the procedure of \cite{tub, emt, bak1, cts, hel, surp}, we have found a solution which saturates the BPS bound on the BI action.  Employing this, we have constructed the hypertube and calculated its energy and radius as functions of the brane charges. The relation between the radius and the charges then expresses the impossibility of collapse.

In section {\bf 3} we present the interesting case when the magnetic field tends to a uniform configuration. The limit of uniform magnetic field on the sphere itself is not physical, but can be approached with the desired accuracy. In this process, the D-string charge grows without limit while the energy and the radius of the brane remain finite. This happens because the length of the dissolved D-strings shrinks in inverse proportion to the magnetic charge while the energy and radius of the brane get their contributions only from the products of their charges and lengths. Finally we present our conclusions in section {\bf 4}.


\section{The hypertube}

\noindent  

We consider a tubular D3-brane with \myHighlight{$\mathbb{R}^1\times\mathbb{S}^2$}\coordHE{} spatial toplogy in flat Minkowski spacetime. The induced metric with constant radius \myHighlight{$r$}\coordHE{} of the sphere then becomes
\begin{equation}\coord{}\boxEquation{
ds^2=-dt^2+dz^2+r^2d{\theta}^2+r^2{\sin}^2\theta\,d{\varphi}^2
\label{bi1}
}{
ds^2=-dt^2+dz^2+r^2d{\theta}^2+r^2{\sin}^2\theta\,d{\varphi}^2
}{ecuacion}\coordE{}\end{equation}
where the variables \myHighlight{$(t,z,\theta ,\varphi )$}\coordHE{} parametrise the brane worldvolume.


The brane action reduces to the BI action
\begin{equation}\coord{}\boxEquation{
I=T_3\int {\cal L}\,\sqrt{-g}\,dtdzd\theta d\varphi,
\label{bi2}
}{
I=T_3\int {\cal L}\,\sqrt{-g}\,dtdzd\theta d\varphi,
}{ecuacion}\coordE{}\end{equation}
where the Lagrangian density \myHighlight{${\cal L}$}\coordHE{} is
\begin{equation}\coord{}\boxEquation{
{\cal L}=-\sqrt{1+{\m}^2-{\e}^2-(\e \cdot \m)^2}
\label{bi3}
}{
{\cal L}=-\sqrt{1+{\m}^2-{\e}^2-(\e \cdot \m)^2}
}{ecuacion}\coordE{}\end{equation}
and \myHighlight{$\sqrt{-g}=r^2\sin\theta$}\coordHE{} is the square root of the induced metric determinant. The \myHighlight{$2\pi$}\coordHE{} coefficients are absorbed in the normalization of the electric intensity \myHighlight{$\e$}\coordHE{} and magnetic induction \myHighlight{$\m$}\coordHE{}
\begin{equation}\coord{}\boxEquation{
E_i=2\pi F_{0i},\qquad 
B_i=2\pi\,\frac{1}{2}\sqrt{-g}\,{\varepsilon}_{ijk}F^{jk} 
\label{bi4}
}{
E_i=2\pi F_{0i},\qquad 
B_i=2\pi\,\frac{1}{2}\sqrt{-g}\,{\varepsilon}_{ijk}F^{jk} 
}{ecuacion}\coordE{}\end{equation}
with the electric displacement \myHighlight{$\d$}\coordHE{} and the magnetic induction \myHighlight{$\h$}\coordHE{} 
\begin{equation}\coord{}\boxEquation{
\d=\frac{\partial {\cal L}}{\partial \e}=\frac{\e+\m (\e\cdot\m)}{\sqrt{1+
{\m}^2-{\e}^2-(\e\cdot \m)^2}}\, ,
\label{bi5}
}{
\d=\frac{\partial {\cal L}}{\partial \e}=\frac{\e+\m (\e\cdot\m)}{\sqrt{1+
{\m}^2-{\e}^2-(\e\cdot \m)^2}}\, ,
}{ecuacion}\coordE{}\end{equation}
\begin{equation}\coord{}\boxEquation{
\h =-\frac{\partial {\cal L}}{\partial \m}=\frac{\m -\e (\e\cdot\m)}{\sqrt{1+\m^2-\e^2-(\e\cdot\m)^2}}\ .
\label{s1}
}{
\h =-\frac{\partial {\cal L}}{\partial \m}=\frac{\m -\e (\e\cdot\m)}{\sqrt{1+\m^2-\e^2-(\e\cdot\m)^2}}\ .
}{ecuacion}\coordE{}\end{equation}
Performing the Legendre transform of \myHighlight{${\cal L}$}\coordHE{} we construct the Hamiltonian density \myHighlight{${\cal H}$}\coordHE{} 
\begin{equation}\coord{}\boxEquation{
{\cal H}=\sqrt{1+{\m}^2+{\d}^2+|\m\times\d|^2}.
\label{bi6}
}{
{\cal H}=\sqrt{1+{\m}^2+{\d}^2+|\m\times\d|^2}.
}{ecuacion}\coordE{}\end{equation}
When the electric and magnetic fields are perpendicular (see the nonperpendicular case in \cite{bak2}) the Hamiltonian density \myHighlight{${\cal H}$}\coordHE{} can be rewritten as
\begin{equation}\coord{}\boxEquation{
{\cal H}=\sqrt{(|\m |+|\d |)^2+(1-|\m ||\d |)^2}\ge |\m |+|\d |
\label{bi7}
}{
{\cal H}=\sqrt{(|\m |+|\d |)^2+(1-|\m ||\d |)^2}\ge |\m |+|\d |
}{ecuacion}\coordE{}\end{equation}
which clearly exhibits the BPS limit
\begin{equation}\coord{}\boxEquation{
{\cal H}_{min}= |\m |+|\d |\, .
\label{bi8}
}{
{\cal H}_{min}= |\m |+|\d |\, .
}{ecuacion}\coordE{}\end{equation}
The BPS saturation condition is
\begin{equation}\coord{}\boxEquation{
|\m ||\d |=1.
\label{bi9}
}{
|\m ||\d |=1.
}{ecuacion}\coordE{}\end{equation}
The physical meaning of eq.(\ref{bi8})  is that in the BPS limit the brane becomes tensionless and its energy comes entirely  from the electromagnetic field. After charge quantisation, the electric and magnetic fields can be interpreted as dissolved  F-strings and D-strings, respectively.

The BPS condition (\ref{bi9}) for mutually perpendicular electric and magnetic fields reduces to \myHighlight{$\e^2 =1$}\coordHE{}, and admits a family of static solutions 
\begin{equation}\coord{}\boxEquation{
E_z=H_{\varphi} =1,\quad D_z=D\sin^{1-\gamma}\theta ,\quad 
B_{\varphi}=B\sin^{\gamma}\theta \ ,
\label{sol}
}{
E_z=H_{\varphi} =1,\quad D_z=D\sin^{1-\gamma}\theta ,\quad 
B_{\varphi}=B\sin^{\gamma}\theta \ ,
}{ecuacion}\coordE{}\end{equation}
where \myHighlight{$B$}\coordHE{} and \myHighlight{$D$}\coordHE{} are constants, related through (\ref{bi9}), by
\begin{equation}\coord{}\boxEquation{
BD=r.
\label{bps}
}{
BD=r.
}{ecuacion}\coordE{}\end{equation}
In (\ref{sol}), one can replace \myHighlight{$\sin^{\gamma}\theta$}\coordHE{} by any function \myHighlight{$f(\theta)$}\coordHE{} which does not change sign. We have made this choice for the sake of simplicity. This solution has two essential properties which stabilisise the brane: the electric field has unit magnitude and the magnetic field does not change sign. These two conditions are also sufficient for preserving 1/4 of the supersymmetry of the cylindrical D2-brane with arbitrary cross-section\cite{bak1, surp}. However these features do not ensure the supersymmetry of the D3-brane with the spherical basesurface. What is more is that the brane with this background electromagnetic field preserves no part of the supersymmetry. This can be checked directly considering the equation 
\begin{equation}\coord{}\boxEquation{
\Gamma\epsilon =\epsilon,
\label{k}
}{
\Gamma\epsilon =\epsilon,
}{ecuacion}\coordE{}\end{equation}
where \myHighlight{$\Gamma$}\coordHE{} is the projection operator appearing in k-symmetry transformations \cite{ced, bk}
\begin{equation}\coord{}\boxEquation{
\Gamma=
\frac{\sqrt{-g}}{\sqrt{-det(g+F)}}(\gamma_{tz\theta\varphi}\otimes I - 
E_z\ \gamma_{\theta\varphi}\otimes J +
B_{\varphi}\ \gamma_{t\varphi}\otimes J)
\label{g}
}{
\Gamma=
\frac{\sqrt{-g}}{\sqrt{-det(g+F)}}(\gamma_{tz\theta\varphi}\otimes I - 
E_z\ \gamma_{\theta\varphi}\otimes J +
B_{\varphi}\ \gamma_{t\varphi}\otimes J)
}{ecuacion}\coordE{}\end{equation}
with
\begin{equation}\coord{}\boxEquation{
\{\gamma_{\alpha},\gamma_{\beta}\}=g_{\alpha\beta}\ ;\qquad
I=i\sigma_2,\quad J=\sigma_1\ .
\label{ij}
}{
\{\gamma_{\alpha},\gamma_{\beta}\}=g_{\alpha\beta}\ ;\qquad
I=i\sigma_2,\quad J=\sigma_1\ .
}{ecuacion}\coordE{}\end{equation}
The \myHighlight{$10$}\coordHE{} dimensional matrices \myHighlight{$\gamma_{\alpha}$}\coordHE{} act on the Weyl-Majorana index while the matrices \myHighlight{$I,J$}\coordHE{} act on the \myHighlight{$SO(2)$}\coordHE{} index, of the spinor \myHighlight{$\epsilon$}\coordHE{}.

Eqn. (\ref{k}) has no constant solution\footnote[1]{to be precise, there is no spinor which can satisfy (\ref{k}), and which is obtained from a constant spinor through the coordinate transformation from Cartesian to \myHighlight{$S^2$}\coordHE{} coordinates in target space.}, and since supersymmetry is not a local symmetry, it is therefore completely broken. Despite this the brane is stable and its collapse is prevented because the Poynting vector is nonzero, as can be seen from the eqn. (\ref{bps}).

An alternative interpretaion of (\ref{bps}) follows from the fact that the BPS equation (\ref{bi9}), from which it ensues, happens to coincide exactly with the consistency conditions for the second order Euler-Lagrange equations
\begin{equation}\coord{}\boxEquation{
\frac{\partial }{\partial x^k}\sqrt{-g}\,{\d}^k=0,\qquad 
{\varepsilon}^{ijk}\frac{\partial {\h}_i}{\partial x^j}=0,
\label{s2}`
}{
\frac{\partial }{\partial x^k}\sqrt{-g}\,{\d}^k=0,\qquad 
{\varepsilon}^{ijk}\frac{\partial {\h}_i}{\partial x^j}=0,
`
}{ecuacion}\coordE{}\end{equation}
and the corresponding Bianchi identities
\begin{equation}\coord{}\boxEquation{
{\varepsilon}^{ijk}\frac{\partial {\e}_i}{\partial x^j}=0,\qquad 
\frac{\partial }{\partial x^k}\sqrt{-g}\,{\m}^k=0\ .
\label{s3}
}{
{\varepsilon}^{ijk}\frac{\partial {\e}_i}{\partial x^j}=0,\qquad 
\frac{\partial }{\partial x^k}\sqrt{-g}\,{\m}^k=0\ .
}{ecuacion}\coordE{}\end{equation}
This consistency check was carried out in Ref. \cite{dual}.
Thus, at least in our limited context of restricting to perendicular electric and magnetic fields, the only solutions of the static equations are those of the BPS constraint (\ref{bi9}).
 
The electric charge quantisation condition
\begin{equation}\coord{}\boxEquation{
T_3\int D_z\sqrt{-g}\;d\theta d\varphi =\frac{n}{2\pi}\ ,
\label{ec}
}{
T_3\int D_z\sqrt{-g}\;d\theta d\varphi =\frac{n}{2\pi}\ ,
}{ecuacion}\coordE{}\end{equation}
together with the magnetic charge quantisation condition
\begin{equation}\coord{}\boxEquation{
\int F_{z\theta}\;dzd\theta =2\pi m\ ,
\label{mc}
}{
\int F_{z\theta}\;dzd\theta =2\pi m\ ,
}{ecuacion}\coordE{}\end{equation}
connect the constants \myHighlight{$D$}\coordHE{} and \myHighlight{$B$}\coordHE{} with the electric \myHighlight{$n$}\coordHE{} and magnetic \myHighlight{$m$}\coordHE{} charges, respectively
\begin{equation}\coord{}\boxEquation{
D=\frac{2ng_s}{r^2}
\frac{\sqrt{\pi}\Gamma (2-\frac{\gamma}{2})}
{\Gamma (\frac{3-\gamma}{2})},\quad 
B=\frac{4\pi m}{l}
\frac{\sqrt{\pi}\Gamma (\frac{\gamma +1}{2})}{\Gamma (\frac{\gamma}{2})}
\label{emg}
}{
D=\frac{2ng_s}{r^2}
\frac{\sqrt{\pi}\Gamma (2-\frac{\gamma}{2})}
{\Gamma (\frac{3-\gamma}{2})},\quad 
B=\frac{4\pi m}{l}
\frac{\sqrt{\pi}\Gamma (\frac{\gamma +1}{2})}{\Gamma (\frac{\gamma}{2})}
}{ecuacion}\coordE{}\end{equation}
where \myHighlight{$l$}\coordHE{} is the compact length in the \myHighlight{$z$}\coordHE{} direction.

Together with (\ref{bps}), (\ref{emg}) gives an expression for the radius \myHighlight{$r$}\coordHE{} of the brane in terms of the electric and magnetic charges
\begin{equation}\coord{}\boxEquation{
r^3={\rm const}\times n\,m\ .
\label{r}
}{
r^3={\rm const}\times n\,m\ .
}{ecuacion}\coordE{}\end{equation}
It follows immediately from (\ref{r}) that the collapse \myHighlight{$r\to 0$}\coordHE{} cannot take place when neither of the two charges \myHighlight{$n$}\coordHE{} and \myHighlight{$m$}\coordHE{} vanishes.

The energy of the brane, in terms of the quantised charges, is
\begin{equation}\coord{}\boxEquation{
{\cal E}=nT_s\int dz\; +\; mT_1\frac{\gamma\pi r}{2}f(\gamma )
\label{en}
}{
{\cal E}=nT_s\int dz\; +\; mT_1\frac{\gamma\pi r}{2}f(\gamma )
}{ecuacion}\coordE{}\end{equation}
where
\begin{equation}\coord{}\boxEquation{
f(\gamma )=\left(\frac{1}{\sqrt{\pi}}\frac{\Gamma (\frac{\gamma +1}{2})}
{\Gamma (\frac{\gamma}{2}+1)}\right)^2
\label{fac}
}{
f(\gamma )=\left(\frac{1}{\sqrt{\pi}}\frac{\Gamma (\frac{\gamma +1}{2})}
{\Gamma (\frac{\gamma}{2}+1)}\right)^2
}{ecuacion}\coordE{}\end{equation}
We recognise the first term as \myHighlight{$n$}\coordHE{} dissolved F-strings along the \myHighlight{$z$}\coordHE{} axis, and the second term as \myHighlight{$m$}\coordHE{} dissolved D-strings along azimuth.

\section{Uniform magnetic field limit}

Now we consider the limit of a uniform magnetic field, which implies \myHighlight{$B=constant$}\coordHE{}, and \myHighlight{$\gamma\to 0$}\coordHE{} in (\ref{sol}). The integral for the magnetic charge 
\begin{equation}\coord{}\boxEquation{
 m=\frac{1}{2\pi}\int F_{z\theta}\,dzd\theta =\frac{1}{(2\pi )^2}
\int\frac{B_{\varphi}}{\sin\theta}\,dzd\theta
\label{u3}
}{
 m=\frac{1}{2\pi}\int F_{z\theta}\,dzd\theta =\frac{1}{(2\pi )^2}
\int\frac{B_{\varphi}}{\sin\theta}\,dzd\theta
}{ecuacion}\coordE{}\end{equation}
diverges as
\begin{equation}\coord{}\boxEquation{
m_{{}_{\gamma\to 0}}\to\frac{Bl}{2\pi^2}\,\frac{1}{\gamma}\ .
\label{u4}
}{
m_{{}_{\gamma\to 0}}\to\frac{Bl}{2\pi^2}\,\frac{1}{\gamma}\ .
}{ecuacion}\coordE{}\end{equation}
We note, immediately, that the product \myHighlight{$m\gamma$}\coordHE{} remains finite in this limit.

The contribution of the magnetic field to the energy
\begin{equation}\coord{}\boxEquation{
{\cal E}_B=T_3\int |\m |\sqrt{-g}\,dzd\theta d\varphi =
T_3\int \frac{B_{\varphi}}{r\sin\theta}\sqrt{-g}\,dzd\theta d\varphi
\label{u1}
}{
{\cal E}_B=T_3\int |\m |\sqrt{-g}\,dzd\theta d\varphi =
T_3\int \frac{B_{\varphi}}{r\sin\theta}\sqrt{-g}\,dzd\theta d\varphi
}{ecuacion}\coordE{}\end{equation}
is finite when \myHighlight{$\gamma >-1$}\coordHE{} and is a well defined quantity having a constant integrand when \myHighlight{$\gamma =0$}\coordHE{}. The total energy in this limit is
\begin{equation}\coord{}\boxEquation{
{\cal E}_{{}_{\gamma\to 0}}=nT_s\int dz\; +\; mT_1\frac{\gamma\pi r}{2}.
\label{u2}
}{
{\cal E}_{{}_{\gamma\to 0}}=nT_s\int dz\; +\; mT_1\frac{\gamma\pi r}{2}.
}{ecuacion}\coordE{}\end{equation}

It may seem strange at first sight that the D3-brane with magnetic field approaching a uniform field, can have arbitrarily large D-string charge and at the same time finite energy. Furthermore, its radius is also finite in the limit of interest here and is given by
\begin{equation}\coord{}\boxEquation{
r^3=8\pi^2g_snm\gamma\ .
\label{u5}
}{
r^3=8\pi^2g_snm\gamma\ .
}{ecuacion}\coordE{}\end{equation}
We interpret this interesting phenomenon as follows. From Eqn.(\ref{u2}) one concludes that the length of the dissolved D-strings is \myHighlight{$\gamma\pi r/2$}\coordHE{}. It shrinks down in the limit of uniform magnetic field when \myHighlight{$B$}\coordHE{} is kept constant and and dependence on the polar angle disappears. However the number of D-strings multipled with their lengths (\myHighlight{$\sim m\gamma$}\coordHE{}) is finite as can be seen from the relation (\ref{u4}). Both the energy and radius of the brane depend on the product \myHighlight{$m\gamma$}\coordHE{} but not \myHighlight{$m$}\coordHE{} or \myHighlight{$\gamma$}\coordHE{} separately. That is why they have a smooth limit when the magnetic charge grows infinitely and simultaneously the length of each D-string shrinks, the latter being inversely proportional to the charge. It is incorrect to consider the limiting value \myHighlight{$\gamma =0$}\coordHE{}. But any configuration described by a small value of \myHighlight{$\gamma$}\coordHE{} is physically valid. As a result there appears a tubular D3-brane with arbitrarily large D-string charge and finite, well defined energy and radius.


The situation is different with the electric charge. This grows infinitely when \myHighlight{$\gamma\to 3$}\coordHE{} and the energy grows infinitely too, while their ratio remains fixed. This is what one expects because F-strings are aligned with the brane axis and have fixed length. However this does mean there is an asymmetry between electric and magnetic charges. The dual picture is achieved by taking a BPS saturating magnetic field aligned with the axis of the cylinder and an electric field along azimutes of the sphere. Now, the ratio of the energy and magnetic charge is fixed while the electric charge may grow keeping the energy and radius finite.

\section{Conclusions}

We considered a hypertube as expanded (\myHighlight{$p,q$}\coordHE{}) strings. The solution we have used has two independent parameters \myHighlight{$B$}\coordHE{} and \myHighlight{$\gamma$}\coordHE{}. In the special case when \myHighlight{$B$}\coordHE{} is kept constant and \myHighlight{$\gamma$}\coordHE{} is very small we observed large D-string charge and almost ``pointlike'' D-strings. These strings do not interact as the total magnetic energy is an additive sum of the energies of the individual strings. The collection of these D-strings forms a hypertube owing to the precise balance of attractive and repulsive forces. This is like a gas of light D-strings supported in equlibrum by the electric flux along the hypertube. This analogy with a usual gas is limited because turning off the electric flux results in the collapse of the system. 



\smallskip

{\bf Acknowledgments:}
\newline
We would like to acknowledge the collaboration of H.J.W. M\"uller-Kirsten and D.K. Park at the initial stages of this work. This work was carried out in the framework of International Collaboration Project IC-2002-005 of Enterprise--Ireland. S.T. acknowledges also support from Project INTAS-2000-00561 of INTAS.

\begin{thebibliography}{99}


\bibitem{tub}D.Mateos and P.K. Townsend, {\it Subertubes}, Phys.Rev.Lett. 
{\bf 87} (2001) 011602, hep-th/0103030.
\bibitem{tunl}R. Emparan, {\it Born-Infeld Strings Tunneling to D-branes}, 
Phys.Lett. {\bf B423} (1998) 71, hep-th/9711106.
\bibitem{my}R.C. Myers, {\it Dielectric-branes}, JHEP9912(1999)22, hep-th/9910053.
\bibitem{bach}C. Bachas, M. Douglas and C. Schweigert, {\it Flux Stabilization of D-branes}, JHEP0005(2000)048, hep-th/0003037.
\bibitem{tay}W. Taylor, {\it D2-branes in B fields}, JHEP0007(2000)039, hep-th/0004141.
\bibitem{my2}R.C. Myers, {\it Nonabelian Phenomena on D-branes}, hep-th/0303072.
\bibitem{emt}R. Emparan, D. Mateos and P.K. Townsend, {\it Supergravity Supertubes}, JHEP0107(2001) 011, hep-th/0106012.
\bibitem{bak1}D. Bak and K.Lee, {\it Noncommutative Supersymmetric Tubes}, Phys. Lett. {\bf B 509(2001)} 168, hep-th/0103148.
\bibitem{cts}D.K. Park, S. Tamaryan and H.J.W. M\"uller-Kirsten, {\it General Criterion for the existence of Supertube and BIon in Curved Target Space}, to appear in Phys.Lett. {\bf B}, hep-th/0302145.
\bibitem{rot}J.-H. Cho and P. Oh, {\it Rotating Supertubes}, hep-th/0302172.
\bibitem{dual}S. Tamaryan, D.K. Park and H.J.W. M\"uller-Kirsten, {\it Tubular D3-branes and their Dualities}, hep-th/0209239.
\bibitem{bak2}D. Bak, N. Ohta and M.M. Sheikh-Jabbari, {\it Supersymmetric Brane--Antibrane Systems: Matrix Model Description, Stability and Decoupling Limits}, hep-th/0205265.
\bibitem{hel}J-H. Cho and P. Oh, {\it Super D-Helix}, Phys.Rev. {\bf D64}. 
(2001) 106010, hep-th/0105095.
\bibitem{flux}R. Emparan, {\it Tubular Branes in Fluxbranes}, Nucl. Phys. {\bf B610} (2001) 064, hep-th/0105062.
\bibitem{evs2}J. Evslin. {\it IIB Soliton Spectra with All Fluxes Activated}, hep-th/0211172.
\bibitem{evs}J. Evslin and U. Varadarajan, {\it K-Theory and S-Duality: Starting Over from Square 3}, hep-th/0112084.
\bibitem{per}M. Kruczenski, R.C. Myers, A.W. Peet and J. Winters, {\it Aspects of supertubes}, JHEP0205(2002)017, hep-th/0204103.
\bibitem{har}T. Harmark and K.G. Savvidy, {\it Ramond-Ramond Field Radiation from Rotating Ellipsoidal Membranes}, Nucl. Phys., {\bf B 585} (200) 567, hep-th/0002157.

\bibitem{surp}P.K. Townsend, {\it Surprises with Angular Momentum}, 
hep-th/0211008. 
\bibitem{ced}M. Cederwall, A.v. Gussich, B.E.W. Nilsson, P. Sundell and A. Westerberg, {\it The Dirichlet Super-p-Branes in Ten-Dimensional Type IIA and IIB Supergravity}, Nucl.Phys. {\bf B490} (1997) 163-178, hep-th/9610148, Nucl.Phys. {\bf B490} (1997) 179-201, hep-th/9611159.
\bibitem{bk}E. Bergshoeff, R. Kallosh, T. Ortin and G. Papadopoulos, {\it k-Symmetry, Supersymmetry and Intersecting Branes},  Nucl.Phys. {\bf B502} (1997) 145, hep-th/9705040.





\end{thebibliography}



\end{document}
\bye
