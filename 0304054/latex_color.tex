
\documentclass[a4paper,12pt]{article}
\setlength{\oddsidemargin}{+2mm}
\setlength{\textwidth}{160mm}
\setlength{\topmargin}{-0.6in}
\setlength{\textheight}{235mm}
\renewcommand{\baselinestretch}{1} 

\usepackage[centertags]{amsmath}
\usepackage{amssymb}
\usepackage{latexsym}
\allowdisplaybreaks

\providecommand{\al}{\ensuremath{\alpha}}
\providecommand{\be}{\ensuremath{\beta}}
\providecommand{\La}{\ensuremath{\Lambda}}
\providecommand{\la}{\ensuremath{\lambda}}
\providecommand{\ga}{\ensuremath{\gamma}}
\providecommand{\Ga}{\ensuremath{\Gamma}}
\providecommand{\de}{\ensuremath{\delta}}
\providecommand{\si}{\ensuremath{\sigma}}
\providecommand{\Si}{\ensuremath{\Sigma}}
\providecommand{\te}{\ensuremath{\theta}}
\providecommand{\ep}{\ensuremath{\epsilon}}
\providecommand{\sfrac}[2]{\ensuremath{{\scriptstyle \frac{#1}{#2}}}}



\usepackage{useful_macros}
\begin{document}


\setlength{\abovedisplayskip}{10pt plus 3pt minus 9pt}
\setlength{\belowdisplayskip}{10pt plus 3pt minus 9pt}
\setlength{\abovedisplayshortskip}{0pt plus 3pt}
\setlength{\belowdisplayshortskip}{5pt plus 3pt minus 4pt}

\title{\bf Supergravity and IOSp\myHighlight{$\mathbf{(3,1|4)}$}\coordHE{} gauge theory\thanks{Research supported by Enterprise Ireland grant SC/2001/041}}

\author{T.G.\ Philbin\thanks{E-mail: tgp@ucc.ie} \\
 \small \it  Department of Mathematics, University College Cork, Cork, Ireland}  


\date{}

\maketitle

\begin{abstract}
Simple \myHighlight{$D=4$}\coordHE{} supergravity is obtained from the gauge theory of the inhomogeneous orthosymplectic group IOSp\myHighlight{$(3,1|4)$}\coordHE{} on a \myHighlight{$(4,4)$}\coordHE{}-dimensional base supermanifold by imposing constraints and talking a limit. Both the constraints and the limiting procedure have a clear {\it a priori} physical motivation. The construction has similarities with the space-time formulation of Newtonian gravity.
\end{abstract}

\noindent
{\small  PACS:  04.65.+e; 11.15.-q  \newline
   Keywords: supergravity; superspace; gauge theory.}
\vspace{10mm}

The fact that general relativity is a theory of the dynamics of space-time geometry leads to the supposition that its supersymmetric version may also be a naturally geometrical theory, where the geometry involved is that of superspace. This does not appear to be the case, however. The analogue in superspace of general relativity, known as gauge supersymmetry~\cite{nat1}, is not supergravity, so one must consider a less ``natural'' superspace geometry. Nath and Arnowitt~\cite{nat2} obtained supergravity from gauge supersymmetry by contracting the tangent-space group OSp\myHighlight{$(3,1|4)$}\coordHE{} of the (4,4)-supermanifold to its SO(3,1) subgroup and taking a limit, while the standard Wess--Zumino formulation~\cite{wes} requires the imposition of constraints on the torsion. A drawback of both these formalisms is that they have elements, viz the limiting procedure of the former and the constraints of the latter, that have no clear {\it a priori} physical motivation.  We therefore seek a formulation that, while not being as naturally geometrical as gauge supersymmetry, derives from a simple, physically reasonable principle. Such a formulation is found by considering the relationship between the group natural to superspace, IOSp\myHighlight{$(3,1|4)$}\coordHE{}, and the group relevant to physics, the super Poincar\'{e} group.

Superspace is remarkable as a supermanifold in that its coordinates\footnote{Superspace coordinate indices are denoted by \myHighlight{$\La=(\mu,\al,\dot{\al})$}\coordHE{}, vielbein indices by \myHighlight{$A=(m,a,\dot{a})$}\coordHE{}; rules for super index positioning and manipulation are those of DeWitt~\cite{dew}. The space-time metric has signature \myHighlight{$+2$}\coordHE{} and the Infeld--van der Waerden symbols are \myHighlight{$\si^{m}_{\ \,a\dot{a}}=\sfrac{i}{\sqrt{2}}(I,\vec{\si})$}\coordHE{}, where \myHighlight{$\vec{\si}$}\coordHE{} are the Pauli matrices.}
\myHighlight{$Z^\La=(x^\mu,\te^\al,\te^{\dot{\al}})$}\coordHE{} have different mass dimensions \myHighlight{$[Z^\La]$}\coordHE{}:
\begin{equation}\coord{}\boxEquation{ \label{coord}
[x^\mu]=-1, \qquad [\te^\al]=[\te^{\dot{\al}}]=-\sfrac{1}{2}.
}{ [x^\mu]=-1, \qquad [\te^\al]=[\te^{\dot{\al}}]=-\sfrac{1}{2}.
}{ecuacion}\coordE{}\end{equation}
The reason for this is that the scale of the supersymmetry generators \myHighlight{$Q_\al,Q_{\dot{\al}}$}\coordHE{} is chosen to avoid introducing a physically irrelevant constant into the supersymmetry algebra~\cite{wes2}. It has the consequence that the canonical metric in flat Riemannian superspace~\cite{dew} cannot have dimensionless components; for distance to be real with units of \myHighlight{$x^\mu$}\coordHE{} the canonical metric is
\begin{equation}\coord{}\boxEquation{ \label{met}
_\La H_\Pi=\left(\begin{array}{ccc} \eta_{\mu\nu} & 0 & 0 \\
0 & k^2\ep_{\al\be} & 0 \\
0 & 0 & -k^2\ep_{\dot{\al}\dot{\be}} \end{array}\right) \quad
\Rightarrow \quad ^\La H^\Pi=\left(\begin{array}{ccc} \eta^{\mu\nu} & 0 & 0 \\
0 & -\frac{1}{k^2}\ep^{\al\be} & 0 \\
0 & 0 & \frac{1}{k^2}\ep^{\dot{\al}\dot{\be}} \end{array}\right),
}{ _\La H_\Pi=\left(\begin{array}{ccc} \eta_{\mu\nu} & 0 & 0 \\
0 & k^2\ep_{\al\be} & 0 \\
0 & 0 & -k^2\ep_{\dot{\al}\dot{\be}} \end{array}\right) \quad
\Rightarrow \quad ^\La H^\Pi=\left(\begin{array}{ccc} \eta^{\mu\nu} & 0 & 0 \\
0 & -\frac{1}{k^2}\ep^{\al\be} & 0 \\
0 & 0 & \frac{1}{k^2}\ep^{\dot{\al}\dot{\be}} \end{array}\right),
}{ecuacion}\coordE{}\end{equation}
where \myHighlight{$k$}\coordHE{} is a real constant with \myHighlight{$[k]=-\sfrac{1}{2}$}\coordHE{}. The group of coordinate transformations that leaves (\ref{met}) unchanged is IOSp\myHighlight{$(3,1|4)$}\coordHE{}, having the infinitesimal form
\begin{equation}\coord{}\boxEquation{ \label{iosp}
\begin{split}
{Z'}^\La& =Z^\La+\,^\La\!\La_\Pi\,Z^\Pi+\Xi^\La, \\
\text{where} \quad _\La\La_\Pi =-&(-1)^{\La+\Pi+\La\Pi}_{\qquad\qquad\!\!\Pi}\La_\La \quad\text{(antisupersymmetric)},
\end{split}
}{ \begin{split}
{Z'}^\La& =Z^\La+\,^\La\!\La_\Pi\,Z^\Pi+\Xi^\La, \\
\text{where} \quad _\La\La_\Pi =-&(-1)^{\La+\Pi+\La\Pi}_{\qquad\qquad\!\!\Pi}\La_\La \quad\text{(antisupersymmetric)},
\end{split}
}{split}\coordE{}\end{equation} 
with constant parameters \myHighlight{$^\La\!\La_\Pi$}\coordHE{} and \myHighlight{$\Xi^\La$}\coordHE{}. Gauge supersymmetry can now be constructed by gauging this group on superspace, in complete analogy to the manner~\cite{gri,sal} in which general relativity is obtained by gauging the Poincar\'{e} group on space-time. The group relevant to supersymmetric physics, however, is the super Poincar\'{e} group, which gives the following infinitesimal transformation on superspace~\cite{wes2}:
\begin{equation}\coord{}\boxEquation{ \label{sp}
\left(\begin{array}{c} {x'}^\mu \\ {\te'}^\al \\ {\te'}^{\dot{\al}} \end{array}\right)
=\left(\begin{array}{c} x^\mu \\ \te^\al \\ \te^{\dot{\al}} \end{array}\right)+
\left(\begin{array}{ccc} \la^\mu_{\ \nu} & \si^\mu_{\ \be\dot{\al}}\;\xi^{\dot{\al}} & \si^\mu_{\ \al\dot{\be}}\;\xi^{\al} \\
0 & \frac{1}{2}\la_{\mu\nu}\;\si^{\mu\nu\al}_{\ \ \ \,\be} & 0 \\
0 & 0 & \frac{1}{2}\la_{\mu\nu}\;\si^{\mu\nu\dot{\al}}_{\ \ \ \,\dot{\be}} \end{array}\right)
\left(\begin{array}{c} x^\nu \\ \te^\be \\ \te^{\dot{\be}} \end{array}\right) +
\left(\begin{array}{c} a^\mu \\ \xi^\al \\ \xi^{\dot{\al}} \end{array}\right).
}{ \left(\begin{array}{c} {x'}^\mu \\ {\te'}^\al \\ {\te'}^{\dot{\al}} \end{array}\right)
=\left(\begin{array}{c} x^\mu \\ \te^\al \\ \te^{\dot{\al}} \end{array}\right)+
\left(\begin{array}{ccc} \la^\mu_{\ \nu} & \si^\mu_{\ \be\dot{\al}}\;\xi^{\dot{\al}} & \si^\mu_{\ \al\dot{\be}}\;\xi^{\al} \\
0 & \frac{1}{2}\la_{\mu\nu}\;\si^{\mu\nu\al}_{\ \ \ \,\be} & 0 \\
0 & 0 & \frac{1}{2}\la_{\mu\nu}\;\si^{\mu\nu\dot{\al}}_{\ \ \ \,\dot{\be}} \end{array}\right)
\left(\begin{array}{c} x^\nu \\ \te^\be \\ \te^{\dot{\be}} \end{array}\right) +
\left(\begin{array}{c} a^\mu \\ \xi^\al \\ \xi^{\dot{\al}} \end{array}\right).
}{ecuacion}\coordE{}\end{equation}
The metric (\ref{met}) is, of course, not invariant under a super Poincar\'{e} transformation; in the notation of (\ref{iosp}) the ``rotation'' matrix in (\ref{sp}) is
\begin{gather}\coord{}\boxAlignEqnarray{ 
\leftCoord{}^\La\!\La_\Pi =\left(\begin{array}{ccc} \la^\mu_{\rightCoord{}\ \nu} & \si^\mu_{\rightCoord{}\ \be\dot{\al}}\;\xi^{\dot{\al}} & \si^\mu_{\rightCoord{}\ \al\dot{\be}}\;\xi^{\al} \rightCoord{}\\
\leftCoord{}0 & \frac{\leftCoord{}1}{\rightCoord{}2}\la_{\rightCoord{}\mu\nu}\;\si^{\mu\nu\al}_{\rightCoord{}\ \ \ \,\be} & 0 \rightCoord{}\\
\leftCoord{}0 & 0 & \frac{\leftCoord{}1}{\rightCoord{}2}\la_{\rightCoord{}\mu\nu}\;\si^{\mu\nu\dot{\al}}_{\rightCoord{}\ \ \ \,\dot{\be}} \end{array}\right) \\[8pt]
 \leftCoord{}\Rightarrow\quad _\La\La_\Pi=\left(\begin{array}{ccc} \la_{\rightCoord{}\mu\nu} & \si_{\rightCoord{}\mu\be\dot{\al}}\;\xi^{\dot{\al}} & \si_{\rightCoord{}\mu\al\dot{\be}}\;\xi^{\al} \rightCoord{}\\
\leftCoord{}0 & -\frac{\leftCoord{}1}{\rightCoord{}2}k^2\la_{\rightCoord{}\mu\nu}\;\si^{\mu\nu}_{\rightCoord{}\ \ \al\be} & 0 \rightCoord{}\\
\leftCoord{}0 & 0 & \frac{\leftCoord{}1}{\rightCoord{}2}k^2\la_{\rightCoord{}\mu\nu}\;\si^{\mu\nu}_{\rightCoord{}\ \ \dot{\al}\dot{\be}} \end{array}\right) \label{rotsp}
}{0mm}{10}{23}{ 
^\La\!\La_\Pi =\left(\begin{array}{ccc} \la^\mu_{\ \nu} & \si^\mu_{\ \be\dot{\al}}\;\xi^{\dot{\al}} & \si^\mu_{\ \al\dot{\be}}\;\xi^{\al} \\
0 & \frac{1}{2}\la_{\mu\nu}\;\si^{\mu\nu\al}_{\ \ \ \,\be} & 0 \\
0 & 0 & \frac{1}{2}\la_{\mu\nu}\;\si^{\mu\nu\dot{\al}}_{\ \ \ \,\dot{\be}} \end{array}\right) \\[8pt]
 \Rightarrow\quad _\La\La_\Pi=\left(\begin{array}{ccc} \la_{\mu\nu} & \si_{\mu\be\dot{\al}}\;\xi^{\dot{\al}} & \si_{\mu\al\dot{\be}}\;\xi^{\al} \\
0 & -\frac{1}{2}k^2\la_{\mu\nu}\;\si^{\mu\nu}_{\ \ \al\be} & 0 \\
0 & 0 & \frac{1}{2}k^2\la_{\mu\nu}\;\si^{\mu\nu}_{\ \ \dot{\al}\dot{\be}} \end{array}\right) }{0}\coordE{}\end{gather}
and the latter is clearly not antisupersymmetric. We can construct an antisupersymmetric matrix from (\ref{rotsp}) by inserting additional elements; the least modification necessary to achieve this results in the transformation
\begin{equation}\coord{}\boxEquation{ \label{modsp}
\left(\begin{array}{c} {x'}^\mu \\ {\te'}^\al \\ {\te'}^{\dot{\al}} \end{array}\right)
=\left(\begin{array}{c} x^\mu \\ \te^\al \\ \te^{\dot{\al}} \end{array}\right)+
\left(\begin{array}{ccc} \la^\mu_{\ \nu} & \si^\mu_{\ \be\dot{\al}}\;\xi^{\dot{\al}} & \si^\mu_{\ \al\dot{\be}}\;\xi^{\al} \\
-\frac{1}{k^2}\si_{\nu\ \dot{\al}}^{\ \al}\;\xi^{\dot{\al}} & \frac{1}{2}\la_{\mu\nu}\;\si^{\mu\nu\al}_{\ \ \ \,\be} & 0 \\
\frac{1}{k^2}\si_{\nu\al}^{\ \ \;\dot{\al}}\;\xi^\al & 0 & \frac{1}{2}\la_{\mu\nu}\;\si^{\mu\nu\dot{\al}}_{\ \ \ \,\dot{\be}} \end{array}\right)
\left(\begin{array}{c} x^\nu \\ \te^\be \\ \te^{\dot{\be}} \end{array}\right) +
\left(\begin{array}{c} a^\mu \\ \xi^\al \\ \xi^{\dot{\al}} \end{array}\right).
 }{ \left(\begin{array}{c} {x'}^\mu \\ {\te'}^\al \\ {\te'}^{\dot{\al}} \end{array}\right)
=\left(\begin{array}{c} x^\mu \\ \te^\al \\ \te^{\dot{\al}} \end{array}\right)+
\left(\begin{array}{ccc} \la^\mu_{\ \nu} & \si^\mu_{\ \be\dot{\al}}\;\xi^{\dot{\al}} & \si^\mu_{\ \al\dot{\be}}\;\xi^{\al} \\
-\frac{1}{k^2}\si_{\nu\ \dot{\al}}^{\ \al}\;\xi^{\dot{\al}} & \frac{1}{2}\la_{\mu\nu}\;\si^{\mu\nu\al}_{\ \ \ \,\be} & 0 \\
\frac{1}{k^2}\si_{\nu\al}^{\ \ \;\dot{\al}}\;\xi^\al & 0 & \frac{1}{2}\la_{\mu\nu}\;\si^{\mu\nu\dot{\al}}_{\ \ \ \,\dot{\be}} \end{array}\right)
\left(\begin{array}{c} x^\nu \\ \te^\be \\ \te^{\dot{\be}} \end{array}\right) +
\left(\begin{array}{c} a^\mu \\ \xi^\al \\ \xi^{\dot{\al}} \end{array}\right).
 }{ecuacion}\coordE{}\end{equation}
Transformations of the form (\ref{modsp}) do not constitute a group: they are a subset of IOSp\myHighlight{$(3,1|4)$}\coordHE{}, not a subgroup. But note that the \myHighlight{$k\to\infty$}\coordHE{} limit of (\ref{modsp}) does give a group---the group of super Poincar\'{e} transformations (\ref{sp}).

These considerations suggest how supergravity may be related to IOSp\myHighlight{$(3,1|4)$}\coordHE{} gauge theory. The latter gives us super one-form gauge potentials \myHighlight{$A^A_{\ B\La}$}\coordHE{} and \myHighlight{$^A\!E_\La$}\coordHE{} (the latter chosen to be the vielbein\footnote{We thus have an {\it affine connection}~\cite{kob} in the principle bundle with fibre IOSp\myHighlight{$(3,1|4)$}\coordHE{}.}) with values in the Lie algebra of IOSp\myHighlight{$(3,1|4)$}\coordHE{}. Now we have seen how to obtain the infinitesimal super Poincar\'{e} group from infinitesimal IOSp\myHighlight{$(3,1|4)$}\coordHE{}---extract all elements of infinitesimal  IOSp\myHighlight{$(3,1|4)$}\coordHE{} of the form (\ref{modsp}) and take \myHighlight{$k\to\infty$}\coordHE{}---so we can perform a similar operation with the respective Lie algebras. In this manner the potentials \myHighlight{$A^A_{\ B\La}$}\coordHE{} and \myHighlight{$^A\!E_\La$}\coordHE{} are turned into super Poincar\'{e}-algebra-valued objects and it is at this point that one might expect to see some physics.

The gauge transformation equations for the IOSp\myHighlight{$(3,1|4)$}\coordHE{} potentials are
\begin{gather}\coord{}\boxAlignEqnarray{
\leftCoord{}\de\,^A\!E_\La=\Xi^A_{\rightCoord{}\ ,\La}+(-1)^{B(A+1)}\;\;\Xi^B\;A^A_{\rightCoord{}\ B\La}-\,^A\!\La_B\;^B\!E_\La,  \label{etrans}\rightCoord{}\\
\leftCoord{}\de\,A^A_{\rightCoord{}\ B\La}=\,^A\!\La_{\rightCoord{}B,\La}+(-1)^{(C+B)(A+C)}\;\;^C\!\La_B\;A^A_{\rightCoord{}\ C\La}-\,^A\!\La_C\;A^C_{\rightCoord{}\ B\La}, \label{atrans}
}{0mm}{2}{8}{
\de\,^A\!E_\La=\Xi^A_{\ ,\La}+(-1)^{B(A+1)}\;\;\Xi^B\;A^A_{\ B\La}-\,^A\!\La_B\;^B\!E_\La,  \\
\de\,A^A_{\ B\La}=\,^A\!\La_{B,\La}+(-1)^{(C+B)(A+C)}\;\;^C\!\La_B\;A^A_{\ C\La}-\,^A\!\La_C\;A^C_{\ B\La}, }{0}\coordE{}\end{gather}
while the curvature two-forms are
\begin{gather}\coord{}\boxAlignEqnarray{
\leftCoord{}R^A_{\rightCoord{}\ \La\Pi}=(-1)^{\La\Pi}\;\;^A\!E_{\rightCoord{}\Pi,\La}-\,^A\!E_{\rightCoord{}\La,\Pi}+(-1)^{\La B}\;A^A_{\rightCoord{}\ B\La}\;^B\!E_\Pi-(-1)^{\Pi(B+\La)}\;A^A_{\rightCoord{}\ B\Pi}\;^B\!E_\La, \label{Tor}\rightCoord{}\\
\leftCoord{}R^A_{\rightCoord{}\ B\La\Pi}=(-1)^{\La\Pi}\; A^A_{\rightCoord{}\ B\Pi,\La}-A^A_{\rightCoord{}\ B\La,\Pi}+(-1)^{\La(C+B)}\; A^A_{\rightCoord{}\ C\La}\,A^C_{\rightCoord{}\ B\Pi}-(-1)^{\Pi(C+B+\La)}\; A^A_{\rightCoord{}\ C\Pi}\,A^C_{\rightCoord{}\ B\La}. \label{Cur}
}{0mm}{2}{14}{
R^A_{\ \La\Pi}=(-1)^{\La\Pi}\;\;^A\!E_{\Pi,\La}-\,^A\!E_{\La,\Pi}+(-1)^{\La B}\;A^A_{\ B\La}\;^B\!E_\Pi-(-1)^{\Pi(B+\La)}\;A^A_{\ B\Pi}\;^B\!E_\La, \\
R^A_{\ B\La\Pi}=(-1)^{\La\Pi}\; A^A_{\ B\Pi,\La}-A^A_{\ B\La,\Pi}+(-1)^{\La(C+B)}\; A^A_{\ C\La}\,A^C_{\ B\Pi}-(-1)^{\Pi(C+B+\La)}\; A^A_{\ C\Pi}\,A^C_{\ B\La}. }{0}\coordE{}\end{gather}
The latter satisfy the following Bianchi identities:
\begin{gather}\coord{}\boxAlignEqnarray{
\leftCoord{}\begin{split}
\leftCoord{}R^A_{\rightCoord{}\ \La\Pi|\Si}&+(-1)^{\La(\Pi+\Si)}\;R^A_{\rightCoord{}\ \Pi\Si|\La}+(-1)^{\Si(\La+\Pi)}\;R^A_{\rightCoord{}\ \Si\La|\Pi}=(-1)^{(B+\La)(\Pi+\Si)}\;R^A_{\rightCoord{}\ B\Pi\Si}\,^B\!E_\La  \rightCoord{}\\
&\leftCoord{}+(-1)^{B(\Si+\La)+\Si(\La+\Pi)}\;R^A_{\rightCoord{}\ B\Si\La}\,^B\!E_\Pi+(-1)^{B(\La+\Pi)}\;R^A_{\rightCoord{}\ B\La\Pi}\,^B\!E_\Si,
\leftCoord{}\end{split}  \label{bian1}\rightCoord{}  \\[8pt]
\leftCoord{}R^A_{\rightCoord{}\ B\La\Pi|\Si}+(-1)^{\La(\Pi+\Si)}\;R^A_{\rightCoord{}\ B\Pi\Si|\La}+(-1)^{\Si(\La+\Pi)}\;R^A_{\rightCoord{}\ B\Si\La|\Pi}=0,  \label{bian2}
}{0mm}{5}{12}{
\begin{split}
R^A_{\ \La\Pi|\Si}&+(-1)^{\La(\Pi+\Si)}\;R^A_{\ \Pi\Si|\La}+(-1)^{\Si(\La+\Pi)}\;R^A_{\ \Si\La|\Pi}=(-1)^{(B+\La)(\Pi+\Si)}\;R^A_{\ B\Pi\Si}\,^B\!E_\La  \\
&+(-1)^{B(\Si+\La)+\Si(\La+\Pi)}\;R^A_{\ B\Si\La}\,^B\!E_\Pi+(-1)^{B(\La+\Pi)}\;R^A_{\ B\La\Pi}\,^B\!E_\Si,
\end{split}  \\[8pt]
R^A_{\ B\La\Pi|\Si}+(-1)^{\La(\Pi+\Si)}\;R^A_{\ B\Pi\Si|\La}+(-1)^{\Si(\La+\Pi)}\;R^A_{\ B\Si\La|\Pi}=0,  }{0}\coordE{}\end{gather}
where \myHighlight{$\scriptstyle |$}\coordHE{} denotes a covariant derivative with connection \myHighlight{$A^A_{\ B\La}$}\coordHE{} that acts only on vielbein indices.

In curved superspace the canonical metric (\ref{met}) with coordinates (\ref{coord}) can only be introduced locally. The corresponding objects now are the vielbein and the vielbein components of the metric, which we choose in line with 
(\ref{coord}) and (\ref{met}):
\begin{gather*}\coord{}\boxAlignEqnarray{
\leftCoord{}[E_m]=1, \qquad [E_a]=[E_{\rightCoord{}\dot{a}}]=\sfrac{1}{2}, \\[8pt]
\leftCoord{}_A H_B=\left(\begin{array}{ccc} \eta_{\rightCoord{}mn} & 0 & 0 \rightCoord{}\\
\leftCoord{}0 & k^2\ep_{\rightCoord{}ab} & 0 \rightCoord{}\\
\leftCoord{}0 & 0 & -k^2\ep_{\rightCoord{}\dot{a}\dot{b}} \end{array}\right) \quad
\leftCoord{}\Rightarrow \quad ^A H^B=\left(\begin{array}{ccc} \eta^{mn} & 0 & 0 \rightCoord{}\\
\leftCoord{}0 & -\frac{\leftCoord{}1}{\rightCoord{}k^2}\ep^{ab} & 0 \rightCoord{}\\
\leftCoord{}0 & 0 & \frac{\leftCoord{}1}{\rightCoord{}k^2}\ep^{\dot{a}\dot{b}} \end{array}\right).
}{0mm}{9}{11}{
[E_m]=1, \qquad [E_a]=[E_{\dot{a}}]=\sfrac{1}{2}, \\[8pt]
_A H_B=\left(\begin{array}{ccc} \eta_{mn} & 0 & 0 \\
0 & k^2\ep_{ab} & 0 \\
0 & 0 & -k^2\ep_{\dot{a}\dot{b}} \end{array}\right) \quad
\Rightarrow \quad ^A H^B=\left(\begin{array}{ccc} \eta^{mn} & 0 & 0 \\
0 & -\frac{1}{k^2}\ep^{ab} & 0 \\
0 & 0 & \frac{1}{k^2}\ep^{\dot{a}\dot{b}} \end{array}\right).
}{0}\coordE{}\end{gather*}
We choose the coordinates \myHighlight{$Z^\La$}\coordHE{} to have dimension
\[\coord{}\boxMath{
[Z^\La]=-1\quad \forall \quad {\scriptstyle \La} 
}{corchetes}{0pt}\coordE{}\]
so that the metric coordinate components 
\[\coord{}\boxMath{
_\La G_\Pi=\,_\La E^A\;_AH_B\;^B\!E_\Pi
}{corchetes}{0pt}\coordE{}\]
are dimensionless. Note that in flat superspace the frame \myHighlight{$E_A$}\coordHE{} may be taken as a coordinate frame and the simplest choice of vielbein (now a coordinate transformation) is then
\begin{equation}\coord{}\boxEquation{ \label{eflat}
^A\!E_\La=\left(\begin{array}{ccc} ^m\de_\mu & 0 & 0 \\
0 & \frac{1}{k}\,^a\de_\al & 0 \\
0 & 0 & \frac{1}{k}\,^{\dot{a}}\de_{\dot{\al}} \end{array}\right).
}{ ^A\!E_\La=\left(\begin{array}{ccc} ^m\de_\mu & 0 & 0 \\
0 & \frac{1}{k}\,^a\de_\al & 0 \\
0 & 0 & \frac{1}{k}\,^{\dot{a}}\de_{\dot{\al}} \end{array}\right).
}{ecuacion}\coordE{}\end{equation}
In light of (\ref{eflat}) we make the standard observation that by a partial choice of gauge we may fix
\begin{gather}\coord{}\boxAlignEqnarray{
\leftCoord{}^A\!E_\La(Z^\al=Z^{\dot{\al}}=0)=\left(\begin{array}{ccc} e^m_{\rightCoord{}\ \,\mu} & 0 & 0 \rightCoord{}\\
\leftCoord{}\frac{\leftCoord{}1}{\rightCoord{}2}\psi^a_{\rightCoord{}\ \mu} & \frac{\leftCoord{}1}{\rightCoord{}k}\,^a\de_\al & 0 \rightCoord{}\\
\leftCoord{}\frac{\leftCoord{}1}{\rightCoord{}2}\psi^{\dot{a}}_{\rightCoord{}\ \mu} & 0 & \frac{\leftCoord{}1}{\rightCoord{}k}\,^{\dot{a}}\de_{\rightCoord{}\dot{\al}} \end{array}\right), \label{ecurv}\rightCoord{}\\
\leftCoord{}A^A_{\rightCoord{}\ B\al}(Z^\al=Z^{\dot{\al}}=0)=A^A_{\rightCoord{}\ B\dot{\al}}(Z^\al=Z^{\dot{\al}}=0)=0, \label{A=0}
}{0mm}{8}{14}{
^A\!E_\La(Z^\al=Z^{\dot{\al}}=0)=\left(\begin{array}{ccc} e^m_{\ \,\mu} & 0 & 0 \\
\frac{1}{2}\psi^a_{\ \mu} & \frac{1}{k}\,^a\de_\al & 0 \\
\frac{1}{2}\psi^{\dot{a}}_{\ \mu} & 0 & \frac{1}{k}\,^{\dot{a}}\de_{\dot{\al}} \end{array}\right), \\
A^A_{\ B\al}(Z^\al=Z^{\dot{\al}}=0)=A^A_{\ B\dot{\al}}(Z^\al=Z^{\dot{\al}}=0)=0, }{0}\coordE{}\end{gather}
where \myHighlight{$e^m_{\ \,\mu}$}\coordHE{}, \myHighlight{$\psi^a_{\ \mu}$}\coordHE{} and \myHighlight{$\psi^{\dot{a}}_{\ \mu}$}\coordHE{} are functions of \myHighlight{$Z^\mu$}\coordHE{} (\myHighlight{$\psi^a_{\ \mu}$}\coordHE{} and \myHighlight{$\psi^{\dot{a}}_{\ \mu}$}\coordHE{}  \myHighlight{$a$}\coordHE{}-type).


Following the path we have set out, translation potentials \myHighlight{$^A\!E_\La$}\coordHE{} and parameters \myHighlight{$\Xi^A$}\coordHE{} are to remain independent, whereas we impose the following constraints on the ``rotation'' potentials \myHighlight{$A^A_{\ B\La}$}\coordHE{} and parameters \myHighlight{$^A\!\La_B$}\coordHE{}:
\begin{gather}\coord{}\boxAlignEqnarray{
\leftCoord{}A^A_{\rightCoord{}\ B\La}=\left(\begin{array}{ccc} A^m_{\rightCoord{}\ \,n\La} & \si^m_{\rightCoord{}\ \,b\dot{a}}\;^{\dot{a}}\!E_\La & \si^m_{\rightCoord{}\ \,a\dot{b}}\;^{a}\!E_\La \rightCoord{}\\
\leftCoord{}-\frac{\leftCoord{}1}{\rightCoord{}k^2}\si_{\rightCoord{}n\ \dot{a}}^{\ a}\;^{\dot{a}}\!E_\La & \frac{\leftCoord{}1}{\rightCoord{}2}A^m_{\rightCoord{}\ \,n\La}\;\si_{\rightCoord{}m\ \ b}^{\ \;na} & 0 \rightCoord{}\\
\leftCoord{}\frac{\leftCoord{}1}{\rightCoord{}k^2}\si_{\rightCoord{}na}^{\ \ \,\dot{a}}\;^a\!E_\La & 0 & \frac{\leftCoord{}1}{\rightCoord{}2}A^m_{\rightCoord{}\ \,n\La}\;\si_{\rightCoord{}m\ \ \dot{b}}^{\ \;n\dot{a}} \end{array}\right),   \label{con1}\rightCoord{}    \\[8pt]
\leftCoord{}^A\!\La_B=\left(\begin{array}{ccc} ^m\!\La_{\rightCoord{}\!\!\!\ n} & \si^m_{\rightCoord{}\ \,b\dot{a}}\;\Xi^{\dot{a}} & \si^m_{\rightCoord{}\ \,a\dot{b}}\;\Xi^{a} \rightCoord{}\\ 
\leftCoord{}-\frac{\leftCoord{}1}{\rightCoord{}k^2}\si_{\rightCoord{}n\ \dot{a}}^{\ a}\;\Xi^{\dot{a}} & \frac{\leftCoord{}1}{\rightCoord{}2}\,^m\!\La_n\;\si_{\rightCoord{}m\ \ b}^{\ \;na} & 0 \rightCoord{}\\
\leftCoord{}\frac{\leftCoord{}1}{\rightCoord{}k^2}\si_{\rightCoord{}na}^{\ \ \,\dot{a}}\;\Xi^a & 0 & \frac{\leftCoord{}1}{\rightCoord{}2}\,^m\!\La_n\;\si_{\rightCoord{}m\ \ \dot{b}}^{\ \;n\dot{a}} \end{array}\right). \label{con2}
}{0mm}{14}{31}{
A^A_{\ B\La}=\left(\begin{array}{ccc} A^m_{\ \,n\La} & \si^m_{\ \,b\dot{a}}\;^{\dot{a}}\!E_\La & \si^m_{\ \,a\dot{b}}\;^{a}\!E_\La \\
-\frac{1}{k^2}\si_{n\ \dot{a}}^{\ a}\;^{\dot{a}}\!E_\La & \frac{1}{2}A^m_{\ \,n\La}\;\si_{m\ \ b}^{\ \;na} & 0 \\
\frac{1}{k^2}\si_{na}^{\ \ \,\dot{a}}\;^a\!E_\La & 0 & \frac{1}{2}A^m_{\ \,n\La}\;\si_{m\ \ \dot{b}}^{\ \;n\dot{a}} \end{array}\right),   \\[8pt]
^A\!\La_B=\left(\begin{array}{ccc} ^m\!\La_{\!\!\!\ n} & \si^m_{\ \,b\dot{a}}\;\Xi^{\dot{a}} & \si^m_{\ \,a\dot{b}}\;\Xi^{a} \\ 
-\frac{1}{k^2}\si_{n\ \dot{a}}^{\ a}\;\Xi^{\dot{a}} & \frac{1}{2}\,^m\!\La_n\;\si_{m\ \ b}^{\ \;na} & 0 \\
\frac{1}{k^2}\si_{na}^{\ \ \,\dot{a}}\;\Xi^a & 0 & \frac{1}{2}\,^m\!\La_n\;\si_{m\ \ \dot{b}}^{\ \;n\dot{a}} \end{array}\right). }{0}\coordE{}\end{gather}
Imposing (\ref{con1}) and (\ref{con2}) on the gauge transformation (\ref{etrans}) of \myHighlight{$^A\!E_\La$}\coordHE{} gives, in the limit \myHighlight{$k\to\infty$}\coordHE{},
\begin{eqnarray}\coord{}\boxAlignEqnarray{\leftCoord{}
\de\rightCoord{}\,^m\!E_\La&=&\Xi^m_{\ ,\La}+\Xi^n\;A^m_{\ \rightCoord{}\,n\La}-2\si^m_{\ \rightCoord{}\,a\dot{a}}(\Xi^a\;^{\dot{a}}\!E_\La+\Xi^{\dot{a}}\;^a\!E_\La)-\rightCoord{}\,^m\!\La_n\;^n\!E_\La,  \rightCoord{}\label{etransa} \\\leftCoord{}[8pt]
\de\rightCoord{}\,^a\!E_\La&=&\Xi^a_{\ ,\La}+\frac{\leftCoord{}1}{\rightCoord{}2}\rightCoord{}\,\Xi^b\;A^m_{\ \rightCoord{}\,n\La}\;\si_{m\ \ b}^{\ \;na}-\frac{\leftCoord{}1}{\rightCoord{}2}\rightCoord{}\,^m\!\La_n\;\si_{m\ \ b}^{\ \;na}\;^b\!E_\La, \rightCoord{}\label{etransb} \\\leftCoord{}[8pt]
\de\rightCoord{}\,^{\dot{a}}\!E_\La&=&\Xi^{\dot{a}}_{\ ,\La}+\frac{\leftCoord{}1}{\rightCoord{}2}\rightCoord{}\,\Xi^{\dot{b}}\;A^m_{\ \rightCoord{}\,n\La}\;\si_{m\ \ \dot{b}}^{\ \;n\dot{a}}-\frac{\leftCoord{}1}{\rightCoord{}2}\rightCoord{}\,^m\!\La_n\;\si_{m\ \ \dot{b}}^{\ \;n\dot{a}}\;^{\dot{b}}E_\La. \rightCoord{}\label{etransc}
\rightCoord{}}{0mm}{7}{21}{
\de\,^m\!E_\La&=&\Xi^m_{\ ,\La}+\Xi^n\;A^m_{\ \,n\La}-2\si^m_{\ \,a\dot{a}}(\Xi^a\;^{\dot{a}}\!E_\La+\Xi^{\dot{a}}\;^a\!E_\La)-\,^m\!\La_n\;^n\!E_\La,  \\[8pt]
\de\,^a\!E_\La&=&\Xi^a_{\ ,\La}+\frac{1}{2}\,\Xi^b\;A^m_{\ \,n\La}\;\si_{m\ \ b}^{\ \;na}-\frac{1}{2}\,^m\!\La_n\;\si_{m\ \ b}^{\ \;na}\;^b\!E_\La, \\[8pt]
\de\,^{\dot{a}}\!E_\La&=&\Xi^{\dot{a}}_{\ ,\La}+\frac{1}{2}\,\Xi^{\dot{b}}\;A^m_{\ \,n\La}\;\si_{m\ \ \dot{b}}^{\ \;n\dot{a}}-\frac{1}{2}\,^m\!\La_n\;\si_{m\ \ \dot{b}}^{\ \;n\dot{a}}\;^{\dot{b}}E_\La. }{1}\coordE{}\end{eqnarray}
On the other hand, with the constraints (\ref{con1}) the only independent \myHighlight{$A^A_{\ B\La}$}\coordHE{} are now the \myHighlight{$6\times 8$}\coordHE{} independent \myHighlight{$A^m_{\ \,n\La}$}\coordHE{}. In the limit \myHighlight{$k\to\infty$}\coordHE{}, (\ref{con1}) and (\ref{atrans}) give
\begin{equation}\coord{}\boxEquation{ \label{atransa}
\de\,A^m_{\ \,n\La}=\,^m\!\La_{n,\La}+\,^r\!\La_n\;A^m_{\ \,r\La}-\,^m\!\La_r\;A^r_{\ n\La}.
}{ \de\,A^m_{\ \,n\La}=\,^m\!\La_{n,\La}+\,^r\!\La_n\;A^m_{\ \,r\La}-\,^m\!\La_r\;A^r_{\ n\La}.
}{ecuacion}\coordE{}\end{equation}
With (\ref{con1}) and \myHighlight{$k\to\infty$}\coordHE{}, the curvature (\ref{Tor}) becomes
\begin{gather}\coord{}\boxAlignEqnarray{
\leftCoord{}\begin{split}
\leftCoord{}R^m_{\rightCoord{}\ \,\La\Pi}=&(-1)^{\La\Pi}\;\;^m\!E_{\rightCoord{}\Pi,\La}-\,^m\!E_{\rightCoord{}\La,\Pi}+A^m_{\rightCoord{}\ \,n\La}\;^n\!E_\Pi-(-1)^{\Pi\La}\;A^m_{\rightCoord{}\ \,n\Pi}\;^n\!E_\La   \rightCoord{}\\
&\leftCoord{}+2(-1)^\La\;\si^m_{\rightCoord{}\ \,a\dot{a}}(^{\dot{a}}\!E_\La\;^a\!E_\Pi+\,^a\!E_\La\;^{\dot{a}}\!E_\Pi),  
\leftCoord{}\end{split} \label{Tora}\rightCoord{}  \\[8pt]
\leftCoord{}R^a_{\rightCoord{}\ \La\Pi}=(-1)^{\La\Pi}\;\;^a\!E_{\rightCoord{}\Pi,\La}-\,^a\!E_{\rightCoord{}\La,\Pi}+\frac{\leftCoord{}1}{\rightCoord{}2}(-1)^\La\;A^m_{\rightCoord{}\ \,n\La}\;\si_{\rightCoord{}m\ \ b}^{\ \;na}\;^b\!E_\Pi-\frac{\leftCoord{}1}{\rightCoord{}2}(-1)^{\Pi(1+\La)}\;A^m_{\rightCoord{}\ \,n\Pi}\;\si_{\rightCoord{}m\ \ b}^{\ \;na}\;^b\!E_\La, \label{Torb}\rightCoord{}\\
\leftCoord{}R^{\dot{a}}_{\rightCoord{}\ \La\Pi}=(-1)^{\La\Pi}\;\;^{\dot{a}}\!E_{\rightCoord{}\Pi,\La}-\,^{\dot{a}}\!E_{\rightCoord{}\La,\Pi}+\frac{\leftCoord{}1}{\rightCoord{}2}(-1)^\La\;A^m_{\rightCoord{}\ \,n\La}\;\si_{\rightCoord{}m\ \ \dot{b}}^{\ \;n\dot{a}}\;^{\dot{b}}\!E_\Pi-\frac{\leftCoord{}1}{\rightCoord{}2}(-1)^{\Pi(1+\La)}\;A^m_{\rightCoord{}\ \,n\Pi}\;\si_{\rightCoord{}m\ \ \dot{b}}^{\ \;n\dot{a}}\;^{\dot{b}}\!E_\La, \label{Torc}
}{0mm}{10}{28}{
\begin{split}
R^m_{\ \,\La\Pi}=&(-1)^{\La\Pi}\;\;^m\!E_{\Pi,\La}-\,^m\!E_{\La,\Pi}+A^m_{\ \,n\La}\;^n\!E_\Pi-(-1)^{\Pi\La}\;A^m_{\ \,n\Pi}\;^n\!E_\La   \\
&+2(-1)^\La\;\si^m_{\ \,a\dot{a}}(^{\dot{a}}\!E_\La\;^a\!E_\Pi+\,^a\!E_\La\;^{\dot{a}}\!E_\Pi),  
\end{split} \\[8pt]
R^a_{\ \La\Pi}=(-1)^{\La\Pi}\;\;^a\!E_{\Pi,\La}-\,^a\!E_{\La,\Pi}+\frac{1}{2}(-1)^\La\;A^m_{\ \,n\La}\;\si_{m\ \ b}^{\ \;na}\;^b\!E_\Pi-\frac{1}{2}(-1)^{\Pi(1+\La)}\;A^m_{\ \,n\Pi}\;\si_{m\ \ b}^{\ \;na}\;^b\!E_\La, \\
R^{\dot{a}}_{\ \La\Pi}=(-1)^{\La\Pi}\;\;^{\dot{a}}\!E_{\Pi,\La}-\,^{\dot{a}}\!E_{\La,\Pi}+\frac{1}{2}(-1)^\La\;A^m_{\ \,n\La}\;\si_{m\ \ \dot{b}}^{\ \;n\dot{a}}\;^{\dot{b}}\!E_\Pi-\frac{1}{2}(-1)^{\Pi(1+\La)}\;A^m_{\ \,n\Pi}\;\si_{m\ \ \dot{b}}^{\ \;n\dot{a}}\;^{\dot{b}}\!E_\La, }{0}\coordE{}\end{gather}
while the curvature (\ref{Cur}) is given by
\begin{gather}\coord{}\boxAlignEqnarray{
\leftCoord{}R^m_{\rightCoord{}\ \,n\La\Pi}=(-1)^{\La\Pi}\,A^m_{\rightCoord{}\ \,n\Pi,\La}-A^m_{\rightCoord{}\ \,n\La,\Pi}+A^m_{\rightCoord{}\ \,r\La}A^r_{\rightCoord{}\ n\Pi}-(-1)^{\La\Pi}\,A^m_{\rightCoord{}\ \,r\Pi}A^r_{\rightCoord{}\ n\La}, \label{Cura}\rightCoord{}\\
\leftCoord{}R^m_{\rightCoord{}\ \,a\La\Pi}=\si^m_{\rightCoord{}\ \,a\dot{a}}\,R^{\dot{a}}_{\rightCoord{}\ \La\Pi}, \quad R^m_{\rightCoord{}\ \,\dot{a}\La\Pi}=\si^m_{\rightCoord{}\ \,a\dot{a}}\,R^a_{\rightCoord{}\ \La\Pi},
\leftCoord{}\quad R^{\dot{a}}_{\rightCoord{}\ m\La\Pi}=R^{\dot{a}}_{\rightCoord{}\ m\La\Pi}=0, \label{Curb}\rightCoord{} \\[8pt]
\leftCoord{}R^a_{\rightCoord{}\ b\La\Pi}=\frac{\leftCoord{}1}{\rightCoord{}2}R^m_{\rightCoord{}\ \,n\La\Pi}\,\si_{\rightCoord{}m\ \ b}^{\ \;na},  \quad R^{\dot{a}}_{\rightCoord{}\ \dot{b}\La\Pi}=\frac{\leftCoord{}1}{\rightCoord{}2}R^m_{\rightCoord{}\ \,n\La\Pi}\,\si_{\rightCoord{}m\ \ \dot{b}}^{\ \;n\dot{a}}.  \label{Curc}
}{0mm}{6}{26}{
R^m_{\ \,n\La\Pi}=(-1)^{\La\Pi}\,A^m_{\ \,n\Pi,\La}-A^m_{\ \,n\La,\Pi}+A^m_{\ \,r\La}A^r_{\ n\Pi}-(-1)^{\La\Pi}\,A^m_{\ \,r\Pi}A^r_{\ n\La}, \\
R^m_{\ \,a\La\Pi}=\si^m_{\ \,a\dot{a}}\,R^{\dot{a}}_{\ \La\Pi}, \quad R^m_{\ \,\dot{a}\La\Pi}=\si^m_{\ \,a\dot{a}}\,R^a_{\ \La\Pi},
\quad R^{\dot{a}}_{\ m\La\Pi}=R^{\dot{a}}_{\ m\La\Pi}=0, \\[8pt]
R^a_{\ b\La\Pi}=\frac{1}{2}R^m_{\ \,n\La\Pi}\,\si_{m\ \ b}^{\ \;na},  \quad R^{\dot{a}}_{\ \dot{b}\La\Pi}=\frac{1}{2}R^m_{\ \,n\La\Pi}\,\si_{m\ \ \dot{b}}^{\ \;n\dot{a}}.  }{0}\coordE{}\end{gather}
The Bianchi identities (\ref{bian1})  and (\ref{bian2}) remain identities for these new curvatures.

We identify the bosonic sector of superspace with space-time. Note that this is the largest sector of superspace for which a metric ``survives'' the \myHighlight{$k\to\infty$}\coordHE{} limit: the matrix \myHighlight{$^A\!E_\La$}\coordHE{} does not have an inverse when \myHighlight{$k\to\infty$}\coordHE{} since its body doesn't~\cite{dew} (see (\ref{ecurv})) but we retain a tetrad \myHighlight{$^m\!E_\mu$}\coordHE{} in space-time. We must then take account of the fact that physical fields show no dependence on the \myHighlight{$a$}\coordHE{}-type coordinates. Accordingly, the fields \myHighlight{$^A\!E_\La$}\coordHE{} and \myHighlight{$A^m_{\ \,n\La}$}\coordHE{} must satisfy
\[\coord{}\boxMath{
^A\!E_\La=\,^A\!E_\La\!\!\mid , \quad A^m_{\ \,n\La}= A^m_{\ \,n\La}\!\!\mid,
}{corchetes}{0pt}\coordE{}\]
where \myHighlight{$\mid$}\coordHE{} means ``\myHighlight{$Z^\al=Z^{\dot{\al}}=0$}\coordHE{} and \myHighlight{$k\to\infty$}\coordHE{}''.\footnote{This is in contrast to the standard approach of gauge completion in which \myHighlight{$^A\!E_\La$}\coordHE{} has a complete expansion in the \myHighlight{$a$}\coordHE{}-type coordinates and space-time is {\it embedded} in the bosonic sector~\cite{nat2,bri,wes}.} 
Then by a partial gauge fixing (see (\ref{ecurv}) and (\ref{etransa})--(\ref{atransa})) we get a physical field content of
\begin{gather}\coord{}\boxAlignEqnarray{
\leftCoord{}^A\!E_\La=\left(\begin{array}{ccc} e^m_{\rightCoord{}\ \,\mu} & 0 & 0 \rightCoord{}\\
\leftCoord{}\frac{\leftCoord{}1}{\rightCoord{}2}\psi^a_{\rightCoord{}\ \mu} & 0 & 0 \rightCoord{}\\
\leftCoord{}\frac{\leftCoord{}1}{\rightCoord{}2}\psi^{\dot{a}}_{\rightCoord{}\ \mu} & 0 & 0 \end{array}\right), \label{physe}\rightCoord{}\\
\leftCoord{}A^m_{\rightCoord{}\ \,n\mu}=:\Ga^m_{\rightCoord{}\ \,n\mu}(Z^\mu),    \qquad A^m_{\rightCoord{}\ \,n\al}=A^m_{\rightCoord{}\ \,n\dot{\al}}=0. \label{physA=0}
}{0mm}{6}{13}{
^A\!E_\La=\left(\begin{array}{ccc} e^m_{\ \,\mu} & 0 & 0 \\
\frac{1}{2}\psi^a_{\ \mu} & 0 & 0 \\
\frac{1}{2}\psi^{\dot{a}}_{\ \mu} & 0 & 0 \end{array}\right), \\
A^m_{\ \,n\mu}=:\Ga^m_{\ \,n\mu}(Z^\mu),    \qquad A^m_{\ \,n\al}=A^m_{\ \,n\dot{\al}}=0. }{0}\coordE{}\end{gather}
Note from (\ref{etransa})--(\ref{atransa}) that the transformation of the physical fields is determined solely by
\begin{gather}\coord{}\boxAlignEqnarray{
\leftCoord{}\Xi^m\!\!\mid\;=:\ep^m, \quad\Xi^a\!\!\mid\;=:-\xi^a, \quad \Xi^{\dot{a}}\!\!\mid\;=:-\xi^{\dot{a}},  \label{par1}\rightCoord{}\\
\leftCoord{}^m\!\La_n\!\!\mid\;=:\la^m_{\rightCoord{}\ \,n}, \label{par2}
}{0mm}{2}{3}{
\Xi^m\!\!\mid\;=:\ep^m, \quad\Xi^a\!\!\mid\;=:-\xi^a, \quad \Xi^{\dot{a}}\!\!\mid\;=:-\xi^{\dot{a}},  \\
^m\!\La_n\!\!\mid\;=:\la^m_{\ \,n}, }{0}\coordE{}\end{gather}
so that (\ref{par1})--(\ref{par2}) are the physically relevant gauge parameters. The transformations (\ref{etransa})--(\ref{atransa}) are now
\begin{eqnarray}\coord{}\boxAlignEqnarray{\leftCoord{}
\de\rightCoord{}\,e^m_{\ \rightCoord{}\,\mu}&=&\ep^m_{\ \rightCoord{}\, ,\mu}+\ep^n\rightCoord{}\,\Ga^m_{\ \rightCoord{}\,n\mu}+\si^m_{\ \rightCoord{}\,a\dot{a}}(\xi^a\psi^{\dot{a}}_{\ \mu}+\xi^{\dot{a}}\psi^a_{\ \mu})-\la^m_{\ \rightCoord{}\,n}\rightCoord{}\,e^n_{\ \mu}, \rightCoord{}\label{sg1} \rightCoord{}\\\leftCoord{}
\de\rightCoord{}\,\psi^a_{\ \mu}&=&-2D_\mu\rightCoord{}\,\xi^a-\sfrac{1}{2}\la^m_{\ \rightCoord{}\,n}\rightCoord{}\,\si_{m\ \ b}^{\ \;na}\rightCoord{}\,\psi^b_{\ \mu}, \rightCoord{}\label{sg2} \rightCoord{}\\\leftCoord{}
\de\rightCoord{}\,\psi^{\dot{a}}_{\ \mu}&=&-2D_\mu\rightCoord{}\,\xi^{\dot{a}}-\sfrac{1}{2}\la^m_{\ \rightCoord{}\,n}\rightCoord{}\,\si_{m\ \ \dot{b}}^{\ \;n\dot{a}}\rightCoord{}\,\psi^{\dot{b}}_{\ \mu}, \rightCoord{}\label{sg3} 
\rightCoord{}}{0mm}{3}{25}{
\de\,e^m_{\ \,\mu}&=&\ep^m_{\ \, ,\mu}+\ep^n\,\Ga^m_{\ \,n\mu}+\si^m_{\ \,a\dot{a}}(\xi^a\psi^{\dot{a}}_{\ \mu}+\xi^{\dot{a}}\psi^a_{\ \mu})-\la^m_{\ \,n}\,e^n_{\ \mu}, \\
\de\,\psi^a_{\ \mu}&=&-2D_\mu\,\xi^a-\sfrac{1}{2}\la^m_{\ \,n}\,\si_{m\ \ b}^{\ \;na}\,\psi^b_{\ \mu}, \\
\de\,\psi^{\dot{a}}_{\ \mu}&=&-2D_\mu\,\xi^{\dot{a}}-\sfrac{1}{2}\la^m_{\ \,n}\,\si_{m\ \ \dot{b}}^{\ \;n\dot{a}}\,\psi^{\dot{b}}_{\ \mu}, }{1}\coordE{}\end{eqnarray}
where \myHighlight{$D_\mu$}\coordHE{} is the familiar space-time covariant derivative with connection \myHighlight{$\Ga^m_{\ \,n\mu}$}\coordHE{} that acts only on tetrad and spinor indices. Eqns.\ (\ref{sg1})--(\ref{sg3}) are the super Poincar\'{e} gauge transformations of supergravity~\cite{cha,sal}. Using (\ref{physe})--(\ref{physA=0}) in  (\ref{Tora})--(\ref{Curc}) we see that the only non-vanishing parts of the curvatures are
\begin{gather}\coord{}\boxAlignEqnarray{
\leftCoord{}R^m_{\rightCoord{}\ \,\mu\nu}\!\!\mid\;=D_\mu\,e^m_{\rightCoord{}\ \,\nu}-D_\nu\,e^m_{\rightCoord{}\ \,\mu}+\sfrac{1}{2}\si^m_{\rightCoord{}\ \,a\dot{a}}(\psi^{\dot{a}}_{\rightCoord{}\ \mu}\,\psi^a_{\rightCoord{}\ \nu}+\psi^a_{\rightCoord{}\ \mu}\psi^{\dot{a}}_{\rightCoord{}\ \nu})  \label{sgcur1}\rightCoord{}\\
\leftCoord{}R^a_{\rightCoord{}\ \mu\nu}\!\!\mid\,=D_{\rightCoord{}[\mu}\psi^a_{\rightCoord{}\ \nu]}, \qquad R^{\dot{a}}_{\rightCoord{}\ \mu\nu}\!\!\mid\,=D_{\rightCoord{}[\mu}\psi^{\dot{a}}_{\rightCoord{}\ \nu]}, \label{sgcur2}\rightCoord{}\\
\leftCoord{}R^m_{\rightCoord{}\ \,n\mu\nu}\!\!\mid\,=\mathcal{R}^m_{\rightCoord{}\ \,n\mu\nu}, \quad R^m_{\rightCoord{}\ \,a\mu\nu}\!\!\mid\,=\si^m_{\rightCoord{}\ \,a\dot{a}}\,R^{\dot{a}}_{\rightCoord{}\ \mu\nu}\!\!\mid, \quad R^m_{\rightCoord{}\ \,\dot{a}\mu\nu}\!\!\mid\,=\si^m_{\rightCoord{}\ \,a\dot{a}}\,R^a_{\rightCoord{}\ \mu\nu}\!\!\mid, \label{sgcur3}\rightCoord{}\\
\leftCoord{}R^a_{\rightCoord{}\ b\mu\nu}\!\!\mid\,=\frac{\leftCoord{}1}{\rightCoord{}2}R^m_{\rightCoord{}\ \,n\mu\nu}\,\si_{\rightCoord{}m\ \ b}^{\ \;na},  \quad R^{\dot{a}}_{\rightCoord{}\ \dot{b}\mu\nu}=\frac{\leftCoord{}1}{\rightCoord{}2}R^m_{\rightCoord{}\ \,n\mu\nu}\,\si_{\rightCoord{}m\ \ \dot{b}}^{\ \;n\dot{a}},  \label{sgcur4}
}{0mm}{6}{34}{
R^m_{\ \,\mu\nu}\!\!\mid\;=D_\mu\,e^m_{\ \,\nu}-D_\nu\,e^m_{\ \,\mu}+\sfrac{1}{2}\si^m_{\ \,a\dot{a}}(\psi^{\dot{a}}_{\ \mu}\,\psi^a_{\ \nu}+\psi^a_{\ \mu}\psi^{\dot{a}}_{\ \nu})  \\
R^a_{\ \mu\nu}\!\!\mid\,=D_{[\mu}\psi^a_{\ \nu]}, \qquad R^{\dot{a}}_{\ \mu\nu}\!\!\mid\,=D_{[\mu}\psi^{\dot{a}}_{\ \nu]}, \\
R^m_{\ \,n\mu\nu}\!\!\mid\,=\mathcal{R}^m_{\ \,n\mu\nu}, \quad R^m_{\ \,a\mu\nu}\!\!\mid\,=\si^m_{\ \,a\dot{a}}\,R^{\dot{a}}_{\ \mu\nu}\!\!\mid, \quad R^m_{\ \,\dot{a}\mu\nu}\!\!\mid\,=\si^m_{\ \,a\dot{a}}\,R^a_{\ \mu\nu}\!\!\mid, \\
R^a_{\ b\mu\nu}\!\!\mid\,=\frac{1}{2}R^m_{\ \,n\mu\nu}\,\si_{m\ \ b}^{\ \;na},  \quad R^{\dot{a}}_{\ \dot{b}\mu\nu}=\frac{1}{2}R^m_{\ \,n\mu\nu}\,\si_{m\ \ \dot{b}}^{\ \;n\dot{a}},  }{0}\coordE{}\end{gather}
where \myHighlight{$\mathcal{R}^m_{\ \,n\mu\nu}$}\coordHE{} is the space-time Riemann tensor. The Bianchi identities (\ref{bian1}) and (\ref{bian2}) are still satisfied by (\ref{sgcur1})--(\ref{sgcur4}) and (\ref{physe}).

The field equations of the theory are to be constructed from the curvatures and must be on-shell (see below) covariant under the super Poincar\'{e} gauge transformations (\ref{sg1})--(\ref{sg3}). Note that we would not obtain suitable field equations by the procedure of imposing our constraints and limit \myHighlight{$k\to\infty$}\coordHE{} on the field equations of IOSp\myHighlight{$(3,1|4)$}\coordHE{} gauge theory: the latter can be obtained from an (on-shell) invariant action by varying {\it independently} the fields \myHighlight{$^A\!E_\La$}\coordHE{} and \myHighlight{$A^A_{\ B\La}$}\coordHE{}, but this is inconsistent with the constraints (though not with the limit \myHighlight{$k\to\infty$}\coordHE{}) so that this procedure would not result in super-Poincar\'{e}-covariant equations. The appropriate field equations are
\begin{gather}\coord{}\boxAlignEqnarray{
\leftCoord{}R^m_{\rightCoord{}\ \,\mu\nu}\!\!\mid\,=0, \label{eqn1}\rightCoord{}\\
\leftCoord{}R^m_{\rightCoord{}\ \,\dot{a}\mu\nu}\!\!\mid e^\mu_{\rightCoord{}\ m}=0, \label{eqn2}\rightCoord{}\\
\leftCoord{}\mathcal{R}_{\rightCoord{}n\nu}\,e^n_{\rightCoord{}\ \mu}\,e^{\nu m}-R^m_{\rightCoord{}\ \,\dot{a}\mu\nu}\!\!\mid\psi^{\dot{a}\nu}-R^m_{\rightCoord{}\ \,a\mu\nu}\!\!\mid\psi^{a\nu}=0, \label{eqn3}
}{0mm}{3}{10}{
R^m_{\ \,\mu\nu}\!\!\mid\,=0, \\
R^m_{\ \,\dot{a}\mu\nu}\!\!\mid e^\mu_{\ m}=0, \\
\mathcal{R}_{n\nu}\,e^n_{\ \mu}\,e^{\nu m}-R^m_{\ \,\dot{a}\mu\nu}\!\!\mid\psi^{\dot{a}\nu}-R^m_{\ \,a\mu\nu}\!\!\mid\psi^{a\nu}=0, }{0}\coordE{}\end{gather}
where \myHighlight{$\mathcal{R}_{n\nu}=R^m_{\ \,n\mu\nu}\!\!\mid e^\mu_{\ m}$}\coordHE{} is the space-time Ricci tensor. From (\ref{sgcur1})--(\ref{sgcur3}) we see that (\ref{eqn1}) determines the space-time torsion \myHighlight{$\mathcal{T}^m_{\ \,\mu\nu}$}\coordHE{},
\[\coord{}\boxMath{
\mathcal{T}^m_{\ \,\mu\nu}=\sfrac{1}{2}\si^m_{\ \,a\dot{a}}(\psi^a_{\ \mu}\,\psi^{\dot{a}}_{\ \nu}+\psi^{\dot{a}}_{\ \mu}\psi^a_{\ \nu}),
}{corchetes}{0pt}\coordE{}\]
(\ref{eqn2}) is the gravitino field equation
\[\coord{}\boxMath{
\si^m_{\ \,a\dot{a}}\,D_{[\mu}\,\psi^a_{\ \nu]}\,e^\mu_{\ m}=0
}{corchetes}{0pt}\coordE{}\]
and (\ref{eqn3}) is the tetrad field equation
\[\coord{}\boxMath{
\mathcal{R}_{\mu}^{\ m}+\si^m_{\ \,a\dot{a}}(\psi^{\dot{a}\nu}\,D_{[\mu}\,\psi^a_{\ \nu]}+\psi^{a\nu}\,D_{[\mu}\,\psi^{\dot{a}}_{\ \nu]})=0.
}{corchetes}{0pt}\coordE{}\]
As discussed in~\cite{sal} and references therein, (\ref{eqn2}) and (\ref{eqn3}) are covariant under (\ref{sg1})--(\ref{sg3}) when (\ref{eqn1}) is satisfied, i.e.\ when \myHighlight{$\Ga^m_{\ \,n\mu}$}\coordHE{} is on-shell.

The approach presented here has been not to view supergravity as a naturally supergeometrical theory, but to show by physically motivated reasoning how  it is related to the natural geometry of a supermanifold. A comparable theory is Newtonian gravity, which also has a rather untidy geometrical formulation, in terms of space-time~\cite{mis}. Indeed, Newtonian gravity in its space-time form can be derived from general relativity, formulated as Poincar\'{e} gauge theory, by a method analogous to that used here~\cite{phi}. In place of (\ref{coord}) and (\ref{met}) one has Minkowski coordinates with \myHighlight{$x^0=t$}\coordHE{}, not \myHighlight{$x^0=ct$}\coordHE{}, so that \myHighlight{$\eta_{00}=-c^2$}\coordHE{}. Instead of (\ref{iosp}) and (\ref{sp}) one has infinitesimal Poincar\'{e} and Galilean transformations; one obtains the latter from the former by taking the analogous limit \myHighlight{$c\to\infty$}\coordHE{}, but here no constraints are required. One then imposes a similar limiting procedure on the Lorentz gauge potential of Poincar\'{e} gauge theory. In curved space-time one takes a local frame with \myHighlight{$\eta_{\widehat{0}\widehat{0}}=-c^2$}\coordHE{} and coordinates with \myHighlight{$[x^\mu]=-1$}\coordHE{}; a 3-space metric, but not a space-time metric, is preserved when \myHighlight{$c\to\infty$}\coordHE{}~\cite{mis}. The crucial difference from our derivation of supergravity is the lack of constraints, which allows one to obtain a suitable Newtonian field equation by taking the \myHighlight{$c\to\infty$}\coordHE{} limit of the Einstein field equation.
\vspace{5mm}

I wish to thank D.\ Hurley and M.\ Vandyck for helpful discussions.

\begin{thebibliography}{99}
\bibitem{nat1} P.\ Nath and R.\ Arnowitt, Phys. Lett. 56B (1975) 177.
\bibitem{nat2} P.\ Nath and R.\ Arnowitt, Phys. Lett. 65B (1976) 73; Nucl. Phys. B165 (1980) 462.
\bibitem{wes} J.\ Wess and B.\ Zumino, Phys. Lett. 66B (1977) 361; 74B (1978) 51; 79B (1978) 394; R.\ Grimm, J.\ Wess and B.\ Zumino, Nucl. Phys. B152 (1979) 255.
\bibitem{dew} B.\ deWitt, {\it Supermanifolds}, Second Edition, (Cambridge University Press, 1992). 
\bibitem{wes2} J.\ Wess and J.\ Bagger, {\it Supersymmetry and Supergravity}, Second Edition (Princeton University Press, 1992).
\bibitem{gri} G.\ Grigani and G.\ Nardelli, Phys. Rev. D45 (1992) 2719. 
\bibitem{sal} P.\ Salgado, M.\ Cataldo and S.\ del Campo, Phys. Rev. D65 (2002) 084032 (gr-qc/0110097).
\bibitem{kob} S.\ Kobayashi and K.\ Nomizu, {\it Foundations of Differential Geometry}, Vol.\ 1, (Interscience, 1963).
\bibitem{bri} L.\ Brink {\it et al.}, Phys. Lett. 74B (1978) 336.
\bibitem{cha} A.H.\ Chamseddine and P.C.\ West, Nucl. Phys. B129 (1977) 39.
\bibitem{mis} C.W.\ Misner, K.S.\ Thorne and J.A.\ Wheeler, {\it Gravitation} (Freeman, 1973), Chapt.~12.
\bibitem{phi} T.G.\ Philbin, unpublished.
\end{thebibliography}




\end{document}\bye
