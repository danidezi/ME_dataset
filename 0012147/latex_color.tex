
\documentclass[a4paper,11pt]{article}
\pagestyle{plain}
\textwidth=16truecm
\textheight=23.5truecm
\topmargin-1.5cm
\hoffset-1.2cm
\baselineskip=24pt 
%
\usepackage{useful_macros}
\begin{document}



\title{On the evaluation of the evolution operator \myHighlight{$Z_{\rm
Reg}(R_2,R_1)$}\coordHE{} in the Diakonov--Petrov approach to the Wilson
loop}

\author{M. Faber\thanks{E--mail: faber@kph.tuwien.ac.at, Tel.:
+43--1--58801--14261, Fax: +43--1--58801--14299} ,
A. N. Ivanov\thanks{E--mail: ivanov@kph.tuwien.ac.at, Tel.:
+43--1--58801--14261, Fax: +43--1--58801--14299}~\myHighlight{$^{\ddagger}$}\coordHE{} , N. I.
Troitskaya\thanks{Permanent Address: State
Technical University, Department of Nuclear Physics, 195251
St. Petersburg, Russian Federation}}

\date{\today}

\maketitle

\begin{center}
{\it Institut f\"ur Kernphysik, Technische Universit\"at Wien, \\ 
Wiedner Hauptstr. 8--10, A--1040 Vienna, Austria}
\end{center}

\begin{abstract}
We evaluate the evolution operator \myHighlight{$Z_{\rm Reg}(R_2,R_1)$}\coordHE{} introduced
by Diakonov and Petrov for the definition of the Wilson loop in terms
of a path integral over gauge degrees of freedom. We use the procedure
suggested by Diakonov and Petrov (Phys. Lett. B224 (1989) 131) and
show that the evolution operator vanishes.
\end{abstract}

\begin{center}
PACS: 11.10.--z, 11.15.--q, 12.38.--t, 12.38.Aw, 12.90.+b\\ 
Keywords: non--Abelian gauge theory, confinement
\end{center}



\section*{Path integral for the evolution operator \myHighlight{$Z_{\rm Reg}(R_2,R_1)$}\coordHE{} } 
\setcounter{equation}{0}

\hspace{0.2in} In Ref.[1] for the representation of the Wilson loop in
terms of the path integral over gauge degrees of freedom Diakonov and
Petrov used the functional \myHighlight{$Z(R_2,R_1)$}\coordHE{} defined by (see Eq.(8) of
Ref.[1])
%
\begin{eqnarray}\coord{}\boxAlignEqnarray{\leftCoord{}\rightCoord{}\label{label1}
Z(R_2,R_1) =
\int\limits^{R_2}_{R_1}DR(t)\rightCoord{}\,\exp\Bigg(iT\int\limits^{t_2}_{t_1}\rightCoord{}\,{\rm
Tr}\rightCoord{}\,(iR\rightCoord{}\,\dot{R}\rightCoord{}\,\tau_3)\Bigg),
\rightCoord{}}{0mm}{1}{8}{Z(R_2,R_1) =
\int\limits^{R_2}_{R_1}DR(t)\,\exp\Bigg(iT\int\limits^{t_2}_{t_1}\,{\rm
Tr}\,(iR\,\dot{R}\,\tau_3)\Bigg),
}{1}\coordE{}\end{eqnarray}
%
where \myHighlight{$\dot{R} = dR/dt$}\coordHE{} and \myHighlight{$T = 1/2,1,3/2,\ldots$}\coordHE{} is the colour
isospin quantum number. According to Diakonov and Petrov \myHighlight{$Z(R_2,R_1)$}\coordHE{}
should be regularized by the analogy to an axial--symmetric top. The
regularized expression of \myHighlight{$Z(R_2,R_1)$}\coordHE{} has been determined in Eq.(9)
of Ref.[1] and reads
%
\begin{eqnarray}\coord{}\boxAlignEqnarray{\leftCoord{}\rightCoord{}\label{label2}
Z_{\rm Reg}(R_2,R_1) =
\int\limits^{R_2}_{R_1}DR(t)\rightCoord{}\,\exp\Bigg(i\int\limits^{t_2}_{t_1}
\Big[\frac{\leftCoord{}1}{\rightCoord{}2}\rightCoord{}\,I_{\perp}\rightCoord{}\,(\Omega^2_1 + \Omega^2_2) +
\frac{\leftCoord{}1}{\rightCoord{}2}\rightCoord{}\,I_{\parallel}\rightCoord{}\,\Omega^2_3 + T\rightCoord{}\,\Omega_3\Big]\Bigg),
\rightCoord{}}{0mm}{3}{11}{Z_{\rm Reg}(R_2,R_1) =
\int\limits^{R_2}_{R_1}DR(t)\,\exp\Bigg(i\int\limits^{t_2}_{t_1}
\Big[\frac{1}{2}\,I_{\perp}\,(\Omega^2_1 + \Omega^2_2) +
\frac{1}{2}\,I_{\parallel}\,\Omega^2_3 + T\,\Omega_3\Big]\Bigg),
}{1}\coordE{}\end{eqnarray}
%
where \myHighlight{$\Omega_a = i\,{\rm Tr}(R\,\dot{R}\,\tau_a)$}\coordHE{} are angular
velocities of the top, \myHighlight{$\tau_a$}\coordHE{} are Pauli matrices \myHighlight{$a=1,2,3$}\coordHE{},
\myHighlight{$I_{\perp}$}\coordHE{} and \myHighlight{$I_{\parallel}$}\coordHE{} are the moments of inertia of the top
which should be taken to zero. According to the prescription of
Ref.[1] one should take first the limit \myHighlight{$I_{\parallel} \to 0$}\coordHE{} and then
\myHighlight{$I_{\perp} \to 0$}\coordHE{}. For the confirmation of the result, given in
Eq.(13) of Ref.[1],
%
\begin{eqnarray}\coord{}\boxAlignEqnarray{\leftCoord{}\rightCoord{}\label{label3}
Z_{\rm Reg}(R_2,R_1) = (2T + 1)\rightCoord{}\,D^T_{TT}(R_2R^{\dagger}_1) = (2T +
\leftCoord{}1)\rightCoord{}\,D^T_{-T-T}(R_1R^{\dagger}_2),
\rightCoord{}}{0mm}{2}{5}{Z_{\rm Reg}(R_2,R_1) = (2T + 1)\,D^T_{TT}(R_2R^{\dagger}_1) = (2T +
1)\,D^T_{-T-T}(R_1R^{\dagger}_2),
}{1}\coordE{}\end{eqnarray}
%
where \myHighlight{$D^T(U)$}\coordHE{} is a Wigner rotational matrix in the representation
\myHighlight{$T$}\coordHE{}, Diakonov and Petrov suggested to evaluate the evolution operator
(\ref{label2}) explicitly via the discretization of the path integral
over \myHighlight{$R$}\coordHE{}.  The discretized form of the path integral Eq.(\ref{label2})
is given by Eq.(14) of Ref.[1] and reads
%
\begin{eqnarray}\coord{}\boxAlignEqnarray{\leftCoord{}\rightCoord{}\label{label4}
\hspace{-0.5in}&&Z_{\rm Reg}(R_{N+1},R_0) = \lim_{\begin{array}{c} N\to
\infty\\\leftCoord{} \delta \to 0\end{array}}{\cal N}\int\prod^{N}_{n=1}dR_n\nonumber\rightCoord{}\\\leftCoord{} 
\hspace{-0.5in}&&\times\rightCoord{}\,\exp\Bigg[\sum^{N}_{n=0}
\Bigg(-\rightCoord{}\,i\rightCoord{}\,\frac{\leftCoord{}I_{\perp}}{\rightCoord{}2\delta}\rightCoord{}\,\Big[({\rm Tr}\rightCoord{}\,V_n\tau_1)^2 +
\leftCoord{}({\rm Tr}\rightCoord{}\,V_n\tau_2)^2\Big] - \rightCoord{}\,i\rightCoord{}\,\frac{\leftCoord{}I_{\parallel}}{\rightCoord{}2\delta}\rightCoord{}\,({\rm
Tr}\rightCoord{}\,V_n\tau_3)^2 -\rightCoord{}\,T\rightCoord{}\,({\rm Tr}\rightCoord{}\,V_n\tau_3)\Bigg)\Bigg],
\rightCoord{}}{0.5in}{6}{19}{\hspace{-0.5in}&&Z_{\rm Reg}(R_{N+1},R_0) = \lim_{\begin{array}{c} N\to
\infty\\ \delta \to 0\end{array}}{\cal N}\int\prod^{N}_{n=1}dR_n\\ 
\hspace{-0.5in}&&\times\,\exp\Bigg[\sum^{N}_{n=0}
\Bigg(-\,i\,\frac{I_{\perp}}{2\delta}\,\Big[({\rm Tr}\,V_n\tau_1)^2 +
({\rm Tr}\,V_n\tau_2)^2\Big] - \,i\,\frac{I_{\parallel}}{2\delta}\,({\rm
Tr}\,V_n\tau_3)^2 -\,T\,({\rm Tr}\,V_n\tau_3)\Bigg)\Bigg],
}{1}\coordE{}\end{eqnarray}
%
where \myHighlight{$R_n = R(s_n)$}\coordHE{} with \myHighlight{$s_n = t_1 + n\,\delta$}\coordHE{} and \myHighlight{$V_n = R_n
R^{\dagger}_{n+1}$}\coordHE{} are the relative orientations of the top at
neighbouring points [1]. The normalization factor \myHighlight{${\cal N}$}\coordHE{} is
determined by
%
\begin{eqnarray}\coord{}\boxAlignEqnarray{\leftCoord{}\rightCoord{}\label{label5}
{\rightCoord{}\leftCoord{}\cal N} = \Bigg(\frac{\leftCoord{}I_{\perp}}{\rightCoord{}2\pi i
\delta}\sqrt{\frac{\leftCoord{}I_{\parallel}}{\rightCoord{}2\pi i \delta}}\rightCoord{}\,\Bigg)^{N+1}.\rightCoord{}
\rightCoord{}}{0mm}{4}{8}{{\cal N} = \Bigg(\frac{I_{\perp}}{2\pi i
\delta}\sqrt{\frac{I_{\parallel}}{2\pi i \delta}}\,\Bigg)^{N+1}.
}{1}\coordE{}\end{eqnarray}
%
(see Eq.(19) of Ref.[1]). Following the prescription of Ref.[1] one
should take the limits \myHighlight{$\delta \to 0$}\coordHE{} and \myHighlight{$I_{\parallel}, I_{\perp}
\to 0$}\coordHE{} but keeping the ratios \myHighlight{$I_i/\delta$}\coordHE{}, where (\myHighlight{$ i= {\parallel},
{\perp}$}\coordHE{}), much greater than unity, \myHighlight{$I_i/\delta \gg 1$}\coordHE{}.

The main point of the evaluation of the path integral is to show that
the evolution operator \myHighlight{$Z_{\rm Reg}(R_2,R_1)$}\coordHE{} given by the path
integral (\ref{label2}) reduces to the representation in the form of
{\it a sum over possible intermediate states}, i.e. eigenfunctions of
the axial--symmetric top [1]
%
\begin{eqnarray}\coord{}\boxAlignEqnarray{\leftCoord{}\rightCoord{}\label{label6}
Z_{\rm Reg}(R_2,R_1) =\sum^{\infty}_{J=0}\sum^J_{m = - J}(2J + 1)\rightCoord{}\,
D^J_{m m}(R_2R^{\dagger}_1)\rightCoord{}\,e^{\textstyle -i(t_2-t_1)\rightCoord{}\,E_{J m}},
\rightCoord{}}{0mm}{1}{6}{Z_{\rm Reg}(R_2,R_1) =\sum^{\infty}_{J=0}\sum^J_{m = - J}(2J + 1)\,
D^J_{m m}(R_2R^{\dagger}_1)\,e^{\textstyle -i(t_2-t_1)\,E_{J m}},
}{1}\coordE{}\end{eqnarray}
%
(see Eq.(12) of Ref.[1]), where \myHighlight{$E_{J m}$}\coordHE{} are the eigenvalues of the
Hamiltonian of the axial--symmetric top 
%
\begin{eqnarray}\coord{}\boxAlignEqnarray{\leftCoord{}\rightCoord{}\label{label7}
E_{J m} = \frac{\leftCoord{}J(J+1) - m^2}{\rightCoord{}2 I_{\perp}} + \frac{\leftCoord{}(m - T)^2}{\rightCoord{}2
I_{\parallel}}
\rightCoord{}}{0mm}{3}{5}{E_{J m} = \frac{J(J+1) - m^2}{2 I_{\perp}} + \frac{(m - T)^2}{2
I_{\parallel}}
}{1}\coordE{}\end{eqnarray}
%
(see Eq.(11) of Ref.[1]).

According to Diakonov\myHighlight{$^{\prime}$}\coordHE{}s and Petrov\myHighlight{$^{\prime}$}\coordHE{}s statement the
integral has a saddle--point at \myHighlight{$V_n \simeq 1$}\coordHE{}. For the calculation of
the integral around the saddle--point Diakonov and Petrov suggested
the following procedure. Let us denote the exponent of
Eq.(\ref{label4}) as
%
\begin{eqnarray}\coord{}\boxAlignEqnarray{\leftCoord{}\rightCoord{}\label{label8}
f[V_n] = -\rightCoord{}\,i\rightCoord{}\,\frac{\leftCoord{}I_{\perp}}{\rightCoord{}2\delta}\rightCoord{}\,\Big[({\rm Tr}\rightCoord{}\,V_n\tau_1)^2 +
\leftCoord{}({\rm Tr}\rightCoord{}\,V_n\tau_2)^2\Big] - \rightCoord{}\,i\rightCoord{}\,\frac{\leftCoord{}I_{\parallel}}{\rightCoord{}2\delta}\rightCoord{}\,({\rm
Tr}\rightCoord{}\,V_n\tau_3)^2 -\rightCoord{}\,T\rightCoord{}\,({\rm Tr}\rightCoord{}\,V_n\tau_3) 
\rightCoord{}}{0mm}{4}{17}{f[V_n] = -\,i\,\frac{I_{\perp}}{2\delta}\,\Big[({\rm Tr}\,V_n\tau_1)^2 +
({\rm Tr}\,V_n\tau_2)^2\Big] - \,i\,\frac{I_{\parallel}}{2\delta}\,({\rm
Tr}\,V_n\tau_3)^2 -\,T\,({\rm Tr}\,V_n\tau_3) 
}{1}\coordE{}\end{eqnarray}
%
and represent the exponential in the following form
%
\begin{eqnarray}\coord{}\boxAlignEqnarray{\leftCoord{}\rightCoord{}\label{label9}
e^{\textstyle f[V_n]} =
\sum^{\infty}_{J=0}\sum^{J}_{p=-J}\sum^{J}_{q=-J}(2J +
\leftCoord{}1)\lambda^J_{pq}D^J_{pq}(V_n).\rightCoord{}
\rightCoord{}}{0mm}{2}{4}{e^{\textstyle f[V_n]} =
\sum^{\infty}_{J=0}\sum^{J}_{p=-J}\sum^{J}_{q=-J}(2J +
1)\lambda^J_{pq}D^J_{pq}(V_n).
}{1}\coordE{}\end{eqnarray}
%
The coefficients \myHighlight{$\lambda^J_{pq}$}\coordHE{} are given by
%
\begin{eqnarray}\coord{}\boxAlignEqnarray{\leftCoord{}\rightCoord{}\label{label10}
\lambda^J_{pq} = \int dU_n\rightCoord{}\,D^J_{qp}(U^{\dagger}_n)\rightCoord{}\,e^{\textstyle f[U_n]}.\rightCoord{}
\rightCoord{}}{0mm}{1}{6}{\lambda^J_{pq} = \int dU_n\,D^J_{qp}(U^{\dagger}_n)\,e^{\textstyle f[U_n]}.
}{1}\coordE{}\end{eqnarray}
%
Substituting Eq.(\ref{label10}) in Eq.(\ref{label9}) we get the identity
%
\begin{eqnarray}\coord{}\boxAlignEqnarray{\leftCoord{}\rightCoord{}\label{label11}
e^{\textstyle f[V_n]} =
\sum^{\infty}_{J=0}\sum^{J}_{p=-J}\sum^{J}_{q=-J}(2J +
\leftCoord{}1)\rightCoord{}\,D^J_{pq}(V_n)\rightCoord{}\, \int dU_n\rightCoord{}\,D^J_{qp}(U^{\dagger}_n)\rightCoord{}\,e^{\textstyle
f[U_n]}.\rightCoord{}
\rightCoord{}}{0mm}{2}{8}{e^{\textstyle f[V_n]} =
\sum^{\infty}_{J=0}\sum^{J}_{p=-J}\sum^{J}_{q=-J}(2J +
1)\,D^J_{pq}(V_n)\, \int dU_n\,D^J_{qp}(U^{\dagger}_n)\,e^{\textstyle
f[U_n]}.
}{1}\coordE{}\end{eqnarray}
%
Let us show that Eq.(\ref{label11}) is the identity. For this aim we
have to use the relation
%
\begin{eqnarray}\coord{}\boxAlignEqnarray{\leftCoord{}\rightCoord{}\label{label12}
\sum^{\infty}_{J=0}\sum^{J}_{p=-J}\sum^{J}_{q=-J}(2J + 1)\rightCoord{}\,
D^J_{pq}(V_n)\rightCoord{}\,D^J_{qp}(U^{\dagger}_n) =
\sum^{\infty}_{J=0}(2J+1)\rightCoord{}\,\chi_J[V_nU^{\dagger}_n].\rightCoord{}
\rightCoord{}}{0mm}{1}{7}{\sum^{\infty}_{J=0}\sum^{J}_{p=-J}\sum^{J}_{q=-J}(2J + 1)\,
D^J_{pq}(V_n)\,D^J_{qp}(U^{\dagger}_n) =
\sum^{\infty}_{J=0}(2J+1)\,\chi_J[V_nU^{\dagger}_n].
}{1}\coordE{}\end{eqnarray}
%
By using Eq.(\ref{label12}) the r.h.s. of Eq.(\ref{label11}) reads
%
\begin{eqnarray}\coord{}\boxAlignEqnarray{\leftCoord{}\rightCoord{}\label{label13}
\int dU_n\rightCoord{}\,e^{\textstyle
f[U_n]}\rightCoord{}\,\sum^{\infty}_{J=0}(2J+1)\rightCoord{}\,\chi_J[V_nU^{\dagger}_n] = \int
dU_n\rightCoord{}\,e^{\textstyle f[U_n]}\rightCoord{}\,\delta(V_nU^{\dagger}_n) = e^{\textstyle
f[V_n]},
\rightCoord{}}{0mm}{1}{8}{\int dU_n\,e^{\textstyle
f[U_n]}\,\sum^{\infty}_{J=0}(2J+1)\,\chi_J[V_nU^{\dagger}_n] = \int
dU_n\,e^{\textstyle f[U_n]}\,\delta(V_nU^{\dagger}_n) = e^{\textstyle
f[V_n]},
}{1}\coordE{}\end{eqnarray}
%
where \myHighlight{$\delta(V_nU^{\dagger}_n)$}\coordHE{} is a \myHighlight{$\delta$}\coordHE{}--function defined by 
%
\begin{eqnarray}\coord{}\boxAlignEqnarray{\leftCoord{}\rightCoord{}\label{label14}
\sum^{\infty}_{J=0}(2J+1)\rightCoord{}\,\chi_J[V_nU^{\dagger}_n] =
\delta(V_nU^{\dagger}_n).\rightCoord{}
\rightCoord{}}{0mm}{1}{5}{\sum^{\infty}_{J=0}(2J+1)\,\chi_J[V_nU^{\dagger}_n] =
\delta(V_nU^{\dagger}_n).
}{1}\coordE{}\end{eqnarray}
%
The important consequence of these steps is that \myHighlight{$dU_n$}\coordHE{} as well as
\myHighlight{$dV_n$}\coordHE{} is a standard Haar measure normalized to unity
%
\begin{eqnarray}\coord{}\boxAlignEqnarray{\leftCoord{}\rightCoord{}\label{label15}
\int dU_n = \int dV_n = 1.\rightCoord{}
\rightCoord{}}{0mm}{1}{4}{\int dU_n = \int dV_n = 1.
}{1}\coordE{}\end{eqnarray}
%
This point alters crucially the results of Ref.[1].

Inserting the expansion Eq.(\ref{label11}) in the r.h.s. of
Eq.(\ref{label4}) we obtain
%
\begin{eqnarray}\coord{}\boxAlignEqnarray{\leftCoord{}\rightCoord{}\label{label16}
\hspace{-0.5in}&&Z_{\rm Reg}(R_{N+1},R_0) = \lim_{\begin{array}{c}
N\to \infty\\\leftCoord{} \delta \to 0\end{array}}{\cal N}\int\ldots\int
dR_1\rightCoord{}\,dR_2\ldots dR_{N-1}\rightCoord{}\,dR_N\nonumber\rightCoord{}\\\leftCoord{}
\hspace{-0.5in}&&\times\rightCoord{}\,\sum^{\infty}_{J_0=0}
\sum^{J_0}_{p_0=-J_0}\sum^{J_0}_{q_0=-J_0}(2J_0
\leftCoord{}+ 1)\rightCoord{}\,D^{J_0}_{p_0q_0}(R_0R^{\dagger}_1)\rightCoord{}\, \int
dU_0\rightCoord{}\,D^{J_0}_{q_0p_0}(U^{\dagger}_0)\rightCoord{}\,e^{\textstyle
f[U_0]}\nonumber\rightCoord{}\\\leftCoord{}
\hspace{-0.5in}&&\times\rightCoord{}\,\sum^{\infty}_{J_1=0}
\sum^{J_1}_{p_1=-J_1}\sum^{J_1}_{q_1=-J_1}(2J_1
\leftCoord{}+ 1)\rightCoord{}\,D^{J_1}_{p_1q_1}(R_1R^{\dagger}_2)\rightCoord{}\, \int
dU_1\rightCoord{}\,D^{J_1}_{q_1p_1}(U^{\dagger}_1)\rightCoord{}\,e^{\textstyle
f[U_1]}\nonumber\rightCoord{}\\\leftCoord{}
\hspace{-0.5in}&&\times\rightCoord{}\,\sum^{\infty}_{J_2=0}
\sum^{J_2}_{p_2=-J_2}\sum^{J_2}_{q_2=-J_2}(2J_2 + 1)
\leftCoord{}\rightCoord{}\,D^{J_2}_{p_2q_2}(R_2R^{\dagger}_3)\rightCoord{}\, \int
dU_2\rightCoord{}\,D^{J_2}_{q_2p_2}(U^{\dagger}_2)\rightCoord{}\,e^{\textstyle
f[U_2]}\nonumber\\\leftCoord{} \hspace{-0.5in}&&\times\rightCoord{}\,\ldots\nonumber\rightCoord{}\\\leftCoord{}
\hspace{-0.5in}&&\times\rightCoord{}\,\sum^{\infty}_{J_N=0} \sum^{J_N}_{p_N
\leftCoord{}=-J_N}\sum^{J_N}_{q_N =-J_N}(2J_N +
\leftCoord{}1)\rightCoord{}\,D^{J_N}_{p_Nq_N}(R_NR^{\dagger}_{N+1})\rightCoord{}\, \int
dU_N\rightCoord{}\,D^{J_N}_{q_Np_N}(U^{\dagger}_N)\rightCoord{}\,e^{\textstyle f[U_N]}
\rightCoord{}}{0.5in}{12}{30}{\hspace{-0.5in}&&Z_{\rm Reg}(R_{N+1},R_0) = \lim_{\begin{array}{c}
N\to \infty\\ \delta \to 0\end{array}}{\cal N}\int\ldots\int
dR_1\,dR_2\ldots dR_{N-1}\,dR_N\\
\hspace{-0.5in}&&\times\,\sum^{\infty}_{J_0=0}
\sum^{J_0}_{p_0=-J_0}\sum^{J_0}_{q_0=-J_0}(2J_0
+ 1)\,D^{J_0}_{p_0q_0}(R_0R^{\dagger}_1)\, \int
dU_0\,D^{J_0}_{q_0p_0}(U^{\dagger}_0)\,e^{\textstyle
f[U_0]}\\
\hspace{-0.5in}&&\times\,\sum^{\infty}_{J_1=0}
\sum^{J_1}_{p_1=-J_1}\sum^{J_1}_{q_1=-J_1}(2J_1
+ 1)\,D^{J_1}_{p_1q_1}(R_1R^{\dagger}_2)\, \int
dU_1\,D^{J_1}_{q_1p_1}(U^{\dagger}_1)\,e^{\textstyle
f[U_1]}\\
\hspace{-0.5in}&&\times\,\sum^{\infty}_{J_2=0}
\sum^{J_2}_{p_2=-J_2}\sum^{J_2}_{q_2=-J_2}(2J_2 + 1)
\,D^{J_2}_{p_2q_2}(R_2R^{\dagger}_3)\, \int
dU_2\,D^{J_2}_{q_2p_2}(U^{\dagger}_2)\,e^{\textstyle
f[U_2]}\\ \hspace{-0.5in}&&\times\,\ldots\\
\hspace{-0.5in}&&\times\,\sum^{\infty}_{J_N=0} \sum^{J_N}_{p_N
=-J_N}\sum^{J_N}_{q_N =-J_N}(2J_N +
1)\,D^{J_N}_{p_Nq_N}(R_NR^{\dagger}_{N+1})\, \int
dU_N\,D^{J_N}_{q_Np_N}(U^{\dagger}_N)\,e^{\textstyle f[U_N]}
}{1}\coordE{}\end{eqnarray}
%
Integrating over \myHighlight{$R_n\,(n=1,2,\ldots,N)$}\coordHE{} and using the
orthogonality relation for the group elements we arrive at the
expression
%
\begin{eqnarray}\coord{}\boxAlignEqnarray{\leftCoord{}\rightCoord{}\label{label17}
\hspace{-0.5in}&&Z_{\rm Reg}(R_{N+1},R_0) = \lim_{\begin{array}{c}
N\to \infty\\\leftCoord{} \delta \to 0\end{array}} \rightCoord{}
\sum^{\infty}_{J=0}\sum^{J}_{p=-J}\sum^{J}_{q=-J}(2J +
\leftCoord{}1)\rightCoord{}\,D^J_{pq}(R_0R^{\dagger}_{N+1})\rightCoord{}\,{\cal N}\int
dU_0\rightCoord{}\,D^J_{qp}(U^{\dagger}_0)\rightCoord{}\,e^{\textstyle
f[U_0]}\nonumber\rightCoord{}\\\leftCoord{}
\hspace{-0.5in}&&\times \int
dU_1\rightCoord{}\,D^J_{qp}(U^{\dagger}_1)\rightCoord{}\,e^{\textstyle f[U_1]}\int
dU_2\rightCoord{}\,D^J_{qp}(U^{\dagger}_2)\rightCoord{}\,e^{\textstyle f[U_2]}\ldots\int
dU_N\rightCoord{}\,D^J_{qp}(U^{\dagger}_N)\rightCoord{}\,e^{\textstyle f[U_N]}=\nonumber\rightCoord{}\\\leftCoord{}
\hspace{-0.5in}&&=\lim_{\begin{array}{c} N\to \infty\\\leftCoord{} \delta \to
0\end{array}}\sum^{\infty}_{J=0}\sum^{J}_{p=-J}
\sum^{J}_{q=-J}(2J + 1)\rightCoord{}\,D^J_{pq}(R_0R^{\dagger}_{N+1})\rightCoord{}\,[Z^J_{qp}]^{N+1},
\rightCoord{}}{0.5in}{6}{18}{\hspace{-0.5in}&&Z_{\rm Reg}(R_{N+1},R_0) = \lim_{\begin{array}{c}
N\to \infty\\ \delta \to 0\end{array}} 
\sum^{\infty}_{J=0}\sum^{J}_{p=-J}\sum^{J}_{q=-J}(2J +
1)\,D^J_{pq}(R_0R^{\dagger}_{N+1})\,{\cal N}\int
dU_0\,D^J_{qp}(U^{\dagger}_0)\,e^{\textstyle
f[U_0]}\\
\hspace{-0.5in}&&\times \int
dU_1\,D^J_{qp}(U^{\dagger}_1)\,e^{\textstyle f[U_1]}\int
dU_2\,D^J_{qp}(U^{\dagger}_2)\,e^{\textstyle f[U_2]}\ldots\int
dU_N\,D^J_{qp}(U^{\dagger}_N)\,e^{\textstyle f[U_N]}=\\
\hspace{-0.5in}&&=\lim_{\begin{array}{c} N\to \infty\\ \delta \to
0\end{array}}\sum^{\infty}_{J=0}\sum^{J}_{p=-J}
\sum^{J}_{q=-J}(2J + 1)\,D^J_{pq}(R_0R^{\dagger}_{N+1})\,[Z^J_{qp}]^{N+1},
}{1}\coordE{}\end{eqnarray}
%
where \myHighlight{$Z^J_{qp}$}\coordHE{} is defined by
%
\begin{eqnarray}\coord{}\boxAlignEqnarray{\leftCoord{}\rightCoord{}\label{label18}
Z^J_{qp} = \frac{\leftCoord{}I_{\perp}}{\rightCoord{}2\pi i
\delta}\sqrt{\frac{\leftCoord{}I_{\parallel}}{\rightCoord{}2\pi i \delta}}\int
dU\rightCoord{}\,D^J_{qp}(U^{\dagger})\rightCoord{}\,e^{\textstyle f[U]}.\rightCoord{}
\rightCoord{}}{0mm}{3}{8}{Z^J_{qp} = \frac{I_{\perp}}{2\pi i
\delta}\sqrt{\frac{I_{\parallel}}{2\pi i \delta}}\int
dU\,D^J_{qp}(U^{\dagger})\,e^{\textstyle f[U]}.
}{1}\coordE{}\end{eqnarray}
%
Recall that \myHighlight{$dU$}\coordHE{} is the Haar measure normalized to unity
Eq.(\ref{label15}).

For the subsequent evaluation of the integral over \myHighlight{$U$}\coordHE{} we follow
Diakonov and Petrov and use
%
\begin{eqnarray}\coord{}\boxAlignEqnarray{\leftCoord{}\rightCoord{}\label{label19}
U = e^{\textstyle i\rightCoord{}\,\frac{\leftCoord{}1}{\rightCoord{}2}\rightCoord{}\,\vec{\omega}\cdot \vec{\tau}} \rightCoord{}
\rightCoord{}}{0mm}{2}{7}{U = e^{\textstyle i\,\frac{1}{2}\,\vec{\omega}\cdot \vec{\tau}} 
}{1}\coordE{}\end{eqnarray}
%
for the fundamental representation and
%
\begin{eqnarray}\coord{}\boxAlignEqnarray{\leftCoord{}\rightCoord{}\label{label20}
D^J_{qp}(U^{\dagger}) = \Big(e^{\textstyle -\rightCoord{}\,i\rightCoord{}\,\vec{\omega}\cdot
\vec{T}}\rightCoord{}\,\Big)_{qp}
\rightCoord{}}{0mm}{1}{6}{D^J_{qp}(U^{\dagger}) = \Big(e^{\textstyle -\,i\,\vec{\omega}\cdot
\vec{T}}\,\Big)_{qp}
}{1}\coordE{}\end{eqnarray}
%
for \myHighlight{$J\not= 1/2$}\coordHE{}. In the parameterization (\ref{label19}) the Haar
measure \myHighlight{$dU$}\coordHE{} reads
%
\begin{eqnarray}\coord{}\boxAlignEqnarray{\leftCoord{}\rightCoord{}\label{label21}
dU = \rightCoord{}
\frac{\leftCoord{}d\omega_1d\omega_2d\omega_3}{\rightCoord{}16\pi^2}\rightCoord{}\,\left(\frac{\leftCoord{}2}{\rightCoord{}\omega}\rightCoord{}\,
\sin\frac{\leftCoord{}\omega}{\rightCoord{}2}\right)^2, \rightCoord{}
\rightCoord{}}{0mm}{4}{10}{dU = 
\frac{d\omega_1d\omega_2d\omega_3}{16\pi^2}\,\left(\frac{2}{\omega}\,
\sin\frac{\omega}{2}\right)^2, 
}{1}\coordE{}\end{eqnarray}
%
where \myHighlight{$\omega = \sqrt{\omega^2_1 + \omega^2_2 + \omega^2_3}$}\coordHE{}.
According to the Diakonov and Petrov point of view the integral over
\myHighlight{$U$}\coordHE{} calculated in the limit \myHighlight{$I_{\parallel}/\delta, I_{\perp}/\delta
\to \infty$}\coordHE{} has a saddle point at \myHighlight{$U\simeq 1$}\coordHE{}\,\footnote{Below we do
not pay attention to the factor \myHighlight{$1/16\pi^2$}\coordHE{} that has to be included in
the normalization factor \myHighlight{${\cal N}$}\coordHE{} in the form
\myHighlight{$(16\pi^2)^{N+1}$}\coordHE{}.}. Expanding the integrand around the saddle--point,
keeping only quadric terms and {\bf neglecting the contribution of the
terms coming from the Haar measure}, we get
%
\begin{eqnarray}\coord{}\boxAlignEqnarray{\leftCoord{}\rightCoord{}\label{label22}
Z^J_{qp} &=& \frac{\leftCoord{}I_{\perp}}{\rightCoord{}2\pi i
\delta}\sqrt{\frac{\leftCoord{}I_{\parallel}}{\rightCoord{}2\pi i
\delta}}\int\limits^{\infty}_{-\infty}d\omega_1
\int\limits^{\infty}_{-\infty}d\omega_2\int\limits^{\infty}_{-\infty}
d\omega_3\rightCoord{}\,\exp\rightCoord{}\, \Bigg\{i\frac{\leftCoord{}I_{\perp}}{\rightCoord{}2\delta}\rightCoord{}\,(\omega^2_1 +
\omega^2_2) +
i\frac{\leftCoord{}I_{\parallel}}{\rightCoord{}2\delta}\rightCoord{}\,\omega^2_3\Bigg\}\nonumber\rightCoord{}\\ &\leftCoord{}&\times
\Bigg[\delta_{qp} - \frac{\leftCoord{}1}{\rightCoord{}2}\rightCoord{}\,[\omega^2_1(T^2_1)_{qp} +
\omega^2_2(T^2_2)_{qp}] - \frac{\leftCoord{}1}{\rightCoord{}2}\rightCoord{}\,\omega^2_3\rightCoord{}\,((T_3+ 
T)^2)_{qp}\Bigg].\rightCoord{}
\rightCoord{}}{0mm}{8}{18}{Z^J_{qp} &=& \frac{I_{\perp}}{2\pi i
\delta}\sqrt{\frac{I_{\parallel}}{2\pi i
\delta}}\int\limits^{\infty}_{-\infty}d\omega_1
\int\limits^{\infty}_{-\infty}d\omega_2\int\limits^{\infty}_{-\infty}
d\omega_3\,\exp\, \Bigg\{i\frac{I_{\perp}}{2\delta}\,(\omega^2_1 +
\omega^2_2) +
i\frac{I_{\parallel}}{2\delta}\,\omega^2_3\Bigg\}\\ &&\times
\Bigg[\delta_{qp} - \frac{1}{2}\,[\omega^2_1(T^2_1)_{qp} +
\omega^2_2(T^2_2)_{qp}] - \frac{1}{2}\,\omega^2_3\,((T_3+ 
T)^2)_{qp}\Bigg].
}{1}\coordE{}\end{eqnarray}
%
Integrating over \myHighlight{$\omega_a\,(a=1,2,3)$}\coordHE{} we arrive at the expression
%
\begin{eqnarray}\coord{}\boxAlignEqnarray{\leftCoord{}\rightCoord{}\label{label23}
Z^J_{qp} &=& \delta_{qp} - i\rightCoord{}\,\delta\rightCoord{}\,\Bigg[\frac{\leftCoord{}(T^2_1 + T^2_2)_{qp}}{\rightCoord{}2I_{\perp}}
\leftCoord{}+ \frac{\leftCoord{}((T_3 + T)^2)_{qp}}{\rightCoord{}2I_{\parallel}}\Bigg]=\nonumber\rightCoord{}\\
&\leftCoord{}=&\delta_{qp}\rightCoord{}\,\Bigg\{1 -i\rightCoord{}\,\delta\rightCoord{}\,\Bigg[\frac{\leftCoord{}(J(J+1) - p^2)}{\rightCoord{}2I_{\perp}}
\leftCoord{}+ \frac{\leftCoord{}(p + T)^2}{\rightCoord{}2I_{\parallel}}\Bigg]\Bigg\}.\rightCoord{}
\rightCoord{}}{0mm}{8}{14}{Z^J_{qp} &=& \delta_{qp} - i\,\delta\,\Bigg[\frac{(T^2_1 + T^2_2)_{qp}}{2I_{\perp}}
+ \frac{((T_3 + T)^2)_{qp}}{2I_{\parallel}}\Bigg]=\\
&=&\delta_{qp}\,\Bigg\{1 -i\,\delta\,\Bigg[\frac{(J(J+1) - p^2)}{2I_{\perp}}
+ \frac{(p + T)^2}{2I_{\parallel}}\Bigg]\Bigg\}.
}{1}\coordE{}\end{eqnarray}
%
This agrees with the result obtained by Diakonov and Petrov (see
Eq.(18) of Ref.[1])

Substituting Eq.(\ref{label23}) in Eq.(\ref{label17}) we obtain the
evolution operator \myHighlight{$Z_{\rm Reg}(R_0R^{\dagger}_{N+1})$}\coordHE{} defined by
%
\begin{eqnarray}\coord{}\boxAlignEqnarray{\leftCoord{}\rightCoord{}\label{label24}
\hspace{-0.5in}&&Z_{\rm Reg}(R_{N+1},R_0) = \lim_{\begin{array}{c}
N\to \infty\\\leftCoord{} \delta \to\rightCoord{}
0\end{array}}\sum^{\infty}_{J=0}\sum^{J}_{p=-J}\sum^{J}_{q=-J}(2J +
\leftCoord{}1)\rightCoord{}\,D^J_{pq}(R_0R^{\dagger}_{N+1})\rightCoord{}\,[Z^J_{qp}]^{N+1}=\nonumber\rightCoord{}\\\leftCoord{}
\hspace{-0.5in}&&=\lim_{\begin{array}{c} N\to \infty\\\leftCoord{} \delta \to
0\end{array}}\sum^{\infty}_{J=0}\sum^{J}_{p = -J}(2J +
\leftCoord{}1)\rightCoord{}\,D^J_{pp}(R_0R^{\dagger}_{N+1})\rightCoord{}\,\Bigg\{1
\leftCoord{}-i\rightCoord{}\,\delta\rightCoord{}\,\Bigg[\frac{\leftCoord{}(J(J+1) - p^2)}{\rightCoord{}2I_{\perp}} + \frac{\leftCoord{}(p +
T)^2}{2I_{\parallel}}\Bigg]\Bigg\}^{N+1}=\nonumber\rightCoord{}\\\leftCoord{}
\hspace{-0.5in}&&=\lim_{N\to \infty}\sum^{\infty}_{J=0}\sum^{J}_{p =
\leftCoord{}-J}(2J + 1)\rightCoord{}\,D^J_{pp}(R_0R^{\dagger}_{N+1})\rightCoord{}\,\Bigg\{1
\leftCoord{}-i\rightCoord{}\,\frac{\leftCoord{}t_2-t_1}{\rightCoord{}N+1}\rightCoord{}\,\Bigg[\frac{\leftCoord{}(J(J+1) - p^2)}{\rightCoord{}2I_{\perp}} +
\frac{\leftCoord{}(p + T)^2}{\rightCoord{}2I_{\parallel}}\Bigg]\Bigg\}^{N+1},
\rightCoord{}}{0.5in}{15}{20}{\hspace{-0.5in}&&Z_{\rm Reg}(R_{N+1},R_0) = \lim_{\begin{array}{c}
N\to \infty\\ \delta \to
0\end{array}}\sum^{\infty}_{J=0}\sum^{J}_{p=-J}\sum^{J}_{q=-J}(2J +
1)\,D^J_{pq}(R_0R^{\dagger}_{N+1})\,[Z^J_{qp}]^{N+1}=\\
\hspace{-0.5in}&&=\lim_{\begin{array}{c} N\to \infty\\ \delta \to
0\end{array}}\sum^{\infty}_{J=0}\sum^{J}_{p = -J}(2J +
1)\,D^J_{pp}(R_0R^{\dagger}_{N+1})\,\Bigg\{1
-i\,\delta\,\Bigg[\frac{(J(J+1) - p^2)}{2I_{\perp}} + \frac{(p +
T)^2}{2I_{\parallel}}\Bigg]\Bigg\}^{N+1}=\\
\hspace{-0.5in}&&=\lim_{N\to \infty}\sum^{\infty}_{J=0}\sum^{J}_{p =
-J}(2J + 1)\,D^J_{pp}(R_0R^{\dagger}_{N+1})\,\Bigg\{1
-i\,\frac{t_2-t_1}{N+1}\,\Bigg[\frac{(J(J+1) - p^2)}{2I_{\perp}} +
\frac{(p + T)^2}{2I_{\parallel}}\Bigg]\Bigg\}^{N+1},
}{1}\coordE{}\end{eqnarray}
%
where we have used the definition of \myHighlight{$\delta$}\coordHE{}: \myHighlight{$\delta = (t_2 -
t_1)/(N+1)$}\coordHE{} [1].

Taking the limit \myHighlight{$N\to \infty$}\coordHE{} we get
%
\begin{eqnarray}\coord{}\boxAlignEqnarray{\leftCoord{}\rightCoord{}\label{label25}
\hspace{-0.1in}Z_{\rm Reg}(R_{\infty},R_0) &=&  \sum^{\infty}_{J=0}
\sum^{J}_{p= -J}(2J +
\leftCoord{}1)\rightCoord{}\,D^J_{pp}(R_0R^{\dagger}_{\infty})\nonumber\rightCoord{}\\
&&\leftCoord{}\times\rightCoord{}\,\exp\Bigg\{-i(t_2-t_1)\rightCoord{}\,\Bigg[\frac{\leftCoord{}(J(J+1)
\leftCoord{}- p^2)}{2I_{\perp}} + \frac{\leftCoord{}(p + T)^2}{\rightCoord{}2I_{\parallel}}\Bigg]\Bigg\}.\rightCoord{}
\rightCoord{}}{0.1in}{6}{9}{\hspace{-0.1in}Z_{\rm Reg}(R_{\infty},R_0) &=&  \sum^{\infty}_{J=0}
\sum^{J}_{p= -J}(2J +
1)\,D^J_{pp}(R_0R^{\dagger}_{\infty})\\
&&\times\,\exp\Bigg\{-i(t_2-t_1)\,\Bigg[\frac{(J(J+1)
- p^2)}{2I_{\perp}} + \frac{(p + T)^2}{2I_{\parallel}}\Bigg]\Bigg\}.
}{1}\coordE{}\end{eqnarray}
%
Replacing \myHighlight{$R_0 \to R_1$}\coordHE{} and \myHighlight{$R^{\dagger}_{\infty} \to R^{\dagger}_2$}\coordHE{}
we arrive at the expression 
%
\begin{eqnarray}\coord{}\boxAlignEqnarray{\leftCoord{}\rightCoord{}\label{label26}
Z_{\rm Reg}(R_2,R_1) &=& \sum^{\infty}_{J=0}\sum^{J}_{p= -J}(2J +
\leftCoord{}1)\rightCoord{}\,D^J_{pp}(R_1R^{\dagger}_2)\nonumber\rightCoord{}\\
&&\leftCoord{}\times\rightCoord{}\,\exp\Bigg\{-i\rightCoord{}\,(t_2-t_1)\rightCoord{}\,
\Bigg[\frac{\leftCoord{}(J(J+1) - p^2)}{\rightCoord{}2I_{\perp}} + \frac{\leftCoord{}(p +
T)^2}{2I_{\parallel}}\Bigg]\Bigg\}.\rightCoord{}
\rightCoord{}}{0mm}{5}{10}{Z_{\rm Reg}(R_2,R_1) &=& \sum^{\infty}_{J=0}\sum^{J}_{p= -J}(2J +
1)\,D^J_{pp}(R_1R^{\dagger}_2)\\
&&\times\,\exp\Bigg\{-i\,(t_2-t_1)\,
\Bigg[\frac{(J(J+1) - p^2)}{2I_{\perp}} + \frac{(p +
T)^2}{2I_{\parallel}}\Bigg]\Bigg\}.
}{1}\coordE{}\end{eqnarray}
%
This expression coincides fully with the result obtained by Diakonov
and Petrov (see Eq.(22) of Ref.[1]) and reproduces the expansion of
the evolution operator (\ref{label6}) (see Eq.(12) of
Ref.[1]).

Now taking the limits \myHighlight{$I_{\parallel} \to 0$}\coordHE{} and \myHighlight{$I_{\perp} \to 0$}\coordHE{} we
have to keep the term \myHighlight{$-p = J = T$}\coordHE{} [1] and obtain
%
\begin{eqnarray}\coord{}\boxAlignEqnarray{\leftCoord{}\rightCoord{}\label{label27}
Z_{\rm Reg} (R_2, R_1) =
\leftCoord{}(2T+1)\rightCoord{}\,D^T_{-T-T}(R_1R^{\dagger}_2)\rightCoord{}\,\exp\Bigg[ -i(t_2-t_1)\rightCoord{}\,\frac{\leftCoord{}T}{\rightCoord{}2
I_{\perp}}\Bigg].\rightCoord{}
\rightCoord{}}{0mm}{3}{8}{Z_{\rm Reg} (R_2, R_1) =
(2T+1)\,D^T_{-T-T}(R_1R^{\dagger}_2)\,\exp\Bigg[ -i(t_2-t_1)\,\frac{T}{2
I_{\perp}}\Bigg].
}{1}\coordE{}\end{eqnarray}
%
In the limit \myHighlight{$I_{\perp} \to 0$}\coordHE{} due to this strongly oscillating factor
the r.h.s. of Eq.(\ref{label27}) vanishes. This point has been
discussed in detail in Refs.[2,3]. Such a vanishing of the evolution
operator confirms the statement in Refs.[2,3] that the path integral
representation of the Wilson loop by Diakonov and Petrov is
erroneous.

We would like to accentuate that following Diakonov\myHighlight{$^{\prime}$}\coordHE{}s and
Petrov\myHighlight{$^{\prime}$}\coordHE{}s evaluation of the integral over \myHighlight{$U$}\coordHE{} we have not
taken into account the contribution of the Haar measure. From the Haar
measure (\ref{label21}) we should get an additional contribution
%
\begin{eqnarray}\coord{}\boxAlignEqnarray{\leftCoord{}\rightCoord{}\label{label28}
dU = \rightCoord{}
\frac{\leftCoord{}d\omega_1d\omega_2d\omega_3}{\rightCoord{}16\pi^2}\rightCoord{}\,
\left(\frac{\leftCoord{}2}{\rightCoord{}\omega}\rightCoord{}\,\sin\frac{\leftCoord{}\omega}{\rightCoord{}2}\right)^2 \rightCoord{}
\leftCoord{}= \frac{\leftCoord{}d\omega_1d\omega_2d\omega_3}{\rightCoord{}16\pi^2}\rightCoord{}\,\Big( 1 -
\frac{\leftCoord{}1}{\rightCoord{}12}\rightCoord{}\,(\omega^2_1 + \omega^2_2 + \omega^2_3)\Big).\rightCoord{}
\rightCoord{}}{0mm}{7}{15}{dU = 
\frac{d\omega_1d\omega_2d\omega_3}{16\pi^2}\,
\left(\frac{2}{\omega}\,\sin\frac{\omega}{2}\right)^2 
= \frac{d\omega_1d\omega_2d\omega_3}{16\pi^2}\,\Big( 1 -
\frac{1}{12}\,(\omega^2_1 + \omega^2_2 + \omega^2_3)\Big).
}{1}\coordE{}\end{eqnarray}
%
This changes the value \myHighlight{$Z^J_{qp}$}\coordHE{} in Eq.(\ref{label23}) as follows
%
\begin{eqnarray}\coord{}\boxAlignEqnarray{\leftCoord{}\rightCoord{}\label{label29}
Z^J_{qp} = \delta_{qp}\rightCoord{}\,\Bigg\{1
\leftCoord{}-i\rightCoord{}\,\delta\rightCoord{}\,\frac{\leftCoord{}1}{\rightCoord{}12}\rightCoord{}\,\Bigg(\frac{\leftCoord{}2}{\rightCoord{}I_{\perp}}+
\frac{\leftCoord{}1}{\rightCoord{}I_{\parallel}}\Bigg) - i\rightCoord{}\,\delta\rightCoord{}\,\Bigg[\frac{\leftCoord{}(J(J+1) -
p^2)}{2I_{\perp}} + \frac{\leftCoord{}(p + T)^2}{\rightCoord{}2I_{\parallel}}\Bigg]\Bigg\}.\rightCoord{}
\rightCoord{}}{0mm}{7}{14}{Z^J_{qp} = \delta_{qp}\,\Bigg\{1
-i\,\delta\,\frac{1}{12}\,\Bigg(\frac{2}{I_{\perp}}+
\frac{1}{I_{\parallel}}\Bigg) - i\,\delta\,\Bigg[\frac{(J(J+1) -
p^2)}{2I_{\perp}} + \frac{(p + T)^2}{2I_{\parallel}}\Bigg]\Bigg\}.
}{1}\coordE{}\end{eqnarray}
%
However, it is not the complete set of contributions of order
\myHighlight{$O(\delta/I_{\perp})$}\coordHE{} and \myHighlight{$O(\delta/I_{\parallel})$}\coordHE{} to \myHighlight{$Z^J_{qp}$}\coordHE{}. In
order to take into account all of them we have to expand too the
exponential \myHighlight{$\exp\,f[U]$}\coordHE{} keeping the terms of order
\myHighlight{$\omega^4_1I_{\perp}/\delta$}\coordHE{}, \myHighlight{$\omega^4_2I_{\perp}/\delta$}\coordHE{},
\myHighlight{$\omega^4_3I_{\parallel}/\delta$}\coordHE{} and so on.  The corresponding expansion
of the exponential \myHighlight{$\exp\,f[U]$}\coordHE{} reads
%
\begin{eqnarray}\coord{}\boxAlignEqnarray{\leftCoord{}\rightCoord{}\label{label30}
\exp\rightCoord{}\,f[U] &=& \exp\rightCoord{}\, \Bigg\{i\frac{\leftCoord{}I_{\perp}}{\rightCoord{}2\delta}\rightCoord{}\,(\omega^2_1 +
\omega^2_2) +
i\frac{\leftCoord{}I_{\parallel}}{\rightCoord{}2\delta}\rightCoord{}\,\omega^2_3\Bigg\}\nonumber\rightCoord{}\\
&&\leftCoord{}\times\rightCoord{}\,\Bigg[1 - i\frac{\leftCoord{}I_{\perp}}{\rightCoord{}24\delta}(\omega^2_1 +
\omega^2_2)^2 - i\frac{\leftCoord{}I_{\perp} + I_{\parallel}}{\rightCoord{}24\delta}(\omega^2_1
\leftCoord{}+ \omega^2_2)\rightCoord{}\,\omega^2_3 - i\frac{\leftCoord{}I_{\parallel}}{\rightCoord{}24\delta}\omega^4_3
\leftCoord{}+\ldots\Bigg], \rightCoord{}
\rightCoord{}}{0mm}{9}{16}{\exp\,f[U] &=& \exp\, \Bigg\{i\frac{I_{\perp}}{2\delta}\,(\omega^2_1 +
\omega^2_2) +
i\frac{I_{\parallel}}{2\delta}\,\omega^2_3\Bigg\}\\
&&\times\,\Bigg[1 - i\frac{I_{\perp}}{24\delta}(\omega^2_1 +
\omega^2_2)^2 - i\frac{I_{\perp} + I_{\parallel}}{24\delta}(\omega^2_1
+ \omega^2_2)\,\omega^2_3 - i\frac{I_{\parallel}}{24\delta}\omega^4_3
+\ldots\Bigg], 
}{1}\coordE{}\end{eqnarray}
%
where ellipses denote the terms that have been taken into account in
(\ref{label22}).

The contribution of the terms in Eq.(\ref{label30}) changes \myHighlight{$Z^J_{qp}$}\coordHE{}
(\ref{label29}) as follows
%
\begin{eqnarray}\coord{}\boxAlignEqnarray{\leftCoord{}\rightCoord{}\label{label31}
Z^J_{qp} = \delta_{qp}\rightCoord{}\,\Bigg\{1
\leftCoord{}+ i\rightCoord{}\,\delta\rightCoord{}\,\frac{\leftCoord{}1}{\rightCoord{}8}\rightCoord{}\,\Bigg(\frac{\leftCoord{}2}{\rightCoord{}I_{\perp}}+
\frac{\leftCoord{}1}{\rightCoord{}I_{\parallel}}\Bigg) - i\rightCoord{}\,\delta\rightCoord{}\,\Bigg[\frac{\leftCoord{}(J(J+1) -
p^2)}{2I_{\perp}} + \frac{\leftCoord{}(p + T)^2}{\rightCoord{}2I_{\parallel}}\Bigg]\Bigg\}.\rightCoord{}
\rightCoord{}}{0mm}{7}{14}{Z^J_{qp} = \delta_{qp}\,\Bigg\{1
+ i\,\delta\,\frac{1}{8}\,\Bigg(\frac{2}{I_{\perp}}+
\frac{1}{I_{\parallel}}\Bigg) - i\,\delta\,\Bigg[\frac{(J(J+1) -
p^2)}{2I_{\perp}} + \frac{(p + T)^2}{2I_{\parallel}}\Bigg]\Bigg\}.
}{1}\coordE{}\end{eqnarray}
%
This describes the total contribution of the terms of order
\myHighlight{$O(\delta/I_{\perp})$}\coordHE{} and \myHighlight{$O(\delta/I_{\parallel})$}\coordHE{}. Due to
Eq.(\ref{label31}) the evolution operator reads
%
\begin{eqnarray}\coord{}\boxAlignEqnarray{\leftCoord{}\rightCoord{}\label{label32}
&&\leftCoord{}Z_{\rm Reg}(R_2,R_1) =
\exp\Bigg\{i\rightCoord{}\,(t_2-t_1)\rightCoord{}\,\frac{\leftCoord{}1}{\rightCoord{}8}\rightCoord{}\,\Bigg(\frac{\leftCoord{}2}{\rightCoord{}I_{\perp}}+
\frac{\leftCoord{}1}{\rightCoord{}I_{\parallel}}\Bigg)\Bigg\}\nonumber\rightCoord{}\\
&&\leftCoord{}\times\sum^{\infty}_{J}\sum^{J}_{p= -J}(2J +
\leftCoord{}1)\rightCoord{}\,D^J_{pp}(R_1R^{\dagger}_2)\rightCoord{}\,\exp\Bigg\{-i\rightCoord{}\,(t_2-t_1)\rightCoord{}\,
\Bigg[\frac{\leftCoord{}(J(J+1) - p^2)}{\rightCoord{}2I_{\perp}} + \frac{\leftCoord{}(p +
T)^2}{2I_{\parallel}}\Bigg]\Bigg\}.\rightCoord{}
\rightCoord{}}{0mm}{9}{16}{&&Z_{\rm Reg}(R_2,R_1) =
\exp\Bigg\{i\,(t_2-t_1)\,\frac{1}{8}\,\Bigg(\frac{2}{I_{\perp}}+
\frac{1}{I_{\parallel}}\Bigg)\Bigg\}\\
&&\times\sum^{\infty}_{J}\sum^{J}_{p= -J}(2J +
1)\,D^J_{pp}(R_1R^{\dagger}_2)\,\exp\Bigg\{-i\,(t_2-t_1)\,
\Bigg[\frac{(J(J+1) - p^2)}{2I_{\perp}} + \frac{(p +
T)^2}{2I_{\parallel}}\Bigg]\Bigg\}.
}{1}\coordE{}\end{eqnarray}
%
{\bf Hence, the evaluation of the path integral (\ref{label2}) with
the correct account for all contributions of order
\myHighlight{$O(\delta/I_{\perp})$}\coordHE{} and \myHighlight{$O(\delta/I_{\parallel})$}\coordHE{} around the
saddle--point, including the contributions of the Haar measure and the
terms of order \myHighlight{$\omega^4_1I_{\perp}/\delta$}\coordHE{},
\myHighlight{$\omega^4_2I_{\perp}/\delta$}\coordHE{}, \myHighlight{$\omega^4_3I_{\parallel}/\delta$}\coordHE{} and so on,
leads to a result that differs fully from the expansion (\ref{label6})
derived from the quantum mechanical consideration of \myHighlight{$Z_{\rm Reg}(R_2,
R_1)$}\coordHE{} in terms of eigenfunctions of the axial--symmetric top. This
means that the path integral (\ref{label2}) representing the evolution
operator \myHighlight{$Z_{\rm Reg}(R_2, R_1)$}\coordHE{} has no relation to the
axial--symmetric top and predicts a completely different energy
spectrum than that given by Eq.(\ref{label7}) for the quantum
axial--symmetric top. In the limit \myHighlight{$I_{\parallel} \to 0$}\coordHE{} and
\myHighlight{$I_{\perp} \to 0$}\coordHE{} the evolution operator vanishes by virtue of the
strongly oscillating factors.}

Thus, the only well defined magnitude of the evolution operator is
zero. This confirms fully the results obtained in Refs.[2,3] that the
evolution operator \myHighlight{$Z_{\rm Reg}(R_2, R_1)$}\coordHE{} vanishes and the path
integral representation of the Wilson loop suggested by Diakonov and
Petrov in terms of the evolution operator \myHighlight{$Z_{\rm Reg}(R_2, R_1)$}\coordHE{} is
erroneous. All of these statements are completely applicable to the
results discussed by Diakonov and Petrov in their recent manuscript
hep--lat/0008004 [4].

\section*{Acknowledgement}

\hspace{0.2in} This investigation has been initiated by Oleg Borisenko
having a great interest in the path integral representation of the
Wilson loop suggested by Diakonov and Petrov in Refs.[1,4].

\newpage

\begin{thebibliography}{9}
\bibitem{[1]} 
D. I. Dyakonov and V. Yu. Petrov, Phys. Lett. B224
(1989) 131.
\bibitem{[2]} 
M. Faber, A. N. Ivanov, N. I. Troitskaya  and M. Zach,
Phys. Rev. D62 (2000) 025019.
\bibitem{[3]} 
M. Faber, A. N. Ivanov and  N. I. Troitskaya, hep--th/0012083.
\bibitem{[4]} 
D. Diakonov and V. Petrov,  hep--lat/0008004.
\end{thebibliography}

\end{document}



\bye
