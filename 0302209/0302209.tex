\documentclass[a4paper,12pt]{article}
\begin{document}

\title{Supersymmetry breaking and 4 dimensional string models}
\author{Michael Hewitt \\
Computing Services, Canterbury Christ Church University College,\\
North Holmes Road, Canterbury, CT1 1QU, U.K. \\ email:
mike@cant.ac.uk,
 tel:+44(0)1227-767700.}
\date{26 February 2003}
\maketitle
\begin{abstract}

A family of superpotentials is constructed which may be relevant
to supersymmetry breaking in 4 dimensional (0,1) heterotic string
models. The scale of supersymmetry breaking, as well as the
coupling constant, would be stable and could not run away to
zero.\

\


PACS numbers:  11.25.Mj, 04.65.+e, 11.25.Sq


\end{abstract}

\section{Introduction}
\ \ \ In this paper we will study a family of superpotentials
which may be relevant to supersymmetry (susy) breaking in 4
dimensional (0,1) heterotic string models, and which generalise
those given in [1,2]. By 4 dimensional string models we have in
mind that the extra string degrees of freedom are subject to
twisted boundary conditions in such a way that no `breathing'
modes for extra dimensions appear in the level zero spectrum [3].
Stability is a feature of such $D=4$ string models, for which
decompactification cannot take place. There is a fixed, arbitrary
gauge coupling, related to string topology through the genus
expansion. There may be a possible restriction of this coupling
constant required for the consistent propagation of knotted and
linked strings, which are features which appear when $D$ is
reduced to 4, with strings producing orbifold-like singularities
in space-time. The origin of susy breaking is often ascribed to
the gaugino condensation mechanism [4,5]. This relies on a
variation of the gauge coupling with a scalar field, which is
therefore absent in these models. Consequently, there would need
to be susy breaking by some fundamental rather than composite
Goldstone supermultiplet. If this is possible, there would be
benefits in avoiding problems with the stability of the gauge
hierarchy and a shallow potential well which could lead to
decompactification during inflation [6]. A vanishing 4 dimensional
cosmological constant also seems more natural in a model without
extra dimensions. For the present, we will consider the scalar
fields only for simplicity. A vanishing cosmological constant may
be viewed as a balance between local and global susy breaking.
This can be described geometrically as a requirement that the
field-space vector formed by all auxiliary fields be null. In
terms of the geometry of the chiral scalar fields, this is
equivalent to the superpotential $G$ having a critical slope of
$\sqrt{3}$ in field space. The polarization of positive and
negative energy auxiliary fields due to this local/global susy
breaking is much greater than the energy due to non-cancellation
between bosons and fermions from global susy breaking. A small
adjustment of the balance between gravitational and other
auxiliaries would cancel the residue to keep $\Lambda =0$. We will
assume that such a balancing mechanism applies. A small failure of
the balance mechanism could give a small $\Lambda$, as may be the
actual case in cosmology, but we will not attempt to understand
this here.

In $N=1$ susy, we may make a decomposition of the chiral fields
into the Goldstino $z$ and other fields $s_{i}$,where $z$,$s_{i}$
are orthogonal at the vacuum expectation values $z_{0},s_{0}$,
\begin{equation}
\frac{\partial^2 G}{\partial z \partial s^{*}_{i}}(z_{0},s_{0}) =
\frac{\partial G}{\partial s_{i}}(z_{0},s_{0}) = 0 \label{PDZ}
\end{equation}
We propose that the Goldstino scalar $z$ can be approximated by
the spin 0 part of the dilaton supermultiplet, which may be
regarded as scalar and pseudoscalar components of the graviton, as
we can form a massive gravitino multiplet, or extended graviton
multiplet by taking the full content of the spin $(1)_{L} \otimes
(1 \oplus 1/2)_{R}$ heterotic level zero fields. Our motivation is
that the dilaton multiplet in $D=4$ is naturally present, and has
a flat potential, or critical-slope superpotential (implementing
the local/global balance principle) in the absence of other
sources of susy breaking [7]. The pseudoscalar graviton mass is
zero, protected by a gauge invariance, and therefore the
pseudoscalar part of $z$ will be a pseudo-Goldstone boson with a
relatively long range. Mixings between the gravitational and other
scalars will be suppressed by factors of $1/m_{3/2}$, since the
contributions of different superfields to the Goldstone
supermultiplet would be in proportion to their auxiliary
components. CP violation may give a small scalar component to the
pseudo-Goldstone boson, but we will neglect this below.
 The situation in realistic models where there are also terms in $V$ from the gauge
auxiliary $D$ fields is more complicated, but providing that the
local/global balance principle applies, there will again be zero
cosmological constant with a small mixing $O(1/m_{3/2})$ between
$z$ and other superfields.
\section{Derivation of the Superpotentials}
\ \ \  Decomposing $z$ into scalar and pseudoscalar parts, we have
$z = x + \mathrm{i}y$ where
 \begin{equation}\label{y}
    dy  = *(dB + \omega_{GS}) + O(1/m_{3/2})
\end{equation}
($\omega_{GS}$ is the Green-Schwarz 3-form) so that under the
transformation $y \rightarrow y + \epsilon$, $\delta G = \epsilon
O(G/m_{3/2})$ or
\begin{equation}\label{z}
 \frac{\partial G(z)}{\partial y} = O(G/m_{3/2}),\  \frac{\partial G(z)}{\partial
 y}(z_{0},s_{0})=0
\end{equation}
i.e. $G(z) = G(x) + O(G/m_{3/2})$. For any function $h(z)$ which
depends only on $x$ we have
\begin{equation}\label{PD}
   \frac{\partial h(x)}{\partial z} = \frac{\partial h(x)}{\partial
   z^{*}} =  1/2 \frac{dh}{dx} \equiv \frac{h^{'}}{2}
\end{equation}
so that the potential for $x$ becomes (neglecting for now the
contribution from other fields)
\begin{equation}\label{V1}
 V(x) = \mathrm{e}^{G}(\frac{G^{'2}}{G^{''}}-3) =
 \frac{\mathrm{e}^{G}}{G^{''}}(G^{'2}-3G^{''}).
\end{equation}
Introducing the notation
\begin{equation}\label{g}
    G^{'} = g(x)
\end{equation}
we can write
\begin{equation}\label{f}
    g^{2} - 3g^{'} = f^{2}.
\end{equation}
For a vanishing cosmological constant, we require that $f$ should
have a zero. The simplest ansatz with this property is essentially
\begin{equation}\label{A}
f^{2} = \alpha (g-3)^{2}.
\end{equation}
Here we have used the linear transformation $x \rightarrow ax+b$
to arrange $x = 0, g = 3$ at $f = 0$ without loss of generality
(these values are chosen to simplify the formulas below). The case
$\alpha = 0$ gives the flat, no-scale solution

\begin{equation}\label{P}
G = -3\ln{(x)},\  \ V=0.
\end{equation}

We will now derive the solution corresponding to eq.(\ref{A}).
Rearranging eq.(\ref{f}) gives the first order differential
equation
\begin{equation}\label{x}
    \frac{dx}{dg} = \frac{1}{(1-\alpha)g^{2} + 6\alpha g -9\alpha}
\end{equation}
with solution
\begin{equation}\label{Sx}
    x = \tan^{-1}{(\frac{1-\alpha}{3\alpha}g + 1)} - \tan^{-1}{(\frac{1}{\alpha})}
\end{equation}
giving
\begin{equation}\label{Sg}
    g = \frac{3\alpha}{1-\alpha}(\tan{(x+c)-1)}
\end{equation}
where
\begin{equation}\label{c}
    c =  \tan^{-1}{(\frac{1}{\alpha})}.
\end{equation}
Now
\begin{equation}\label{G}
    G(x) = \int_{u = 0}^{x}g(u) du + G_{0}
\end{equation}
where the constant of integration $G_{0}$ gives the value of
$\ln{(m_{3/2}^{2})}$ at the potential minimum. We have
\begin{equation}\label{SG}
    G = G_{0}  + \frac{3\alpha}{1-
    \alpha}(\log \sec (x+c) - \log \sec (c)-x)
\end{equation}
and using
\begin{equation}\label{Gp}
G^{''} = g^{'} = \frac{3\alpha}{1-\alpha}\sec^{2}{(x+c)}
\end{equation}
gives us, on eliminating $c$ by using elementary identities,
\begin{equation}\label{SV2}
    V =3 \mathrm{e}^{G_{0}}\frac{(1+ \alpha^{2})}{1 - \alpha}[\mathrm{e}^{-x}(\cos{(x)} - \frac{1}{\alpha}\sin{(x)})]^{\frac{
    3\alpha}{\alpha -1}}\sin^{2}{(x)}.
\end{equation}
Here $x$ will be confined to the range where $V$ is finite i.e.
$\tan(x) < \alpha$ for $0 < \alpha < 1$. For the mass of the
scalar $x$ we have

\begin{equation}\label{ms}
m_{S}^{2} = \frac{V^{''}}{G^{''}}|_{x = 0}  =
\frac{\mathrm{e}^{G_{0}}}{G^{''}}
\frac{(f^{'})^{2}}{G^{''}}|_{x=0} = \alpha m_{3/2}^{2}
\end{equation}
giving the interpretation of the parameter $\alpha$ as
$m_{S}^{2}/m_{3/2}^{2}$. As argued above the mass of the
pseudoscalar $y$ will be
\begin{equation}\label{mp}
m_{P}^{2} = O(m_{3/2}^4).
\end{equation}
The solution eq.(\ref{SV2}) may be written implicity as a function
of the sigma model parameter

\begin{equation}\label{phi}
    \phi(x) = \int_{u=0}^{x}\sqrt{2G^{''}(u)}du =
    \sqrt{\frac{2\alpha}{3(1-\alpha)}}\ln{(\frac{\sec(x+c)+\tan(x+c)}{\sec(c)+ \tan(c)})}
\end{equation}
with standard kinetic term. Now consider the limit $\alpha
\rightarrow 1$ corresponding to the physically interesting case
$m_{S} = m_{3/2}$. Going back to eq.(\ref{f}) with $\alpha = 1$
gives

\begin{equation}\label{L}
   x = 1/2 \ln{(2g/3 - 1)}.
\end{equation}
Introducing an alternative parametrization by
\begin{equation}\label{w}
    w = \exp{(z)}, |w| = \exp{(x)}
\end{equation}
(so that $y$ and $y + 2\pi n$ are identified), which is natural
when we recall the origin of $x$ as the 4 dimensional dilaton,
gives the solution
\begin{equation}\label{SL}
    V = \frac{3 \mathrm{e}^{G_{0}}}{4}(|w|^{2} \exp{(|w|^{2}-1)})^{3/4}(|w| -
    |w|^{-1})^{2}.
\end{equation}
Here the value of $w$ is kept away from zero by the divergence in
$V(0)$. In this case the sigma model field is $\phi =
\sqrt{6}|w|$, so that if we write $\psi = |w|$ we can write the
Lagrangian terms for $\psi$ as

\begin{equation}\label{psi}
    L_{\psi} = 3\sqrt{-g}((\partial \psi)^{2} - \frac{\mathrm{e}^{G_{0}}}{4}
    (\psi^{2}\exp{\psi^{2}-1})^{3/4}(\psi - \psi^{-1})^{2}).
\end{equation}
(Here $g$ is the usual $D=4$ metric determinant.) An interesting
feature of this model is that the scalar and pseudoscalar fields
will be described by a sigma model with the flat Kahler metric

\begin{equation}\label{sig}
    d \sigma^{2} = 6dwdw^{*} = 6(d\psi^{2} + \psi^{2}d\theta^{2}).
\end{equation}

 We will now consider the origin of the integration constant $G_{0}$, which determines $m_{3/2}$.
 Suppose that, near $(z_{0},s_{0})$, $G$ takes the form
\begin{equation}\label{K}
    G(z,s) = G_{g}(z)+ G_{s}(s) + H(z,s) = G_{g}(z)+ K(s) +
    \ln(|W(s)|^{2}) + H(z,s)
\end{equation}
solving eq.(\ref{PDZ}), where $G_{g}(z)$ takes the form discussed
above and for the remaining chiral fields $s$, $K(s)$ is the
Kahler potential (generating the field kinetic terms) and $W(s)$
represents the chiral scalar superfield interactions. $H(z,s)$
represents mixing terms of higher order in $1/m_{3/2}$ relative to
$G(z,s)$, with $H(z_{0},s_{0})=0$. We expect $K_{0} \sim 0$,
giving
\begin{equation}\label{W}
    \exp{(G_{0})} \sim |W_{0}|^{2}
\end{equation}
where $W_{0}$ is the minimum value of $W(s)$ - since the vacuum
represents the minimum of $G$ w.r.t. $s$, it also represents the
minimum of $W(s)$. Now the leading renormalizable contribution
gives $W \sim g_{4}X_{1}X_{2}X_{3}$ where $X_{i}$ may acquire
expectation values by a Coleman-Weinberg mechanism well below the
string tension scale. For $m_{3/2}\sim$ 1TeV, the $X_{i0}$ may
come from GUT symmetry breaking, a neutrino mass see-saw mechanism
or some other hidden sector. String diagrams with zero background
field give $V(z) = 0$. While this may represent the $\alpha
\rightarrow 0$ limit of the family of models represented by
eq.(\ref{A}), it may alternatively represent the limit
$\exp{(G_{0})} \rightarrow 0$, in particular with $\alpha = 1$,
allowing stable supersymmetry breaking in 4 dimensional string
models. For $\alpha \leq 1$, $m_{3/2}$ is prevented from running
away to zero, so that it is natural in this scenario for susy
breaking to survive inflation.

\

\textbf{References}


[1] J.Polonyi, Budapest Preprint KFKI-1997-93.

[2] E.Cremmer, S.Ferrara, C.Kounnas, and D.V.Nanopoulos, {\em
Phys. Lett B\/}{\bf 133} (1983) 61.

[3] H.Kawai, D.C.Lewellen and S.H.H.Tye, {\em Nucl. Phys. B\/}{\bf
288} (1987) 1.

[4] S.Ferrara, L.Girardello and H.P.Nilles, {\em Phys. Lett
B\/}{\bf 125} (1983) 457.

[5] M.Dine, R.Rohm, N.Seiberg and E.Witten, {\em Phys. Lett
B\/}{\bf 156} (1985) 55.

[6] R.Brustein and P.J.Steinhardt, {\em Phys. Lett B\/}{\bf 302}
(1993) 196.

[7] M.Dine and N.Seiberg, {\em Phys. Rev. Lett\/}{\bf 57} (1986)
2625.





\end{document}

