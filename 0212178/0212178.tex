\documentclass[a4paper,12pt]{article}%
\usepackage{ amsmath, graphics, rotating}% 

\textwidth      6.00in
\textheight     8.50in
\topmargin     -0.25in
\leftmargin -10mm

\baselineskip 2pc
\parindent 2cm

\begin{document}
\large
\begin{center}{\large\bf REDUCED SPIN-STATISTICS THEOREM}
\end{center}
\vskip 1em \begin{center} {\large Felix M. Lev} \end{center}
\vskip 1em \begin{center} {\it Artwork Conversion Software Inc.,
1201 Morningside Drive, Manhattan Beach, CA 90266, USA 
(E-mail:  felixlev@hotmail.com)} \end{center}
\vskip 1em

{\it Abstract:}
\vskip 0.5em

As argued in our previous papers, it would be more natural to
modify the standard approach to quantum theory by requiring
that i) one unitary irreducible representation (UIR) of the 
symmetry algebra should describe a particle and its antiparticle 
simultaneously. This would automatically
explain the existence of antiparticles and show that a 
particle and its antiparticle are different states of the same
object. If i) is adopted then among the Poincare, so(2,3) 
and so(1,4) algebras only the latter is a candidate for 
constructing elementary particle theory. We extend our
analysis in hep-th/0210144 and prove that: 
1) UIRs of the so(1,4) algebra
can indeed be interpreted in the framework of i) and cannot
be interpreted in the framework of the standard approach;
2) as a consequence of a new symmetry (called AB one) 
between particles
and antiparticles for UIRs satisfying i), elementary 
particles described by UIRs of the so(1,4) 
algebra can be only fermions; 3) as a consequence
of the AB symmetry, the vacuum condition can be
consistent only for particles with the 
half-integer spin (in conventional units) and therefore
only such particles can be elementary. 
In our approach the well known fact that fermions have
imaginary parity is a consequence of the AB symmetry.

\begin{flushleft} PACS: 11.30Cp, 11.30.Ly\end{flushleft}

\section{Introduction}
\label{S1} 
 
\subsection{Motivation}

The phenomenon of local quantum field theory (LQFT) has no 
analogs in the history of science. There is no branch of 
science where so impressive agreements between theory and 
experiment have been
achieved. At the same time, the level of mathematical rigor 
in the LQFT is very poor and, as a consequence,
the LQFT has several well known difficulties and 
inconsistencies. The absolute majority of physicists believes
that agreement with experiment is much more important than the
lack of mathematical rigor, but not all of them think so.
For example, Dirac wrote in Ref. \cite{DirMath}: 
$'$The agreement
with observation is presumably by coincidence, just like the
original calculation of the hydrogen spectrum with Bohr orbits.
Such coincidences are no reason for turning a blind eye to the
faults of the theory. Quantum electrodynamics is rather like
Klein-Gordon equation. It was built up from physical ideas
that were not correctly incorporated into the theory and it
has no sound mathematical foundation.$'$  

One could agree or disagree with this statement, but in any
case, the majority of physicists believes that the LQFT should
be treated \cite{Wein} $'$in the way it is$'$, but at the same
time it is \cite{Wein} a $'$low energy approximation to a
deeper theory that may not even be a field theory, but something
different like a string theory$'$. 

The main problem of course is the choice of strategy
for constructing a new quantum theory. Since 
nobody knows for sure what strategy is the best one, 
different approaches should be investigated. Dirac's 
advice given in Ref. \cite{DirMath} is as follows: 
$'$I learned to distrust all
physical concepts as a basis for a theory. Instead one should
put one's trust in a mathematical scheme, even if the scheme
does not appear at first sight to be connected with physics.
One should concentrate on getting an interesting mathematics.$'$ 

Typically the LQFT starts from a local Lagrangian for which, by
using the canonical Noether formalism, one can determine a set of
conserving physical observables. After quantizing the local fields
in question, these observables become quantum operators 
and the quantum Lagrangian obtained in such a way contains
products of local quantum fields at coinciding points. 
However, interacting field operators can be treated only
as operator valued distributions \cite{Haag} and therefore their
products at coinciding points are not well defined.
Although there exists a wide literature on this problem, 
a universal solution has not been found yet.

There exist two essentially different approaches to quantum 
theory --- the standard operator approach and the path 
integral approach. We accept the operator approach. In this 
case, to be consistent, we should assume that {\it any} 
physical quantity is described by a selfadjoint operator 
in the Hilbert space of states for the system under 
consideration (we will not discuss the difference between 
selfadjoint and Hermitian operators).
Then the first question which immediately arises is that, 
even in the nonrelativistic quantum mechanics, there is no 
operator corresponding to time \cite{time}. It is also 
well known that when quantum mechanics is combined with 
relativity, then there is no operator satisfying all the 
properties of the spatial position operator 
(see e.g. Ref. \cite{Schweber}) . For these reasons the 
quantity $x$ in the Lagrangian density $L(x)$ is not 
the coordinate but a parameter which becomes the 
coordinate in the classical limit.

These facts were well known already in 30th of the 20th 
century. As a result of evolution of these ideas, it 
was widely believed in 50th and 60th that spacetime is 
a rudimentary notion which will disappear in the 
ultimate quantum theory. Since that time, no arguments 
questioning those ideas have been given, but in view of 
the great success of gauge theories in 70th and 80th , 
such ideas became almost forgotten. 

At present, the predictions of the standard model are 
in agreement with experiment with an unprecedented 
accuracy. Nevertheless, the difficulties of the 
canonical LQFT have not been overcome. For this reason 
there exist several approaches, the goal of which is 
to avoid the product of interacting quantum fields at 
the same spacetime point. In addition to the well 
known theories (string theory, noncommutative quantum 
theory etc.) we also would like to mention a very 
interesting approach by Saniga where the classical 
spacetime is replaced by a Galois field \cite{Saniga}.

In view of this situation, the problem arises of how 
one should define the notion of elementary particles. 

Although particles are observable and fields are not, 
in the spirit of the LQFT, fields are more fundamental 
than particles, and a possible definition is as 
follows \cite{Wein1}: 'It is simply a particle whose
field appears in the Lagrangian. It does not matter if
it's stable, unstable, heavy, light --- if its field 
appears in the Lagrangian then it's elementary, 
otherwise it's composite'.
    
Another approach has been developed by Wigner in his 
investigations of unitary irreducible representations 
(UIRs) of the Poincare group \cite{Wigner}. In view 
of this approach, one might postulate that in the 
general case, a particle is elementary if the space 
of its wave functions is the space of UIR of the 
symmetry group in the given theory.

In standard well-known theories (QED, electroweak 
theory and QCD) the above approaches coincide 
although the problem arises of whether the second 
definition is compatible with confinement. However, 
when the symmetry group is not the Poincare one, 
additional problems arise. For example, in view of 
modern approaches to the LQFT in curved spacetime, 
the de Sitter group SO(1,4) cannot be the symmetry 
group since, from the standpoint of any local 
observer, the vacuum has a finite temperature and 
admits particle destruction and creation (see e.g. 
Refs. \cite{Narlikar,Susskind}). We discuss this problem 
in Sect. \ref{S7}.   

Although the Wigner approach is well defined in the 
framework of standard mathematics,
the following problem arises. The symmetry group is 
usually chosen as the group of motions of some classical 
manifold. How does this agree with the above discussion 
that quantum theory in the operator formulation should 
not contain spacetime? A possible answer is as follows. 
One can notice that for computing observables (e.g. the
spectrum of the Hamiltonian) we need in fact not a 
representation of the group but a representation of its 
Lie algebra by Hermitian operators. After such a 
representation has been constructed, we have only 
operators acting in the Hilbert space and this is all 
we need in the operator approach. The representation 
operators of the group are needed only if it is 
necessary to calculate some macroscopic 
transformation, e.g. time evolution. In the approximation 
when classical time is a good approximate parameter, 
one can calculate evolution, but nothing guarantees 
that this is always the case. Let us also note that in
the stationary formulation of scattering theory, the
S-matrix can be defined without any mentioning of
time (see e.g. Ref. \cite{Kato}). For these reasons 
we can assume that on quantum level the symmetry 
algebra is more fundamental than the symmetry group
(see also the discussion in Sect. \ref{S7}). 

In other words, instead of saying that some operators 
satisfy commutation relations of a Lie algebra 
$A$ because spacetime $X$ has a group of motions $G$ such 
that $A$ is the Lie algebra of $G$, we say that there
exist operators satisfying  commutation 
relations of a Lie algebra $A$ such that: for some 
operator functions $\{O\}$ of them, the classical 
limit is a good approximation, a set $X$ of the eigenvalues
of the operators $\{O\}$ represents a classical manifold with 
the group of motions $G$ and its Lie algebra is $A$ (see also Sect. 
\ref{S2}). This is not of course in the spirit of 
famous Klein's Erlangen program \cite{Klein} or LQFT.

Summarizing our discussion, we assume that, 
{\it by definition}, on quantum level a Lie algebra is 
the symmetry algebra if there exist physical
observables such that their operators  
satisfy the commutation relations characterizing the
algebra. Then, a particle is called elementary if the 
space of its wave functions is a space of irreducible 
representation of this algebra by Hermitian operators. 
In the literature such representations usually also are 
called UIRs meaning that the representation of the 
algebra can be extended to an UIR of the corresponding 
Lie group. 
 
The approach we accept is in the spirit of that considered 
by Dirac in Ref. \cite{Dir}. Let us also note that 
although we treat the description of quantum systems in 
terms of representations of algebras as more 
fundamental, this does not mean that for investigating 
properties of algebra representations we cannot use
mathematical results on group representations. 

In our papers \cite{lev2}, we discussed an approach when 
quantum  states are described by elements of a linear 
space over a Galois field, and the operators of 
physical quantities --- by operators in this space. 
It has been argued that such an approach is more 
natural than the standard one and the future quantum
physics will be based on Galois fields. However, in 
the present paper we work in the framework of the 
standard quantum theory based on complex numbers.

\subsection{Statement of the problem}

In standard Poincare or anti de Sitter (AdS) invariant
theories, the field theoretical and Wigner definitions
of elementary particles (see the preceding subsection)
do not contradict each other. Therefore, each elementary
particle can be described by using an UIR of the Poincare 
or AdS group or algebra. In addition, it also can be described 
by using a Poincare or AdS covariant 
equation. In these theories the existence of antiparticles
is explained as follows. For each values of the mass 
and spin, there exist two UIRs - with positive and 
negative energies, respectively.
At the same time, the corresponding covariant equation 
has solutions with both positive and negative energies.    
As noted by Dirac (see e.g. his Nobel
lecture \cite{DirNobel}), the existence of the negative
energy solutions represents a difficulty which should be
resolved. In the standard approach, the solution is given
in the framework of quantization such that the
creation and annihilation operators for the antiparticle
have the usual meaning but they enter the quantum Lagrangian 
with the coefficients representing the negative energy
solutions. 

Such an approach has lead to impressive success in 
describing various experimental data. However, as noted by
Weinberg \cite{Wein1}, 'this is our aim in physics, not
just to describe nature, but to explain nature'. From this 
point of view, it seems unnatural that the covariant
equation describes the particle and antiparticle
simultaneously while the UIRs for them are fully 
independent of each other. Moreover, the UIRs with
negative energies are not used at all. 

The necessity to have negative energy solutions is
related to the implementation of the idea that the
creation or annihilation of an antiparticle can be treated,
respectively, as the annihilation or creation of the 
corresponding particle with the negative energy. However,
since negative energies have no direct physical meaning in
the standard theory, this idea is implemented implicitly
rather than explicitly. 

The above program cannot be implemented if
the de Sitter (dS) group SO(1,4) is chosen as the 
symmetry group or the dS algebra so(1,4) is chosen as
the symmetry algebra. Some of the reasons have been already
indicated in the preceding subsection. Also, it is 
well known that in UIRs of the dS 
algebra, the dS Hamiltonian is not positive definite and
has the spectrum in the interval $(-\infty,+\infty)$ 
see e.g. Refs. \cite{Dobrev,Men,Moy,lev1,lev1a}). 
Note also that in contrast to the AdS
algebra so(2,3), the dS one does not have a supersymmetric
generalization. For this and other reasons it was believed 
that the dS group or algebra were not suitable for constructing 
elementary particle theory. Although our approach
considerably differs from that in Refs. \cite{Narlikar,Susskind}
and references therein, we come to the same conclusion
(see Sect. \ref{S3}) that in the standard approach the dS 
group cannot be a symmetry group. However, it is possible to 
modify the standard approach in such a way (see below) that 
theories with the dS symmetry become consistent.

It is well known that the group SO(1,4) is the symmetry 
group of the four-dimensional manifold in the five-dimensional 
space, defined by the equation 
\begin{equation}
 x_0^2-x_1^2-x_2^2-x_3^2-x_4^2=-R^2
\label{1}
\end{equation}
where a constant $R$ has the dimension of length.
The quantity $R^2$ is often written as $R^2=3/\Lambda$ where
$\Lambda$ is the cosmological constant. 
The nomenclature is such that $\Lambda < 0$ for  
the AdS symmetry while $\Lambda >0$ - for the dS one.
The recent astronomical data show that, although
$\Lambda$ is very small, it is probably positive 
(see e.g. Ref. \cite{Perlmutter}). For this reason the 
interest to dS theories has increased. Nevertheless, the existing
difficulties have not been overcome (see e.g. Ref. \cite{Witten}).

As shown in Ref. \cite{lev2}, in 
quantum theory based on a Galois field, Galois field analogs
of UIRs of the AdS algebra so(2,3) have a property 
that a particle and 
its antiparticle are described by the same irreducible
representation of the symmetry algebra. This automatically 
explains the existence of antiparticles and shows that a 
particle and its antiparticle represent different states of 
the same object. 
As argued in Ref. \cite{lev2}, 
the description of quantum theory in terms of Galois fields
is more natural than the standard description based on the
field of complex numbers. However, in the present paper we
consider only the standard approach based on complex numbers,
but with the following 
modification. Instead of saying that UIRs 
describe elementary particles, we assume that

{\it Supposition 1:} Any UIR of the symmetry algebra should 
describe a particle and its antiparticle simultaneously.

With such a requirement, among the Poincare, AdS and dS
algebras, only the latter can be a candidate for 
constructing the
elementary particle theory. As shown in Ref. \cite{hep}, 
UIRs of the dS algebra are indeed compatible with Supposition 1. 
By quantizing such UIRs and requiring that the energy should be
positive definite in the Poincare limit, it has been shown 
that only fermions can be elementary. 

In the present paper we analyze UIRs of the so(1,4) algebra
not only in the Poincare limit but in the general case
as well.
In Sect. \ref{S2} we describe well known results on the
derivation of explicit expressions for the representation
generators of the dS group. In Sect. \ref{S3} the
Poincare limit is discussed. It is explained why UIRs
of the dS algebra can be treated in the framework of
Supposition 1 and cannot be treated in the framework of
the standard approach. In Sect. \ref{S4} we describe in
detail a basis of UIRs such that all the quantum numbers
are discrete, and in Sect. \ref{S5} the generators are
explicitly written down in the quantized form. 
In Sect. \ref{Physical} we discuss the implementation
of Supposition 1 in
the case when the dS algebra is the exact symmetry algebra.
In Sect. \ref{S6} it is shown that UIRs of the dS algebra
possess a new symmetry between particles and 
antiparticles. Following Ref. \cite{lev2}, we call this
symmetry the AB one. It is shown that the AB symmetry
is compatible only with the anticommutation relations
and therefore only fermions can be elementary. As
shown in Sect. \ref{S7}, the vacuum condition is
consistent only for particles with the half-integer
spin (in conventional units), and therefore only such
particles can be elementary. Finally, in Sect. \ref{S8}
we argue that neutral elementary particles cannot
exist and show that the well known fact that fermions
have imaginary parity follows from the AB symmetry. 

\section{UIRs of the SO(1,4) group}
\label{S2}

As already noted, the de Sitter group SO(1,4) is the symmetry 
group of the four-dimensional manifold defined by Eq. (\ref{1}).
Elements of a map of the point $(0,0,0,0,R)$ (or $(0,0,0,0,-R)$) 
can be parametrized by the
coordinates $(x_0,x_1,x_2,x_3)$. If $R$ is very large then such a
map  proceeds to Minkowski space and the action of the dS group 
on this map --- to  the action of the Poincare group.

\begin{sloppypar}
In the present paper it will be convenient for us to work with 
the units $\hbar/2=c=1$. Then the spin of any particle is
always an integer. For the normal relation between spin and 
statistics, the spin of fermions is odd and the spin of bosons
is even. In this system of units the representation generators 
of the SO(1,4) group
$M^{ab}$ ($a,b=0,1,2,3,4$, $M^{ab}=-M^{ba}$) should satisfy the
commutation relations
\begin{equation}
[M^{ab},M^{cd}]=-2i (\eta^{ac}M^{bd}+\eta^{bd}M^{as}-
\eta^{ad}M^{bc}-\eta^{bc}M^{ad})
\label{2}
\end{equation}
where $\eta^{ab}$ is the diagonal metric tensor such that
$\eta^{00}=-\eta^{11}=-\eta^{22}=-\eta^{33}=-\eta^{44}=1$.
\end{sloppypar}

An important observation is as follows. If we accept that the
symmetry on quantum level means that proper commutation relations
are satisfied (see the preceding section) then Eq. (\ref{2})
can be treated as the {\it definition} of the dS symmetry on quantum
level. In our system of units, all the operators $M^{ab}$ are 
dimensionless, in contrast to the situation with the Poincare
algebra, where the representation generators of the Lorentz
group are dimensionless while the momentum operators have the
dimension $(length)^{-1}$. For this reason it is natural to
think that the dS or AdS symmetries are more fundamental that
the Poincare symmetry. Note that such a definition does not
involve the cosmological constant at all. It appears only if 
one is interested in interpreting results in terms of
the dS spacetime or in the Poincare limit.

If one assumes that spacetime is fundamental then in the 
spirit of General Relativity it is natural to think that 
the empty space is flat, i.e. that the cosmological 
constant is equal to zero. This was the subject of the 
well-known dispute between Einstein and de Sitter. In the 
modern approach to the LQFT, the cosmological constant 
is given by a contribution of vacuum diagrams, 
and the problem is to explain why it is so small. On the 
other hand, if we assume that symmetry on quantum level in 
our formulation is more fundamental, then the problem of 
the cosmological constant does not exist at all. Instead we 
have a problem of why nowadays the Poincare symmetry is so 
good approximate symmetry. It seems natural to 
involve the anthropic principle for the explanation of 
this phenomenon (see e.g. Ref. \cite{Linde} and references 
therein).

There exists a wide literature devoted to 
UIRs of the dS group and algebra
(see e.g. Refs. \cite{Dix1,Tak,Hann,Str,Schwarz,
Men,Moy,Dobrev,Mielke,Klimyk,lev1,lev1a}).
In particular the first complete mathematical classification of the
UIRs has been given in Ref. \cite{Dix1}, three well-known
realizations of the UIRs have been first considered in Ref. \cite{Tak}
and their physical context has been first discussed in Ref. \cite{Hann}.

It is well known that for classification of UIRs of the dS group,
one should, strictly speaking, consider not the group SO(1,4) 
itself but its
universal covering group. The investigation carried out in
Refs. \cite{Dix1,Tak,Hann,Str,Moy} has shown that this 
involves only replacement of the SO(3) group by its universal 
covering group SU(2). Since this procedure is well known then
for illustrations we will work with the SO(1,4) group itself and 
follow a very elegant presentation for physicists in terms
of induced representations, given in the book \cite{Men}
(see also Refs. \cite{Dobrev,Barut,Str}). 
The elements of the SO(1,4) group can be described in the
block form
\begin{equation}
g=\left\|\begin{array}{ccc}
g_0^0&{\bf a}^T&g_4^0\\
{\bf b}&r&{\bf c}\\
g_0^4&{\bf d}^T&g_4^4
\end{array}\right\|\ 
\label{3}
\end{equation}
where 
\begin{equation}
\label{4}
{\bf a}=\left\|\begin{array}{c}a^1\\a^2\\a^3\end{array}\right\| \quad
{\bf b}^T=\left\|\begin{array}{ccc}b_1&b_2&b_3\end{array}\right\|
\quad r\in SO(3)
\end{equation}
(the subscript $^T$ means a transposed vector).

UIRs of the SO(1,4) 
group are induced from UIRs of the subgroup $H$ defined
as follows \cite{Str,Men,Dobrev}. Each element of $H$ can be uniquely
represented as a product of elements of the subgroups
SO(3), $A$ and ${\bf T}$: $h=r\tau_A{\bf a}_{\bf T}$ where 
\begin{equation}
\tau_A=\left\|\begin{array}{ccc}
cosh(\tau)&0&sinh(\tau)\\
0&1&0\\
sinh(\tau)&0&cosh(\tau)
\end{array}\right\|\ \quad
{\bf a}_{\bf T}=\left\|\begin{array}{ccc}
1+{\bf a}^2/2&-{\bf a}^T&{\bf a}^2/2\\
-{\bf a}&1&-{\bf a}\\
-{\bf a}^2/2&{\bf a}^T&1-{\bf a}^2/2
\end{array}\right\|\ 
\label{5}
\end{equation}

The subgroup $A$ is one-dimensional and the three-dimensional
group ${\bf T}$ is the dS analog of the conventional
translation group (see e.g. Ref. \cite{Men}). We hope it 
should not cause misunderstandings when 1 is used in its
usual meaning and when to denote the unit element of the
SO(3) group. It should also be clear when $r$ is a true
element of the SO(3) group or belongs to the SO(3) subgroup
of the SO(1,4) group. 

Let $r\rightarrow \Delta(r;{\bf s})$ be a UIR of the group
SO(3) with the spin ${\bf s}$ and 
$\tau_A\rightarrow exp(i\mu\tau)$ be a
one-dimensional UIR of the group $A$, where $\mu$ is a real
parameter. Then UIRs of the group $H$ used for inducing to
the SO(1,4) group, have the form
\begin{equation}
\Delta(r\tau_A{\bf a}_{\bf T};\mu,{\bf s})=
exp(i\mu\tau)\Delta(r;{\bf s})
\label{6}
\end{equation} 
We will see below that $\mu$ has the meaning of the dS
mass and therefore UIRs of the SO(1,4) group are
defined by the mass and spin, by analogy with UIRs
in Poincare invariant theory.

Let $G$=SO(1,4) and $X=G/H$ be a factor space (or
coset space) of $G$ over $H$. The notion of the factor 
space is well known (see e.g. Refs. 
\cite{Naimark,Dobrev,Str,Men,Barut}).
Each element $x\in X$ is a class containing the
elements $x_Gh$ where $h\in H$, and $x_G\in G$ is a
representative of the class $x$. The choice of
representatives is not unique since if $x_G$ is
a representative of the class $x\in G/H$ then
$x_Gh_0$, where $h_0$ is an arbitrary element
from $H$, also is a representative of the same 
class. It is well known that $X$ can be treated 
as a left $G$ space. This means that if $x\in X$
then the action of the group $G$ on $X$ can be
defined as follows: if $g\in G$ then $gx$ is a
class containing $gx_G$ (it is easy to verify
that such an action is correctly defined). 
Suppose that the choice of representatives
is somehow fixed. Then $gx_G=(gx)_G(g,x)_H$
where $(g,x)_H$ is an element of $H$. This
element is called a factor.

As noted in the preceding section, although we
can use well known facts about group representations,
our final goal is the construction of the
generators. The explicit form of the generators $M^{ab}$
depends on the choice of representatives in
the space $G/H$. As explained in several
papers devoted to UIRs of the SO(1,4) group
(see e.g. Ref. \cite{Men}), to obtain
the  possible closest analogy between UIRs of
the SO(1,4) and Poincare groups, one should proceed
as follows. Let ${\bf v}_L$ be a representative 
of the Lorentz group in the factor space SO(1,3)/SO(3)
(strictly speaking, we should consider $SL(2,c)/SU(2)$).
This space can be represented as the well known velocity
hyperboloid with the Lorentz invariant measure
\begin{equation}
d\rho({\bf v})=d^3{\bf v}/v_0
\label{7}
\end{equation}
where $v_0=(1+{\bf v}^2)^{1/2}$. Let $I\in SO(1,4)$ be a
matrix which formally has the same form as
the metric tensor $\eta$. One can show 
(see e.g. Ref. \cite{Men} for details) that 
$X=G/H$ can be represented as a union of three
spaces, $X_+$, $X_-$ and $X_0$ such that 
$X_+$ contains classes ${\bf v}_Lh$, $X_-$
contains classes ${\bf v}_LIh$ and $X_0$ is of
no interest for UIRs describing elementary particles 
since it has measure zero relative to the spaces
$X_+$ and $X_-$.

As a consequence of these results, the space of UIR
of the SO(1,4) algebra can be implemented as follows.  
If $s$ is the spin of the particle under 
consideration, then we
use $||...||$ to denote the norm in the space of 
UIR of the su(2) algebra with the spin $s$. 
Then the space of UIR in question is the space of 
functions $\{f_1({\bf v}),f_2({\bf v})\}$ on
two Lorentz hyperboloids with the range in the space of
UIR of the su(2) algebra with the spin $s$ and such that
\begin{equation}
\int\nolimits [||f_1({\bf v})||^2+
||f_2({\bf v})||^2]d\rho({\bf v}) <\infty
\label{8}
\end{equation}

We see that, in contrast with UIRs of the Poincare 
algebra (and AdS one), where UIRs are implemented on
one Lorentz hyperboloid, UIRs of the dS algebra can be
implemented only on two Lorentz hyperboloids, $X_+$
and $X_-$. Even this
fact (which is well known) is a strong indication 
that UIRs of the dS algebra might
have a natural interpretation in the framework of
Supposition 1.

In the case of the Poincare and AdS algebras, the positive
energy UIRs are implemented on an analog of $X_+$ and 
negative energy UIRs - on an analog of $X_-$. Since the 
Poincare and AdS groups
do not contain elements transforming these spaces
to one another, the positive and negative energy UIRs 
are fully independent. At the same time, the dS 
group contains 
such elements (e.g. $I$ \cite{Men,Dobrev,Mielke}) and for 
this reason its UIRs cannot be implemented only on 
one hyperboloid. 

\begin{sloppypar}
In Ref. \cite{hep} we have described all the technical details
needed for computing the explicit form of the generators $M^{ab}$.
In our system of units the results are as follows.
The action  of the generators on functions with the supporter in
$X_+$ has the form
\begin{eqnarray}
&&{\bf M}^{(+)}=2l({\bf v})+{\bf s},\quad {\bf N}^{(+)}==-2i v_0
\frac{\partial}{\partial {\bf v}}+\frac{{\bf s}\times {\bf v}}
{v_0+1}, \nonumber\\
&& {\bf B}^{(+)}=\mu {\bf v}+2i [\frac{\partial}{\partial {\bf v}}+
{\bf v}({\bf v}\frac{\partial}{\partial {\bf v}})+\frac{3}{2}{\bf v}]+
\frac{{\bf s}\times {\bf v}}{v_0+1},\nonumber\\
&& M_{04}^{(+)}=\mu v_0+2i v_0({\bf v}
\frac{\partial}{\partial {\bf v}}+\frac{3}{2})
\label{9}
\end{eqnarray}
where ${\bf M}=\{M^{23},M^{31},M^{12}\}$,
${\bf N}=\{M^{01},M^{02},M^{03}\}$,
${\bf B}=-\{M^{14},M^{24},M^{34}\}$, ${\bf s}$ is the spin operator,
and ${\bf l}({\bf v})=-i{\bf v}
\times \partial/\partial {\bf v}$.
At the same time, the action of the generators on 
functions with the supporter 
in $X_-$ is given by
\begin{eqnarray}
&&{\bf M}^{(-)}=2l({\bf v})+{\bf s},\quad {\bf N}^{(-)}==-2i v_0
\frac{\partial}{\partial {\bf v}}+\frac{{\bf s}\times {\bf v}}
{v_0+1}, \nonumber\\
&& {\bf B}^{(-)}=-\mu {\bf v}-2i [\frac{\partial}{\partial {\bf v}}+
{\bf v}({\bf v}\frac{\partial}{\partial {\bf v}})+\frac{3}{2}{\bf v}]-
\frac{{\bf s}\times {\bf v}}{v_0+1},\nonumber\\
&& M_{04}^{(-)}=-\mu v_0-2i v_0({\bf v}
\frac{\partial}{\partial {\bf v}}+\frac{3}{2})
\label{10}
\end{eqnarray}
\end{sloppypar}

In view of the fact that SO(1,4)=SO(4)$AT$ and $H$=SO(3)$AT$, there
also exists a choice of representatives which is probably even
more natural than that described above \cite{Men,Dobrev,Moy}. 
Namely, we can choose as
representatives the elements from the coset space SO(4)/SO(3).
Since the universal covering group for SO(4) is SU(2)$\times$SU(2)
and for SO(3) --- SU(2), we can choose as representatives the
elements of the first multiplier in the product SU(2)$\times$SU(2). 
Elements of SU(2) can be represented by the points $u=({\bf u},u_4)$
of the three-dimensional sphere $S^3$ in the four-dimensional
space as $u_4+i{\bf \sigma u}$ where ${\bf \sigma}$ are the Pauli
matrices and $u_4=\pm (1-{\bf u}^2)^{1/2}$ for the upper and
lower hemispheres, respectively. Then the calculation of the
generators is similar to that described above. Since such a form
of generators will be needed only for illustrative purposes, we
will not discuss technical details and describe only the 
result.

The Hilbert space for such a choice of representatives is 
the space of functions $\varphi (u)$ on $S^3$ 
with the range in the space of the UIR of the su(2) algebra 
with the spin $s$ and such that
\begin{equation}
\int\nolimits ||\varphi(u)||^2du <\infty
\label{11}
\end{equation}
where $du$ is the SO(4) invariant volume element on $S^3$.
The explicit calculation  shows  that  the  generators for  this
realization have the form
\begin{eqnarray}
&&{\bf M}=2l({\bf u})+{\bf s},\quad {\bf B}=2\imath u_4
\frac{\partial}{\partial {\bf u}}-{\bf s}, \nonumber\\
&& {\bf N}=2\imath [\frac{\partial}{\partial {\bf u}}-
{\bf u}({\bf u}\frac{\partial}{\partial {\bf u}})]
-(\mu +3\imath){\bf u}+{\bf u}\times {\bf s}-u_4{\bf s},\nonumber\\
&& M_{04}=(\mu +3\imath)u_4+2\imath u_4{\bf u}
\frac{\partial}{\partial {\bf u}}
\label{12}
\end{eqnarray}

Since Eqs. (\ref{8}-\ref{10}) on the one hand and
Eqs. (\ref{11}) and (\ref{12}) on  the
other  are  the  different  realization  of  one  
and   the   same
representation, there exists a unitary operator transforming
functions $f(v)$ into $\varphi (u)$ and operators 
(\ref{9},\ref{10}) into
operators (\ref{12}). For example in the spinless case the
operators (\ref{9}) and (\ref{12}) are related to each other
by a unitary transformation 
\begin{equation}
\varphi (u)=exp(-\frac{\imath}{2}\,\mu \,lnv_0)v_0^{3/2}f(v)
\label{13}
\end{equation}
where ${\bf u}=-{\bf v}/v_0$. 

In view of this relation, the
sphere $S^3$ is usually interpreted in the literature as the
velocity space (see e.g. Refs. \cite{Men,Dobrev,Moy,Mielke}). 
As argued
in Refs. \cite{lev1,lev1a,lev3}, there also exist reasons 
to interpret
$S^3$ as the coordinate space. However, the behavior of a particle
in the dS space is rather unusual (see e.g. Refs.
\cite{Men,Mielke,Narlikar}). For this reason the standard 
physical intuition
is expected to work only for elements of the full 
Hilbert space
which become physical states in the Poincare limit. 
Unitary transformations 
similar to those in Eq. (\ref{13}) transform such states in
such a way that the standard contraction to the 
Poincare group 
is impossible for them. For these reasons, $S^3$ probably 
does not have
a universal interpretation (see also Sect. \ref{S7}). 

\section{Poincare limit}
\label{S3}

A general notion of contraction has been developed in 
Ref. \cite{IW}. In our case it can be performed
as follows. Let us assume that $\mu > 0$ and denote
$m=\mu /2R$, ${\bf P}={\bf B}/2R$ and $E=M_{04}/2R$.
Then, as follows from  Eq. (\ref{9}), in the limit
when $R\rightarrow \infty$, $\mu\rightarrow \infty$
but $\mu /R$ is finite,   
one obtains a standard representation of the
Poincare algebra for a particle with the mass $m$ such 
that ${\bf P}=m{\bf v}$ is the particle momentum
and $E=mv_0$ is the particle energy. In that case
the generators of the Lorentz algebra have the same form
for the Poincare and dS algebras. Analogously the
operators given by Eq. (\ref{10}) are contracted to
ones describing negative energy UIRs of the Poincare
algebra.

In the standard interpretation of UIRs
it is assumed that each element of the full 
representation space represents a possible physical
state for the elementary particle in question. 
It is also well known (see e.g. Ref. 
\cite{Dobrev,Men,Moy,Mielke})
that the dS group contains elements (e.g. $I$)
such that the corresponding representation operator
transforms positive energy states to negative energy
ones and {\it vice versa}. Are these properties 
compatible with the fact that in the Poincare
limit there exist states with negative energies?

One might say that the choice of the energy sign is
only a matter of convention. For example, in the
standard theory we can define the energy not as
$(m^2+{\bf p}^2)^{1/2}$ but as 
$-(m^2+{\bf p}^2)^{1/2}$. However, let us consider,
for example, a system of two free noninteracting
particles. The fact that they do not interact means
mathematically that the representation describing
the system is the tensor product of single-particle
UIRs. The generators of the tensor product are equal
to sums of the corresponding single-particle
generators. In the Poincare limit the energy and
momentum can be chosen diagonal. If we assume that
both positive and negative energies are possible 
then a system of two free particles with the equal
masses can have the same quantum numbers as the
vacuum (for example, if the first particle has the
energy $E$ and momentum ${\bf p}$ while the second
one has the energy $-E$ and the momentum $-{\bf p}$)
what obviously contradicts experiment. For this
and other reasons it is well known that in the
Poincare invariant theory all the particles in
question should have the same energy sign.

We conclude that UIRs of the dS algebra cannot be 
interpreted in the standard way since such an
interpretation is physically meaningless even in
the Poincare limit. Although our approach 
considerably differs from that in LQFT in curved
spacetime, this conclusion is in agreement with
that in Refs. \cite{Narlikar,Susskind} and references 
therein (see Sect. \ref{S1}). 

In the framework of Supposition 1, one could try to
interpret the operators (\ref{10}) as those describing
a particle while the operators (\ref{11}) as those
describing the corresponding antiparticle. Such a
program has been implemented in Ref. \cite{hep}.
If one requires that the dS Hamiltonian should be
positive definite in the Poincare limit, then, as shown in Ref.
\cite{hep}, the annihilation and creation operators for
the particle and antiparticle in question can satisfy
only anticommutation relations, i.e. the particle and
antiparticle can be only fermions. 

If one assumes that the dS algebra
is the symmetry algebra in the elementary particle theory
then one has to consider not only the limit when
the contraction to the Poincare algebra is possible, but
the general case as well. This is just the goal of the 
present paper.

Concluding this section, let us note the following.
As assumed by Mensky \cite{Men}, UIRs of the dS
group could be a basis for new approaches to the CPT
theorem. We believe that Supposition 1 is in the spirit
of Mensky's idea. Indeed, a comparison of Eqs.
(\ref{9}) and (\ref{10}) shows that the operators
$M_{ab}$ in these expressions not containing the 
subscript 4 are the same while those containing 
this subscript have different signs (the operator
$M_{44}$ is of no interest since it is identical 
zero). If the coordinates $x^{\mu}$ ($\mu=0,1,2,3$)
are inversed (i.e. one applies the PT transformation)
then the operators $M_{\mu 4}$ should change their
sign while the other operators remain unchanged.
In particular, a positive definite Hamiltonian 
becomes negative definite. To avoid such an
undesirable behavior we can relate the new Hamiltonian
to antiparticles and quantize it in a proper way.
In other words, the PT transformation should be
necessarily accompanied by transition from particles to
antiparticles and {\it vice versa}, i.e. the PT 
transformation should be replaced by the CPT one.    

\section{UIRs in the su(2)$\times$su(2) basis}
\label{S4}

Proceeding from the method of su(2)$\times$su(2) shift
operators, developed by Hughes \cite{Hug} for constructing
UIRs of the group SO(5), and following Ref. \cite{lev3},
we now give a pure algebraic description of UIRs of
the so(1,4) algebra. It will be convenient for us to  deal
with the set of operators $({\bf J}',{\bf J}",R_{ij})$ ($i,j=1,2$)
instead  of $M^{ab}$.
Here ${\bf J}'$ and ${\bf J}"$ are two independent su(2) algebras
(i.e. $[{\bf J}',{\bf J}"]=0$).
In each of them one chooses as the basis the operators  
$(J_+,J_-,J_3)$ such that in our system of units 
$J_1=J_++J_-$, $J_2=-\imath (J_+-J_-)$ and the commutation
relations have the form
\begin{equation}
[J_3,J_+]=2J_+,\quad [J_3,J_-]=-2J_-,\quad [J_+,J_-]=J_3
\label{14}
\end{equation}

The commutation relations of the operators ${\bf J}'$ and
${\bf J}"$  with $R_{ij}$ have the form
\begin{eqnarray}
&&[J_3',R_{1j}]=R_{1j},\quad [J_3',R_{2j}]=-R_{2j},\quad
[J_3",R_{i1}]=R_{i1},\nonumber\\
&& [J_3",R_{i2}]=-R_{i2},\quad
[J_+',R_{2j}]=R_{1j},\quad [J_+",R_{i2}]=R_{i1},\nonumber\\
&&[J_-',R_{1j}]=R_{2j},\quad [J_-",R_{i1}]=R_{i2},\quad
[J_+',R_{1j}]=\nonumber\\
&&[J_+",R_{i1}]=[J_-',R_{2j}]=[J_-",R_{i2}]=0,\nonumber\\
\label{15}
\end{eqnarray}
and the commutation relations of the operators $R_{ij}$ 
with each other have the form
\begin{eqnarray}
&&[R_{11},R_{12}]=2J_+',\quad 
[R_{11},R_{21}]=2J_+",\nonumber\\
&& [R_{11},R_{22}]=-(J_3'+J_3"),\quad 
[R_{12},R_{21}]=J_3'-J_3"\nonumber\\
&& [R_{11},R_{22}]=-2J_-",\quad [R_{21},R_{22}]=-2J_-'
\label{16}
\end{eqnarray}

The relation between the sets $({\bf J}',{\bf J}",R_{ij})$ and
$M^{ab}$  is given by
\begin{eqnarray}
&&{\bf M}={\bf J}'+{\bf J}", \quad {\bf B}={\bf J}'-{\bf J}",
\quad M_{01}=\imath (R_{11}-R_{22}), \nonumber\\
&& M_{02}=R_{11}+R_{22}, \quad  
M_{03}=-i(R_{12}+R_{21}),\nonumber\\
&&  M_{04}=R_{12}-R_{21}
\label{17}
\end{eqnarray}
Then it is easy to see that Eq. (\ref{2}) 
follows from Eqs. (\ref{15}-\ref{17}) and {\it vice versa}.

Consider the space of maximal  $su(2)\times su(2)$  vectors,
i.e.  such vectors $x$ that $J_+'x=J_+"x=0$. Then from
Eqs. (\ref{15}) and (\ref{16}) it follows that the operators
\begin{eqnarray}
&&A^{++}=R_{11},\quad  A^{+-}=R_{12}-
J_-"R_{11}(J_3"+1)^{-1},  \nonumber\\
&&A^{-+}=R_{21}-J_-'R_{11}(J_3'+1)^{-1},\nonumber\\
&&A^{--}=-R_{22}+J_-"R_{21}(J_3"+1)^{-1}+\nonumber\\
&&J_-'R_{12}(J_3'+1)^{-1}-J_-'J_-"R_{11}[(J_3'+1)(J_3"+1)]^{-1}
\label{18}
\end{eqnarray}
act invariantly on this space.
The notations are related to the property  that
if $x^{kl}$  ($k,l>0$) is the maximal su(2)$\times$su(2)
vector and simultaneously
the eigenvector of operators $J_3'$ and $J_3"$ with the 
eigenvalues $k$ and $l$, respectively, then  $A^{++}x^{kl}$
is  the  eigenvector  of  the  same
operators with the values $k+1$ and $l+1$, $A^{+-}x^{kl}$ - the
eigenvector  with
the values $k+1$ and $l-1$, $A^{-+}x^{kl}$ - 
the eigenvector with the values  $k-1$ and $l+1$ and 
$A^{--}x^{kl}$ - the eigenvector with the
values $k-1$ and $l-1$.

As follows from Eq. (\ref{14}), the vector $x_{ij}^{kl}
=(J_-')^i(J_-")^jx^{kl}$ is
the eigenvector of the operators $J_3'$ and $J_3"$ with 
the eigenvalues $k-2i$ and $l-2j$, respectively. 
Since 
$${\bf J}^2=J_3^2-2J_3+4J_+J_-=J_3^2+2J_3+4J_-J_+$$
is the Casimir operator for the ${\bf J}$  algebra, and the
Hermiticity condition can be written as $J_-^*=J_+$, 
it  follows  in  addition that 
\begin{equation}
{\bf J}^{'2}x_{ij}^{kl}=k(k+2)x_{ij}^{kl},\quad
{\bf J}^{"2}x_{ij}^{kl}=l(l+2)x_{ij}^{kl}
\label{19}
\end{equation}
\begin{equation}
J_+'x_{ij}^{kl}=i (k+1-i)x_{i -1,j}^{kl},
\quad  J_+"x_{ij}^{kl}=j (l+1-j)x_{i,j -1}^{kl}
\label{20}
\end{equation}
\begin{equation}
(x_{ij}^{kl},x_{ij}^{kl}))=
\frac{i !j !k!l!}{(k-i !)(l-j !)}(x^{kl},x^{kl})
\label{21}
\end{equation}
where $(...,...)$ is the scalar product in the representation space.
From these formulas it follows that the action of 
the operators ${\bf J}'$
and ${\bf J}"$ on $x^{kl}$  generates a space with the  
dimension $(k+1)(l+1)$  and  the
basis $x_{ij}^{kl}$ ($i=0,1,...k$, $j=0,1,...l$).
Note that the vectors $x_{ij}^{kl}$   are
orthogonal but we deliberately do not normalize them to 
unity since the normalization (\ref{21}) will be convenient below.

The Casimir operator of the second order for the 
algebra  (\ref{2}) can be written as
\begin{eqnarray}
&&I_2 =-\frac{1}{2}\sum_{ab} M_{ab}M^{ab}=\nonumber\\
&&4(R_{22}R_{11}-R_{21}R_{12}-J_3')-
2({\bf J}^{'2}+{\bf J}^{"2})
\label{22}
\end{eqnarray}
A direct calculation shows that for the
generators given by Eqs. (\ref{9}), (\ref{10}) and 
(\ref{12}), $I_2$  has the numerical  value
\begin{equation}
I_2 =w-s(s+2)+9
\label{23}
\end{equation}
where $w=\mu^2$. As noted in the preceding section,
$\mu = 2mR$ where $m$ is the conventional mass. If $m \neq 0$
then $\mu$ is very big since $R$ is very big. We conclude
that for massive UIRs the quantity $I_2$ is a big positive
number.

 The basis in the representation space
can be explicitly constructed assuming that there exists a
vector $e^0$ which is the maximal su(2)$\times$su(2)
vector such that
\begin{equation}
J_3'e_0=n_1e_0\quad J_3"e_0=n_2e_0
\label{24}
\end{equation}
and $n_1$ is the minimum possible eigenvalue of $J_3'$ in
the space of the maximal vectors. Then $e_0$ should also
satisfy the conditions
\begin{equation}
A^{--}e_0=A^{-+}e_0=0
\label{25}
\end{equation}
We use ${\tilde I}$ to denote the operator 
$R_{22}R_{11}-R_{21}R_{12}$.
Then as follows from Eqs. (\ref{15}), (\ref{16}), (\ref{18}),
(\ref{22}), (\ref{24}) and (\ref{25}),
$${\tilde I}n_1e_0=2n_1(n_1+1)e_0.$$
Therefore, if $n_1\neq 0$ the vector $e_0$ is the eigenvector
of the operator ${\tilde I}$ with the eigenvalue 
$2(n_1+1)$ and the
eigenvector of the operator $I_2$ with the eigenvalue
$$-2[(n_1+2)(n_2-2)+n_2(n_2+2)].$$ 
The latter is obviously incompatible with Eq. (\ref{23})
for massive UIRs. Therefore the compatibility can be
achieved only if $n_1=0$. In that case we use $s$ to denote
$n_2$ since it will be clear soon that the value of $n_2$
indeed has the meaning of spin. Then, as follows from 
Eqs. (\ref{23}) and (\ref{24}), the vector $e_0$ should 
satisfy the conditions  
\begin{eqnarray}
&&{\bf J}'e^0=J_+"e^0=0,\quad J_3"e^0 =se^0, \nonumber\\
&&I_2e^0 =[w-s(s+2)+9] e^0
\label{26}
\end{eqnarray}
where $w,s>0$ and $s$ is an integer.  

Define the vectors
\begin{equation}
e^{nr}=(A^{++})^n(A^{+-})^re^0
\label{27}
\end{equation}
Then a direct calculation taking into account Eqs.
(\ref{14})-(\ref{16}), (\ref{18}), (\ref{19}), (\ref{22}), 
(\ref{25}) and (\ref{26}) gives
\begin{equation}
A^{++}e^{nr}=e^{n+1,r}\quad A^{+-}e^{nr}=
\frac{s-r+1}{n+s-r+1}e^{n,r+1}
\label{28}
\end{equation}
\begin{equation}
A^{--}e^{nr}=-\frac{n(n+s+1)[w+(2n+s+1)^2]}
{4(n+r+1)(n+s-r+1)}e^{n-1,r}
\label{29}
\end{equation}
\begin{equation}
A^{-+}e^{nr}=-\frac{r(s+1-r)[w+(s+1-2r)^2]}
{4(n+r+1)(s+2-r)}e^{n,r-1}
\label{30}
\end{equation}
As follows from Eqs. (\ref{29}) and (\ref{30}), 
the possible values of $n$ are $n=0,1,2,...$ while 
$r$ can take only the values of $0,1,....s$
(and therefore $s$ indeed has the meaning of the 
particle spin). 
Since $e^{nr}$ is the maximal $su(2)\times su(2)$ vector with the
eigenvalues of the operators ${\bf J}'$ and ${\bf J}"$ equal to
$n+r$ and $n+s-r$, respectively, then
as a basis of the representation space one can take the vectors
$e_{ij}^{nr}=(J_-')^i (J_-")^je^{nr}$
where, for the given $n$ and $s$, the quantity $i$ can
take the values of $0,1,...n+r$ and $j$ - the values 
of $0,1,...n+s-r$. 

A direct calculation shows that 
$Norm(i,j,n,r)=(e_{ij}^{nr},e_{ij}^{nr})$ can be represented
as
\begin{equation}
Norm(i,j,n,r)=G(n,r)F(i,j,n,r)
\label{31}
\end{equation}
where
\begin{eqnarray}
&&G(n,r)=\frac{(s+1-r)r!n!(n+1+s)!(s+1-r)!}
{4^{(n+r)}(s+1)(s+1)!(n+r+1)(n+s-r+1)}\nonumber\\
&&\{1/[w+(s+1)^2]\}\prod_{l=-r}^{n}[w+(s+1+2l)^2]
\label{32}
\end{eqnarray}
\begin{equation}
F(i,j,n,r)=i!j!/[n+r-i)!(n+s-r-j)!]
\label{33}
\end{equation}
It is obvious that for the allowed values of $(ijnr)$
the norm defined by Eq. (\ref{31}) is positive definite.

One can show that the basis discussed in this section
is an implementation of the generators (\ref{12}) but
not (\ref{9}) and (\ref{10}). In particular, the elements
$e^{0r}$ correspond to functions $\varphi(u)$ not
depending on $u$. If the elements $e^{0r}$ are interpreted
as rest states then $S^3$ could be interpreted as the
coordinate space since the wave function of the rest
space does not depend on coordinates. However, let us
stress again, that the generators (\ref{12}) do not
allow a direct contraction to the Poincare group, and
the standard intuition does not apply.

\section{Quantization of representation generators}
\label{S5}
 
In standard approach to quantum theory, the operators of 
physical quantities act in the Fock space of the system under 
consideration. Suppose that the system consists
of free particles and their antiparticles.
Strictly speaking, in our approach it is not clear yet
what should be treated as a particle or antiparticle.
The considered UIRs of the dS algebra describe objects
such that $(ijnr)$ is the full set of
their quantum numbers. Therefore we can introduce the
annihilation and creation operators 
$(a(i,j,n,r),a(i,j,n,r)^*)$ for these objects. 
Taking into account the fact that the elements $e^{nr}_{ij}$
are not normalized to one (see Eq. (\ref{31})), we require
that in the case of anticommutation relations the 
operators $(a(i,j,n,r),a(i,j,n,r)^*)$ satisfy the conditions
\begin{equation}
\{a(i,j,n,r),a(i',j',n',r')^*\}=Norm(i,j,n,r)
\delta_{ii'}\delta_{jj'}\delta_{nn'}\delta_{rr'}
\label{34}
\end{equation}
while in the case of commutation relations
\begin{equation}
[a(i,j,n,r),a(i',j',n',r')^*]=Norm(i,j,n,r)
\delta_{ii'}\delta_{jj'}\delta_{nn'}\delta_{rr'}
\label{35}
\end{equation}
In the first case, any two $a$-operators or any two
$a^*$ operators anticommute with each other while
in the second case they commute with each other.

Eqs. (\ref{34}) and (\ref{35}) have a clear physical
meaning if $a$ is the annihilation operator and $a^*$
is the creation operator. A discussion whether
such an interpretation is consistent will be
given in Sect. \ref{S7}.

The problem of quantization of representation 
generators can now be formulated as follows. 
One should construct a linear map ${\it F}$ from 
the Lie algebra of representation generators to a 
Lie algebra of operators acting in the Fock space. 
If ${\it F}(M_{ab})$ is the image of the operator
$M_{ab}$ in the Fock space and ${\it F}(M_{cd})$ 
is the image of the operator $M_{cd}$ then the
image of $[M_{ab},M_{cd}]$ should be equal to  
$[{\it F}(M_{ab}),{\it F}(M_{cd})]$.
In other words, we should have a homomorphism
of Lie algebras of operators acting in the space of UIR
and in the Fock space. We can also require that our 
map should be compatible with the Hermitian
conjugation in both spaces. In what follows it will
be always clear whether an operator acts in the
space of UIR or in the Fock space. For this reason
we will not use the notation ${\it F}(M_{ab})$
and will simply write $M_{ab}$ instead.

The matrix elements of the operator $M_{ab}$ in
the space of UIR can be defined as
\begin{equation}
M_{ab}e^{nr}_{ij}=\sum_{i'j'n'r'}
M_{ab}(i',j',n',r';i,j,n,r)e^{n'r'}_{i'j'}
\label{36}
\end{equation}
where the sum is taken over all possible values of
$(i'j'n'r')$ for the UIR in question. Then one can
verify that if the image of the operator $M_{ab}$
in the Fock space is defined as
\begin{eqnarray}
&&M_{ab}=\sum_{i'j'n'r'ijnr}
M_{ab}(i',j',n',r';i,j,n,r)\nonumber\\
&&a(i',j',n',r')^*a(i,j,n,r)/Norm(i,j,n,r)
\label{37}
\end{eqnarray}
then the images of any two representation generators
will properly commute with each other if the
operators $(a(i,j,n,r),a(i,j,n,r)^*)$ satisfy either
Eq. (\ref{34}) or Eq. (\ref{35}).

Our next goal is to write down explicit expressions
for $M_{ab}$ in quantized form. 

Since the elements $e^{nr}_{ij}$ are the eigenvectors
of the operators $J_3'$ and $J"_3$ with the eigenvalues
$n+r-2i$ and $n+s-r-2j$, respectively, the form of
the matrix elements of these operators is obvious.
Then, as follows from Eq. (\ref{37}), the operators
$J_3'$ and $J_3"$ in quantized form can
be written as
\begin{equation}
J_3'=\sum_{ijnr}(n+r-2i)a(i,j,n,r)^*a(i,j,n,r)/Norm(i,j,n,r)
\label{38}
\end{equation}
\begin{equation}
J_3"=\sum_{ijnr}(n+s-r-2j)a(i,j,n,r)^*a(i,j,n,r)/Norm(i,j,n,r)
\label{39}
\end{equation}

Since $J_-'e^{nr}_{ij}=e^{nr}_{i+1,j}$ and
$J_-"e^{nr}_{ij}=e^{nr}_{i,j+1}$, the quantized
form of the operators $J_-'$ and $J_-"$ is as follows
\begin{equation}
J_-'=\sum_{ijnr}a(i+1,j,n,r)^*a(i,j,n,r)/Norm(i,j,n,r)
\label{40}
\end{equation}
\begin{equation}
J_-"=\sum_{ijnr}a(i,j+1,n,r)^*a(i,j,n,r)/Norm(i,j,n,r)
\label{41}
\end{equation}

The quantized form of the operators $J_+'$
and $J_+"$ easily follows from Eq. (21):
\begin{equation}
J_+'=\sum_{ijnr}i(n+r+1-i)a(i-1,j,n,r)^*a(i,j,n,r)/Norm(i,j,n,r)
\label{42}
\end{equation}
\begin{equation}
J_+"=\sum_{ijnr}j(n+s-r+1-j)a(i,j-1,n,r)^*a(i,j,n,r)/Norm(i,j,n,r)
\label{43}
\end{equation}

The expressions for the quantized form of the
operators $R_{\alpha\beta}$ ($\alpha,\beta =1,2$) can
be obtained as follows. First one can use the fact that
$e^{nr}_{ij}=(J_-')^i(J_-")^je^{nr}$. Therefore, by using
the commutation relations (\ref{15}) one can express the
action of $R_{\alpha\beta}$ on $e^{nr}_{ij}$ in terms of
the action of $R_{\alpha\beta}$ on $e^{nr}$. Since $e^{nr}$
is the maximal su(2)$\times$su(2) vector, then by using 
Eq. (\ref{18}) one can express the action of 
$R_{\alpha\beta}$ on $e^{nr}$ in terms of the action
of the $A$-operators on $e^{nr}$. The final expressions
for the matrix elements can be obtained by using 
Eqs. (\ref{28}-\ref{30}) and then the final expressions
in the quantized form --- by using Eq. 
(\ref{37}). The result of the calculations is as follows.
\begin{eqnarray}
&&R_{11}=\sum_{ijnr} \{4(n+r+1-i)(n+s-r+1-j)
a(i,j,n+1,r)^*-\nonumber\\
&&4j(s+1-r)(n+r+1-i)a(i,j-1,n,r+1)^*+\nonumber\\
&&ir(s+1-r)(n+s-r+1-j)[w+(s+1-2r)^2]\nonumber\\
&&a(i-1,j,n,r-1)^*/(s+2-r) +\nonumber\\ 
&&ijn(n+s+1)[w+(2n+s+1)^2]a(i-1,j-1,n-1,r)^*\}\nonumber\\
&&a(i,j,n,r)/[4(n+r+1)(n+s-r+1)Norm(i,j,n,r)]
\label{44}
\end{eqnarray}
\begin{eqnarray}
&&R_{21}=\sum_{ijnr} \{-jn(n+s+1)[w+(2n+s+1)^2]\nonumber\\
&&a(i,j-1,n-1,r)^*-(n+s-r+1-j)r(s+1-r)\nonumber\\
&&[w+(s+1-2r)^2]a(i,j,n,r-1)^*/(s+2-r)-\nonumber\\
&&4j(s+1-r)a(i+1,j-1,n,r+1)^*+\nonumber\\
&&4(n+s-r+1-j)a(i+1,j,n+1,r)^*\}\nonumber\\
&&a(i,j,n,r)/[4(n+r+1)(n+s-r+1)Norm(i,j,n,r)]
\label{45}
\end{eqnarray}
\begin{eqnarray}
&&R_{12}=\sum_{ijnr}\{-in(n+s+1)[w+(2n+s+1)^2]\nonumber\\
&&a(i-1,j,n-1,r)^*+ir(s+1-r)[w+(s+1-2r)^2]\nonumber\\
&&a(i-1,j+1,n,r-1)^*/(s+2-r)+\nonumber\\
&&4(s+1-r)(n+r+1-i)a(i,j,n,r+1)^*+\nonumber\\
&&4(n+r+1+1-i)a(i,j+1,n+1,r)^*\}\nonumber\\
&&a(i,j,n,r)/[4(n+r+1)(n+s-r+1)Norm(i,j,n,r)]
\label{46}
\end{eqnarray}
\begin{eqnarray}
&&R_{22}=\sum_{ijnr}\{n(n+s+1)[w+(2n+s+1)^2]\nonumber\\
&&a(i,j,n-1,r)^*-r(s+1-r)[w+\nonumber\\
&&(s+1-2r)^2]a(i,j+1,n,r-1)^*/(s+2-r)+\nonumber\\
&&4(s+1-r)a(i+1,j,n,r+1)^*+4a(i+1,j+1,n+1,r)^*\}\nonumber\\
&&a(i,j,n,r)/[4(n+r+1)(n+s-r+1)Norm(i,j,n,r)]
\label{47}
\end{eqnarray}

\section{Problem of physical and nonphysical states}
\label{Physical}

Consider now the following question. As noted in 
Sect. \ref{S3}, in the Poincare limit the 
operator ${\bf B}$ is such that ${\bf P}={\bf B}/2R$ 
becomes the ordinary momentum. In this limit
the energy $E=M_{04}/2R$ commutes with ${\bf P}$,
and the sign of the energy is a good criterion for
distinguishing physical and nonphysical states.
However, if the basis is chosen as in Eq. (\ref{12}) or 
Sect. \ref{S4} then the operator ${\bf B}$ has
no longer such a simple meaning and the {\it standard}
contraction is impossible. Nevertheless there should
exist conditions when the Poincare algebra is an
approximate symmetry algebra. What are these conditions?

Note that if ${\bf p}$ is the ordinary momentum
and $p=|{\bf p}|$ then in the Poincare limit 
$|{\bf B}|$ is of order $pR$, i.e. is much greater 
than the ordinary angular momentum. We can assume
that at the conditions when the Poincare algebra is
the approximate symmetry algebra, the value of
$|{\bf B}|$ is still very big.
As follows from Eq. (\ref{17}), $|{\bf B}|$ is 
much greater than $|{\bf M}|$ if ${\bf J}'$ 
and ${\bf J}"$ have approximately the same 
magnitude and are
approximately anticollinear. Since in the state
$e^{nr}_{ij}$ the magnitude of ${\bf J}'$ is 
$n+r$, the magnitude of ${\bf J}"$ is $n+s-r$
and $r=0,1,...s$, then it is natural to think
that in the Poincare limit the quantity $n$ is
very big and much greater than $s$ and $r$.

\begin{sloppypar}
Let us calculate the dS energy operator $M_{04}$
in the limit when $n$ is much greater than $s$
and $r$, and $i$ and $j$ are of order $n$. 
It will be convenient for this purpose to normalize
the basis vectors not to $Norm(i,j,n,r)$, but to
one. Accordingly, the operators $(a,a^*)$ should
now satisfy not the condition (\ref{34}) but
\begin{equation}
\{a(i,j,n,r),a(i',j',n',r')^*\}=
\delta_{ii'}\delta_{jj'}\delta_{nn'}\delta_{rr'}
\label{59}
\end{equation}
and analogously in the case of Eq. (\ref{35}). 
This can be achieved if the operators $a(i,j,n,r)$
in Eqs. (\ref{38}-\ref{47}) are replaced by
$Norm(i,j,n,r)^{1/2}a(i,j,n,r)$. Then a direct
calculation using Eqs. (\ref{31}-\ref{33}) gives
that in the above approximation Eq. (\ref{45}) becomes
\begin{eqnarray}
&&R_{21}=\sum_{ijnr}\{[i(n-j)]^{1/2}
a(i+1,j,n+1,r)^*-[j(n-i)]^{1/2}\nonumber\\
&&a(i,j-1,n-1,r)^*\}(w+4n^2)^{1/2}a(i,j,n,r)/2n
\label{60}
\end{eqnarray}
and Eq. (\ref{46}) becomes.
\begin{eqnarray}
&&R_{12}=\sum_{ijnr}\{[j(n-i)]^{1/2}
a(i,j+1,n+1,r)^*-[i(n-j)]^{1/2}\nonumber\\
&&a(i-1,j,n-1,r)^*\}(w+4n^2)^{1/2}a(i,j,n,r)/2n
\label{61}
\end{eqnarray}
Now, as follows from Eqs. (\ref{17}), (\ref{60})
and (\ref{61}), in this approximation $e^{nr}_{ij}$
is the eigenvector of the operator $M_{04}$ with
the eigenvalue
$$E=(w+4n^2)^{1/2}\{[j(n-i)]^{1/2}-[i(n-j)]^{1/2}\}/n.$$ 
It is obvious that $E>0$ if $j>i$ and $E<0$ if $j<i$. 
\end{sloppypar}


The element $e^{nr}_{ij}$ is the eigenvector of the operator 
$B^3=J_3'-J_3"$ with the eigenvalue $B^3(i,j,r)=2(r+j-i)-s$. 
In the above approximation $B^3(i,j,r)=2(j-i)$ 
and therefore the condition $B^3(i,j,r)>0$ is equivalent to 
$E>0$ and the condition $B^3(i,j,r)<0$ is equivalent to $E<0$.
Note also that if $i\neq j$ then $|E|>2|j-i|$ if $w\geq 0$,
by analogy with the standard case. At the same time,
in contrast with the standard case, the quantities $E$
and $2(j-i)$ always have the same sign.

Let us recall (see Sect. \ref{S4}) that the quantity
$i$ can take the values $0,1,...N_1(n,r)$ and $j$
can take the values $0,1,...N_2(n,r)$ where 
$N_1(n,r)=n+r$ and $N_2=n+s-r$. The vector $e^{nr}_{ij}$
is the eigenvector of the operator $J_3'$ with the
eigenvalue $n+r-2i$ and the eigenvector of the operator 
$J_3"$ with the eigenvalue $n+s-r-2j$. Therefore when
$i\rightarrow N_1(n,r)-i$ and $j\rightarrow N_2(n,r)-j$,
the eigenvalues change their signs and 
$B^3(i,j,r)\rightarrow -B^3(i,j,r)$. In view of this 
observation, the following question arises. If
$e^{nr}_{ij}$ is a physical state then is 
$e^{nr}_{N_1(n,r)-i,N_2(n,r)-j}$ a physical state too?

As follows from the above discussion, if $n$ is big,
$e^{nr}_{ij}$ and $e^{nr}_{N_1(n,r)-i,N_2(n,r)-j}$
are also eigenvectors of the operator $M_{04}$ with
the opposite eigenvalues. If the both states are
treated as physical, we can consider a system of two
particles with the equal mass and spin, such that the
first particle is in the state $e^{nr}_{ij}$ and the
second one - in the state 
$e^{nr}_{N_1(n,r)-i,N_2(n,r)-j}$. Such a system is
the eigenstate of the operators $(J_3'J_3"M_{04})$
with all the eigenvalues equal to zero. 

How can we investigate the system in the general
case, when $n$ is arbitrary? One of the possibilities
is to use the theory of decomposition of the tensor
product of two induced UIRs into UIRs 
\cite{Naimark,Klimyk2,Barut}. As follows from the
results of Chapt. 18 in Ref. \cite{Barut}, in the 
given case of two equal masses and spins one has
to induce the tensor product of two UIRs 
$\Delta(r;{\bf s})$ in SO(1,4) and decompose it
into UIRs. The latter task is not easy from the
technical point of view. Another possible way is
as follows. We can use the result of Sect. 5 in 
Ref. \cite{lev1} where the mass operator of a system
of two dS particles has been explicitly calculated
assuming that all the states are physical.
As a consequence of this result we have 

{\it Statement 1:} The decomposition of the state 
vector of the system of two free particles with the 
equal mass and spin, and 
such that the first particle is in the state 
$e^{nr}_{ij}$ and the second one - in the state 
$e^{nr}_{N_1(n,r)-i,N_2(n,r)-j}$, contains a state
with zero values of mass and spin.    

Therefore, we have a 
situation analogous to that discussed in Sect.
\ref{S3} when the state of a system of two 
particles contains a state with the 
same quantum numbers as the vacuum. Since such a
situation is unacceptable, we should conclude that
the states $e^{nr}_{ij}$ and 
$e^{nr}_{N_1(n,r)-i,N_2(n,r)-j}$ cannot be
physical simultaneously. This is another proof
that UIRs of the dS group cannot be treated in the
standard way.

In the framework of Supposition 1, the problem
posed by Statement 1 can be resolved if we accept

{\it Supposition 2:} If $e^{nr}_{ij}$ is 
a physical state then $e^{nr}_{N_1(n,r)-i,N_2(n,r)-j}$
is a nonphysical state and {\it vice versa}. 

In the general case the operator $M_{04}$ does not
commute with $J_3'$ and $J_3"$, and $e^{nr}_{ij}$ 
is not the eigenvector of $M_{04}$.  
Therefore, the sign of the dS energy cannot be used 
for distinguishing physical 
and nonphysical states. However, a reasonable 
assumption is that the sign of $B^3(i,j,r)$ can be
used for this purpose. Indeed, when $n$ is big,
this is the case since the sign of $B^3(i,j,r)$ is
the same as the sign of $M_{04}$, and in the general
case $B^3(i,j,r)$ satisfies the condition
$B^3(N_1(n,r)-i,N_2(n,r)-j,n,r)=-B^3(i,j,n,r)$. 

In what follows we will use only Supposition 2 and
no explicit criterion for distinguishing physical
and nonphysical states will be used.

\begin{sloppypar}
Supposition 2 can be consistent only if the
relations $i=N_1(n,r)-i$ and $j=N_2(n,r)-j$ 
{\it cannot} be satisfied simultaneously. This
question is discussed in Sect. \ref{S7}.
\end{sloppypar}

\section{AB symmetry}
\label{S6}

In the standard approach, where a particle and its 
antiparticle are described by independent UIRs, Eq. 
(\ref{37}) describes either the quantized field for
particles or antiparticles. In the standard theory
the notations are such that the operators 
$a(i,j,n,r)$ and $a(i,j,n,r)^*$ are related
to particles while the operators $b(i,j,n,r)$ and
$b(i,j,n,r)^*$ satisfy the analogous commutation or
anticommutation relations
and describe the annihilation and creation of antiparticles.
Then the operators of the quantized
particle-antiparticle field are given by
\begin{eqnarray}  
&M_{standard}^{ab}=\sum_{i'j'n'r'ijnr} M_{particle}^{ab}
(i',j',n'r';i,j,n,r)\nonumber\\
&a(i',j',n',r')^*a(i,j,n,r)/Norm(i,j,n,r)+\nonumber\\
&\sum_{i'j'n'r'ijnr} 
M_{antiparticle}^{ab}(i',j',n',r';i,j,n,r)\nonumber\\
&b(i',j',n',r')^*b(i,j,n,r)/Norm(i,j,n,r)
\label{48}
\end{eqnarray}
where the quantum numbers in each sum take the
values allowable for the corresponding UIR. 

In contrast to the standard approach, Eq. (\ref{37}) 
describes the quantized field for particles and 
antiparticles simultaneously. More precisely, it
describes the quantized field for some object such
that a particle and its antiparticle are different
states of this object. The problem arises how to
interpret Eq. (\ref{37}) in usual terms, i.e. in
terms of particles and antiparticles. 

As noted in Sect. \ref{S3}, in the limit when the
dS algebra can be contracted to the Poincare one, we
can use the following procedure (see also Ref. 
\cite{hep}). The states for which the energy
is positive can be treated as physical 
states describing the particle and for them the 
$(a,a^*)$ operators have the usual meaning. 
On the other hand, the states for which the energy
is negative are nonphysical and for
them the operators $(a,a^*)$ cannot have the usual
meaning. In this case we can use the idea (see
Sect. \ref{S1}) that annihilation of the
particle with the negative energy can be treated
as the creation of the antiparticle with the
positive energy, and creation of the
particle with the negative energy can be treated
as the annihilation of the antiparticle with the
positive energy. As noted in Sect. \ref{S1}, in the
standard approach this idea is implemented
implicitly while in our approach it can be
implemented explicitly. 

\begin{sloppypar}
When the contraction to the 
Poincare algebra is not possible, we can use
Statement 2 and generalize the approach of Ref.
\cite{hep} as follows. If $e^{nr}_{ij}$
is a physical state then we assume as usual that
$a(i,j,n,r)$ is the operator annihilating this
state while $a(i,j,n,r)^*$ is the operator creating
this state. However, if $e^{nr}_{ij}$
is a nonphysical state then it should be related
to the antiparticle as follows. Since the state
$e^{nr}_{N_1(n,r)-i,N_2(n,r)-j}$ is now physical,
we can {\bf define} the antiparticle annihilation
and creation operators $(b,b^*)$ such that
$b(N_1(n,r)-i,N_2(n,r)-j,n,r)$ should 
be proportional to $a(i,j,n,r)^*$ and
$b(N_1(n,r)-i,N_2(n,r)-j,n,r)^*$  
should be proportional to $a(i,j,n,r)$. 
\end{sloppypar}

We define the $b$-operators as follows. 
\begin{eqnarray}
&&a(i,j,n,r)^*=\eta(i,j,n,r) F(i,j,n,r)\nonumber\\
&&b(N_1(n,r)-i,N_2(n,r)-j,n,r)\nonumber\\ 
&&a(i,j,n,r)=\eta(i,j,n,r)^* F(i,j,n,r)\nonumber\\
&&b(N_1(n,r)-i,N_2(n,r)-j,n,r)^*\nonumber\\ 
\label{49}
\end{eqnarray}
where $\eta(i,j,n,r)$ is some function. 
We assume that Eq. (\ref{49}) defines the
$(b,b^*)$ operators regardless of whether
the state $e^{nr}_{ij}$ is physical or
nonphysical. Then, as follows from the above
remarks, the $(b,b^*)$ operators have the 
usual meaning of antiparticle operators only 
for such $(ijnr)$ that $e^{nr}_{ij}$ is physical. 

The $(b,b^*)$ operators should satisfy either
\begin{equation}
\{b(i,j,n,r),b(i',j',n',r')^*\}=Norm(i,j,n,r)
\delta_{ii'}\delta_{jj'}\delta_{nn'}\delta_{rr'}
\label{50}
\end{equation}
in the case of anticommutators, and
\begin{equation}
[b(i,j,n,r),b(i',j',n',r')^*]=Norm(i,j,n,r)
\delta_{ii'}\delta_{jj'}\delta_{nn'}\delta_{rr'}
\label{51}
\end{equation}
in the case of commutators.

As follows from Eq. (\ref{33})
\begin{equation}
F(N_1(n,r)-i,N_2(n,r)-j,nr)F(i,j,n,r)=1
\label{52}
\end{equation}
Therefore, as follows from Eq. (\ref{31}) and
(\ref{49}), Eq. (\ref{50}) can be satisfied 
only if
\begin{equation}
\eta(i,j,n,r)\eta(i,j,n,r)^*=1
\label{53}
\end{equation}
i.e. $\eta(i,j,n,r)$ is a phase factor. At the
same time, Eq. (\ref{51}) obviously cannot be satisfied
for any choice of $\eta(i,j,n,r)$.

In general, we can conclude that if $b$ is proportional
to $a^*$ and $b^*$ is proportional to $a$ then the
$(b,b^*)$ operators can satisfy only anticommutation
relations.

Consider now the following question. If we take the
generators in the quantized form and 
replace the $(a,a^*)$ operators in them by the
$(b,b^*)$ operators using Eq. (\ref{49}) then is
it possible that the generators in terms of
$(b,b^*)$ will have the same form as in $(a,a^*)$?
If such a property is satisfied and the 
$(b,b^*)$ operators satisfy the same anticommutation 
or commutation relations as the $(a,a^*)$ operators 
then, following Ref. \cite{lev2}, we will say that 
the AB symmetry takes place.

For testing the AB symmetry one needs the following
property of the matrix elements:
\begin{equation}
\sum_{ijnr} M_{ab}(i,j,n,r;i,j,n,r)=0
\label{54}
\end{equation}
i.e. the trace of any generator in the space of UIR
is equal to zero. For the diagonal operators $J_3'$
and $J_3"$ the proof follows from the fact that the
l.h.s. of Eq. (\ref{54}) for them is the sum of all
eigenvalues, and for any UIR
of the su(2) algebra the sum of all eigenvalues is
equal to zero. Therefore for any fixed $(nr)$ the
sum of all eigenvalues is equal to zero. For the
remaining nondiagonal operators Eq. (\ref{54})
also is satisfied since the nondiagonal 
operators necessarily change at least one quantum
number $(ijnr)$. 

Let us now take Eqs. (\ref{38}-\ref{47}) and replace
the $(a,a^*)$ operators by the $(b,b^*)$ operators
using Eq. (\ref{49}). Then a direct calculation using 
Eqs (\ref{31}-\ref{33}) and (\ref{54}) shows that
the generators in terms of $(b,b^*)$ have the same
form as in terms of $(a,a^*)$ if and only if
\begin{equation}
\eta(i,j,n,r)=\alpha (-1)^{i+j+n+r}
\label{55}
\end{equation} 
where $\alpha$ is a constant such that 
$\alpha\alpha^*=1$.

Let us note that the AB symmetry has no analog in the standard
theory where the sets $(a,a^*)$ and $(b,b^*)$ are fully 
independent. They are defined only on physical states and 
are related to each other by the CPT transformation in 
Schwinger's formulation (see e.g. Refs. \cite{Novozhilov,Wein}).  
On the contrary, Eq. (\ref{49}) represents not a 
transformation but a definition relating the operators 
in physical and nonphysical states.

\section{Vacuum condition}
\label{S7}

The results of the preceding section are based on the
assumption that the $(a,a^*)$ operators satisfy either
Eq. (\ref{34}) or Eq. (\ref{35}). This is the case if
$a$ has the meaning of annihilation operator and $a^*$
has the meaning of creation operator. Analogously, the
$(b,b^*)$ operators satisfy either Eq. (\ref{50}) or 
Eq. (\ref{51}) if $b$ has the meaning of annihilation 
operator and $b^*$ has the meaning of creation operator.
However, no arguments have been given yet that
these operators indeed can be treated in such a way. 

By analogy with the standard approach, one
might define the vacuum vector $\Phi_0$ such
that 
\begin{equation}
a(i,j,n,r)\Phi_0=b(i,j,n,r)\Phi_0=0\quad 
\forall\,\, (i,j,n,r) 
\label{56}
\end{equation}
Then the elements 
\begin{equation}
\Phi_+(i,j,n,r)=a(i,j,n,r)^*\Phi_0\quad
\Phi_-(i,j,n,r)=b(i,j,n,r)^*\Phi_0
\label{57}
\end{equation}
might be treated as one-particle states for particles 
and antiparticles, respectively. 

However, if one requires the condition (\ref{56})
then it is obvious from Eqs. (\ref{49})
that the elements defined by 
Eq. (\ref{57}) are null vectors. 
We can therefore try to modify
Eq. (\ref{56}) as follows. Let $S_+$ be a set 
of elements $(ijnr)$ such that $e^{nr}_{ij}$ is
a physical state and $S_-$ be a set 
of elements $(ijnr)$ such that $e^{nr}_{ij}$ is
a nonphysical state. Then if Supposition 2 is
consistent, $S_+$ and $S_-$ do not intersect
and each element $(ijnr)$ belongs either to
$S_+$ or $S_-$. Now instead of the condition 
(\ref{56}) we require 
\begin{equation}
a(i,j,n,r)\Phi_0=b(i,j,n,r)\Phi_0=0\quad 
\forall\,\, (i,j,n,r)\in S_+ 
\label{58}
\end{equation}
In that case the elements defined by Eq. (\ref{57}) 
will indeed have the meaning of one-particle states
for $(i,j,n,r)\in S_+$. 

In our approach there is no problem with the stability
of the vacuum, at least in the absence of interactions.
At the same time, as noted in Sect. \ref{S1}, in modern
LQFT in curved spacetime \cite{Narlikar} the vacuum in 
the SO(1,4) invariant theory is unstable. Let us 
discuss this question in greater details. 
Consider particles described by the operators (\ref{9}).
By analogy with the nonrelativistic quantum mechanics, 
one can define the position operator as 
$i/(m \partial {\bf v})$
and time can be defined by the condition that the dS
Hamiltonian is the operator describing evolution. 
Then one can show that, in the quasiclassical limit, the 
motion of the particles is in agreement with the 
classical motion in the dS space. The proof is 
especially simple for nonrelativistic particles 
(see e.g. Refs. 
\cite{lev1,lev1a,lev3}). Therefore, in our approach 
the classical dS space exists only in the approximation
when there are quasiclassical particles. If the system
is in the state described by the vacuum vector $\Phi_0$
then, in our approach, there are no particles and no 
classical dS space exists (in the spirit of Mach's principle). 
On the contrary, in the approach of Ref. \cite{Narlikar},
the classical dS space always exists and one can consider
properties of quantum states from the standpoint of a
geodesic observer. It is well known that in Copenhagen 
formulation of quantum theory, the presence of
a classical observer is always assumed. It is not clear 
whether the formulation is universal, e.g. whether it applies
at the very early stage of the Universe. We will not dwell
on the discussion of this problem since it has 
been extensively discussed in the literature.
 
We now return to the discussion of quantum states in our
approach.
Regardless of how the sets $S_+$ and $S_-$ are
defined, it is clear that the above construction
can be consistent only if there are no such values
of $(ijnr)$ that the equalities $i=N_1(n,r)-i$
and $j=N_2(n,r)-j$ are satisfied simultaneously.
Indeed, suppose that there exist such elements
$(ijnr)$ that the both equalities are satisfied
simultaneously. Then these elements belong to
both $S_+$ and $S_-$ (see Sect. \ref{Physical}).
Since $b(i,j,n,r)$ is proportional to
$a(N_1(n,r)-i,N_2(n,r)-j,n,r)^*$ then, as follows
from Eq. (\ref{49}), if $i=N_1(n,r)-i$, 
$j=N_2(n,r)-j$ and $\Phi_0$
is annihilated by both $a(i,j,n,r)$ and 
$b(i,j,n,r)$, it is also annihilated by both
$a(i,j,n,r)$ and $a(i,j,n,r)^*$. However this
contradicts Eqs. (\ref{34}) and (\ref{35}).   

Since $N_1(n,r)=n+r$ and $N_2(n,r)=n+s-r$, it is
obvious that if $s$ is even then $N_1(n,k)$ and $N_2(n,k)$
are either both even or both odd. Therefore in that
case we will necessarily have a situation when for
some values of $(nr)$, $N_1(n,k)$ and $N_2(n,k)$
are both even. In that case $i=N_1(n,r)-i$
and $j=N_2(n,r)-j$ necessarily takes place for 
$i=N_1(n,k)/2$ and $j=N_2(n,k)/2$. Moreover, since 
for each $(nr)$ the number of all possible values of 
$(ijnr)$
is equal to $(N_1(n,r)+1)(N_2(n,r)+1)$, this 
number is odd (therefore one cannot divide the set
of all possible values into the equal 
nonintersecting parts $S_+$ and $S_-$). 

On the other hand, if $s$ is odd then
for all the values of $(nr)$ we will necessarily have
a situation when either $N_1(n,r)$ is even and
$N_2(n,r)$ is odd or $N_1(n,r)$ is odd and
$N_2(n,r)$ is even. Therefore for each value of 
$(nr)$ the case when the equalities $i=N_1(n,r)-i$
and $j=N_2(n,r)-j$ are satisfied simultaneously
is impossible, 
and the number of all possible values of $(ijnr)$  
is even.

We conclude that the condition (\ref{58}) is
consistent only if $s$ is odd,
or in other words, if the particle spin in usual units
is half-integer. 

In Sect. \ref{Physical} we argued that the sign of
$B^3(i,j,r)$ is a good criterion for 
distinguishing physical and nonphysical states. If
$i=N_1(n,r)-i$ and $j=N_2(n,r)-j$ are satisfied 
simultaneously, then it is obvious that $B^3(i,j,r)=0$.
Therefore Supposition 2 will be always consistent
if $B^3(i,j,r)=0$ is impossible. This is automatically
satisfied if $s$ is odd, since $B^3(i,j,r)=2(r+j-i)-s$. 
Let us stress, however, that the conclusion of this
section that $s$ should be odd, does not depend on 
the explicit way of breaking the set 
of elements $(ijnr)$ into $S_+$ and $S_-$.  

\section{Neutral particles, AB$^2$ parity and space
inversion}
\label{S8}  
 
Suppose that the particle in question is neutral, i.e.
the particle coincides with its antiparticle. On the language
of the operators $(a,a^*)$ and $(b,b^*)$ this means that these
sets are the same, i.e. $a(i,j,n,r)=b(i,j,n,r)$ and
$a(i,j,n,r)^*=b(i,j,n,r)^*$. Then as follows from
Eqs. (\ref{52}) and (\ref{55}), Eq. (\ref{49}) is
consistent only if $s$ is even. This means that in 
our approach neutral elementary particles with
the half-integer spin (in conventional units) cannot
exist. At the same time, as shown in the preceding section,
the integer spin in conventional units is incompatible
with the vacuum condition. For this reason we conclude
that in our approach there can be no neutral elementary
particles. As argued in Ref. \cite{lev2}, this conclusion
is natural in view of the following observation. 
If one irreducible representation describes a 
particle and its antiparticle simultaneously, the energy 
operator necessarily contains the contribution of
the both parts of the spectrum, corresponding to the particle
and its antiparticle. If a particle were the same as 
its antiparticle then the energy operator would 
contain two equal contributions and thus the value of 
the energy would be twice as big as necessary. 

Consider now the following question.
In Sect. \ref{S6} the AB symmetry has been formulated as the
condition that the $(b,b^*)$ operators satisfy the same
anticommutation or commutation relations as the
$(a,a^*)$ operators and the representation generators
have the same form
in terms of $(a,a^*)$ and $(b,b^*)$. 
In that case the operators
$(b,b^*)$ are defined in terms of $(a,a^*)$ by 
Eqs. (\ref{49}). A desire to have operators which can be
interpreted as those relating separately 
to particles and antiparticles is natural in view of our
experience in the standard approach. However, in the spirit
of our approach, there is no need to have separate 
operators for
particles and antiparticles since they are different states
of the same object. For this reason the operators $(b,b^*)$
are stricly speaking redundant. We can therefore reformulate
the AB symmetry as follows. Instead of Eqs. (\ref{49}), 
we consider a {\it transformation} defined as 
\begin{eqnarray}
&&a(i,j,n,r)^*\rightarrow\eta(i,j,n,r) F(i,j,n,r)\nonumber\\
&&a(N_1(n,r)-i,N_2(n,r)-j,n,r)\nonumber\\ 
&&a(i,j,n,r)\rightarrow\eta(i,j,n,r)^* F(i,j,n,r)\nonumber\\
&&a(N_1(n,r)-i,N_2(n,r)-j,n,r)^*\nonumber\\ 
\label{62}
\end{eqnarray}
Then the AB symmetry can be formulated as a requirement that
the anticommutation or commutation relations and operators 
related to physical quantities should be invariant
under this transformation.

The results of Sect. \ref{S6} can now be reformulated in
such a way that the representation generators are compatible 
with this new formulation of the AB symmetry (strictly 
speaking, the name "AB symmetry" is not appropriate anymore
but we retain it for "backward compatibility"). 

Let us now apply the AB transformation twice. Then, as follows 
from Eqs. (\ref{52}) and (\ref{55}), 
\begin{equation}
a(i,j,n,r)^*\rightarrow (-1)^sa(i,j,n,r)^*\quad
a(i,j,n,r)\rightarrow (-1)^sa(i,j,n,r)
\label{63}
\end{equation} 
Since only an odd value of $s$ is compatible with
the vacuum condition, we can formulate this result
by saying that the AB$^2$ parity of each elementary
particle is equal to -1. Therefore, as a consequence
of the AB symmetry, any interaction can involve 
only an even number of creation and annihilation 
operators. Since in our approach only fermions 
with the half-integer spin (in conventional units)
can be elementary, this result is obvious.

The results of Sect. \ref{S7} can be easily
reformulated for the case when only the $(a,a^*)$
operators are used. In this case $a(i,j,n,r)$ can
be treated as annihilation operator when 
$(ijnr)\in S_+$ and as creation one when
$(ijnr)\in S_-$. Analogously, $a(i,j,n,r)^*$ can
be treated as creation operator when 
$(ijnr)\in S_+$ and as annihilation one when
$(ijnr)\in S_-$. The consistency condition is
the requirement that there should be no such 
$(ijnr)$ that $i=N_1(n,r)-i$ and
$j=N_2(i,j)-j$. As shown in Sect. \ref{S7},
this condition can be satisfied only for particles
with the half-integer spin (in conventional units).

Finally, consider the space inversion in our approach.
We define the space inversion as the transformation
\begin{eqnarray}
&&a(i,j,n,r)^*\rightarrow \eta_P^* (-1)^n 
[G(r)/G(s-r)]^{1/2}a(j,i,n,s-r)^*\nonumber\\
&&a(i,j,n,r)\rightarrow \eta_P (-1)^n 
[G(r)/G(s-r)]^{1/2}a(j,i,n,s-r)
\label{64}
\end{eqnarray}
where $\eta_P$ is the spatial parity and
\begin{equation}
G(r)=[(s+1-r)/2^r]^2 \prod_{l=1}^r [w+(s+1-2l)^2]
\label{65}
\end{equation} 
Then, as follows from Eqs. (\ref{31}-\ref{33}),
the anticommutation or commutation relations,
(\ref{34}) or (\ref{35}), are invariant under 
the transformation (\ref{64}) if $|\eta_P|=1$,
i.e. $\eta_P$ is a phase factor. 
If we apply the transformation (\ref{64}) to
Eqs. (\ref{38}-\ref{47}) then a direct calculation
using Eqs. (\ref{17}), (\ref{31}-\ref{33}) and
(\ref{65}) shows that ${\bf M}\rightarrow {\bf M}$,
${\bf B}\rightarrow -{\bf B}$, 
${\bf N}\rightarrow -{\bf N}$ and
$M_{04}\rightarrow M_{04}$. Therefore Eq. (\ref{64})
indeed has the meaning of the space inversion. 

Consider now whether the space inversion is 
compatible with the AB symmetry. Let us first
apply the transformation (\ref{64}) to the
second relation in Eq. (\ref{62}). Taking into
account Eq. (\ref{55}) we obtain
\begin{eqnarray}
&&\eta_Pa(j,i,n,s-r)\rightarrow (-1)^{i+j+n+r}
(\alpha\eta_P)^*F(i,j,n,r)\nonumber\\
&&a(n+s-r-j,n+r-i,n,s-r)
\label{66}
\end{eqnarray}
On the other hand, by applying the transformation
(\ref{62}) to $a(j,i,n,s-r)$ we obtain
\begin{eqnarray}
&&a(j,i,n,s-r)\rightarrow (-1)^{i+j+n+s-r}
\alpha^*F(j,i,n,s-r)\nonumber\\
&&a(n+s-r-j,n+r-i,n,s-r)
\label{67}
\end{eqnarray}
Now, as follows from Eq. (\ref{33}), Eqs. (\ref{66})
and (\ref{67}) are compatible with each other if and
only if
\begin{equation}
\eta_P^* =(-1)^s \eta_P
\label{68}
\end{equation}
We have obtained a well known result (see e.g. Ref.
\cite{Wein-super}) that particles with the 
half-integer spin (in conventional units) have
imaginary parity. In our approach this result is
a direct consequence of the AB symmetry.

\section{Discussion}
\label{S9}

In the present paper we have reformulated the 
standard approach
to quantum theory as follows. Instead of requiring that
each elementary particle is described by its own UIR of
the symmetry algebra, we require that one UIR should 
describe a particle and its antiparticle simultaneously.
In that case, among the Poincare, AdS and dS algebras,
only the latter can be a candidate for constructing
elementary particle theory. 

\begin{sloppypar}
Although our approach considerably differs from that
in LQFT in curved spacetime, our results confirm the 
conclusion of Refs. \cite{Narlikar,Susskind} and references
therein that the dS group cannot be a symmetry group
in the standard approach (see Sects. \ref{S1} and 
\ref{S3}). In contrast with the standard approach, 
the space of UIR
in our one contains two sets --- physical and 
nonphysical states. As a consequence, the dS algebra
can be a symmetry algebra if one accepts Supposition 1. 
\end{sloppypar}

In Ref. \cite{Susskind}, which appeared after the
original version of the present paper, problems with
the dS spacetime have been discussed in the framework of
the thermofield theory, which was originally invented
in many-body theory (see Ref. \cite{Susskind} for
references). The existence of physical and nonphysical
states in our approach looks similar to the analogous 
feature in the thermofield theory, but the
interpretation of such states in our approach fully
differs from that in the thermofield theory.
Nevertheless, this example confirms that similar 
ideas can be invented in approaches which considerably
differ each other.  

For the physical states the 
operators $(a,a^*)$ have the usual meaning while for 
nonphysical states $a$ becomes the creation operator while 
$a^*$ - the annihilation one. It is obvious that only
anticommutation relations are consistent with such an 
interchange. This simple observation immediately explains why 
in our approach only fermions can 
be elementary (see Sect. \ref{S6}). 

The fact that the vacuum condition is consistent only for 
particles with the half-integer spin (in conventional units) 
has been proved in Sect. \ref{S7}. A simple explanation is 
as follows. Since there should be equal numbers of physical 
and nonphysical states and they should not intersect with 
each other, the number of states in each su(2)$\times$su(2) 
multiplet should be even. These conditions are satisfied 
only if the spin is half-integer.

Our results can be summarized as follows. {\it If the de 
Sitter algebra so(1,4) is the symmetry algebra in elementary 
particle theory then only fermions can be elementary and 
they can have only the half-integer spin}.

In the standard theory there exists the well-known Pauli 
spin-statistics theorem \cite{Pauli} stating that 
elementary particles with the half-integer spin are 
fermions while the particles with the integer spin --- 
bosons. The proof has been given in the framework of the 
standard local quantum field theory. After the original 
Pauli proof, many authors investigated more general 
approaches to the spin-statistics theorem (see e.g. Ref. 
\cite{Kuckert} and references cited therein). On the 
other hand, in the framework of the Skyrme model 
\cite{Skyrme} and its generalizations (see e.g. 
Refs. \cite{Zahed}) fermions can be built of bosons.

Our result has been proved only for free particles (as well
as the original proof \cite{Pauli}). It follows only from 
general properties of representations of groups and algebras
in Hilbert spaces and does not involve any locality
conditions.

In Ref. \cite{hep}, where the Poincare limit has been
considered, we have proved that only fermions can be
elementary but no restriction on spin has been 
obtained. The reason is that in the Poincare limit
the physical and nonphysical states are fully disjoint
(they have supporters on the upper and lower Lorentz
hyperboloids, respectively), and the number of all
possible states is always even for any spin. On the other
hand, as noted in Sect. \ref{S1}, in the Poincare limit,
the Wigner approach to elementary particles is
compatible with that in the LQFT.
 
In our opinion, the very possibility that only fermions 
with the half-integer spin can be elementary, is very 
attractive from the
aesthetic point of view. Indeed, what was the reason for
nature to create elementary fermions and bosons if the 
latter can be built of the former? A well known
historical analogy is that before the discovery of
the Dirac equation, it was believed that nothing could
be simpler than the Klein-Gordon equation for spinless
particles. However, it has turned out that the spin 1/2
particles are simpler since the covariant equation
for them is of the first order, not the second one as the
Klein-Gordon equation. A very interesting possibility
(which has been probably considered first by Heisenberg) is 
that only spin 1/2 particles are elementary.

In our recent series of papers \cite{lev2} 
it has been argued that quantum
theory based on a Galois field (GFQT) is more natural
than the standard one. In the GFQT the property that
one IR simultaneously describes a particle and its
antiparticle, is satisfied automatically, and it was
the main reason for the present investigation. In other
words, in the GFQT there are no IRs describing only 
particles without antiparticles. It has been proved
in Ref. \cite{lev2} that in the GFQT only a half-integer
spin is compatible with the vacuum condition, and the
proof given in Sect. \ref{S7} of the present paper is
similar. At the same time, in Ref. \cite{lev2} we have 
not succeeded in proving that only fermions are 
elementary. Roughly speaking, the reason is that in 
Galois fields the quantity $\alpha \alpha^*$ is not 
necessary greater than zero when $\alpha \neq 0$.
It has been also noted that if only fermions are 
elementary, then the actual infinity is not present in the
theory in any form, and for each sort of elementary
particles their number in the Universe cannot be 
greater than $p^3$ where $p$ is the characteristic
of the Galois field.

{\it Acknowledgements: } The author is grateful to 
V. Dobrev, L. Koyrakh, M.B. Mensky, M.A. Olshanetsky,
E. Pace, G. Salme and M. Saniga for fruitful discussions. 

\begin{thebibliography}{99}
\bibitem{DirMath} P.A.M. Dirac, {\it in} Mathematical Foundations of
Quantum Theory, A.R. Marlow ed. (Academic Press, New York -
San Francisco - London, 1978).
\bibitem{Wein} S. Weinberg, Quantum Theory of Fields  
(Cambridge, Cambridge University Press, 1995).
\bibitem{Haag} R. Haag, Kgl. Danske Videnskab. Selsk. Mat.-Fys. Medd.
{\bf 29}, No. 12  (1955); R.F. Streater and A.S. Wightman, 
PCT, Spin, Statistics and All That (W.A.Benjamin Inc. 
New York-Amsterdam, 1964);
R. Jost, The General Theory of Quantized Fields, Ed. M.Kac
(American Mathematical Society, Providence, 1965);
N.N. Bogolubov, A.A. Logunov, A.I. Oksak and I.T. Todorov,
General Principles of Quantum Field Theory (Nauka, Moscow, 1987);
R. Haag, Local Quantum Physics (Springer Verlag,
Heidelberg, 1996); D. Buchholz and R. Haag, hep-th/9910243.
\bibitem{time} W. Pauli, Handbuch der Physik, vol. V/1
(Berlin, 1958); F.T. Smith, Phys. Rev. {\bf 118}, 349 (1960);
Y. Aharonov and D. Bohm, Phys. Rev. {\bf 122}, 1649 (1961),
{\bf 134} 1417 (1964); V.A. Fock, ZhETF {\bf 42}, 1135 (1962);
B.A. Lippman, Phys. Rev. {\bf 151}, 1023 (1966).
\bibitem{Schweber} T.D. Newton and E.P. Wigner, Rev. Mod. Phys.
{\bf 21}, 400 (1949). 
\bibitem{Saniga} M. Saniga, Chaos, Solitons and Fractals, 
{\bf 9}, 1769 (1998);  {\bf ibid} 1087 (1998); {\bf ibid}
1095 (1998).
\bibitem{Wein1} S. Weinberg, hep-th/9702027.
\bibitem{Wigner} E.P. Wigner, Ann. Math. {\bf 40}, 149 (1939).
\bibitem{Narlikar} J.V. Narlikar and T. Padmanabhan,
Gravity, Gauge Theories and Quantum Cosmology (D. Reidel
Publishing Company, Dordrecht, 1986);
R.M. Wald {\it in} Gravitation and
Quantization, B. Julia and J. Zinn-Justin eds. (Elsevier,
Amsterdam, 1986) p.63, Quantum Field Theory in curved
spacetime and black hole thermodynamics (University of
Chicago Press, 1994);
S.A. Fulling, Aspects of quantum field
theory in curved spacetime (Cambridge University Press, 1989);
H.J. Borchers and D. Buchholz, Annales Poincare Phys.
Theor. {\bf 70}, 23 (1999).
\bibitem{Susskind} N. Goheer, M. Kleban and L. Susskind,
hep-th/0212209.
\bibitem{Kato} T. Kato, Perturbation Theory for Linear
Operators (Springer-Verlag, Berlin - Heidelberg - New York,
1966). 
\bibitem{Klein} R.W. Sharpe, Differential Geometry: Cartan's
Generalization of Klein's Erlangen Program (Graduate Texts
in Mathematics, Vol. 166, Springer Verlag, 
Berlin - Heidelberg - New York, 1997).
\bibitem{Dir} P.A.M. Dirac, Rev. Mod. Phys. {\bf 21}, 392 (1949).
\bibitem{lev2} F. Lev, hep-th/0209001 and references therein.
\bibitem{DirNobel} P.A.M. Dirac, in {\it The World Treasury
of Physics, Astronomy and Mathematics}, p. 80, Timothy Ferris ed.
(Little Brown and Company, Boston-New York-London, 1991).
\bibitem{Dobrev} V.K. Dobrev, G. Mack, V.B. Petkova, 
S. Petrova and I.T. Todorov, IAS preprint, Princeton, (May 
1975); Rep. Math. Phys. {\bf 9}, 219 (1976);
Harmonic Analysis on the n-Dimensional Lorentz 
group and Its Application to Conformal Quantum Field Theory,
Lecture notes in Physics Vol. 63 (Springer Verlag,
Berlin - Heidelberg - New York, 1977). 
\bibitem{Men} M.B. Mensky, Method of Induced Representations.
Space-Time and Concept of Particles (Moscow, Nauka, 1976). 
\bibitem{Moy} P. Moylan, J. Math. Phys. {\bf 24}, 2706 (1983);
{\bf 26}, 29 (1985).
\bibitem{lev1} F.M. Lev, J. Phys. {\bf A21}, 599 (1988).
\bibitem{lev1a} F.M. Lev, J. Phys. {\bf A32}, 1225 (1999).
\bibitem{Perlmutter} S. Perlmutter et. al. Astrophys. J. {\bf 517},
565 (1999); A. Melchiori et. al. astro-ph/9911445.
\bibitem{Witten} E. Witten, hep-th/0106109; T.Banks, W. Fishler
and S. Paban, hep-th/0210160.
\bibitem{hep} F. Lev, hep-th/0210144.
\bibitem{Linde} R. Kallosh and A. Linde, hep-th/0208157;
A. Linde, hep-th/0211048.
\bibitem{Dix1} J. Dixmier, Bull. Soc Math. France, {\bf 89}, 9 (1961).
\bibitem{Tak} R. Takahashi Bull. Soc. Math. France, {\bf 91}, 289 (1963).
\bibitem{Hann} K.C. Hannabus, Proc. Camb. Phil. Soc. {\bf 70}, 283 (1971).
\bibitem{Str} S. Stroem,  Ann. Inst. Henri Poincare, {\bf 58}, 
77 (1970); {\it in} De Sitter and Conformal Groups and
Their Applications, Lectures in Theoretical Physics, Vol. XIII
(Boulder, Colorado, 1971) p.97.
\bibitem{Schwarz} F. Schwarz {\it in} De Sitter and Conformal Groups and
Their Applications, Lectures in Theoretical Physics, Vol. XIII
(Boulder, Colorado, 1971) p. 53. 
\bibitem{Mielke} E.W. Mielke, Fortschr. Phys. {\bf 25}, 401 (1977).
\bibitem{Klimyk} A.U. Klimyk and I.I. Kachurik, Computational 
Methods in Group Representation Theory 
(Vyshcha Shkola, Kiev, 1986).
\bibitem{Naimark} G.W. Mackey, Ann. Math. {\bf 55}, 101 (1952); 
{\bf 58}, 193 (1953); M.A. Naimark, Normalized rings 
(Nauka, Moscow, 1968);
J. Dixmier, Les  algebres  d'operateurs dans l'espace
   hilbertien (Gauthier-Villars, Paris, 1969);
\bibitem{Barut} A.O. Barut and R. Raczka, Theory of group 
   representations  and
   applications (Polish  Scientific  Publishers, Warsaw, 1977). 
\bibitem{lev3} F. Lev, J. Math. Phys., {\bf 34}, 490 (1993).
\bibitem{IW} E. Inonu and E.P. Wigner, Nuovo Cimento, 
{\bf IX}, 705 (1952).
\bibitem{Hug} W.B. Hughes, J. Math. Phys. {\bf 24}, 1015 (1983).
\bibitem{Klimyk2} A.U. Klimyk, Amer. Math. Soc. Trans. Ser.
2 {\bf 76}, 63 (1968).
\bibitem{Novozhilov} Yu.V. Novozhilov, 
An Introduction to Elementary Particle
Theory (Nauka, Moscow, 1972).
\bibitem{Wein-super} S. Weinberg, The Quantum 
Theory of Fields, Volume III Supersymmetry, (Cambridge 
University Press, Cambridge, 2000).
\bibitem{Pauli} W. Pauli, Phys. Rev. {\bf 58}, 116 (1940).
\bibitem{Kuckert} R. Verch, Commun. Math. Phys. {\bf 223},
261 (2001); B. Kuckert, quant-ph/0208151.
\bibitem{Skyrme} T.H.R. Skyrme, Proc. Roy. Soc. London
{\bf 260}, 127 (1961), {\bf ibid} {\bf 262}, 237 (1961).
\bibitem{Zahed} E. Witten, Nucl. Phys. {\bf B233}, 433 (1983);
I. Zahed and G.E. Brown, Phys. Rep. {\bf 142}, 1 (1986);
V. Makhankov, Y.P. Rybakov, V.I. Sanyuk, The Skyrme Model
(Springer Verlag, Berlin - Heidelberg - New York, 1993)
\end{thebibliography}
\end{document}
%
%
%


