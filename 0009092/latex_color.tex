
\documentclass[11pt,a4paper]{article}
\renewcommand{\theequation}{\thesection.\arabic{equation}}
\let\ssection=\section
\renewcommand{\section}{\setcounter{equation}{0}\ssection}

\title{Symmetries of fluid dynamics\\
with polytropic exponent\footnote{Dedicated to the memory of
Lochlainn O'Raifeartaigh, our late friend and teacher.}
\\
}
\author{M.~Hassa\"{\i}ne and P. A.~Horv\'athy\\
\\
Institute for Theoretical Physics, Roland E\"{o}tv\"{o}s
University,\\
H-1117, BUDAPEST, P\'azm\'any P. s\'et\'any 1/A (Hungary)
\\
and
\\
Laboratoire de Math\'ematiques et de Physique Th\'eorique\\
Universit\'e de Tours, Parc de Grandmont\\
F-37200 TOURS (France)\\
}
\def\D{{\cal D}}
\def\K{{\cal K}}
\def\H{{\cal H}}
\def\IR{{\bf R}}
\def\smallcirc{{\raise 0.5pt \hbox{\myHighlight{$\scriptstyle\circ$}\coordHE{}}}}
\def\smallover#1/#2{\hbox{\myHighlight{$\textstyle{#1\over#2}$}\coordHE{}}} %
\def\o{{\rm o}}
\def\O{{\rm O}}
\def\so{{\rm so}}
\def\parag{\hfil\break} %%%%% paragraph
\def\kikezd{\parag\underbar}


\usepackage{useful_macros}
\begin{document}

\maketitle
\begin{abstract}
The symmetries of the general Euler equations of fluid dynamics with
polytropic exponent are determined using the Kaluza-Klein
type framework of Duval et \myHighlight{$\it{al}$}\coordHE{}.
In the standard polytropic case the recent results of O'Raifeartaigh
and Sreedhar are confirmed. Similar
results are proved for
 polytropic exponent \myHighlight{$\gamma=-1$}\coordHE{}, which corresponds to
the dimensional reduction of \myHighlight{$d$}\coordHE{}-branes.
The relation between the duality transformation used in describing
supernova explosion and Cosmology is explained.
\end{abstract}


%%%%%%%%%%%%%%%%%%%%%%
\section{Introduction}
%%%%%%%%%%%%%%%%%%%%%%

The amazing similarity of supernova explosion and plasma implosion
has been explained not less amazingly by Drury and
Mendon\c ca \cite{DM}, who pointed out that the two situations
 can be related by the ``duality'' transformation
\myHighlight{$
\Sigma~: t\to -1/t,\; {\bf x}\to {\bf x}/t$}\coordHE{}.
This strange-looking transformation
belongs to the \myHighlight{${\rm SL}(2,\IR)$}\coordHE{}  group generated by
the
dilatations, \myHighlight{$\D:t\to \delta^2t,\, {\bf x}\to \delta{\bf x}$}\coordHE{},
expansions,
\myHighlight{$\K~: t\to t/1+\kappa t,\; {\bf x}\to{\bf x}(1+\kappa t)^{-1}$}\coordHE{},
 and time-translation, \myHighlight{${\cal H}~:t\to t+\epsilon$}\coordHE{},
which are indeed symmetries of a free non-relativistic particle
\cite{NH, DGH}. In fact, \myHighlight{$\Sigma=\H_{-1}\smallcirc \K_{1}\smallcirc
H_{-1}$}\coordHE{}.


Motivated by the results of Drury and Mendon\c ca, O'Raifeartaigh
and Sreedhar \cite{RS}
performed a systematic study of the symmetries of the Euler equations
of fluid dynamics,
\begin{eqnarray}\coord{}\boxAlignEqnarray{\leftCoord{}
 D\rho & = &-\rho\vec{\nabla}\cdot{\bf u},\rightCoord{}\\\leftCoord{}
\rho\rightCoord{}\, D{\bf u} & = &-\Lambda(\gamma-1)\vec{\nabla}(\chi
\rho^{\gamma})+{\bf V},\rightCoord{}\\\leftCoord{}
D\chi & = &0, \rightCoord{}
\label{eulergeneral}
\rightCoord{}}{0mm}{3}{6}{
 D\rho & = &-\rho\vec{\nabla}\cdot{\bf u},\\
\rho\, D{\bf u} & = &-\Lambda(\gamma-1)\vec{\nabla}(\chi
\rho^{\gamma})+{\bf V},\\
D\chi & = &0, 
}{1}\coordE{}\end{eqnarray}
where \myHighlight{$D$}\coordHE{} is the convective derivative,
\myHighlight{$D=\partial_t+{\bf u}\cdot\vec{\nabla}$}\coordHE{}, and the fields \myHighlight{$\rho$}\coordHE{} and
\myHighlight{${\bf u}$}\coordHE{} are the density and the velocity. \myHighlight{${\bf V}$}\coordHE{} is
the viscosity term, with components
\begin{eqnarray}\coord{}\boxAlignEqnarray{\leftCoord{}
V_i=\partial_j\left(\eta\left(\partial_j u_i+\partial_iu_j
\leftCoord{}-\frac{\leftCoord{}2}{\rightCoord{}d}\delta_{ij}\partial_k u_k\right)\right)+\partial_i
\left(\xi\partial_k u_k\right),
\label{viscosityterms}
\rightCoord{}}{0mm}{3}{3}{
V_i=\partial_j\left(\eta\left(\partial_j u_i+\partial_iu_j
-\frac{2}{d}\delta_{ij}\partial_k u_k\right)\right)+\partial_i
\left(\xi\partial_k u_k\right),
}{1}\coordE{}\end{eqnarray}
where  \myHighlight{$d$}\coordHE{} is the spatial dimension,
\myHighlight{$\xi$}\coordHE{} and \myHighlight{$\eta$}\coordHE{} represent the bulk  and shear viscosity fields,
respectively.  \myHighlight{$\gamma$}\coordHE{} is the polytropic exponent and
 \myHighlight{$\Lambda$}\coordHE{} is the coupling constant of a
potential, \myHighlight{$U(\rho)=\Lambda\, \rho^{\gamma}$}\coordHE{}. The field \myHighlight{$\chi$}\coordHE{} is
related to the energy density \myHighlight{$\epsilon$}\coordHE{} by \myHighlight{$\epsilon=\chi\rho^{\gamma}$}\coordHE{}.

O'Raifeartaigh and Sreedhar consider first the sub-class of
(\myHighlight{$1.1$}\coordHE{})-(\myHighlight{$1.3$}\coordHE{}) characterised
by
\myHighlight{$(i)$}\coordHE{} the absence of viscosity terms, \myHighlight{${\bf V}=0$}\coordHE{};
\myHighlight{$(ii)$}\coordHE{} the dynamical field \myHighlight{$\chi$}\coordHE{} is choosen to be
\myHighlight{$\chi=1$}\coordHE{};
\myHighlight{$(iii)$}\coordHE{} the motion is assumed irrotational,
\myHighlight{${\rm rot}\,{\bf u}=0$}\coordHE{}.
Then they show that
when the polytropic exponent takes the standard value \myHighlight{$\gamma=1+2/d$}\coordHE{},
the equations
(\myHighlight{$1.1$}\coordHE{})-(\myHighlight{$1.3$}\coordHE{}) are invariant w.r.t. Schr\"{o}dinger
transformations, composed of Galilei transformations, augmented by
dilatations and expansions
\cite{NH}. When the conditions \myHighlight{$(i), (ii)$}\coordHE{} and \myHighlight{$(iii)$}\coordHE{} are relaxed,
the expansions are generally broken by the viscosity term;
dilatations remain, however, symmetries \cite{RS}.

Similar questions were
investigated by Bordemann and Hoppe, and Jevicki \cite{BoHo},
and by Jackiw, Polychronakos, and Bazeia
\cite{JAC, BJ}, who found that the dimensional reduction
of d-brane theory yields a
viscosity--free, isentropic and irrotational hydrodynamical model
called the Chaplygin gas,
eqns. (\myHighlight{$1.1$}\coordHE{})-(\myHighlight{$1.3$}\coordHE{}) with \myHighlight{${\bf V}=0$}\coordHE{} and \myHighlight{$\chi=1$}\coordHE{} and
with effective
potential \myHighlight{$U \propto 1/\rho$}\coordHE{}.
Remarkably, their system
admits a hidden Poincar\'e symmetry \cite{BoHo, JAC, BJ},
composed of the Galilei transformations,
augmented by \myHighlight{$(d+1)$}\coordHE{} generators we called time-dilatation and antiboost
 \cite{HH}.

\goodbreak
In this Letter, we combine and generalize these results in a unified
framework.
First, we confirm  the results of O'Raifeartaigh
et {\it al}. by dropping condition \myHighlight{$(iii)$}\coordHE{} right on from the
beginning.
Then we extend the d-brane results in \cite{BoHo, JAC, BJ}
showing that, for \myHighlight{$U \propto 1/\rho$}\coordHE{}, the symmetries of
the general equations (\myHighlight{$1.1$}\coordHE{})-(\myHighlight{$1.3$}\coordHE{}) with conditions \myHighlight{$(i)$}\coordHE{} and \myHighlight{$(ii)$}\coordHE{} alone
 still admit a Poincar\'e symmetry.
Viscosity breaks part of this large symmetry.
There remains, however,  time-dilatation,
\myHighlight{$\Delta~: t\to e^{\alpha}t,\; {\bf x}\to {\bf x}$}\coordHE{}, analogous to
dilatations, \myHighlight{$\D$}\coordHE{},  in the standard case.
\goodbreak

 The relation of the duality transformation
\myHighlight{$\Sigma$}\coordHE{} and newtonian cosmology is also explained.
Although our results could also be obtained in a classical
approach \cite{NH,  RS, BJ},
we found it more convenient to use  Duval's Kaluza
Klein--type framework \cite{DGH}, which sheds a new light
on the arisal of these symmetries.

%%%%%%%%%%%%%%%%%%%%%%%%%%%%%%%%%%%%%%%%%%
\section{Symmetries of the Euler equations}
%%%%%%%%%%%%%%%%%%%%%%%%%%%%%%%%%%%%%%%%%%

The simplest way to confirm the result of O'Raifeartaigh and
Sreedhar \cite{RS},
is to consider \cite{HH}, Sect. 2, p. 224
(see also \cite{JPRIV}), the stress--energy tensor \myHighlight{$T^{\alpha\beta}$}\coordHE{}.
In the absence of viscosity, \myHighlight{${\bf V}=0$}\coordHE{}
and for \myHighlight{$\chi=1$}\coordHE{}, they are given, e. g., in Eq. (2.2) in the first
reference of
\cite{JAC}, as
\begin{equation}\coord{}\boxEquation{
	T^{00}=\rho\frac{{\bf u}^2}{2}+U(\rho),
	\qquad
	T^{ij}=\rho u^{i}u^{j}-\delta^{ij}(U-\rho\partial_{\rho}U),
    \label{emtensor}
}{
	T^{00}=\rho\frac{{\bf u}^2}{2}+U(\rho),
	\qquad
	T^{ij}=\rho u^{i}u^{j}-\delta^{ij}(U-\rho\partial_{\rho}U),
    }{ecuacion}\coordE{}\end{equation}
where \myHighlight{$\partial_{\rho}U$}\coordHE{} is the enthalpy\footnote{It is worth
noting that, although it has been derived assuming irrotationality,
(\ref{emtensor}) actually provides us with a conserved energy-momentum
tensor
in the general case, as it can be verified by a directly,
using the Euler equations.}.
Next recall (e. g.
\cite{JP}, Eq. (2.261) ) the criterion
of Schr\"odinger symmetry:
\begin{equation}\coord{}\boxEquation{
    2T^{00}=\sum_{i}T^{ii},
    \label{nrtracecond}
 }{
    2T^{00}=\sum_{i}T^{ii},
    }{ecuacion}\coordE{}\end{equation}
which replaces, in the non-relativistic context,
the familiar  condition for relativistic conformal
invariance, {\it viz}. \myHighlight{$T^{\mu}_{\ \mu}=0$}\coordHE{}. With the above expression
for \myHighlight{$T^{00}$}\coordHE{} and \myHighlight{$T^{ij}$}\coordHE{}, we get
 a differential equation for \myHighlight{$U$}\coordHE{}, namely
\myHighlight{$\rho\partial_{\rho}U=(2/d+1)U$}\coordHE{} or
\myHighlight{$
U=\Lambda\rho^{1+2/d},
$}\coordHE{}
which is the result in \cite{RS}.

More generally, let us first consider
the sub-class of (\myHighlight{$1.1$}\coordHE{})-(\myHighlight{$1.3$}\coordHE{}) with
conditions \myHighlight{$(i)$}\coordHE{} and \myHighlight{$(ii)$}\coordHE{} alone. Using the
Clebsch parametrization \cite{CLE},
\myHighlight{${\bf u}=\vec{\nabla}\phi-\nu\vec{\nabla}\theta$}\coordHE{},
provides us with a local lagrangian theory \cite{RS}.
Then eliminating the Lagrange multiplier \myHighlight{$\nu$}\coordHE{}
yields the equations of motion
\begin{eqnarray}\coord{}\boxAlignEqnarray{\leftCoord{}
\begin{array}{ll} \rightCoord{}
\left(E_{\gamma}\right) &\rightCoord{}
\left\lbrace\rightCoord{}
\begin{array}{l} \rightCoord{}
\partial_t\rho+\partial_k\left(\rho\partial_k\phi+\rightCoord{}
\displaystyle{\frac{\leftCoord{}\rho}\rightCoord{}
{\rightCoord{}\leftCoord{}\vert\vec{\nabla}\theta\vert^2}}\partial_k\rightCoord{}
\theta\left(\partial_t\theta+\rightCoord{}
{\rightCoord{}\leftCoord{}\vec\nabla}\rightCoord{}
\theta\cdot\vec{\nabla}\phi\right)\right)=0,\rightCoord{}
\leftCoord{}\\\leftCoord{}[3mm]\rightCoord{}
\partial_t\phi+\frac{\leftCoord{}1}{\rightCoord{}2}\rightCoord{}
\vert{\vec\nabla}\phi\vert^2+\rightCoord{}
\displaystyle{\frac{\leftCoord{}1}\rightCoord{}
{\rightCoord{}\leftCoord{}2\vert{\vec\nabla}\theta\vert^2}}\rightCoord{}
\left(\partial_t\theta+\rightCoord{}
\vec{\nabla}\theta\cdot{\vec\nabla}\phi\right)^2\rightCoord{}
\leftCoord{}-\gamma\Lambda\rho^{\gamma-1}=0,\rightCoord{}
\leftCoord{}\\\leftCoord{}[3mm]\rightCoord{}
\partial_t\left(\displaystyle{\frac{\leftCoord{}\rho}{\rightCoord{}\vert{\bf\rightCoord{}
\vec\nabla}\theta\vert^2}}\left(\partial_t\theta\rightCoord{}
\leftCoord{}+\vec{\nabla}\theta\cdot{\vec\nabla}\phi\right)\right)+\rightCoord{}
\leftCoord{}\\\leftCoord{}[2mm]\rightCoord{}
\qquad\rightCoord{}
\partial_k\left(\displaystyle{\frac{\leftCoord{}\rho\rightCoord{}\,\partial_k\phi}\rightCoord{}
{\rightCoord{}\leftCoord{}\vert{{\vec{\nabla}}}\theta\vert^2}}\left(\partial_t\theta\rightCoord{}
\leftCoord{}+\vec{\nabla}\theta\cdot{{\vec{\nabla}}}\phi\right)\rightCoord{}
\leftCoord{}-\displaystyle{\frac{\leftCoord{}\rho\rightCoord{}\,\partial_k\theta}\rightCoord{}
{\rightCoord{}\leftCoord{}\vert{{\vec{\nabla}}}\theta\vert^{4}}}\rightCoord{}
\left(\partial_t\theta\rightCoord{}
\leftCoord{}+\vec{\nabla}\rightCoord{}
\theta\cdot{{\vec{\nabla}}}\phi\right)^2\right)=0.\rightCoord{}
\label{equationsQ}\rightCoord{}
\end{array} \rightCoord{}
\right. \rightCoord{}
\end{array} \rightCoord{}
\rightCoord{}}{0mm}{23}{47}{
\begin{array}{ll} 
\left(E_{\gamma}\right) &
\left\lbrace
\begin{array}{l} 
\partial_t\rho+\partial_k\left(\rho\partial_k\phi+
\displaystyle{\frac{\rho}
{\vert\vec{\nabla}\theta\vert^2}}\partial_k
\theta\left(\partial_t\theta+
{\vec\nabla}
\theta\cdot\vec{\nabla}\phi\right)\right)=0,
\\[3mm]
\partial_t\phi+\frac{1}{2}
\vert{\vec\nabla}\phi\vert^2+
\displaystyle{\frac{1}
{2\vert{\vec\nabla}\theta\vert^2}}
\left(\partial_t\theta+
\vec{\nabla}\theta\cdot{\vec\nabla}\phi\right)^2
-\gamma\Lambda\rho^{\gamma-1}=0,
\\[3mm]
\partial_t\left(\displaystyle{\frac{\rho}{\vert{\bf
\vec\nabla}\theta\vert^2}}\left(\partial_t\theta
+\vec{\nabla}\theta\cdot{\vec\nabla}\phi\right)\right)+
\\[2mm]
\qquad
\partial_k\left(\displaystyle{\frac{\rho\,\partial_k\phi}
{\vert{{\vec{\nabla}}}\theta\vert^2}}\left(\partial_t\theta
+\vec{\nabla}\theta\cdot{{\vec{\nabla}}}\phi\right)
-\displaystyle{\frac{\rho\,\partial_k\theta}
{\vert{{\vec{\nabla}}}\theta\vert^{4}}}
\left(\partial_t\theta
+\vec{\nabla}
\theta\cdot{{\vec{\nabla}}}\phi\right)^2\right)=0.
\end{array} 
\right. 
\end{array} 
}{1}\coordE{}\end{eqnarray}
The velocity field
\myHighlight{${\bf u}$}\coordHE{} here is expressed in terms of \myHighlight{$\theta$}\coordHE{} and \myHighlight{$\phi$}\coordHE{} by
\myHighlight{${\bf u}=\vec{\nabla}\phi-
(\vec{\nabla}\theta/\vert{\vec{\nabla}}\theta\vert^2)
\left(\partial_t\theta
+\vec{\nabla}\theta\cdot{\vec\nabla}\phi\right).
$}\coordHE{}

Below we analyse the symmetries of
(\ref{equationsQ}) in the Kaluza-Klein type framework of \cite{DGH}.
Non-relativistic
space-time, \myHighlight{$Q$}\coordHE{}, has coordinates \myHighlight{$({\bf x},t)$}\coordHE{}, and
can also be obtained from one higher dimensional manifold
\myHighlight{$M$}\coordHE{} with coordinates \myHighlight{$({\bf x},t,s)$}\coordHE{}, when the coordinate \myHighlight{$s$}\coordHE{}
 is factored out.
\myHighlight{$M$}\coordHE{} is endowed with the flat Lorentz
metric \myHighlight{$d{\bf x}^2+2dtds$}\coordHE{};
\myHighlight{$\Xi=\partial_s$}\coordHE{} light-like vector field.
\myHighlight{$M$}\coordHE{} is a relativistic spacetime, upon which we consider
the real fields \myHighlight{$R$}\coordHE{}, \myHighlight{$\Theta$}\coordHE{} and \myHighlight{$\Phi$}\coordHE{}. Inspired by (\ref{equationsQ}),
we postulate
\begin{eqnarray}\coord{}\boxAlignEqnarray{\leftCoord{}
\begin{array}{ll} \rightCoord{}
\left({\cal E}_{\gamma}\right) &\rightCoord{}
\left\lbrace\rightCoord{}
\begin{array}{l} \rightCoord{}
\partial_\mu\left(\displaystyle{\frac{\leftCoord{}R}{\rightCoord{}2}}\rightCoord{}
\leftCoord{}\rightCoord{}\,\partial^\mu\Phi+\displaystyle\rightCoord{}
{\rightCoord{}\leftCoord{}\frac{\leftCoord{}R\rightCoord{}\,\partial^\mu\Theta}\rightCoord{}
{\rightCoord{}\leftCoord{}(\partial_\sigma\Theta)(\partial^\sigma\Theta)}}\rightCoord{}
\partial_\nu\Theta\rightCoord{}\,\partial^\nu\Phi\right)=0\rightCoord{}
\leftCoord{}\\\leftCoord{}[3,5mm]\rightCoord{}
\partial_\mu\Phi\rightCoord{}\,\partial^\mu\Phi\rightCoord{}
\leftCoord{}+\displaystyle{\frac{\leftCoord{}1}{\rightCoord{}(\partial_\mu\Theta)\rightCoord{}
\leftCoord{}(\partial^\mu\Theta)}}\left(\partial_\nu\Phi\rightCoord{}\,\rightCoord{}
\partial^\nu\Theta\right)^2-\rightCoord{}
\gamma\Lambda\rightCoord{}
R^{\gamma-1}=0,\\\leftCoord{}[3,5mm]\rightCoord{}
\partial_\mu\left(R\rightCoord{}\,\partial^\mu\Phi\rightCoord{}\,\rightCoord{}
\displaystyle{\frac{\leftCoord{}\partial_\sigma\Phi\rightCoord{}\,\rightCoord{}
\partial^\sigma\Theta}\rightCoord{}
{\rightCoord{}\leftCoord{}(\partial_\nu\Theta\rightCoord{}\,\partial^\nu\Theta)}}-\frac{\leftCoord{}R}{\rightCoord{}2}\rightCoord{}\,\rightCoord{}
\partial^\mu\Theta\rightCoord{}\,\rightCoord{}
\displaystyle{\frac{\leftCoord{}(\partial_\sigma\Phi\rightCoord{}\,\partial^\sigma\rightCoord{}
\Theta)^2}\rightCoord{}
{\rightCoord{}\leftCoord{}(\partial_\nu\Theta\rightCoord{}\,\partial^\nu\Theta)^2}}\right)=0.\rightCoord{}
\label{equationsM}\rightCoord{}
\end{array} \rightCoord{}
\right. \rightCoord{}
\end{array} \rightCoord{}
\rightCoord{}}{0mm}{17}{50}{
\begin{array}{ll} 
\left({\cal E}_{\gamma}\right) &
\left\lbrace
\begin{array}{l} 
\partial_\mu\left(\displaystyle{\frac{R}{2}}
\,\partial^\mu\Phi+\displaystyle
{\frac{R\,\partial^\mu\Theta}
{(\partial_\sigma\Theta)(\partial^\sigma\Theta)}}
\partial_\nu\Theta\,\partial^\nu\Phi\right)=0
\\[3,5mm]
\partial_\mu\Phi\,\partial^\mu\Phi
+\displaystyle{\frac{1}{(\partial_\mu\Theta)
(\partial^\mu\Theta)}}\left(\partial_\nu\Phi\,
\partial^\nu\Theta\right)^2-
\gamma\Lambda
R^{\gamma-1}=0,\\[3,5mm]
\partial_\mu\left(R\,\partial^\mu\Phi\,
\displaystyle{\frac{\partial_\sigma\Phi\,
\partial^\sigma\Theta}
{(\partial_\nu\Theta\,\partial^\nu\Theta)}}-\frac{R}{2}\,
\partial^\mu\Theta\,
\displaystyle{\frac{(\partial_\sigma\Phi\,\partial^\sigma
\Theta)^2}
{(\partial_\nu\Theta\,\partial^\nu\Theta)^2}}\right)=0.
\end{array} 
\right. 
\end{array} 
}{1}\coordE{}\end{eqnarray}

To complete our Kaluza-Klein framework, we need to establish a correspondance
between the systems \myHighlight{$(\ref{equationsQ})$}\coordHE{} and \myHighlight{$(\ref{equationsM})$}\coordHE{}.
Below we define, for both critical values of \myHighlight{$\gamma$}\coordHE{}, a judicious (and
different) relation between the fields on \myHighlight{$M$}\coordHE{} and those on \myHighlight{$Q$}\coordHE{}, such
that the relativistic system \myHighlight{$\displaystyle{({\cal E}_{\gamma})}$}\coordHE{}
projects to the non-relativistic one
\myHighlight{$\displaystyle{({E}_{\gamma})}$}\coordHE{}. Then the
symmetries of the latter arise by projection.

\myHighlight{$\bullet$}\coordHE{} Let us first consider the standard case, \myHighlight{$\gamma=1+2/d$}\coordHE{}.
If the fields \myHighlight{$R$}\coordHE{},
\myHighlight{$\Theta$}\coordHE{} and \myHighlight{$\Phi$}\coordHE{}
are of the particular form
\begin{eqnarray}\coord{}\boxAlignEqnarray{\leftCoord{}
R({\bf x},t,s)= \rightCoord{}
\rho({\bf x},t),\;\; \rightCoord{}
\Theta({\bf x},t,s)=\theta({\bf x},t)\;\; \rightCoord{}
\Phi({\bf x},t,s)=\phi({\bf x},t)+s \rightCoord{}
\label{equivarianceduval}
\rightCoord{}}{0mm}{1}{6}{
R({\bf x},t,s)= 
\rho({\bf x},t),\;\; 
\Theta({\bf x},t,s)=\theta({\bf x},t)\;\; 
\Phi({\bf x},t,s)=\phi({\bf x},t)+s 
}{1}\coordE{}\end{eqnarray}
(which is in fact the usual equivariance condition \cite{DGH}),
then the equations \myHighlight{$\displaystyle{({\cal
E}_{1+2/d})}$}\coordHE{} project to
\myHighlight{$\displaystyle{({E}_{1+2/d})}$}\coordHE{}.

Now we determine the symmetries. One shows readily  that
 if the fields \myHighlight{$R$}\coordHE{},
\myHighlight{$\Phi$}\coordHE{} and \myHighlight{$\Theta$}\coordHE{} are solutions of equations
\myHighlight{$\displaystyle{({\cal E}_{1+2/d})}$}\coordHE{},
then their images under a conformal
transformation of \myHighlight{$M$}\coordHE{}, \myHighlight{$\varphi^\star g=\Omega^2 g$}\coordHE{}, implemented as
\myHighlight{$\tilde{R}=\Omega^d\,\varphi^\star R$}\coordHE{},
\myHighlight{$\tilde{\Phi}=\varphi^\star\Phi$}\coordHE{} and
\myHighlight{$\tilde{\Theta}=\,\varphi^\star\Theta$}\coordHE{}, also satisfy the same equations.
They are hence symmetries for (\ref{equationsM}).
To make the transformed
fields equivariant in the sense (\ref{equivarianceduval}),
however, we must restrict
ourselves to transformations
which preserve the ``vertical''vector field \myHighlight{$\Xi$}\coordHE{}. Their action on
\myHighlight{$M$}\coordHE{},
\begin{eqnarray}\coord{}\boxAlignEqnarray{\leftCoord{}
\left\lbrace
\begin{array}{l} \rightCoord{}
\tilde{{\bf x}}={\vec\gamma}-{\vec\beta}t+\rightCoord{}
\displaystyle{\frac{\leftCoord{}\delta\rightCoord{}\,{\cal\rightCoord{}
R}{\bf x}}{(1+\kappa t)}},\rightCoord{}
\leftCoord{}\\\leftCoord{}[2mm]\rightCoord{}
\tilde{t}=\displaystyle{\frac{\leftCoord{}\epsilon+\delta^2\rightCoord{}\,t}{\rightCoord{}(1+\kappa t)}},\rightCoord{}
\leftCoord{}\\\leftCoord{}[2mm]\rightCoord{}
\tilde{s}=s+\lambda(t,{\bf x}),\rightCoord{}
\qquad\rightCoord{}
\lambda(t,{\bf x})\equiv\rightCoord{}
{\rightCoord{}\leftCoord{}\vec\beta}\cdot{\bf x}-\frac{\leftCoord{}1}{\rightCoord{}2}\vert{\vec\beta}\vert^2\rightCoord{}\,t\rightCoord{}
\leftCoord{}+\displaystyle{\frac{\leftCoord{}\kappa}{\rightCoord{}2}\rightCoord{}
\frac{\leftCoord{}\vert{\bf x}\vert^2}{\rightCoord{}(1+\kappa t)}},\rightCoord{}
\label{transfM}\rightCoord{}
\end{array} \rightCoord{}
\right. \rightCoord{}
\rightCoord{}}{0mm}{12}{26}{
\left\lbrace
\begin{array}{l} 
\tilde{{\bf x}}={\vec\gamma}-{\vec\beta}t+
\displaystyle{\frac{\delta\,{\cal
R}{\bf x}}{(1+\kappa t)}},
\\[2mm]
\tilde{t}=\displaystyle{\frac{\epsilon+\delta^2\,t}{(1+\kappa t)}},
\\[2mm]
\tilde{s}=s+\lambda(t,{\bf x}),
\qquad
\lambda(t,{\bf x})\equiv
{\vec\beta}\cdot{\bf x}-\frac{1}{2}\vert{\vec\beta}\vert^2\,t
+\displaystyle{\frac{\kappa}{2}
\frac{\vert{\bf x}\vert^2}{(1+\kappa t)}},
\end{array} 
\right. 
}{1}\coordE{}\end{eqnarray}
(where \myHighlight{${\cal R}\in so(2)$}\coordHE{}, \myHighlight{${\vec\beta},
{\vec\gamma}, \epsilon, \kappa$}\coordHE{} and
\myHighlight{$\delta$}\coordHE{} are interpreted as rotation, boost, space
translation, time translation, expansion and dilatation)
projects into non-relativistic space-time, \myHighlight{$Q$}\coordHE{}, according to the classical
Schr\"{o}dinger transformations
\cite{NH, DGH}. The action on fields are
obtained by using the previous relations. Setting
\myHighlight{$M=\displaystyle{\left(\partial \tilde{x}_i/\partial x_j\right)}$}\coordHE{}, we
get
\begin{eqnarray}\coord{}\boxAlignEqnarray{\leftCoord{}
\begin{array}{l} \rightCoord{}
\left\lbrace\rightCoord{}
\begin{array}{l} \rightCoord{}
\tilde{\rho}(t,{\bf x})=\displaystyle{\frac{\leftCoord{}\delta^d}{\rightCoord{}(1+\kappa\rightCoord{}
t)^d}}\rightCoord{}\,\rho(\tilde{t},\tilde{{\bf\rightCoord{}
x}})=\hbox{det}(M)\rightCoord{}\,\rho(\tilde{t},\tilde{{\bf x}}),\rightCoord{}
\leftCoord{}\\\leftCoord{}[2.8mm]\rightCoord{}
\tilde{\phi}(t,{\bf x})=\phi(\tilde{t},\tilde{{\bf x}})\rightCoord{}
\leftCoord{}+\lambda(t,{\bf x}),\rightCoord{}
\leftCoord{}\\\leftCoord{}[2mm]\rightCoord{}
\tilde{\theta}(t,{\bf x})=\rightCoord{}
\theta(\tilde{t},\tilde{{\bf x}}).\rightCoord{}
\end{array} \rightCoord{}
\right. \rightCoord{}
\end{array} \rightCoord{}
\label{dilatation}
\rightCoord{}}{0mm}{7}{20}{
\begin{array}{l} 
\left\lbrace
\begin{array}{l} 
\tilde{\rho}(t,{\bf x})=\displaystyle{\frac{\delta^d}{(1+\kappa
t)^d}}\,\rho(\tilde{t},\tilde{{\bf
x}})=\hbox{det}(M)\,\rho(\tilde{t},\tilde{{\bf x}}),
\\[2.8mm]
\tilde{\phi}(t,{\bf x})=\phi(\tilde{t},\tilde{{\bf x}})
+\lambda(t,{\bf x}),
\\[2mm]
\tilde{\theta}(t,{\bf x})=
\theta(\tilde{t},\tilde{{\bf x}}).
\end{array} 
\right. 
\end{array} 
}{1}\coordE{}\end{eqnarray}
\goodbreak

Since the \myHighlight{$\Xi$}\coordHE{}-preserving symmetries of (\ref{equationsM}) project to
symmetries, we
 conclude that, in the viscosity--free case \myHighlight{$\xi=\eta=0$}\coordHE{}, the
(not necessarily irrotational)
system has a full Schr\"{o}dinger symmetry, as stated above.

Another way of reaching this result, closer in spirit to our first proof,
is to observe that
 Eqns. (\ref{equationsM}) derive from the relativistic Action
\begin{equation}\coord{}\boxEquation{
S=\int\Big(R\partial_\mu\Phi\,\partial^\mu\Phi
+\frac{R}{\partial_\mu\Theta\,\partial^\mu\Theta}\,
\left(\partial_\sigma\Phi\,
\partial^\sigma\Theta\right)^2-2\Lambda
R^{\gamma}\Big)\sqrt{g}d^{d+2}x,
}{
S=\int\Big(R\partial_\mu\Phi\,\partial^\mu\Phi
+\frac{R}{\partial_\mu\Theta\,\partial^\mu\Theta}\,
\left(\partial_\sigma\Phi\,
\partial^\sigma\Theta\right)^2-2\Lambda
R^{\gamma}\Big)\sqrt{g}d^{d+2}x,
}{ecuacion}\coordE{}\end{equation}
where, for convenience, we moved to a general Lorentz metric
\myHighlight{$g_{\mu\nu}$}\coordHE{} on \myHighlight{$M$}\coordHE{}.
The associated energy-momentum tensor
\myHighlight{${\cal T}_{\mu\nu}=2\delta S/\delta g^{\mu\nu}$}\coordHE{}, i.~e.,
\begin{eqnarray}\coord{}\boxAlignEqnarray{\leftCoord{}
{\rightCoord{}\leftCoord{}\cal T}_{\mu\nu} &=& R\rightCoord{}\,\partial_\mu\Phi\rightCoord{}\,\partial_\nu\Phi
\leftCoord{}-\frac{\leftCoord{}R}{\rightCoord{}2}(\partial_\sigma\Phi\rightCoord{}\,\partial^\sigma\Phi)\rightCoord{}\,g_{\mu\nu}
\leftCoord{}+\Lambda\rightCoord{}\,R^{\gamma}g_{\mu\nu}\nonumber\rightCoord{}\\
&\leftCoord{}+&\frac{\leftCoord{}R}{\rightCoord{}\partial_\sigma\Theta\rightCoord{}\,\partial^\sigma\Theta}
     \left(\partial_\mu\Phi\rightCoord{}\,\partial_\nu\Theta
     \leftCoord{}+\partial_\mu\Theta\rightCoord{}\,\partial_\nu\Phi\right)
     \leftCoord{}(\partial_\sigma\Theta\rightCoord{}\,\partial^\sigma\Phi)\nonumber\rightCoord{}\\
&\leftCoord{}-&R\rightCoord{}\, \rightCoord{}
      \partial_\mu\Theta\rightCoord{}\,\partial_\nu\Theta
      \frac{\leftCoord{}\left(\partial_\sigma\Theta\rightCoord{}\,\partial^\sigma\Phi\right)^2}
{\rightCoord{}\leftCoord{}\left(\partial_\sigma\Theta\rightCoord{}\,\partial^\sigma\Theta\right)^2}
	   \leftCoord{}-\frac{\leftCoord{}R}{\rightCoord{}2}g_{\mu\nu}
	   \frac{\leftCoord{}\left(\partial_\sigma\Theta\rightCoord{}\,\partial^\sigma\Phi
	   \right)^2} \rightCoord{}
{\rightCoord{}\leftCoord{}\left(\partial_\sigma\Theta\rightCoord{}\,\partial^\sigma\Theta\right)},
\rightCoord{}}{0mm}{16}{27}{
{\cal T}_{\mu\nu} &=& R\,\partial_\mu\Phi\,\partial_\nu\Phi
-\frac{R}{2}(\partial_\sigma\Phi\,\partial^\sigma\Phi)\,g_{\mu\nu}
+\Lambda\,R^{\gamma}g_{\mu\nu}\\
&+&\frac{R}{\partial_\sigma\Theta\,\partial^\sigma\Theta}
     \left(\partial_\mu\Phi\,\partial_\nu\Theta
     +\partial_\mu\Theta\,\partial_\nu\Phi\right)
     (\partial_\sigma\Theta\,\partial^\sigma\Phi)\\
&-&R\, 
      \partial_\mu\Theta\,\partial_\nu\Theta
      \frac{\left(\partial_\sigma\Theta\,\partial^\sigma\Phi\right)^2}
{\left(\partial_\sigma\Theta\,\partial^\sigma\Theta\right)^2}
	   -\frac{R}{2}g_{\mu\nu}
	   \frac{\left(\partial_\sigma\Theta\,\partial^\sigma\Phi
	   \right)^2} 
{\left(\partial_\sigma\Theta\,\partial^\sigma\Theta\right)},
}{1}\coordE{}\end{eqnarray}
(which generalizes the expression given in \cite{HH})
is seen to be symmetric and conserved. Relativistic conformal
invariance  requires the vanishing of its trace,
\begin{equation}\coord{}\boxEquation{
\sum_{\mu}{\cal T}^\mu_{\ \mu}=
\Lambda d\,R^{\gamma} \left(\gamma-\big[1+
\frac{2}{d}]\right)=0,
\label{relattrace}
}{
\sum_{\mu}{\cal T}^\mu_{\ \mu}=
\Lambda d\,R^{\gamma} \left(\gamma-\big[1+
\frac{2}{d}]\right)=0,
}{ecuacion}\coordE{}\end{equation}
which yields the correct polytropic exponent
\myHighlight{$\gamma=1+2/d$}\coordHE{} once again. To conclude, the Schr\"odinger
group is the \myHighlight{$\Xi$}\coordHE{}-preserving part of the (relativistic) conformal group.
It is worth mentionning that the \myHighlight{$ti,\, it$}\coordHE{} and \myHighlight{$ij$}\coordHE{} components of
 the relativistic \myHighlight{${\cal T}^{\mu\nu}$}\coordHE{} are related to
 the non-relativistic \myHighlight{$T^{\alpha\beta}$}\coordHE{} by surface terms,
 and that the non-relativistic trace condition (\ref{nrtracecond})
 follows from \myHighlight{$-T^{00}={\cal T}^s_{\ s}={\cal T}^t_{\ t}$}\coordHE{}.


\goodbreak
Let us now return to the general equations (\myHighlight{$1.1$}\coordHE{})-(\myHighlight{$1.3$}\coordHE{})
 including viscosity.  We first determine  how \myHighlight{${\bf u}$}\coordHE{}
transforms. Let us define on \myHighlight{$M$}\coordHE{} an \myHighlight{$s$}\coordHE{}-independent  vector
\myHighlight{$(k_\nu)\equiv (k_t,{\bf u},k_s)$}\coordHE{},
\begin{eqnarray}\coord{}\boxAlignEqnarray{\leftCoord{}
k_\nu=
\partial_\nu\Phi-\displaystyle{\frac{\leftCoord{}\partial_\nu\Theta}
{\rightCoord{}\leftCoord{}\left(\partial_
\sigma\Theta
\leftCoord{}\rightCoord{}\,\partial^\sigma\Theta\right)}\left(\partial_\mu\Theta
\leftCoord{}\rightCoord{}\,\partial^\mu\Phi\right)}. \rightCoord{}
\label{budapest}
\rightCoord{}}{0mm}{5}{6}{
k_\nu=
\partial_\nu\Phi-\displaystyle{\frac{\partial_\nu\Theta}
{\left(\partial_
\sigma\Theta
\,\partial^\sigma\Theta\right)}\left(\partial_\mu\Theta
\,\partial^\mu\Phi\right)}. 
}{1}\coordE{}\end{eqnarray}
Using the transformation rule on \myHighlight{$M$}\coordHE{} of this vector,
\myHighlight{$\displaystyle{\tilde{k}_\mu=
\left(\partial \tilde{x}^\nu/\partial {x}^\mu
\right)k_\nu}$}\coordHE{}, the action on \myHighlight{${\bf u}$}\coordHE{}, the
space component of \myHighlight{$k_{\nu}$}\coordHE{}, is obtained, namely
\begin{eqnarray}\coord{}\boxAlignEqnarray{\leftCoord{}
\tilde{{\bf u}}\rightCoord{}\,(t,{\bf x})=[{\cal
R}\rightCoord{}\,(\hbox{det}\rightCoord{}\,M)^{1/d}]\rightCoord{}\,{\bf u}\rightCoord{}\,(\tilde{t},{\tilde{{\bf x}}}) \rightCoord{}
\leftCoord{}+\vec{\nabla}\lambda. \rightCoord{}
\rightCoord{}}{0mm}{2}{9}{
\tilde{{\bf u}}\,(t,{\bf x})=[{\cal
R}\,(\hbox{det}\,M)^{1/d}]\,{\bf u}\,(\tilde{t},{\tilde{{\bf x}}}) 
+\vec{\nabla}\lambda. 
}{1}\coordE{}\end{eqnarray}

It is interesting to observe that the restriction (ii), {\it viz}.
\myHighlight{$\chi=1$}\coordHE{}, can actually be relaxed:
 the viscosity--free Euler equations
 are invariant w.r.t. transformations (\ref{transfM})
and (\ref{dilatation}), whenever  \myHighlight{$\tilde{\chi}=\chi$}\coordHE{}.
The first term in  (1.2), \myHighlight{$\rho D{\bf u}$}\coordHE{},  transforms in fact into
\myHighlight{$({\rm det}\,M)^{1+3/d}\rho D{\bf u}$}\coordHE{}, and if
\myHighlight{$\tilde{\chi}=\chi$}\coordHE{}, then the term \myHighlight{$\vec{\nabla}(\chi\rho^{1+2/d})$}\coordHE{} becomes
\myHighlight{$({\rm det}\,M)^{1+3/d}\vec{\nabla}(\chi\rho^{1+2/d})$}\coordHE{} so that eqn. (1.2)
merely gets multiplied by an overall factor. The other equations are
plainly invariant.


Now, if
\myHighlight{$\tilde{\eta}=(\hbox{det}\,M)\,\eta$}\coordHE{} and
\myHighlight{$\tilde{\xi}=(\hbox{det}\,M)\,\xi$}\coordHE{}, the viscosity term
transforms as
\begin{equation}\coord{}\boxEquation{
   \begin{array}{c}
V_i\to \tilde{V}_i=(\hbox{det}\,M)^{1+\frac{3}{d}}\,V_i+\hfill
\\[2mm]
(\hbox{det}\,M)^{1+\frac{1}{d}}
\left[\tilde{\partial}_i(\xi\Delta
\lambda)+\tilde{\partial}_j\left(\eta[2
\partial_i\partial_j\lambda-\frac{2}{d}\delta_{ij}\Delta
\lambda]\right)\right].
\label{obstruct}
\end{array}
}{
   \begin{array}{c}
V_i\to \tilde{V}_i=(\hbox{det}\,M)^{1+\frac{3}{d}}\,V_i+\hfill
\\[2mm]
(\hbox{det}\,M)^{1+\frac{1}{d}}
\left[\tilde{\partial}_i(\xi\Delta
\lambda)+\tilde{\partial}_j\left(\eta[2
\partial_i\partial_j\lambda-\frac{2}{d}\delta_{ij}\Delta
\lambda]\right)\right].
\end{array}
}{ecuacion}\coordE{}\end{equation}

Invariance of Eqn. \myHighlight{$(1.2)$}\coordHE{} requires the second term here to vanish.
For  \myHighlight{$\lambda$}\coordHE{} in (\ref{transfM}), this is automatical for the shear
viscosity field \myHighlight{$\eta$}\coordHE{}. The bulk viscosity field, \myHighlight{$\xi$}\coordHE{}, however,
breaks the expansions, leaving us with dilatational
symmetry only. For time-independent fields
one also have time-translations. (This is consistent, owing to \myHighlight{$\{\H,
\D\}=\H$}\coordHE{}). When the viscosity fields only depend on time, though,
the residual symmetry includes the expansions but break the
time-translational invariance. These results confirm the conclusion
of \cite{RS} obtained in a rather different way.



\myHighlight{$\bullet$}\coordHE{} Next, we consider the d-brane potential, \myHighlight{$\gamma=-1$}\coordHE{}. The
``non-relativistic conformal symmetries'' (i. e. dilatations and
expansions) are plainly broken.
However, when the motion is irrotional and viscosity--free, this
\myHighlight{$(d+1)$}\coordHE{} dimensional non-relativistic model
 admits the \myHighlight{$(d+1,1)$}\coordHE{}-dimensional Poincar\'e group
 as symmetry \cite{BoHo, JAC, BJ}. Generalising the results and
 the procedure presented in
\cite{HH}, now we show that the not necessarily irrotational but
still viscosity--free system
\myHighlight{$\displaystyle{({E}_{-1})}$}\coordHE{} is Poincar\'e
symmetric. Our previous equivariance condition (\ref{equivarianceduval})
is seen to be be too restrictive and we propose to
 relate  instead
the fields defined on \myHighlight{$M$}\coordHE{} and \myHighlight{$Q$}\coordHE{}
according to
\begin{eqnarray}\coord{}\boxAlignEqnarray{\leftCoord{}
\left\lbrace
\begin{array}{l} \rightCoord{}
\rho({\bf x},t)=R\left({\bf x},t,-\phi({\bf\rightCoord{}
x},t)\right)\rightCoord{}\,\partial_s\Phi\left({\bf x},t,-\phi({\bf\rightCoord{}
x},t)\right),\\\leftCoord{}[1,2mm]\rightCoord{}
\Phi\left({\bf x},t,-\phi({\bf x},t)\right)=0,\\\leftCoord{}[1.2mm]\rightCoord{}
\Theta({\bf x},t,s)=\theta({\bf x},t).\rightCoord{}
\end{array} \rightCoord{}
\right. \rightCoord{}
\label{equivariancehh}
\rightCoord{}}{0mm}{3}{11}{
\left\lbrace
\begin{array}{l} 
\rho({\bf x},t)=R\left({\bf x},t,-\phi({\bf
x},t)\right)\,\partial_s\Phi\left({\bf x},t,-\phi({\bf
x},t)\right),\\[1,2mm]
\Phi\left({\bf x},t,-\phi({\bf x},t)\right)=0,\\[1.2mm]
\Theta({\bf x},t,s)=\theta({\bf x},t).
\end{array} 
\right. 
}{1}\coordE{}\end{eqnarray}


Here the point
\myHighlight{$\left(t,{\bf x},-\phi(t,{\bf x})\right)$}\coordHE{} in \myHighlight{$M$}\coordHE{} is defined as a zero
of the field \myHighlight{$\Phi=0$}\coordHE{}. Note that \myHighlight{$R$}\coordHE{}
can depend on the \myHighlight{$s$}\coordHE{} variable; however,
 \myHighlight{$\rho$}\coordHE{} is already defined  \myHighlight{$Q$}\coordHE{}.
It is easy to see
that this condition is more general than  classical equivariance
(\ref{equivarianceduval}). As previously,
\myHighlight{$\displaystyle{({\cal E}_{-1})}$}\coordHE{}
with the constraint (\ref{equivariancehh}), project into \myHighlight{$Q$}\coordHE{} as
\myHighlight{$\displaystyle{({E}_{-1})}$}\coordHE{}. Let us  insist that this projection is
only possible for the d-brane potential \cite{HH}. The advantage of the general
equivariance is that, now, we can consider
transformations which do not  necessarily preserve \myHighlight{$\Xi$}\coordHE{}.
But the
particular form of our potential restricts ourselves to
consider only isometric transformations. These latter are symmetries of
equations
\myHighlight{$\displaystyle{({\cal E}_{-1})}$}\coordHE{} coupled to the constraint
(\ref{equivariancehh}). The action of the \myHighlight{$\Xi$}\coordHE{}-preserving
isometries lead to the extended Galilei transformations. The
non-preserving part is composed by \myHighlight{$(d+1)$}\coordHE{} generators whose action on \myHighlight{$M$}\coordHE{}
is given by \cite{HH} :
\begin{eqnarray}\coord{}\boxAlignEqnarray{\leftCoord{}
\begin{array}{l} \rightCoord{}
\left\lbrace\rightCoord{}
\begin{array}{l} \rightCoord{}
\tilde{{\bf x}}={\bf x}-\vec{\omega}s\rightCoord{}
\rightCoord{}\\\leftCoord{}
\tilde{t}=\rightCoord{}
e^{\alpha}\left(t+\vec{\omega}\cdot{\bf x}-\frac{\leftCoord{}1}{\rightCoord{}2}\rightCoord{}
\vert\vec{\omega}\vert^2\rightCoord{}\,s\right),\rightCoord{}
\rightCoord{}\\\leftCoord{}
\tilde{s}=e^{-\alpha}s,\rightCoord{}
\end{array} \rightCoord{}
\right. \rightCoord{}
\end{array} \rightCoord{}
\label{timedilatationM}
\rightCoord{}}{0mm}{4}{17}{
\begin{array}{l} 
\left\lbrace
\begin{array}{l} 
\tilde{{\bf x}}={\bf x}-\vec{\omega}s
\\
\tilde{t}=
e^{\alpha}\left(t+\vec{\omega}\cdot{\bf x}-\frac{1}{2}
\vert\vec{\omega}\vert^2\,s\right),
\\
\tilde{s}=e^{-\alpha}s,
\end{array} 
\right. 
\end{array} 
}{1}\coordE{}\end{eqnarray}
where \myHighlight{$\alpha$}\coordHE{} and \myHighlight{$\vec{\omega}$}\coordHE{} are the parameters associated with
time dilatation and antiboost, respectively.
Our transformations act on
fields naturally, as
\myHighlight{$\tilde{R}(x,t,s)=R(\tilde{\bf x},\tilde{t},\tilde{s})$}\coordHE{}, etc.
The projection into \myHighlight{$Q$}\coordHE{} yields  \cite{JAC, BJ}
\begin{eqnarray}\coord{}\boxAlignEqnarray{\leftCoord{}
\begin{array}{lllll} \rightCoord{}
\left\lbrace\rightCoord{}
\begin{array}{l} \rightCoord{}
\tilde{\bf x}={\bf x}+\vec{\omega}\rightCoord{}\,\phi(\tilde{\bf x},\tilde{t}),\rightCoord{}
\leftCoord{}\\\leftCoord{}[2mm]\rightCoord{}
\tilde{t}=e^{\alpha}\rightCoord{}
\left(t+\frac{\leftCoord{}1}{\rightCoord{}2}\vec{\omega}\cdot({\bf x}+\rightCoord{}
\tilde{\bf x})\right)\rightCoord{}
\end{array} \rightCoord{}
\right. \rightCoord{}
&\leftCoord{}\rightCoord{}
\hbox{and} \rightCoord{}
&\leftCoord{}\rightCoord{}
\left\lbrace
\begin{array}{l} \rightCoord{}
\tilde{\rho}({\bf x},t)=\rho(\tilde{\bf x},\tilde{t})\rightCoord{}\,J^{-1}\rightCoord{}
\leftCoord{}\\\leftCoord{}[2mm]\rightCoord{}
\tilde{\phi}({\bf x},t)=e^\alpha\phi(\tilde{\bf x},\tilde{t})\rightCoord{}
\leftCoord{}\\\leftCoord{}[2mm]\rightCoord{}
\tilde{\theta}({\bf x},t)=\theta(\tilde{\bf x},\tilde{t})\rightCoord{}
\end{array} \rightCoord{}
\right. \rightCoord{}
&\leftCoord{}\rightCoord{}
\hbox{} \rightCoord{}
\end{array} \rightCoord{}
\label{lise}
\rightCoord{}}{0mm}{11}{29}{
\begin{array}{lllll} 
\left\lbrace
\begin{array}{l} 
\tilde{\bf x}={\bf x}+\vec{\omega}\,\phi(\tilde{\bf x},\tilde{t}),
\\[2mm]
\tilde{t}=e^{\alpha}
\left(t+\frac{1}{2}\vec{\omega}\cdot({\bf x}+
\tilde{\bf x})\right)
\end{array} 
\right. 
&
\hbox{and} 
&
\left\lbrace
\begin{array}{l} 
\tilde{\rho}({\bf x},t)=\rho(\tilde{\bf x},\tilde{t})\,J^{-1}
\\[2mm]
\tilde{\phi}({\bf x},t)=e^\alpha\phi(\tilde{\bf x},\tilde{t})
\\[2mm]
\tilde{\theta}({\bf x},t)=\theta(\tilde{\bf x},\tilde{t})
\end{array} 
\right. 
&
\hbox{} 
\end{array} 
}{1}\coordE{}\end{eqnarray}
where \myHighlight{$J$}\coordHE{} is the Jacobian of the transformation given by
\begin{eqnarray}\coord{}\boxAlignEqnarray{\leftCoord{}
J=e^\alpha\left[1-\sum_{\rightCoord{}k\rightCoord{}}\rightCoord{}\,\omega_k\rightCoord{}\,\tilde{\partial}_{k}
\phi(\tilde{\bf x},\tilde{t}) \rightCoord{}
\leftCoord{}-\frac{\leftCoord{}1}{\rightCoord{}2}\vert\vec{\omega}\vert^2\rightCoord{}\,\partial_{\tilde{t}}
\phi(\tilde{\bf x},\tilde{t})\right]^{-1}. \rightCoord{}
\rightCoord{}}{0mm}{3}{10}{
J=e^\alpha\left[1-\sum_{k}\,\omega_k\,\tilde{\partial}_{k}
\phi(\tilde{\bf x},\tilde{t}) 
-\frac{1}{2}\vert\vec{\omega}\vert^2\,\partial_{\tilde{t}}
\phi(\tilde{\bf x},\tilde{t})\right]^{-1}. 
}{1}\coordE{}\end{eqnarray}

As in the standard case, the vector \myHighlight{$k_\mu$}\coordHE{}
(\ref{budapest}) can be used to determine the transformation on the
velocity. But now because of this particular equivariance,
the velocity is equal to
\myHighlight{${\bf u}=\displaystyle{({\bf k}/\partial_s\Phi)(t,{\bf x},
-\phi(t,{\bf x}))}$}\coordHE{} and a similar calculation yields instead
\begin{eqnarray}\coord{}\boxAlignEqnarray{\leftCoord{}
\tilde{{\bf u}}(t,{\bf x})=J\left[{\bf u}(\tilde{t},\tilde{\bf x}) \rightCoord{}
\leftCoord{}+\vec{\omega} \rightCoord{}
\left(\tilde{\partial}_{t}\phi-
\frac{\leftCoord{}\partial_{\tilde{t}}\theta}{\rightCoord{}\sum(
\tilde{\partial}_{k}\theta)^2}
\left(\partial_{\tilde{t}}\theta+\tilde{\partial}_{m}
\theta\;\tilde{\partial}_{m}\phi\right)
\right)\right]. \rightCoord{}
\label{chat}
\rightCoord{}}{0mm}{3}{6}{
\tilde{{\bf u}}(t,{\bf x})=J\left[{\bf u}(\tilde{t},\tilde{\bf x}) 
+\vec{\omega} 
\left(\tilde{\partial}_{t}\phi-
\frac{\partial_{\tilde{t}}\theta}{\sum(
\tilde{\partial}_{k}\theta)^2}
\left(\partial_{\tilde{t}}\theta+\tilde{\partial}_{m}
\theta\;\tilde{\partial}_{m}\phi\right)
\right)\right]. 
}{1}\coordE{}\end{eqnarray}


As in
the standard case, the viscosity term breaks most of the symmetry. A rather
tedious
calculation shows in fact that, under a Poincar\'e transformation,
the viscosity term (\ref{viscosityterms})
transforms as
\begin{equation}\coord{}\boxEquation{
    \tilde{V}_{i}=e^{\alpha}\,V_{i}+F(\vec{\omega},\xi,\eta),
}{
    \tilde{V}_{i}=e^{\alpha}\,V_{i}+F(\vec{\omega},\xi,\eta),
}{ecuacion}\coordE{}\end{equation}
where \myHighlight{$F(\vec{\omega},\xi,\eta)$}\coordHE{} is a complicated expression
which vanishes for \myHighlight{$\vec{\omega},\ \xi,$}\coordHE{} or \myHighlight{$\eta$}\coordHE{} equal zero.
For non-trivial viscosity, this means that the antiboosts are broken.
Eq. (1.2) is, however, merely multiplied by \myHighlight{$e^{\alpha}$}\coordHE{}
 under \myHighlight{$\Delta~: t\to e^{\alpha}t$}\coordHE{}~:
time  (rather then
non-relativistic) dilatation, \myHighlight{$\Delta$}\coordHE{},
is a residual symmetry.


%%%%%%%%%%%%%%%%%%%%%%%%%%%%%%%%%%%%%%%%%%%%%%%%%%
\section{Explosion/implosion duality and cosmology}
%%%%%%%%%%%%%%%%%%%%%%%%%%%%%%%%%%%%%%%%%%%%%%%%%%

The clue of Drury and Mendon\c ca \cite{DM} is to map, using the ``duality
transformation''
 \myHighlight{$\Sigma: \tilde{t}=-1/t,\;
\tilde{\bf x}={\bf x}/t$}\coordHE{}, supernova
{\it explosion} at time \myHighlight{$t=0$}\coordHE{}
into  an {\it implosion} starting at \myHighlight{$\tilde{t}=-\infty$}\coordHE{} and evolving to
\myHighlight{$\tilde{t}=0$}\coordHE{}. Then they find that, implementing \myHighlight{$\Sigma$}\coordHE{} on the fields as
\myHighlight{$\tilde{\rho}=a^3\rho
$}\coordHE{} and
\myHighlight{$\tilde{\bf u}=a\,{\bf u}-\dot{a}\,{\bf x}$}\coordHE{},
the equations
of viscosity--free polytropic
hydrodynamical system with \myHighlight{$\chi=1$}\coordHE{}
are  invariant when \myHighlight{$a(t)\propto t$}\coordHE{} and \myHighlight{$\gamma=5/3$}\coordHE{}.

Curiously, their \myHighlight{$\Sigma$}\coordHE{}
appeared before in cosmology.
The relation is explained as follows.
 In the uniformly expanding newtonian cosmological model \cite{SOU},
the gravitational acceleration has the form
\myHighlight{${\bf g}=-(B/a^3){\bf x}$}\coordHE{},
where \myHighlight{$B$}\coordHE{} is a constant related to the scale factor \myHighlight{$a(t)$}\coordHE{}
as \myHighlight{$B=-a^2\ddot{a}$}\coordHE{}.  The Hubble constant is \myHighlight{$H=\dot{a}/a$}\coordHE{},
and \myHighlight{${\bf g}$}\coordHE{} satisfies
\myHighlight{$\vec\nabla\cdot{\bf g}=-4\pi G\rho$}\coordHE{}
(rather than the Einstein equations, as in relativity).
Combining this with \myHighlight{$\dot{\bf x}=H{\bf x}$}\coordHE{} and \myHighlight{$\ddot{\bf x}={\bf g}$}\coordHE{} yields
\begin{equation}\coord{}\boxEquation{
    \big(\dot{a}\big)^2=\frac{2B}{a}-K
    \qquad\hbox{and}\qquad
    \rho=\frac{3B}{4\pi Ga^{3}},
}{
    \big(\dot{a}\big)^2=\frac{2B}{a}-K
    \qquad\hbox{and}\qquad
    \rho=\frac{3B}{4\pi Ga^{3}},
}{ecuacion}\coordE{}\end{equation}
where \myHighlight{$K$}\coordHE{} is another constant now unrelated to space curvature.
This non-relativistic model is, however,
equivalent to the relativistic
Friedmann universe with constant curvature \myHighlight{$K$}\coordHE{} \cite{HE}.
The model is also conveniently described \cite{DGH}
by the (``Kaluza--Klein'') \myHighlight{$5$}\coordHE{}-metric
\begin{equation}\coord{}\boxEquation{
    d{\bf x}^2+2dtds-\frac{B\,{\bf x}^{2}}{a^3}dt^2,
    \label{expuniv}
}{
    d{\bf x}^2+2dtds-\frac{B\,{\bf x}^{2}}{a^3}dt^2,
    }{ecuacion}\coordE{}\end{equation}
whose gravitational field equation requires indeed
\myHighlight{$\bigtriangleup \big(B{\bf x}^2a(t)^{-3}\big)=8\pi G\rho$}\coordHE{} as above.
Now this metric can be conformally mapped to flat space
with metric \myHighlight{$d{\tilde{\bf x}}^2+2d\tilde{t}d\tilde{s}$}\coordHE{}, using
\begin{equation}\coord{}\boxEquation{
    \tilde{\bf x}=\frac{{\bf x}}{a},
    \qquad
    \tilde{t}=\int \frac{dt}{a^{2}},
    \qquad
    \tilde{s}=s+\smallover1/2H{\bf x}^2.
    \label{confflat}
}{
    \tilde{\bf x}=\frac{{\bf x}}{a},
    \qquad
    \tilde{t}=\int \frac{dt}{a^{2}},
    \qquad
    \tilde{s}=s+\smallover1/2H{\bf x}^2.
    }{ecuacion}\coordE{}\end{equation}
The (inverse of) (\ref{confflat}) carries the flat-space
hydrodynamical equations
into those valid in the expanding universe.

For the choice of Drury and Mendon\c ca   \myHighlight{$B=0$}\coordHE{}, so that
their expanding metric (\ref{expuniv}) is flat
and has therefore little cosmological interest since then also \myHighlight{$\rho=0$}\coordHE{}.
Ignoring this aspect, we note that the transformation (\ref{confflat}),
which becomes now precisely \myHighlight{$\Sigma$}\coordHE{} completed with
\myHighlight{$s\to s+{\bf x}^2/2t$}\coordHE{},  is a conformal transformation of flat space into
itself. The invariance of the Euler equations
under \myHighlight{$\Sigma$}\coordHE{} follows. This is of course consistent with \myHighlight{$\Sigma$}\coordHE{}
belonging to the \myHighlight{${\rm SL}(2,\IR)$}\coordHE{} invariance group of the free system
discussed above. Unfortunately, this symmetry is broken by the viscosity.

Interestingly, the  map \myHighlight{$\Sigma$}\coordHE{} has also been used
to solve planetary motion when the gravitational constant changes
inversely with time \cite{VIN, DGH}.
It is worth mentionning also that a Friedmann metric
containing a perfect fluid with equation of state
\myHighlight{$p=(\gamma-1)\rho$}\coordHE{} has also been studied \cite{BARROW}.


%%%%%%%%%%%%%%%%%%%%%%%%%%%%%%%%%%%%%%%%%%%%%%%%%%%%%
\section{Schr\"odinger fields and the Madelung fluid}
%%%%%%%%%%%%%%%%%%%%%%%%%%%%%%%%%%%%%%%%%%%%%%%%%%%%%

Let us conclude with a remark on the  well-known
Schr\"odinger invariance
of the non-linear Schr\"odinger equation
\myHighlight{$i\partial_{t}\psi=-\bigtriangleup\psi/2+\lambda\,\vert\psi\vert^{4/d+1}\psi$}\coordHE{}.
Decomposing the Schr\"odinger field
into module and phase, \myHighlight{$\psi=\sqrt{\rho}\,e^{i\phi}$}\coordHE{}, yields in fact
the hydrodynamical system referred to as the Madelung
fluid \cite{MAD},
\begin{eqnarray}\coord{}\boxAlignEqnarray{\leftCoord{}
\partial_t\rho &+ &\vec{\nabla}\cdot(\rho\vec{\nabla}\phi)=0,
\rightCoord{}\\\leftCoord{}
\partial_t\phi&+&\frac{\leftCoord{}1}{\rightCoord{}2}\vert\vec{\nabla}\phi\vert^2
\leftCoord{}=-\frac{\leftCoord{}1}{\rightCoord{}4\rho}\left[\frac{\leftCoord{}1}{\rightCoord{}2} \rightCoord{}
\frac{\leftCoord{}\vert\vec{\nabla}\rho\vert^2}{\rightCoord{}\rho} \rightCoord{}
\leftCoord{}-{\Delta\rho}\right] \rightCoord{}
\leftCoord{}+\partial_{\rho}U,
\label{madelungequations}
\rightCoord{}}{0mm}{9}{10}{
\partial_t\rho &+ &\vec{\nabla}\cdot(\rho\vec{\nabla}\phi)=0,
\\
\partial_t\phi&+&\frac{1}{2}\vert\vec{\nabla}\phi\vert^2
=-\frac{1}{4\rho}\left[\frac{1}{2} 
\frac{\vert\vec{\nabla}\rho\vert^2}{\rho} 
-{\Delta\rho}\right] 
+\partial_{\rho}U,
}{1}\coordE{}\end{eqnarray}
where \myHighlight{$U=\lambda\rho^{(2/d+1)}$}\coordHE{}.
Eqns \myHighlight{$(3.1)$}\coordHE{} and \myHighlight{$(3.2)$}\coordHE{} can be obtained from the
irrotational and viscosity--free Euler equations
choosing the field \myHighlight{$\chi$}\coordHE{} non-trivially,
\begin{eqnarray}\coord{}\boxAlignEqnarray{\leftCoord{}
\chi=\frac{\leftCoord{}d}{\rightCoord{}8{\Lambda}\rho^{2/d+1}}\rightCoord{}\,\left[ \rightCoord{}
\frac{\leftCoord{}\vert{\vec\nabla}\rho\vert^2}{\rightCoord{}2\rho} \rightCoord{}
\leftCoord{}-\bigtriangleup\rho\right]. \rightCoord{}
\label{chirelation}
\rightCoord{}}{0mm}{4}{8}{
\chi=\frac{d}{8{\Lambda}\rho^{2/d+1}}\,\left[ 
\frac{\vert{\vec\nabla}\rho\vert^2}{2\rho} 
-\bigtriangleup\rho\right]. 
}{1}\coordE{}\end{eqnarray}
Now, as seen above, the general Euler equations with the standard
polytropic exponent
\myHighlight{$\gamma=1+2/d$}\coordHE{},
 are
Schr\"{o}dinger invariant whenever \myHighlight{$\tilde{\chi}=\chi$}\coordHE{}.
Using (\ref{dilatation}),
we can show that our \myHighlight{$\chi$}\coordHE{} transforms precisely in this way.
Therefore, the Madelung equations
 are Schr\"{o}dinger invariant.

In is worth noting that for the membrane potential \myHighlight{$\gamma=-1$}\coordHE{}
one can still choose such a \myHighlight{$\chi$}\coordHE{}. However, owing to the bracketed
term, \myHighlight{$\widetilde{\chi}\neq\chi$}\coordHE{}, so that  the Poincar\'e symmetry is
broken.
The non-relativistic conformal symmetries are also broken, and
we are left with a mere Galilei symmetry.

\kikezd{Note added}. After this paper has been accepted, we became
aware of a paper by Bordemann and Hoppe \cite{BoHo2}, which offers
yet another way to derive the Schr\"odinger invariance.
For simplicity,
we only spell this out in the irrotational case \myHighlight{$\theta=0$}\coordHE{}.
 %Let us assume that \myHighlight{$\gamma\neq1$}\coordHE{}.
Expressing
\myHighlight{$\rho$}\coordHE{} from the second equation in (\ref{equationsQ}) and
inserting into the first one yields the
so-called ``Steichen equation'',
which in fact derives from the Lagrangian
\begin{equation}\coord{}\boxEquation{
{\cal L}=\left[\partial_t\phi+
\frac{1}{2}\big(\vec\nabla\phi)^2
\right]^{\frac{\gamma}{\gamma-1}}.
\label{BI}
}{
{\cal L}=\left[\partial_t\phi+
\frac{1}{2}\big(\vec\nabla\phi)^2
\right]^{\frac{\gamma}{\gamma-1}}.
}{ecuacion}\coordE{}\end{equation}
Under a non-relativistic
dilatation \myHighlight{${\cal L}$}\coordHE{} scales as
\myHighlight{${\cal L}\to\delta^{2\gamma/1-\gamma}\,{\cal L}$}\coordHE{};
taking into account
 the scaling of the volume element, %\myHighlight{$d^dxdt\to\delta^{d+2}d^dxdt$}\coordHE{},
invariance is obtained precisely when \myHighlight{$\gamma=1+2/d$}\coordHE{}.
Note that for the Chaplygin value \myHighlight{$\gamma=-1$}\coordHE{},
(\ref{BI}) becomes the
Lagrangian used by Jackiw and Polychronakos
\cite{JAC}.



\goodbreak
\kikezd{Acknowledgements}.
The authors are indebted to Professor R. Jackiw for his interest, advice and
for sending them his unpublished remark \cite{JPRIV}, as well as
Professor C. Duval and Dr. O. Ley for discussions.
They acknowledge the {\it Institute for Theoretical Physics}
of E\"{o}tv\"{o}s University (Budapest, Hungary) for hospitality
extended to them.

\goodbreak
\begin{thebibliography}{2}


\bibitem{DM}
{\sc L. O'C. Drury and J. T. Mendon\c ca}, astro-ph/0003385,
to appear in {\em Physics of Plasmas}, Jan. (2001).

\bibitem{NH}
{\sc R. Jackiw}, {\em Phys. Today} {\bf 25},\, 23\,
(1972);
{\sc U. Niederer}, {\em Helv. Phys. Acta} {\bf 45},\,
802\, (1972);
{\sc C. R. Hagen}, {\em Phys. Rev. D} {\bf 5},\, 377\, (1972).

\bibitem{DGH}{\sc
C. Duval, G. Gibbons and P. Horv\'athy},
{\em Phys. Rev.  D} {\bf 43},\, 3907\, (1991).
The non-relativistic
Kaluza-Klein--type framework was  proposed in
{\sc C. Duval, G. Burdet, H. P. K\"{u}nzle and
M. Perrin}, {\em Phys. Rev. D} {\bf 31},\, 1841\, (1985).

\bibitem{RS}{\sc L. O'Raifeartaigh and V. V. Sreedhar},
hep-th/0007199, submitted to {\em Physics of Fluids}.

\bibitem{BoHo}{\sc M.~Bordemann and J.~Hoppe},
{\em Phys. Lett.} {\bf B317},\,315\, (1993);
{\sc A. Jevicki}, {\em Phys. Rev.} {\bf D 57}, R5955 (1998).

\bibitem{JAC}{\sc R. Jackiw and A. P. Polychronakos}, {\em
Comm. Math. Phys.} {\bf 207},\, 107\, (1999); {\sc R. Jackiw},\,
(hep-th/9911235)\,(1999). For a review, see
{\sc R. Jackiw},\, (physics/0010042).

\bibitem{BJ}{\sc D. Bazeia and R. Jackiw}, {\em Ann. Phys.} {\bf
270},\, 146\, (1998);
{\sc D. Bazeia}, {\em Phys. Rev. D} {\bf 59},\, 085007\, (1999).

\bibitem{HH}{\sc M. Hassa\"{\i}ne and P. Horv\'athy}, {\em
Ann. Phys.} {\bf 282},\, 218\, (2000).

\bibitem{JPRIV}
{\sc R. Jackiw}, Private Communication (2000).

\bibitem{JP}
{\sc R. Jackiw and S.-Y. Pi},
{\em Nucl. Phys.} {\bf B} (Proc. Suppl). {\bf 33C}, 104 (1993).

\bibitem{CLE}{\sc H. Lamb}, Hydrodynamics, Cambridge University Press, 1942.


\bibitem{SOU}
{\sc J.-M. Souriau}, in {\em G\'eom\'etrie Symplectique et Physique
Math\'ematique}. Coll. Int. CNRS,  No. {\bf 237},
ed. J.-M. Souriau. Editions du CNRS, p. 59 (1975).

\bibitem{HE}
{\sc S. W. Hawking and G. F. R. Ellis},
 {\em The Large Scale Structure of space-time}. Cambridge  U. P. (1973).

\bibitem{VIN}
{\sc J. P. Vinti},
{\em Mon.  Not. R. Astron. Soc.} {\bf 169}, 417  (1974);
{\sc J. D. Barrow},
{\it ibid}. {\bf 282}, 1397 (1996).

\bibitem{BARROW}
{\sc J. D. Barrow},
{\em The Observatory} {\bf 113}, No 1115, 210 (1993);
{\sc H. C. Rosu},
gr-qc/0003108, {\em Mod. Phys. Lett.} {\bf A 15},  979 (2000).

\bibitem{MAD}{\sc E. Madelung}, {\em Z. Phys} {\bf 40},\, 332\, (1926).

\bibitem{BoHo2}{\sc M.~Bordemann and J.~Hoppe},
{\em Phys. Lett.} {\bf B325},\,359\, (1994).

\end{thebibliography}

\end{document}

\bye
