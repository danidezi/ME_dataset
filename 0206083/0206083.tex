
%%%%%%%%%%%%%%%%%%%%% Start \document\NeueVer.tex %%%%%%%%%%%%%%%%%%%%%

%% This document created by Scientific Word (R) Version 2.5
%% Starting shell: article


\documentclass[a4paper,11pt,oneside]{article}
\usepackage{amssymb}
\newcommand{\text}[1]{\mbox{#1}}
\newcommand{\underleftrightarrow}{\stackrel{\;+\,\longleftrightarrow\,-\;}{\longleftrightarrow}}
\newcommand{\underleftrightarrowf}{\stackrel{\;j\,\longleftrightarrow\,j'\;}{\longleftrightarrow}}
%%%%%%%%%%%%%%%%%%%%%%%%%%%%%%%%%%%%%%%%%%%%%%%%%%%%%%%%%%%%%%%%%%%%%%%%%%%%%%%%%%%%%%%%%%%%%%%%%%%
%\usepackage{makeidx}
%\usepackage{sw20lart}

%TCIDATA{TCIstyle=article/art4.lat,lart,article}

%TCIDATA{Created=Fri Dec 21 12:51:20 2001}
%TCIDATA{LastRevised=Fri Apr 05 16:39:25 2002}
%TCIDATA{Language=American English}

%\input{tcilatex}
\setlength{\textwidth}{14.7cm}
\setlength{\textheight}{20.5cm}
\setlength{\oddsidemargin}{1.0cm}
\begin{document}

\author{Hartmut Wachter \\
%EndAName
Sektion Physik der Ludwig-Maximilians-Universit\"{a}t M\"{u}nchen\\
Theresienstr. 37, D-80333 M\"{u}nchen\thanks{%
e-mail: Hartmut.Wachter@physik.uni-muenchen.de}}
\title{q-Integration on Quantum Spaces}
\maketitle

\begin{abstract}
In this article we present explicit formulae for q-integration on quantum
spaces which could be of particular importance in physics, i.e. q-deformed
Minkowski space and q-deformed Euclidean space in 3 or 4 dimensions.
Furthermore, our formulae can be regarded as a generalization of Jackson's
q-integral to 3 and 4 dimensions.
\end{abstract}

\section{Introduction}

One might say the ideas of differential calculus are as old as physical
science itself. Since its invention by J. Newton and G.W. Leibniz there
hasn't been a necessity for an essential change. Although this can be seen
as a great success one cannot ignore the fact that up to now physicists
haven't been able to present a unified description of nature by using this
traditional tool, i.e., a theory which does not break down at any possible
space-time distances.

Quantum spaces, however, which are defined as co-module algebras of quantum
groups and which can be interpreted as deformations of ordinary co-ordinate
algebras \cite{RTF90} could provide a proper framework for developing a new
kind of non-commutative analysis \cite{Wes00}, \cite{Maj93}. For our \
purposes it is sufficient to consider a quantum space as an algebra $%
\mathcal{A}_{q}$ of formal power series in the non-commuting co-ordinates $%
X_{1},X_{2},\ldots ,X_{n}$

\begin{equation}
\mathcal{A}_{q}=\mathbb{C}\left[ \left[ X_{1},\ldots X_{n}\right] \right] /%
\mathcal{I}
\end{equation}
where $\mathcal{I}$ denotes the ideal generated by the relations of the
non-commuting co-ordinates.

The algebra $\mathcal{A}_{q}$ satisfies the Poincar\'{e}-Birkhoff-Witt
property, i.e., the dimension of the subspace of homogenous polynomials
should be the same as for commuting co-ordinates. This property is the
deeper reason why the monomials of normal ordering $X_{1}X_{2}\ldots X_{n}$
constitute a basis of $\mathcal{A}_{q}$. In particular, we can establish a
vector space isomorphism between $\mathcal{A}_{q}$ and the commutative
algebra $\mathcal{A}$ generated by ordinary co-ordinates $x_{1},x_{2},\ldots
,x_{n}$: 
\begin{eqnarray}
\mathcal{W} &:&\mathcal{A}\longrightarrow \mathcal{A}_{q}, \\
\mathcal{W}\left( x_{1}^{i_{1}}\ldots x_{n}^{i_{n}}\right)
&=&X_{1}^{i_{1}}\ldots X_{n}^{i_{n}}.  \nonumber
\end{eqnarray}
This vector space isomorphism can be extended to an algebra isomorphism
introducing a non-commutative product in $\mathcal{A}$, the so-called $\star 
$-product \cite{Moy49}, \cite{MSSW00}. This product is defined by the
relation 
\begin{equation}
\mathcal{W}\left( f\star g\right) =\mathcal{W}\left( f\right) \cdot \mathcal{%
W}\left( g\right)
\end{equation}
where $f$ and $g$ are formal power series in $\mathcal{A}$. In \cite{WW01}
we have calculated the $\star $-product for quantum spaces which could be of
particular importance in physics, i.e., q-deformed Minkowski space and
q-deformed Euclidean space in three or four dimensions.

Additionally, for each of these quantum spaces exists a symmetry algebra 
\cite{Dri85}, \cite{Jim85} and a covariant differential calculus \cite{WZ91}%
, which can provide an action upon the quantum spaces under consideration.
If $\mathcal{H}$ is an algebra which acts on $\mathcal{A}$ covariantly we
are also able to introduce an action upon the corresponding
commutative algebra by means of the relation 
\begin{equation}
\mathcal{W}\left( h\triangleright f\right) :=h\triangleright \mathcal{W}%
\left( f\right) \text{,\quad }h\in \mathcal{H}\text{, }f\in \mathcal{A}\text{%
,}.
\end{equation}
In our previous work \cite{BW01} we have presented explicit
formulae for such representations.

In the following it is our aim to derive some sort of q-integration on
commutative algebras which are, via the algebra isomorphism $\mathcal{W}$,
related to q-deformed Minkowski space or q-deformed Euclidean space in three
and four dimensions. In some sense our considerations can be thought of as a
generalization of the celebrated Jackson-integral \cite{Jac27} to higher
dimensions. To this end we will start with introducing new elements which
are formally inverse to the partial derivatives of the given covariant
differential calculi. Consequently, such an extension of the algebra of
partial derivatives will lead to additional commutation relations. Finally,
the representations in \cite{BW01} will aid us in identifying this new
elements with particular solutions of some q-difference equations.

In this way, we can interpret our results as a method to discretise
classical integrals of more than one dimension. It is worth noting that from
a mathematical point of view it is not quite clear for which functions
apart from polynomials our integral expressions will converge. However,
because of some physical arguments, we will cover later on, it is reasonable
to assume that functions which do not lead to finite integrals should
be of no relevance in physics.

Furthermore, it is necessary to distinguish between left and right
integrals. For both types formulae for integration by parts can be derived.
It is also possible to define volume integrals which are invariant under
translations or the action of symmetry generators if surface terms are
neglected.

\section{q-deformed Euclidean Space in three Dimensions}

The formulae for the $\star$-product in \cite{WW01} allow us to
introduce inverse objects on the
co-ordinate algebra of the quantum spaces under consideration. Furthermore, the commutation
relations the new elements have to be subject to can be read off
from these formulae quite easily. As the partial derivatives obey the
same commutation relations as the non-commutative co-ordinates \cite{LWW97},
namely 
\begin{equation}
\partial ^{3}\partial ^{+}=q^{2}\partial ^{+}\partial ^{3},\quad \partial
^{-}\partial ^{3}=q^{2}\partial ^{3}\partial ^{-},\quad \partial
^{-}\partial ^{+}=\partial ^{+}\partial ^{-}+\lambda \left( \partial
^{3}\right) ^{2}
\end{equation}
where $\lambda =q-q^{-1}$ and $q>1$, we can also apply this method in order
to introduce elements $\left( \partial ^{A}\right) ^{-1}$, $A=\pm ,3$, with 
\begin{eqnarray}
\partial ^{+}\left( \partial ^{+}\right) ^{-1} &=&\left( \partial
^{+}\right) ^{-1}\partial ^{+}=1, \\
\partial ^{3}\left( \partial ^{3}\right) ^{-1} &=&\left( \partial
^{3}\right) ^{-1}\partial ^{3}=1,  \nonumber \\
\partial ^{-}\left( \partial ^{-}\right) ^{-1} &=&\left( \partial
^{-}\right) ^{-1}\partial ^{-}=1.  \nonumber
\end{eqnarray}
Then the remaining commutation relations take the form 
\begin{eqnarray}
\left( \partial ^{3}\right) ^{-1}\partial ^{+} &=&q^{-2}\partial ^{+}\left(
\partial ^{3}\right) ^{-1},\quad \partial ^{3}\left( \partial ^{+}\right)
^{-1}=q^{-2}\left( \partial ^{+}\right) ^{-1}\partial ^{3}, \\
\left( \partial ^{-}\right) ^{-1}\partial ^{3} &=&q^{-2}\partial ^{3}\left(
\partial ^{-}\right) ^{-1},\quad \partial ^{-}\left( \partial ^{3}\right)
^{-1}=q^{-2}\left( \partial ^{3}\right) ^{-1}\partial ^{-},  \nonumber \\
\left( \partial ^{-}\right) ^{-1}\partial ^{+} &=&\partial ^{+}\left(
\partial ^{-}\right) ^{-1}-q^{-4}\lambda \left( \partial ^{3}\right)
^{2}\left( \partial ^{-}\right) ^{-2},  \nonumber \\
\partial ^{-}\left( \partial ^{+}\right) ^{-1} &=&\left( \partial
^{+}\right) ^{-1}\partial ^{-}-q^{-4}\lambda \left( \partial ^{+}\right)
^{-2}\left( \partial ^{3}\right) ^{2}.  \nonumber
\end{eqnarray}
In addition, the inverse elements $\left( \partial ^{A}\right) ^{-1}$, $%
A=\pm ,3$, have to be subject to the following identities: 
\begin{eqnarray} \label{inv}
\left( \partial ^{3}\right) ^{-1}\left( \partial ^{+}\right) ^{-1}
&=&q^{2}\left( \partial ^{+}\right) ^{-1}\left( \partial ^{3}\right)
^{-1},\quad \left( \partial ^{-}\right) ^{-1}\left( \partial ^{3}\right)
^{-1}=q^{2}\left( \partial ^{3}\right) ^{-1}\left( \partial ^{-}\right)
^{-1}, \\
\left( \partial ^{-}\right) ^{-1}\left( \partial ^{+}\right) ^{-1}
&=&\sum_{i=0}^{\infty }\lambda ^{i}\left[ \left[ i\right] \right]
_{q^{4}}!\left(\left[\begin{array}{c} -1\\i\end{array}\right]_{q^{4}}\right) ^{2}\left( \partial
^{+}\right) ^{-\left( i+1\right) }\left( \partial ^{3}\right) ^{2i}\left(
\partial ^{-}\right) ^{-\left( i+1\right) }.  \nonumber
\end{eqnarray}
For the commutation relations with the symmetry generators we find the
expressions 
\begin{eqnarray}
L^{+}\left( \partial ^{+}\right) ^{-1} &=&\left( \partial ^{+}\right)
^{-1}L^{+}, \\
L^{+}\left( \partial ^{3}\right) ^{-1} &=&\left( \partial ^{3}\right)
^{-1}L^{+}+q^{-1}\partial ^{+}\left( \partial ^{3}\right) ^{-2}\tau ^{-\frac{%
1}{2}},  \nonumber \\
L^{+}\left( \partial ^{-}\right) ^{-1} &=&\left( \partial ^{-}\right)
^{-1}L^{+}+q^{-1}\partial ^{3}\left( \partial ^{-}\right) ^{-2}\tau ^{-\frac{%
1}{2}},  \nonumber \\[0.1in]
L^{-}\left( \partial ^{-}\right) ^{-1} &=&\left( \partial ^{-}\right)
^{-1}L^{-}, \\
L^{-}\left( \partial ^{3}\right) ^{-1} &=&\left( \partial ^{3}\right)
^{-1}L^{-}-q^{-3}\left( \partial ^{3}\right) ^{-2}\partial ^{-}\tau ^{-\frac{%
1}{2}},  \nonumber \\
L^{-}\left( \partial ^{+}\right) ^{-1} &=&\left( \partial ^{+}\right)
^{-1}L^{-}-q^{-4}\left( \partial ^{+}\right) ^{-2}\partial ^{3}\tau ^{-\frac{%
1}{2}},  \nonumber \\[0.1in]
\tau ^{-\frac{1}{2}}\left( \partial ^{+}\right) ^{-1} &=&q^{-2}\left(
\partial ^{+}\right) ^{-1}\tau ^{-\frac{1}{2}}, \\
\tau ^{-\frac{1}{2}}\left( \partial ^{-}\right) ^{-1} &=&q^{2}\left(
\partial ^{-}\right) ^{-1}\tau ^{-\frac{1}{2}},  \nonumber \\
\tau ^{-\frac{1}{2}}\left( \partial ^{3}\right) ^{-1} &=&\left( \partial
^{3}\right) ^{-1}\tau ^{-\frac{1}{2}},  \nonumber \\[0.1in]
\Lambda ^{-\frac{1}{2}}\left( \partial ^{A}\right) ^{-1} &=&q^{2}\left(
\partial ^{A}\right) ^{-1}\Lambda ^{\frac{1}{2}},\quad A=\pm ,3.
\end{eqnarray}
Applying the substitutions 
\begin{equation}
\partial ^{A}\rightarrow \hat{\partial}^{A},\quad \left( \partial
^{A}\right) ^{-1}\rightarrow \left( \hat{\partial}^{A}\right) ^{-1},\quad
A=\pm ,3
\end{equation}
and

\begin{equation}
\partial ^{A}\rightarrow P^{A},\quad \left( \partial ^{A}\right)
^{-1}\rightarrow \left( P^{A}\right) ^{-1},\quad A=\pm ,3
\end{equation}
to all expressions presented so far we get the corresponding relations of
the second differential calculus (generated by the conjugated partial
derivatives $\hat{\partial}^A$) and that of the algebra of hermitean
momentum generators $P^A$, respectively \cite{LWW97}.

In \cite{BW01} it was shown that according to 
\begin{equation}
\partial ^{A}F=\left( \left( \partial _{\left( i=0\right) }^{A}\right)
+\left( \partial _{\left( i>0\right) }^{A}\right) \right) F
\end{equation}
the representations of our partial derivatives can be divided up into a
classical part and corrections vanishing in the undeformed limit $%
q\rightarrow 1$. Thus, seeking a solution to the equation $\partial ^{A}F=f$
\ for given f it is reasonable to consider the following expression: 
\begin{eqnarray}
F &=&\left( \partial ^{A}\right) ^{-1}f=\frac{1}{\left( \partial _{\left(
i=0\right) }^{A}\right) +\left( \partial _{\left( i>0\right) }^{A}\right) }f
\label{IntegralE3} \\
&=&\frac{1}{\left( \partial _{\left( i=0\right) }^{A}\right) \left( 1+\left(
\partial _{\left( i=0\right) }^{A}\right) ^{-1}\left( \partial _{\left(
i>0\right) }^{A}\right) \right) }f  \nonumber \\
&=&\frac{1}{\left( 1+\left( \partial _{\left( i=0\right) }^{A}\right)
^{-1}\left( \partial _{\left( i>0\right) }^{A}\right) \right) }\cdot \frac{1%
}{\left( \partial _{\left( i=0\right) }^{A}\right) }f  \nonumber \\
&=&\sum_{k=0}^{\infty }\left( -1\right) ^{k}\left[ \left( \partial _{\left(
i=0\right) }^{A}\right) ^{-1}\left( \partial _{\left( i>0\right)
}^{A}\right) \right] ^{k}\left( \partial _{\left( i=0\right) }^{A}\right)
^{-1}f.  \nonumber
\end{eqnarray}
With this formula at hand the new elements $\left( \partial ^{A}\right)
^{-1} $, $\,A=\pm ,3,$ can be represented on the commutative algebra as
\begin{eqnarray}
\left( \partial ^{-}\right) _{L}^{-1}f &=&-q\left( D_{q^{4}}^{+}\right)
^{-1}f,  \label{repres1} \\
\left( \partial ^{3}\right) _{L}^{-1}f &=&\left( D_{q^{2}}^{3}\right)
^{-1}f\left( q^{-2}x^{+}\right) ,  \nonumber \\
\left( \partial ^{+}\right) _{L}^{-1}f &=&-q^{-1}\sum_{k=0}^{\infty }\left(
-\lambda \right) ^{k}q^{2k\left( k+1\right) }\left[ \left(
D_{q^{4}}^{-}\right) ^{-1}x^{+}\left( D_{q^{2}}^{3}\right) ^{2}\right] ^{k} 
\nonumber \\
&&\times \left( D_{q^{4}}^{-}\right) ^{-1}f\left( q^{-2\left( k+1\right)
}x^{3}\right)   \nonumber
\end{eqnarray}
where the symbols $D_{q^{a}}^{A}$ and $\left( D_{q^{a}}^{A}\right) ^{-1}$, $%
\,A=\pm ,3,$ denote Jackson derivatives and Jackson integrals, respectively.
Their explicit form can be found in appendix \ref{AppA}. It is worth noting
that these expressions have to be considered as definite integrals with
their integration limits determining the limits of the appearing Jackson
integrals. We have to emphasize that in general the identity
$\left(\partial^A \right)^{-1}\partial^Af=f$ does not hold any longer, as 
$\left(\partial^A\right)^{-1}F$ is only one possible solution to the
equation $\partial^AF=f$. However, this problem should not arise, if
we restrict attention to continuous functions which fulfill certain
boundary conditions. 

Analogously, in the case of the second differential calculus the inverse
elements $\left( \hat{\partial}^{A}\right) ^{-1}$, $\,A=\pm ,3,$ can now be
written as 
\begin{eqnarray}
\allowbreak \left( \hat{\partial}^{+}\right) _{L}^{-1}f &=&-q^{-1}\left(
D_{q^{4}}^{-}\right) ^{-1}f\left( q^{2}x^{3},q^{4}x^{-}\right) ,
\label{repres2} \\
\left( \hat{\partial}^{3}\right) _{L}^{-1}f &=&\sum_{k=0}^{\infty }\lambda
^{k}\left( q\lambda _{+}\right) ^{k}\left[ \left( D_{q^{2}}^{3}\right)
^{-1}x^{3}D_{q^{4}}^{+}D_{q^{4}}^{-}\right] ^{k}  \nonumber \\
&&\times \left( D_{q^{2}}^{3}\right) ^{-1}f\left(
q^{2}x^{+},q^{2}x^{3},q^{4}x^{-}\right) ,  \nonumber \\
\left( \hat{\partial}^{-}\right) _{L}^{-1}f &=&-\sum_{k=0}^{\infty }\lambda
^{k}q^{k}\left[ H^{-1}\left( \left( D_{q^{4}}^{+}\right) ^{-1}x^{-}\left(
D_{q^{2}}^{3}\right) ^{2}\right) -q^{-2}\lambda \lambda _{+}\left(
x^{3}\right) D_{q^{4}}^{+}D_{q^{4}}^{-k}\right] ^{k}  \nonumber \\
&&\times H^{-1}\left( D_{q^{4}}^{+}\right) ^{-1}f\left(
q^{4}x^{+},q^{-2k}x^{3},q^{-4k}x^{-}\right)  \nonumber
\end{eqnarray}
\newline
where $\lambda _{+}=q+q^{-1}$ and $H^{-1}$ denotes the inverse of the
operator $H$ introduced in \cite{BW01}. Their explicit form can be found in
appendix \ref{AppB}.

Finally, formula 
(\ref{IntegralE3}) can also be applied to the algebra of 
momentum generators $P^A$, yielding 
\begin{eqnarray}
-%
\frac{i}{\left[ 2\right] _{q^{3}}}\left( P^{+}\right) _{L}^{-1}f
&=&-q^{-2}\sum_{k=0}^{\infty }\left( -q^{2}\lambda \right) ^{k}\left[ \left(
H^{+}\right) ^{-1}\left( D_{q^{4}}^{-}\right) ^{-1}x^{+}\left(
D_{q^{2}}^{3}\right) ^{2}\right] ^{k}  \label{Impuls} \\
&&\times \left( H^{+}\right) ^{-1}\left( D_{q^{4}}^{-}\right) ^{-1}f, 
\nonumber \\
-\frac{i}{\left[ 2\right] _{q^{3}}}\left( P^{3}\right) _{L}^{-1}f
&=&-q^{-1}\sum_{k=0}^{\infty }\left( \lambda \lambda _{+}\right)
^{k}q^{-k\left( k+4\right) }\left[ \left( H^{3}\right) ^{-1}\left(
D_{q^{2}}^{3}\right) ^{-1}x^{3}D_{q^{4}}^{+}D_{q^{-4}}^{-}\right] ^{k} 
\nonumber \\
&&\times \left( H^{3}\right) ^{-1}\left( D_{q^{2}}^{3}\right) ^{-1}f\left(
q^{-2k}x^{+},q^{-2k}x^{3}\right) ,  \nonumber \\
-\frac{i}{\left[ 2\right] _{q^{3}}}\left( P^{-}\right) _{L}^{-1}f
&=&-q^{4}\sum_{k=0}^{\infty }\left( q^{2}\lambda \right) ^{k}  \nonumber \\
&&\times \left[ \left( H^{-}\right) ^{-1}\left( \left( D_{q^{4}}^{+}\right)
^{-1}x^{-}\left( D_{q^{2}}^{3}\right) ^{2}-q^{-2}\lambda \lambda _{+}\left(
x^{3}\right) ^{2}D_{q^{4}}^{+}D_{q^{4}}^{-}\right) \right] ^{k}  \nonumber \\
&&\times \left( H^{-}\right) ^{-1}\left( D_{q^{4}}^{+}\right) ^{-1}f\left(
q^{4}x^{+},q^{-2k}x^{3},q^{-4k}x^{-}\right) .  \nonumber
\end{eqnarray}
\linebreak Again, the operators $\left( H^{A}\right) ^{-1}$, $A=\pm ,3,$
denote the inverse of the operators $H^{A}$, $\,A=\pm ,3,$ introduced in 
\cite{BW01}. (Their explicit form is written out in appendix 
\ref{AppB}.)

As indicated by an index $L$ all of the above integral operators have been
derived from left derivatives. Depending on the fact whether the expression
for the integral under consideration refers to left or right derivatives we
will call it from now on a left or right integral and denote this by an
additional index $L$ or $R$. From the relation between left and right
derivatives presented in \cite{BW01} we can at once deduce simple rules for
transforming left and right integrals into each other, namely 
\begin{eqnarray}
\left( \partial ^{+}\right) _{R}^{-1}f&\underleftrightarrow&-q^{6}\left( \hat{\partial}^{-}\right)
_{L}^{-1}f, \\
\left( \partial ^{3}\right) _{R}^{-1}f&\underleftrightarrow&-q^{6}\left( \hat{\partial}^{3}\right)
_{L}^{-1}f,  \nonumber \\
\left( \partial ^{-}\right) _{R}^{-1}f&\underleftrightarrow&-q^{6}\left( \hat{\partial}^{+}\right)
_{L}^{-1}f.  \nonumber
\end{eqnarray}
And in the same manner we also get 
\begin{eqnarray}
\left( \hat{\partial}^{+}\right) _{R}^{-1}f&\underleftrightarrow&-q^{-6}\left( \partial ^{-}\right)
_{L}^{-1}f, \\
\left( \hat{\partial}^{3}\right) _{R}^{-1}f&\underleftrightarrow&-q^{-6}\left( \partial ^{3}\right)
_{L}^{-1}f,  \nonumber \\
\left( \hat{\partial}^{-}\right) _{R}^{-1}f&\underleftrightarrow&-q^{-6}\left( \partial ^{+}\right)
_{L}^{-1}f.  \nonumber
\end{eqnarray}
In the case of the algebra of momentum generators we can finally write 
\begin{eqnarray}
\left( P^{+}\right) _{R}^{-1}f&\underleftrightarrow&-\left( P^{-}\right) _{L}^{-1}f, \\
\left( P^{3}\right) _{R}^{-1}f&\underleftrightarrow&-\left( P^{3}\right) _{L}^{-1}f,  \nonumber \\
\left( P^{-}\right) _{R}^{-1}f&\underleftrightarrow&-\left( P^{+}\right) _{L}^{-1}f \nonumber
\end{eqnarray}
where the symbol $\underleftrightarrow$
denotes that one can make a transition between the two expressions by
applying the substitutions \footnote{%
Let us note that the operators $H^{-1},$ $\left( H^{A}\right) ^{-1},$ $A=\pm
,3,$ have to be affected by this substitution, too.} 
\begin{equation}
x^{+}\longleftrightarrow x^{-},\qquad \left( D_{q^{a}}^{+}\right) ^{\pm
1}\longleftrightarrow \left( D_{q^{a}}^{-}\right) ^{\pm 1}.
\end{equation}

For both types of integrals we can write down formulae for integration by
parts. In the case of left integrals they read 
\begin{eqnarray}
\left( \partial ^{-}\right) _{L}^{-1}\left( \partial _{L}^{-}f\right) \star
g\left| _{x^{+}=a}^{b}\right. &=&f\star g\left| _{x^{+}=a}^{b}\right.
-\left( \partial ^{-}\right) _{L}^{-1}\left( \Lambda ^{1/2}\tau
^{-1/2}f\right) \star \partial _{L}^{-}g\left| _{x^{+}=a}^{b},\right. \\
\left( \partial ^{3}\right) _{L}^{-1}\left( \partial _{L}^{3}f\right) \star
g\left| _{x^{3}=a}^{b}\right. &=&f\star g\left| _{x^{3}=a}^{b}\right.
-\left( \partial ^{3}\right) _{L}^{-1}\left( \Lambda ^{1/2}f\right) \star
\partial _{L}^{3}g\left| _{x^{3}=a}^{b}\right.  \nonumber \\
&&-\,\lambda \lambda _{+}\left( \partial ^{3}\right) _{L}^{-1}\left( \Lambda
^{1/2}L^{+}f\right) \star \partial _{L}^{-}g\left| _{x^{3}=a}^{b}\right. , 
\nonumber \\
\left( \partial ^{+}\right) _{L}^{-1}\left( \partial _{L}^{+}f\right) \star
g\left| _{x^{-}=a}^{b}\right. &=&f\star g\left| _{x^{-}=a}^{b}\right.
-\left( \partial ^{+}\right) _{L}^{-1}\left( \Lambda ^{1/2}\tau
^{1/2}f\right) \star \partial _{L}^{+}g\left| _{x^{-}=a}^{b}\right. 
\nonumber \\
&&-\,q\lambda \lambda _{+}\left( \partial ^{+}\right) _{L}^{-1}\left(
\Lambda ^{1/2}\tau ^{1/2}L^{+}f\right) \star \partial _{L}^{3}g\left|
_{x^{-}=a}^{b}\right.  \nonumber \\
&&-\,q^{2}\lambda ^{2}\lambda _{+}\left( \partial ^{+}\right)
_{L}^{-1}\left( \Lambda ^{1/2}\tau ^{1/2}\left( L^{+}\right) ^{2}f\right)
\star \partial _{L}^{-}g\left| _{x^{-}=a}^{b}\right.  \nonumber
\end{eqnarray}
and for the second covariant differential calculus we get accordingly 
\begin{eqnarray}
\left( \hat{\partial}^{+}\right) _{L}^{-1}\left( \hat{\partial}%
_{L}^{+}f\right) \star g\left| _{x^{-}=a}^{b}\right. &=&f\star g\left|
_{x^{-}=a}^{b}\right. -\left( \hat{\partial}^{+}\right) _{L}^{-1}\left(
\Lambda ^{-1/2}\tau ^{-1/2}f\right) \star \hat{\partial}_{L}^{+}g\left|
_{x^{-}=a}^{b},\right. \\
\left( \hat{\partial}^{3}\right) _{L}^{-1}\left( \hat{\partial}%
_{L}^{3}f\right) \star g\left| _{x^{3}=a}^{b}\right. &=&f\star g\left|
_{x^{3}=a}^{b}\right. -\left( \hat{\partial}^{3}\right) _{L}^{-1}\left(
\Lambda ^{-1/2}f\right) \star \hat{\partial}_{L}^{3}g\left|
_{x^{3}=a}^{b}\right.  \nonumber \\
&&-\,\lambda \lambda _{+}\left( \hat{\partial}^{3}\right) _{L}^{-1}\left(
\Lambda ^{-1/2}L^{-}f\right) \star \hat{\partial}_{L}^{+}g\left|
_{x^{3}=a}^{b}\right. ,  \nonumber \\
\left( \hat{\partial}^{-}\right) _{L}^{-1}\left( \hat{\partial}%
_{L}^{-}f\right) \star g\left| _{x^{+}=a}^{b}\right. &=&f\star g\left|
_{x^{+}=a}^{b}\right. -\left( \hat{\partial}^{-}\right) _{L}^{-1}\left(
\Lambda ^{-1/2}\tau ^{1/2}f\right) \star \hat{\partial}_{L}^{-}g\left|
_{x^{+}=a}^{b}\right.  \nonumber \\
&&-q^{-1}\lambda \lambda _{+}\left( \hat{\partial}^{-}\right)
_{L}^{-1}\left( \Lambda ^{-1/2}\tau ^{1/2}L^{-}f\right) \star \hat{\partial}%
_{L}^{3}g\left| _{x^{+}=a}^{b}\right.  \nonumber \\
&&-q^{-2}\lambda ^{2}\lambda _{+}\left( \hat{\partial}^{-}\right)
_{L}^{-1}\left( \Lambda ^{-1/2}\tau ^{1/2}\left( L^{-}\right) ^{2}f\right)
\star \hat{\partial}_{L}^{+}g\left| _{x^{+}=a}^{b}\right. .  \nonumber
\end{eqnarray}
In the case of right integrals the rules for integration by parts take the
form 
\begin{eqnarray}
\left( \partial ^{-}\right) _{R}^{-1}f\star \partial _{R}^{-}g\left|
_{x^{+}=a}^{b}\right. &=&f\star g\left| _{x^{+}=a}^{b}\right. -\left(
\partial ^{-}\right) _{R}^{-1}\left( \partial _{R}^{-}f\right) \star \left(
\Lambda ^{-1/2}\tau ^{1/2}g\right) \left| _{x^{+}=a}^{b},\right. \\
\left( \partial ^{3}\right) _{R}^{-1}f\star \partial _{R}^{3}g\left|
_{x^{3}=a}^{b}\right. &=&f\star g\left| _{x^{3}=a}^{b}\right. -\left(
\partial ^{3}\right) _{R}^{-1}\left( \partial _{R}^{3}f\right) \star \left(
\Lambda ^{-1/2}g\right) \left| _{x^{3}=a}^{b}\right.  \nonumber \\
&&+\,\lambda \lambda _{+}\left( \partial ^{3}\right) _{R}^{-1}\left(
\partial _{R}^{-}f\right) \star \left( \Lambda ^{-1/2}\tau
^{1/2}L^{+}g\right) \left| _{x^{3}=a}^{b},\right.  \nonumber \\
\left( \partial ^{+}\right) _{R}^{-1}f\star \partial _{R}^{+}g\left|
_{x^{-}=a}^{b}\right. &=&f\star g\left| _{x^{-}=a}^{b}\right. -\left(
\partial ^{+}\right) _{R}^{-1}\left( \partial _{R}^{+}f\right) \star \left(
\Lambda ^{-1/2}\tau ^{-1/2}g\right) \left| _{x^{-}=a}^{b}\right.  \nonumber
\\
&&+\,q^{-1}\lambda \lambda _{+}\left( \partial ^{+}\right) _{R}^{-1}\left(
\partial _{R}^{3}f\right) \star \left( \Lambda ^{-1/2}L^{+}g\right) \left|
_{x^{-}=a}^{b}\right.  \nonumber \\
&&-\,\lambda ^{2}\lambda _{+}\left( \partial ^{+}\right) _{R}^{-1}\left(
\partial _{R}^{-}f\right) \star \left( \Lambda ^{-1/2}\tau ^{1/2}\left(
L^{+}\right) ^{2}g\right) \left| _{x^{-}=a}^{b}\right.  \nonumber
\end{eqnarray}
and for the second differential calculus we have again the formulae 
\begin{eqnarray}
\left( \hat{\partial}^{+}\right) _{R}^{-1}f\star \hat{\partial}%
_{R}^{+}g\left| _{x^{-}=a}^{b}\right. &=&f\star g\left|
_{x^{-}=a}^{b}\right. -\left( \hat{\partial}^{+}\right) _{R}^{-1}\left( \hat{%
\partial}_{R}^{+}f\right) \star \left( \Lambda ^{\ 1/2}\tau ^{1/2}g\right)
\left| _{x^{-}=a}^{b},\right. \\
\left( \hat{\partial}^{3}\right) _{R}^{-1}f\star \hat{\partial}%
_{R}^{3}g\left| _{x^{3}=a}^{b}\right. &=&f\star g\left|
_{x^{3}=a}^{b}\right. -\left( \hat{\partial}^{3}\right) _{R}^{-1}\left( \hat{%
\partial}_{R}^{3}f\right) \star \left( \Lambda ^{1/2}g\right) \left|
_{x^{3}=a}^{b}\right.  \nonumber \\
&&+\,\lambda \lambda _{+}\left( \hat{\partial}^{3}\right) _{R}^{-1}\left( 
\hat{\partial}_{R}^{+}f\right) \star \left( \Lambda ^{1/2}\tau
^{1/2}L^{-}g\right) \left| _{x^{3}=a}^{b},\right.  \nonumber \\
\left( \hat{\partial}^{-}\right) _{R}^{-1}f\star \hat{\partial}%
_{R}^{-}g\left| _{x^{+}=a}^{b}\right. &=&f\star g\left|
_{x^{+}=a}^{b}\right. -\left( \hat{\partial}^{-}\right) _{R}^{-1}\left( \hat{%
\partial}_{R}^{-}f\right) \star \left( \Lambda ^{1/2}\tau ^{-1/2}g\right)
\left| _{x^{+}=a}^{b}\right.  \nonumber \\
&&+\,q\lambda \lambda _{+}\left( \hat{\partial}^{-}\right) _{R}^{-1}\left( 
\hat{\partial}_{R}^{3}f\right) \star \left( \Lambda ^{1/2}L^{-}g\right)
\left| _{x^{+}=a}^{b}\right.  \nonumber \\
&&-\,\lambda ^{2}\lambda _{+}\left( \hat{\partial}^{-}\right)
_{R}^{-1}\left( \hat{\partial}_{R}^{+}f\right) \star \left( \Lambda
^{1/2}\tau ^{1/2}\left( L^{-}\right) ^{2}g\right) \left|
_{x^{+}=a}^{b}\right. .  \nonumber
\end{eqnarray}

Let us now make contact with expressions of the form $\left( \partial
^{+}\right) ^{-1}\left( \partial ^{3}\right) ^{-1}\left( \partial
^{-}\right) ^{-1}$ which can be interpreted as some sort of 3-dimensional
volume integrals. As the operators $\left( \partial ^{A}\right) ^{-1}$, $%
A=\pm ,3,$ do not mutually commute, it should be clear that in general the
result of our volume integration depends on the order in which the operators 
$\left( \partial ^{A}\right) ^{-1}$, $A=\pm ,3$ are arranged. However, if we
consequently drop surface terms, the considered volume integrals will only
change their normalisation, as the following calculation illustrates: 
\begin{eqnarray}
\left( \partial ^{-}\right) ^{-1}\left( \partial ^{3}\right) ^{-1}\left(
\partial ^{+}\right) ^{-1}f &=&q^{2}\left( \partial ^{-}\right) ^{-1}\left(
\partial ^{+}\right) ^{-1}\left( \partial ^{3}\right) ^{-1}f \\
&&\hspace*{-2.5cm}=\sum_{i=0}^{\infty }q^{2\left( i+2\right) }\lambda
^{i}\left[ \left[ i\right] \right] _{q^{4}}!\left(\left[
    \begin{array}{c} -1\\i\end{array} \right]_{q^{4}}\right)
^{2}\left( \partial ^{+}\right) ^{-\left( i+1\right) }\left( \partial
^{3}\right) ^{2i-1}\left( \partial ^{-}\right) ^{-\left( i+1\right) }f 
\nonumber \\
&&\hspace*{-2.5cm}=q^{4}\left( \partial ^{+}\right) ^{-1}\left( \partial
^{3}\right) ^{-1}\left( \partial ^{-}\right) ^{-1}f+S.T.  \nonumber
\end{eqnarray}
where $S.T.$ stands for neglected surface terms. For the first and
second equality we have used here the identities of
(\ref{inv}). Finally, the list
containing all possible relations between the different volume integrals is
given by 
\begin{eqnarray}
\left( \partial ^{+}\right) ^{-1}\left( \partial ^{3}\right) ^{-1}\left(
\partial ^{-}\right) ^{-1}f &=&q^{-4}\left( \partial ^{-}\right) ^{-1}\left(
\partial ^{3}\right) ^{-1}\left( \partial ^{+}\right) ^{-1}f=
\label{volume3} \\
q^{-2}\left( \partial ^{3}\right) ^{-1}\left( \partial ^{-}\right)
^{-1}\left( \partial ^{+}\right) ^{-1}f &=&q^{-2}\left( \partial ^{3}\right)
^{-1}\left( \partial ^{+}\right) ^{-1}\left( \partial ^{-}\right) ^{-1}f= 
\nonumber \\
q^{-2}\left( \partial ^{-}\right) ^{-1}\left( \partial ^{+}\right)
^{-1}\left( \partial ^{3}\right) ^{-1}f &=&q^{-2}\left( \partial ^{+}\right)
^{-1}\left( \partial ^{-}\right) ^{-1}\left( \partial ^{3}\right) ^{-1}f. 
\nonumber
\end{eqnarray}
Let us note that the same identities also hold for the operators $\left( 
\hat{\partial}^{A}\right) ^{-1}$ and $\left( P^{A}\right) ^{-1}$, $A=\pm ,3$.

Our next goal is to provide formulae which enable an explicit
calculation of the new volume integrals. To this end we have to insert the
above representations (\ref{repres1}), (\ref{repres2}) and (\ref{Impuls})
for $\left( \partial ^{A}\right) ^{-1}$, $\left( \hat{\partial}^{A}\right)
^{-1}$ and $\left( P^{A}\right) ^{-1}$, respectively, into 
expressions (\ref{volume3}) defining our volume integrals. A detailed
analysis of the resulting formulae shows that all contributions
depending on $\lambda ^{k}$, $k\geq 1$, lead to surface terms and can, in
turn, be neglected. Due to this fact we end up with the identities 
\begin{eqnarray}
\lefteqn{\left( \partial ^{+}\right) _{L}^{-1}\left( \partial ^{3}\right)
_{L}^{-1}\left( \partial ^{-}\right) _{L}^{-1}f =} \\
&&\hspace*{0 cm}=q^{-4}\left( D_{q^{4}}^{-}\right) ^{-1}\left(
D_{q^{2}}^{3}\right) ^{-1}\left( D_{q^{4}}^{+}\right) ^{-1}f\left(
q^{-2}x^{+},q^{-2}x^{3}\right) ,  \nonumber \\[0.1in]
\lefteqn{ \left( \hat{\partial}^{-}\right) _{L}^{-1}\left( \hat{\partial}^{3}\right)
_{L}^{-1}\left( \hat{\partial}^{+}\right) _{L}^{-1}f =}  \nonumber \\
&&\hspace*{0 cm}=-q^{3}H^{-1}\left( D_{q^{-4}}^{+}\right) ^{-1}\left(
D_{q^{-2}}^{3}\right) ^{-1}\left( D_{q^{-4}}^{-}\right) ^{-1}f\left(
q^{2}x^{+},q^{2}x^{3},q^{4}x^{-}\right) ,  \nonumber \\[0.1in]
\lefteqn{\left( -\frac{i}{\left[ 2\right] _{q^{3}}}\right) ^{3}\left( P^{-}\right)
_{L}^{-1}\left( P^{3}\right) _{L}^{-1}\left( P^{+}\right) _{L}^{-1}f =}& 
\nonumber \\
&&\hspace*{0 cm}=q\left( H^{-}\right) ^{-1}\left( D_{q^{4}}^{+}\right)
^{-1}\left( H^{3}\right) ^{-1}\left( D_{q^{2}}^{3}\right) ^{-1}\left(
H^{+}\right) ^{-1}\left( D_{q^{4}}^{-}\right) ^{-1}f\left( q^{4}x^{+}\right)
.  \nonumber
\end{eqnarray}
As our volume integrals are built up from operators which are inverse to
partial derivatives or momentum generators their translation invariance is
obvious if surface terms are dropped. Likewise, we can easily prove
rotation invariance by applying the adjoint action of the symmetry
generators $L^{+}$, $L^{-}$ and $\tau ^{-1/2}$ to our volume integrals \cite
{KS97}, \cite{Maj95}. For example, we have 
\begin{eqnarray}
L^{+}\triangleright \left( \left( \partial ^{+}\right) ^{-1}\left( \partial
^{3}\right) ^{-1}\left( \partial ^{-}\right) ^{-1}\right) &=&\left( \partial
^{+}\right) ^{-1}L^{+}\triangleright \left( \left( \partial ^{3}\right)
^{-1}\left( \partial ^{-}\right) ^{-1}\right) \\
&&\hspace*{-4.5cm}=\left( \partial ^{+}\right) ^{-1}\left( \partial
^{3}\right) ^{-1}L^{+}\triangleright \left( \partial ^{-}\right)
^{-1}+q^{-1}\left( \partial ^{+}\right) ^{-1}\partial ^{+}\left( \partial
^{3}\right) ^{-2}\tau ^{-1/2}\triangleright \left( \partial ^{-}\right) ^{-1}
\nonumber \\
&&\hspace*{-4.5cm}=\left( \partial ^{+}\right) ^{-1}\left( \partial
^{3}\right) ^{-1}\left( \partial ^{-}\right) ^{-1}L^{+}\triangleright
1+q\left( \partial ^{3}\right) ^{-2}\left( \partial ^{-}\right) ^{-1} 
\nonumber \\
&&\hspace*{-4.5cm}=0+S.T.  \nonumber
\end{eqnarray}
Repeating this calculation for $L^{-}$ and $\tau ^{-1/2}$ yields the same
result and therefore shows rotation invariance of the volume integrals (\ref
{volume3}), if again surface terms are dropped.

\section{q-deformed Euclidean Space in four Dimensions}

The 4-dimensional Euclidean space can be treated in very much the same way
as the 3-dimensional one. Again, we start with the commutation relations 
\cite{Oca96} 
\begin{eqnarray}
\partial ^{1}\partial ^{2}=q\partial ^{2}\partial ^{1}, &&\qquad \partial
^{1}\partial ^{3}=q\partial ^{3}\partial ^{1}, \\
\partial ^{2}\partial ^{4}=q\partial ^{4}\partial ^{2}, &&\qquad \partial
^{3}\partial ^{4}=q\partial ^{4}\partial ^{3},  \nonumber \\
\partial ^{2}\partial ^{3}=\partial ^{3}\partial ^{2}, &&\qquad \partial
^{4}\partial ^{1}=\partial ^{1}\partial ^{4}+\lambda \partial ^{2}\partial
^{3}\qquad  \nonumber
\end{eqnarray}
where $\lambda =q-q^{-1}$ with $q>1$, and introduce inverse elements $\left(
\partial ^{i}\right) ^{-1}$, $i=1,\ldots,4$, by 
\begin{equation}
\partial ^{i}\left( \partial ^{i}\right) ^{-1} = \left( \partial
^{i}\right) ^{-1}\partial ^{i}=1.
\end{equation}
Now, the remaining commutation relations become 
\begin{eqnarray}
\left( \partial ^{2}\right) ^{-1}\partial ^{1} &=&q\partial ^{1}\left(
\partial ^{2}\right) ^{-1},\quad \partial ^{2}\left( \partial ^{1}\right)
^{-1}=q\left( \partial ^{1}\right) ^{-1}\partial ^{2}, \\
\left( \partial ^{3}\right) ^{-1}\partial ^{1} &=&q\partial ^{1}\left(
\partial ^{3}\right) ^{-1},\quad \partial ^{3}\left( \partial ^{1}\right)
^{-1}=q\left( \partial ^{1}\right) ^{-1}\partial ^{3},  \nonumber \\
\left( \partial ^{4}\right) ^{-1}\partial ^{2} &=&q\partial ^{2}\left(
\partial ^{4}\right) ^{-1},\quad \partial ^{4}\left( \partial ^{2}\right)
^{-1}=q\left( \partial ^{2}\right) ^{-1}\partial ^{4},  \nonumber \\
\left( \partial ^{4}\right) ^{-1}\partial ^{3} &=&q\partial ^{3}\left(
\partial ^{4}\right) ^{-1},\quad \partial ^{4}\left( \partial ^{3}\right)
^{-1}=q\left( \partial ^{3}\right) ^{-1}\partial ^{4},  \nonumber \\
\left( \partial ^{3}\right) ^{-1}\partial ^{2} &=&\partial ^{2}\left(
\partial ^{3}\right) ^{-1},\quad \partial ^{3}\left( \partial ^{2}\right)
^{-1}=\left( \partial ^{2}\right) ^{-1}\partial ^{3},  \nonumber \\
\left( \partial ^{4}\right) ^{-1}\partial ^{1} &=&\partial ^{1}\left(
\partial ^{4}\right) ^{-1}-q^{2}\lambda \partial ^{2}\partial ^{3}\left(
\partial ^{4}\right) ^{-2},  \nonumber \\
\quad \partial ^{4}\left( \partial ^{1}\right) ^{-1} &=&\left( \partial
^{1}\right) ^{-1}\partial ^{4}-q^{2}\lambda \left( \partial ^{1}\right)
^{-2}\partial ^{2}\partial ^{3}.  \nonumber
\end{eqnarray}
In addition, we obtain further identities of the form 
\begin{eqnarray}
\left( \partial ^{2}\right) ^{-1}\left( \partial ^{1}\right) ^{-1}
&=&q^{-1}\left( \partial ^{1}\right) ^{-1}\left( \partial ^{2}\right)
^{-1},\quad \left( \partial ^{3}\right) ^{-1}\left( \partial ^{1}\right)
^{-1}=q^{-1}\left( \partial ^{1}\right) ^{-1}\left( \partial ^{3}\right)
^{-1}, \\
\left( \partial ^{4}\right) ^{-1}\left( \partial ^{2}\right) ^{-1}
&=&q^{-1}\left( \partial ^{2}\right) ^{-1}\left( \partial ^{4}\right)
^{-1},\quad \left( \partial ^{4}\right) ^{-1}\left( \partial ^{3}\right)
^{-1}=q^{-1}\left( \partial ^{3}\right) ^{-1}\left( \partial ^{4}\right)
^{-1},  \nonumber \\
\left( \partial ^{3}\right) ^{-1}\left( \partial ^{2}\right) ^{-1} &=&\left(
\partial ^{2}\right) ^{-1}\left( \partial ^{3}\right) ^{-1},  \nonumber \\
\left( \partial ^{4}\right) ^{-1}\left( \partial ^{1}\right) ^{-1}
&=&\sum_{i=0}^{\infty }\lambda ^{i}\left[ \left[ i\right] \right]
_{q^{-2}}!\left(\left[\begin{array}{c} -1\\i\end{array}\right]_{q^{-2}}\right) ^{2}\left( \partial
^{1}\right) ^{-\left( i+1\right) }\left( \partial ^{2}\right) ^{i}\left(
\partial ^{3}\right) ^{i}\left( \partial ^{4}\right) ^{-\left( i+1\right) }.
\nonumber
\end{eqnarray}
Next, the commutation relations with the symmetry generators read 
\begin{eqnarray}
L_{1}^{+}\left( \partial ^{1}\right) ^{-1} &=&q^{-1}\left( \partial
^{1}\right) ^{-1}L_{1}^{+}+q^{-1}\left( \partial ^{1}\right) ^{-2}\partial
^{2}, \\
L_{1}^{+}\left( \partial ^{2}\right) ^{-1} &=&q\left( \partial ^{2}\right)
^{-1}L_{1}^{+},  \nonumber \\
L_{1}^{+}\left( \partial ^{3}\right) ^{-1} &=&q^{-1}\left( \partial
^{3}\right) ^{-1}L_{1}^{+}-q^{-1}\left( \partial ^{3}\right) ^{-2}\partial
^{4},  \nonumber \\
L_{1}^{+}\left( \partial ^{4}\right) ^{-1} &=&q\left( \partial ^{4}\right)
^{-1}L_{1}^{+},  \nonumber \\[0.16in]
L_{2}^{+}\left( \partial ^{1}\right) ^{-1} &=&q^{-1}\left( \partial
^{1}\right) ^{-1}L_{2}^{+}+q^{-1}\left( \partial ^{1}\right) ^{-2}\partial
^{3}, \\
L_{2}^{+}\left( \partial ^{2}\right) ^{-1} &=&q^{-1}\left( \partial
^{2}\right) ^{-1}L_{2}^{+}-q^{-1}\left( \partial ^{2}\right) ^{-2}\partial
^{4},  \nonumber \\
L_{2}^{+}\left( \partial ^{3}\right) ^{-1} &=&q\left( \partial ^{3}\right)
^{-1}L_{2}^{+},  \nonumber \\
L_{2}^{+}\left( \partial ^{4}\right) ^{-1} &=&q\left( \partial ^{4}\right)
^{-1}L_{2}^{+},  \nonumber \\[0.4cm]
L_{1}^{-}\left( \partial ^{1}\right) ^{-1} &=&q^{-1}\left( \partial
^{1}\right) ^{-1}L_{1}^{-}, \\
L_{1}^{-}\left( \partial ^{2}\right) ^{-1} &=&q\left( \partial ^{2}\right)
^{-1}L_{1}^{-}+q^{3}\partial ^{1}\left( \partial ^{2}\right) ^{-2}, 
\nonumber \\
L_{1}^{-}\left( \partial ^{3}\right) ^{-1} &=&q^{-1}\left( \partial
^{3}\right) ^{-1}L_{1}^{-},  \nonumber \\
L_{1}^{-}\left( \partial ^{4}\right) ^{-1} &=&q\left( \partial ^{4}\right)
^{-1}L_{1}^{-}-q^{3}\partial ^{3}\left( \partial ^{4}\right) ^{-2}, 
\nonumber \\[0.4cm]
L_{2}^{-}\left( \partial ^{1}\right) ^{-1} &=&q^{-1}\left( \partial
^{1}\right) ^{-1}L_{2}^{-}, \\
L_{2}^{-}\left( \partial ^{2}\right) ^{-1} &=&q^{-1}\left( \partial
^{2}\right) ^{-1}L_{2}^{-},  \nonumber \\
L_{2}^{-}\left( \partial ^{3}\right) ^{-1} &=&q\left( \partial ^{3}\right)
^{-1}L_{2}^{-}+q^{3}\partial ^{1}\left( \partial ^{3}\right) ^{-2}, 
\nonumber \\
L_{2}^{-}\left( \partial ^{4}\right) ^{-1} &=&q\left( \partial ^{4}\right)
^{-1}L_{2}^{-}-q^{3}\partial ^{2}\left( \partial ^{4}\right) ^{-2}, 
\nonumber \\[0.4cm]
K_{1}\left( \partial ^{1}\right) ^{-1} &=&q\left( \partial ^{1}\right)
^{-1}K_{1}, \\
K_{1}\left( \partial ^{2}\right) ^{-1} &=&q^{-1}\left( \partial ^{2}\right)
^{-1}K_{1},  \nonumber \\
K_{1}\left( \partial ^{3}\right) ^{-1} &=&q\left( \partial ^{3}\right)
^{-1}K_{1},  \nonumber \\
K_{1}\left( \partial ^{4}\right) ^{-1} &=&q^{-1}\left( \partial ^{4}\right)
^{-1}K_{1},  \nonumber \\[0.4cm]
K_{2}\left( \partial ^{1}\right) ^{-1} &=&q\left( \partial ^{1}\right)
^{-1}K_{2}, \\
K_{2}\left( \partial ^{2}\right) ^{-1} &=&q\left( \partial ^{2}\right)
^{-1}K_{2},  \nonumber \\
K_{2}\left( \partial ^{3}\right) ^{-1} &=&q^{-1}\left( \partial ^{3}\right)
^{-1}K_{2},  \nonumber \\
K_{2}\left( \partial ^{4}\right) ^{-1} &=&q^{-1}\left( \partial ^{4}\right)
^{-1}K_{2},  \nonumber \\[0.4cm]
\Lambda \left( \partial ^{i}\right) ^{-1} &=&q^{2}\left( \partial
^{i}\right) ^{-1}\Lambda ,\qquad i=1,\ldots ,4.
\end{eqnarray}
With the substitutions 
\begin{equation}
\partial ^{i}\rightarrow \hat{\partial}^{i},\quad \left( \partial
^{i}\right) ^{-1}\rightarrow \left( \hat{\partial}^{i}\right) ^{-1},\quad
i=1,\ldots ,4
\end{equation}
and 
\begin{equation}
\partial ^{i}\rightarrow P^{i},\quad \left( \partial ^{i}\right)
^{-1}\rightarrow \left( P^{i}\right) ^{-1},\quad i=1,\ldots ,4
\end{equation}
we get the corresponding relations for the second differential calculus and
the algebra of momentum generators, respectively.

Now, the same reasoning we have already applied to the 3-dimensional Euclidean
space leads immediately to the representations 
\begin{eqnarray}
\left( \partial ^{1}\right) _{L}^{-1}f &=&q\left( D_{q^{2}}^{4}\right)
^{-1}f\left( q^{-1}x^{2},q^{-1}x^{3}\right) , \\
\left( \partial ^{2}\right) _{L}^{-1}f &=&\left( D_{q^{2}}^{3}\right)
^{-1}f\left( q^{-1}x^{1},q^{-2}x^{4}\right)  \nonumber \\
&&+\,q^{-1}\sum_{k=1}^{\infty }\lambda ^{k}q^{k^{2}}\left[ \left(
D_{q^{2}}^{3}\right) ^{-1}x^{2}D_{q^{-2}}^{1}D_{q^{-2}}^{4}\right] ^{k} 
\nonumber \\
&&\times \left( D_{q^{2}}^{3}\right) ^{-1}f\left(
q^{-1}x^{1},q^{k}x^{2},q^{k}x^{3},q^{-2}x^{4}\right) ,  \nonumber \\
\left( \partial ^{3}\right) _{L}^{-1}f &=&\left( D_{q^{2}}^{2}\right)
^{-1}f\left( q^{-1}x^{1},q^{-2}x^{4}\right)  \nonumber \\
&&+\,q^{-1}\sum_{k=1}^{\infty }\lambda ^{k}q^{k^{2}}\left[ \left(
D_{q^{2}}^{2}\right) ^{-1}x^{3}D_{q^{-2}}^{1}D_{q^{-2}}^{4}\right] ^{k} 
\nonumber \\
&&\times \left( D_{q^{2}}^{2}\right) ^{-1}f\left(
q^{-1}x^{1},q^{k}x^{2},q^{k}x^{3},q^{-2}x^{4}\right) ,  \nonumber \\
\left( \partial ^{4}\right) _{L}^{-1}f &=&q^{-1}\sum_{k=0}^{\infty }\lambda
^{k}q^{k}  \nonumber \\
&&\times \left\{ N^{-1}\left( D_{q^{2}}^{1}\right) ^{-1}\left[
q^{-1}x^{4}D_{q^{-2}}^{2}D_{q^{-2}}^{3}+\lambda x^{2}x^{3}\left(
D_{q^{-2}}^{1}\right) ^{2}D_{q^{-2}}^{4}\right] \right\} ^{k}  \nonumber \\
&&\times N^{-1}\left( D_{q^{2}}^{1}\right) ^{-1}f\left(
q^{2k}x^{1},q^{k}x^{2},q^{k}x^{3},q^{-2}x^{4}\right) .  \nonumber
\end{eqnarray}
Similarly, we have for the second differential calculus 
\begin{eqnarray}
\left( \hat{\partial}^{1}\right) _{L}^{-1}f &=&q\sum_{k=0}^{\infty }\left(
-\lambda \right) ^{k}q^{-k\left( k+1\right) }\left[ \left(
D_{q^{-2}}^{4}\right) ^{-1}x^{1}D_{q^{-2}}^{2}D_{q^{-2}}^{3}\right] ^{k} \\
&&\times \left( D_{q^{-2}}^{4}\right) ^{-1}f\left(
q^{k+1}x^{2},q^{k+1}x^{3}\right) ,  \nonumber \\ 
\left( \hat{\partial}^{2}\right) _{L}^{-1}f &=&\left( D_{q^{-2}}^{3}\right)
^{-1}f\left( qx^{1}\right) ,  \nonumber \\
\left( \hat{\partial}^{3}\right) _{L}^{-1}f &=&\left( D_{q^{-2}}^{2}\right)
^{-1}f\left( qx^{1}\right) ,  \nonumber \\
\left( \hat{\partial}^{4}\right) _{L}^{-1}f &=&q^{-1}\left(
D_{q^{-2}}^{1}\right) ^{-1}f.  \nonumber
\end{eqnarray}
Finally, we obtain for the inverse momentum generators the expressions 
\begin{eqnarray}
-\frac{i}{\left[ 2\right] _{q^{2}}}\left( P^{1}\right) _{L}^{-1}f
&=&\sum_{k=0}^{\infty }\left( -q\lambda \right) ^{k}\left[ \left(
N^{1}\right) ^{-1}\left( D_{q^{2}}^{4}\right)
^{-1}x^{1}D_{q^{-2}}^{2}D_{q^{-2}}^{3}\right] ^{k} \\
&&\times \left( N^{1}\right) ^{-1}\left( D_{q^{2}}^{4}\right) ^{-1}f, 
\nonumber \\
-\frac{i}{\left[ 2\right] _{q^{2}}}\left( P^{2}\right) _{L}^{-1}f
&=&q^{-1}\left( N^{2}\right) ^{-1}\left( D_{q^{2}}^{3}\right) ^{-1}f 
\nonumber \\
&&+\,q^{-1}\sum_{k=1}^{\infty }\lambda ^{k}q^{\frac{1}{2}k\left( 3k-5\right)
}\left[ \left( N^{2}\right) ^{-1}\left( D_{q^{2}}^{3}\right)
^{-1}x^{2}D_{q^{2}}^{1}D_{q^{2}}^{4}\right] ^{k}  \nonumber \\
&&\times \left( N^{2}\right) ^{-1}\left( D_{q^{2}}^{3}\right) ^{-1}f\left(
q^{-k}x^{1},q^{k}x^{2},q^{k}x^{3}\right) ,  \nonumber \\
-\frac{i}{\left[ 2\right] _{q^{2}}}\left( P^{3}\right) _{L}^{-1}f
&=&q^{-1}\left( N^{3}\right) ^{-1}\left( D_{q^{2}}^{2}\right) ^{-1}f 
\nonumber \\
&&+\,q^{-1}\sum_{k=1}^{\infty }\lambda ^{k}q^{\frac{1}{2}k\left( 3k-5\right)
}\left[ \left( N^{3}\right) ^{-1}\left( D_{q^{2}}^{2}\right)
^{-1}x^{3}D_{q^{2}}^{1}D_{q^{2}}^{4}\right] ^{k}  \nonumber \\
&&\times \left( N^{3}\right) ^{-1}\left( D_{q^{2}}^{2}\right) ^{-1}f\left(
q^{-k}x^{1},q^{k}x^{2},q^{k}x^{3}\right) ,  \nonumber \\
-\frac{i}{\left[ 2\right] _{q^{2}}}\left( P^{4}\right) _{L}^{-1}f
&=&q^{-2}\sum_{k=0}^{\infty }\left( q^{-4}\lambda \right) ^{k}  \nonumber \\
&&\times \left\{ \left( N^{4}\right) ^{-1}\left( D_{q^{2}}^{1}\right)
^{-1}\left[ q^{-1}x^{4}D_{q^{-2}}^{2}D_{q^{-2}}^{3}+\lambda x^{2}x^{3}\left(
D_{q^{-2}}^{1}\right) ^{2}D_{q^{-2}}^{4}\right] \right\} ^{k}  \nonumber \\
&&\times \left( N^{4}\right) ^{-1}\left( D_{q^{2}}^{1}\right) ^{-1}f\left(
q^{2k}x^{1},q^{k}x^{2},q^{k}x^{3},q^{-2}x^{4}\right)   \nonumber
\end{eqnarray}
where $N^{-1}$ and $\left( N^{i}\right) ^{-1}$, $i=1,\ldots ,4,$ denote the
inverse of the operators $N$ and $N^{i}$, $1,\ldots ,4,$ as introduced
previously in \cite{BW01}. Their explicit form can be looked up in appendix
\ref{AppB}. One should also recall that the above expressions have to
be considered
as definite integrals with their integration limits determining the limits
of the appearing Jackson integrals $\left( D_{q^{a}}^{i}\right) ^{-1}$, $%
i=1,\ldots ,4.$

Again we can establish a correspondence between left and right integrals 
\begin{eqnarray}
\left( \partial ^{i}\right) _{R}^{-1}f&\underleftrightarrowf&-q^{4}\left( \hat{\partial}%
^{i^{\prime }}\right) _{L}^{-1}f, \\
\left( \hat{\partial}^{i}\right) _{R}^{-1}f&\underleftrightarrowf&-q^{-4}\left( \partial ^{i^{\prime
}}\right) _{L}^{-1}f,  \nonumber \\
\left( P^{i}\right) _{R}^{-1}f&\underleftrightarrowf&-\left( P^{i^{\prime }}\right) _{L}^{-1}f\qquad
\nonumber
\end{eqnarray}
where $i=1,\ldots ,4$ and $i^{\prime }=5-i.$ The symbol $\underleftrightarrowf$ now denotes
that one can make a transition between the two expressions by applying the
substitutions 
\begin{equation}
x^{i}\rightarrow x^{i^{\prime }},\qquad \hat{\sigma}_{i}\rightarrow \hat{%
\sigma}_{i^{\prime }},\qquad \hat{\sigma}_{i}\equiv x^{i}\frac{\partial }{%
\partial x^{i}}\ 
\end{equation}
and 
\begin{equation}
D_{q^{a}}^{i}\rightarrow D_{q^{a}}^{i^{\prime }},\qquad \left(
D_{q^{a}}^{i}\right) ^{-1}\rightarrow \left( D_{q^{a}}^{i^{\prime }}\right)
^{-1}.
\end{equation}
It is important to realize that the expressions for $N^{-1}$ and $\left(
N^{i}\right) ^{-1}$, $i=1,\ldots ,4$, have to be affected by the above
substitutions,too. A simple example shall clarify the meaning of this point: 
\begin{equation}
\left( D_{q^{-2}}^{4}\right) ^{-1}D_{q^{-2}}^{2}q^{\left( \hat{\sigma}%
_{4}-2\right) \hat{\sigma}_{3}}f\left( q^{2}x^{2}\right) \;\underleftrightarrowf\;\left(
D_{q^{-2}}^{1}\right) ^{-1}D_{q^{-2}}^{3}q^{\left( \hat{\sigma}_{1}-2\right) 
\hat{\sigma}_{2}}f\left( q^{2}x^{3}\right) .
\end{equation}
The formulae for integration by parts are now given by 
\begin{eqnarray}
\left. \left( \partial ^{1}\right) _{L}^{-1}\left( \partial _{L}^{1}f\right)
\star g\right| _{x^{4}=a}^{b} &=&\left. f\star g\right|
_{x^{4}=a}^{b}-\left. \left( \partial^{1}\right) _{L}^{-1}\left(
\Lambda ^{1/2}K_{1}^{1/2}K_{2}^{1/2}f\right) \star \partial _{L}^{1}g\right|
_{x^{4}=a}^{b}, \\
\left. \left( \partial ^{2}\right) _{L}^{-1}\left( \partial _{L}^{2}f\right)
\star g\right| _{x^{3}=a}^{b} &=&\left. f\star g\right|
_{x^{3}=a}^{b}-\left. \left( \partial^{2}\right) _{L}^{-1}\left(
\Lambda ^{1/2}K_{1}^{-1/2}K_{2}^{1/2}f\right) \star \partial
_{L}^{2}g\right| _{x^{3}=a}^{b}  \nonumber \\
&&-\,q\lambda \left. \left( \partial ^{2}\right) _{L}^{-1}\left( \Lambda
^{1/2}K_{1}^{1/2}K_{2}^{1/2}L_{1}^{+}f\right) \star \partial
_{L}^{1}g\right| _{x^{3}=a}^{b},  \nonumber \\
\left. \left( \partial ^{3}\right) _{L}^{-1}\left( \partial _{L}^{3}f\right)
\star g\right| _{x^{2}=a}^{b} &=&\left. f\star g\right|
_{x^{2}=a}^{b}-\left. \left( \partial^{3}\right) _{L}^{-1}\left(
\Lambda ^{1/2}K_{1}^{1/2}K_{2}^{-1/2}f\right) \star \partial
_{L}^{3}g\right| _{x^{3}=a}^{b}  \nonumber \\
&&-\,q\lambda \left. \left( \partial ^{3}\right) _{L}^{-1}\left( \Lambda
^{1/2}K_{1}^{1/2}K_{2}^{1/2}L_{2}^{+}f\right) \star \partial
_{L}^{1}g\right| _{x^{2}=a}^{b},  \nonumber \\
\left. \left( \partial ^{4}\right) _{L}^{-1}\left( \partial _{L}^{4}f\right)
\star g\right| _{x^{1}=a}^{b} &=&\left. f\star g\right|
_{x^{1}=a}^{b}-\left. \left( \partial^{4}\right) _{L}^{-1}\left(
\Lambda ^{1/2}K_{1}^{-1/2}K_{2}^{-1/2}f\right) \star \partial
_{L}^{4}g\right| _{x^{1}=a}^{b}  \nonumber \\
&&+\,q\lambda \left. \left( \partial ^{4}\right) _{L}^{-1}\left( \Lambda
^{1/2}K_{1}^{1/2}K_{2}^{-1/2}L_{1}^{+}f\right) \star \partial
_{L}^{3}g\right| _{x^{1}=a}^{b}  \nonumber \\
&&+\,q\lambda \left. \left( \partial ^{4}\right) _{L}^{-1}\left( \Lambda
^{1/2}K_{1}^{-1/2}K_{2}^{1/2}L_{2}^{+}f\right) \star \partial
_{L}^{2}g\right| _{x^{1}=a}^{b}  \nonumber \\
&&+\,q^{2}\lambda ^{2}\left. \left( \partial ^{4}\right) _{L}^{-1}\left(
\Lambda ^{1/2}K_{1}^{1/2}K_{2}^{1/2}L_{1}^{+}L_{2}^{+}f\right) \star
\partial _{L}^{1}g\right| _{x^{1}=a}^{b}.  \nonumber
\end{eqnarray}
For the second covariant differential calculus one can compute likewise 
\begin{eqnarray}
\left. \left( \hat{\partial}^{1}\right) _{L}^{-1}\left( \hat{\partial}%
_{L}^{1}f\right) \star g\right| _{x^{4}=a}^{b} &=&\left. f\star g\right|
_{x^{4}=a}^{b}-\left. \left( \hat{\partial}^{1}\right)_L^{-1} \left( \Lambda
^{-1/2}K_{1}^{-1/2}K_{2}^{-1/2}f\right) \star \hat{\partial}_{L}^{1}g\right|
_{x^{4}=a}^{b} \\
&&+\,q^{-1}\lambda \left. \left( \hat{\partial}^{1}\right)
_{L}^{-1}\left( \Lambda ^{-1/2}K_{1}^{1/2}K_{2}^{-1/2}L_{1}^{-}f\right)
\star \hat{\partial}_{L}^{2}g\right| _{x^{4}=a}^{b}  \nonumber \\
&&+\,q^{-1}\lambda \left. \left( \hat{\partial}^{1}\right)
_{L}^{-1}\left( \Lambda ^{-1/2}K_{1}^{-1/2}K_{2}^{1/2}L_{2}^{-}f\right)
\star \hat{\partial}_{L}^{3}g\right| _{x^{4}=a}^{b}  \nonumber \\
&&+\,q^{-2}\lambda ^{2}\left. \left( \hat{\partial}^{1}\right)
_{L}^{-1}\left( \Lambda
^{-1/2}K_{1}^{1/2}K_{2}^{-1/2}L_{1}^{-}L_{2}^{-}f\right) \star \partial
_{L}^{4}g\right| _{x^{4}=a}^{b},  \nonumber \\
\left. \left( \hat{\partial}^{2}\right) _{L}^{-1}\left( \hat{\partial}%
_{L}^{2}f\right) \star g\right| _{x^{3}=a}^{b} &=&\left. f\star g\right|
_{x^{3}=a}^{b}-\left. \left( \hat{\partial}^{2}\right) _{L}^{-1}\left(
\Lambda ^{-1/2}K_{1}^{1/2}K_{2}^{-1/2}f\right) \star \hat{\partial}%
_{L}^{2}g\right| _{x^{3}=a}^{b}  \nonumber \\
&&-\,q^{-1}\lambda \left. \left( \hat{\partial}^{2}\right)
_{L}^{-1}\left( \Lambda ^{-1/2}K_{1}^{1/2}K_{2}^{1/2}L_{2}^{-}f\right) \star 
\hat{\partial}_{L}^{4}g\right| _{x^{3}=a}^{b},  \nonumber \\
\left. \left( \hat{\partial}^{3}\right) _{L}^{-1}\left( \hat{\partial}%
_{L}^{3}f\right) \star g\right| _{x^{2}=a}^{b} &=&\left. f\star g\right|
_{x^{2}=a}^{b}-\left. \left( \hat{\partial}^{3}\right) _{L}^{-1}\left(
\Lambda ^{-1/2}K_{1}^{-1/2}K_{2}^{1/2}f\right) \star \hat{\partial}%
_{L}^{3}g\right| _{x^{2}=a}^{b}  \nonumber \\
&&-\,q^{-1}\lambda \left. \left( \hat{\partial}^{3}\right)
_{L}^{-1}\left( \Lambda ^{-1/2}K_{1}^{1/2}K_{2}^{1/2}L_{1}^{-}f\right) \star 
\hat{\partial}_{L}^{4}g\right| _{x^{2}=a}^{b},  \nonumber \\
\left. \left( \hat{\partial}^{4}\right) _{L}^{-1}\left( \hat{\partial}%
_{L}^{4}f\right) \star g\right| _{x^{1}=a}^{b} &=&\left. f\star g\right|
_{x^{1}=a}^{b}-\left. \left( \hat{\partial}^{4}\right)_L^{-1} \left( \Lambda
^{-1/2}K_{1}^{1/2}K_{2}^{1/2}f\right) \star \hat{\partial}_{L}^{4}g\right|
_{x^{1}=a}^{b}.  \nonumber
\end{eqnarray}
For right integrals, however, we have 
\begin{eqnarray}
\left. \left( \partial ^{1}\right) _{R}^{-1}f\star \partial _{R}^{1}g\right|
_{x^{4}=a}^{b} &=&\left. f\star g\right| _{x^{4}=a}^{b}-\left. \left(
\partial ^{1}\right) _{R}^{-1}\left( \partial _{R}^{1}f\right) \star \left(
\Lambda ^{-1/2}K_{1}^{-1/2}K_{2}^{-1/2}g\right) \right| _{x^{4}=a}^{b} \\
\left. \left( \partial ^{2}\right) _{R}^{-1}f\star \partial _{R}^{2}g\right|
_{x^{3}=a}^{b} &=&\left. f\star g\right| _{x^{3}=a}^{b}-\left. \left(
\partial ^{2}\right) _{R}^{-1}\left( \partial _{R}^{2}f\right) \star \left(
\Lambda ^{-1/2}K_{1}^{1/2}K_{2}^{-1/2}g\right) \right| _{x^{3}=a}^{b} 
\nonumber \\
&&+\,\lambda \left. \left( \partial ^{2}\right) _{R}^{-1}\left( \partial
_{R}^{1}f\right) \star \left( \Lambda
^{-1/2}K_{1}^{1/2}K_{2}^{-1/2}L_{1}^{+}g\right) \right| _{x^{3}=a}^{b}, 
\nonumber \\
\left. \left( \partial ^{3}\right) _{R}^{-1}f\star \partial _{R}^{3}g\right|
_{x^{2}=a}^{b} &=&\left. f\star g\right| _{x^{2}=a}^{b}-\left. \left(
\partial ^{3}\right) _{R}^{-1}\left( \partial _{R}^{3}f\right) \star \left(
\Lambda ^{-1/2}K_{1}^{-1/2}K_{2}^{1/2}g\right) \right| _{x^{2}=a}^{b} 
\nonumber \\
&&+\,\lambda \left. \left( \partial ^{3}\right) _{R}^{-1}\left( \partial
_{R}^{1}f\right) \star \left( \Lambda
^{-1/2}K_{1}^{-1/2}K_{2}^{1/2}L_{2}^{+}g\right) \right| _{x^{2}=a}^{b}, 
\nonumber \\
\left. \left( \partial ^{4}\right) _{R}^{-1}f\star \partial _{R}^{4}g\right|
_{x^{1}=a}^{b} &=&\left. f\star g\right| _{x^{1}=a}^{b}-\left. \left(
\partial ^{4}\right) _{R}^{-1}\left( \partial _{R}^{4}f\right) \star \left(
\Lambda ^{-1/2}K_{1}^{1/2}K_{2}^{1/2}g\right) \right| _{x^{1}=a}^{b} 
\nonumber \\
&&-\,\lambda \left. \left( \partial ^{4}\right) _{R}^{-1}\left( \partial
_{R}^{3}f\right) \star \left( \Lambda
^{-1/2}K_{1}^{1/2}K_{2}^{1/2}L_{1}^{+}g\right) \right| _{x^{1}=a}^{b} 
\nonumber \\
&&-\,\lambda \left. \left( \partial ^{4}\right) _{R}^{-1}\left( \partial
_{R}^{2}f\right) \star \left( \Lambda
^{-1/2}K_{1}^{1/2}K_{2}^{1/2}L_{2}^{+}g\right) \right| _{x^{1}=a}^{b} 
\nonumber \\
&&+\,\lambda ^{2}\left. \left( \partial ^{4}\right) _{R}^{-1}\left( \partial
_{R}^{1}f\right) \star \left( \Lambda
^{-1/2}K_{1}^{1/2}K_{2}^{1/2}L_{1}^{+}L_{2}^{+}g\right) \right|
_{x^{1}=a}^{b}  \nonumber
\end{eqnarray}
and for the second covariant differential calculus 
\begin{eqnarray}
\left. \left( \hat{\partial}^{1}\right) _{R}^{-1}f\star \hat{\partial}%
_{R}^{1}g\right| _{x^{4}=a}^{b} &=&\left. f\star g\right|
_{x^{4}=a}^{b}-\left. \left( \hat{\partial}^{1}\right) _{R}^{-1}\left( \hat{%
\partial}_{R}^{1}f\right) \star \left( \Lambda
^{1/2}K_{1}^{1/2}K_{2}^{1/2}g\right) \right| _{x^{4}=a}^{b} \\
&&-\,\lambda \left. \left( \hat{\partial}^{1}\right) _{R}^{-1}\left( \hat{%
\partial}_{R}^{2}f\right) \star \left( \Lambda
^{1/2}K_{1}^{1/2}K_{2}^{1/2}L_{1}^{-}g\right) \right| _{x^{4}=a}^{b} 
\nonumber \\
&&-\,\lambda \left. \left( \hat{\partial}^{1}\right) _{R}^{-1}\left( \hat{%
\partial}_{R}^{3}f\right) \star \left( \Lambda
^{1/2}K_{1}^{1/2}K_{2}^{1/2}L_{2}^{-}g\right) \right| _{x^{4}=a}^{b} 
\nonumber \\
&&+\,\lambda ^{2}\left. \left( \hat{\partial}^{1}\right) _{R}^{-1}\left( 
\hat{\partial}_{R}^{4}f\right) \star \left( \Lambda
^{1/2}K_{1}^{1/2}K_{2}^{1/2}L_{1}^{-}L_{1}^{-}g\right) \right|
_{x^{4}=a}^{b},  \nonumber \\
\left. \left( \hat{\partial}^{2}\right) _{R}^{-1}f\star \hat{\partial}%
_{R}^{2}g\right| _{x^{3}=a}^{b} &=&\left. f\star g\right|
_{x^{3}=a}^{b}-\left. \left( \hat{\partial}^{2}\right) _{R}^{-1}\left( \hat{%
\partial}_{R}^{2}f\right) \star \left( \Lambda
^{1/2}K_{1}^{-1/2}K_{2}^{1/2}g\right) \right| _{x^{3}=a}^{b}  \nonumber \\
&&+\,\lambda \left. \left( \hat{\partial}^{2}\right) _{R}^{-1}\left( \hat{%
\partial}_{R}^{4}f\right) \star \left( \Lambda
^{1/2}K_{1}^{-1/2}K_{2}^{1/2}L_{2}^{-}g\right) \right| _{x^{3}=a}^{b}, 
\nonumber \\
\left. \left( \hat{\partial}^{3}\right) _{R}^{-1}f\star \hat{\partial}%
_{R}^{3}g\right| _{x^{2}=a}^{b} &=&\left. f\star g\right|
_{x^{2}=a}^{b}-\left. \left( \hat{\partial}^{3}\right) _{R}^{-1}\left( \hat{%
\partial}_{R}^{3}f\right) \star \left( \Lambda
^{1/2}K_{1}^{-1/2}K_{2}^{-1/2}g\right) \right| _{x^{2}=a}^{b}  \nonumber \\
&&+\,\lambda \left. \left( \hat{\partial}^{3}\right) _{R}^{-1}\left( \hat{%
\partial}_{R}^{4}f\right) \star \left( \Lambda
^{1/2}K_{1}^{1/2}K_{2}^{-1/2}L_{1}^{-}g\right) \right| _{x^{2}=a}^{b}, 
\nonumber \\
\left. \left( \hat{\partial}^{4}\right) _{R}^{-1}f\star \hat{\partial}%
_{R}^{4}g\right| _{x^{1}=a}^{b} &=&\left. f\star g\right|
_{x^{1}=a}^{b}-\left. \left( \hat{\partial}^{4}\right) _{R}^{-1}\left( \hat{%
\partial}_{R}^{4}f\right) \star \left( \Lambda
^{1/2}K_{1}^{-1/2}K_{2}^{-1/2}g\right) \right| _{x^{1}=a}^{b}.  \nonumber
\end{eqnarray}

Next we turn to expressions of the form $\left( \partial ^{1}\right)
^{-1}\left( \partial ^{2}\right) ^{-1}\left( \partial ^{3}\right)
^{-1}\left( \partial ^{4}\right) ^{-1}$ which can again be regarded as a
q-deformed version of four dimensional volume integrals. If surface terms
are neglected, we can state the identities 
\begin{eqnarray}
\left( \partial ^{1}\right) ^{-1}\left( \partial ^{2}\right) ^{-1}\left(
\partial ^{3}\right) ^{-1}\left( \partial ^{4}\right) ^{-1} &=&\left(
\partial ^{1}\right) ^{-1}\left( \partial ^{3}\right) ^{-1}\left( \partial
^{2}\right) ^{-1}\left( \partial ^{4}\right) ^{-1}=  \label{volume4} \\
q\left( \partial ^{1}\right) ^{-1}\left( \partial ^{2}\right) ^{-1}\left(
\partial ^{4}\right) ^{-1}\left( \partial ^{3}\right) ^{-1} &=&q\left(
\partial ^{1}\right) ^{-1}\left( \partial ^{3}\right) ^{-1}\left( \partial
^{4}\right) ^{-1}\left( \partial ^{2}\right) ^{-1}=  \nonumber \\
q\left( \partial ^{2}\right) ^{-1}\left( \partial ^{1}\right) ^{-1}\left(
\partial ^{3}\right) ^{-1}\left( \partial ^{4}\right) ^{-1} &=&q\left(
\partial ^{3}\right) ^{-1}\left( \partial ^{1}\right) ^{-1}\left( \partial
^{2}\right) ^{-1}\left( \partial ^{4}\right) ^{-1}=  \nonumber \\
q^{3}\left( \partial ^{2}\right) ^{-1}\left( \partial ^{4}\right)
^{-1}\left( \partial ^{3}\right) ^{-1}\left( \partial ^{1}\right) ^{-1}
&=&q^{3}\left( \partial ^{3}\right) ^{-1}\left( \partial ^{4}\right)
^{-1}\left( \partial ^{2}\right) ^{-1}\left( \partial ^{1}\right) ^{-1}= 
\nonumber \\
q^{3}\left( \partial ^{4}\right) ^{-1}\left( \partial ^{2}\right)
^{-1}\left( \partial ^{1}\right) ^{-1}\left( \partial ^{3}\right) ^{-1}
&=&q^{3}\left( \partial ^{4}\right) ^{-1}\left( \partial ^{3}\right)
^{-1}\left( \partial ^{1}\right) ^{-1}\left( \partial ^{2}\right) ^{-1}= 
\nonumber \\
q^{4}\left( \partial ^{4}\right) ^{-1}\left( \partial ^{2}\right)
^{-1}\left( \partial ^{3}\right) ^{-1}\left( \partial ^{1}\right) ^{-1}
&=&q^{4}\left( \partial ^{4}\right) ^{-1}\left( \partial ^{3}\right)
^{-1}\left( \partial ^{2}\right) ^{-1}\left( \partial ^{1}\right) ^{-1}= 
\nonumber \\
&=&q^{2}\;\times \text{remaining combinations.}  \nonumber
\end{eqnarray} 
Let us note that the elements $\left( \hat{\partial}^{i}\right) ^{-1}$ and $%
\left( P^{i}\right) ^{-1},$ $i=1,\ldots ,4,$ obey the same relations.
Since the results of the various volume integrals in (\ref{volume4}) differ
by a normalisation factor only, we can restrict attention to one of the
above expressions. Hence, it is sufficient to consider the following explicit
formulae: 
\begin{eqnarray}
\lefteqn{\left( \partial ^{4}\right) _{L}^{-1}\left( \partial ^{3}\right)
_{L}^{-1}\left( \partial ^{2}\right) _{L}^{-1}\left( \partial ^{1}\right)
_{L}^{-1}f =} \\
&=&\hspace*{0cm}q^{-4}N^{-1}\left( D_{q^{2}}^{1}\right) ^{-1}\left(
D_{q^{2}}^{2}\right) ^{-1}\left( D_{q^{2}}^{3}\right) ^{-1}\left(
D_{q^{2}}^{4}\right) ^{-1}  \nonumber \\
&&\hspace*{0cm}\times f(q^{-2}x^{1},q^{-1}x^{2},q^{-1}x^{3},q^{-4}x^{4}), 
\nonumber \\
\lefteqn{\left( \hat{\partial}^{1}\right) _{L}^{-1}\left( \hat{\partial}^{2}\right)
_{L}^{-1}\left( \hat{\partial}^{3}\right) ^{-1}\left( \hat{\partial}%
^{4}\right) ^{-1}f =}  \nonumber \\
&=&\hspace*{0cm}q^{4}\left( D_{q^{-2}}^{4}\right) ^{-1}\left(
D_{q^{-2}}^{3}\right) ^{-1}\left( D_{q^{-2}}^{2}\right) ^{-1}\left(
D_{q^{-2}}^{1}\right) ^{-1}f(q^{2}x^{1},qx^{2},qx^{3}),  \nonumber \\
\lefteqn{\left( -\frac{i}{\left[ 2\right] _{q^{2}}}\right) ^{4}\left( P^{4}\right)
_{L}^{-1}\left( P^{3}\right) _{L}^{-1}\left( P^{2}\right) _{L}^{-1}\left(
P^{1}\right) _{L}^{-1}f =}  \nonumber \\
&=&\hspace*{0cm}q^{-4}\left( N^{4}\right) ^{-1}\left( D_{q^{2}}^{1}\right)
^{-1}\left( N^{3}\right) ^{-1}\left( D_{q^{2}}^{2}\right) ^{-1}  \nonumber \\
&&\hspace*{0cm}\times\, \left( N^{2}\right) ^{-1}\left( D_{q^{2}}^{3}\right)
^{-1}\left( N^{1}\right) ^{-1}\left( D_{q^{2}}^{4}\right) ^{-1}f.  \nonumber
\end{eqnarray}
With the same reasoning applied to q-deformed Euclidean space in three
dimensions we can immediately varify rotation and translation invariance of
our four dimensional volume integrals.

\section{q-deformed Minkowski-Space\label{KapMin}}

In principle all considerations of the previous two sections pertain equally
to q-deformed Minkowski space \cite{LWW97} apart from the fact that the
results now obey a more involved structure. The partial derivatives of
q-deformed Minkowski space satisfy the relations 
\begin{eqnarray}
\partial ^{0}\partial ^{-} &=&\partial ^{-}\partial ^{0},\quad \partial
^{0}\partial ^{+}=\partial ^{+}\partial ^{0},\quad \partial ^{0}\tilde{%
\partial}^{3}=\tilde{\partial}^{3}\partial ^{0},  \label{Derivative} \\
\tilde{\partial}^{3}\partial ^{+} &=&q^{2}\partial ^{+}\tilde{\partial}%
^{3},\quad \partial ^{-}\tilde{\partial}^{3}=q^{2}\tilde{\partial}%
^{3}\partial ^{-},  \nonumber \\
\partial ^{-}\partial ^{+} &=&\partial ^{+}\partial ^{-}+\lambda \left( 
\tilde{\partial}^{3}\tilde{\partial}^{3}+\partial ^{0}\tilde{\partial}%
^{3}\right) ,  \nonumber
\end{eqnarray} 
with $\lambda =q-q^{-1}$ and $q>1.$ As usual, the inverse elements $\left(
\partial ^{\mu}\right) ^{-1}$, $\mu=\pm ,0,\tilde{3},$ are defined by 
\begin{equation}
\left( \partial ^{\mu}\right) ^{-1}\partial ^{\mu} =\partial ^{\mu}\left(
\partial ^{\mu}\right) ^{-1}=1
\end{equation}
and the remaining commutation relations involving partial derivatives
 are given by 
\begin{eqnarray}
\left( \tilde{\partial}^{3}\right) ^{-1}\partial ^{+} &=&q^{-2}\partial
^{+}\left( \tilde{\partial}^{3}\right) ^{-1},\quad \tilde{\partial}%
^{3}\left( \partial ^{+}\right) ^{-1}=q^{-2}\left( \partial ^{+}\right) ^{-1}%
\tilde{\partial}^{3}, \\
\left( \partial ^{-}\right) ^{-1}\tilde{\partial}^{3} &=&q^{-2}\tilde{%
\partial}^{3}\left( \partial ^{-}\right) ^{-1},\quad \partial ^{-}\left( 
\tilde{\partial}^{3}\right) ^{-1}=q^{-2}\left( \tilde{\partial}^{3}\right)
^{-1}\partial ^{-},  \nonumber \\
\left( \partial ^{-}\right) ^{-1}\partial ^{+} &=&\partial ^{+}\left(
\partial ^{-}\right) ^{-1}-q^{-2}\lambda \partial ^{0}\tilde{\partial}%
^{3}\left( \partial ^{-}\right) ^{-2}-q^{-4}\lambda \left( \tilde{\partial}%
^{3}\right) ^{2}\left( \partial ^{-}\right) ^{-2},  \nonumber \\
\partial ^{-}\left( \partial ^{+}\right) ^{-1} &=&\left( \partial
^{+}\right) ^{-1}\partial ^{-}-q^{-2}\lambda \left( \partial ^{+}\right)
^{-2}\partial ^{0}\tilde{\partial}^{3}-q^{-4}\lambda \left( \partial
^{+}\right) ^{-2}\left( \tilde{\partial}^{3}\right) ^{2},  \nonumber \\
\left( \partial ^{0}\right) ^{-1}\partial ^{A} &=&\partial ^{A}\left(
\partial ^{0}\right) ^{-1},\quad \partial ^{0}\left( \partial ^{A}\right)
^{-1}=\left( \partial ^{A}\right) ^{-1}\partial ^{0},\quad A=\pm ,\tilde{3}.
\nonumber
\end{eqnarray}
Finally, there are the commutation relations 
\begin{eqnarray}
\left( \partial ^{0}\right) ^{-1}\left( \partial ^{\mu}\right) ^{-1} &=&\left(
\partial ^{\mu}\right) ^{-1}\left( \partial ^{0}\right) ^{-1},\quad \mu=\pm ,%
\tilde{3}, \\
\left( \partial ^{-}\right) ^{-1}\left( \tilde{\partial}^{3}\right) ^{-1}
&=&q^{2}\left( \tilde{\partial}^{3}\right) ^{-1}\left( \partial ^{-}\right)
^{-1},\quad \left( \tilde{\partial}^{3}\right) ^{-1}\left( \partial
^{+}\right) ^{-1}=q^{2}\left( \partial ^{+}\right) ^{-1}\left( \tilde{%
\partial}^{3}\right) ^{-1},  \nonumber \\
\left( \partial ^{-}\right) ^{-1}\left( \partial ^{+}\right) ^{-1}
&=&q^{2}\sum_{i=0}^{\infty }\left( \frac{\lambda}{\lambda _{+}}\right)
^{i}\left[ \left[ i\right] \right] _{q^{2}}!\left(
  \left[\begin{array}{c} -1\\i\end{array}\right]_{q^{2}}\right) ^{2}\sum_{j+k=i}\left( -q^{6}\right) q^{i\left( i+2k\right) }%
\left[ \begin{array}{c} i\\k\end{array}\right]_{q^{2}}  \nonumber \\
&&\times \sum_{p=0}^{k}\left( q^{4j}\lambda _{+}\right) ^{p}\left( \partial
^{+}\right) ^{p-\left( i+1\right) }\left( \tilde{\partial}^{3}\right)
^{2j}S_{k,p}\left( \partial ^{0},\tilde{\partial}^{3}\right) \left( \partial
^{-}\right) ^{p-\left( i+1\right) }  \nonumber
\end{eqnarray}
where the symbols $S_{k,p}$ stand for polynomials of degree $2(k-p),$ with
their explicit form presented in appendix \ref{AppA}.

Next, we come to the commutation relations with the Lorentz generators \cite
{OSWZ92}, \cite{SWZ91}, \cite{RW99} which become 
\begin{eqnarray}
T^{+}\left( \partial ^{0}\right) ^{-1} &=&\left( \partial ^{0}\right)
^{-1}T^{+}, \\
T^{+}\left( \tilde{\partial}^{3}\right) ^{-1} &=&\left( \tilde{\partial}%
^{3}\right) ^{-1}T^{+}-q^{1/2}\lambda _{+}^{1/2}\left( \tilde{\partial}%
^{3}\right) ^{-2}\partial ^{+},  \nonumber \\
T^{+}\left( \partial ^{+}\right) ^{-1} &=&q^{2}\left( \partial ^{+}\right)
^{-1}T^{+},  \nonumber \\
T^{+}\left( \partial ^{-}\right) ^{-1} &=&q^{-2}\left( \partial ^{-}\right)
^{-1}T^{+}-q^{-1/2}\lambda _{+}^{1/2}\left( \partial ^{-}\right) ^{-2}\left( 
\tilde{\partial}^{3}+q^{-2}\partial ^{0}\right) ,  \nonumber \\[0.16in]
T^{-}\left( \partial ^{0}\right) ^{-1} &=&\left( \partial ^{0}\right)
^{-1}T^{-}, \\
T^{-}\left( \tilde{\partial}^{3}\right) ^{-1} &=&\left( \tilde{\partial}%
^{3}\right) ^{-1}T^{-}-q^{-1/2}\lambda _{+}^{1/2}\left( \tilde{\partial}%
^{3}\right) ^{-2}\partial ^{-},  \nonumber \\
T^{-}\left( \partial ^{-}\right) ^{-1} &=&q^{-2}\left( \partial ^{-}\right)
^{-1}T^{-},  \nonumber \\
T^{-}\left( \partial ^{+}\right) ^{-1} &=&q^{2}\left( \partial ^{+}\right)
^{-1}T^{-}-q^{1/2}\lambda _{+}^{1/2}\left( \partial ^{+}\right) ^{-2}\left( 
\tilde{\partial}^{3}+q^{2}\partial ^{0}\right) ,  \nonumber \\[0.16in]
\tau ^{3}\left( \partial ^{0}\right) ^{-1} &=&\left( \partial ^{0}\right)
^{-1}\tau ^{3},\quad \tau ^{3}\left( \tilde{\partial}^{3}\right)
^{-1}=\left( \tilde{\partial}^{3}\right) ^{-1}\tau ^{3}, \\
\tau ^{3}\left( \partial ^{+}\right) ^{-1} &=&q^{4}\left( \partial
^{+}\right) ^{-1}\tau ^{3},\quad \tau ^{3}\left( \partial ^{-}\right)
^{-1}=q^{-4}\left( \partial ^{-}\right) ^{-1}\tau ^{3},  \nonumber \\[0.16in]
T^{2}\left( \tilde{\partial}^{3}\right) ^{-1} &=&q\left( \tilde{\partial}%
^{3}\right) ^{-1}T^{2},\quad T^{2}\left( \partial ^{+}\right)
^{-1}=q^{-1}\left( \partial ^{+}\right) ^{-1}T^{2}, \\
T^{2}\left( \partial ^{-}\right) ^{-1} &=&q\left( \partial ^{-}\right)
^{-1}T^{2}-q^{-3/2}\lambda _{+}^{-1/2}\tilde{\partial}^{3}\left( \partial
^{-}\right) ^{-2}\tau ^{1},  \nonumber \\
T^{2}\left( \partial ^{3}\right) ^{-1} &=&\left( T^{2}\triangleright \left(
\partial ^{3}\right) ^{-1}\right) \tau ^{1}+\left( \sigma ^{2}\triangleright
\left( \partial ^{3}\right) ^{-1}\right) T^{2},  \nonumber \\[0.16in]
S^{1}\left( \tilde{\partial}^{3}\right) ^{-1} &=&q^{-1}\left( \tilde{\partial%
}^{3}\right) ^{-1}S^{1},\quad S^{1}\left( \partial ^{-}\right)
^{-1}=q^{-1}\left( \partial ^{-}\right) ^{-1}S^{1}, \\
S^{1}\left( \partial ^{+}\right) ^{-1} &=&q\left( \partial ^{+}\right)
^{-1}S^{1}+q^{-1/2}\lambda _{+}^{-1/2}\left( \partial ^{+}\right) ^{-2}%
\tilde{\partial}^{3}\sigma ^{2},  \nonumber \\
S^{1}\left( \partial ^{3}\right) ^{-1} &=&\left( S^{1}\triangleright \left(
\partial ^{3}\right) ^{-1}\right) \sigma ^{2}+\left( \tau ^{1}\triangleright
\left( \partial ^{3}\right) ^{-1}\right) S^{1},  \nonumber \\[0.16in]
\tau ^{1}\left( \tilde{\partial}^{3}\right) ^{-1} &=&q^{-1}\left( \tilde{%
\partial}^{3}\right) ^{-1}\tau ^{1},\quad \tau ^{1}\left( \partial
^{-}\right) ^{-1}=q\left( \partial ^{-}\right) ^{-1}\tau ^{1}, \\
\tau ^{1}\left( \partial ^{+}\right) ^{-1} &=&q^{-1}\left( \partial
^{+}\right) ^{-1}\tau ^{1}+q^{3/2}\lambda _{+}^{-1/2}\lambda ^{2}\tilde{%
\partial}^{3}\left( \partial ^{+}\right) ^{-2}T^{2},  \nonumber \\
\tau ^{1}\left( \partial ^{3}\right) ^{-1} &=&\left( \tau ^{1}\triangleright
\left( \partial ^{3}\right) ^{-1}\right) \tau ^{1}+\lambda ^{2}\left(
S^{1}\triangleright \left( \partial ^{3}\right) ^{-1}\right) T^{2}, 
\nonumber \\[0.16in]
\sigma ^{2}\left( \tilde{\partial}^{3}\right) ^{-1} &=&q\left( \tilde{%
\partial}^{3}\right) ^{-1}\sigma ^{2},\quad \sigma ^{2}\left( \partial
^{+}\right) ^{-1}=q\left( \partial ^{+}\right) ^{-1}\sigma ^{2}, \\
\sigma ^{2}\left( \partial ^{-}\right) ^{-1} &=&q^{-1}\left( \partial
^{-}\right) ^{-1}\sigma ^{2}-q^{1/2}\lambda _{+}^{-1/2}\lambda ^{2}\left(
\partial ^{-}\right) ^{-2}\tilde{\partial}^{3}S^{1},  \nonumber \\
\sigma ^{2}\left( \partial ^{3}\right) ^{-1} &=&\left( \sigma
^{2}\triangleright \left( \partial ^{3}\right) ^{-1}\right) \sigma
^{2}+\lambda ^{2}\left( T^{2}\triangleright \left( \partial ^{3}\right)
^{-1}\right) S^{1}  \nonumber
\end{eqnarray} 
where we have to insert the actions
\begin{eqnarray}
T^{2}\triangleright \left( \partial ^{3}\right) ^{-1}
&=&q^{-3/2}\sum_{k=0}^{\infty }\left( -q^{2}\frac{\lambda ^{2}}{\lambda
_{+}^{2}}\right) ^{k}\left( K_{1,q^{2}}\right) _{-1}^{\left( k,k+1\right) }
 \\
&&\times \sum_{0\leq i+j\leq k}\lambda _{+}^{j-1/2} {k \choose i}\left(
\partial ^{+}\right) ^{j+1}\left( a_{+}\left( q^{2j}\tilde{\partial}%
^{3}\right) \right) ^{i}  \nonumber \\
&&\times S_{k-i,j}\left( \partial ^{0},\tilde{\partial}^{3}\right) \left(
\partial ^{0}+\frac{2q^{2j+1}}{\left[ 2\right] _{q}}\tilde{\partial}%
^{3}\right) ^{-2\left( k+1\right) }\left( \partial ^{-}\right) ^{j}, 
\nonumber \\[0.1in]
S^{1}\triangleright \left( \partial ^{3}\right) ^{-1}
&=&-q^{-3/2}\sum_{k=0}^{\infty }\left( -q^{-2}\frac{\lambda ^{2}}{\lambda
_{+}^{2}}\right) ^{k}\left( K_{1,q^{-2}}\right) _{-1}^{\left( k,k+1\right) }
\\
&&\times \sum_{0\leq i+j\leq k}\lambda _{+}^{j-1/2}{k \choose i}\left(
\partial ^{+}\right) ^{j}\left( a_{+}\left( q^{2j}\tilde{\partial}%
^{3}\right) \right) ^{i}  \nonumber \\
&&\times S_{k-i,j}\left( \partial ^{0},\tilde{\partial}^{3}\right) \left(
\partial ^{0}+\frac{2q^{2j+1}}{\left[ 2\right] _{q}}\tilde{\partial}%
^{3}\right) ^{-2\left( k+1\right) }\left( \partial ^{-}\right) ^{j+1}, 
\nonumber \\[0.1in]
\tau ^{1}\triangleright \left( \partial ^{3}\right) ^{-1}
&=&q\sum_{k=0}^{\infty }\left( -q^{2}\frac{\lambda ^{2}}{\lambda _{+}^{2}}%
\right) ^{k}\left( K_{1,q^{2}}\right) _{-1}^{\left( k,k\right) } \\
&&\times \sum_{0\leq i+j\leq k}\lambda _{+}^{j}{k \choose i}\left( \partial
^{+}\right) ^{j}\left( a_{+}\left( q^{2j}\tilde{\partial}^{3}\right) \right)
^{i}  \nonumber \\
&&\times S_{k-i,j}\left( \partial ^{0},\tilde{\partial}^{3}\right) \left(
\partial ^{0}+\frac{2q^{2j+1}}{\left[ 2\right] _{q}}\tilde{\partial}%
^{3}\right) ^{-2k-1}\left( \partial ^{-}\right) ^{j},  \nonumber \\[0.1in]
\sigma ^{2}\triangleright \left( \partial ^{3}\right) ^{-1}
&=&q^{-1}\sum_{k=0}^{\infty }\left( -q^{-2}\frac{\lambda ^{2}}{\lambda
_{+}^{2}}\right) ^{k}\left( K_{1,q^{-2}}\right) _{-1}^{\left(
k,k\right) }
\label{Wirkungen} \\
&&\times \sum_{0\leq i+j\leq k}\lambda _{+}^{j}{k \choose i}\left( \partial
^{+}\right) ^{j}\left( a_{-}\left( q^{2j}\tilde{\partial}^{3}\right) \right)
^{i}  \nonumber \\
&&S_{k-i,j}\left( \partial ^{0},\tilde{\partial}^{3}\right) \left( \partial
^{0}+\frac{2q^{2j-1}}{\left[ 2\right] _{q}}\tilde{\partial}^{3}\right)
^{-2k-1}\left( \partial ^{-}\right) ^{j}.  \nonumber
\end{eqnarray}
Note that these expressions have been formulated by using the abbreviations
and conventions listed in appendix \ref{AppA}. Applying the substitutions 
\begin{equation}
\partial ^{\mu}\rightarrow \hat{\partial}^{\mu},\qquad \left( \partial
^{\mu}\right) ^{-1}\rightarrow \left( \hat{\partial}^{\mu}\right) ^{-1}
\end{equation}
and 
\begin{equation}
\partial ^{\mu}\rightarrow P^{\mu},\qquad \left( \partial ^{\mu}\right)
^{-1}\rightarrow \left( P^{\mu}\right) ^{-1}
\end{equation} 
to the formulae (\ref{Derivative})-(\ref{Wirkungen}) again yields the
corresponding expressions for the second covariant differential calculus and
for the algebra of momentum generators, respectively.

As in the Euclidean case the representations of partial derivatives of
q-deformed Minkowski space \cite{BW01} can again be split into two parts
\begin{equation}
\partial ^{\mu}F=\left( \left( \partial _{(i=0)}^{\mu}\right) +\left( \partial
_{\left( i>0\right) }^{\mu}\right) \right) F, \quad \mu=\pm, 0,\tilde{3}.
\end{equation}
Now, we can follow the same lines as in the previous sections. Thus,
solutions to the difference equations 
\begin{equation}
\partial ^{\mu}F=f,\quad \mu=\pm ,0,\tilde{3},
\end{equation}
are given by 
\begin{eqnarray}
F &=&\left( \partial ^{\mu}\right) ^{-1}f  \label{IntegralM} \\
&=&\sum_{k=0}^{\infty }\left( -1\right) ^{k}\left[ \left( \partial _{\left(
i=0\right) }^{\mu}\right) ^{-1}\left( \partial _{\left( i>0\right)
}^{\mu}\right) \right] ^{k}\left( \partial _{\left( i=0\right) }^{\mu}\right)
^{-1}f.  \nonumber
\end{eqnarray}

For the differential calculus with partial derivatives $\partial ^{\mu},$ $\mu=\pm ,0,\tilde{3},$ the operators
needed for computing (\ref{IntegralM}) can in explicit terms be written as 
\begin{eqnarray}
\left( \tilde{\partial}_{\left( i=0\right) }^{3}\right) _{L}^{-1}f &=&\left(
D_{q_{+}q^{-2}}^{3}\right) ^{-1}f,  \label{DerMin1} \\
\left( \tilde{\partial}_{\left( i>0\right) }^{3}\right) _{L}f &=&-q
\lambda ^{2}\lambda_{+}^{-1}x^{+}\tilde{x}^{3}D_{q^{2}}^{+}\left(
D_{2,q^{-1}}^{3}\right) ^{1,1}f  \nonumber \\
&&\hspace{-2cm}-\,\lambda \left\{ q^{-1}x^{-}D_{q^{-2}}^{-}\left(
D_{q^{-2}}^{3}f\right) \left( q_{+}x^{3}\right) +\tilde{x}%
^{3}D_{q^{2}}^{+}D_{q^{-2}}^{-}f\left( q^{-2}\tilde{x}^{3},q_{-}x^{3}\right)
\right\}  \nonumber \\
&&\hspace{-2cm}+\,\sum_{l=1}^{\infty }\alpha _{-}^{l}\sum_{0\leq i+j\leq
l}\left. \left\{ -q^{2}\lambda \left( M^{-}\right) _{i,j}^{l}\left( \underline{x}%
\right)\left(\tilde{T}^3_1\right)^l_jf +\,\left( M^{+}\right) _{i,j}^{l}\left( \underline{x}%
\right)  \left(\tilde{T}^3_2\right)^l_jf\right\}
\right| _{x^{0}\rightarrow x^{3}-\tilde{x}^{3}}
,  \nonumber \\[0.16in]
\left( \partial _{\left( i=0\right) }^{+}\right) _{L}^{-1}f &=&-q^{-1}\left(
D_{q^{-2}}^{-}\right) ^{-1}f\left( q^{2}\tilde{x}^{3},q_{+}x^{3}\right) , \\
\left( \partial _{\left( i>0\right) }^{+}\right) _{L}f &=&-\frac{\lambda }{%
\lambda _{+}}x^{+}\left( D_{2,q^{-1}}^{3}\right) ^{1,1}f  \nonumber \\
&&\hspace{-2cm}+\sum_{l=1}^{\infty }\alpha _{-}^{l}\sum_{0\leq i+j\leq
l}\left.\left\{ -q\left( M^{-}\right) _{i,j}^{l}\left(
  \underline{x}\right)\left(T^+_1\right)^l_jf -\frac{\lambda }{\lambda _{+}}\left( M^{+}\right)
_{i,j}^{l}\left( \underline{x}\right) 
\left(T^+_2\right)^l_jf\right\}\right| _{x^{0}\rightarrow x^{3}-\tilde{x}%
^{3}} ,  \nonumber \\[0.16in]
\left( \partial _{\left( i=0\right) }^{0}\right) _{L}^{-1}f &=&\left( \tilde{%
D}_{q^{-2}}^{3}\right) ^{-1}f\left( q^{2}x^{+},q_{+}x^{3},q^{2}x^{-}\right) ,
\\
\left( \partial _{\left( i>0\right) }^{0}\right) _{L}f &=&-\frac{\lambda}{ \lambda_{+}}x^{+}D_{q^{-2}}^{+}D_{q_{+}q^{-2}}^{3}f+  \nonumber \\
&&\hspace{-2cm}-\,\lambda \left( x^{3}-q^{-1}\lambda _{+}^{-1}\tilde{x}%
^{3}\right) D_{q^{-2}}^{+}D_{q^{-2}}^{-}f\left( q^{-2}\tilde{x}%
^{3},q_{-}x^{3}\right)  \nonumber \\
&&\hspace{-2cm}+\,q^{-1}\lambda _{+}^{-1}\lambda
^{2}x^{+}x^{-}D_{q^{-2}}^{+}D_{q^{-2}}^{-}\left( D_{q^{-2}}^{3}f\right)
f\left( q_{+}x^{3}\right)  \nonumber \\
&&\hspace{-2cm}+\,q^{-1}\lambda _{+}^{-1}\lambda ^{2}x^{+}\left(
x^{3}-q^{-1}\lambda _{+}^{-1}\tilde{x}^{3}\right) D_{q^{-2}}^{+}\left(
D_{2,q^{-1}}^{3}\right) ^{1,1}f  \nonumber \\
&&\hspace{-2cm}+\,\sum_{l=1}^{\infty }\alpha _{-}^{l}\sum_{0\leq i+j\leq
l}\left.\left\{ \left( M^{-}\right) _{i,j}^{l}\left( \underline{x}\right)
 \left(T^0_1\right)^l_jf-\frac{\lambda }{\lambda _{+}}\left( M^{+}\right)
_{i,j}^{l}\left( \underline{x}\right) \left(T^0_2\right)^l_jf\right\}\right|
_{x^{0}\rightarrow x^{3}-\tilde{x}^{3}}  \nonumber \\
&&\hspace{-2cm}+\,\frac{\lambda }{\lambda _{+}}\sum_{0\leq l+m<\infty
}\alpha _{-}^{l+m}\sum_{i=0}^{l}\sum_{j=0}^{m}\sum_{0\leq u\leq i+j}\left(
M_{q^{-1}}^{-+}\right) _{i,j,u}^{l,m}\left( \underline{x}\right) \left.\left(T^0_3\right)^{l,m}_uf \right| _{x^{0}\rightarrow x^{3}-\tilde{x}^{3}} 
\nonumber \\
&&\hspace{-2cm}+\,q^{-1}\frac{\beta_-}{\lambda_+}  \sum_{0\leq
l+m<\infty }\alpha _{-}^{l+m+1}\sum_{i=0}^{l+1}\sum_{j=0}^{m}\sum_{0\leq
u\leq i+j}\left( M_{q^{-1}}^{-+}\right) _{i,j,u}^{l+1,m}\left( \underline{x}%
\right)\left.\left(T^0_4\right)^{l,m}_uf 
\right| _{x^{0}\rightarrow x^{3}-\tilde{x}%
^{3}}  \nonumber \\
&&\hspace{-2cm}-\,q^{-1}\lambda _{+}^{-1}\sum_{0\leq l+m<\infty }\alpha
_{-}^{l+m+1}\sum_{i=0}^{l}\sum_{j=0}^{m+1}\sum_{0\leq u\leq i+j}\left(
M_{q^{-1}}^{-+}\right) _{i,j,u}^{l,m+1}\left( \underline{x}\right) 
\left. \left(T^0_5\right)^{l,m}_uf \right| _{x^{0}\rightarrow x^{3}-\tilde{x}^{3}}, 
\nonumber \\[0.16in]
\left( \partial _{\left( i=0\right) }^{-}\right) _{L}^{-1}f &=&-q\left(
D_{q^{-2}}^{+}\right) ^{-1}f, \\
\left( \partial _{\left( i>0\right) }^{-}\right) _{L}f &=&\nonumber\\
&&\hspace{-2cm}\lambda
\sum_{l=0}^{\infty }\alpha _{-}^{l}\sum_{0\leq i+j\leq l}\left.\left\{ \left(
M^{-}\right) _{i,j}^{l}\left( \underline{x}\right)\left(T^-_1\right)^l_jf 
 +\left( M^{+}\right) _{i,j}^{l}\left( \underline{x}%
\right) \left(T^-_2\right)^l_jf\right\} 
\right| _{x^{0}\rightarrow x^{3}-\tilde{x}^{3}}
\nonumber \\
&&\hspace{-2cm}+\,q^{2}\lambda \sum_{l=0}^{\infty }\alpha
_{-}^{l+1}\sum_{0\leq i+j\leq l+1}\left( M^{+}\right) _{i,j}^{l+1}\left( 
\underline{x}\right) \left.\left(T^-_3\right)^l_jf
\right| _{x^{0}\rightarrow x^{3}-\tilde{x}^{3}} 
\nonumber \\
&&\hspace{-2cm}+\,\frac{\lambda }{\lambda _{+}}\sum_{0\leq l+m<\infty
}\alpha _{-}^{l+m}\sum_{i=0}^{l}\sum_{j=0}^{m}\sum_{0\leq u\leq i+j}\bigg \{ %
q^{-1}\left( M_{q}^{-+}\right) _{i,j,u}^{l,m}\left( \underline{x}\right) 
\left(T^-_4\right)^{l,m}_uf\\ 
&&\hspace{4cm} \left.+\,\lambda \left( M_{q^{-1}}^{-+}\right)
  _{i,j,u}^{l,m}
\left( \underline{x}\right) \left(T^-_5\right)^{l,m}_uf \bigg \}
\right| _{x^{0}\rightarrow x^{3}-\tilde{x}^{3}} 
\nonumber \\
&&\hspace{-2cm}+\,\beta_-\frac{\lambda }{\lambda_+}
\sum_{0\leq l+m<\infty }\alpha
_{-}^{l+m+1}\sum_{i=0}^{l+1}\sum_{j=0}^{m}\sum_{0\leq u\leq i+j}\left(
M_{q^{-1}}^{-+}\right) _{i,j,u}^{l+1,m}\left( \underline{x}\right)
\left. \left(T^-_6\right)^{l,m}_uf
\right| _{x^{0}\rightarrow x^{3}-\tilde{x}%
^{3}}  \nonumber \\
&&\hspace{-2cm}+\,\frac{\lambda }{\lambda _{+}}\sum_{0\leq l+m<\infty
}\alpha _{-}^{l+m+1}\sum_{i=0}^{l}\sum_{j=0}^{m+1}\sum_{0\leq u\leq i+j}%
\bigg \{ \left( M_{q^{-1}}^{-+}\right) _{i,j,u}^{l,m+1}\left( \underline{x}%
\right) \left(T^-_7\right)_u^{l,m}f \nonumber \\
&&\hspace{4cm}\left. +\,q\left( M_{q^{-1}}^{-+}\right)
_{i,j,u}^{l,m+1}\left( \underline{x}\right)\left(T^-_8\right)^{l,m}_uf
\bigg \}
\right| _{x^{0}\rightarrow x^{3}-\tilde{x}^{3}}   \nonumber
\end{eqnarray}
where 
\begin{eqnarray}
\left(\tilde{T}^3_1\right)_j^lf&=&\left[ \left. \left( \tilde{O}_{1}^{3}\right) _{l}f\right|
_{x^{3}\rightarrow y_{+}}\right] \left( q^{2\left( j-1\right) }\tilde{x}%
^{3}\right),\\
\left(\tilde{T}^3_2\right)_j^lf&=&\left[ \left. \left( \tilde{O}_{2}^{3}\right) _{l}f\right|
_{x^{3}\rightarrow x^{0}+\tilde{x}^{3}}-q^{-1}\lambda \left. \left( \tilde{O}%
_{3}^{3}\right) _{l}f\right| _{x^{3}\rightarrow y_{+}}\right] \left( q^{2j}%
\tilde{x}^{3}\right) ,\nonumber\\[.16in]
\left(\tilde{T}^+_1\right)_j^lf&=&\left[ \left. \left( O_{1}^{+}\right) _{l}f\right| _{x^{3}\rightarrow
y_{+}}\right] \left( q^{2\left( j-1\right) }\tilde{x}^{3}\right) ,\\
\left(T^+_2\right)^l_jf&=&\left[ \left. \left(
O_{2}^{+}\right) _{l}f\right| _{x^{3}\rightarrow x^{0}+\tilde{x}^{3}}\right]
\left( q^{2j}\tilde{x}^{3}\right) ,\nonumber\\[.16in]
\left(T^0_1\right)^l_jf&=&\left[ \left. \left( O_{1}^{0}\right) _{l}f\right| _{x^{3}\rightarrow
y_{-}}\right] \left( q^{2j}\tilde{x}^{3}\right),\\
\left(T^0_2\right)^l_jf&=&\left[ \left. \left(
O_{2}^{0}\right) _{l}f\right| _{x^{3}\rightarrow x^{0}+\tilde{x}%
^{3}}-q^{-1}\lambda \left. \left( O_{3}^{0}\right) _{l}f\right|
_{x^{3}\rightarrow y_{+}}\right] \left(
q^{2j}\tilde{x}^{3}\right),\nonumber\\
\left(T^0_3\right)^{l,m}_uf&=&\left[ q^{-2}\left. \left( Q_{1}^{0}\right)
_{l,m}f\right| _{x^{3}\rightarrow x^{0}+\tilde{x}^{3}}-\left. \left(
Q_{2}^{0}\right) _{l,m}f\right| _{x^{3}\rightarrow y_{+}}\right] \left(
q^{2u}\tilde{x}^{3}\right),\nonumber\\
\left(T^0_4\right)^{l,m}_uf&=&\left[ \left. \left( Q_{3}^{0}\right)
_{l,m}f\right| _{x^{3}\rightarrow x^{0}+\tilde{x}^{3}}-q^{-1}\lambda \left.
\left( Q_{4}^{0}\right) _{l,m}f\right| _{x^{3}\rightarrow y_{+}}\right]
\left( q^{2u}\tilde{x}^{3}\right),\nonumber\\
\left(T^0_5\right)^{l,m}_uf&=&\left[ \left. \left( Q_{5}^{0}\right)
_{l,m}f\right| _{x^{3}\rightarrow x^{0}+\tilde{x}^{3}}\right] \left( q^{2u}%
\tilde{x}^{3}\right),\nonumber\\[.16in]
\left(T^-_1\right)^l_jf&=&\left[ \left.
\left( O_{1}^{-}\right) _{l}f\right| _{x^{3}\rightarrow y_{-}}\right] \left(
q^{2j}\tilde{x}^{3}\right),\\
\left(T^-_2\right)^l_jf&=&\left[ \left. \left( O_{2}^{-}\right) _{l}f\right|
_{x^{3}\rightarrow x^{0}+\tilde{x}^{3}}+\left. \left( O_{3}^{-}\right)
_{l}f\right| _{x^{3}\rightarrow y_{+}}\right] \left( q^{2j}\tilde{x}%
^{3}\right),\nonumber\\
\left(T^-_3\right)_j^lf&=& \left[ \left. \left( O_{4}^{-}\right)
_{l}f\right| _{x^{3}\rightarrow x^{0}+\tilde{x}^{3}}\right] \left( q^{2j}%
\tilde{x}^{3}\right), \nonumber\\
\left(T_4^-\right)^{l,m}_uf&=&\left[ \left. \left( Q_{1}^{-}\right)
_{l,m}f\right| _{x^{3}\rightarrow x^{0}+\tilde{x}^{3}}-\lambda \left. \left(
Q_{2}^{-}\right) _{l,m}f\right| _{x^{3}\rightarrow y_{+}}\right] \left(
q^{2\left( u+1\right) }\tilde{x}^{3}\right),\nonumber\\
\left(T^-_5\right)^{l,m}_uf&=&\left[ q^{-2}\left. \left( Q_{3}^{-}\right)
_{l,m}f\right| _{x^{3}\rightarrow
x^{0}+\tilde{x}^{3}}
-\left. \left( Q_{4}^{-}\right)
_{l,m}f\right| _{x^{3}\rightarrow y_{+}}\right] \left( q^{2u}\tilde{x}%
^{3}\right),\nonumber\\
\left(T^-_6\right)^{l,m}_uf&=& \left[ \left. \left( Q_{5}^{-}\right)
_{l,m}f\right| _{x^{3}\rightarrow x^{0}+\tilde{x}^{3}}-q^{-1}\lambda \left.
\left( Q_{6}^{-}\right) _{l,m}f\right| _{x^{3}\rightarrow y_{+}}\right]
\left( q^{2u}\tilde{x}^{3}\right),\nonumber\\
\left(T^-_7\right)^{l,m}_uf&=&\left[ \left. \left( Q_{7}^{-}\right)
_{l,m}f\right| _{x^{3}\rightarrow x^{0}+\tilde{x}^{3}}\right] \left(
q^{2\left( u+1\right) }\tilde{x}^{3}\right),\nonumber\\
\left(T^-_8\right)^{l,m}_uf&=&\left[ \left. \left(
Q_{8}^{-}\right) _{l,m}f\right| _{x^{3}\rightarrow x^{0}+\tilde{x}%
^{3}}\right] \left( q^{2u}\tilde{x}^{3}\right).\nonumber
\end{eqnarray} 
The operators $\left( O_{i}^{\mu}\right) _{l}$ and $\left( Q_{i}^{\mu}\right)
_{l,m}$, $\mu=\pm ,\tilde{3},0,$ as well as the polynomials $\left( M^{\pm
}\right) _{i,j}^{l}\left( \underline{x}\right) $ and $\left( M_{q^{\pm
}}^{-+}\right) _{i,j,u}^{l,m}\left( \underline{x}\right) $ have already been
introduced in \cite{BW01}. Their explicit form is once again listed in
appendix \ref{AppA}. In addition, we have used the abbreviations 
\begin{eqnarray}
\alpha _{\pm }=-q^{\pm 2}\frac{\lambda ^{2}}{\lambda _{+}^{2}},&&\qquad q_{\pm
}=1\pm \frac{\lambda }{\lambda _{+}}\frac{\tilde{x}^{3}}{x^{3}},\\
\beta_\pm=\left(q^\pm +\lambda_+\right),&&\qquad
y_{\pm }=x^{0}+\frac{2}{\lambda_+}q^{\pm 1}\tilde{x}^{3}.\nonumber
\end{eqnarray}
The derivative operator $\left( D_{2,q^{-1}}^{3}\right) ^{1,1}$appearing in
the expression for $\left( \tilde{\partial}_{\left( i>0\right) }^{3}\right)
_{L}f$ has also been defined in \cite{BW01}. It is important to note 
that the operators $\partial^\mu_{(i=0)}$ and
$\left(\partial^\mu_{(i>0)}\right)^{-1}$ have to be applied to functions
which can explicitly depend on the coordinates $x^\pm, \tilde{x}^3$ and
$x^3$ only.

For the second differential calculus we have to insert the following
expressions into formula (\ref{IntegralM}): 
\begin{eqnarray}
\left( \hat{\tilde{\partial}^{3}}_{\left( i=0\right) }\right) _{L}^{-1}f
&=&\left( D_{q_{-}q^{2}}^{3}\right) ^{-1}f\left( q^{-2}x^{+}\right) ,
\label{DerMin2} \\
\left( \hat{\tilde{\partial}^{3}}_{\left( i>0\right) }\right) _{L}f
&=&\sum_{l=1}^{\infty }\alpha _{+}^{l}\sum_{0\leq i+j\leq l}
\left. \left(M^{-}\right) _{i,j}^{l}\left( \underline{x}\right) 
\left(\hat{\tilde{T}^3}\right)^l_jf \right|
_{x^{0}\rightarrow x^{3}-\tilde{x}^{3}},  \nonumber \\[0.16in]
\left( \hat{\partial}_{\left( i=0\right) }^{-}\right) _{L}^{-1}f &=&-q\left(
D_{q^{2}}^{+}\right) ^{-1}f, \\
\left( \hat{\partial}_{\left( i>0\right) }^{-}\right) _{L}f &=&\\ \nonumber
&&\hspace{-2.2cm}\frac{\lambda 
}{\lambda _{+}}\sum_{l=0}^{\infty }\alpha _{+}^{l}\sum_{0\leq i+j\leq l}%
\left. \left\{ \left( M^{+}\right) _{i,j}^{l}\left( \underline{x}\right) 
\left(\hat{T}^-_1\right)^l_jf+q^{-1}\left( M^{-}\right) _{i,j}^{l}\left( 
\underline{x}\right)\left(\hat{T}^-_2\right)^l_jf\right\}
 \right| _{x^{0}\rightarrow x^{3}-\tilde{x}^{3}}
,  \nonumber \\[0.16in]
\left( \hat{\partial}_{\left( i=0\right) }^{0}\right) _{L}^{-1}f &=&\left( 
\tilde{D}_{q^{2}}^{3}\right) ^{-1}f\left( q_{-}x^{3}\right) , \\
\left( \hat{\partial}_{\left( i>0\right) }^{0}\right) _{L}f &=&-q\frac{%
\lambda }{\lambda _{+}}\left\{ qx^{-}D_{q^{-2}}^{-}\left(
D_{q^{-2}}^{3}f\right) \left( q^{2}\tilde{x}^{3},q^{2}q_{+}x^{3}\right)
+qx^{+}D_{q^{2}}^{+}D_{q^{2}q_{-}}^{3}f\right\}  \nonumber \\
&&\hspace{-2.2cm}-\,q\frac{\lambda }{\lambda _{+}}\left\{ \tilde{x}%
^{3}D_{q^{2}}^{+}D_{q^{2}}^{-}f\left( q^{2}q_{-}x^{3}\right) +q^{2}\lambda
x^{+}\tilde{x}^{3}D_{q^{2}}^{+}\left( \tilde{D}_{q^{2}}^{3}D_{q^{2}}^{3}f%
\right) \left( q_{+}x^{3}\right) \right\}  \nonumber \\
&&\hspace{-2.2cm}+\,\sum_{l=1}^{\infty }\alpha _{+}^{l}\sum_{0\leq i+j\leq l}%
\left. \left\{ \left( M^{+}\right) _{i,j}^{l}\left( \underline{x}\right)  
\left(\hat{T}^0_1\right)^l_jf
+q\frac{\lambda }{\lambda _{+}}\left( M^{-}\right)
_{i,j}^{l}\left( \underline{x}\right)\left(\hat{T}^0_2\right)^l_jf
\right\} \right|
_{x^{0}\rightarrow x^{3}-\tilde{x}^{3}}  \nonumber \\
&&\hspace*{-2.2cm}-\,q^{2}\frac{\lambda }{\lambda _{+}}\sum_{0\leq l+m<\infty
}\alpha _{+}^{l+m}\sum_{i=0}^{l}\sum_{j=0}^{m}\sum_{0\leq u\leq i+j}
\left.\left(
M_{q^{-1}}^{+-}\right) _{i,j,u}^{l,m}\left( \underline{x}\right) 
\left(\hat{T}^0_3\right)_u^{l,m}f
\right| _{x^{0}\rightarrow x^{3}-\tilde{x}^{3}} 
\nonumber \\
&&\hspace*{-2.2cm}+\,\frac{\beta_{+}}{\lambda_+}\sum_{0\leq l+m<\infty
}\alpha _{+}^{l+m+1}\sum_{i=0}^{l+1}\sum_{j=0}^{m}\sum_{0\leq u\leq
i+j}\left( M_{q^{-1}}^{+-}\right) _{i,j,u}^{l+1,m}\left( \underline{x}\right)
\left. \left(\hat{T}^0_4\right)^{l,m}_uf
\right| _{x^{0}\rightarrow x^{3}-\tilde{x}^{3}} 
\nonumber \\
&&\hspace*{-2.2cm}+\,\lambda _{+}^{-1}\sum_{0\leq l+m<\infty }\alpha
_{+}^{l+m+1}\sum_{i=0}^{l}\sum_{j=0}^{m+1}\sum_{0\leq u\leq i+j}\left(
\left. M_{q^{-1}}^{+-}\right) _{i,j,u}^{l,m+1}\left( \underline{x}\right) 
\left(\hat{T}^0_5\right)_u^{l,m}f
\right| _{x^{0}\rightarrow x^{3}-\tilde{x}^{3}}, 
\nonumber \\[0.16in]
\left( \hat{\partial}_{\left( i=0\right) }^{+}\right) _{L}^{-1}f
&=&-q^{-1}\left( D_{q^{2}}^{-}\right) ^{-1}f\left( q^{-2}q_{+}x^{3}\right) ,
\\
\left( \hat{\partial}_{\left( i>0\right) }^{+}\right) _{L}f
&=&-q\lambda x^{+}\left( \tilde{D}_{q^{2}}^{3}D_{q^{2}}^{3}f\right)
^{-1}f\left( q_{+}x^{3}\right)  \nonumber \\
&&\hspace*{-2.2cm}-\,q\sum_{l=1}^{\infty }\alpha _{+}^{l}\sum_{0\leq i+j\leq
l}\left.\left\{ \left( M^{-}\right) _{i,j}^{l}\left( \underline{x}\right) 
\left(\hat{T}^+_1\right)_u^{l,m}f
+\lambda \left( M^{+}\right) _{i,j}^{l}\left( 
\underline{x}\right)\left(\hat{T}^+_2\right)_u^{l,m}f \right\}
\right| _{x^{0}\rightarrow x^{3}-\tilde{x}^{3}}
\nonumber \\
&&\hspace*{-2.2cm}-\,q\frac{\lambda }{\lambda _{+}}\sum_{0\leq l+m<\infty
}\alpha _{+}^{l+m}\sum_{i=0}^{l}\sum_{j=0}^{m}\sum_{0\leq u\leq  i+j}\left.\left(
M_{q^{-1}}^{+-}\right) _{i,j,u}^{l,m}\left( \underline{x}\right) 
\left(\hat{T}_3^+\right)^{l,m}_uf
\right| _{x^{0}\rightarrow x^{3}-\tilde{x}^{3}} 
\nonumber \\
&&\hspace*{-2.2cm}-\,\frac{\lambda }{\lambda _{+}}\sum_{0\leq l+m<\infty
}\alpha _{+}^{l+m+1}\sum_{i=0}^{l}\sum_{j=0}^{m+1}\sum_{0\leq u\leq
i+j}\left.\left( M_{q^{-1}}^{+-}\right) _{i,j,u}^{l,m+1}\left( \underline{x}\right)
\left(\hat{T}^+_4\right)^{l,m}_uf
\right| _{x^{0}\rightarrow x^{3}-\tilde{x}^{3}} 
\nonumber
\end{eqnarray} 
where
\begin{eqnarray}
\left(\hat{\tilde{T}^3}\right)^l_jf&=& \left[ \left.
\left( \hat{\tilde{O}^{3}}\right) _{l}f\right| _{x^{3}\rightarrow x^{0}+%
\tilde{x}^{3}}\right] \left( q^{2j}\tilde{x}^{3}\right),\\[0.16in]
\left(\hat{T}_1^-\right)^l_jf&=&\left[ \left. \left( \hat{O}_{1}^{-}\right)
_{l}f\right| _{x^{3}\rightarrow x^{0}+\tilde{x}^{3}}\right] \left(
q^{2\left( j+1\right) }\tilde{x}^{3}\right),\\
\left(\hat{T}^-_2\right)^l_jf&=&\left[ \left. \left( \hat{O}_{2}^{-}\right)
_{l}f\right| _{x^{3}\rightarrow x^{0}+\tilde{x}^{3}}\right] \left( q^{2j}%
\tilde{x}^{3}\right), \nonumber\\[0.16in]
\left(\hat{T}^0_1\right)^l_jf&=&\left[ \left. \left( \hat{O}_{1}^{0}\right)
_{l}f\right| _{x^{3}\rightarrow y_{+}}-q^{2}\frac{\lambda }{\lambda _{+}}%
\left. \left( \hat{O}_{2}^{0}\right) _{l}f\right| _{x^{3}\rightarrow
q^{2}y^{+}}\right] \left( q^{2j}\tilde{x}^{3}\right),\\
\left(\hat{T}^0_2\right)^l_jf&=&\left[ \left. \left( \hat{O}%
_{3}^{0}\right) _{l}f\right| _{x^{3}\rightarrow x^{0}+\tilde{x}^{3}}+\left.
\left( \hat{O}_{4}^{0}\right) _{l}f\right| _{x^{3}\rightarrow
q^{2}y_{-}}\right] \left( q^{2j}\tilde{x}^{3}\right),\nonumber\\
\left(\hat{T}^0_3\right)^{l,m}_uf&=&\left[ \left. \left( \hat{Q}_{1}^{0}\right)
_{l,m}f\right| _{x^{3}\rightarrow x^{0}+\tilde{x}^{3}}\right] \left( q^{2u}%
\tilde{x}^{3}\right),\nonumber\\
\left(\hat{T}^0_4\right)^{l,m}_uf&=&
\left[ \left. \left( \hat{Q}_{2}^{0}\right)
_{l,m}f\right| _{x^{3}\rightarrow x^{0}+\tilde{x}^{3}}\right] \left( q^{2u}%
\tilde{x}^{3}\right),\nonumber\\
\left(\hat{T}^0_5\right)^{l,m}_uf&=&
\left[ \left. \left( \hat{Q}_{3}^{0}\right)
_{l,m}f\right| _{x^{3}\rightarrow x^{0}+\tilde{x}^{3}}\right] \left( q^{2u}%
\tilde{x}^{3}\right),\nonumber \\[0.16in]
\left(\hat{T}^+_1\right)_j^lf&=&
\left[ \left. \left( \hat{O}_{1}^{+}\right) _{l}f\right| _{x^{3}\rightarrow
q^{2}y_{-}}\right] \left( q^{2j}\tilde{x}^{3}\right),\\
\left(\hat{T}^+_2\right)_j^lf&=&
\left[ \left. \left( \hat{O}_{2}^{+}\right)
_{l}f\right| _{x^{3}\rightarrow y_{+}}\right] \left( q^{2j}\tilde{x}%
^{3}\right),\nonumber\\
\left(\hat{T}^+_3\right)_u^{l,m}f&=&
 \left[ \left. \left( \hat{Q}_{1}^{+}\right)
_{l,m}f\right| _{x^{3}\rightarrow x^{0}+\tilde{x}^{3}}\right] \left( q^{2u}%
\tilde{x}^{3}\right),\nonumber\\
\left(\hat{T}^+_4\right)^{l,m}_uf&=&
\left[ \left. \left( \hat{Q}_{2}^{+}\right)
_{l,m}f\right| _{x^{3}\rightarrow x^{0}+\tilde{x}^{3}}\right] \left( q^{2u}%
\tilde{x}^{3}\right).\nonumber
\end{eqnarray}
And likewise for the algebra of momentum generators we have 
\begin{eqnarray}
-\frac{i}{\left[ 2\right] _{q^{2}}}\left( \tilde{P}_{\left( i=0\right)
}^{3}\right) _{L}^{-1}f &=&\left( \tilde{J}^{3}\right) ^{-1}\left(
D_{q^{2}}^{3}\right) ^{-1}f,  \label{DerMinP} \\
i\left[ 2\right] _{q^{2}}\left( \tilde{P}_{\left( i>0\right) }^{3}\right)
_{L}f &=&-q^{-3}\lambda x^{-}D_{q^{-2}}^{-}\left( D_{q^{-2}}^{3}f\right)
\left( q_{+}x^{3}\right)  \nonumber \\
&&\hspace*{-3.cm}-\,q^{-1}\lambda \left\{ q^{-1}\tilde{x}%
^{3}D_{q^{2}}^{+}D_{q^{-2}}^{-}f\left( q^{-2}\tilde{x}^{3},q_{-}x^{3}\right)
+\frac{\lambda }{\lambda _{+}}x^{+}\tilde{x}^{3}D_{q^{2}}^{+}\left(
D_{2,q^{-1}}^{3}\right) ^{1,1}f\right\}  \nonumber \\
&&\hspace*{-3.cm}+\,\sum_{l=1}^{\infty }\alpha _{0}^{l}\sum_{0\leq i+j\leq
l}\left. \left\{ \left( M^{+}\right) _{i,j}^{l}\left(\underline{x}\right)
\left(T^{P_3}_1\right)_j^lf 
+\left( M^{-}\right) _{i,j}^{l}\left( \underline{x%
}\right)
\left(T^{P_3}_2\right)_j^lf \right\}
\right| _{x^{0}\rightarrow
x^{3}-\tilde{x}^{3}} ,  \nonumber \\[0.16in]
-\frac{i}{\left[ 2\right] _{q^{2}}}\left( P_{\left( i=0\right) }^{+}\right)
_{L}^{-1}f &=&-q^{-1}\left( J^{+}\right) ^{-1}\left( D_{q^{2}}^{-}\right)
^{-1}f, \\
i\left[ 2\right] _{q^{2}}\left( P_{\left( i>0\right) }^{+}\right) _{L}f
&=&-\lambda x^{+}\left\{ q^{-2}\lambda _{+}^{-1}\left(
D_{2,q^{-1}}^{3}\right) ^{1,1}f+q^{3}\left( \tilde{D}%
_{q^{2}}^{3}D_{q^{2}}^{3}f\right) \left( q_{+}x^{3}\right) \right\} 
\nonumber \\
&&\hspace*{-3.cm}-\sum_{l=1}^{\infty }\alpha _{0}^{l}\sum_{0\leq i+j\leq
l}\left.\left\{ q\left( M^{-}\right) _{i,j}^{l}\left(\underline{x}\right)
\left(T^{P_+}_1\right)^l_jf 
+\lambda \left( M^{+}\right) _{i,j}^{l}\left( \underline{%
x}\right)\left(T^{P_+}_2\right)^l_jf
\right\}\right| _{x^{0}\rightarrow x^{3}-\tilde{x}%
^{3}}\nonumber \\
&&\hspace*{-3.cm}-\,q\frac{\lambda }{\lambda _{+}}\sum_{0\leq l+m<\infty
}\alpha _{0}^{l+m}\sum_{i=0}^{l}\sum_{j=0}^{m}\sum_{0\leq u\leq i+j}\left(
\left.M_{q^{-1}}^{+-}\right) _{i,j,u}^{l,m}\left( \underline{x}\right) 
\left(T^{P_+}_3\right)^{l,m}_uf
\right| _{x^{0}\rightarrow x^{3}-\tilde{x}^{3}} 
\nonumber \\
&&\hspace*{-3.cm}-\,\frac{\lambda }{\lambda _{+}}\sum_{0\leq l+m<\infty
}\alpha _{0}^{l+m+1}\sum_{i=0}^{l}\sum_{j=0}^{m+1}\sum_{0\leq u\leq
i+j}\left.\left( M_{q^{-1}}^{+-}\right) _{i,j,u}^{l,m+1}\left( \underline{x}\right)
\left(T^{P_+}_4\right)^{l,m}_uf 
 \right| _{x^{0}\rightarrow x^{3}-\tilde{x}^{3}}, 
\nonumber \\[0.16in]
-\frac{i}{\left[ 2\right] _{q^{2}}}\left( P_{\left( i=0\right) }^{0}\right)
_{L}^{-1}f &=&\left( J^{0}\right) ^{-1}\left( \tilde{D}_{q^{2}}^{3}\right)
^{-1}f, \\
i\left[ 2\right] _{q^{2}}\left( P_{\left( i>0\right) }^{0}\right) _{L}f
&=&-q^{-2}\lambda x^{3}D_{q^{-2}}^{+}D_{q^{-2}}^{-}f\left( q^{-2}\tilde{x}%
^{3},q_{-}x^{3}\right)  \nonumber \\
&&\hspace*{-3.cm}-\,\frac{\lambda}{ \lambda _{+}}x^{+}D_{q^{2}}^{+}\left(
D_{q_{+}q^{-2}}^{3}f\left( q^{-2}x^{+}\right) +q^{4}D_{q_{-}q^{2}}f\right) 
\nonumber \\
&&\hspace*{-3.cm}-\,q^{4}\frac{\lambda}{\lambda_{+}} x^{-}D_{q^{-2}}^{-}\left(
D_{q^{-2}}^{3}f\right) \left( q^{2}\tilde{x}^{3},q_{+}q^{2}x^{3}\right) 
\nonumber \\
&&\hspace*{-3.cm}-\,q^{3}\frac{\lambda}{\lambda_{+}} \tilde{x}%
^{3}D_{q^{2}}^{+}D_{q^{2}}^{-}\left( f\left( q_{-}q^{2}x^{3}\right) -f\left(
q^{-2}x^{+},q^{-2}\tilde{x}^{3},q_{-}x^{3},q^{-2}x^{-}\right) \right) 
\nonumber \\
&&\hspace*{-3.cm}+\,q^{-3}\lambda _{+}^{-1}\lambda
^{2}x^{+}x^{-}D_{q^{-2}}^{+}D_{q^{-2}}^{-}\left( D_{q^{-2}}^{3}f\right)
\left( q_{+}x^{3}\right)  \nonumber \\
&&\hspace*{-3.cm}+\,q^{-3}\lambda _{+}^{-1}\lambda ^{2}x^{+}\left(
x^{3}-q^{-1}\lambda _{+}^{-1}\tilde{x}^{3}\right) D_{q^{-2}}^{+}\left(
D_{2,q^{-1}}^{3}\right) ^{1,1}f  \nonumber \\
&&\hspace*{-3.cm}-\,q^{5}\lambda _{+}^{-1}\lambda ^{2}x^{+}\tilde{x}%
^{3}D_{q^{2}}^{+}\left( \tilde{D}_{q^{2}}^{3}D_{q^{2}}^{3}f\right) \left(
q_{+}x^{3}\right)  \nonumber \\
&&\hspace*{-3.cm}+\,\sum_{l=1}^{\infty }\alpha _{0}^{l}\sum_{0\leq i+j\leq
l}\left.\left\{ \left( M^{-}\right) _{i,j}^{l}\left( \underline{x}\right) 
\left(T^{P_0}_1\right)_j^lf
+\left( M^{+}\right) _{i,j}^{l}\left( \underline{x%
}\right) \left(T^{P_0}_2\right)^l_jf
 \right\}\right| _{x^{0}\rightarrow
x^{3}-\tilde{x}^{3}}  \nonumber \\
&&\hspace*{-3.cm}-\,\frac{\lambda }{\lambda _{+}}\sum_{0\leq l+m<\infty
}\alpha _{0}^{l+m}\sum_{i=0}^{l}\sum_{j=0}^{m}\sum_{0\leq u\leq i+j}\left.\left(
M_{q^{-1}}^{-+}\right) _{i,j,u}^{l,m}\left( \underline{x}\right) 
\left(T^{P_0}_3\right)^{l,m}_uf
\right| _{x^{0}\rightarrow x^{3}-\tilde{x}^{3}} 
\nonumber \\
&&\hspace*{-3.cm}+\,\sum_{0\leq l+m<\infty }\alpha
_{0}^{l+m+1}\sum_{i=0}^{l+1}\sum_{j=0}^{m}\sum_{0\leq u\leq i+j}\left.\left(
M_{q^{-1}}^{-+}\right) _{i,j,u}^{l+1,m}\left( \underline{x}\right)
\left(T^{P_0}_4\right)^{l,m}_uf
\right| _{x^{0}\rightarrow x^{3}-\tilde{x}^{3}}  \nonumber \\
&&\hspace*{-3.cm}-\,\sum_{0\leq l+m<\infty }\alpha
_{0}^{l+m+1}\sum_{i=0}^{l}\sum_{j=0}^{m+1}\sum_{0\leq u\leq i+j}\left.\left(
M_{q^{-1}}^{-+}\right) _{i,j,u}^{l,m+1}\left( \underline{x}\right)
\left(T^{P_0}_5\right)_u^{l,m}f 
\right| _{x^{0}\rightarrow x^{3}-\tilde{x}^{3}}, 
\nonumber \\[0.16in]
-\frac{i}{\left[ 2\right] _{q^{2}}}\left( P_{\left( i=0\right) }^{-}\right)
_{L}^{-1}f &=&-q\left( J^{-}\right) ^{-1}\left( D_{q^{2}}^{+}\right) ^{-1}f,
\\
i\left[ 2\right] _{q^{2}}\left( P_{\left( i>0\right) }^{-}\right) _{L}f
&=&\\&&\hspace*{-3.cm}
\lambda \sum_{l=0}^{\infty }\alpha _{0}^{l}\sum_{0\leq i+j\leq l} %
\left.\left\{\left( M^{-}\right) _{i,j}^{l}\left( \underline{x}\right) 
\left(T^{P_-}_1\right)^l_jf
+\left( M^{+}\right) _{i,j}^{l}\left( \underline{x}%
\right)\left(T^{P_-}_2\right)^l_jf  \right\}
\right| _{x^{0}\rightarrow x^{3}-\tilde{x}^{3}}
\nonumber \\
&&\hspace*{-3.cm}+\,q^{2}\lambda \sum_{l=0}^{\infty }\alpha
_{0}^{l+1}\sum_{0\leq i+j\leq l+1}\left.\left( M^{+}\right) _{i,j}^{l+1}\left( 
\underline{x}\right)\left(T^{P_-}_3\right)^l_jf
\right| _{x^{0}\rightarrow x^{3}-\tilde{x}^{3}} 
\nonumber \\
&&\hspace*{-3.cm}+\,\frac{\lambda }{\lambda _{+}}\sum_{0\leq l+m<\infty
}\alpha _{0}^{l+m}\sum_{i=0}^{l}\sum_{j=0}^{m}\sum_{0\leq u\leq i+j}
\left\{ %
q^{-1}\left( M_{q}^{-+}\right) _{i,j,u}^{l,m}\left(\underline{x}\right) 
\left(T^{P_-}_4\right)^{l,m}_uf\right.\nonumber\\ 
&&\hspace*{3.cm}+\left.\left.\,\lambda \left( M_{q^{-1}}^{-+}\right)
_{i,j,u}^{l,m}\left( \underline{x}\right)
\left(T^{P_-}_5\right)^{l,m}_uf \right\}
\right| _{x^{0}\rightarrow x^{3}-\tilde{x}^{3}} 
\nonumber \\
&&\hspace*{-3.cm}+\,\beta_-\frac{\lambda }{\lambda _{+}}
 \sum_{0\leq l+m<\infty }\alpha
_{0}^{l+m+1}\sum_{i=0}^{l+1}\sum_{j=0}^{m}\sum_{0\leq u\leq i+j}\left.\left(
M_{q^{-1}}^{-+}\right) _{i,j,u}^{l+1,m}\left( \underline{x}\right)   
\left(T^{P_-}_6\right)^{l,m}_uf
\right| _{x^{0}\rightarrow x^{3}-\tilde{x}%
^{3}}  \nonumber \\
&&\hspace*{-3.cm}+\,\frac{\lambda }{\lambda _{+}}\sum_{0\leq l+m<\infty
}\alpha _{0}^{l+m+1}\sum_{i=0}^{l}\sum_{j=0}^{m+1}\sum_{0\leq u\leq i+j}%
\left\{ \left( M_{q}^{-+}\right) _{i,j,u}^{l,m+1}
\left( \underline{x}\right)\left(T^{P_-}_7\right)^{l,m}_uf\right.\nonumber\\
&&\hspace*{3cm}\left.\left.+\,q\left( M_{q^{-1}}^{-+}\right)
_{i,j,u}^{l,m+1}\left( \underline{x}\right) 
\left(T^{P_-}_8\right)^{l,m}_uf\right\}
\right| _{x^{0}\rightarrow
x^{3}-\tilde{x}^{3}}   \nonumber
\end{eqnarray}
where
\begin{eqnarray}
\left(T^{P_3}_1\right)_j^lf&=&
\left[ \left. \left( \tilde{O}%
_{1}^{P_{3}}\right) _{l}f\right| _{x^{3}\rightarrow x^{0}+\tilde{x}%
^{3}}-q^{-1}\lambda \left. \left( \tilde{O}_{2}^{P_{3}}\right) _{L}f\right|
_{x^{3}\rightarrow y_{+}}\right] \left( q^{2j}\tilde{x}^{3}\right),\\
\left(T^{P_3}_2\right)_j^lf&=&
\left[ \left. \left( \tilde{O}_{3}^{P_{3}}\right)
_{l}f\right| _{x^{3}\rightarrow x^{0}+\tilde{x}^{3}}-q^{2}\lambda \left.
\left( \tilde{O}_{4}^{P_{3}}\right) _{L}f\right| _{x^{3}\rightarrow
y_{-}}\right] \left( q^{2j}\tilde{x}^{3}\right),\nonumber\\[0.16in]
\left(T^{P_+}_1\right)_j^lf&=&
\left[ \left. \left( O_{1}^{P_{+}}\right) _{l}f\right| _{x^{3}\rightarrow
y_{-}}\right] \left( q^{2j}\tilde{x}^{3}\right),\\
\left(T^{P_+}_2\right)_j^lf&=&
\left[ \lambda _{+}^{-1}\left. \left( O_{2}^{P_{+}}\right)
_{l}f\right| _{x^{3}\rightarrow x^{0}+\tilde{x}^{3}} 
+ q\left. \left(
O_{3}^{P_{+}}\right) _{L}f\right| _{x^{3}\rightarrow y_{+}}\right] \left(
q^{2j}\tilde{x}^{3}\right),\nonumber\\
\left(T^{P_+}_3\right)^{l,m}_uf&=&
\left[ \left. \left( Q_{1}^{P_{+}}\right)
_{l,m}f\right| _{x^{3}\rightarrow x^{0}+\tilde{x}^{3}}\right] \left( q^{2u}%
\tilde{x}^{3}\right),\nonumber\\
\left(T^{P_+}_4\right)^{l,m}_uf&=&
\left[ \left. \left( Q_{2}^{P_{+}}\right)
_{l,m}f\right| _{x^{3}\rightarrow x^{0}+\tilde{x}^{3}}\right] \left( q^{2u}%
\tilde{x}^{3}\right),\nonumber\\[0.16in]
\left(T^{P_0}_1\right)_j^lf&=&
\left[ \left. \left( O_{1}^{P_{0}}\right)
_{l}f\right| _{x^{3}\rightarrow y_{-}}-q\frac{\lambda }{\lambda _{+}}\left.
\left( O_{2}^{P_{0}}\right) _{l}f\right| _{x^{3}\rightarrow x^{0}+\tilde{x}%
^{3}}\right] \left( q^{2j}\tilde{x}^{3}\right),\\
\left(T^{P_0}_2\right)^l_jf&=&
\left[ \left. \left( O_{3}^{P_{0}}\right) _{l}f\right|
_{x^{3}\rightarrow y_{+}}-\frac{\lambda }{\lambda _{+}}\left. \left(
O_{4}^{P_{0}}\right) _{L}f\right| _{x^{3}\rightarrow x^{0}+\tilde{x}%
^{3}}\right] \left( q^{2j}\tilde{x}^{3}\right),\nonumber\\
\left(T^{P_0}_3\right)^{l,m}_uf&=&
 \left[ \left. \left( Q_{1}^{P_{0}}\right)
_{l,m}f\right| _{x^{3}\rightarrow y_{+}}-\left. \left( Q_{2}^{P_{0}}\right)
_{l,m}f\right| _{x^{3}\rightarrow x^{0}+\tilde{x}^{3}}\right] \left( q^{2u}%
\tilde{x}^{3}\right),\nonumber\\
\left(T^{P_0}_4\right)^{l,m}_uf&=&
 \left[
\left. \left( Q_{3}^{P_{0}}\right) _{l,m}f\right| _{x^{3}\rightarrow x^{0}+%
\tilde{x}^{3}} 
-q^{-1}\beta_-\frac{\lambda}{\lambda_+}
\left. \left( Q_{4}^{P_{0}}\right)
_{l,m}f\right| _{x^{3}\rightarrow y_{+}}\right]\left(q^{2u}\tilde{x}^3\right),\nonumber\\
\left(T^{P_0}_5\right)^{l,m}_uf&=&
\left[ \left. \left( Q_{5}^{P_{0}}\right)
_{l,m}f\right| _{x^{3}\rightarrow x^{0}+\tilde{x}^{3}}\right] \left( q^{2u}%
\tilde{x}^{3}\right),\nonumber\\[0.16in]
\left(T^{P_-}_1\right)^l_jf&=&
\left[ q^{-1}\lambda _{+}^{-1}\left. \left(
O_{1}^{P_{-}}\right) _{l}f\right| _{x^{3}\rightarrow x^{0}+\tilde{x}%
^{3}}+\left. \left( O_{2}^{P_{-}}\right) _{l}f\right| _{x^{3}\rightarrow
y_{-}}\right] \left( q^{2j}\tilde{x}^{3}\right),\\
\left(T^{P_-}_2\right)^l_uf&=&
\left[ \left. \left( O_{3}^{P_{-}}\right) _{l}f\right|
_{x^{3}\rightarrow x^{0}+\tilde{x}^{3}}
+\left. \left( O_{4}^{P_{-}}\right)
_{l}f\right| _{x^{3}\rightarrow y_{+}} \right.\nonumber\\
&&+\,\left.\left. \left( O_{5}^{P_{-}}\right)
f\right| _{x^{3}\rightarrow x^{0}+q^{2}\tilde{x}^{3}}\right] \left( q^{2j}%
\tilde{x}^{3}\right),\nonumber\\
\left(T^{P_-}_3\right)^l_jf&=&
\left[ \left. \left( O_{6}^{P_{-}}\right)
_{l}f\right| _{x^{3}\rightarrow x^{0}+\tilde{x}^{3}}\right] \left( q^{2j}%
\tilde{x}^{3}\right),\nonumber\\
\left(T^{P_-}_4\right)^{l,m}_uf&=&
\left[ \left. \left( Q_{1}^{P_{-}}\right)
_{l,m}f\right| _{x^{3}\rightarrow x^{0}+\tilde{x}^{3}}-\lambda \left. \left(
Q_{2}^{P_{-}}\right) _{l,m}f\right| _{x^{3}\rightarrow y_{+}}\right] \left(
q^{2\left( u+1\right) }\tilde{x}^{3}\right),\nonumber\\
\left(T^{P_-}_5\right)^{l,m}_uf&=&
\left[ \left. q^{-2}\left(
Q_{3}^{P_{-}}\right) _{l,m}f\right| _{x^{3}\rightarrow x^{0}+\tilde{x}%
^{3}}
- \left. \left( Q_{4}^{P_{-}}\right)
_{l,m}f\right| _{x^{3}\rightarrow y_{+}}\right] \left( q^{2u}\tilde{x}%
^{3}\right),\nonumber\\
\left(T^{P_-}_6\right)^{l,m}_uf&=&
\left[ \left. \left( Q_{5}^{P_{-}}\right)
_{l,m}f\right| _{x^{3}\rightarrow x^{0}+\tilde{x}^{3}}-q^{-1}\lambda \left.
\left( Q_{6}^{P_{-}}\right) _{l,m}f\right| _{x^{3}\rightarrow y_{+}}\right]
\left( q^{2u}\tilde{x}^{3}\right),\nonumber\\
\left(T^{P_-}_7\right)^{l,m}_uf&=&
\left[ \left. \left( Q_{7}^{P_{-}}\right)
_{l,m}f\right| _{x^{3}\rightarrow x^{0}+\tilde{x}^{3}}\right] \left(
q^{2\left( u+1\right) }\tilde{x}^{3}\right),\nonumber\\
\left(T^{P_-}_8\right)^{l,m}_uf&=&
 \left[ \left. \left(
Q_{8}^{P_{-}}\right) _{l,m}f\right| _{x^{3}\rightarrow x^{0}+\tilde{x}%
^{3}}\right] \left( q^{2u}\tilde{x}^{3}\right).\nonumber
\end{eqnarray}
The explicit form of the operators $\left( \hat{O}_{i}^{\mu}\right) _{l},$ $%
\left( \hat{Q}_{i}^{\mu}\right) _{l,m}$ and $\left( O_{i}^{P_{\mu}}\right) _{l},$
$\left( Q_{i}^{P_{\mu}}\right) _{l,m}$ can be found in appendix \ref{AppA},
whereas the scaling operators $\left( J^{\mu}\right) ^{-1}$ are defined in
appendix \ref{AppB}. For the purpose of abbreviation we have additionally
$\alpha _{0}=-\lambda ^{2}/\lambda _{+}^{2}$ introduced.

Again, left and right integrals are related to each other by the following
transitions: 
\begin{eqnarray}
\left( \partial ^{0}\right) _{R}^{-1}f&\underleftrightarrow&-q^{-4}\left( \hat{\partial}^{0}\right)
_{L}^{-1}f, \\
\left( \tilde{\partial}^{3}\right) _{R}^{-1}f&\underleftrightarrow&-q^{-4}\left( \hat{\tilde{\partial}^{3}}%
\right) _{L}^{-1}f,  \nonumber \\
\left( \partial ^{+}\right) _{R}^{-1}f&\underleftrightarrow&-q^{-4}\left( \hat{\partial}^{-}\right)
_{L}^{-1}f,  \nonumber \\
\left( \partial ^{-}\right) _{R}^{-1}f&\underleftrightarrow&-q^{-4}\left( \hat{\partial}^{+}\right)
_{L}^{-1}f.  \nonumber
\end{eqnarray}
Analogously, for the second differential calculus we have 
\begin{eqnarray}
\left( \hat{\partial}^{0}\right) _{R}^{-1}f&\underleftrightarrow&-q^{4}\left( \partial ^{0}\right) _{L}^{-1}f,\\
\left( \hat{\tilde{\partial}^{3}}\right) _{R}^{-1}f&\underleftrightarrow&-q^{4}\left( \tilde{%
\partial}^{3}\right) _{L}^{-1}f,  \nonumber \\
\left( \hat{\partial}^{+}\right) _{R}^{-1}f&\underleftrightarrow&-q^{4}\left( \partial ^{-}\right) _{L}^{-1}f,
\nonumber \\
\left( \hat{\partial}^{-}\right) _{R}^{-1}f&\underleftrightarrow&-q^{4}\left( \partial ^{+}\right) _{L}^{-1}f.
\nonumber
\end{eqnarray}
It remains to treat the case of the algebra of momentum generators for which
we find 
\begin{eqnarray}
\left( P^{0}\right) _{R}^{-1}f&\underleftrightarrow&-\left( P^{0}\right) _{L}^{-1}f, \\
\left( \tilde{P}^{3}\right) _{R}^{-1}f&\underleftrightarrow&-\left( \tilde{P}^{3}\right) _{L}^{-1}f, 
\nonumber \\
\left( P^{+}\right) _{R}^{-1}f&\underleftrightarrow&-\left( P^{-}\right) _{L}^{-1}f,  \nonumber \\
\left( P^{-}\right) _{R}^{-1}f&\underleftrightarrow&-\left( P^{+}\right) _{L}^{-1}f.  \nonumber
\end{eqnarray}

The rules for integration by parts in the case of
left integrals are given by
\begin{eqnarray}
\left. \left( \tilde{\partial}^{3}\right) _{L}^{-1}\left( \tilde{\partial}%
_{L}^{3}f\right) \star g\right| _{x^{3}=a}^{b} &=&\left. f\star g\right|
_{x^{3}=a}^{b}-\left( \tilde{\partial}^{3}\right) _{L}^{-1}\left( \Lambda
^{1/2}\tau ^{1}f\right) \star \left. \tilde{\partial}_{L}^{3}g\right|
_{x^{3}=a}^{b} \\
&&\hspace*{-2.9cm}+\,q^{1/2}\lambda _{+}^{1/2}\lambda \left( \tilde{\partial}%
^{3}\right) _{L}^{-1}\left( \Lambda ^{1/2}\left( \tau ^{3}\right)
^{-1/2}S^{1}f\right) \star \left. \partial _{L}^{+}g\right| _{x^{3}=a}^{b}, 
\nonumber \\
\left. \left( \partial ^{+}\right) _{L}^{-1}\left( \partial _{L}^{+}f\right)
\star g\right| _{x^{-}=a}^{b} &=&\left. f\star g\right|
_{x^{-}=a}^{b}-\left( \partial ^{+}\right) _{L}^{-1}\left( \Lambda
^{1/2}\left( \tau ^{3}\right) ^{-1/2}\sigma ^{2}f\right) \star \left.
\partial _{L}^{+}g\right| _{x^{-}=a}^{b}  \nonumber \\
&&\hspace*{-2.9cm}+\,q^{3/2}\lambda _{+}^{-1/2}\lambda \left( \partial
^{+}\right) _{L}^{-1}\left( \Lambda ^{1/2}T^{2}f\right) \star \left. \tilde{%
\partial}_{L}^{3}g\right| _{x^{-}=a}^{b},  \nonumber \\
\left. \left( \partial ^{-}\right) _{L}^{-1}\left( \partial _{L}^{-}f\right)
\star g\right| _{x^{+}=a}^{b} &=&\left. f\star g\right|
_{x^{+}=a}^{b}-\left( \partial ^{-}\right) _{L}^{-1}\left( \Lambda
^{1/2}\left( \tau ^{3}\right) ^{1/2}\tau ^{1}f\right) \star \left. \partial
_{L}^{-}g\right| _{x^{+}=a}^{b}  \nonumber \\
&&\hspace*{-2.9cm}+\,q^{-1/2}\lambda _{+}^{1/2}\lambda \left( \partial
^{-}\right) _{L}^{-1}\left( \Lambda ^{1/2}S^{1}f\right) \star \left.
\partial _{L}^{0}g\right| _{x^{+}=a}^{b}  \nonumber \\
&&\hspace*{-2.9cm}+\,\lambda ^{2}\left( \partial ^{-}\right) _{L}^{-1}\left(
\Lambda ^{1/2}\left( \tau ^{3}\right) ^{-1/2}T^{-}S^{1}f\right) \star \left.
\partial _{L}^{+}g\right| _{x^{+}=a}^{b}  \nonumber \\
&&\hspace*{-2.9cm}-\,q^{-1/2}\lambda _{+}^{-1/2}\lambda \left( \partial
^{-}\right) _{L}^{-1}\left( \Lambda ^{1/2}\left( \tau
^{1}T^{-}-q^{-1}S^{1}\right) f\right) \star \left. \tilde{\partial}%
_{L}^{3}g\right| _{x^{+}=a}^{b},  \nonumber \\
\left. \left( \partial ^{0}\right) _{L}^{-1}\left( \partial _{L}^{0}f\right)
\star g\right| _{\tilde{x}^{3}=a}^{b} &=&\left. f\star g\right| _{\tilde{x}
^{3}=a}^{b}-\left( \partial ^{0}\right) _{L}^{-1}\left( \Lambda ^{1/2}\sigma
^{2}f\right) \star \left. \partial _{L}^{0}g\right| _{\tilde{x}^{3}=a}^{b} 
\nonumber \\
&&\hspace*{-2.9cm}+\,q^{1/2}\lambda _{+}^{-1/2}\lambda \left( \partial
^{0}\right) _{L}^{-1}\left( \Lambda ^{1/2}T^{2}\left( \tau ^{3}\right)
^{1/2}f\right) \star \left. \partial _{L}^{-}g\right| _{\tilde{x}^{3}=a}^{b}
\nonumber \\
&&\hspace*{-2.9cm}-\,q^{-1/2}\lambda _{+}^{-1/2}\lambda \left( \partial
^{0}\right) _{L}^{-1}\left( \Lambda ^{1/2}\left( \tau ^{3}\right)
^{-1/2}\left( T^{-}\sigma ^{2}+qS^{1}\right) f\right) \star \left. \partial
_{L}^{+}g\right| _{\tilde{x}^{3}=a}^{b}  \nonumber \\
&&\hspace*{-2.9cm}+\,\lambda _{+}^{-1}\left( \partial ^{0}\right)
_{L}^{-1}\left( \Lambda ^{1/2}\left( \lambda _{-}^{2}T^{-}T^{2}+q\left( \tau
^{1}-\sigma ^{2}\right) \right) f\right) \star \left. \tilde{\partial}%
_{L}^{3}g\right| _{\tilde{x}^{3}=a}^{b}.  \nonumber
\end{eqnarray}
Similarly in the case of the second differential calculus we get 
\begin{eqnarray}
\left. \left( \hat{\tilde{\partial}^{3}}\right) _{L}^{-1}\left( \hat{\tilde{%
\partial}_{L}^{3}}f\right) \star g\right| _{x^{3}=a}^{b} &=&\left. f\star
g\right| _{x^{3}=a}^{b} \\
&&\hspace*{-3.cm}-\,\left( \hat{\tilde{\partial}^{3}}\right)
_{L}^{-1}\left( \Lambda ^{-1/2}\left( \tau ^{3}\right) ^{-1/2}\sigma
^{2}f\right) \star \left. \hat{\tilde{\partial}_{L}^{3}}g\right|
_{x^{3}=a}^{b}  \nonumber \\
&&\hspace*{-3.cm}+\,q^{3/2}\lambda _{+}^{1/2}\lambda \left( \hat{\tilde{%
\partial}^{3}}\right) _{L}^{-1}\left( \Lambda ^{-1/2}T^{2}f\right) \star
\left. \hat{\partial}_{L}^{-}g\right| _{x^{3}=a}^{b},  \nonumber \\
\left. \left( \hat{\partial}^{-}\right) _{L}^{-1}\left( \hat{\partial}%
_{L}^{-}f\right) \star g\right| _{x^{+}=a}^{b} &=&\left. f\star g\right|
_{x^{+}=a}^{b}-\left( \hat{\partial}^{-}\right) _{L}^{-1}\left( \Lambda
^{-1/2}\tau ^{1}f\right) \star \left. \hat{\partial}_{L}^{-}g\right|
_{x^{+}=a}^{b}  \nonumber \\
&&\hspace*{-3.cm}+\,q^{1/2}\lambda _{+}^{-1/2}\lambda \left( \hat{\partial}%
^{-}\right) _{L}^{-1}\left( \Lambda ^{-1/2}\left( \tau ^{3}\right)
^{-1/2}S^{1}f\right) \star \left. \hat{\tilde{\partial}_{L}^{3}}g\right|
_{x^{+}=a}^{b},  \nonumber \\
\left. \left( \hat{\partial}^{+}\right) _{L}^{-1}\left( \hat{\partial}%
_{L}^{+}f\right) \star g\right| _{x^{-}=a}^{b} &=&\left. f\star g\right|
_{x^{-}=a}^{b}-\left( \hat{\partial}^{+}\right) _{L}^{-1}\left( \Lambda
^{-1/2}\sigma ^{2}f\right) \star \left. \hat{\partial}_{L}^{+}g\right|
_{x^{-}=a}^{b}  \nonumber \\
&&\hspace*{-3.cm}+\,q^{1/2}\lambda _{+}^{1/2}\lambda \left( \hat{\partial}%
^{+}\right) _{L}^{-1}\left( \Lambda ^{-1/2}T^{2}\left( \tau ^{3}\right)
^{1/2}f\right) \star \left. \hat{\partial}_{L}^{0}g\right| _{x^{-}=a}^{b} 
\nonumber \\
&&\hspace*{-3.cm}+\,q^{1/2}\lambda _{+}^{-1/2}\lambda \left( \hat{\partial}%
^{+}\right) _{L}^{-1}\left( \Lambda ^{-1/2}\left( \tau ^{3}\right)
^{-1/2}\left( T^{+}\sigma ^{2}+q\tau ^{3}T^{2}\right) f\right) \star \left. 
\hat{\tilde{\partial}_{L}^{3}}g\right| _{x^{-}=a}^{b}  \nonumber \\
&&\hspace*{-3.cm}-\,q^{2}\lambda ^{2}\left( \hat{\partial}^{+}\right)
_{L}^{-1}\left( \Lambda ^{-1/2}T^{2}T^{+}f\right) \star \left. \hat{\partial}%
_{L}^{-}g\right| _{x^{-}=a}^{b},  \nonumber \\
\left. \left( \hat{\partial}^{0}\right) _{L}^{-1}\left( \hat{\partial}%
_{L}^{0}f\right) \star g\right| _{\tilde{x}^{3}=a}^{b} &=&\left. f\star
g\right| _{\tilde{x}^{3}=a}^{b}-\left( \hat{\partial}^{0}\right)
_{L}^{-1}\left( \Lambda ^{-1/2}\left( \tau ^{3}\right) ^{1/2}\tau
^{1}f\right) \star \left. \hat{\partial}_{L}^{0}g\right| _{\tilde{x}%
^{3}=a}^{b}  \nonumber \\
&&\hspace*{-3.cm}+\,q^{-1/2}\lambda _{+}^{-1/2}\lambda \left( \hat{\partial}%
^{0}\right) _{L}^{-1}\left( \Lambda ^{-1/2}S^{1}f\right) \star \left. \hat{%
\partial}_{L}^{+}g\right| _{\tilde{x}^{3}=a}^{b}  \nonumber \\
&&\hspace*{-3.cm}+\,q^{1/2}\lambda _{+}^{-1/2}\lambda \left( \hat{\partial}%
^{0}\right) _{L}^{-1}\left( \Lambda ^{-1/2}\left( qT^{+}\tau
^{1}-T^{2}\right) f\right) \star \left. \hat{\partial}_{L}^{-}g\right| _{%
\tilde{x}^{3}=a}^{b}  \nonumber \\
&&\hspace*{-3.cm}-\,\lambda _{+}^{-1}\left( \hat{\partial}_{L}^{0}\right)
^{-1}\left( \Lambda ^{-1/2}\left( \tau ^{3}\right) ^{-1/2}\left( \lambda
^{2}T^{+}S^{1}+q^{-1}\left( \tau ^{3}\tau ^{1}-\sigma ^{2}\right) \right)
f\right) \star \left. \hat{\tilde{\partial}_{L}^{3}}g\right| _{\tilde{x}%
^{3}=a}^{b}.  \nonumber
\end{eqnarray}
In the case of right integrals the rules for integration by parts take the
form 
\begin{eqnarray}
\left( \tilde{\partial}^{3}\right) _{R}^{-1}f\star \left. \left( \tilde{%
\partial}_{R}^{3}g\right) \right| _{x^{3}=a}^{b} &=&\left. f\star g\right|
_{x^{3}=a}^{b}-\left( \tilde{\partial}^{3}\right) _{R}^{-1}\left( \tilde{%
\partial}_{R}^{3}f\right) \star \left. \left( \Lambda ^{-1/2}\sigma
^{2}g\right) \right| _{x^{3}=a}^{b} \\
&&\hspace*{-2.9cm}-\,q^{1/2}\lambda _{+}^{1/2}\lambda \left( \tilde{\partial}%
^{3}\right) _{R}^{-1}\left( \partial _{R}^{+}f\right) \star \left. \left(
\Lambda ^{-1/2}S^{1}g\right) \right| _{x^{3}=a}^{b},  \nonumber \\
\left( \partial ^{+}\right) _{R}^{-1}f\star \left. \left( \partial
_{R}^{+}g\right) \right| _{x^{-}=a}^{b} &=&\left. f\star g\right|
_{x^{-}=a}^{b}-\left( \partial ^{+}\right) _{R}^{-1}\left( \partial
_{R}^{+}f\right) \star \left. \left( \Lambda ^{-1/2}\left( \tau ^{3}\right)
^{1/2}\tau ^{1}g\right) \right| _{x^{-}=a}^{b}  \nonumber \\
&&\hspace*{-2.9cm}-\,q^{3/2}\lambda _{+}^{-1/2}\lambda \left( \partial
^{+}\right) _{R}^{-1}\left( \tilde{\partial}_{R}^{3}f\right) \star \left.
\left( \Lambda ^{-1/2}\left( \tau ^{3}\right) ^{1/2}T^{2}g\right) \right|
_{x^{-}=a}^{b},  \nonumber \\
\left( \partial ^{-}\right) _{R}^{-1}f\star \left. \left( \partial
_{R}^{-}g\right) \right| _{x^{+}=a}^{b} &=&\left. f\star g\right|
_{x^{+}=a}^{b}-\left( \partial ^{-}\right) _{R}^{-1}\left( \partial
_{R}^{-}f\right) \star \left. \left( \Lambda ^{-1/2}\left( \tau ^{3}\right)
^{-1/2}\sigma ^{2}g\right) \right| _{x^{+}=a}^{b}  \nonumber \\
&&\hspace*{-2.9cm}-\,q^{3/2}\lambda _{+}^{1/2}\lambda \left( \partial
^{-}\right) _{R}^{-1}\left( \partial _{R}^{0}f\right) \star \left. \left(
\Lambda ^{-1/2}\left( \tau ^{3}\right) ^{-1/2}S^{1}g\right) \right|
_{x^{+}=a}^{b}  \nonumber \\
&&\hspace*{-2.9cm}-\,q^{2}\lambda ^{2}\left( \partial ^{-}\right)
_{R}^{-1}\left( \partial _{R}^{+}f\right) \star \left. \left( \Lambda
^{-1/2}\left( \tau ^{3}\right) ^{-1/2}T^{-}S^{1}g\right) \right|
_{x^{+}=a}^{b}  \nonumber \\
&&\hspace*{-2.9cm}-\,q^{1/2}\lambda _{+}^{-1/2}\lambda \left( \partial
^{-}\right) _{R}^{-1}\left( \tilde{\partial}_{R}^{3}f\right) \star \left.
\left( \Lambda ^{-1/2}\left( \tau ^{3}\right) ^{-1/2}\left(
S^{1}-qT^{-}\sigma ^{2}\right) g\right) \right| _{x^{+}=a}^{b},  \nonumber \\
\left( \partial ^{0}\right) _{R}^{-1}f\star \left. \left( \partial
_{R}^{0}g\right) \right| _{\tilde{x}^{3}=a}^{b} &=&\left. f\star g\right| _{%
\tilde{x}^{3}=a}^{b}-\left( \partial ^{0}\right) _{R}^{-1}\left( \partial
_{R}^{0}f\right) \star \left. \left( \Lambda ^{-1/2}\tau ^{1}g\right)
\right| _{\tilde{x}^{3}=a}^{b}  \nonumber \\
&&\hspace*{-2.9cm}\,-\,q^{1/2}\lambda _{+}^{-1/2}\lambda \left( \partial
^{0}\right) _{R}^{-1}\left( \partial _{R}^{-}f\right) \star \left. \left(
\Lambda ^{-1/2}T^{2}g\right) \right| _{\tilde{x}^{3}=a}^{b}  \nonumber \\
&&\hspace*{-2.9cm}-\,q^{1/2}\lambda _{+}^{-1/2}\lambda \left( \partial
^{0}\right) _{R}^{-1}\left( \partial _{R}^{+}f\right) \star \left. \left(
\Lambda ^{-1/2}\left( qS^{1}-\tau ^{1}T^{-}\right) g\right) \right| _{\tilde{%
x}^{3}=a}^{b}  \nonumber \\
&&\hspace*{-2.9cm}+\,\lambda _{+}^{-1}\left( \partial ^{0}\right)
_{R}^{-1}\left( \tilde{\partial}_{R}^{3}f\right) \star \left. \Lambda
^{-1/2}\left( \lambda ^{2}T^{2}T^{-}+q\left( \sigma ^{2}-\tau ^{1}\right)
\right) g\right| _{\tilde{x}^{3}=a}^{b}  \nonumber
\end{eqnarray}
and analogously for the second differential calculus we have 
\begin{eqnarray}
\left. \left( \hat{\tilde{\partial}^{3}}\right) _{R}^{-1}f\star \left( \hat{%
\tilde{\partial}_{R}^{3}}g\right) \right| _{x^{3}=a}^{b} &=&\left. f\star
g\right| _{x^{3}=a}^{b} \\
&&\hspace*{-2.9cm}-\,\left( \hat{\tilde{\partial}^{3}}\right)
_{R}^{-1}\left( \hat{\tilde{\partial}_{R}^{3}}g\right) \star \left. \left(
\Lambda ^{1/2}\left( \tau ^{3}\right) ^{1/2}\tau ^{1}g\right) \right|
_{x^{3}=a}^{b}  \nonumber \\
&&\hspace*{-2.9cm}-\,q^{3/2}\lambda _{+}^{1/2}\lambda \left( \hat{\tilde{%
\partial}^{3}}\right) _{R}^{-1}\left( \hat{\partial}_{R}^{-}f\right) \star
\left. \left( \Lambda ^{1/2}\left( \tau ^{3}\right) ^{1/2}T^{2}g\right)
\right| _{x^{3}=a}^{b},  \nonumber \\
\left. \left( \hat{\partial}^{-}\right) _{R}^{-1}f\star \left( \hat{\partial}%
_{R}^{-}g\right) \right| _{x^{+}=a}^{b} &=&\left. f\star g\right|
_{x^{+}=a}^{b}-\left( \hat{\partial}^{-}\right) _{R}^{-1}\left( \hat{\partial%
}_{R}^{-}f\right) \star \left. \left( \Lambda ^{1/2}\sigma ^{2}g\right)
\right| _{x^{+}=a}^{b}  \nonumber \\
&&\hspace*{-2.9cm}-\,q^{1/2}\lambda _{+}^{-1/2}\lambda \left( \hat{\partial}%
^{-}\right) _{R}^{-1}\left( \hat{\tilde{\partial}_{R}^{3}}f\right) \star
\left. \left( \Lambda ^{1/2}S^{1}g\right) \right| _{x^{+}=a}^{b},  \nonumber
\\
\left. \left( \hat{\partial}^{+}\right) _{R}^{-1}f\star \left( \hat{\partial}%
_{R}^{+}g\right) \right| _{x^{-}=a}^{b} &=&\left. f\star g\right|
_{x^{-}=a}^{b}-\left( \hat{\partial}^{+}\right) _{R}^{-1}\left( \hat{\partial%
}_{R}^{+}f\right) \star \left. \left( \Lambda ^{1/2}\tau ^{1}g\right)
\right| _{x^{-}=a}^{b}  \nonumber \\
&&\hspace*{-2.9cm}-\,q^{1/2}\lambda _{+}^{1/2}\lambda \left( \hat{\partial}%
^{+}\right) _{R}^{-1}\left( \hat{\tilde{\partial}_{R}^{3}}f\right) \star
\left. \left( \Lambda ^{1/2}\left( \tau ^{1}T^{+}+q^{-1}\tau
^{3}T^{2}\right) g\right) \right| _{x^{-}=a}^{b}  \nonumber \\
&&\hspace*{-2.9cm}-\,\lambda ^{2}\left( \hat{\partial}^{+}\right)
_{R}^{-1}\left( \hat{\partial}_{R}^{-}f\right) \star \left. \left( \Lambda
^{1/2}T^{2}T^{+}g\right) \right| _{x^{-}=a}^{b},  \nonumber \\
\left. \left( \hat{\partial}^{0}\right) _{R}^{-1}f\star \left( \hat{\partial}%
_{R}^{0}g\right) \right| _{\tilde{x}^{3}=a}^{b} &=&\left. f\star g\right| _{%
\tilde{x}^{3}=a}^{b}-\left( \hat{\partial}^{0}\right) _{R}^{-1}\left( \hat{%
\partial}_{R}^{0}f\right) \star \left. \left( \Lambda ^{1/2}\left( \tau
^{3}\right) ^{1/2}\sigma ^{2}g\right) \right| _{\tilde{x}^{3}=a}^{b} 
\nonumber \\
&&\hspace*{-2.9cm}-\,q^{3/2}\lambda _{+}^{-1/2}\lambda \left( \hat{\partial}%
^{0}\right) _{R}^{-1}\left( \hat{\partial}_{R}^{+}f\right) \star \left.
\left( \Lambda ^{1/2}\left( \tau ^{3}\right) ^{-1/2}S^{1}g\right) \right| _{%
\tilde{x}^{3}=a}^{b}  \nonumber \\
&&\hspace*{-2.9cm}+\,q^{-1/2}\lambda _{+}^{-1/2}\lambda \left( \hat{\partial}%
^{0}\right) _{R}^{-1}\left( \hat{\partial}_{R}^{-}f\right) \star \left.
\left( \Lambda ^{1/2}\left( \tau ^{3}\right) ^{-1/2}\left( q\tau
^{3}T^{2}-\sigma ^{2}T^{+}\right) g\right) \right| _{\tilde{x}^{3}=a}^{b} 
\nonumber \\
&&\hspace*{-2.9cm}-\,q^{-1}\lambda _{+}^{-1}\lambda \left( \hat{\partial}%
^{0}\right) _{R}^{-1}\left( \hat{\tilde{\partial}_{R}^{3}}f\right)  \nonumber
\\
&&\hspace*{-2.9cm}\star \left. \left( \Lambda ^{1/2}\left( \tau ^{3}\right)
^{1/2}\left( q^{-1}\lambda ^{2}S^{1}T^{+}+\left( \tau ^{3}\right)
^{-1}\sigma ^{2}-\tau ^{2}\right) g\right) \right| _{\tilde{x}^{3}=a}^{b}. 
\nonumber
\end{eqnarray}

As $\left( \partial ^{0}\right) ^{-1}$ commutes with the elements $\left( \partial ^{\mu}\right) ^{-1},$ $\mu=\pm ,\tilde{3}$ one can
at once realize that the different possibilities for a volume integration
are related to each other by the following identities, if surface terms are
neglected :

\begin{eqnarray}
\hspace*{-0.7cm}\left( \partial ^{+}\right) ^{-1}\left( \tilde{\partial}%
^{3}\right) ^{-1}\left( \partial ^{-}\right) ^{-1}\left( \partial
^{0}\right) ^{-1}f &=&q^{-4}\left( \partial ^{-}\right) ^{-1}\left( \tilde{%
\partial}^{3}\right) ^{-1}\left( \partial ^{+}\right) ^{-1}\left( \partial
^{0}\right) ^{-1}f \\
\hspace*{-0.7cm}=q^{-2}\left( \tilde{\partial}^{3}\right) ^{-1}\left(
\partial ^{-}\right) ^{-1}\left( \partial ^{+}\right) ^{-1}\left( \partial
^{0}\right) ^{-1}f &=&q^{-2}\left( \tilde{\partial}^{3}\right) ^{-1}\left(
\partial ^{+}\right) ^{-1}\left( \partial ^{-}\right) ^{-1}\left( \partial
^{0}\right) ^{-1}f  \nonumber \\
\hspace*{-0.7cm}=q^{-2}\left( \partial ^{-}\right) ^{-1}\left( \partial
^{+}\right) ^{-1}\left( \tilde{\partial}^{3}\right) ^{-1}\left( \partial
^{0}\right) ^{-1}f &=&q^{-2}\left( \partial ^{+}\right) ^{-1}\left( \partial
^{-}\right) ^{-1}\left( \tilde{\partial}^{3}\right) ^{-1}\left( \partial
^{0}\right) ^{-1}f.  \nonumber
\end{eqnarray}
These relations follow from the same reasonings we have already applied to the
Euclidean cases and carry over directly to volume integrals with the
operators $\left( \hat{\partial}^{\mu}\right) ^{-1}$ or $\left( P^{\mu}\right)
^{-1},$ $\mu=\pm ,\tilde{3},0.$
For calculating volume integrals over the whole Minkowski-space one
can use the
expressions 
\begin{eqnarray}
\lefteqn{\left( \partial ^{-}\right) ^{-1}\left( \partial ^{0}\right) ^{-1}\left(
\partial ^{+}\right) ^{-1}\left( \tilde{\partial}^{3}\right) ^{-1}f
=}\\ &=&q^{2}\left( D_{q^{-2}}^{+}\right) \left( \tilde{D}_{q^{-2}}^{3}\right)
^{-1} \hspace*{.5cm} \left( D_{q^{-2}}^{-}\right)\nonumber\\ \hspace{.5cm}&&\times \left( \left(
D_{q_{+}q^{-2}}^{3}\right) f\right) \left( q^{2}x^{+},q^{2}\tilde{x}%
^{3},\left( 1+2\frac{\lambda}{\lambda_{+}}\frac{\tilde{x}^{3}}{x^{3}}%
,q^{2}x^{-}\right) \right) ,  \nonumber \\
\lefteqn{\left( \hat{\partial}^{0}\right) ^{-1}\left( \hat{\partial}^{+}\right)
^{-1}\left( \hat{\partial}^{-}\right) ^{-1}\left( \hat{\tilde{\partial}^{3}}%
\right) ^{-1}f =} \nonumber \\
\hspace*{.5cm}&=&\left( \tilde{D}_{q^{2}}^{3}\right) \left(
D_{q^{2}}^{-}\right) ^{-1}\left( D_{q^{2}}^{+}\right) \left( \left(
D_{q_{-}q^{2}}^{3}\right) f\right) \left( q^{-2}x^{+},q^{-2}x^{3}\right) , 
\nonumber \\
\lefteqn{\left( -\frac{i}{\left[ 2\right] _{q^{2}}}\right) ^{4}\left( P^{-}\right)
^{-1}\left( P^{0}\right) ^{-1}\left( P^{+}\right) ^{-1}\left( \tilde{P}%
^{3}\right) ^{-1}f =} \nonumber  \\
\hspace*{.5cm}&=&\left( J^{-}\right) ^{-1}\left( D_{q^{2}}^{+}\right)
\left( J^{0}\right) ^{-1}\left( \tilde{D}_{q^{2}}^{3}\right) ^{-1}
\nonumber\\\hspace{.5cm}&&\times\left(
J^{+}\right) ^{-1}\left( D_{q^{2}}^{-}\right) \left( \tilde{J}^{3}\right)
^{-1}\left( D_{q^{2}}^{3}\right) ^{-1}f  \nonumber
\end{eqnarray}
where again surface terms have been neglected. Translation and
Lorentz invariance of these volume integrals can be shown by the same
method already applied to the Euclidean cases.

\section{Remarks}

In the past several attempts have been made to introduce the notion of
integration on quantum spaces \cite{Ste96},\cite{CZ93},\cite{Fio93},\cite
{HW92},\cite{Die01}. However, our way of introducing integration is much
more in the spirit of \cite{KM94}, where integration has also been
considered as an inverse of q-differentiation. While in \cite{KM94} the
algebraic point of view dominates, our integration formulae are directly
derived by inversion of q-difference operators. Thus, this sort of integrals
represent nothing else than a generalization of Jackson's celebrated
q-integral to higher dimensions.

The expressions for our integrals are affected by the non-commutative
structure of the underlying quantum space in two ways. First of all,
integration can always be reduced to a process of summation. Additionally,
there is a correction term for each order of $\lambda $ vanishing in the
undeformed limit $(q=1).$ This new kind of terms result from the finite
boundaries of integration and are responsible for the fact that integral
operators referring to different directions do not commute.

Let us end with a few comments on the question of integrability. It can
easily be seen that polynomials lead to finite integrals, thus, they are
always integrable. Furthermore, it can be shown by numerical simulations or
a detailed analysis of our formulae that for functions with large 
characteristic wave length the correction
terms in our integral formulae become very small. If we assume $\lambda $
to be some sort of a fundamental length in space-time, from a physical point
of view, only such functions should be considered whose characteristic
wave length is not essentially smaller than $\lambda $. For these functions
the assumption seems to be reasonable that the corresponding correction
terms constitute a convergent series.

\appendix 

\section{Notation\label{AppA}}

\begin{enumerate}
\item  The \textit{q-number} is defined by \cite{KS97} 
\begin{equation}
\left[ \left[ c\right] \right] _{q^{a}}\equiv \frac{1-q^{ac}}{1-q^{a}}%
,\qquad a,c\in \mathbb{C}.
\end{equation}
For $m\in \mathbb{N}$, we can introduce the \textit{q-factorial }by setting 
\begin{equation}
\left[ \left[ m\right] \right] _{q^{a}}!\equiv \left[ \left[ 1\right]
\right] _{q^{a}}\left[ \left[ 2\right] \right] _{q^{a}}\ldots \left[ \left[
m\right] \right] _{q^{a}},\qquad \left[ \left[ 0\right] \right]
_{q^{a}}!\equiv 1.
\end{equation}
There is also a q-analogue of the usual binomial coefficients, the so-called 
\textit{q-binomial coefficients} defined by the formula 
\begin{equation}
\left[ \begin{array}{c} \alpha\\m \end{array}\right]_{q^{a}}\equiv \frac{\left[ \left[ \alpha \right]
\right] _{q^{a}}\left[ \left[ \alpha -1\right] \right] _{q^{a}}\ldots \left[
\left[ \alpha -m+1\right] \right] _{q^{a}}}{\left[ \left[ m\right] \right]
_{q^{a}}!}
\end{equation}
where $\alpha \in \mathbb{C},$ $m\in \mathbb{N}$.

\item  Note that in functions only such variables are explicitly displayed
which are effected by a scaling. For example, 
\begin{equation}
f\left( q^{2}x^{+}\right) \qquad \text{instead of}\qquad f\left(
q^{2}x^{+},x^{3},x^{-}\right) .
\end{equation}

\item  The \textit{Jackson derivative} referring to the coordinate $x^{A}$ is
defined by 
\begin{equation}
D_{q^{a}}^{A}f:=\frac{f\left( x^{A}\right) -f\left( q^{a}x^{A}\right) }{%
\left( 1-q^{a}\right) x^{A}}
\end{equation}
where f may depend on all other coordinates as well. Higher Jackson
derivatives are obtained by applying the above operator $D_{q^{a}}^{A}$
several times: 
\begin{equation}
\left( D_{q^{a}}^{A}\right) ^{i}f:=\underbrace{D_{q^{a}}^{A}D_{q^{a}}^{A}%
\ldots D_{q^{a}}^{A}}_{i\text{ times}}f.
\end{equation}

\item  For $a>0,$ $q>1$ and $x^{A}>0,$ the definition of the \textit{Jackson
integral }is 
\begin{eqnarray}
\left. \left( D_{q^{a}}^{A}\right) ^{-1}f\right| _{0}^{x^{A}} &=&-\left(
1-q^{a}\right) \sum_{k=1}^{\infty }\left( q^{-ak}x^{A}\right) f\left(
q^{-ak}x^{A}\right) ,  \label{Jackson1} \\
\left. \left( D_{q^{a}}^{A}\right) ^{-1}f\right| _{x^{A}}^{\infty }
&=&-\left( 1-q^{a}\right) \sum_{k=0}^{\infty }\left( q^{ak}x^{A}\right)
f\left( q^{ak}x^{A}\right) ,  \nonumber \\
\left. \left( D_{q^{-a}}^{A}\right) ^{-1}f\right| _{0}^{x^{A}} &=&\left(
1-q^{-a}\right) \sum_{k=0}^{\infty }\left( q^{-ak}x^{A}\right) f\left(
q^{-ak}x^{A}\right) ,  \nonumber \\
\left. \left( D_{q^{-a}}^{A}\right) ^{-1}f\right| _{x^{A}}^{\infty }
&=&\left( 1-q^{-a}\right) \sum_{k=1}^{\infty }\left( q^{ak}x^{A}\right)
f\left( q^{ak}x^{A}\right) .  \nonumber
\end{eqnarray}
For $a>0,$ $q>1$ and $x^{A}<0,$we set 
\begin{eqnarray}
\left. \left( D_{q^{a}}^{A}\right) ^{-1}f\right| _{x^{A}}^{0} &=&\left(
1-q^{a}\right) \sum_{k=1}^{\infty }\left( q^{-ak}x^{A}\right) f\left(
q^{-ak}x^{A}\right) ,  \label{Jackson2} \\
\left. \left( D_{q^{a}}^{A}\right) ^{-1}f\right| _{-\infty }^{x^{A}}
&=&\left( 1-q^{a}\right) \sum_{k=0}^{\infty }\left( q^{ak}x^{A}\right)
f\left( q^{ak}x^{A}\right) ,  \nonumber \\
\left. \left( D_{q^{-a}}^{A}\right) ^{-1}f\right| _{x^{A}}^{0} &=&-\left(
1-q^{-a}\right) \sum_{k=0}^{\infty }\left( q^{-ak}x^{A}\right) f\left(
q^{-ak}x^{A}\right) ,  \nonumber \\
\left. \left( D_{q^{-a}}^{A}\right) ^{-1}f\right| _{-\infty }^{x^{A}}
&=&-\left( 1-q^{-a}\right) \sum_{k=0}^{\infty }\left( q^{ak}x^{A}\right)
f\left( q^{ak}x^{A}\right) .  \nonumber
\end{eqnarray}
Note that the formulae (\ref{Jackson1}) and (\ref{Jackson2}) also yield
expressions for the q-integrals over any other interval \cite{KS97}.

\item  Arguments enclosed in brackets refer to the first object on their
left. For example, we have 
\begin{equation}
D_{q^{2}}^{+}f\left( q^{2}x^{+}\right) =D_{q^{2}}^{+}\left( f\left(
q^{2}x^{+}\right) \right)
\end{equation}
or 
\begin{equation}
D_{q^{2}}^{+}\left[ D_{q^{2}}^{+}f+D_{q^{2}}^{-}f\right] \left(
q^{2}x^{+}\right) =D_{q^{2}}^{+}\left( \left[
D_{q^{2}}^{+}f+D_{q^{2}}^{-}f\right] \left( q^{2}x^{+}\right) \right) .
\end{equation}
However, the symbol $\mid _{x^{\prime }\rightarrow x}$ applies to the whole
expression on its left side reaching up to the next opening bracket or $\pm $
sign.

\item  Calculations for q-deformed Minkowski space show that it is
reasonable to give the following repeatedly appearing polynomials a name of
their own: 
\begin{eqnarray}
S_{i,j}\left( x^{0},\tilde{x}^{3}\right) &\equiv &\left\{ 
\begin{array}{c}
\sum\limits_{p_{1}=0}^{j}\sum\limits_{p_{2}=0}^{p_{1}}\ldots
\sum\limits_{p_{i-j}=0}^{p_{i-j-1}}\prod\limits_{l=0}^{i-j}a_{-}\left(
x^{0},q^{2p_{l}}\tilde{x}^{3}\right) ,\quad \\ 
1,\quad
\end{array}
\right. \left. 
\begin{array}{l}
\text{if }0\leq j<i, \\ 
\text{if }j=i,
\end{array}
\right.  \nonumber \\
a_{\pm }\left( x^{0},\tilde{x}^{3}\right) &\equiv &q^{\pm 1}\tilde{x}%
^{3}\left( q^{\pm 1}\tilde{x}^{3}+\lambda _{+}x^{0}\right) , \\
\left( M^{\pm }\right) _{i,j}^{k}\left( \underline{x}\right) &\equiv &\left(
M^{\pm }\right) _{i,j}^{k}\left( x^{0},x^{+},\tilde{x}^{3},x^{-}\right) 
\nonumber \\
&=&{k \choose i}\lambda _{+}^{j}\left( a_{\pm }\left( q^{2j}\tilde{x}%
^{3}\right) \right) ^{i}\left( x^{+}x^{-}\right) ^{j}S_{k-j,j}\left( x^{0},%
\tilde{x}^{3}\right) ,  \nonumber \\
\left( M_{\alpha }^{-+}\right) _{i,j,u}^{k,l}\left( \underline{x}\right)
&\equiv &\left( M_{\alpha }^{-+}\right) _{i,j,u}^{k,l}\left( x^{0},x^{+},%
\tilde{x}^{3},x^{-}\right)  \nonumber \\
&=&{k \choose i}{l \choose j}\lambda _{+}^{u}\left( a_{^{-}}\left( \alpha
q^{2u+1}\tilde{x}^{3}\right) ^{k-i}\right) \left( a_{+}\left( \alpha q^{2u+1}%
\tilde{x}^{3}\right) \right) ^{l-j}  \nonumber \\
&&\times \left( x^{+}x^{-}\right) ^{u}S_{i+j,u}\left( x^{0},\tilde{x}%
^{3}\right) ,  \nonumber \\
\left( M_{\alpha }^{+-}\right) _{i,j,u}^{k,l}\left( \underline{x}\right)
&\equiv &\left( M_{\alpha }^{-+}\right) _{i,j,u}^{k,l}\left( \underline{x}%
\right) .  \nonumber
\end{eqnarray}

\item  In \cite{BW01} we have introduced the quantities $\left(
K_{a_{1},\ldots ,a_{l}}\right) _{\alpha }^{\left( k_{1},\ldots ,k_{l}\right)
},$ $k_{i}\in \mathbb{N},$ $a_{i},\alpha \in \mathbb{R}$. We also refer to \cite
{BW01} for a review of their explicit calculation. These quantities can be
used to define new differentiation operators by setting 
\begin{equation}
\left( D_{a_{1},\ldots ,a_{l}}\right) ^{\left( k_{1},\ldots ,k_{l}\right)
}x^{n}=\left\{ 
\begin{array}{c}
\left( K_{a_{1},\ldots ,a_{l}}\right) _{n}^{\left( k_{1},\ldots
,k_{l}\right) }x^{n-k_{1}-\ldots -k_{l}}, \\ 
0,\qquad \text{if }n<k_{1}+\ldots +k_{l},
\end{array}
\right.
\end{equation}
Especially, in the case of q-deformed Minkowski space we need the following
operators: 
\begin{eqnarray}
\left( D_{1,q}^{3}\right) ^{k,l} &\equiv &\left( D_{1,q^{2}}\right) ^{\left(
k,l\right) }, \\
\left( D_{2,q}^{3}\right) ^{k,l} &\equiv &\left(
D_{y_{-}/x^{3},q^{2}y_{-}/x^{3}}\right) ^{\left( k,l\right) },  \nonumber \\
\left( D_{3,q}^{3}\right) _{i,j}^{k,l} &\equiv &\left(
D_{y_{+}/x^{3},q^{2}y_{+}/x^{3},y_{-}/x^{3},q^{2}y_{-}/x^{3}}\right)
^{\left( k,l,i,j\right) },  \nonumber \\
\left( D_{4,q}^{3}\right) _{i,j}^{k,l} &\equiv &\left(
D_{y_{+}/y_{-},q^{2}y_{+}/y_{-},1,q^{2}}\right) ^{\left( k,l,i,j\right) }. 
\nonumber
\end{eqnarray}
Notice that these operators should act upon the coordinate $x^{3}$ only.

\item  In section 4 we have set 
\begin{eqnarray}
\left( \tilde{O}_{1}^{3}\right) _{k}f &=&\tilde{x}^{3}D_{q^{2}}^{+}\left(
D_{1,q^{-1}}^{3}\right) ^{k,k}D_{q^{-2}}^{-}f, \\
\left( \tilde{O}_{2}^{3}\right) _{k}f &=&\left( D_{2,q^{-1}}^{3}\right)
^{k,k+1}f-q\lambda _{+}^{-1}\lambda ^{2}x^{+}\tilde{x}^{3}D_{q^{2}}^{+}%
\left( D_{2,q^{-1}}^{3}\right) ^{k+1,k+1}f,  \nonumber \\
\left( \tilde{O}_{3}^{3}\right) _{k}f &=&x^{-}\left( D_{1,q^{-1}}^{3}\right)
^{k,k+1}D_{q^{-2}}^{-}f,  \nonumber \\[0.16in]
\left( O_{1}^{+}\right) _{k}f &=&D_{q^{-2}}^{-}\left(
D_{1,q^{-1}}^{3}\right) ^{k,k}f, \\
\left( O_{2}^{+}\right) _{k}f &=&x^{+}\left( D_{2,q^{-1}}^{3}\right)
^{k+1,k+1}f,  \nonumber \\[0.16in]
\left( O_{1}^{0}\right) _{k}f &=&\tilde{D}_{q^{-2}}^{3}\left(
D_{1,q^{-1}}^{3}\right) ^{k,k}f\left( q^{-2}x^{+},q^{-2}x^{-}\right) \\
&&-\,\lambda \left( x^{0}+q^{-1}\lambda _{+}^{-1}\tilde{x}^{3}\right)
D_{q^{-2}}^{+}\left( D_{1,q^{-1}}^{3}\right) ^{k,k}D_{q^{-2}}^{-}f\left(
q^{-2}\tilde{x}^{3}\right) ,  \nonumber \\
\left( O_{2}^{0}\right) _{k}f &=&x^{+}D_{q^{-2}}^{+}\left(
D_{2,q^{-1}}^{3}\right) ^{k,k+1}f  \nonumber \\
&&+\,q^{-1}\lambda x^{+}\left( x^{0}+q\lambda _{+}^{-1}\tilde{x}^{3}\right)
D_{q^{-2}}^{+}\left( D_{2,q^{-1}}^{3}\right) ^{k+1,k+1}f,  \nonumber \\
\left( O_{3}^{0}\right) _{k}f &=&x^{+}x^{-}D_{q^{-2}}^{+}\left(
D_{1,q^{-1}}^{3}\right) ^{k,k+1}D_{q^{-2}}^{-}f,  \nonumber \\[0.16in]
\left( Q_{1}^{0}\right) _{k,l}f &=&\left( x^{0}-q\lambda \tilde{x}%
^{3}\right) \left( D_{3,q^{-1}}^{3}\right) _{l,l+1}^{k+1,k}f\left(
q^{-2}x^{+}\right) , \\
\left( Q_{2}^{0}\right) _{k,l}f &=&x^{-}\left( D_{4,q^{-1}}^{3}\right)
_{l,l}^{k+1,k}D_{q^{-2}}f\left( q^{-2}x^{+}\right)  \nonumber \\
&&+\,q^{-3}\lambda \left( x^{0}-q\lambda \tilde{x}^{3}\right) x^{-}\left(
D_{4,q^{-1}}^{3}\right) _{l,l+1}^{k+1,k}D_{q^{-2}}^{-}f\left(
q^{-2}x^{+}\right) ,  \nonumber \\
\left( Q_{3}^{0}\right) _{k,l}f &=&\left( D_{3,q^{-1}}^{3}\right)
_{l,l+1}^{k+1,k+1}f\left( q^{-2}x^{+}\right) ,  \nonumber \\
\left( Q_{4}^{0}\right) _{k,l}f &=&x^{-}\left( D_{4,q^{-1}}^{3}\right)
_{l,l+1}^{k+1,k+1}D_{q^{-2}}^{-}f\left( q^{-2}x^{+}\right) ,  \nonumber \\
\left( Q_{5}^{0}\right) _{k,l}f &=&\left( D_{3,q^{-1}}^{3}\right)
_{l+1,l+1}^{k+1,k}f\left( q^{-2}x^{+}\right) ,  \nonumber \\[0.16in]
\left( O_{1}^{-}\right) _{k}f &=&x^{3}D_{q^{2}}^{+}\tilde{D}%
_{q^{-2}}^{3}\left( D_{1,q^{-1}}^{3}\right) ^{k,k}f\left(
q^{-2}x^{+},q^{-2}x^{-}\right) \\
&&-\,q^{2}\lambda \tilde{x}^{3}\left( x^{0}+q^{-1}\lambda _{+}^{-1}\tilde{x}%
^{3}\right)  \nonumber \\
&&\times \left( D_{q^{2}}^{+}\right) ^{2}\left( D_{1,q^{-1}}^{3}\right)
^{k,k}D_{q^{-2}}^{-}f\left( q^{-2}x^{+},q^{-2}\tilde{x}^{3}\right)  \nonumber
\\
&&-\,q^{-1}\frac{\lambda }{\lambda _{+}}\left( \tilde{x}^{3}\right)
^{2}\left( D_{q^{2}}^{+}\right) ^{2}\left( D_{1,q^{-1}}^{3}\right)
^{k,k}D_{q^{-2}}^{-}f\left( q^{-2}\tilde{x}^{3}\right) ,  \nonumber \\
\left( O_{2}^{-}\right) _{k}f &=&x^{3}D_{q^{2}}^{+}\left(
D_{2,q^{-1}}^{3}\right) ^{k,k+1}f\left( q^{-2}x^{+}\right)  \nonumber \\
&&+\,q^{-1}\lambda _{+}^{-1}\tilde{x}^{3}D_{q^{2}}^{+}\left(
D_{2,q^{-1}}^{3}\right) ^{k,k+1}f  \nonumber \\
&&-\,q^{3}\lambda _{+}^{-1}\lambda ^{2}x^{+}\tilde{x}^{3}\left(
x^{0}+q\lambda _{+}^{-1}\tilde{x}^{3}\right)  \nonumber \\
&&\times \left( D_{q^{2}}^{+}\right) ^{2}\left( D_{2,q^{-1}}^{3}\right)
^{k+1,k+1}f\left( q^{-2}x^{+}\right)  \nonumber \\
&&-\,q^{2}\frac{\lambda ^{2}}{\lambda _{+}^{2}}x^{+}\left( \tilde{x}%
^{3}\right) ^{2}\left( D_{q^{2}}^{+}\right) ^{2}\left(
D_{2,q^{-1}}^{3}\right) ^{k+1,k+1}f,  \nonumber \\
\left( O_{3}^{-}\right) _{k}f &=&q^{-1}x^{-}\tilde{D}_{q^{2}}^{3}\left(
D_{1,q^{-1}}^{3}\right) ^{k,k+1}f\left( q^{-2}x^{+},q^{-2}x^{-}\right) 
\nonumber \\
&&-\,q^{-1}\lambda \left( x^{0}+\tilde{x}^{3}\right)
x^{-}D_{q^{2}}^{+}\left( D_{1,q^{-1}}^{3}\right)
^{k,k+1}D_{q^{-2}}^{-}f\left( q^{-2}x^{+}\right)  \nonumber \\
&&-\,q\lambda x^{-}D_{q^{2}}^{+}\left( D_{1,q^{-1}}^{3}\right)
^{k,k}D_{q^{-2}}^{-}f\left( q^{-2}x^{+}\right)  \nonumber \\
&&-\,q^{-2}\frac{\lambda }{\lambda _{+}}\tilde{x}^{3}x^{-}D_{q^{2}}^{+}%
\left( D_{1,q^{-1}}^{3}\right) ^{k,k+1}D_{q^{-2}}^{-}f,  \nonumber \\
\left( O_{4}^{-}\right) _{k}f &=&D_{q^{2}}^{+}\left( D_{2,q^{-1}}^{3}\right)
^{k+1,k+1}f\left( q^{-2}x^{+}\right) ,  \nonumber \\[0.16in]
\left( Q_{1}^{-}\right) _{k,l}f &=&\left( q^{-1}+\lambda _{+}\right)
x^{-}\left( D_{3,q^{-1}}^{3}\right) _{l,l+1}^{k,k+1}f\left(
q^{-2}x^{+}\right) \\
&&+\,q^{-2}\lambda \left( x^{0}-q\lambda \tilde{x}^{3}\right) x^{-}\left(
D_{3,q^{-1}}^{3}\right) _{l,l+1}^{k+1,k+1}f\left( q^{-2}x^{+}\right) , 
\nonumber \\
\left( Q_{2}^{-}\right) _{k,l}f &=&\left( x^{-}\right) ^{2}\left(
D_{4,q^{-1}}^{3}\right) _{l,l}^{k+1,k+1}D_{q^{-2}}^{-}f\left(
q^{-2}x^{+}\right)  \nonumber \\
&&+\,q^{-1}\left( q^{-1}+\lambda _{+}\right) \left( x^{-}\right) ^{2}\left(
D_{4,q^{-1}}^{3}\right) _{l,l+1}^{k,k+1}D_{q^{-2}}^{-}f\left(
q^{-2}x^{+}\right)  \nonumber \\
&&+\,q^{-3}\lambda \left( x^{0}-q\lambda \tilde{x}^{3}\right) \left(
x^{-}\right) ^{2}\left( D_{4,q^{-1}}^{3}\right) _{l,l+1}^{k+1,k+1}f\left(
q^{-2}x^{+}\right) ,  \nonumber \\
\left( Q_{3}^{-}\right) _{k,l}f &=&\left( x^{0}-q\lambda \tilde{x}%
^{3}\right) \tilde{x}^{3}D_{q^{2}}^{+}\left( D_{3,q^{-1}}^{3}\right)
_{l,l+1}^{k+1,k}f\left( q^{-2}x^{+}\right) ,  \nonumber \\
\left( Q_{4}^{-}\right) _{k,l}f &=&\tilde{x}^{3}x^{-}D_{q^{2}}^{+}\left(
D_{4,q^{-1}}^{3}\right) _{l,l}^{k+1,k}f\left( q^{-2}x^{+}\right)  \nonumber
\\
&&+\,q^{-3}\lambda \left( x^{0}-q\lambda \,\tilde{x}^{3}\right) \tilde{x}%
^{3}x^{-}D_{q^{2}}^{+}\left( D_{4,q^{-1}}^{3}\right)
_{l,l+1}^{k+1,k}D_{q^{-2}}^{-}f\left( q^{-2}x^{+}\right) ,  \nonumber \\
\left( Q_{5}^{-}\right) _{k,l}f &=&\tilde{x}^{3}D_{q^{2}}^{+}\left(
D_{3,q^{-1}}^{3}\right) _{l,l+1}^{k+1,k+1}f\left( q^{-2}x^{+}\right) , 
\nonumber \\
\left( Q_{6}^{-}\right) _{k,l}f &=&\tilde{x}^{3}x^{-}D_{q^{2}}^{+}\left(
D_{4,q^{-1}}^{3}\right) _{l,l+1}^{k+1,k+1}D_{q^{-2}}^{-}f\left(
q^{-2}x^{+}\right) ,  \nonumber \\
\left( Q_{7}^{-}\right) _{k,l}f &=&x^{-}\left( D_{3,q^{-1}}^{3}\right)
_{l+1,l+1}^{k+1,k+1}f\left( q^{-2}x^{+}\right) ,  \nonumber \\
\left( Q_{8}^{-}\right) _{k,l}f &=&\tilde{x}^{3}D_{q^{2}}^{+}\left(
D_{3,q^{-1}}^{3}\right) _{l+1,l}^{k+1,k}f\left( q^{-2}x^{+}\right) . 
\nonumber
\end{eqnarray}
Additionally, for the representations of $\left( \hat{\partial}^{\mu}\right)
^{-1},$ $\mu=\pm ,0,\tilde{3}$ we need the operators 
\begin{eqnarray}
\left( \hat{\tilde{O}^{3}}\right) _{k}f &=&\left( D_{2,q}^{3}\right)
^{k,k+1}f\left( q^{2}x^{+}\right) , \\[0.16in]
\left( \hat{O}_{1}^{-}\right) _{k}f &=&x^{-}\left( D_{2,q}^{3}\right)
^{k+1,k+1}f\left( q^{2}x^{+}\right) , \\
\left( \hat{O}_{2}^{-}\right) _{k}f &=&\tilde{x}^{3}D_{q^{2}}^{+}\left(
D_{2,q}^{3}\right) ^{k,k+1}f\left( q^{2}x^{+}\right) ,  \nonumber \\[0.16in]
\left( \hat{O}_{1}^{0}\right) _{k}f &=&\tilde{D}_{q^{2}}^{3}\left(
D_{1,q}^{3}\right) ^{k,k}f \\
&&-\,q^{3}\lambda _{+}^{-1}\lambda ^{2}x^{+}\tilde{x}^{3}D_{q^{2}}^{+}\tilde{%
D}_{q^{2}}^{3}\left( D_{1,q}^{3}\right) ^{k,k+1}f,  \nonumber \\
\left( \hat{O}_{2}^{0}\right) _{k}f &=&x^{-}\left( D_{1,q^{-1}}^{3}\right)
^{k,k+1}D_{q^{2}}^{-}f\left( q^{2}\tilde{x}^{3}\right) ,  \nonumber \\
\left( \hat{O}_{3}^{0}\right) _{k}f &=&qx^{+}D_{q^{2}}^{+}\left(
D_{2,q}^{3}\right) ^{k,k+1}f,  \nonumber \\
\left( \hat{O}_{4}^{0}\right) _{k}f &=&\tilde{x}^{3}D_{q^{2}}^{+}\left(
D_{1,q^{-1}}^{3}\right) ^{k,k}D_{q^{2}}^{-}f,  \nonumber \\[0.16in]
\left( \hat{Q}_{1}^{0}\right) _{k,l}f &=&\left( x^{0}+q^{-1}\lambda \tilde{x}%
^{3}\right) \left( D_{3,q}^{3}\right) _{l,l+1}^{k+1,k}f \\
&&+\,q\lambda _{+}^{-1}\lambda \left( q+\lambda _{+}\right) x^{+}\tilde{x}%
^{3}D_{q^{2}}^{+}\left( D_{3,q}^{3}\right) _{l,l+1}^{k,k+1}f  \nonumber \\
&&-\,q^{3}\lambda _{+}^{-1}\lambda ^{2}x^{+}\tilde{x}^{3}\left(
x^{0}+q^{-1}\lambda \tilde{x}^{3}\right) D_{q^{2}}^{+}\left(
D_{3,q}^{3}\right) _{l,l+1}^{k+1,k+1}f,  \nonumber \\
\left( \hat{Q}_{2}^{0}\right) _{k,l}f &=&\left( D_{3,q}^{3}\right)
_{l,l+1}^{k+1,k+1}f,  \nonumber \\
\left( \hat{Q}_{3}^{0}\right) _{k,l}f &=&q^{-1}\left( D_{3,q}^{3}\right)
_{l+1,l+1}^{k+1,k}f-q^{2}\lambda _{+}^{-1}\lambda ^{2}x^{+}\tilde{x}%
^{3}D_{q^{2}}^{+}\left( D_{3,q}^{3}\right) _{l+1,l+1}^{k+1,k+1}f,  \nonumber
\\[0.16in]
\left( \hat{O}_{1}^{+}\right) _{k}f &=&\left( D_{1,q^{-1}}^{3}\right)
^{k,k}D_{q^{2}}^{-}f, \\
\left( \hat{O}_{2}^{+}\right) _{k}f &=&x^{+}\tilde{D}_{q^{2}}^{3}\left(
D_{1,q}^{3}\right) ^{k,k+1}f,  \nonumber \\[0.16in]
\left( \hat{Q}_{1}^{+}\right) _{k,l}f &=&\left( q+\lambda _{+}\right)
x^{+}\left( D_{3,q}^{3}\right) _{l,l+1}^{k,k+1}f \\
&&-\,q^{2}\lambda x^{+}\left( x^{0}+q^{-1}\lambda \tilde{x}^{3}\right)
\left( D_{3,q}^{3}\right) _{l,l+1}^{k+1,k+1}f,  \nonumber \\
\left( \hat{Q}_{2}^{+}\right) _{k,l}f &=&x^{+}\left( D_{3,q}^{3}\right)
_{l+1,l+1}^{k+1,k+1}f.  \nonumber
\end{eqnarray}
Finally, the representations of $\left( P^{\mu}\right) ^{-1},$ $\mu=\pm ,0,\tilde{3%
},$ have been formulated by using the following abbreviations: 
\begin{eqnarray}
\left( \tilde{O}_{1}^{p_{3}}\right) _{k}f &=&q^{-2\left( k+1\right) }\left( 
\tilde{O}_{2}^{3}\right) _{k}f, \\
\left( \tilde{O}_{2}^{p_{3}}\right) _{k}f &=&q^{-2\left( k+1\right) }\left( 
\tilde{O}_{3}^{3}\right) _{k}f,  \nonumber \\
\left( \tilde{O}_{3}^{p_{3}}\right) _{k}f &=&q^{2\left( k+1\right) }\left( 
\hat{\tilde{O}^{3}}\right) _{k}f,  \nonumber \\
\left( \tilde{O}_{4}^{p_{3}}\right) _{k}f &=&q^{-2\left( k+1\right) }\left(
\left( \tilde{O}_{1}^{3}\right) _{k}f\right) \left( q^{-2}\tilde{x}%
^{3}\right) ,  \nonumber \\[0.16in]
\left( O_{1}^{p_{+}}\right) _{k}f &=&q^{-2\left( k+1\right) }\left( \left(
O_{1}^{+}\right) _{k}f\right) \left( q^{-2}\tilde{x}^{3}\right) \\
&&+\,q^{2\left( k+1\right) }\left( \left( \hat{O}_{1}^{+}\right) f\right)
\left( q^{2}\tilde{x}^{3}\right) ,  \nonumber \\
\left( O_{2}^{p_{+}}\right) _{k}f &=&q^{-2\left( k+1\right) }\left(
O_{2}^{+}\right) _{k}f,  \nonumber \\
\left( O_{3}^{p_{+}}\right) _{k}f &=&q^{2\left( k+1\right) }\left( \hat{O}%
_{2}^{+}\right) _{k}f,  \nonumber \\[0.16in]
\left( Q_{1}^{p_{+}}\right) _{k,l}f &=&q^{2\left( k+l+1\right) }\left( \hat{Q%
}_{1}^{+}\right) _{k,l}f, \\
\left( Q_{2}^{p_{+}}\right) _{k,l}f &=&q^{2\left( k+l+2\right) }\left( \hat{Q%
}_{2}^{+}\right) _{k,l}f,  \nonumber \\[0.16in]
\left( O_{1}^{p_{0}}\right) _{k}f &=&q^{-2\left( k+1\right) }\left(
O_{1}^{0}\right) _{k}f-q\lambda _{+}^{-1}\lambda q^{2\left( k+1\right)
}\left( \left( \hat{O}_{4}^{0}\right) f\right) \left( q^{2}x^{3}\right) , \\
\left( O_{2}^{p_{0}}\right) _{k}f &=&q^{2\left( k+1\right) }\left( \hat{O}%
_{3}^{0}\right) _{k}f,  \nonumber \\
\left( O_{3}^{p_{0}}\right) _{k}f &=&q^{2\left( k+1\right) }\left( \hat{O}%
_{1}^{0}\right) _{k}f-\lambda _{+}^{-1}\lambda q^{2\left( k+2\right) }\left(
\left( \hat{O}_{2}^{0}\right) f\right) \left( q^{2}x^{3}\right)  \nonumber \\
&&+\,q^{-1}\lambda ^{2}\lambda _{+}^{-1}q^{-2\left( k+1\right) }\left(
O_{3}^{0}\right) _{k}f,  \nonumber \\
\left( O_{4}^{p_{0}}\right) _{k}f &=&q^{-2\left( k+1\right) }\left(
O_{2}^{0}\right) _{k}f,  \nonumber \\[0.16in]
\left( Q_{1}^{p_{0}}\right) _{k,l}f &=&q^{-2\left( k+l+1\right) }\left(
Q_{2}^{0}\right) _{k,l}f, \\
\left( Q_{2}^{p_{0}}\right) _{k,l}f &=&q^{-2\left( k+l+2\right) }\left(
Q_{1}^{0}\right) _{k,l}f-q^{2\left( k+l+2\right) }\left( \hat{Q}%
_{1}^{0}\right) _{l,k}f,  \nonumber \\
\left( Q_{3}^{p_{0}}\right) _{k,l}f &=&\left( 1+q^{-1}\lambda
_{+}^{-1}\right) q^{-2\left( k+l+2\right) }\left( Q_{3}^{0}\right)
_{k,l}f+\lambda _{+}^{-1}q^{2\left( k+l+2\right) }\left( \hat{Q}%
_{3}^{0}\right) _{l,k}f,  \nonumber \\
\left( Q_{4}^{p_{0}}\right) _{k,l}f &=&q^{-2\left( k+l+2\right) }\left(
Q_{4}^{0}\right) _{k,l}f,  \nonumber \\
\left( Q_{5}^{p_{0}}\right) _{k,l}f &=&q\lambda _{+}^{-1}q^{-2\left(
k+l+2\right) }\left( Q_{5}^{0}\right) _{k,l}f-\left( 1+q\lambda
_{+}^{-1}\right) q^{2\left( k+l+2\right) }\left( \hat{Q}_{2}^{0}\right)
_{l,k}f,  \nonumber \\[0.16in]
\left( O_{1}^{p_{-}}\right) _{k}f &=&q^{2\left( k+1\right) }\left( \hat{O}%
_{2}^{-}\right) _{k}f, \\
\left( O_{2}^{p_{-}}\right) _{k}f &=&q^{-2\left( k+1\right) }\left(
O_{1}^{-}\right) _{k}f,  \nonumber \\
\left( O_{3}^{p_{-}}\right) _{k}f &=&q^{-2\left( k+1\right) }\left(
O_{2}^{-}\right) _{k}f,  \nonumber \\
\left( O_{4}^{p_{-}}\right) _{k}f &=&q^{-2\left( k+1\right) }\left(
O_{3}^{-}\right) _{k}f,  \nonumber \\
\left( O_{5}^{p_{-}}\right) _{k}f &=&\lambda _{+}^{-1}q^{2\left( k+1\right)
}\left( \left( \hat{O}_{1}^{-}\right) _{k}f\right) \left( q^{2}\tilde{x}%
^{3}\right) ,  \nonumber \\
\left( O_{6}^{p_{-}}\right) _{k}f &=&q^{-2\left( k+2\right) }\left(
O_{4}^{-}\right) _{k}f,  \nonumber \\[0.16in]
\left( Q_{i}^{p_{-}}\right) _{k,l}f &=&q^{-2\left( k+l+1\right) }\left(
Q_{i}^{-}\right) _{k,l}f,\qquad i=1,\ldots ,4, \\
\left( Q_{i}^{p_{-}}\right) _{k,l}f &=&q^{-2\left( k+l+2\right) }\left(
Q_{i}^{-}\right) _{k,l}f,\qquad i=5,\ldots ,8.  \nonumber
\end{eqnarray}
\end{enumerate}

\section{Scaling Operators\label{AppB}}

For the representations of the covariant differential calculus of three
dimensional Euclidean space we need the following operators:
\begin{eqnarray}
Hf &=&qf-q^{-1}\left[ 2\right] _{q^{2}}f\left( q^{-4}x^{-}\right)
+q^{-2}\left[ 2\right] _{q}f\left( q^{-2}x^{3},q^{-4}x^{-}\right) , \\
H^{-1}f &=&q^{-1}\sum_{k=0}^{\infty }\sum_{l=0}^{k}\left( -1\right)
^{l}q^{-2k-l}{k \choose i}\left( \left[ 2\right] _{q^{2}}\right) ^{k-l}\left(
\left[ 2\right] _{q}\right) ^{l}f\left( q^{-2l}x^{3},q^{-4k}x^{-}\right) , 
\nonumber \\[0.16in]
H^{+}f &=&q^{2}f\left( q^{2}x^{3}\right) +f\left(
q^{-2}x^{+},q^{-4}x^{-}\right) , \\
\left( H^{+}\right) ^{-1}f &=&q^{-2}\sum_{k=0}^{\infty }\left( -1\right)
^{k}q^{-2k}f\left( q^{-4k-2}x^{3},q^{-4k}x^{-}\right) ,  \nonumber \\[0.16in]
H^{3}f &=&q^{2}f\left( q^{2}x^{+}\right) +q^{-2}f\left(
q^{-2}x^{+},q^{-2}x^{3},q^{-4}x^{-}\right) , \\
\left( H^{3}\right) ^{-1}f &=&q^{-2}\sum_{k=0}^{\infty }\left( -1\right)
^{k}q^{-4k}f\left( q^{-4k-2}x^{+},q^{-2k}x^{3},q^{-4k}x^{-}\right) , 
\nonumber \\[0.16in]
H^{-}f &=&q^{2}\left( f+f\left( q^{4}x^{+}\right) \right) -\left[ 2\right]
_{q^{2}}f\left( q^{-4}x^{-}\right) +q^{-1}\left[ 2\right] _{q}f\left(
q^{-2}x^{3},q^{-4}x^{-}\right) , \\
\left( H^{-}\right) ^{-1}f &=&q^{-2}\sum_{i=0}^{\infty }\sum_{j+k+l=i}\left(
-1\right) ^{j+l}\frac{i!}{j!k!l!}q^{-2k-3l}\left( \left[ 2\right]
_{q^{2}}\right) ^{k}\left( \left[ 2\right] _{q}\right) ^{l}  \nonumber \\
&&\times f\left( q^{-4\left( i+k+l+1\right) }x^{+},q^{-2l}x^{3},q^{-4\left(
k+l\right) }x^{-}\right)  \nonumber
\end{eqnarray}
where $\left[2\right]_{q^a}=q^a+q^{-a},\, a\in\mathbb{C}$.
It is worth noting that this fomulae are only valid for $q>1.$

Furthermore, we introduce the functions 
\begin{equation}
W_{\left[ a,b\right] }(x)\equiv \frac{1}{\sqrt{2\pi }}\int\limits_{-\infty
}^{\infty }F\left( y\right) e^{ixy}dy=\left\{ 
\begin{array}{c}
1, \\ 
0,
\end{array}
\right. \left. 
\begin{array}{l}
\text{if }a\leq x\leq b, \\ 
\text{otherwise}
\end{array}
\right. 
\end{equation}
where 
\begin{equation}
F\left( y\right) =\left\{ 
\begin{array}{c}
\frac{i}{y\sqrt{2\pi }}\left( e^{-iby}-e^{-iay}\right) ,\quad \text{if }%
y\neq 0, \\ 
\frac{b-a}{\sqrt{2\pi }},\quad \text{if }y=0,
\end{array}
\right. .
\end{equation}
With these functions at hand we can also define 
\begin{eqnarray}
W_{\left] 0,1\right] }\left( x\right)  &\equiv &W_{\left[ -1,1\right]
}\left( x\right) -W_{\left[ -1,0\right] }\left( x\right) , \\
W_{\left[ -1,0\right[ }\left( x\right)  &\equiv &W_{\left[ -1,1\right]
}\left( x\right) -W_{\left[ 0,1\right] }\left( x\right) ,  \nonumber \\
W_{\overline{\left[ 0,1\right] }}\left( x\right)  &\equiv &1-W_{\left[
0,1\right] }\left( x\right) ,  \nonumber \\
W_{0}\left( x\right)  &\equiv &W_{\left[ -1,0\right] }\left( x\right)
+W_{\left[ 0,1\right] }\left( x\right) -W_{\left[ -1,1\right] }\left(
x\right) .  \nonumber
\end{eqnarray}
Using the operators \cite{WW01} 
\begin{equation}
\hat{\sigma}_{i}\equiv x^{i}\frac{\partial }{\partial x^{i}},\quad
i=1,\ldots ,4,
\end{equation}
and
\begin{equation}
\hat{L}^{i_1,i_4}_{i_2,i_3} f\equiv f(q^{i_1} x^1,q^{i_2}
x^2,q^{i_3}x^3,q^{i_4}x^4),
\quad i_1,i_2,i_3,i_4 = 1,\dots ,4,
\end{equation}
the scaling operators needed for q-deformed Euclidean space in four
dimensions take the form 
\begin{eqnarray}
Nf &=&f\left( q^{-2}x^{4}\right) +q\lambda \left(
x^{2}D_{q^{2}}^{2}+x^{3}D_{q^{2}}^{3}\right) f+q^{-2}\lambda
^{2}x^{4}D_{q^{-2}}^{4}f, \\
N^{-1}f &=&\frac{1}{2}q^{2}W_{0}\left(\hat{S}_1^0
\right) f\left( q^{2}x^{4}\right)   \nonumber \\
&&\hspace*{-1.2cm}+\,q^{2}\sum_{k=0}^{\infty }\left( -q^{2}\right)
^{k}\sum_{l_{1}+l_{2}+l_{3}=k}\frac{k!}{l_{1}!l_{2}!l_{3}!}\left(
-q^{-1}\lambda _{+}\right) ^{l_{3}} W_{\left] 0,1\right] }\left(\hat{S}_1^0
\right) \hat{L}^{0,2(k+1)}_{2l_1,2l_2}f 
\nonumber \\ 
%&&\hspace*{-1.2cm}f\left( q^{2l_{1}}x^{2},q^{2l_{2}}x^{3},q^{2\left(
%k+1\right) }x^{4}\right)   \nonumber \\
&&\hspace*{-1.2cm}+\,\sum_{k=0}^{\infty }\left( -q^{-2}\right)
^{k}\sum_{l=0}^{\infty }{-k-1 \choose l}\left( -q^{-1}\lambda _{+}\right) ^{l}
W_{\overline{\left[ 0,1\right] }}\left(\hat{S}_1^0
 \right)   \nonumber \\
&&\hspace*{-.0cm}\times\bigg \{ 2^{-\left( k+l+1\right) }W_{0}\left(
\hat{S}_2^0\right) \hat{L}^{0,-2k}_{-(k+l+1),-(k+l+1)} f
% f\left(
%q^{-\left( k+l+1\right) }x^{2},q^{-\left( k+l+1\right)
%}x^{3},q^{-2k}x^{4}\right) 
  \nonumber \\
&&\hspace*{-.0cm}+\sum_{u=0}^{\infty }{-k-l-1 \choose u}\left[ W_{\left]
0,1\right] }\left(
\hat{S}_2^0
\right) \hat{L}^{0,-2k}_{-2(k+l+u+1),2u} f
% f\left( q^{-2\left( k+l+u+1\right)
%}x^{2},q^{2u}x^{3},q^{-2k}x^{4}\right)
 \right.
   \nonumber \\
&&\hspace*{-.0cm}
\left. +\,W_{\left] 0,1\right] }\left(
\hat{S}_3^0
\right) \hat{L}^{0,-2k}_{2u,-2(k+l+u+1)} f
%f\left( q^{2u}x^{2},q^{-2\left(
%k+l+u+1\right) }x^{3},q^{-2k}x^{4}\right)
 \right] \bigg\},  \nonumber \\%
[0.16in]
N^{1}f &=&q^{-3}f\left( qx^{2},qx^{3}\right) +q^{3}f\left(
q^{-1}x^{2},q^{-1}x^{3},q^{-2}x^{4}\right) , \\
\left( N^{1}\right) ^{-1}f &=&\frac{1}{2}q^{3}W_{0}\left( 
\hat{S}^1
\right) 
f\left(
q^{-1}x^{2},q^{-1}x^{3}\right)   \nonumber \\
&&\hspace*{-1.2cm}+\,q^{3}\sum_{k=0}^{\infty }\left( -q^{6}\right)
^{k}W_{\left] 0,1\right] }\left(
\hat{S}^1
\right) \hat{L}^{0,-2k}_{-(2k+1),-2(k+1)} f
%\left( q^{-\left( 2k+1\right)
%}x^{2},q^{-2\left( k+1\right) }x^{3},q^{-2k}x^{4}\right) 
  \nonumber \\
&&\hspace*{-1.2cm}+\,q^{-3}\sum_{k=0}^{\infty }\left( -q^{-6}\right) ^{k}W_{%
\overline{\left[ 0,1\right] }}\left(
\hat{S}^1
\right)  \hat{L}^{0,2(k+1)}_{2k+1,2k+1} f
%\left(q^{2k+1}x^{2},q^{2k+1}x^{3},q^{2\left( k+1\right) }x^{4}\right)
 ,  \nonumber
\\[0.16in]
N^{2}f &=&q^{-3}f\left( qx^{1},q^{2}x^{4}\right) +q^{3}f\left(
q^{-1}x^{1},q^{-2}x^{3}\right) , \\
\left( N^{2}\right) ^{-1}f &=&\frac{1}{2}q^{3}W_{0}\left(
\hat{S}^2
\right) f\left(
q^{-1}x^{2},q^{-2}x^{4}\right)   \nonumber \\
&&\hspace*{-1.2cm}+\,q^{3}\sum_{k=0}^{\infty }\left( -q^{6}\right)
^{k}W_{\left] 0,1\right] }\left( 
\hat{S}^2
\right) \hat{L}^{-(2k+1),-(2k+1)}_{0,-2k} f
%f\left( q^{-\left( 2k+1\right)
%}x^{1},q^{-2k}x^{3},q^{-2\left( k+1\right) }x^{4}\right)  
 \nonumber \\
&&\hspace*{-1.2cm}+\,q^{-3}\sum_{k=0}^{\infty }\left( -q^{-6}\right) ^{k}W_{%
\overline{\left[ 0,1\right] }}\left( 
\hat{S}^2
\right) \hat{L}^{2k+1,2k}_{0,2(k+1)} f
%f\left(
%q^{2k+1}x^{1},q^{2\left( k+1\right) }x^{3},q^{2k}x^{4}\right) ,
  \nonumber \\%
[0.16in]
N^{3}f &=&q^{-3}f\left( qx^{1},q^{2}x^{4}\right) +q^{3}f\left(
q^{-1}x^{1},q^{-2}x^{2}\right) , \\
\left( N^{3}\right) ^{-1}f &=&\frac{1}{2}q^{3}W_{0}\left(
\hat{S}^3
\right)  f\left(
q^{-1}x^{2},q^{-2}x^{4}\right)   \nonumber \\
&&\hspace*{-1.2cm}+\,q^{3}\sum_{k=0}^{\infty }\left( -q^{6}\right)
^{k}W_{\left] 0,1\right] }\left(
\hat{S}^3
\right) \hat{L}^{-2(k+1),-2(k+1)}_{-2k,0}f
%f\left( q^{-2\left( k+1\right)
%}x^{1},q^{-2k}x^{2},q^{-2\left( k+1\right) }x^{4}\right)
   \nonumber \\
&&\hspace*{-1.2cm}+\,q^{-3}\sum_{k=0}^{\infty }\left( -q^{-6}\right) ^{k}W_{%
\overline{\left[ 0,1\right] }}\left(
\hat{S}^3
\right) \hat{L}^{2k+1,2k}_{2(k+1),0} f
%f\left(
%q^{2k+1}x^{1},q^{2\left( k+1\right) }x^{2},q^{2k}x^{4}\right) 
, \nonumber \\%
[0.16in]
N^{4}f &=&q^{-3}f\left( q^{-2}x^{4}\right) +q^{3}f\left(
q^{-2}x^{1},q^{-2}x^{4}\right)  \\
&&\hspace*{-1.2cm}+\,q^{-2}\lambda \left(
x^{2}D_{q^{2}}^{2}+x^{3}D_{q^{2}}^{3}\right) f+q^{-5}\lambda
^{2}x^{4}D_{q^{-2}}^{4}f,  \nonumber \\
\left( N^{4}\right) ^{-1}f &=&\frac{1}{2}q^{5}W_{0}\left(\hat{S}^4_1
\right) f\left( q^{2}x^{4}\right)  
\nonumber \\
&&\hspace*{-1.2cm}+\,q^{5}\sum_{k=0}^{\infty }\left( -q^{5}\right)
^{k}\sum_{l_{1}+\ldots +l_{4}=k}\frac{k!}{l_{1}!\ldots l_{4}!}\left(
-q^{-4}\lambda _{+}\right) ^{l_{1}}q^{3\left( l_{2}-l_{3}-l_{4}\right) } 
\nonumber \\
&&\hspace*{-.0cm}\times
 W_{\left] 0,1\right] }\left(\hat{S}^4_1
 \right) \hat{L}^{-2l_2,2(k-m_2+1)}_{2l_3,2l_4} f
%f\left(
%q^{-2l_{2}}x^{1},q^{2l_{3}}x^{2},q^{2l_{4}}x^{3},q^{2\left( k-m_{2}+1\right)
%}x^{4}\right) 
  \nonumber \\
&&\hspace*{-1.2cm}+\,\frac{1}{2}q^{-3}\sum_{k=0}^{\infty }\left(
-2q^{-8}\right) ^{k}W_{\overline{\left[ 0,1\right] }}\left(\hat{S}^4_1
\right) 
% \nonumber \\
%&\hspace*{-1.2cm}\times 
W_{0}\left(
\hat{S}^4_2 
\right) \hat{L}^{2(k+1),2}_{0,0} f
%f\left( q^{2\left( k+1\right) }x^{1},q^{2}x^{4}\right)  
\nonumber \\
&&\hspace*{-1.2cm}+\,q^{-3}\sum_{k=0}^{\infty }\left( -q^{-8}\right)
^{k}\sum_{l=0}^{\infty }{-k-1 \choose l}q^{-6l}\sum_{m_{1}+m_{2}+m_{3}=l}%
\frac{l!}{m_{1}!m_{2}!m_{3}!}\left( -q^{-1}\lambda _{+}\right) ^{m_{1}} 
\nonumber\\&&\hspace*{-0.cm}
\times W_{\overline{\left[ 0,1\right] }}\left(\hat{S}^4_1
 \right)  
% \nonumber \\
%&&\hspace*{-1.2cm}\times
 W_{\left] 0,1\right] }\left( 
\hat{S}^4_2
\right) \hat{L}^{2(k+l+1),2(l+1)}_{2m_2,2m_3} f
\nonumber \\
%&&\hspace*{-1.2cm}\times f\left( q^{2\left( k+l+1\right)
%}x^{1},q^{2m_{2}}x^{2},q^{2m_{3}}x^{3},q^{2\left( l+1\right) }x^{4}\right)  
%\nonumber \\
&&\hspace*{-1.2cm}+\,q^{3}\sum_{k=0}^{\infty }\left( -q^{-2}\right)
^{k}\sum_{l=0}^{\infty }{-k-1 \choose l}q^{6l}\sum_{u=0}^{\infty
}{-k-l-1 \choose u}\left( -q^{-1}\lambda _{+}\right) ^{u}\nonumber\\
&&\hspace*{-.0cm}\times W_{\overline{\left[ 0,1\right] }}\left(\hat{S}^4_1
\right)  
% \nonumber \\
%&&\hspace*{-1.2cm}\times
 W_{\overline{\left[ 0,1\right] }}\left( 
\hat{S}^4_2 
\right)\nonumber\\ 
&&\hspace*{-0.cm}\times \bigg \{ 2^{-\left(
k+l+u+1\right) }  W_{0}\left(
\hat{S}^4_3
\right) \hat{L}^{-2l,-2(k+l)}_{-(k+l+u+1),-(k+l+u+1)} f
%f\left( q^{-2l}x^{1},q^{-\left( k+l+u+1\right)
%}x^{2},q^{-\left( k+l+u+1\right) }x^{3},q^{-2\left( k+l\right) }x^{4}\right) 
\nonumber \\
&&\hspace*{-.0cm}+\sum_{v=0}^{\infty }{-k-l-u-1 \choose v} 
 \left[ W_{\left] 0,1\right] }\left( 
\hat{S}^4_3
\right) \hat{L}^{-2l,-2(k+l)}_{2v,-2(k+l+u+v+1)}\right.
% f\left(
%q^{-2l}x^{1},q^{2v}x^{2},q^{-2\left( k+l+u+v+1\right) }x^{3},q^{-2\left(
%k+l\right) }x^{4}\right) \right.
   \nonumber \\
&&\hspace*{-.0cm}\left.
+\,W_{\overline{\left[ 0,1\right]} }\left( 
\hat{S}^4_3  
\right) \hat{L}^{-2l,-2(k+l)}_{-2(k+l+u+v+1),2v}f
%f\left(
%q^{-2l}x^{1},q^{-2\left( k+l+u+v+1\right) }x^{3},q^{-2\left( k+l\right)
%}x^{4}\right)
 \right] \bigg\}  \nonumber
\end{eqnarray}
where
\begin{eqnarray}
\hat{S}_1^0&=&q^{-2}-q^{2\hat{\sigma}_{4}}\left(
q^{2\hat{\sigma}_{2}}+q^{2\hat{\sigma}_{3}}-q^{-1}\lambda
_{+}\right),\\
\hat{S}_2^0&=& 1-q^{2\left( \hat{%
\sigma}_{3}-\hat{\sigma}_{2}\right) },
%\nonumber\\
\nonumber\\[0.16in]
\hat{S}^1&=& 
q^{-6}-q^{-2\left( 
\hat{\sigma}_{2}+\hat{\sigma}_{3}+\hat{\sigma}_{4}\right) },\\[0.16in]
\hat{S}^2&=& q^{-6}-q^{-2\left( 
\hat{\sigma}_{1}+\hat{\sigma}_{3}+\hat{\sigma}_{4}\right) },\\[0.16in]
\hat{S}^3&=& q^{-6}-q^{-2\left( 
\hat{\sigma}_{1}+\hat{\sigma}_{2}+\hat{\sigma}_{4}\right) },\\[0.16in]
\hat{S}^4_1&=&q^{-5}-q^{3-2\hat{%
\sigma}_{1}}-q^{-3+2\hat{\sigma}_{4}}\left( q^{2\hat{\sigma}_{2}}+q^{2\hat{%
\sigma}_{3}}-q^{-1}\lambda _{+}\right),\\
\hat{S}_2^4&=& q^{3}-q^{-3+2\left(\hat{\sigma}_1+\hat{%
\sigma}_{4}\right)}\left( q^{2\hat{\sigma}_{2}}+
q^{2\hat{\sigma}_{3}}-q^{-1}\lambda_{+}\right),\nonumber\\
\hat{S}^4_3&=& 1-q^{2\left( \hat{%
\sigma}_{2}-\hat{\sigma}_{3}\right)}
%\nonumber\\
\nonumber . 
\end{eqnarray}
Again we have to emphasize that these expressions hold for $q>1$ only.

Finally, in the case of q-deformed Minkowski space we need the following
operators to perform our integrals: 
\begin{eqnarray}
J^{-}f &=&f\left( q^{-2}x^{+}\right) +q^{2}f, \\
\left( J^{-}\right) ^{-1} &=&q^{-2}\sum_{k=0}^{\infty }\left( -q^{-2}\right)
^{k}f\left( q^{-2k}x^{+}\right) ,  \nonumber \\[0.16in]
J^{+}f &=&f\left( q^{-2}\tilde{x}^{3},q_{-}x^{3},q^{-2}x^{-}\right)
+q^{2}f\left( q_{-}q^{2}x^{3}\right) , \\
\left( J^{+}\right) ^{-1}f &=&q^{-2}\sum_{k=0}^{\infty }\left(
-q^{-2}\right) ^{k}f\left( q_{-}^{-1}q^{-2\left( k+1\right) }x^{3},q^{-2}%
\tilde{x}^{3},q^{-2}x^{-}\right) ,  \nonumber \\[0.16in]
J_{k}^{0}f &=&\frac{\left( \tilde{x}^{3}\partial _{3}\right) ^{k}}{k!}\left(
q^{2}+\left( -1\right) ^{k}q^{-2\left( \hat{\sigma}_{+}+\hat{\tilde{\sigma}%
_{3}}+\hat{\sigma}_{-}\right) }\right) , \\
\left( J_{0}^{0}\right) ^{-1}f &=&q^{-2}\sum_{k=0}^{\infty }\left(
-q^{-2}\right) ^{k}f\left( q^{-2k}x^{+},q^{-2k}\tilde{x}^{3},q^{-2k}x^{-}%
\right) ,  \nonumber \\
\left( J^{0}\right) ^{-1}f &=&\left( J_{0}^{0}\right)
^{-1}f-\sum_{k=1}^{\infty }\left( \frac{\lambda }{\lambda _{+}}\right)
^{k}\left( J_{0}^{0}\right) ^{-1}J_{k}^{0}\left( J_{0}^{0}\right) ^{-1}f 
\nonumber \\
&&-\sum_{k=0}^{\infty }\left( \frac{\lambda }{\lambda _{+}}\right)
^{k+1}\sum_{l=1}^{k}\left( -1\right) ^{l}\sum_{0\leq m_{1}+\ldots +m_{l}\leq
k-l}\left( \prod_{i=1}^{l}\left( J_{0}^{0}\right)
^{-1}J_{m_{i}+1}^{0}\right)   \nonumber \\
&&\times \left( J_{0}^{0}\right) ^{-1}J_{k+1-l-m_{1}-\ldots
-m_{l}}^{0}\left( J_{0}^{0}\right) ^{-1}f,  \nonumber \\
[.16in]\tilde{J}_{0}^{3}f &=&f\left( q^{-2}x^{3}\right) +q^{2}f\left(
q^{2}x^{+}\right) ,  \\
\tilde{J}_{k}^{3}f &=&\left( \tilde{x}^{3}\right) ^{k}\left[ \left(
D_{1,q^{2}}^{3}\right) ^{\left( k,1\right) }f\left( q^{-2}x^{3}\right)
+\left( -1\right) ^{k}\left( D_{1,q^{-2}}\right) ^{\left( k,1\right)
}f\left( q^{2}x^{+},q^{2}x^{3}\right) \right] ,  \nonumber \\
\left( \tilde{J}_{0}^{3}\right) ^{-1}f &=&q^{-2}\sum_{k=0}^{\infty }\left(
-q^{-2}\right) ^{k}f\left( q^{-2\left( k+1\right) }x^{+},q^{-2k}x^{3}\right)
,  \nonumber \\
\left( \tilde{J}^{3}\right) ^{-1}f &=&\left( \tilde{J}_{0}^{3}\right)
^{-1}f-\sum_{k=1}^{\infty }\left( \frac{\lambda }{\lambda _{+}}\right)
^{k}\left( \tilde{J}_{0}^{3}\right) ^{-1}\left( D_{q^{2}}^{3}\right) ^{-1}%
\tilde{J}_{k}^{3}\left( \tilde{J}_{0}^{3}\right) ^{-1}f  \nonumber \\
&&-\sum_{k=1}^{\infty }\left( \frac{\lambda }{\lambda _{+}}\right)
^{k+1}\sum_{l=1}^{k}\left( -1\right) ^{l}\sum_{0\leq m_{1}+\ldots +m_{l}\leq
k-l}\left( \tilde{J}_{0}^{3}\right) ^{-1}\left( D_{q^{2}}^{3}\right) ^{-1} 
\nonumber \\
&&\times \left( \prod_{i=1}^{l}\tilde{J}_{m_{i}+1}^{3}\left( \tilde{J}%
_{0}^{3}\right) ^{-1}\left( D_{q^{2}}^{3}\right) ^{-1}\right) \tilde{J}%
_{k+1-l-m_{1}-\ldots -m_{l}}^{3}\left( \tilde{J}_{0}^{3}\right) ^{-1}f 
\nonumber
\end{eqnarray}
where again $q>1.$ Let us also note that the product $%
\prod_{i=1}^{n}\hat{a}_{i}$ of non-commuting operators $\hat{a}_{i}$ is
given by 
\begin{equation}
\prod_{i=1}^{n}\hat{a}^{i}\equiv \hat{a}^{1}\cdot \ldots \cdot \hat{a}^{n}.
\end{equation}
\textbf{Acknowledgement}\newline
First of all I want to express my gratitude to Julius Wess for his efforts,
suggestions and discussions. And I would like to thank Michael Wohlgenannt,
Fabian Bachmeier, Christian Blohmann, and Marcus Dietz for useful
discussions and their steady support.

\begin{thebibliography}{99}
\bibitem{RTF90}  N.Yu. Reshetikhin, L.A. Takhtadzhyan, L.D. Faddeev, \textit{%
Quantization of Lie Groups and Lie Algebras}, Leningrad Math. J. \textbf{1}
(1990) 193.

\bibitem{Wes00}  J. Wess, \textit{q-deformed Heisenberg Algebras}, in H.
Gausterer, H. Grosse and L.Pittner, eds., Proceedings of the 38.
Internationale Universit\"{a}tswochen f\"{u}r Kern- und Teilchen physik, no.
543 in Lect. Notes in Phys., Springer -Verlag, Schladming (2000), math-ph
9910013.

\bibitem{Maj93}  S. Majid,\textit{\ Braided geometry: A new approach to
q-deformations, }in R. Coquereaux et al., eds., First Caribbean Spring
School of Mathematics and Theoretical\ Physics, World Sci., Guadeloupe
(1993).

\bibitem{Moy49}  J.E. Moyal, \textit{Quantum mechanics as a statistical
theory, }Proc. Camb. Phil. Soc. \textbf{45} (1949) 99.

\bibitem{MSSW00}  J. Madore, S. Schraml, P. Schupp, J. Wess, \textit{Gauge
Theory on Noncommutative Spaces, }Eur. Phys. J. C\textbf{16} (2000) 161,
hep-th/0103120.

\bibitem{WW01}  H. Wachter, M. Wohlgenannt, \textit{*-Products on Quantum
Spaces, }hep-th/0103120, to appear in Eur. Phys. Journal C.

\bibitem{Dri85}  V.G. Drinfeld, \textit{Hopf algebras and the quantum
Yang-Baxter equation, }Soviet Math. Dokl. \textbf{32} (1985)  254-258.

\bibitem{Jim85}  M. Jimbo, \textit{A q-analogue of U(g) and the Yang-Baxter
equation,} Lett. Math. Phys. \textbf{10} (1985) 63-69.

\bibitem{WZ91}  J. Wess, B. Zumino,\textit{\ Covariant differential calculus
on the quantum hyperplane}, Nucl. Phys. B. Suppl. \textbf{18} (1991)
302-312.

\bibitem{BW01}  C. Bauer, H. Wachter, \textit{Operator Representations on
Quantum Spaces, }math-ph/0201023.

\bibitem{Jac27}  F.H. Jackson, \textit{q-Integration,} Proc. Durham Phil.
Soc. \textbf{7} (1927)  182-189.


\bibitem{LWW97}  A. Lorek, W. Weich, J. Wess, \textit{Non Commutative
Euclidean and Minkowski Structures, }Z. Phys. C\textbf{76} (1997) 375,
q-alg/9702025.

\bibitem{KS97}  A. Klimyk, K. Schm\"{u}dgen, \textit{Quantum Groups and
their Representations,} Springer Verlag, Berlin (1997).

\bibitem{Maj95}  S. Majid, \textit{Foundations of Quantum Group Theory, }%
University Press, Cambridge (1995).

\bibitem{Oca96}  H. Ocambo, \textit{SO}$_{q}$(4) \textit{quantum mechanics},
Z. Phys. C\textbf{70} (1996) 525.

\bibitem{OSWZ92}  O. Ogievetsky, W.B. Schmittke, J. Wess, B. Zumino, \textit{%
q-deformed Poincar\'{e} Algebra,} Commun. Math. Phys. \textbf{150} (1992) 495.

\bibitem{SWZ91}  W.B. Schmidtke, J. Wess, B. Zumino, \textit{A q-deformed
Lorentz Algebra in Minkowski phase Space, }Z. Phys.  C\textbf{52 }(1991) 471.

\bibitem{RW99}  M. Rohregger, J. Wess, \textit{q-Deformed Lorentz Algebra in
Minkowski phase space, }Eur. Phys. J. \textbf{7} (1999) 177.

\bibitem{Ste96}  H. Steinacker, \textit{Integration on quantum Euclidean
Space and sphere,} J. Math. Phys. \textbf{37} (1996) 4738.

\bibitem{CZ93}  C. Chryssomalakos, B. Zumino, \textit{Translations integrals
and Fourier transforms in the quantum plane, }preprint LBL-34803,
UCCB-PTH-93/30, in Salamfestschrift, edited by A. Ali, J. Ellis, and S.
Randjbar-Daemi, World Sci., Singapore, 1993.

\bibitem{Fio93}  G. Fiore, \textit{The SO}$_{q}$\textit{(N)-symmmetric
harmonic oszillator on the quantum Euclidean space...,} Int. J. Mod. Phys. A%
\textbf{8} (1993) 4679.

\bibitem{HW92}  A. Hebecker, W. Weich, \textit{Free particle in q-deformed
configuration space, }Lett. Math. Phys. \textbf{26} (1992) 245.

\bibitem{Die01}  M.A. Dietz, \textit{Symmetrische Formen auf
Quantenalgebren, }diploma-thesis, Universit\"{a}t Hamburg, Fakult\"{a}t
f\"{u}r Physik (2001).

\bibitem{KM94}  A. Kempf, S. Majid, \textit{Algebraic q-integration and
Fourier theory on quantum and braided spaces, }J. Math. Phys. \textbf{35}
(1994) 6802.
\end{thebibliography}

\end{document}

%%%%%%%%%%%%%%%%%%%%%% End \document\NeueVer.tex %%%%%%%%%%%%%%%%%%%%%%

