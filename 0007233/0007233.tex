\documentclass[a4paper,12pt]{article}

\usepackage{epsfig}
\usepackage{amsmath}
\usepackage{amssymb}

\setlength{\voffset}{-1cm}
\setlength{\evensidemargin}{0mm}
\setlength{\oddsidemargin}{0mm}
\setlength{\textwidth}{16cm}
\setlength{\textheight}{22cm}
\setlength{\floatsep}{0pt}
\setlength{\parskip}{3mm}
\renewcommand{\baselinestretch}{1.2}

\providecommand{\ads}{\text{AdS}}
\providecommand{\ap}{\alpha^\prime}
\providecommand{\Mink}{\mathbb{R}^{1,9}}
\providecommand{\Ibar}{\bar{I}}
\providecommand{\Jbar}{\bar{J}}
\providecommand{\Kbar}{\bar{K}}
\providecommand{\Lbar}{\bar{L}}
\providecommand{\Mbar}{\bar{M}}
\providecommand{\Nbar}{\bar{N}}
\providecommand{\Abar}{\bar{A}}
\providecommand{\Bbar}{\bar{B}}
\providecommand{\Cbar}{\bar{C}}
\providecommand{\Dbar}{\bar{D}}
\providecommand{\Ebar}{\bar{E}}
\providecommand{\Fbar}{\bar{F}}
\providecommand{\Gbar}{\bar{G}}
\providecommand{\etat}{\tilde{\eta}}
\providecommand{\epsh}{\hat{\epsilon}}
\providecommand{\Gamh}{\hat{\Gamma}}
\providecommand{\eh}{\hat{e}}
\providecommand{\gh}{\hat{g}}
\providecommand{\ab}{\bar{a}}
\providecommand{\bb}{\bar{b}}
\providecommand{\cb}{\bar{c}}
\providecommand{\db}{\bar{d}}
\providecommand{\eb}{\bar{e}}
\providecommand{\Ib}{\bar{I}}
\providecommand{\Jb}{\bar{J}}
\providecommand{\Kb}{\bar{K}}
\providecommand{\Ab}{\bar{A}}
\providecommand{\Bb}{\bar{B}}
\providecommand{\Cb}{\bar{C}}
\providecommand{\mub}{\bar{\mu}}
\providecommand{\nub}{\bar{\nu}}
\providecommand{\rhob}{\bar{\rho}}
\providecommand{\btil}{{\tilde b}}
\providecommand{\ctil}{\tilde c}
\providecommand{\etil}{\tilde e}
\providecommand{\ftil}{{\tilde f}}
\providecommand{\gtil}{\tilde g}
\providecommand{\psitil}{\tilde \psi}
\providecommand{\Gam}{\Gamma}
\providecommand{\psibar}{\bar{\psi}}
\providecommand{\Psibar}{\bar{\Psi}}
\providecommand{\chibar}{\bar{\chi}}
\providecommand{\zetabar}{\bar{\zeta}}
\providecommand{\lambdabar}{\bar{\lambda}}
\providecommand{\pslash}{\hspace{-1.9mm} \not \!p}
\providecommand{\Pslash}{\hspace{-1.9mm} \not \!P}
\providecommand{\qslash}{\hspace{-1.9mm} \not \!q}
\providecommand{\epsslash}{\hspace{-1.9mm} \not \!\epsilon}
\providecommand{\Prefac}{\frac{1}{(4\pi)^{5/3} \kappa^{1/3}}}
\providecommand{\Ztwo}{\mathbb{Z}_2}
\providecommand{\mydot}{\hspace{-1mm}\cdot\hspace{-1mm}}
\providecommand{\tr}{\text{tr}}


\begin{document}

\begin{titlepage}
\begin{flushright}
HU-EP-00/22 \\
\today
\end{flushright}

\vspace{1cm}
\begin{center}
\baselineskip25pt
{\Large\bf A Small Cosmological Constant and Backreaction of
           Non-Finetuned Parameters}
\end{center}
\vspace{1cm}
\begin{center}
\baselineskip12pt
{Axel Krause\footnote{E-mail: axkrause@physik.hu-berlin.de}}
\vspace{.3truecm}
\vspace{1cm}

{\it Humboldt-Universit\"{a}t, Institut f\"{u}r Physik, D-10115 Berlin,
     Germany}

\vspace{0.3cm}
\end{center}
\vspace*{\fill}

\begin{abstract}
We include the backreaction on the warped geometry induced by
non-finetuned parameters for a recently proposed mechanism to obtain
an exponentially small cosmological constant $\Lambda_4$. It is shown
that by separating two domain-walls by a distance $2l$ the
cosmological constant appears exponentially suppressed with
suppression-length $l$. Thus no huge hierarchy is required to obtain a
realistic $\Lambda_4$. Moreover, we find a smooth connection to the
limit with finetuned parameters.
\end{abstract}

\noindent
PACS: 4.50.+h, 11.10.Kk \\
Keywords: Cosmological Constant, Warp-Geometry, Extra Dimensions \\
Los Alamos Database Number: hep-th/0007233

\vspace*{\fill}

\end{titlepage}


Recently, a mechanism for obtaining the small observed value of the
cosmological constant $\Lambda_4 \simeq 10^{-47}\text{GeV}^4$ has been
proposed in \cite{AK3}. It requires a five-dimensional (or higher)
set-up consisting of two four-dimensional positive-tension $T$
domain-walls (there is no need for either the bulk or the walls to be
supersymmetric) separated by a distance $2l$ along the fifth noncompact
dimension. Together with bulk gravity and a non-positive bulk
cosmological constant $\Lambda(x^5)\le 0$ the set-up is described by
the action
\begin{equation}
  S=-\int d^5x \left( \sqrt{G}\left[ M^3R(G)+\Lambda(x^5) \right]
                     +\sqrt{g^{(4)}}T\left[ \delta(x^5+l)+\delta(x^5-l)
                                     \right]
               \right) \; .
  \label{WallAction}
\end{equation}
Neither of the walls is conceived as hidden but instead they are
together responsible for the remaining three forces beyond gravity. By
a string-embedding of the set-up and the realization of the
domain-walls as two stacks of D3-branes, one can think of the
Standard-Model gauge group SU(3) arising from one stack and the
$\text{SU(2)}\times\text{U(1)}$ from the other \cite{AK3}. For
finetuned parameters the set-up leads to a warped geometry containing
a flat 4-dimensional spacetime
\begin{alignat}{3}
  ds^2 &= e^{-A(x^5)}\eta_{\mu\nu}dx^\mu dx^\nu + (dx^5)^2 \;\; ; \;\;
                            \mu,\nu=1,\hdots,4 \notag \\
  A(x^5) &= \frac{k}{2}
            \left( |x^5+l|+|x^5-l| \right) \;\; , \;\;
  k = \sqrt{-\Lambda_e/3M^3}
  \label{FlatSolution}
\end{alignat}
with the bulk cosmological constant $\Lambda(x^5)$ and wall-tension $T$ given
by
\begin{equation}
  \Lambda(x^5) = 
            \left\{ \begin{array}{cc}
      \Lambda_e   &, \;\;|x^5| > l \\
      \Lambda_e/4 &, \;\;|x^5| = l \\
          0       &, \;\;|x^5| < l
                    \end{array}
            \right.   
  \; , \qquad   
  T=\sqrt{-3M^3 \Lambda_e} \; . 
  \label{Tension}
\end{equation}
A non-vanishing $\Lambda$ in the interior region $|x^5|<l$ could be
turned on if the tension of both walls would no longer be equal
\cite{AK3}.  The exponential warp-factor eventually accounts for an
exponential suppression by $e^{-kl}$ of the effective 4-dimensional
$\Lambda_4$ if the finetuning is suspended to such a degree that the
backreaction on the warp-factor is negligible. The more we pull the
two walls apart (enlarge the thickness of our world) the
more we are able to lower $\Lambda_4$. Due to the exponential factor
no huge hierarchy is required and a distance $2l\simeq
1/M_{\text{GUT}}$ is already enough to obtain a realistic $\Lambda_4$.
Note, that unlike other approaches to achieve a zero/small
cosmological constant (e.g.~\cite{ADKS},\cite{KSS}), we do not need
the addition of a further bulk scalar field.

In this letter, we want to determine the full backreaction of
non-finetuned parameters on the warped geometry and demonstrate that
the resulting $\Lambda_4$ still comes out exponentially
suppressed. Therefore, the mechanism to obtain a hierarchically small
$\Lambda_4$ by separating the two walls sufficiently far from each
works without the ususal finetuning between fundamental bulk
parameters and wall tensions.

To this aim, we have to determine the resulting 5-dimensional geometry for
general non-positive $\Lambda \le 0$ and positive $T>0$.
Let us start with a $D$-dimensional warped geometry
\begin{equation}
  ds^2=G_{MN}dx^M dx^N=f(x^D)g_{\mu\nu}(x^\rho)dx^\mu dx^\nu + (dx^D)^2 \; ,
\end{equation}
with $\mu,\nu,\rho=1,\hdots,D-1$ and the warp-factor
$f(x^D)$. The induced metric on a $(D-1)$-dimensional section defined by
$x^D=$ const, will be denoted by $g^{(D-1)}_{\mu\nu}(x^\rho,x^D)
=f(x^D)g_{\mu\nu}(x^\rho)$. Eventually, we want to solve the Einstein equation
to determine the lower-dimensional $\Lambda_4$
for the case $D=5$. Therefore, we decompose the $D$-dimensional Ricci-tensor
$R_{MN}$ into its $\mu$ and $D$ components
\begin{alignat}{3}
  R_{\mu\nu}(G) &= R_{\mu\nu}(g)+\frac{1}{4}g_{\mu\nu}
            \left( 2f^{\prime\prime}+(D-3)f\left[(\ln f)^\prime\right]^2 
            \right) \notag \\
  R_{\mu D}(G) &= 0 \\
  R_{DD}(G) &= \frac{1}{4}(D-1)\left( 2\frac{f^{\prime\prime}}{f}
                                    - \left[(\ln f)^\prime\right]^2
                               \right) \; . \notag                                   
\end{alignat}
This allows to decompose the $D$-dimensional Einstein-tensor
$E_{MN}(G)=R_{MN}-\frac{1}{2}R(G) G_{MN}$ as
\begin{alignat}{3}
  E_{\mu\nu}(G) &= E_{\mu\nu}(g)+g_{\mu\nu}\frac{(D-2)}{2}
                   \left[ \left(1-\frac{(D-1)}{4}\right) 
                          f\left[(\ln f)^\prime\right]^2
                         -f^{\prime\prime}
                   \right] \notag \\
  E_{\mu D}(G) &= 0 \\
  E_{DD}(G) &= -\frac{1}{2f}R(g)
               -\frac{(D-1)(D-2)}{8}
                \left[(\ln f)^\prime\right]^2 \; . \notag
\end{alignat}
Let us now restrict to $D=5$, where the expressions simplify to
\begin{alignat}{3}
  E_{\mu\nu}(G) &= E_{\mu\nu}(g)-\frac{3}{2}g_{\mu\nu}f^{\prime\prime} 
   \notag \\
  E_{\mu 5}(G) &= 0 
   \label{Ein2} \\
  E_{55}(G) &= -\frac{1}{2f}R(g)
               -\frac{3}{2}\left[(\ln f)^\prime\right]^2 
    \; . \notag
\end{alignat}
For the action (\ref{WallAction}) specifying the set-up, the gravitational
sources consist of a non-positive bulk cosmological constant $\Lambda(x^5) \le
0$ and walls with tension $T$ placed at $x^5=l$ and $x^5=-l$, which amounts to
the following energy-momentum tensor
\begin{equation}
  T_{MN} = -\Lambda(x^5) G_{MN} - T\left[\delta(x^5+l)+\delta(x^5-l)\right]
            g^{(4)}_{\mu\nu}\delta^\mu_M\delta^\nu_N \; .
\end{equation}
Decomposing the 5-dimensional Einstein-equation, $E_{MN}(G)=-T_{MN}/(2M^3)$,
with the help of (\ref{Ein2}) into its $\mu$ and 5 components, we
receive from the $\mu\nu$ part the 4-dimensional Einstein-equation
\begin{equation}
  E_{\mu\nu}(g)=\left[ \frac{3}{2}f^{\prime\prime}+\frac{f}{2M^3}
                       \left[ \Lambda(x^5)+T\delta(x^5+l)+T\delta(x^5-l)
                       \right]
                \right] g_{\mu\nu} \; .
  \label{FourEinstein}
\end{equation}
From the $55$ part follows an expression for the 4-dimensional curvature scalar
\begin{equation}
  R(g)=-f\left[3\left[(\ln f)^\prime\right]^2+\frac{\Lambda(x^5)}{M^3}
         \right] \; ,
  \label{FourScalar}
\end{equation}
whereas the $\mu 5$ part is satisfied trivially.

Contraction of $E_{\mu\nu}(g)$ with $g^{\mu\nu}$ gives
$E^\mu_{\phantom{\mu}\mu}(g)=\frac{3-D}{2} R(g)\rightarrow -R(g)$ and therefore
leads to the following consistency equation among (\ref{FourEinstein}) and
(\ref{FourScalar})
\begin{equation}
    2\frac{f^{\prime\prime}}{f}-\left[(\ln f)^\prime\right]^2
  = -\frac{1}{3M^3}\left[ \Lambda(x^5)+2T\delta(x^5+l)+2T\delta(x^5-l)
                   \right] \; .
    \label{Consistency}
\end{equation} 

It is evident that the right-hand-sides of (\ref{FourEinstein}) and
(\ref{FourScalar}) must be piecewise constant with respect to $x^5$, since
both left-hand-sides are at least piecewise independent of $x^5$.
This is a consequence of the simple warp-factor Ansatz.
It means that the 4-dimensional sections $\Sigma_4$, defined by $x^5=const$,
must be spacetimes of constant curvature. For $R(g)<0$ we have de Sitter
($\text{dS}_4$) and for $R(g)>0$ Anti-de Sitter ($\text{AdS}_4$)
spacetime. Since this already determines the solution to the Einstein equation
up to a scalar quantity -- the curvature -- the equations
(\ref{FourEinstein}),(\ref{FourScalar}),(\ref{Consistency}) become linear
dependent and it suffices to solve only two of them.

When we foliate
the 5-dimensional spacetime into sections $\Sigma_4$, we see that the
Einstein-equations (\ref{FourEinstein}),(\ref{FourScalar}) also follow from
the 4-dimensional action on $\Sigma_4$
\begin{equation}
   S_{D=4}(x^5) =-\int_{\Sigma_4} d^4x \sqrt{g}\left( M^2_{\text{eff}} R(g)
                                          +\Lambda_4(x^5) 
                                    \right)
\end{equation} 
if we make the following identifications\footnote{The 4-dimensional sections
  exhibit
                 \begin{equation*}
                    E_{\mu\nu}(g) = \frac{\Lambda_4(x^5)}{2M_{\text{eff}}^2}
                                    g_{\mu\nu} \; , \qquad
                    R(g) = -2\frac{\Lambda_4(x^5)}{M_{\text{eff}}^2} \; , 
                 \end{equation*}
  with $\text{dS}_4:\; R(g)<0,\Lambda_4(x^5)>0$ and $\text{AdS}_4:\;
  R(g)>0,\Lambda_4(x^5)<0$.}
\begin{alignat}{3}
    \frac{3}{2}f^{\prime\prime}+\frac{f}{2M^3}
    \left[ \Lambda(x^5)+T\delta(x^5+l)+T\delta(x^5-l) \right]
 &= \frac{\Lambda_4(x^5)}{2M_{\text{eff}}^2} 
     \label{FourScalarEquiv1} \\
    -f\left[3\left[(\ln f)^\prime\right]^2+\frac{\Lambda(x^5)}{M^3}
      \right]
 &= -\frac{2\Lambda_4(x^5)}{M_{\text{eff}}^2} \; .
     \label{FourScalarEquiv2} 
\end{alignat}
Here $M_{\text{eff}}$ is the effective Planck-scale, as obtained by
integrating the 5-dimensional action (\ref{WallAction}) over $x^5$
\begin{equation}
  M_{\text{eff}}^2 = M^3 \int dx^5 f(x^5) \; .
\end{equation}
The Einstein equations (\ref{FourEinstein}),(\ref{FourScalar}) now become
replaced by (\ref{FourScalarEquiv1}),(\ref{FourScalarEquiv2}).

To recognize the relation between the cosmological constant $\Lambda_4(x^5)$
on sections $\Sigma_4$ and the final effective $\Lambda_4$ obtained
by integrating out the fifth direction of (\ref{WallAction}), we note that
$\Lambda_4$ is given by \cite{AK3}
\begin{equation}
  \Lambda_4 = \int dx^5 f^2 
              \bigg(M^3 \big[ [(\ln f)^\prime]^2
                                +4\frac{f^{\prime\prime}}{f}
                         \big]
                     +   \big[ \Lambda(x^5)+T\delta(x^5+l)+T\delta(x^5-l)
                         \big]
              \bigg) \; . 
  \label{DefinitionLambda}
\end{equation}
Using (\ref{FourScalarEquiv1}) for the second term in square brackets, we
obtain the simple relationship
\begin{equation}
  \Lambda_4 = f^\prime f |^{x_R^5}_{x_L^5} + \langle\Lambda_4(x^5)\rangle
                                                                    \; ,
  \label{DeterminingLambda}
\end{equation}
where $x^5_R,x^5_L$ denote the right and left boundary of the $x^5$
integration region and the mean is weighted with the profile of the warp-factor
\begin{equation}
         \langle\Lambda_4(x^5)\rangle 
  \equiv \frac{\int dx^5 f \Lambda_4(x^5)}{\int dx^5 f} \; . 
\end{equation}
Since we will see that the total derivative contribution $f^\prime f
|^{x_R^5}_{x_L^5}$ will
vanish in our case of interest, we learn that the 4-dimensional effective
action $S_{D=4}$ is related to the sectionwise action by taking the mean,
$S_{D=4}=\langle S_{D=4}(x^5) \rangle$.

Since only two of the equations
(\ref{Consistency}),(\ref{FourScalarEquiv1}),(\ref{FourScalarEquiv2}) are
independent, it is most convenient to choose (\ref{Consistency}) to
determine the warp-factor in terms of the fundamental ``input'' parameters
$\Lambda(x^5),M$ and $T$. In a further step, we will then obtain
$\Lambda_4(x^5)$ from (\ref{FourScalarEquiv2}). Expressing the
warp-factor through $f=e^{-A(x^5)}$ and denoting $Y(x^5)=A^\prime(x^5)$, we can
write (\ref{Consistency}) as
\begin{equation}
  -2Y^\prime+Y^2+\frac{\Lambda(x^5)}{3M^3}
 =-\frac{2T}{3M^3}\left[ \delta(x^5+l)+\delta(x^5-l) \right] \; ,
    \label{YEquation}
\end{equation}
With the signature-function
defined by $\text{sign}(x)=-1$ if $x\le 0$ and $\text{sign}(x)=1$ if $x>0$,
the solution to this differential equation is given by
\begin{equation}
  Y(x^5)= -\frac{k}{2}
      \left(\text{sign}(x^5+l)+\text{sign}(x^5-l)\right)
      \coth\left(\frac{k}{4}
                 \left[ |x^5+l| + |x^5-l| - 2a \right]
           \right)  
        \label{YSol}
\end{equation}
together with the following $\Lambda(x^5)$ profile with arbitrary but
non-positive $\Lambda_e \le 0$
\begin{equation}
  \Lambda(x^5)  = \left\{ \begin{array}{cc}
                  \Lambda_e &,\;\; |x^5| > l \\
                  \Lambda_e/4 \le 0&,\;\; |x^5| = l \\
                      0  &,\; |x^5| < l
                          \end{array}
                   \right.
   \label{CosmStep}
\end{equation}
and the wall-tension
\begin{equation}
  \frac{T}{3M^3} = k\coth\left(\frac{k}{2}(a-l)\right) \; .
   \label{IntConstTension}
\end{equation}
Here, as in the introduction, $k=\sqrt{-\Lambda_e/3M^3}$ and $a$ is an
integration constant. The last
relation which determines $a$ through the bulk
cosmological constant $\Lambda_e$ and the wall-tension $T$ has been gained by
satisfying the boundary conditions at the wall-locations, which are encoded in
the $\delta$-function terms in (\ref{YEquation}). A matching of the
$\delta$-function terms arising from $Y^\prime$ with those proportional to $T$
leads to (\ref{IntConstTension}).
The symmetry of the set-up -- caused by the equality of both wall-tensions --
forces the bulk cosmological constant between them to be zero. A non-vanishing
value can be obtained, if wished, if we introduce an asymmetry of the set-up
through unequal wall-tensions.
A further integration of $Y$ yields the warp-function
\begin{equation}
  A(x^5)=-2\ln\left|\sinh\left(\frac{k}{4}
                                \left[|x^5+l|+|x^5-l|-2a\right]
                          \right)
              \right| + b \; ,
  \label{WarpFunction}
\end{equation}
where $b$ is a second integration constant.
Note, that the above solution is valid for the parameter-range $T\ge
3M^3 k$ as can be easily recognized from
(\ref{IntConstTension}). If $T < 3M^3 k$, we have to
substitute a ``tanh'' for the ``coth'' appearing in (\ref{YSol}) and
(\ref{IntConstTension}), while (\ref{CosmStep}) remains the same. This amounts
to a change from ``sinh'' to ``cosh'' in (\ref{WarpFunction})
Since we assume a positive wall-tension $T>0$, the integration constant
$a$ is constrained through (\ref{IntConstTension}) over the whole
parameter-region, $T>0$, $\Lambda_e \le 0$, by the lower bound $a>l$. 

An important observation is that the warp-factor $f=e^{-A(x^5)}$
vanishes at $x^5=\pm a$. If $Q<0$ (which later will turn out to be the
$\text{AdS}_4$ case, whereas the physically more relevant -- since
observations point to a positive $\Lambda_4$ -- $\text{dS}_4$ case is
free of singularities) this gives rise to a singular
5-dimensional curvature at these points
\begin{equation}
  \lim_{x^5 \rightarrow \pm a} R(G) 
  \rightarrow
    \frac{24\Theta(-Q)}{(|x^5|-a)^2}   
    \; , \qquad
    Q = \frac{T-3M^3 k}{T+3M^3 k}
    \; ,
\end{equation}
where the Heaviside step-function is defined by $\Theta(x)=0,x<0$ and
$\Theta(x)=1, x>0$. Due to the vanishing of the warp-factor at these
points we expect a tremendous red-shift in signals originating
there. Indeed, let us conceive a wave signal emitted with frequency
$\nu_e$ at $x^5=\pm a$. Then that wave will be observed in the
interior region $x^5 \in (-a,a)$ with frequency $\nu_o$ given by
\begin{equation}
  \frac{\nu_o}{\nu_e} = \sqrt{\frac{G_{11}(x^5=\pm a)}{G_{11}(|x^5|<a)}}
                      = 0 \; ,
\end{equation}
due to the vanishing of the warp-factor at $x^5=\pm a$. Hence, an infinite
redshift makes it impossible for the region $|x^5|\ge a$ to communicate to our
world (at least via electromagnetic radiation). Therefore, we should restrict
the $x^5$ integration region to the causally connected interval $x^5\in
(-a,a)$.

Since recently there has been a discussion in the literature
\cite{Gubser},\cite{FLLN1},\cite{FLLN2} about which singularities are
permissible and which
have better to be avoided, it is interesting to see the verdict on our
singularities in the case of $Q<0$. In
\cite{Gubser} it has been argued that in a gravitational system exhibiting a
4-dimensional flat solution together with bulk scalars, only those
singularities are allowed, which leave the scalar potential bounded
from above. In our case, where we do not have any scalars, the role of
the scalar potential is played by the bulk cosmological constant
$\Lambda_e$ (together with the tension $T$ at the wall-positions),
which is clearly bounded from above.  If the criterion of
\cite{Gubser} generalizes to the case where the 4-dimensional metric
deviates slightly (since in the end $\Lambda_4$ turns out to be
exponentially small) from the flat case, we would conclude that the
singularities encountered above for $Q<0$ are of the permissible type.

Furthermore, in \cite{FLLN2} a consistency condition has been derived which
should hold for the effective cosmological constant obtained by integration
over the causally connected $x^5$-region. We will now demonstrate that this
consistency condition is a simple consequence of
(\ref{FourScalarEquiv1}),(\ref{FourScalarEquiv2}) and the expression
(\ref{DefinitionLambda}), which defines $\Lambda_4$. Starting with
(\ref{DefinitionLambda}) and employing
(\ref{FourScalarEquiv1}),(\ref{FourScalarEquiv2}) to eliminate the derivatives
$[(\ln f)^\prime]^2$ and $f^{\prime\prime}$, (\ref{DefinitionLambda}) becomes
\begin{equation}
  \Lambda_4 = 2\langle \Lambda_4 \rangle
             -\frac{1}{3}\int_{-a}^a dx^5 f^2 \left( 2\Lambda(x^5)
             +T\delta(x^5+l)+T\delta(x^5-l)\right) \; .
\end{equation}  
Noticing that $f^\prime f(x^5=\pm a)=0$, we use (\ref{DeterminingLambda}) to
obtain
\begin{alignat}{3}
  \Lambda_4 &= \frac{1}{3}\int_{-a}^a dx^5 f^2 \left( 2\Lambda(x^5)
              +T\delta(x^5+l)+T\delta(x^5-l)\right) \notag \\
            &= -\frac{1}{3}\int_{-a}^a dx^5 f^2 
               \left( T_1^{\phantom{1}1}+T_5^{\phantom{5}5} \right) \; ,
  \label{NillesConsistency}
\end{alignat}
which is nothing but the consistency condition of \cite{FLLN2}. Since
our solution has been derived from
(\ref{Consistency}),(\ref{FourScalarEquiv2}) which are equivalent to
(\ref{FourScalarEquiv1}),(\ref{FourScalarEquiv2}) and we will furthermore only
require
(\ref{DefinitionLambda}) to obtain $\Lambda_4$, we conclude that the
consistency condition (\ref{NillesConsistency}) of \cite{FLLN2} should be
satisfied for our solution.

After this short intermezzo on singularities, let us proceed by
inverting (\ref{IntConstTension}), to express $a$ explicitly through
the input values $T$ and $\Lambda_e$
\begin{equation}
   a = -\frac{1}{k}\ln |Q| + l \; ,
   \label{aConstant}
\end{equation}
which is valid for both $T \ge 3M^3 k$ and
$T < 3M^3 k$. This shows how the parameters $T,M,\Lambda_e$
influence the width of the $x^5$ domain. 

In order to determine $\Lambda_4(x^5)$, note that to obey the Einstein
equations, we have to fulfill (\ref{FourScalarEquiv2}). This can be used
to derive the following expressions for $\Lambda_4(x^5)$
\begin{equation}
            \Lambda_4(x^5)
          = \pm\frac{3}{2}e^{-b}M^2_{\text{eff}}
            \left\{ \begin{array}{cc}
                    k^2   &,\;\; |x^5| > l \\
                    k^2/4 &,\;\; x^5 = \pm l \\
                      0      &,\;\; |x^5| < l
                    \end{array}
            \right.
     \label{SectionCoCo}
\end{equation}
Here, the plus-sign applies to the case $T\ge 3M^3 k$, whereas the
minus-sign applies to the complementary case in which $T < 3M^3 k$.
Since we do not want to use $\Lambda_4(x^5)$ as an input to determine
$b$, but rather focus on the opposite, we are looking for an
additional constraint, which allows for a determination of the
constant $b$. This extra constraint comes from considering a smooth
transition to the flat solution (\ref{FlatSolution}) with
$\Lambda_4=0$. As can be seen from (\ref{Tension}), we reach the flat
limit by sending $T\rightarrow 3M^3 k$. Via (\ref{aConstant}) this
limit corresponds to sending the constant $a\rightarrow \infty$. Thus
we see, that the integration region $x^5 \in (-a,a)$ extends over the
whole real line in this limit and the warp-function
(\ref{WarpFunction}) becomes
\begin{equation}
  A(x^5) \rightarrow \frac{k}{2}\left(|x^5+l|+|x^5-l|\right)
                     +2\ln 2-ka+b \; .
\end{equation}
Thus, to guarantee a smooth transition to the flat solution
(\ref{FlatSolution}), we have to identify 
the integration constants $a$ and $b$ as follows
\begin{equation}
  b = -2\ln 2+ka \; .
\end{equation}
This, together with (\ref{aConstant}) and (\ref{SectionCoCo}) yields
the following expression for $\Lambda_4(x^5)$ in terms of physical
``input'' parameters
\begin{equation}
     \Lambda_4(x^5)
   =  6 e^{-kl} Q M^2_{\text{eff}}
     \left\{ \begin{array}{cc}
                k^2     &,\; |x^5| > l \\
                k^2/4   &,\; x^5 = \pm l \\
                   0        &,\; |x^5| < l
             \end{array}
     \right.
    \label{SectionCoCoFinal}
\end{equation}
Notice, that this formula is valid for both parameter-regions $T \ge 3M^3 k$
and $T < 3M^3 k$.

Finally, to obtain the observable four-dimensional $\Lambda_4$, we
have to take the mean of $\Lambda_4(x^5)$. Again using that $f^\prime
f(x^5=\pm a)=0$, we employ (\ref{DeterminingLambda}) and arrive at
\begin{equation}
  \Lambda_4 = \frac{\int_{-a}^a dx^5 e^{-A(x^5)}\Lambda_4(x^5)}
                    {\int_{-a}^a dx^5 e^{-A(x^5)}}
            = 12e^{-2kl}M^3 k Q F(|Q|) \; ,
  \label{PreLambda}
\end{equation}
where we defined $F(|Q|)=1-|Q|^2+2|Q|\ln|Q|$. In addition we obtain
the following effective Planck-scale
\begin{equation}
  M_{\text{eff}}^2 = M^3 \int_{-a}^a dx^5 e^{-A(x^5)}
                   = 2e^{-kl}M^3\left(l(1-|Q|)^2+\frac{F(|Q|)}{k}\right) 
                      \; .
  \label{PreMass}
\end{equation}

There is an exponential-factor occuring in $\Lambda_4$ which is the
square of the one occuring in $M_{\text{eff}}^2$. At the classical
level (classical in the bulk of the five-dimensional spacetime -- the
field-theories on the walls are however considered quantum
mechanically) an overall constant $e^{-kl}$ multiplying the whole
effective 4-dimensional action $S_{D=4}= -\int
d^4x\sqrt{g}(M^2_{\text{eff}}R(g)+\Lambda_4)= -e^{-kl}\int
d^4x\sqrt{g}({\tilde M}^2_{\text{eff}}R(g)+{\tilde \Lambda}_4)=
-e^{-kl}{\tilde M}^2_{\text{eff}}\int d^4x\sqrt{g}(R(g)+\lambda_4)$ is
immaterial -- it simply drops out of the field equation. Therefore, we
can neglect the overall factor $e^{-kl}$. The physically observable
cosmological constant -- invariant under any overall rescaling -- is
given by $\lambda_4=\Lambda_4/M^2_{\text{eff}}= {\tilde
\Lambda_4}/{\tilde M}^2_{\text{eff}}$. With (\ref{PreLambda}) and 
(\ref{PreMass}) we thus obtain
\begin{equation}
  \lambda_4 = e^{-kl}
              \left( \frac{6k^2 Q F(|Q|)}{kl(1-|Q|)^2 + F(|Q|)}
              \right) \; .
\end{equation}

Let us make some comments about this formula. The physical range of
the parameter $Q$ lies between $0\le |Q|\le 1$, where we presuppose a
non-negative wall-tension $T>0$. The lower bound corresponds to the
finetuned flat $\Lambda_4=0$ limit, while the upper bound is reached
for vanishing bulk cosmological constant $\Lambda_e=0$. Over that
domain we have $1
\ge F(|Q|<1)>0,\,F(1)=0$. Hence, we recognize that starting with some
fundamental values for $\Lambda_e\le 0,M,T>0$ we obtain a positive or
negative $\lambda_4$ depending on the sign of $Q$. For
$T>\sqrt{-3M^3\Lambda_e}$ the 4-dimensional spacetime will be
$\text{dS}_4$, whereas for $T<\sqrt{-3M^3\Lambda_e}$ it will be
$\text{AdS}_4$. Furthermore, we see a smooth connection to the case
with flat 4-dimensional Minkowski spacetime for finetuned parameters
$T=\sqrt{-3M^3\Lambda_e} \Leftrightarrow Q=0$.  Most importantly,
there is no need for a finetuning of the fundamental parameters to
receive a small $\lambda_4$. By adapting the distance $2l$ between
both walls, one arrives at a huge enough suppression through the
exponential factor such that the observed value could be accounted
for. Moreover, thanks to the exponential suppression this does not
amount to an extremely large hierarchy between the fundamental scale
$M$ and the separation-scale $1/2l$.


\bigskip
\noindent {\large \bf Acknowledgements}\\[2ex] 
We are grateful to R.~Blumenhagen, G.~Curio and particularly A.~Karch for
discussions.


 \newcommand{\zpc}[3]{{\sl Z. Phys.} {\bf C\,#1} (#2) #3}
 \newcommand{\npb}[3]{{\sl Nucl. Phys.} {\bf B\,#1} (#2)~#3}
 \newcommand{\plb}[3]{{\sl Phys. Lett.} {\bf B\,#1} (#2) #3}
 \newcommand{\prd}[3]{{\sl Phys. Rev.} {\bf D\,#1} (#2) #3}
 \newcommand{\prb}[3]{{\sl Phys. Rev.} {\bf B\,#1} (#2) #3}
 \newcommand{\pr}[3]{{\sl Phys. Rev.} {\bf #1} (#2) #3}
 \newcommand{\prl}[3]{{\sl Phys. Rev. Lett.} {\bf #1} (#2) #3}
 \newcommand{\jhep}[3]{{\sl JHEP} {\bf #1} (#2) #3}
 \newcommand{\cqg}[3]{{\sl Class. Quant. Grav.} {\bf #1} (#2) #3}
 \newcommand{\prep}[3]{{\sl Phys. Rep.} {\bf #1} (#2) #3}
 \newcommand{\fp}[3]{{\sl Fortschr. Phys.} {\bf #1} (#2) #3}
 \newcommand{\nc}[3]{{\sl Nuovo Cimento} {\bf #1} (#2) #3}
 \newcommand{\nca}[3]{{\sl Nuovo Cimento} {\bf A\,#1} (#2) #3}
 \newcommand{\lnc}[3]{{\sl Lett. Nuovo Cimento} {\bf #1} (#2) #3}
 \newcommand{\ijmpa}[3]{{\sl Int. J. Mod. Phys.} {\bf A\,#1} (#2) #3}
 \newcommand{\rmp}[3]{{\sl Rev. Mod. Phys.} {\bf #1} (#2) #3}
 \newcommand{\ptp}[3]{{\sl Prog. Theor. Phys.} {\bf #1} (#2) #3}
 \newcommand{\sjnp}[3]{{\sl Sov. J. Nucl. Phys.} {\bf #1} (#2) #3}
 \newcommand{\sjpn}[3]{{\sl Sov. J. Particles \& Nuclei} {\bf #1} (#2) #3}
 \newcommand{\splir}[3]{{\sl Sov. Phys. Leb. Inst. Rep.} {\bf #1} (#2) #3}
 \newcommand{\tmf}[3]{{\sl Teor. Mat. Fiz.} {\bf #1} (#2) #3}
 \newcommand{\jcp}[3]{{\sl J. Comp. Phys.} {\bf #1} (#2) #3}
 \newcommand{\cpc}[3]{{\sl Comp. Phys. Commun.} {\bf #1} (#2) #3}
 \newcommand{\mpla}[3]{{\sl Mod. Phys. Lett.} {\bf A\,#1} (#2) #3}
 \newcommand{\cmp}[3]{{\sl Comm. Math. Phys.} {\bf #1} (#2) #3}
 \newcommand{\jmp}[3]{{\sl J. Math. Phys.} {\bf #1} (#2) #3}
 \newcommand{\pa}[3]{{\sl Physica} {\bf A\,#1} (#2) #3}
 \newcommand{\nim}[3]{{\sl Nucl. Instr. Meth.} {\bf #1} (#2) #3}
 \newcommand{\el}[3]{{\sl Europhysics Letters} {\bf #1} (#2) #3}
 \newcommand{\aop}[3]{{\sl Ann. of Phys.} {\bf #1} (#2) #3}
 \newcommand{\jetp}[3]{{\sl JETP} {\bf #1} (#2) #3}
 \newcommand{\jetpl}[3]{{\sl JETP Lett.} {\bf #1} (#2) #3}
 \newcommand{\acpp}[3]{{\sl Acta Physica Polonica} {\bf #1} (#2) #3}
 \newcommand{\sci}[3]{{\sl Science} {\bf #1} (#2) #3}
 \newcommand{\vj}[4]{{\sl #1~}{\bf #2} (#3) #4}
 \newcommand{\ej}[3]{{\bf #1} (#2) #3}
 \newcommand{\vjs}[2]{{\sl #1~}{\bf #2}}
 \newcommand{\hepph}[1]{{\sl hep--ph/}{#1}}
 \newcommand{\desy}[1]{{\sl DESY-Report~}{#1}}

\bibliographystyle{plain}
\begin{thebibliography}{99}
  \bibitem{AK3} A.~Krause,
                {\it A Small Cosmological Constant, Grand Unification and
                Warped Geometry},
                hep-th/0006226;
  \bibitem{ADKS} N.~Arkani-Hamed, S.~Dimopoulos, N.~Kaloper and R.~Sundrum,
                 {\it A Small Cosmological Constant from a Large Extra
                 Dimension},
                 hep-th/0001197;
  \bibitem{KSS} S.~Kachru, M.~Schulz and E.~Silverstein,
                {\it Selftuning Flat Domain Walls in 5-D Gravity and
                String Theory},
                hep-th/0001206;
  \bibitem{Gubser} S.S.~Gubser,
                   {\it Curvature Singularities: The Good, the Bad, and the
                   Naked},
                   hep-th/0002160;
  \bibitem{FLLN1} S.~F\"orste, Z.~Lalak, S.~Lavignac and H.P.~Nilles,
                  {\it A Comment On Selftuning and Vanishing Cosmological
                  Constant in the Brane World},
                  \plb{481}{2000}{360}, hep-th/0002164;
  \bibitem{FLLN2} S.~F\"orste, Z.~Lalak, S.~Lavignac and H.P.~Nilles,
                  {\it The Cosmological Constant Problem from a Brane-World
                  Perspective},
                  hep-th/0006139;
\end{thebibliography}

\end{document}

















