\documentclass[a4paper,12pt,draft]{article}

\usepackage{amssymb}
\usepackage{latexsym}

\setlength{\evensidemargin}{-0.2in}
\setlength{\oddsidemargin}{-0.2in}
\setlength{\textwidth}{6.8in}
\setlength{\topmargin}{-0.5in}

\setlength{\textheight}{9.4in}

\newcommand{\Z}{{\mathbb Z}}
\newcommand{\R}{{\mathbb R}}
\newcommand{\g}{{\mathfrak g}}
\newcommand{\dd}{{\rm d}}   
\newcommand{\tr}{{\rm tr}}  

\font\fone=cmr10 scaled\magstep3
\font\ftwo=cmr7 scaled\magstep3
\begin{document}
\thispagestyle{empty}
\vspace{1.5in}

\centerline{\fone Homotopic Classification of Yang--Mills Vacua}
\vspace{0.1in}

\centerline{\fone Taking into Account Causality}
\vspace{ 0.3in}

\centerline{\ftwo G\'abor Etesi}
\vspace{0.1in}

\centerline{Yukawa Institute for Theoretical Physics,}
\centerline{Kyoto University,}
\centerline{606-8502 Kyoto, Japan}
\centerline{\tt etesi@yukawa.kyoto-u.ac.jp}
\vspace{0.2 in}

\begin{abstract}
In this letter we study the existence of $\theta$-vacuum states in
Yang--Mills theories defined over asymptotically flat, stationary
space-times taking into account not only the topology but the complicated
causal structure of these space-times, too. By a result of
Chru\'sciel and Wald, apparently causality makes all vacuum states, seen
by a distant observer, homotopically equivalent making the
introduction of $\theta$-terms unnecessary in (causally) effective
Lagrangians. 

But a more careful study shows that certain twisted classical vacuum
states survive even in this case eventually leading to the
conclusion that the concept of ``$\theta$-vacua'' is meaningful in
the case of general Yang--Mills theories. We give a classification of
these vacuum states based on Isham's results showing that the Yang--Mills
vacuum has the same complexity as in the flat Minkowskian case hence the
general CP-problem is not more complicated than the well-known flat one.
\end{abstract}
\vspace{0.1in}

\centerline{Keywords: {\it $\theta$-vacua, causality, black holes}}
\centerline{PACS numbers: 11.15, 11.30.E, 04.20.G, 04.70}
\vspace{0.1in}

\pagestyle{myheadings}
\markright{G. Etesi: The Structure of the Yang--Mills Vacuum}
\section{Introduction}
The famous solution of the long-standing $U(1)$-problem in the Standard 
Model via instanton effects was presented by 't Hooft about two
decades ago \cite{tho}. This solution demonstrated that {\it instantons},
i.e. finite-action solutions of the {\it Euclidean} Yang--Mills-equations
discovered by Belavin et al. \cite{bpst} should be taken seriously
in gauge theories. Another problem arose in these models over 
{\it flat} space-times, however: if instantons really exist, they induce a
P- hence CP-violating so-called $\theta$-term in the effective Yang--Mills
action. But according to accurate experimental results, such a
CP-violation does not exists in QCD, for instance. The most accepted
solution to this  problem is the so-called {\it Peccei--Quinn mechanism}
\cite{pec}. A consequence of this mechanism is the existence of a light
particle, the so-called {\it axion}. This particle has not been observed
yet, however.

The question naturally arises whether or not such problematic
$\theta$-term must be introduced over more general space-times. The aim of
our paper is to claim that the answer is yes.
 
First, let us summarize the vacuum structure of a gauge theory over 
Minkowski space-time. We are going to use an Yang--Mills theory
framework over $(\R^4 , g)$ where $g$ is a Lorentzian metric. Let
$E$ be a (trivial) complex vector bundle over $\R^4$ belonging to a finite
dimensional complex representation of $G$. Without loss of generality we
choose the gauge group $G$ to be a compact Lie group. Consider a
$G$-connection on this bundle; choosing a particular frame on $E$, this
connection can be identified globally with a $\g$-valued $1$-form $A$
with curvature $F_A$. We choose the usual Yang--Mills action (by
fixing the coupling to be $1$): 
\begin{equation}
S(A,g)=-{1\over 8\pi^2}\int\limits_M\tr\left(F_A\wedge
*F_A\right) ,
\label{eym}
\end{equation}
in this case $M=\R^4$ and $*$ denotes the Hodge-operator
induced by the metric $g$ on $\R^4$. Usually the metric $g$ is fixed and
supposed to be the Minkowski metric on $\R^4$. The 
Euler--Lagrange equations of (\ref{eym}) are the Yang--Mills equations
and read as follows: 
\[\dd_AF_A=0,\:\:\:\:\:\dd_A*F_A=0.\]
The simplest solution is the vacuum. The gauge field $A$ is a vacuum field
if $F_A=0$ i.e. 
its field strength or curvature is equal to zero. By simply connectedness
of $\R^4$ such gauge fields can be written in the form 
$A=f^{-1}\dd f$, where $f: \R^4\rightarrow G$ is a smooth function.

But by the existence of a global temporal gauge and the
stationarity of the flat metric on $\R^4$ it is enough to consider
the restriction of $f$ to a space-like submanifold of Minkowski
space-time, i.e. $f: \R^3\rightarrow G$. Minkowski space-time is
asymptotically flat as well, so there is a point $i^0$ called 
space-like infinity. This point represents the ``infinity of space''
hence can be added to $\R^3$ completing it to the
three-sphere $\R^3\cup\{i^0\}=S^3$. It is well-known that vacuum fields
(possibly after a null-homotopic gauge-transformation around $i^0$) 
extend to the whole $S^3$ consequently classical vacua are classified by
maps $f: S^3\rightarrow G$. These maps up to homotopy are given by
elements of $\pi_3 (G)$. For typical compact Lie groups $\pi_3(G)\cong\Z$. 
This fact can be interpreted as classical vacua are separated 
from each other by barriers of finite height i.e. it is impossible to
develop two vacua of different winding numbers into each other only
through vacuum states. Hence homotopy equivalence reflects the dynamical
structure of the theory.

On the other hand if $f_1$, $f_2$ are vacua of winding numbers $n_1$,
$n_2$ respectively, there is a gauge transformation $g: S^3\rightarrow G$
of winding number $n_2-n_1$ satisfying $gf_1=f_2$. Consequently we can see
that the concept of {\it dynamical equivalence} of vacua given by the {\it
dynamics} of the theory (i.e. the {\it homotopy equivalence} of maps $f:
S^3\rightarrow G$) is different from that of {\it symmetry-equivalence} of
vacua provided by the {\it symmetry} of the gauge theory (i.e. the {\it
gauge equivalence} of the above maps). 

To avoid this discrepancy, we proceed as follows. Suppose we have
constructed the Hilbert space ${\cal
H}_{\R^4}$ of the corresponding quantum gauge theory. If $\vert
n\rangle\in{\cal H}_{\R^4}$ denotes the  quantum vacuum state belonging to
a classical vacuum $f$ of winding number $n$, the simplest way to
construct a state which is invariant (up to phase) under both dynamical
(i.e. homotopy) and symmetry- (i.e. gauge)  equivalence is to introduce
the state
\begin{equation}
\vert\theta\rangle 
:=\sum\limits_{n=-\infty}^{\infty}e^{in\theta}\vert n\rangle\:\in{\cal
H}_{\R^4} ,\:\:\:\:\theta\in U(1).
\label{teta}
\end{equation}
These quantum vacuum states are referred to as ``$\theta$-vacua''. 

From the physical point of view, the introduction of $\theta$-vacua is
also necessary because the vacuum states of different winding numbers can
be joined semi-classically i.e. by a tunneling process induced by
non-trivial instantons of the corresponding {\it Euclidean} gauge theory.
Indeed, as it is well known \cite{che}\cite{kak}, in the case
of $G=SU(2)$, the instanton number of an instanton is an element $k\in
H^4(S^4, \Z )\simeq\Z$ (here $S^4$ is the one-point conformal
compactification of the Euclidean flat $\R^4$. Note that the notion of 
``instanton number'' comes from a very different
compactification process comparing to the derivation of ``vacuum winding
number''). If two vacua, $\vert n_1\rangle$, $\vert n_2\rangle$ ($n_1,
n_2\in \pi_3(SU(2))\simeq\Z$) are given then there is an instanton of instanton
number $n_2-n_1\in H^4(S^4,\Z )\simeq\Z$ tunneling between them in
temporal gauge \cite{che}\cite{kak}. In other words the true vacuum
states are linear combinations of the vacuum states of unique winding numbers 
yielding again (\ref{teta}). 

But the value of $\theta$ cannot be changed in any order of perturbation,
i.e. it should be treated as a physical parameter of the theory; this
implies that tunnelings induce the effective term
\[{\theta\over 8\pi^2}\int\limits_{\R^4}\tr\left( F_A\wedge F_A\right)\]
in addition to action (\ref{eym}). But it is not difficult to see that
such a term violates the parity symmetry $P$ of the theory resulting in 
the violation of the CP-symmetry.

In summary, we have seen that there are at least three different ways to
introduce $\theta$-parameters in Yang--Mills theories {\it over
Minkowskian space-time}:

(i) $\theta$ is introduced to fill in the gap between the notions of
dynamical (i.e. homotopy) and symmetry- (i.e. gauge)
equivalence of Yang--Mills vacua. This approach is pure mathematical in
its nature;

(ii) $\theta$ must be introduced because by instanton effects vacua of
definite winding numbers are superposed in the underlying semi-classical
Yang--Mills theory;

(iii) $\theta$ must be introduced by ``naturality arguments'', i.e.
nothing prevents us to extend the Yang--Mills action at the full quantum 
level by a $P$-violating term $\tr\left( F_A\wedge F_A\right)$ with
coupling constant $\theta$.

There is a correspondence between the above three characterizations of the
$\theta$ in {\it flat Min\-kow\-ski\-an space-time} but in the case of
general space-times, clear and careful distinction must be made until a
relation or correspondence between the three notions is established.
Clearly, (i) is related to the {\it topology} of the space-time and
the gauge group hence it is relatively easy to check whether or not it
remains valid in the general case. Concept (ii) is related to the
semi-classical structure of the
general Yang--Mills theory especially to the existence of instanton
solutions in the Wick-rotated theory and their relationship with vacuum
tunneling. The validity of concept (iii) is the most subtle one: we need
lot of information on global non-perturbative aspects of the general
quantum Yang--Mills theory to check if any $\theta$-term survives quantum
corrections. In the present state of affairs, having no adequate general
theory of Wick rotation, instantons and their physical interpretation,
non-perturbative aspects of general Yang--Mills theories, we can examine
only the validity of concept (i) in the general case. Its validity or
invalidity may serve as a good indicator for the existence and role of
$\theta$-terms in general Yang--Mills theories.

The analysis of the vacuum structure of general Yang--Mills
theories from the point of view of (i) was carried out by Isham et al.
\cite{isham1}\cite{isham2}\cite{isham3}\cite{isham4}\cite{isham5}. In
these papers Isham et al. argue that in the general case concept (i) for
introducing $\theta$-terms still continue to hold due to the complicated
topology of the spatial surface $S$ and the gauge group $G$ \cite{isham1}.
The classical vacuum structure of these theories becomes more complicated
and we cannot avoid the introduction of various CP-violating terms into
the effective Lagrangian \cite{isham5}. 

We have to emphasize that the approach of Isham et al. to the problem is
pure topological in its nature, however. By a result of Witt \cite{wit}
every oriented, connected three-manifold $S$ appears as Cauchy-surface
of a physically reasonable initial data set. It is well-known
that the complicated topology of the space-like submanifold $S$ leads to
appearance of singularities in space-time if it arises as the
Cauchy development of $S$. Indeed, an early result of Gannon \cite{gan}
shows that the Cauchy development of a non-simply connected Cauchy surface
is geodesically incomplete i.e. singularities occur. If we accept the
Cosmic Censorship Hypothesis, these singularities are hidden behind event
horizons resulting in a non-trivial causal structure for these
space-times, too. A theorem of Chru\'sciel--Wald \cite{chr-wal} shows that
distant observers can observe only simply-connected portions of 
asymptotically flat space-times: all topological properties are hidden
behind event horizons, eventually resulting again in a topologically
simple {\it effective} space-time. Hence one may doubt if Isham's
conclusions remain valid.

In Section 2 we formulate Yang--Mills theories with 
an arbitrary compact gauge group over general asymptotically flat,
stationary space-times. This model provides a good framework for 
studying classical Yang--Mills vacua  over causally non-trivial
space-times. In this setup we simply mimic the above analysis concerning
classical Yang--Mills vacua and find that although
all vacua are topologically equivalent on the causally connected 
regime of the space-time, the appearance of a natural boundary condition
on the event horizons (also a consequence of the causal structure) finally
makes concept (i) still remain valid. 

In Section 3 we calculate explicitely the homotopy classes of vacua for
the classical groups. A modification appears compared with Isham and
other's pure topological considerations in the sense that
generally the vacuum structure in our case has the same complexity as in
the flat Minkowskian case. This demonstrates the ``stability'' of the
$\theta$-problem.

The idea of studying relationship between micro- or virtual black
holes, wormholes and $\theta$-vacua is not new. For example, see 
Hawking \cite{haw} and Preskil et al. \cite{pres-triv}.

\section{Vacua: A General Space-Time Model}
The general reference for this chapter is \cite{wal}. Let $\widetilde{S}$ be
a connected, oriented, closed three-manifold and let
$S:=\widetilde{S}\setminus\{i^0\}$ where
$i^0\in\widetilde{S}$ is a point. Using the result of Witt \cite{wit} we
can choose an asymptotically flat initial data set
$(S, h, k)$, where $h$ is a smooth Riemannian metric while $k$ is a
symmetric $(0,2)$-type tensor field on $S$ both satisfying suitable
fall-off conditions in a neighbourhood of $i^0$ (this point is called {\it
spatial infinity}). We suppose these initial data are given by some matter
field represented by a stress-energy tensor $T\vert_S$ obeying the
dominant energy condition. Consider the maximal Cauchy development of 
$(S, h, k)$ denoted by $(M, g)$. This space-time is {\it globally
hyperbolic} with Cauchy surface $S$ by construction and $M\cong
S\times\R$. 

Choose a complex vector bundle $E$
over $M$ associated to the gauge group $G$ via a complex representation 
of $G$ and a $G$-connection $A$ on it.
Consider a Yang--Mills theory with action (\ref{eym}). We will focus on
{\it vacuum solutions on a gravitational background} i.e.
pairs $(A, g)$ where $A$ is a smooth flat $G$-connection on the bundle $E$
while $g$ is a smooth Lorentzian metric on $M$ (which is a solution of the
Einstein's equations with a matter field given by a stress-energy tensor
$T$. For technical reasons we suppose $T$ satisfies the strong
energy condition). We will refer the collection $(E, A, M, g)$ to as an
{\it Yang--Mills vacuum setup}.

We take two more restrictions. First, we will assume that $(M, g)$ is {\it
asymptotically flat}. At a first look this means that there is
a conformal embedding $i: (M, g)\rightarrow (\widetilde{M},\widetilde{g})$
such that the image of the Cauchy surface can be
completed to a maximal space-like submanifold $\widetilde{S}$ by adding
the space-like infinity $i^0\in\widetilde{M}$ to it: $i(S)\cup\{ i^0\}
=\widetilde{S}$. Moreover
the infinitely distant points of $M$, represented by $\partial i(M)$ are
divided naturally into three classes: the future and past null infinities
${\cal I}^\pm$ and the already mentioned spatial infinity $i^0$. The {\it
asymptotically flat outer region} of $M$ is defined to be the set
$N:=J^-({\cal I}^+)\cap i(M)$ which is a manifold (here $J^\pm (A)$
denotes the causal past or future of a set $A\subset\widetilde{M}$). The 
metric $\widetilde{g}$ is related by a conformal factor $\Omega$ to
the original metric; although the details are unrelevant in our 
considerations, we remark that $\widetilde{g}$ is not necessarily smooth
in $i^0$. Note that if $M\setminus N=:B$ is not empty then $(M, g)$
contains a {\it black hole region $B$}. We denote by $H\subset M$
its {\it event horizon}. Clearly, $\partial N = \partial B=H$. For
details, see \cite{wal}.

Secondly, we assume that $(M, g)$ is {\it stationary}.

In summary, we focus our attention to each stationary, asymptotically 
flat, globally hyperbolic Yang--Mills vacuum setups $(E, A, M,
g)$ with an arbitrary compact gauge group $G$. We address the problem of
describing the topology of Yang--Mills vacua {\it seen by an observer in
the asymptotically flat region} of the space-time $(M,g)$. Clearly, at
least classically, only this part of the space-time can be relevant for
ordinary macroscopic observers. To achieve this, we refer to a general 
result of Chru\'sciel and Wald on asymptotically flat outer regions
\cite{chr-wal}.
\vspace{0.1in}

{\bf Theorem 1}. (Chru\'sciel--Wald). {\it Let $(M, g)$ be a globally
hyperbolic, asymptotically flat, stationary space-time with a matter field
represented by a stress-energy tensor $T$ satisfying the strong
energy condition. Then the outer asymptotically flat
region $N$ of $(M,g)$ is simply connected, i.e. $\pi_1(N)=1$.

Moreover, if $M$ contains a black hole region, then all connected
components of the event horizon $H\subset M$ are homeomorphic to
$S^2\times\R$.} $\Diamond$
\vspace{0.1in}

\noindent By global hyperbolicity, there is a global time function $T:
M\rightarrow\R$ . Let $S_t:=T^{-1}(t)$ ($t\in\R$) be a Cauchy surface and
$\widetilde{S}_t=i(S_t)\cup\{ i^0\}$ its conformal completion. Of course
$\widetilde{S}_t\cong\widetilde{S}$ for all $t\in\R$.

In light of the above theorem, $V_t:=N\cap S_t$, a Cauchy surface for the
outer region, is
an oriented simply connected three-manifold. If $M$ contains black hole
domains then $\partial V_t\not=\emptyset$  and all boundary components
are homeomorphic to a two-sphere $S^2$ (``the event horizon of a
stationary black hole has no handles''). By conformal completion we may
consider rather the three-manifold
$\widetilde{V}_t:=\widetilde{N}\cap\widetilde{S}_t$ which is
moreover compact. 

Now we are ready to describe the Yang--Mills-vacuum structure over $(N,
g\vert_N)$. The simply connectedness of $N$ implies that a restricted
flat Yang--Mills bundle $E\vert_N$ is trivial whatever $G$ is hence a
vacuum Yang--Mills connection $A\vert_N$ may be regarded as a $\g$-valued
$1$-form on $E\vert_N$. In an appropriate trivialization of the bundle
$E\vert_N$, we may regard $A\vert_N$ as a $\g$-valued $1$-form over $N$
instead of $E\vert_N$. Moreover, also by simply connectedness, a smooth
Yang--Mills field $A\vert_N$ is vacuum if and only if there is a
smooth function $f:N\rightarrow G$ obeying $A\vert_N=f^{-1}\dd f$. Using
again the simply connectedness of $N$, there is a gauge-transformation
over $N$ such that all vacuum fields can be transformed into temporal
gauge, i.e. $A_0\vert_N=0$ where the $A_0$ component is defined by the
time function $T$. Consequently in temporal gauge, by exploiting the
stationarity of the metric $g$, all flat connections are characterized by
smooth functions $f_t: V_t\rightarrow G$. It is easily seen that
$\pi_2(G)=0$ implies that always
exists a {\it null-homotopic} gauge transformation in the neighbourhood of
the space-like infinity $i^0\in\widetilde{V}_t$ such that the
gauge-transformed function $\tilde{f}_t$ extends as the identity to
$i^0$ hence we are dealing with smooth functions
$\tilde{f}:\widetilde{V}\rightarrow G$ (we denote $\widetilde{V}_t$ and
$\tilde{f}_t$ simply by $\widetilde{V}$ and $\tilde{f}$ if there is no
danger of confusion).

A pure Yang--Mills theory being conformally invariant, we may consider
our original Ein\-stein--Yang--Mills theory over $(\widetilde{M},
\widetilde{g})$ instead of the original space-time. The
restriction of the extended flat Yang--Mills bundle $\widetilde{E}
\vert_{\widetilde{N}}$ is trivial even in this case. Certain physical
quantities of the extended theory may suffer from singularities on the
boundary $\partial i(M)$ but classical Yang--Mills vacua extend smoothly
to the whole $(\widetilde{M}, \widetilde{g})$ as we have seen. In other
words the studying of the vacuum sector of the extended Yang--Mills theory
is correct.

Summing up, we can see that dynamically (i.e. homotopically) inequivalent
vacua of the Yang--Mills theory are classified by the
homotopy classes of smooth maps $\tilde{f}: \widetilde{V}\rightarrow G$
satisfying $\tilde{f}(i^0)=e\in G$, usually written as  
\begin{equation}
\left[(\widetilde{V}, i^0), (G, e)\right].
\label{vakuum}
\end{equation}
Now suppose that $(M, g)$ contains black hole(s). In this case
$\widetilde{V}$ is a simply connected manifold {\it with boundary}. 
Such manifolds, considered as CW-complexes, have only cells of dimension
less than three. Hence by the Cellular Approximation Theorem, every
map $\tilde{f}: \widetilde{V}\rightarrow G$ descends to a homotopic map
with values only on the cells of $G$ having dimension less than three.
Consequently, being $\pi_2(G)=0$, $G$ can be replaced by the simple
Postnikov-tower $P_2=K(\pi_1(G), 1)$ where $K(\pi_1(G), 1)$ is an
Eilenberg--Mac Lane space yielding
\begin{equation}
\left[(\widetilde{V}, i^0), (G,e)\right]\cong
\left[\widetilde{V},
K(\pi_1(G), 1)\right]\cong H^1(\widetilde{V},
\pi_1(G))=0.
\label{szamolas}
\end{equation}
The result is zero because $\widetilde{V}$ is simply connected. For
details, see for instance \cite{spa}. Consequently all vacuum states are
homotopy equivalent i.e. can be deformed into each other only through
vacuum states {\it over an outer, asymptotically flat portion $N$ of
the space-time $(M,g)$}. Clearly, classically only this part is relevant
for a distant observer. 

This result can be explained from a different point of view as well. Since
the outer part $N$ of $M$ is also globally hyperbolic, the space-like
submanifold $V$ forms a Cauchy surface for $N$. Consequently if we know
the initial values of two gauge fields, $A$ and  $A'$ say, on $V\subset
S$, we can determine their values over the whole {\it outer} space-time
$N\subset M$ by using the field equations. This implies that the values of
the fields $A$ and $A'$ ``beyond'' the event horizon in a moment are
irrelevant for an observer outside the black hole. But we just proved that
every vacuum fields restricted to $V$ are homotopic. Roughly speaking,
homotopical differences between Yang--Mills vacua ``can be swept'' into a
stationary black hole. 

Via (\ref{szamolas}) for arbitrary smooth functions
$\tilde{f}_0:\widetilde{V}\rightarrow G$ and 
$\tilde{f}_1:\widetilde{V}\rightarrow G$ there is a homotopy
\begin{equation}
\widetilde{F}_T: \widetilde{V}\times [0,1]\rightarrow G
\label{T-homotopia}
\end{equation}
satisfying $\widetilde{F}_T(x,0)=\tilde{f}_0 (x)$ and
$\widetilde{F}_T(x,1)=\tilde{f}_1 (x)$ for all $x\in\widetilde{V}$.
Taking two Cauchy-surfaces $V_0\subset S_0:=T^{-1}(0)$ and $V_1\subset
S_1:=T^{-1}(1)$ we can regard the two functions as vacua $\tilde{f}_0
:\widetilde{V}_0\rightarrow G$ and $\tilde{f}_1 : 
\widetilde{V}_1\rightarrow G$. In the homotopy $\widetilde{F}_T$ the
subscript $T$ shows that the ``time'' required for the homotopy is
measured by the time function $T$ naturally associated to the globally
hyperbolic space-time $(M,g)$. We call this homotopy as {\it
$T$-homotopy}.

But on physical grounds, such a deformation or homotopy is effective 
only if it can be carried out {\it in finite proper time according to a
distant observer's clock.} Such homotopies will be referred to as
{\it finite $\gamma$-homotopies} or {\it effective homotopies}.

Let $\gamma :\R^+\rightarrow N$ be a smooth, time-like, future directed
curve in the outer region $N\subset M$ with $\gamma (0)\in
V_0$ representing an observer moving in $N$. We denote by $\tau$ its
proper time i.e. the natural affine parameter of the curve
$\gamma$ obeying $g(\dot\gamma (\tau) ,\dot\gamma (\tau ))=-1$. 
Moreover, let $\beta_1: [0,1]\rightarrow V_1$ be a continuous space-like
curve in the fixed Cauchy-surface $V_1$ approaching one connected
component of the horizon i.e. $\beta_1(0)\notin H_1 =H\cap
V_1$ while $\beta_1(1)\in H_1$. Define
\[\tau_{\beta_1(s)}:=\inf\limits_{\tau\in\R}\left\{\gamma (\tau )\in
J^+(\beta_1(s))\right\} .\]
We prove the following simple lemma:
\vspace{0.1in}

{\bf Lemma}. {\it Let $(M,g)$ be an asymptotically flat, stationary
space-time with outer asymptotically flat region $N$ and black hole
region $B$ and event horizon $H=\partial N=\partial B$. Consider the
curves $\beta_1 :[0,1]\rightarrow V_1$ and $\gamma :\R^+\rightarrow N$
defined above. Then}
\[\lim\limits_{s\rightarrow 1}\tau_{\beta_1(s)}=\infty .\]


{\it Proof.} The proof is very simple. Clearly, if for a non-space-like
curve $\alpha :\R\rightarrow M$ the condition ${\rm im}\alpha\cap
B\not=\emptyset$ holds then ${\rm im}\alpha\subset B$ by the definition
of the black hole $B\subset M$. Consequently if by assumption
$\beta_1(1)\in H_1\subset B$ then all non-space-like curves $\alpha$ with
the property $\alpha (\tau ')=\beta_1(1)$ ($\tau '\in\R$) never enter $N$
hence never meet $\gamma$ showing $\gamma (\tau )\notin J^+(\beta_1(1))$
for all $\tau\in\R^+$ i.e. $\tau_{\beta_1(1)}=\infty$. By continuity
we get the result. $\Diamond$
\vspace{0.1in}
  
\noindent Now we are ready to understand the above homotopy from the
point of view of a distant observer. Fix an observer $\gamma$. Clearly,
the $T$-homotopy (\ref{T-homotopia}) can be written as a
$\gamma$-homotopy as
\[\widetilde{F}_T(x,t)=\tilde{f}_t(x)=\widetilde{F}_{\gamma}(x, \tau ).\]
Take $x:=\beta_1(s)$ then we get
\[\widetilde{F}_T(\beta_1(s),
1)=\tilde{f}_1(\beta_1(s))=\widetilde{F}_\gamma (\beta_1(s),
\tau_{\beta_1(s)}).\]
But by the Lemma we can see that as we approach the horizon, i.e.
$s\rightarrow 1$, $\tau_{\beta_1(s)}$ diverges hence typically the
$T$-homotopy cannot be a finite $\gamma$-homotopy in other
words it cannot be ``finished'' in finite proper time measured by
$\gamma$.

From here we can see that given a $T$-homotopy (\ref{T-homotopia}), it
gives rise to an effective homotopy if and only if there is a
neighbourhood $H\subset U\subset N$ of the horizon in $N$ with the
property $\tilde{f}_0\vert_{U_0}=\tilde{f}_1\vert_{U_1}$. This implies
\[\tilde{f}_0\vert_{H_0}=\tilde{f}_1\vert_{H_1}.\]
This result can be interpreted as a natural boundary condition on each
connected component of the horizon for effectively deformable vacua.
Since each boundary component is homeomorphic to the two-sphere $S^2$
and $\pi_2(G)=0$ we may select a function in each homotopy class
obeying $\tilde{f}_0\vert_{H_0}=e$.
We just remark that exactly this is the physical reason for keeping the
functions as identity in the space-like infinity $i^0$ when we discuss
homotopy classes of vacua over Minkowskian space-time.

Taking into account that $\partial\widetilde{V}_t=H_t$, {\it the classes 
of effectively deformable vacua} are given by the homotopy classes of
functions $\tilde{f}:\widetilde{V}\rightarrow G$ with the property
$\tilde{f}(\partial\widetilde{V})=\tilde{f}(i^0)=e\in G$. The homotopy is
restricted to obey these boundary conditions. This set is denoted by
\begin{equation}
\left[ (\widetilde{V}, \partial\widetilde{V}, i^0), (G,
e)\right]
\label{ujvakuum}
\end{equation}
and replaces (\ref{vakuum}). To get a more explicit description of
this set, we proceed as follows.


\section{Homotopic Classification}
First taking into account that a function
$\tilde{f}:\widetilde{V}\rightarrow G$ we are interested in satisfies
that it sends each connected component of $\partial\widetilde{V}$ into the
unit element $e\in G$, we can replace the 
three-manifold-with-boundary $\widetilde{V}$ with an oriented, closed,
simply connected three-manifold $W_k$ in the following way.
Let denote by $k>0$ the number of connected components of
$\partial\widetilde{V}$ (i.e. the number of black holes). As we have seen,
all such component is an $S^2$. Hence we can glue to each such component a
three-ball $B^3$ using the identity function of $S^2$ to get 
\[W_k:=\widetilde{V}\cup_{\partial\widetilde{V}}\underbrace{B^3
\cup ...\cup B^3}_k.\]
Clearly, $\tilde{f}$ extends as the identity to each ball giving rise to
the function $f: W_k\rightarrow G$.
Consequently, if we fix a point $x_n$ in each ball ($n\leq k$), then we
may equivalently consider functions obeying 
$f(x_1)=...=f(x_k)=f(i^0)=e$. Modifying the
allowed homotopies to obey this constraint, we can replace the homotopy
set (\ref{ujvakuum}) by
\[\left[ (W_k, x_1,...,x_k, i^0), (G,e)\right]\]
(of course if $k=0$ then no point except $i^0$ is distinguished in
$W_0$). We prove the following proposition:
\vspace{0.1in}

{\bf Proposition}. {\it Fix a number $k>0$ and consider the
connected, closed, simply connected three-manifold with $k+1$ distinguished 
points $(W_k, x_1,...,x_k, i^0)$ constructed above. Denote by
$(W_k, i^0)$ the same space with only one distinguished point.
Then there is a natural bijection}
\[\left[ (W_k, x_1,...,x_k, i^0), (G,e)\right]\cong\left[ (W_k, i^0),
(G,e)\right] .\]

 
{\it Proof.} Fix a number $k\geq 0$. First it is straightforward that if
two functions, $f_0$ and $f_1$ are homotopic in $\left[
(W_k, x_1,...,x_k, i^0), (G,e)\right]$ then they represent
the same homotopy class in $\left[ (W_k, i^0), (G,e)\right]$
i.e. they are homotopic in the later set as well. This is because the
allowed homotopies in $\left[ (W_k, i^0), (G,e)\right]$ are
less restrictive than in $\left[ (W_k, x_1,...,x_k, i^0), (G,e)\right]$. 

Consider the opposite way. It is not difficult to see that in each class
$[f]\in\left[ (W_k, i^0), (G,e)\right]$
there is a representant which belongs to $\left[ (W_k, 
x_1,...,x_k, i^0), (G,e)\right]$. Indeed, choose an arbitrary representant
$f\in [f]\in\left[ (W_k, i^0), (G,e)\right]$
and consider the pre-image $f^{-1}(e)\subset W_k$. This
pre-image contains the point $i^0\in W_k$ by construction.
Taking into account that $W_k$ is path-connected, we can 
deform $f^{-1}(e)$ to contain the points $x_1,...,x_k$ as well
producing a representant which belongs to $\left[ (W_k, x_1,...,x_k, i^0), 
(G,e)\right]$. 

Now suppose that there are two functions $f_0$ and $f_1$
which are homotopic in $\left[ (W_k, i^0), (G,e)\right]$ i.e.
there is a continuous function $F: (W_k, i^0)\times
[0,1]\rightarrow (G,e)$ with
\[F(x, 0)=f_0(x),\:\:\:F(x,1)=f_1(x),\:\:\:F(i^0, t)=e.\]
For the sake of simplicity, assume they represent elements in $\left[
(W_k, x_1,...,x_k, i^0), (G,e)\right]$, too. Then we have to
prove that they are also homotopic in $\left[ (W_k, 
x_1,...,x_k, i^0), (G,e)\right]$ i.e. there is a function $F':
(W_k, x_1,...,x_k, i^0)\times 
[0,1]\rightarrow (G,e)$ with the property
\[F'(x,0)=f_0(x),\:\:\:F'(x,1)=f_1(x),\:\:\:F'(x_1,t)=...=F'(x_k,t)=F'(i^0, 
t)=e.\]
From here we can see that the orbit of an arbitrary distinguished point
$x_n$ is a loop $l_n: [0,1]\rightarrow G$ under the homotopy $F$ while
the constant loop in the case of $F'$. Hence if these loops are
homotopically trivial in $G$ then we can deform $F$ into the homotopy $F'$
by shrinking the loops $l_1$,...,$l_k$.

Now we prove that this is always possible. First, if $\pi_1(G)=1$ i.e. the
compact Lie group is simply connected then certainly each loop $l_n$ is
homotopic to the constant loop. Consequently assume $\pi_1(G)\not=1$.
Consider a distinguished point $x_n\in W_k$ and two path $a_n:
[0, 1/2]\rightarrow W_k$ with $a_n(0)=i^0$ and $a_n(1/2)=x_n$
and $b_n: [1/2, 1]\rightarrow W_k$ with $b_n(1/2)= x_n$ and
$b_n(1)=i^0$. These give rise to a continuous loop $b_n*a_n
:[0,1]\rightarrow W_k$ with $b_n*a_n(0)=b_n*a_n(1)=i^0$. Here $*$ refers
to juxtaposition of curves, loops etc. 
The loop $b_n*a_n$ is homotopic to the trivial loop since $W_k$ is
simply connected. Consider the maps $\alpha^0_n:=f_0\circ a_n: [0,
1/2]\rightarrow G$ and $\beta^0_n:=f_0\circ b_n: [1/2,
1]\rightarrow G$. These are loops in $G$ hence so is their product
$\beta^0_n*\alpha^0_n$. Construct the same kind of loops
$\alpha^1_n:=f_1\circ a_n$ and $\beta^1_n:=f_1\circ b_n$.
The product loop $\beta_n^1*\alpha_n^1$ is homotopic in $G$ to
$\beta^0_n*\alpha^0_n$ i.e. 
$[\beta^0_n*\alpha^0_n]=[\beta_n^1*\alpha_n^1]$ because $f_0$ is
homotopic to $f_1$. It is clear that 
\[\beta_n^1*\alpha_n^1 =\beta^0_n*l_n*\alpha^0_n.\]
Consequently we can write for the homotopy classes in question
\[[\beta_n^1*\alpha_n^1]=[\beta^0_n*l_n*\alpha^0_n]=[\beta^0_n*\alpha^0_n
*l_n]=[\beta^0_n*\alpha^0_n][l_n]=[\beta_n^1*\alpha_n^1][l_n].\] 
In the second step we exploited the fact that a topological
group always has commutative fundamental group. This shows that $[l_n]=1$ 
that is the loop $l_n$ is contractible in $G$ for all $0\leq n\leq k$ in
other words the homotopy $F$ is deformable into a homotopy $F'$ yielding
$f_0$ and $f_1$ are homotopic in $\left[ (W_k, x_1,...,x_k, i^0),
(G,e)\right]$ as well. $\Diamond$
\vspace{0.1in}

\noindent The above Proposition enables us to give a more explicit
description of the set (\ref{ujvakuum}).
\vspace{0.1in}

{\bf Theorem 2}. {\it Let $G$ be a typical classical compact Lie group
i.e. let $G$ be $U(n)$ with $n\geq 2$, or $SO(n)$,} Spin$(n)$ {\it with
$n\not=4$, or $SU(n)$, $Sp(n)$ for all $n$, or
$G_2$, $F_4$, $E_6$, $E_7$, $E_8$. Then we have
\[\left[ (\widetilde{V}, \partial\widetilde{V}, i^0), (G,
e)\right]\cong\Z .\]
Moreover we have  
\[\left[ (\widetilde{V}, \partial\widetilde{V}, i^0), (U(1),    
e)\right]\cong 0,\]
and}
\[\left[ (\widetilde{V}, \partial\widetilde{V}, i^0), (SO(4),
e)\right]\cong\left[ (\widetilde{V},
\partial\widetilde{V}, i^0), (\mbox{Spin}(4), e)\right]\cong
\Z\oplus\Z\]
{\it for the remaining cases.}
\vspace{0.1in}

{\it Proof.} In light of the above considerations and the Proposition, we
have 
\[[(\widetilde{V}, \partial\widetilde{V}, i^0), (G,e)]\cong [(W_k,
x_1,...,x_k, i^0), (G,e)]\cong [(W_k, i^0), (G,e)].\]
Hence we can use the results of Isham who classified the set $[(W_k,
i^0), (G,e)]$ and it is summarized in \cite{isham1} in Table 1 on pp. 207.
But in our case $W_k$ is a connected, closed, simply connected
three manifold hence the above result follows. $\Diamond$
\vspace{0.1in}

\noindent {\it Remark.} We mention that assuming the validity of the three
dimensional Poincar\'e conjecture i.e. if $W_k\cong S^3$, then Theorem 2
can be derived without using Isham's result since in this case we have
simply $[(\widetilde{V}, \partial\widetilde{V}, i^0),
(G,e)]\cong\pi_3(G)$.

We can see by Theorem 2 that although the homotopy set
(\ref{ujvakuum}) of effectively deformable vacua is typically non-trivial,
it is remarkable more simple than in the original calculations of Isham et
al. based on topological considerations only. The homotopy sets listed in
Theorem 2 are exactly the same as for the flat Minkowskian case.
Being these vacua of definite winding numbers non gauge invariant, we have
to introduce again linear combinations as (\ref{teta}) in this more
general situation. Consequently we can see that approach (i) to the
$\theta$-parameter, mentioned in the Introduction, still makes sense in
the general case.
 
\section{Concluding Remarks}
In this letter we have studied the concept of $\theta$-vacua
in general Yang--Mills theories. In light of our results, we can see that
for outer observers in stationary, asymptotically flat space-times
$\theta$-vacua do occur in a Yang--Mills theory. Despite the possible 
complicated topology of the underlying Cauchy surface however, their
structure is similar to the flat Minkowskian case, due to the
causal structure of these space-times which is complicated in the general
case, too. Hence the introduction of the various new CP-violating terms
studied in \cite{isham5} is unnecessary.

The suppression of the topology of the underlying Cauchy surface is due to
Theorem 1 which is a consequence of the so-called Topological Censorship
Theorem of Friedman--Schleich--Witt \cite{fri-sch}. Consequently, the
reduction of the problem of the general CP-violation to the flat
Minkowskian case is essentially due to this result.  

The natural question arises: are there instanton solutions in the
corresponding Wick-rotated theories? What is the physical relevance of
these solutions? Do they induce semi-classical tunneling between the
vacuum states of different effective winding numbers? If yes, beyond (i)
we have another, more physical, reason to introduce $\theta$-vacua because
of concept (ii), also mentioned in the Introduction.

{\bf Acknowledgement}. The author is grateful to Prof. G.W. Gibbons for
the useful discussions and calling the author's attention to Isham's paper
and to Prof. H. Kodama for his important remarks concerning the subject
of this article.

\begin{thebibliography}{99}

\bibitem{tho} G. 't Hooft, Phys. Rev. Lett. {\bf 37} (1976) 8;

\bibitem{bpst}  A.A. Belavin, A.M. Polyakov, A.S. Schwarz,
Yu.S. Tyupkin, Phys. Lett. {\bf B59} (1975) 85;

\bibitem{pec} R.P. Peccei, H.R. Quinn, Phys. Rev. {\bf D16} (1977) 1791;

\bibitem{che} L-P. Cheng, L-F.Li: Gauge Theory of Elementary
Particle Physics, Clarendon Press (1984);

\bibitem{kak} M. Kaku: Quantum Field Theory, Oxford University
Press, Oxford (1993);

\bibitem{isham1} C.J. Isham, in: Old and new Questions in Physics,
Cosmology, Philosophy and Theoretical Biology, Ed.: A. Van Der Merwe,
Plenum Press, New York (1983) 189; 

\bibitem{isham2} C.J. Isham, Trieste Diff. Geom. Meth. (1981) 171;

\bibitem{isham3} C.J. Isham, G. Kunstatter, Journ. Math. Phys. 
{\bf 23} (1982) 1668;

\bibitem{isham4} C.J. Isham, G. Kunstatter, Phys. Lett. {\bf B102}
(1981) 417;

\bibitem{isham5} S. Deser, M.J. Duff, C.J. Isham, Phys. Lett. {\bf B93} 
(1980) 419;

\bibitem{wit} D.M. Witt, Phys. Rev. Lett. {\bf 57} (1986) 1386;

\bibitem{gan} D. Gannon, J. Math. Phys. {\bf 16} (1975) 2364;

\bibitem{chr-wal} P.T. Chru\'sciel, R.M. Wald, Class. Quant. Grav.
{\bf 11} (1994) L147;

\bibitem{haw} S.W. Hawking, Phys. Rev. {\bf D53} (1996) 3099;

\bibitem{pres-triv} J. Preskill, S.P. Trivedi, M.B. Wise, Phys. Lett.
{\bf B223}, no.1. (1989) 26;

\bibitem{wal} R.M. Wald: General Relativity, Univ. of Chicago Press,
Chicago (1984);

\bibitem{spa} E.H. Spanier: Algebraic Topology, Springer--Verlag, Berlin
(1966);

\bibitem{fri-sch} J.L. Friedman, K. Schleich, D.M. Witt, Phys. Rev. Lett.
{\bf 71} (1993) 1486;

\end{thebibliography}
\end{document}

