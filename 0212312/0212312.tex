\documentclass[a4paper,aps,twocolumn,prd,showpacs,nofootinbib]{revtex4}
\usepackage{graphicx}
\usepackage{dcolumn}
\usepackage{bm}
\usepackage{amssymb}

\bibliographystyle{unsrt}

\def\setC{\mathbb{C}}
\def\setG{\mathbb{G}}
\def\setO{\mathbb{O}}
\def\setH{\mathbb{H}}
\def\setR{\mathbb{R}}
\def\setC{\mathbb{C}}
\newcommand{\mP}{M_{_{\rm Pl}}}
\newcommand{\GN}{G_{_{\rm N}}}
\newcommand{\dd}{\mathrm{d}}
\newcommand{\ee}{\mathrm{e}}
\newcommand{\DD}{\mathcal{D}}
\newcommand{\bx}{\mathbf{x}}
\newcommand{\bk}{\mathbf{k}}
\newcommand{\gsim}{\gtrsim}
\newcommand{\lsim}{\lesssim}
\newcommand{\rdim}{r}
\newcommand{\si}{\scriptscriptstyle}
\newcommand{\radim}{\rho}
\newcommand{\calF}{\mathcal{F}}
\newcommand{\calV}{\mathcal{V}_{_1}}
\newcommand{\calVV}{\mathcal{V}_{_2}}
\newcommand{\sfx}{s_{_\mathrm{F}}}
\newcommand{\lfx}{l_{_\mathrm{F}}}
\newcommand{\mfx}{m_{_\mathrm{F}}}
\newcommand{\ffx}{f_{_\mathrm{F}}}
\newcommand{\Qfx}{Q_{_\mathrm{F}}}
\newcommand{\wfx}{w_{_\mathrm{F}}}
\newcommand{\zero}{{_{0}}}
\newcommand{\one}{{_{1}}}
\newcommand{\uw}{\mathrm{w}}
\newcommand{\uf}{\mathrm{f}}
\newcommand{\us}{\mathrm{s}}
\newcommand{\kappaw}{\kappa_{_\uw}}
\newcommand{\GReCO}{${\cal G}\setR\varepsilon\setC{\cal O}$}
\newcommand{\kappaf}{\kappa_{_\uf}}
\newcommand{\kappas}{\kappa_{_\us}}
\newcommand{\radimF}{\radim_{_\infty}}
\newcommand{\Qsminf}{\left(Q/\sqrt{m}\right)_{_\infty}}
\newcommand{\minf}{m_{_\infty}} \newcommand{\etal}{{\sl et al.}}
\def\Integer{\mathsf{Z\hspace{-0.4em}Z}} \def\Natural{\mathrm{I\!N}}
\def\Real{\mathrm{I\!R}}

\begin{document}

\title{Regular cosmological bouncing solutions in \\ low energy
effective action from string theories.}

\author{J. C. Fabris}
\email{fabris@cce.ufes.br}
\affiliation{Departamento de F\'{\i}sica, Universidade Federal do
Esp\'{\i}rito Santo, 29060-900, Vit\'oria, Esp\'{\i}rito Santo,
Brazil}

\author{R. G. Furtado}
\email{furtado@cce.ufes.br}
\affiliation{Departamento de
F\'{\i}sica, Universidade Federal do Esp\'{\i}rito Santo,
29060-900, Vit\'oria, Esp\'{\i}rito Santo, Brazil}

\author{Patrick Peter}
\email{peter@iap.fr}
\affiliation{Institut d'Astrophysique de
Paris, \GReCO, FRE 2435-CNRS, 98bis boulevard Arago, 75014 Paris,
France}

\author{N. Pinto-Neto}
\email{nelsonpn@cbpf.br}
\affiliation{Lafex - Centro Brasileiro
de Pesquisas F\'{\i}sicas, CBPF. Rua Xavier Sigaud, 150, Urca,
CEP22290-180, Rio de Janeiro, Brazil}

\date{7 January 2003}

\begin{abstract}
The possibility of obtaining singularity free cosmological solutions
in four dimensional effective actions motivated by string theory is
investigated. In these effective actions, besides the Einstein-Hilbert
term, the dilatonic and the axionic fields are also considered as well
as terms coming from the Ramond-Ramond sector. A radiation fluid is
coupled to the field equations, which appears as a consequence of the
Maxwellian terms in the Ramond-Ramond sector. Singularity free
bouncing solutions in which the dilaton is finite and strictly
positive are obtained for models with flat or negative curvature
spatial sections when the dilatonic coupling constant is such that
$\omega < - 3/2$, and only for models with negative curvature spatial
sections when $\omega > - 3/2$, including the pure string case
$\omega=-1$.  The bounces are smoothly connected to the radiation
dominated expansion phase of the standard cosmological model, and the
asymptotic pasts correspond to very large flat spacetimes.
\end{abstract}

\pacs{04.20.Cv, 04.20.Dw, 98.80.Cq} \maketitle

\section{Introduction}

Superstring is the most promising candidate to describe a unified
theory of all interactions, gravity included. There are five
consistent superstring theories in 10 dimensions, which are connected
among themselves through duality transformations. To each superstring
theory, there is a corresponding supergravity theory in 10
dimensions. All of them can be obtained from the 11 dimensional
supergravity theory. This indicates that those superstring theories
are different manifestations of a unique 11 dimensional framework,
that has been named
$M$-theory~\cite{polchinski,green,kiritsis}. Moreover, the superstring
type-IIB can be recast in a more geometrical form in a 12 dimensional
model, suggesting that perhaps a yet more fundamental framework may
exist in 12 dimensions, which has been called $F$-theory~\cite{pope}.

The physical properties of superstring theories become relevant at
energy scales comparable with the Planck scale. This renders very
improbable that superstring phenomenology may be tested in the near
future in some laboratory experiment (see, however, Ref.~\cite{randal}
in which the Planck mass is lowered to TeV scale by accounting for
large extra-dimensions). According to the hot big-bang scenario,
however, energy scales even as high as the usual Planck scale
($\mP\sim 10^{19}$GeV) may have been reached in the very early
universe. Hence, for the moment, cosmology seems to be the most
natural arena where the consequences of superstring theories may be
tested. Perhaps, some relics of a cosmological string phase may be
identified~\cite{bdp}, opening the possibility of testing superstring
models. Furthermore, superstring theories open the possibility that
some typical drawbacks of the standard cosmological model, like the
existence of an initial singularity, may be solved in the context of
superstring cosmological models.  The goal of the present paper is to
show that, under certain conditions, it is possible to obtain
completely regular bouncing cosmological models in the context of
effective actions constructed from superstring theories (not
involving, in particular, negative energies~\cite{ppnpn2}), for which,
moreover, the dilaton is strictly positive and never diverges.

The search of singularity free cosmology in string theories is not a
new subject~\cite{lidsey,vasquez,picco,kirill,branden}. The string
action at tree level does not lead in general to singularity free
cosmological solutions, at least when the strict string case ($\omega
= - 1$, $\omega$ being the dilatonic coupling parameter) is
considered. The pre-big bang model~\cite{gasperini}, which is an
example of a string cosmology, requires the introduction of non-linear
curvature terms in order to achieve a smooth transition from a
curvature growing phase to a curvature decreasing phase. If large
negative values of the dilatonic coupling parameter $\omega$ are
allowed, it is possible, in some cases, to obtain completely regular
models, including in the dilatonic sector~\cite{kirill}. This may be
achieved mainly in models with spatial sections with negative
curvature.  Here, it will be shown that regular cosmological models
may be obtained, even in the string case, if a radiation fluid is
coupled to the string action at the tree level. Such a radiation fluid
can have a fundamental motivation, for example, in the case of the
superstring type IIB theory, where a 5-form appears in the
Ramond-Ramond sector. Truncation and dimensional reduction of this
5-form lead to a Maxwell term in four dimensions with the desired
features~\cite{fabris}. Hence, the model to be studied here is totally
based on superstring theories.  The string motivated phenomenological
term included under the form of a radiation fluid makes possible to
connect smoothly such string cosmological models to the radiation
phase of the standard cosmological model before nucleosynthesis.

In Ref.~\cite{picco}, models motivated by string theory similarly
including a radiation fluid have been studied, restricted to flat
spatial sections and $\omega > -3/2$.  In such cases, bouncing
solutions have been obtained only for $\omega < -4/3$. Furthermore,
for these solutions, the dilaton vanishes in the infinite past,
raising doubts on the validity of the tree level action in such
region. In the present paper, the curvature of the spatial sections
and the value of $\omega$ are kept arbitrary.  New bouncing regular
solutions are then obtained, for which, as mentioned above, the
dilaton remains finite and strictly positive at all times, even for
the strict string case ($\omega = -1$) provided the spatial section
has negative curvature.

In the following section, we derive effective string motivated actions
in four dimensions, which we use in section III to derive non singular
cosmological solutions which are thoroughly discussed.  We end up with
the conclusions in section IV.

\section{The effective action}

Our analysis is based on the following effective action at tree level:
\begin{eqnarray}
L = \sqrt{-\tilde g}\ee^{-\tilde\sigma}\biggl(\tilde R -
\omega\tilde\sigma_{;\si{A}}\tilde\sigma^{;\si{A}} -
\frac{1}{12}H_{\si{ABC}}H^{\si{ABC}}\biggr)\nonumber \\ - \sqrt{-
\tilde g}\biggr(\frac{1}{2}\xi_{;\si{A}}\xi^{;\si{A}} +
\frac{1}{240}F_{\si{ABCDE}}F^{\si{ABCDE}}\biggr)\label{lagrange}
\end{eqnarray}
where $\tilde\sigma$ is the dilatonic field, $H_{\si{ABC}}$ is the
axionic field, and $\omega$ is the dilatonic coupling constant.  The
two last terms come from the Ramond-Ramond sector of superstring type
IIB. The tildes indicate that all quantities are considered in a
$D$-dimensional space-time, $D = 10$ in the pure superstring context.

The dilatonic coupling constant is $\omega = - 1$ for usual
superstring theory. However, this may not necessarily be the case for
some ten dimensional theories stemming from a more fundamental one in
higher dimensions. In some specific situations, the value of $\omega$
can be found to be even less than $-3/2$.  As an example, the
superstring type IIB action may be reformulated in 12 dimensions, in
the context of the so-called $F$-theory. A low energy limit of the
$F$-theory has been studied by~\cite{pope}, where an action in 12
dimensions has been established, which was shown to lead to the low
energy limit of the superstring type IIB in 10 dimensions through
truncation and dimensional reduction. Let us consider this twelve
dimensional action, given by~\cite{pope}
\begin{eqnarray}
L_{12} = \sqrt{-\tilde g}\biggl(\tilde R -
\frac{1}{2}\Psi_{;\si{A}}\Psi^{;\si{A}} -
\frac{1}{48}\ee^{a\Psi}F_{\si{ABCD}}F^{\si{ABCD}} \nonumber \\
-\frac{1}{240}\ee^{b\Psi}G_{\si{ABCDE}}G^{\si{ABCDE}} + \lambda
B_4{\scriptstyle \wedge} \dd A_3{\scriptstyle \wedge} \dd A_3\biggr),
\label{f12}
\end{eqnarray}
with $a^2 = - 1/5$ and $b^2 = - 4/5$, $\lambda$ being a coupling
parameter for the Chern-Simons type term involving the potentials of
the five and four-forms. Writing the metric as
\begin{equation}
\dd s_{12}^2 = g_{\mu\nu}\dd x^\mu \dd x^\nu - \ee^{2\beta}\dd x_i \dd
x^i ,
\end{equation}
with Greek indices $\mu, \nu$ running from 0 to 9 and Latin indices
$i\in [10,11]$, and setting the five-form equal to zero, we obtain the
following Lagrangian
\begin{eqnarray}
L_{10} & = & \sqrt{-g}\ee^{2\beta}\biggr(R +
2\beta_{;\rho}\beta^{;\rho} - \frac{1}{2}\Psi_{;\rho}\Psi^{;\rho}
\nonumber \\ &- &\frac{1}{12}\ee^{a\Psi - 2\beta}
F_{\mu\nu\lambda}F^{\mu\nu\lambda} - \frac{1}{8}\ee^{a\Psi -
4\beta}F_{\mu\nu}F^{\mu\nu}\biggl),
\end{eqnarray}
where we have retained just the two and three-forms coming from the
four-form in the original action. The term originating the three-form
was made purely imaginary in 12 dimensions. Choosing $\Psi = 2\beta/a$
and defining $\phi = \ee^{2\beta}$, one ends up with the following
action in 10 dimensions:
\begin{equation}
L_{10} = \sqrt{-g}\biggl[\phi\biggl(R +
3\frac{\phi_{;\rho}\phi^{;\rho}}{\phi^2} -
\frac{1}{12}F_{\mu\nu\lambda}F^{\mu\nu\lambda}\biggl) -
\frac{1}{8}F_{\mu\nu}F^{\mu\nu}\biggr]\label{f10}.
\end{equation}
One can see that, in this case, we obtain an action with $\omega = -
3$ together with a Maxwell term (which generates the radiation
fluid). This is a remarkable example on how an effective string action
with $\omega \neq -1$ (in this case, $\omega = -3 < -3/2$) can be
realized. That is why we will maintain the value of $\omega$ in
Eq.~(\ref{lagrange}) arbitrary in what follows, unless otherwise
specified.

The $D$-dimensional metric is written as
\begin{equation}
\dd s^2 = g_{\mu\nu}\dd x^\mu \dd x^\nu -
\ee^{2\beta}\delta_{ij}\dd x^i \dd x^j,
\end{equation}
where $g_{\mu\nu}$ is the four dimensional metric, $\ee^\beta$ is the
scale factor of the $d = D - 4$ dimensional internal space which we
suppose to be homogeneous and flat. For now on, we will consider a
static internal space. This is not obligatory in some of the cases to
be analyzed latter, but such a restriction considerably simplifies the
unified presentation of many different cases allowed by the action
given by Eq.~(\ref{lagrange}).

Dimensional reduction and isotropization of the Maxwellian term [which
may come from the Ramond-Ramond sector, or as described in the passage
from Eq.~(\ref{f12}) to Eq.~(\ref{f10})], lead to the following
effective action in four dimensions
\begin{equation}
\label{l1} {\it L} = \sqrt{-g}\biggl[\phi\biggl(R -
\omega\frac{\phi_{;\rho}\phi^{;\rho}}{\phi^2} -
\frac{\Psi_{;\rho}\Psi^{;\rho}}{\phi^{2}}\biggr) -
\frac{1}{2}\xi_{;\rho}\xi^{;\rho}\biggr] + L_{\rm r},
\end{equation}
where for now on Greek indices run from 0 to 3.  In this action, $\phi
= \ee^{-\tilde\sigma}$ is the dilaton, the field $\Psi$ comes from the
axionic term, and is thus called the axion, while $L_{\rm r}$
represents an ordinary radiation fluid term, which can be obtained
from the five-form existing in the Ramond-Ramond sector, as was
stressed above. We shall also call $\xi$ the RR-scalar as is
originates from the same sector.

{}From Eq.~(\ref{l1}), we obtain the field equations
\begin{eqnarray}
R_{\mu\nu} - \frac{1}{2}g_{\mu\nu}R &=& 8\pi\frac{T}{\phi} +
\frac{\omega}{\phi^2}\biggr(\phi_{;\mu}\phi_{;\nu} -
\frac{1}{2}g_{\mu\nu}\phi_{;\rho}\phi^{\rho}\biggl) \nonumber \\ & &
+\frac{1}{\phi}\biggr(\phi_{\mu\nu} - g_{\mu\nu}\Box\phi\biggl)
\nonumber\\ & &+\frac{1}{\phi^2}\biggr(\Psi_{;\mu}\Psi_{;\nu} -
\frac{1}{2}g_{\mu\nu}\Psi_{;\rho} \Psi^{;\rho}\biggl) \nonumber \\ & &
+ \frac{1}{\phi}\biggr(\xi_{;\mu}\xi_{;\nu} -
\frac{1}{2}g_{\mu\nu}\xi_{;\rho}\xi^{;\rho}\biggl),
\end{eqnarray}
for the Einstein part,
\begin{equation}
\Box\phi + \frac{2}{3 + 2\omega}\phi^{-1}\Psi_{;\rho}\Psi^{;\rho} +
\frac{1}{3 + 2\omega}\xi_{;\rho}\xi^{;\rho} = \frac{8\pi T}{3 +
2\omega},
\end{equation}
with $T\equiv T^\mu_{\ \ \mu}$, for the dilaton $\phi$, while we get
\begin{equation}
\Box\Psi - \Psi_{;\rho}\frac{\phi^{;\rho}}{\phi}=0,
\end{equation}
to describe the dynamics of the axion $\Psi$, and finally
\begin{eqnarray}
\Box\xi &=& 0,\\ {T^{\mu\nu}}_{;\mu} &=& 0,
\end{eqnarray}
for the RR-scalar $\xi$ and the radiation fluid respectively. These
equations we now implement in a cosmological context.

\section{Cosmological solutions}

Introducing the Friedman-Robertson-Walker metric
\begin{equation}
\dd s^2 = \dd t^2 - a(t)^2\left[\frac{\dd r^2}{1 - kr^2} +
r^2\left(\dd\theta^2 + \sin^2\theta \dd\phi^2\right)\right],
\end{equation}
$k$ being the normalized curvature of the maximally symmetric spatial
sections ($k = 0, \pm 1$), and assuming the fields now depend only on
time, the field equations derived above reduce to the following
equations of motion:
\begin{equation}
\label{em1} 3\biggr(\frac{\dot a}{a}\biggl)^2 + 3\frac{k}{a^2}
= 8\pi\frac{\rho}{\phi} +
\frac{\omega}{2}\biggr(\frac{\dot\phi}{\phi}\biggl)^2 - 3\frac{\dot
a}{a} \frac{\dot\phi}{\phi} + \frac{\dot\Psi^2}{2\phi^{2}} +
\frac{\dot\xi^2}{2\phi},
\end{equation}
which is the generalization of the Friedman equation, and
\begin{eqnarray}
\label{em2}
\ddot\phi + 3\frac{\dot a}{a}\dot\phi + \frac{2}{(3 +
2\omega)}\frac{\dot\Psi^2}{\phi} + \frac{\dot\xi^2}{(3 + 2\omega)} &=&
\frac{8\pi(\rho - 3p)}{(3 + 2\omega)},\nonumber \\ \\ \label{em3}
\ddot\Psi + 3 \frac{\dot a}{a}\dot\Psi - \dot\Psi\frac{\dot\phi}{\phi}
&=& 0,\\
\label{em4}
\ddot\xi + 3\frac{\dot a}{a}\dot\xi &=& 0,\\
\label{em5} \dot\rho + 3\frac{\dot a}{a}(\rho + p) &=& 0.
\end{eqnarray}
In these expressions, $\rho$ is the energy density and $p$ is the
pressure of some perfect fluid which obeys, for the sake of
generality, a barotropic equation of state, $p = \lambda\rho$, with
$\lambda$ an arbitrary constant. Later on, we will specialize this
fluid to the case we are interested in, namely, radiation.  A dot
stands for a derivative with respect to the cosmic time $t$.

Eqs.~(\ref{em3}), (\ref{em4}) and (\ref{em5}) admit the first
integrals
\begin{equation}
\dot\Psi = \frac{A\phi}{a^3}, \quad \dot\xi = \frac{B}{a^3}, \quad
\rho = D a^{- 3(1 + \lambda)}, \label{solPsi}
\end{equation}
where $A$, $B$ and $D$ are integration constants. According to the
string motivated action discussed above, let us now specialize the
equations for the radiation fluid case ($\lambda = 1/3$).  For this
specific case, Eq.~(\ref{em2}) simplifies to
\begin{equation}
\label{scalar} \ddot\phi + 3\frac{\dot a}{a}\dot\phi + \frac{2}{(3
+ 2\omega)}\frac{A^2}{a^6}\phi + \frac{B^2}{(3 + 2 \omega)a^6} = 0,
\end{equation}
which can be solved in the following way. It is convenient to define a
new time coordinate $\theta$ given by the relation 
\begin{equation}
\dd t = a^3 \dd\theta.
\label{parameter}
\end{equation}
In terms of this new time coordinate, Eq.~(\ref{scalar}) reads
\begin{equation}
\label{scalar'} \phi'' + \frac{2A^2}{(3 + 2\omega)}\phi +
\frac{B^2}{(3 + 2 \omega)} = 0,
\end{equation}
where primes denote differentiations with respect to $\theta$.
Similarly, Eq.~(\ref{em1}), when expressed in terms of $\theta$, reads
\begin{equation} \left({a'\over a^3}\right)^2 +k={M\over a^2 \phi} +
{\omega \over 6} {\phi'^2 \over a^4 \phi^2} - {a'\phi'\over a^5 \phi}
+{1\over 6 a^4} \left( A^2 + {B^2\over
\phi}\right),\label{aprim}\end{equation} in which use has been made of
Eq.~(\ref{solPsi}), 
and we have set $M = 8\pi D/3$, which is dimensionless in the
radiation case.

Eq.~(\ref{aprim}) may be recast in a very convenient form through the
redefinition $a = \phi^{-1/2}b$, which implies to change to the
so-called Einstein frame. This yields
\begin{equation}
\left({b'\over b}\right)^2 + ( kb^2-M) {b^2\over\phi^2} = {1\over
6}\left( A^2 +{B^2\over \phi} + {3+2\omega\over 2}
{\phi'^2\over\phi^2}\right), \label{bprim}
\end{equation}
whose solution we next investigate. Notice, however, that we want to
keep considering the Jordan frame as the physical frame; the conformal
transformation above is introduced only for technical reasons. The
solutions of Eqs.~(\ref{scalar'}) and (\ref{em1}), with the
redefinition made above for the scale factor, depend on the sign of
the term $3 + 2\omega$ and on the presence of the Ramond-Ramond scalar
field. We will consider each case separately. For simplicity, we will
call $3 + 2\omega > 0$ (respectively $< 0$) as the normal
(resp. anomalous) case, and $\xi =$ constant (respectively $\xi \neq$
constant) as the axionic (resp. RR) case.  In what follows, the
quantities $\phi_0$ and $a_0$ are constants of integration subject to
the constraints indicated in each case.

\subsection{Normal axionic case}

In this first case for which $\xi$ is constant [i.e., $B=0$ in
Eq.~(\ref{solPsi})] and $\omega > -3/2$, the solution of
Eq.~(\ref{scalar'}) is given by
\begin{equation}
\phi(\theta) = \phi_0\sin(\alpha\theta),\label{phiNA}
\end{equation}
where
\begin{equation}
\alpha = \sqrt{\frac{2A^2}{3 + 2\omega}}
\end{equation}
and we have chosen $\phi (0)=0$.

Plugging this solution into Eq.~(\ref{bprim}) yields
\begin{equation}
\phi_0^2 \sin^2(\alpha\theta) b'^2 = (C^2 + M b^2 - k b^4 ) b^2,
\label{bNA}
\end{equation}
where 
\begin{equation} 
C^2 = \frac{1}{6}A^2\phi_0^2.\label{C}
\end{equation}
We are seeking regular bouncing solutions for which the scale factor
is bounded from below but can grow arbitrarily large, while $\phi$ is
non vanishing and finite. This means that the function $b$ should also
grow indefinitely on both sides of the bounce. As can be seen by
inspection of Eq.~(\ref{bNA}), a necessary condition for this to
happen is that the curvature be non-positive. This is to be contrasted
with the general relativistic case for which a positive curvature is a
pre-requisite to ensure that a bounce is possible~\cite{ppnpn1}, and
can be understood by stating that, in the case at hand, a positive
curvature implies a finite scale factor at all times.

Under the assumption that both sides are positive definite, one can
integrate Eq.~(\ref{bNA}), written as
\begin{equation}
\int_{b_0}^{b} {\dd \tilde b \over \tilde b \sqrt{C^2 + M \tilde b^2 -
k \tilde b^4}} = \pm \int_{\theta_0}^\theta {\dd \tilde \theta\over
\phi_0 \sin(\alpha\tilde \theta)}, \label{integr}
\end{equation}
to provide the solution \{see, e.g., Ref.~\cite{Grad}, Eq.~(2.266)\}
\begin{equation}
{g(b)\over g(b_0)} = {f(\theta)\over f(\theta_0)},
\end{equation}
where
\begin{equation}
f(\theta) = \left| \tan
\left(\frac{\alpha\theta}{2}\right)\right|^p,\label{fNA}
\end{equation}
and
\begin{equation}
g(b) = {M\over C} + {2\over b^2} \left( C+\sqrt{C^2 + M b^2
-kb^4}\right),
\end{equation}
$b_0$ and $\theta_0$ being constants of integration that we choose
such that $C g(b_0) = f(\theta_0)$ for further convenience, and
\begin{equation}
p = \pm \sqrt{1 + \frac{2}{3}\omega}.
\end{equation}
Setting $a_0^2 = 4 C^2/\phi_0$, we finally get
\begin{equation}
a(\theta) =
\frac{a_0}{\sqrt{\sin\alpha\theta}}\left\{\frac{f(\theta)}
{\left[M - f(\theta)\right]^2 + 4C^2k}\right\}^{1/2}\label{ana},
\end{equation}
which is the desired result for the scale factor. Note that because
of the trigonometric identity
\begin{equation}
\tan \left[ -\left( {\alpha\theta +\pi/2\over 2}\right) \right] =
\left[ \tan \left( {\pi/2-\alpha\theta \over 2}\right) \right]^{-1},
\end{equation}
the solution~(\ref{ana}) with $p\to -p$ can be straightforwardly
deduced from the original one by a mirror symmetry with respect to the
point $\alpha\theta=\pi/2$. It is thus sufficient to consider $p>0$
and we shall in what follows restrict our attention to this case.

These solutions have some interesting features. As we have already
discussed, for $k = 1$, there are no bouncing solutions. On the other
hand, for $k = 0$ or $k= - 1$, it is possible to choose the parameters
in such a way that the extremes of the range of validity of the
variable $\theta$ occur for $t \rightarrow \pm \infty$, where
spacetime becomes flat.

The case $k=0$ was presented in Ref.~\cite{picco}; let us recall it
briefly for the sake of completeness.  The denominator in
Eq.~(\ref{ana}) has only two roots if $k=0$, and the parameter
$\theta$ varies from $\theta_{\rm i} =0$ to $\theta_{\rm f} =
2\alpha^{-1}\arctan(M^{1/p})$. Bouncing non singular solutions are
possible only when $-3/2 < \omega < -4/3$. This can be seen by
considering the limit for which $\theta \to \theta_{\rm i}=0$. There,
the scale factor is $a\propto\theta^{(p-1)/2}$, and, from
Eq.~(\ref{parameter}), $t\propto \theta^{(3p-1)/3}$, yielding
$a\propto |t|^{(p-1)/(3p-1)}$. As $a(t)$ is a power law (disregarding
the exceptional cases $p=1\Leftrightarrow\omega=0$, and
$p=1/3\Leftrightarrow\omega = -4/3$, also discussed in
Ref.\cite{picco}), the scalar curvature for $k=0$, is proportional to
$t^{-2}$, which converges (in fact, goes to zero) only if $t \to
\infty$ as $\theta \to 0$. This happens only for $p<1/3$, which yields
$-3/2 < \omega < -4/3$.  Note, however, that for $\theta=0$ the
dilaton $\phi$ vanishes, independently of the value of $\omega$,
rendering dubious the validity of the tree level action
(\ref{lagrange}) in this region.

When $k = -1$, the denominator in Eq.~(\ref{ana}) has now three
roots. One can take the parameter $\theta$ varying from $\theta_{\rm
i} =2\alpha^{-1}\arctan[(M-2C)^{1/p}]$ to $\theta_{\rm f} =
2\alpha^{-1}\arctan[(M+2C)^{1/p}]$. Provided\footnote{Considering
$G_{\rm eff} = \GN/\phi$, where $\GN$ is the value of the
gravitational coupling today, recovering the units in the string case
($\omega =-1$) by making the replacements $\phi \rightarrow \phi/\GN$,
$\Psi \rightarrow \Psi/\GN$, $t\rightarrow a_0 t$, $a_0 \approx 1/H_0$
($H_0$ being the present Hubble parameter, which we choose to be our
inverse unit of time), and assuming the present amount of radiation
($\rho _{0{\rm r}} = \Omega_{0{\rm r}} \rho_{\rm c} \sim 10^{-4}
\rho_{\rm c} \approx 10^{-33} {\rm g/cm}^3$, with $\rho_{\rm c}$ the
critical density), we obtain from Eq.~(\ref{em1}), assuming the
radiation term to dominate at the time under consideration, that
$M\sim 8\pi G \rho_{0{\rm r}} H_0^{-2}/3 = \Omega_{0{\rm r}} \approx
10^{-4}$. This implies, provided $C\approx 10^{-4}\alt M/2$, that
$\sin(\alpha\theta_{\rm f})\approx 10^{-6}$, so that we must set
$\phi_0\approx 10^{6}$ in order to have $\phi_{\rm f} =1$ now. The
value of $C$, in turn, fixes $A\approx 10^{-10}$ through
Eq.~(\ref{C}). As a consequence, we find that the effective
gravitational constant $\GN/\phi$ was one order of magnitude larger in
the past than it is today. More significant enhancements are possible
but they require some fine tuning between $M$ and $2C$.} $2C < M$, we
have $0 < \alpha\theta < \pi$, and the dilatonic field given by
Eq.~(\ref{phiNA}) is strictly positive and finite, taking constant
values in the asymptotic regions.

Let us now consider the limit $\theta\to\theta_{\rm i}$ or
$\theta\to\theta_{\rm f}$. Setting $\alpha\theta = \alpha\theta_{\rm
i} +\varepsilon$ or $\alpha\theta = \alpha\theta_{\rm f}
-\varepsilon$, and expanding the denominator around $\varepsilon =0$,
we get $a\propto \varepsilon^{-1/2}$, from Eq.~(\ref{parameter})
$|t|\propto \varepsilon^{-1/2}$, and finally $a\propto |t|$,
independently on the value of $p$ or $\omega$.  As we are considering
$k=-1$, this limit corresponds to Milne flat spacetime. The scale
factor~(\ref{ana}) thus represents, with this choice of range for
$\theta$, a universe contracting from a Milne spacetime to a minimum
size, bouncing to an expansion phase, and ending asymptotically also
in a Milne spacetime, passing smoothly through a standard cosmological
model radiation dominated phase before that.  It is very important to
notice that these solutions are valid for $\omega > - 3/2$, which
includes the string case ($\omega = -1$). As the dilaton is finite and
strictly positive, there are also no singularities in the string
expansion parameter given by $g_s^2 = \phi^{-1}$ (which is not the
case for $k=0$), and the tree level approximation can be trusted all
along.  Consequently, we have obtained a perfectly regular bouncing
solution in the string framework, without any singularity, even in the
dilatonic field, when the curvature of the spatial section is
negative.

\subsection{Anomalous axionic case}

For $3 + 2\omega <0$, the previous solution for Eq.~(\ref{scalar'})
must be replaced by
\begin{equation}
\phi(\theta) = \phi_0\sinh (\alpha\theta),\label{phiAA}
\end{equation}
where now
\begin{equation}
\alpha = \sqrt{\frac{-2A^2}{3 + 2\omega}},
\end{equation}
and, as before, we have imposed $\phi(0)=0$.

Again, inserting this solution into Eq.~(\ref{bprim}) yields
\begin{equation}
\phi_0^2 \sinh^2(\alpha\theta) b'^2 = (M b^2 -C^2 - k b^4 ) b^2,
\label{bAA}
\end{equation}
where $C$ is as before [Eq.~(\ref{C})]. The same argument concerning
the existence of a bouncing solution apply, namely, that such
solutions cannot exist for $k=1$. Manipulations similar to those of
the previous case then lead to
\begin{eqnarray}
f(\theta) &=& \ln\left[\left| \tanh
\left(\frac{\alpha\theta}{2}\right) \right|^p \right],\\ g(b) &=&
\arcsin \left( {M b^2 - 2 C^2\over b^2 \sqrt{M^2-4kC^2} }\right) ,
\end{eqnarray}
where we have assumed $M^2-4kC^2 > 0$ (recall we are only interested
in the cases $k=0$ and $k=-1$). We now choose $f(\theta_0) = g(b_0)$
and set
\begin{equation}
p = \pm \sqrt{-\left( 1 + \frac{2}{3}\omega\right)}, \ \ \hbox{ and }
\ \ \ a_0^2 = {A^2 \phi_0\over 3 M},
\end{equation}
to obtain the scale factor as
\begin{equation}
a(\theta) = \frac{a_0}{\sqrt{\sinh\alpha\theta}}\left[1 \pm \sqrt{1 -
4\displaystyle{\frac{kC^2}{M^2}}}\sin f(\theta)\right]^{-1/2}.
\end{equation}
Regular bouncing solutions may be obtained for $k = 0$ or $k= - 1$.
Furthermore, differently from the previous situation, the flat case
also does not exhibit any singularity in the string expansion
parameter. It is interesting to note that these models can provide a
quite effective way of enhancing the gravitational
coupling\footnote{To illustrate this point, let us choose $p = 1$
($\omega = - 3$), $f(\theta_{\rm i})=-7\pi/2$ and $f(\theta_{\rm
f})=-3\pi/2$. One then obtains $\phi_{\rm i}\approx 10^{-3}$ and
$\phi_{\rm f}\approx 1$, where the constant $\phi_0$ is chosen
$\phi_0\approx 10^{-2}$ in order to obtain the effective gravitational
``constant'' today equal to Newton constant $\GN$.  With this choice
of parameters, the enhancement of the effective gravitational
``constant'' in the past was therefore of three orders of magnitude.
Note that the dilaton is strictly positive and finite in this
range.}. Investigation of the asymptotic behavior reveals that, for $k
= 0$, the universe displays a radiation dominated behavior in both
extremities of the range ($a\propto |t|^{1/2}$ for $t \rightarrow \pm
\infty$), while for $k = - 1$, the curvature dominates in the
asymptotic regions, leading to a Milne universe.

\subsection{Normal RR case}

Integrating the equations of motion (\ref{scalar'}) after inclusion of
$\xi$, i.e., with a non vanishing $B$ and still for $\omega > -3/2$,
simply turns the solution given by Eq.~(\ref{phiNA}) into
\begin{equation}
\phi(\theta) = \phi_0\left(\sin\alpha\theta - s\right),
\end{equation}
where
\begin{equation}
s = \frac{B^2}{2A^2\phi_0},
\end{equation}
provides the particular solution of the inhomogeneous equation, and we
have assumed the same initial condition for the homogeneous part. The
constant $\alpha$ is defined as in the normal axionic case.

After some straightforward calculations, we get that
Eq.~(\ref{integr}) is modified into
\begin{equation}
\int_{b_0}^{b} {\dd \tilde b \over \tilde b \sqrt{\pm C^2 + M \tilde
b^2 - k \tilde b^4}} = \pm \int_{\theta_0}^\theta {\dd \tilde
\theta\over \phi_0 \left[\sin(\alpha\tilde\theta)-s\right]},
\label{integr2}
\end{equation}
where now
\begin{equation}
C^2 = \frac{1}{6}A^2\phi_0^2 |1 - s^2| \label{Cs}
\end{equation}
takes into account the inhomogeneous part. In Eq.~(\ref{integr2}), the
sign in front of the factor $C^2$ in the denominator of the right-hand
side integrand is positive or negative depending on whether $s^2<1$ or
$s^2<1$ respectively. We shall treat both cases separately.

\subsubsection{Small RR-scalar}

We assume for now on that even though we allow variations for $\xi$,
those are limited in such a way that $s^2 <1$. Eq.~(\ref{integr2}),
being in a form similar to Eq.~(\ref{integr}), yields the same result
that bounces cannot be realized unless $k\leq 0$.

Integrating both sides of Eq.~(\ref{integr2}), we obtain the function
$b$, thanks to which we can write the scale factor as
\begin{equation}
a(\theta) = \frac{a_0}{\sqrt{\sin\alpha\theta -
s}}\left\{\frac{f(\theta)} {\left[ M - f(\theta)\right]^2 + 4C^2k }
\right\}^{1/2},
\end{equation}
with \{see again Ref.~\cite{Grad}, Eq.~(2.251/3)\}
\begin{equation}
f(\theta) = \left| {2\over s}
\left[\frac{s\tan\left(\alpha\theta/2\right) - 1 + \sqrt{1 - s^2}}{1 +
\sqrt{1 -s^2}- s\tan\left(\alpha\theta/2\right)}\right]\right|^p,
\label{fNRR}
\end{equation}
where $a_0$, $p$ and the choice for the relationship between
$f(\theta_0)$ and $g(b_0)$ are the same as in the normal axionic case,
except for the new definition~(\ref{Cs}) of the constant $C$. The
normalization in Eq.~(\ref{fNRR}) has been chosen in such a way that
the limit $s\to 0$ gets indeed back to the normal axionic case
(\ref{fNA}).

The properties of these solutions are the same as in the normal
axionic case, with singularity free models and regular dilatonic
behavior for any $\omega >-3/2$ only when\footnote{In string theory
($\omega=-1$), with $M\sim 10^{-4}$ and $M-2C>0$, one finds that $\phi
> 0$ for any value of $s<1$.}  $k = -1$.

\subsubsection{Large RR-scalar}

In the opposite situation for which $s^2 >1$, one can normalize the
solution in such a way that \{see Ref.~\cite{Grad}, Eq.~(2.551/3)\}
\begin{equation}
f(\theta) = 2 p \arctan \left[
{1-s\tan\left(\alpha\theta/2\right)\over \sqrt{s^2-1} }\right],
\end{equation}
and, provided $a_0^2 = 2 C/(M\phi_0)$, the solution can be written as
\begin{equation}
a(\theta) = \frac{a_0}{\sqrt{s - \sin(\alpha\theta)}}\left[1 \pm
\sqrt{1 - 4\displaystyle{\frac{kC^2}{M^2}}}\sin
f(\theta)\right]^{-1/2} \!\!\!\!\!\!\!\!\!\!.
\end{equation}

Bouncing solutions may be obtained but they are unstable. In fact, to
obtain regular solutions, the dilatonic field must be negative, at
least in a certain range of validity of the solutions. This implies a
repulsive gravity effect. In order to connect such model with the real
Universe, where gravity is attractive and the dilaton must be
positive, the dilatonic field must pass by a zero value, and
instabilities will appear.

\subsection{Anomalous RR case}

Finally, the last situation, for which Eq.~(\ref{scalar'}) is solved
by
\begin{equation}
\phi =\phi_0 \left[\sinh\left(\alpha\theta\right)-s\right],
\end{equation}
is very similar to the anomalous axionic case except that the
hyperbolic sine in Eq.~(\ref{bAA}) is replaced by $[\sinh^2
(\alpha\theta) -s]$, with the same definition for the constant $s$ and
$C$ as in the anomalous axionic situation. This case is essentially
similar to the normal RR one, and we obtain a different function \{see
Ref.~\cite{Grad}, Eq.~(2.441/3)\}
\begin{equation}
f(\theta) = p\ln\left| {2\over s}
\left[\frac{s\tanh\left(\alpha\theta/2\right) + 1 - \sqrt{1 + s^2}}{1
+ \sqrt{1 + s^2}+ s\tanh\left(\alpha\theta/2\right)}\right]\right|,
\end{equation}
where the normalization again ensures that the limit $s\to 0$ is
equivalent to the anomalous axionic case. With the new scale factor
normalization
\begin{equation}
a_0^2 = {A^2 \phi_0\over 3M}(1+s^2),
\end{equation}
the new solution is expressed as
\begin{equation}
a(\theta) = \frac{a_0}{\sqrt{\sinh\alpha\theta - s}}\left[1 \pm
\sqrt{1 - 4\displaystyle{\frac{kC^2}{M^2}}}\sin
f(\theta)\right]^{-1/2} \!\!\!\!\!\!\!\!\!\!.
\end{equation}
Again, as in the anomalous axionic case, completely non singular
solutions, also with respect to the dilatonic field, are obtained for
$k = 0$ or $k= - 1$. The properties of both anomalous (axionic and RR)
cases are very similar, even in the asymptotic regions.  The
significant feature of this case is that, for $k=0$, it is not
difficult to choose the free parameters in order to allow huge
increases of the dilaton along the evolution of such universes.

\section{Conclusions}

We have constructed fully regular cosmological solutions in the
framework of effective actions derived from string theory
principles. These solutions present bouncing behaviors for a wide
range of parameters, and are singularity free; furthermore, the
spacetimes they lead to are geodesically complete, thereby improving
the so-called horizon problem of standard cosmology. Stemming from
string theory, they have a reasonably sound basis as long as the
dilaton is strictly positive and finite in all such solutions. As a
consequence, it is not necessary to go beyond the tree level
approximation in any part of their histories: the analytic solutions
exhibited above can describe the whole history of the cosmological
models they represent.  Their consequences may, in turn, be used as
cosmological tests.

In the normal axionic and Ramond-Ramond cases there are non singular
bouncing solutions even for the string case if $k = - 1$. Remembering
that the radiation fluid included here has also a motivation in the
superstring type IIB action, this reveals to be, to our knowledge, the
first case where a complete regular bouncing cosmological solution is
obtained in the strict string framework ($\omega = -1$), which is
smoothly connected with the standard model radiation dominated
phase. Moreover, the anomalous axionic and RR cases also exhibit
complete non-singular solutions for flat and negative curvature
spatial sections.

As all these models have the interesting feature to approach flat
spacetime in the infinity past (either in Milne coordinates for
$k=-1$, or infinitely large radiation dominated standard model with
$k=0$), there is the possibility to implement a quantum spectrum of
perturbations in the initial asymptotic without any trans-Planckian
problem, and, at the same time, to accomplish a smooth transition to
the standard cosmological model when, after the bounce, a standard
radiation dominated phase is recovered (asymptotically in the $k=0$
case), preserving some of its main achievements like primordial
nucleosynthesis and not involving some exotic matter~\cite{ppnpn2}.

Notice that, in all these cases, the initial value of the dilatonic
field can be made smaller than its final value. Hence, the
gravitational coupling has initially a greater value than it would
have today. This opens the possibility to solve the hierarchical
problem of the gravitational coupling, in a spirit similar to the
so-called brane cosmology~\cite{langlois}.

\acknowledgments We thank CNPq (Brazil) for financial support.

\begin{thebibliography}{90}
\bibitem{polchinski} J.~Polchinski, String theory, Cambridge
University Press (Cambridge, 1998), Vols. I \& II.
\bibitem{green} M.~B.~Green, J.~H.~Schwarz, and E.~Witten, Superstring
theory, Cambridge University Press (Cambridge, 1987), Vols. I \& II.
\bibitem{kiritsis} E.~Kiritsis, {\it Introduction to superstring
theory}, {\tt hep-th/9709062}.
\bibitem{pope} N.~Khviengia, Z.~Khviengia, H.~Lu, and C.~N.~Pope,
Class. Quantum Grav.  {\bf 15}, 759 (1998).
\bibitem{randal} I.~Antoniadis, N.~Arkani-Hamed, S.~Dimopoulos, and
G.~Dvali, Phys. Lett. B {\bf 436}, 257 (1998); L.~Randall and
R.~Sundrum, \prl {\bf 83}, 4690 (1999).
\bibitem{bdp} P.~Bin\'etruy, C.~Deffayet and P.~Peter, Phys. Lett.  B
{\bf 441}, 52 (1998).
\bibitem{ppnpn2} P.~Peter and N.~Pinto-Neto,\prd {\bf 66}, 063509
(2002).
\bibitem{lidsey} J.~E.~Lidsey, D.~Wands, and E.~J.~Copeland,
Phys. Rep.  {\bf 337}, 343 (2000).
\bibitem{vasquez} A.~Feinstein and M.~A.~Vazquez-Mozo, Nucl. Phys.
{\bf B568}, 405 (2000).
\bibitem{picco} C.~P.~Constantinidis, J.~C.~Fabris, R.~G.~Furtado, and
M. Picco,\prd {\bf 61}, 043503 (2000).
\bibitem{kirill} K.~A.~Bronnikov and J.~C.~Fabris, JHEP {\bf 209}, 62
(2002).
\bibitem{branden} R.~Brandenberger, R.~Easther, and J.~Maia, JHEP {\bf
9808}, 007 (1998); D.~A.~Easson, R.~H.~Brandenberger, JHEP {\bf 9909},
003 (1999).
\bibitem{gasperini} M.~Gasperini and G.~Veneziano, {\it The pre-big
bang scenario in string cosmology}, {\tt hep-th/0207130}.
\bibitem{fabris} J.~C.~Fabris, Phys. Lett. B {\bf 267}, 30 (1991).
\bibitem{ppnpn1} P.~Peter and N.~Pinto-Neto, \prd {\bf 65}, 023513
(2001) and references therein.
\bibitem{Grad} I.~S.~Gradshteyn, I.~M.~Ryshik, {\sl Table of
Integrals, Series and Products}, (Academic Press, New-York, 1980).
\bibitem{langlois} D.~Langlois, {\it Brane cosmology: an
introduction}, {\tt hep-th/0209261}, in {\sl Proceedings of the YITP
workshop on "Braneworld - Dynamics of spacetime boundary"}.
\end{thebibliography}

\end{document}

