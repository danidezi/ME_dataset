
%%%%%%%%%%%%%%%%%%%%%%%%%%%%%%%%%%%%%%%%%%%%%%%%%%%%%%
%%%%%%   template.tex for PTPTeX.cls <ver.0.88>  %%%%%
%%%%%%%%%%%%%%%%%%%%%%%%%%%%%%%%%%%%%%%%%%%%%%%%%%%%%%
\documentclass[a4paper,seceq,preprint]{ptptex}


%%%%% Personal Macros %%%%%%%%%%%%%%%%%%%
\providecommand{\dz}{\frac{dz}{2\pi i}}
\providecommand{\norm}[1]{{}^\times_\times{#1}^\times_\times}
%%%%%%%%%%%%%%%%%%%%%%%%%%%%%%%%%%%%%%%%%

\preprintnumber[3cm]{%<-- [..]: optional width of preprint # column.
hep-th/0303039\\YITP-03-8\\ March, 2003}

\markboth{%     %running head for odd-page (authors' name)
Hiroshi Kunitomo
}{%             %running head for even-page (`short' title)
Hybrid Superstrings in NS-NS Plane Waves
}

\title{%        %You can use \\ for explicit line-break
Hybrid Superstrings in NS-NS Plane Waves
}


\author{%       %Use \scshape  for the family name
Hiroshi \textsc{Kunitomo}%
}

\inst{%         %Affiliation, neglected when [addenda] or [errata]
Yukawa Institute for Theoretical Physics\\
Kyoto University, Kyoto 606-8502, Japan
}



\abst{%         %this abstract is neglected when [addenda] or [errata]
By using the hybrid formalism, superstrings in four-dimensional
NS-NS plane waves are studied in a manifest
supersymmetric manner. 
This description of the superstring is obtained by
a field redefinition of the RNS worldsheet fields
and defined as a topological \myHighlight{$N=4$}\coordHE{} string theory.
The Hilbert space consists of two types of representations
describing short and long strings. We study the physical spectrum 
to find boson-fermion asymmetry in the massless spectrum
of the short string. Some massive spectrum of the short string 
and the massless spectrum of the long string are also studied.
}

\usepackage{useful_macros}
\begin{document}

\maketitle

\section{Introduction}\label{intro}

Plane-wave backgrounds are exact string vacua and 
give examples of string theories in non-compact 
curved space-times. These backgrounds are obtained 
by Penrose limit of {\it AdS} spaces and have 
attracted much attention to study the {\it AdS/CFT} 
duality beyond the supergravity approximation.\cite{BMN} 
Despite much progress in this subject, analyses do not
have preserved whole supersymmetry of these 
backgrounds so far. The superstring in R-R plane waves has 
been quantized in the light-cone gauge\cite{M} or using 
a non-covariant formalism.\cite{BM} Only a part of 
supersymmetry are linearly realized in these formulations.
For superstrings in NS-NS plane waves, on the other hand,
we can use the Ramond-Neveu-Schwarz (RNS)
formalism for covariant quantization
in which, however, the space-time
supersymmetry is not manifest.
This is not always necessary but desirable to make 
transparent any cancellations coming from the supersymmetry. 

In this paper, we study the four-dimensional superstring 
in NS-NS plane waves in terms of the hybrid formalism
which has been developed in Refs.~\citen{B} and \citen{BV}, 
and applied to several compactified string theories.\cite{compact}
This description of the superstring is obtained by 
a field redefinition of the RNS worldsheet fields and 
manifestly preserve all isometries of the background 
including supersymmetry.

Strings in NS-NS plane waves are described
by the Nappi-Witten model,\cite{NW} which is the WZW model 
on the group manifold \myHighlight{$H_4$}\coordHE{} generated by the four-dimensional 
Heisenberg algebra:
\begin{align}\coord{}\boxAlignEqnarray{
  \leftCoord{}\left[{\cal J},{\cal P}\right]&={\cal P},\qquad
  \leftCoord{}\left[{\cal J},{\cal P}^*\right]=-{\cal P}^*,\nonumber\rightCoord{}\\
  \leftCoord{}\left[{\cal P},{\cal P}^*\right]&={\cal F}.\label{glbnw}
}{0mm}{3}{2}{
  \left[{\cal J},{\cal P}\right]&={\cal P},\qquad
  \left[{\cal J},{\cal P}^*\right]=-{\cal P}^*,\nonumber\\
  \left[{\cal P},{\cal P}^*\right]&={\cal F}.}{0}\coordE{}\end{align}
The Hilbert space of the NW model consists of two 
distinct representations, discrete (type II and III)
and continuous (type I). The model has a spectral 
flow symmetry and all flowed representations must be 
also included.\cite{KK,KP} 
The spectrally flowed discrete representations are 
viewed as describing short strings localized 
in the transverse space to the plane wave.
The flowed continuous representations define
long string states propagating 
in the whole four-dimensional space.
These structure is similar to the spectrum of 
the string in \myHighlight{$AdS_3$}\coordHE{},\cite{MO} 
which in fact obtained by its Penrose limit.\cite{HS2}

The RNS superstring in this background is expressed
by superconformal field theories of the type
\myHighlight{${H_4\times{\cal M}}$}\coordHE{}, where \myHighlight{$H_4$}\coordHE{} denotes the super NW
model and \myHighlight{${\cal M}$}\coordHE{} is represented by an arbitrary \myHighlight{$N=2$}\coordHE{} 
unitary superconformal field theory 
with \myHighlight{$c=9$}\coordHE{}.\cite{HS2}
This \myHighlight{${\cal M}$}\coordHE{} sector represents a Calabi-Yau compactification 
if we project into the integral sector of the \myHighlight{$U(1)_R$}\coordHE{}
charge \myHighlight{$I_{{\cal M},0}$}\coordHE{}.
However, it was found in \citen{HS2} that 
the string vacua have enhanced supersymmetry if we take 
a generalized GSO projection which restricts the {\it total} 
\myHighlight{$U(1)_R$}\coordHE{} charge to be integer, while fractional \myHighlight{$I_{{\cal M},0}$}\coordHE{}
is allowed. We adopt this weak GSO projection throughout. 

The hybrid superstrings in NS-NS plane waves are
related to the RNS suprestrings by a field redefinition of
worldsheet fields. We show how to perform
a redefinition from RNS fields to hybrid fields 
making all the space-time supersymmetry manifest. 
The model can be formulated as a \myHighlight{$N=4$}\coordHE{} topological string theory.\cite{BV} 

Then we examine the physical spectrum at several lower mass levels.
We find that the massless spectrum of the short string
has boson-fermion asymmetry, which is allowed without breaking
supersymmetry. There are two massless bosons without fermionic
partners, which can be clarified due to the manifest 
supersymmetry. Some massive spectrum of the short string 
and the massless spectrum of the long string 
are also studied in a manifest supersymmetric manner.

This paper is organized as follows. In section \ref{nsr} 
we begin with a brief review of the super-NW model 
in the RNS formalism. The space-time supercharges 
are given in the form satisfying the conventional 
supersymmetry algebra.
Hybrid worldsheet fields are introduced
in section \ref{hyb} by a redefinition of RNS worldsheet 
fields. The model is reformulated as a topological \myHighlight{$N=4$}\coordHE{} 
string theory.
In section \ref{string}, we construct the Hilbert space 
of the hybrid superstring by using hybrid fields, 
which consists of two sectors representing short and long 
strings respectively. The physical spectrum is studied 
in section \ref{physspec}. It is found that the massless 
spectrum of the short string has boson-fermion asymmetry. 
The results are summarized with some discussions 
in section \ref{summary}.

\section{RNS superstrings in the NS-NS plane waves}\label{nsr}

The RNS superstrings propagating in the four-dimensional
NS-NS plane waves is provided by superconformal 
field theories of the type \myHighlight{$H_4\times{\cal M}$}\coordHE{}.\cite{HS1,HS2} 
Here \myHighlight{$H_4$}\coordHE{} denotes the super-NW model described by 
the super WZW model on the four-dimensional Heisenberg group. 
The Hilbert space of this model is constructed by representations of 
the \myHighlight{$H_4$}\coordHE{} super current algebra\footnote{
We refer only to the holomorphic sector in this paper.
It can be easily combined with the anti-holomorphic sector
if necessary.\cite{HS2,KK,KP,MO}}
\begin{alignat}{2}\label{h4sca}
 J(z)P(w)&\sim\frac{P(w)}{z-w},&\qquad
 J(z)P^*(w)&\sim-\frac{P^*(w)}{z-w},\nonumber\\
%%%%%%%%%%%%%%%%%%%%%%%%%%%%%%%%
 P(z)P^*(w)&\sim\frac{1}{(z-w)^2}+\frac{F(w)}{z-w},&\qquad
 J(z)F(w)&\sim\frac{1}{(z-w)^2}, \nonumber\\
  \psi_P(z)\psi_{P^*}(w)&\sim\frac{1}{z-w},&\qquad
 \psi_J(z)\psi_F(w)&\sim\frac{1}{z-w},\nonumber\\
%%%%%%%%%%%%%%%%%%%%%%%%%%%%%%%%
J(z)\psi_P(w)&\sim\psi_J(z)P(w)\sim 
\frac{\psi_P(w)}{z-w},&\qquad
J(z)\psi_{P^*}(w)&\sim\psi_J(z)P^*(w)\sim
-\frac{\psi_{P^*}(w)}{z-w},\nonumber\\
%%%%%%%%%%%%%%%%%%%%%%%%%%%%%%%%
P(z)\psi_{P^*}(w)&\sim\psi_P(z)P^*(w)\sim
\frac{\psi_F(w)}{z-w}.& &
\end{alignat}
Representations of this algebra are easily obtained by using
a free-field realization\cite{KK}
\begin{alignat}{2}
 J&=i\partial X^-,&\qquad F&=i\partial X^+,\nonumber\\
%%%%%%%%%%%%%%%%%%%%%%%%%%%%%%%%
 P&=e^{iX^+}(i\partial Z+\psi^+\psi),&\qquad
 P^*&=e^{-iX^+}(i\partial Z^*-\psi^+\psi^*),\nonumber\\
%%%%%%%%%%%%%%%%%%%%%%%%%%%%%%%%
\psi_F&=\psi^+\qquad
\psi_J=\psi^-,&\qquad
\psi_P&=e^{iX^+}\psi,\qquad
\psi_{P^*}=e^{-iX^+}\psi^*,
\end{alignat}
where operator products of free fields are defined by
\begin{align}\coord{}\boxAlignEqnarray{
 \leftCoord{}X^+(z)X^-(w)&\sim Z(z)Z^*(w)\sim -\log(z-w),\nonumber\rightCoord{}\\
\leftCoord{}\psi^+(z)\psi^-(w)&\sim\psi(z)\psi^*(w)\sim\frac{\leftCoord{}1}{\rightCoord{}z-w}.
}{0mm}{3}{3}{
 X^+(z)X^-(w)&\sim Z(z)Z^*(w)\sim -\log(z-w),\nonumber\\
\psi^+(z)\psi^-(w)&\sim\psi(z)\psi^*(w)\sim\frac{1}{z-w}.
}{0}\coordE{}\end{align}
The zero modes of bosonic currents provide generators of 
the space-time symmetry (\ref{glbnw}):
 \begin{alignat}{2}
 {\cal J}&=\oint\dz i\partial X^-,&\qquad
 {\cal F}&=\oint\dz i\partial X^+,\nonumber\\
%%%%%%%%%%%%%%%%%%%%%%%%%%%%%%%%%%%%%%%%%%%%%%%%%%%%%%%%%
 {\cal P}&=\oint\dz e^{iX^+}(i\partial Z+\psi^+\psi),&\qquad
 {\cal P}^*&=\oint\dz e^{-iX^+}(i\partial Z^*-\psi^+\psi^*).
\label{glbnwgen}
 \end{alignat}

The \myHighlight{$N=1$}\coordHE{} worldsheet superconformal symmetry is actually
enhanced to \myHighlight{$N=2$}\coordHE{} generated by
\begin{align}\coord{}\boxAlignEqnarray{
 \leftCoord{}T_{\rightCoord{}H_4}&=-\partial X^+\partial X^--\partial Z\partial Z^*
 \leftCoord{}-\frac{\leftCoord{}1}{\rightCoord{}2}\psi^+\partial \psi^-
 \leftCoord{}-\frac{\leftCoord{}1}{\rightCoord{}2}\psi^-\partial\psi^+
 \leftCoord{}-\frac{\leftCoord{}1}{\rightCoord{}2}\psi\partial\psi^*
 \leftCoord{}-\frac{\leftCoord{}1}{\rightCoord{}2}\psi^*\partial\psi,\nonumber\rightCoord{}\\
%%%%%%%%%%%%%%%%%%%%%%%
 \leftCoord{}G^+_{\rightCoord{}H_4}&=\psi^+i\partial X^-+\psi i\partial Z^*,\qquad
\leftCoord{}G^-_{\rightCoord{}H_4}=\psi^-i\partial X^++\psi^*i\partial Z,\nonumber\rightCoord{}\\
%%%%%%%%%%%%%%%%%%%%%%%
 \leftCoord{}I_{\rightCoord{}H_4}&=\psi^+\psi^-+\psi\psi^*.
}{0mm}{1}{2}{theequation}{0}\coordE{}\end{align}
The model has the central charge \myHighlight{$c=6$}\coordHE{}, which is the same with 
the superstring in the flat four-dimensional space-time.

The \myHighlight{${\cal M}$}\coordHE{} sector is represented by 
an arbitrary unitary representation of \myHighlight{$N=2$}\coordHE{} rational 
superconformal field theory with \myHighlight{$c=9$}\coordHE{}. We denote generators of 
this \myHighlight{$N=2$}\coordHE{} superconformal symmetry by 
\myHighlight{$(T_{\cal M},G^\pm_{\cal M},I_{\cal M})$}\coordHE{}. 

In order to covariantly quantize the RNS superstring,
fermionic ghosts \myHighlight{$(b,c)$}\coordHE{} and bosonic ghosts \myHighlight{$(\beta,\gamma)$}\coordHE{}
must be introduced. These superconformal ghosts satisfy 
\begin{equation}\coord{}\boxEquation{
 c(z)b(z)\sim\gamma(z)\beta(w)\ \sim\ \frac{1}{z-w},
}{
 c(z)b(z)\sim\gamma(z)\beta(w)\ \sim\ \frac{1}{z-w},
}{ecuacion}\coordE{}\end{equation}
and have \myHighlight{$N=1$}\coordHE{} superconformal invariance generated by
\begin{align}\coord{}\boxAlignEqnarray{
 \leftCoord{}T_{\rightCoord{}gh}&=-2b\partial c-\partial bc
\leftCoord{}-\frac{\leftCoord{}3}{\rightCoord{}2}\beta\partial\gamma-\frac{\leftCoord{}1}{\rightCoord{}2}\partial\beta\gamma,
\leftCoord{}\nonumber\rightCoord{}\\
%%%%%%%%%%%%%%%%%%%%%%%%%
 \leftCoord{}G_{\rightCoord{}gh}&=\frac{\leftCoord{}3}{\rightCoord{}2}\beta\partial c+\partial\beta c-2b\gamma.
}{0mm}{2}{3}{theequation}{0}\coordE{}\end{align}
The physical Hilbert space is defined by the cohomology 
\myHighlight{${\cal H}_{phys}={\rm Ker}Q_{BRST}/{\rm Im}Q_{BRST}$}\coordHE{}
of the BRST charge
\begin{eqnarray}\coord{}\boxAlignEqnarray{\leftCoord{}
 Q_{BRST}&=&\oint\dz\left(c\left(T_m+\frac{\leftCoord{}1}{\rightCoord{}2}T_{gh}\right)
\leftCoord{}+\gamma\left(G_m+\frac{\leftCoord{}1}{\rightCoord{}2}G_{gh}\right)\right),
\rightCoord{}}{0mm}{4}{4}{
 Q_{BRST}&=&\oint\dz\left(c\left(T_m+\frac{1}{2}T_{gh}\right)
+\gamma\left(G_m+\frac{1}{2}G_{gh}\right)\right),
}{1}\coordE{}\end{eqnarray}
where
\begin{eqnarray}\coord{}\boxAlignEqnarray{\leftCoord{}
 T_m&=&T_{H_4}+T_{\cal M},\qquad 
G_m\ =\ G^+_{H_4}+G^-_{H_4}+G^+_{\cal M}+G^-_{\cal M}.\rightCoord{}
\rightCoord{}}{0mm}{1}{3}{
 T_m&=&T_{H_4}+T_{\cal M},\qquad 
G_m\ =\ G^+_{H_4}+G^-_{H_4}+G^+_{\cal M}+G^-_{\cal M}.
}{1}\coordE{}\end{eqnarray}

Then we bosonize the worldsheet fermions and the \myHighlight{$U(1)$}\coordHE{}
current in the \myHighlight{${\cal M}$}\coordHE{} sector as
\begin{subequations}\label{bp}
\begin{align}\coord{}\boxAlignEqnarray{
 \leftCoord{}\psi^+\psi^-&=i\partial H_0,\qquad
 \leftCoord{}\psi\psi^*=i\partial H_1,\rightCoord{}\\
 \leftCoord{}I_{\rightCoord{}\cal M}&=-\sqrt{3}i\partial H_2,\label{h2}
}{0mm}{3}{3}{
 \psi^+\psi^-&=i\partial H_0,\qquad
 \psi\psi^*=i\partial H_1,\\
 I_{\cal M}&=-\sqrt{3}i\partial H_2,}{0}\coordE{}\end{align}
\end{subequations}
where bosons \myHighlight{$H_I(z)\ (I=0,1,2)$}\coordHE{} satisfy
the standard OPE's
\begin{equation}\coord{}\boxEquation{
 H_I(z)H_J(w)\sim -\delta_{IJ}\log(z-w).
}{
 H_I(z)H_J(w)\sim -\delta_{IJ}\log(z-w).
}{ecuacion}\coordE{}\end{equation}
The superconformal ghosts are also bosonized by\cite{FMS}
\begin{align}\coord{}\boxAlignEqnarray{
 \leftCoord{}c&=e^\sigma,\qquad b=e^{-\sigma},\nonumber\rightCoord{}\\
%%%%%%%%%%%%%%%%%%%%%%%%%%%
 \leftCoord{}\gamma&=\eta e^\phi=e^{\phi-\chi},\nonumber\rightCoord{}\\
%%%%%%%%%%%%%%%%%%%%%%%%%%%
 \leftCoord{}\beta&=e^{-\phi}\partial\xi=\partial\chi e^{-\phi+\chi},
\label{bg}
}{0mm}{1}{1}{theequation}{0}\coordE{}\end{align}
with
\begin{align}\coord{}\boxAlignEqnarray{
 \leftCoord{}\phi(z)\phi(w)&\sim-\log(z-w),\nonumber\rightCoord{}\\
 \leftCoord{}\sigma(z)\sigma(w)&\sim\chi(z)\chi(w)\sim+\log(z-w).
}{0mm}{2}{2}{
 \phi(z)\phi(w)&\sim-\log(z-w),\nonumber\\
 \sigma(z)\sigma(w)&\sim\chi(z)\chi(w)\sim+\log(z-w).
}{0}\coordE{}\end{align}
Here it is important that since the zero-mode \myHighlight{$\xi_0$}\coordHE{} is
not included in these formulas, the Hilbert space of 
the original bosonic ghosts \myHighlight{$(\beta,\gamma)$}\coordHE{} is different 
from the one of the bosonized fields \myHighlight{$(\phi,\xi,\eta)$}\coordHE{} 
(or \myHighlight{$(\phi,\chi)$}\coordHE{}). The former (latter) is called small 
(large) Hilbert space \myHighlight{${\cal H}_{small}$}\coordHE{} 
(\myHighlight{${\cal H}_{large}$}\coordHE{}). This extension of 
the Hilbert space is essential to realize
the supersymmetry.

In order to obtain supersymmetric spectrum
in the RNS formalism, we must impose GSO projection which
guarantees the mutual locality of space-time supercharges.
If we take a weak GSO condition
\begin{eqnarray}\coord{}\boxAlignEqnarray{\leftCoord{}
I_0&=&I_{H_4,0}+I_{{\cal M},0} \in \mathbb Z,\rightCoord{}\label{gso}
\rightCoord{}}{0mm}{1}{3}{
I_0&=&I_{H_4,0}+I_{{\cal M},0} \in \mathbb Z,}{1}\coordE{}\end{eqnarray}
the model has the enhanced supersymmetry 
generated by four supercharges\cite{HS2}
\begin{eqnarray}\coord{}\boxAlignEqnarray{\leftCoord{}
 {\rightCoord{}\leftCoord{}\cal Q}^{\pm\pm}_{-\frac{\leftCoord{}1}{\rightCoord{}2}}&=&\oint\dz
e^{-\frac{\leftCoord{}\phi}{\rightCoord{}2}} \rightCoord{}
e^{\pm iX^+}e^{\frac{\leftCoord{}i}{\rightCoord{}2}(H_0\pm(H_1+\sqrt{3}H_2))},\nonumber\rightCoord{}\\\leftCoord{}
 {\rightCoord{}\leftCoord{}\cal Q}^{\pm\mp}_{-\frac{\leftCoord{}1}{\rightCoord{}2}}&=&\oint\dz
e^{-\frac{\leftCoord{}\phi}{\rightCoord{}2}}e^{\frac{\leftCoord{}i}{\rightCoord{}2}(-H_0\pm(H_1-\sqrt{3}H_2))}.\rightCoord{}
\label{susy}
\rightCoord{}}{0mm}{10}{13}{
 {\cal Q}^{\pm\pm}_{-\frac{1}{2}}&=&\oint\dz
e^{-\frac{\phi}{2}} 
e^{\pm iX^+}e^{\frac{i}{2}(H_0\pm(H_1+\sqrt{3}H_2))},\\
 {\cal Q}^{\pm\mp}_{-\frac{1}{2}}&=&\oint\dz
e^{-\frac{\phi}{2}}e^{\frac{i}{2}(-H_0\pm(H_1-\sqrt{3}H_2))}.
}{1}\coordE{}\end{eqnarray}
The subscript \myHighlight{$-\frac{1}{2}$}\coordHE{} indicates that these operators
are given in the \myHighlight{$-\frac{1}{2}$}\coordHE{} picture, which is generally
read from the eigenvalue of the operator
\begin{equation}\coord{}\boxEquation{
 {\cal R}=\oint\dz(\xi-\partial\phi).\label{picture}
}{
 {\cal R}=\oint\dz(\xi-\partial\phi).}{ecuacion}\coordE{}\end{equation}

We note that these supercharges obey a peculiar algebra
due to the infinite degeneracy of pictures. This algebra is 
equivalent to the supersymmetry only in the on-shell physical 
amplitudes. We change the picture of the half of the 
supercharges 
\myHighlight{$({\cal Q}^{--}_{-\frac{1}{2}},{\cal Q}^{+-}_{-\frac{1}{2}})$}\coordHE{} to 
\myHighlight{$+\frac{1}{2}$}\coordHE{} so that the conventional supersymmetry algebra
may hold without any condition:\footnote{
Rigorously speaking, the relative signs between terms in the explicit 
forms (\ref{susyp}) are not fixed without specifying the cocycle factors 
usually omitted. This fact makes practical calculations difficult
although it is often fixed by Lorentz covariance of the result. This
complexity disappears in the hybrid formalism, which is
actually one of the important advantages of the hybrid formalism.}
\begin{align}\coord{}\boxAlignEqnarray{
 {\rightCoord{}\leftCoord{}\cal Q}^{--}_{\rightCoord{}\frac{\leftCoord{}1}{\rightCoord{}2}}&=\oint\dz\left\{Q_{\rightCoord{}BRST},\xi
\leftCoord{}e^{-\frac{\leftCoord{}\phi}{\rightCoord{}2}}e^{-iX^+}e^{\frac{\leftCoord{}i}{\rightCoord{}2}(H_0-(H_1+\sqrt{3}H_2))}\right\},
\leftCoord{}\nonumber\rightCoord{}\\
&\leftCoord{}=\oint\dz e^{-iX^+}\Bigg(
\leftCoord{}\eta be^{\frac{\leftCoord{}3}{\rightCoord{}2}\phi+\frac{\leftCoord{}i}{\rightCoord{}2}(H_0-(H_1+\sqrt{3}H_2))}
\leftCoord{}\nonumber\rightCoord{}\\
&\leftCoord{}\hspace{2.5cm}
\leftCoord{}+i\partial X^+e^{\frac{\leftCoord{}\phi}{\rightCoord{}2}-\frac{\leftCoord{}i}{\rightCoord{}2}(H_0+(H_1+\sqrt{3}H_2))}
\leftCoord{}+i\partial Z^*e^{\frac{\leftCoord{}\phi}{\rightCoord{}2}+\frac{\leftCoord{}i}{\rightCoord{}2}(-H_0-(H_1-\sqrt{3}H_2))}
\leftCoord{}\nonumber\rightCoord{}\\
&\leftCoord{}\hspace{4cm}
\leftCoord{}-\psi^+e^{\frac{\leftCoord{}\phi}{\rightCoord{}2}+\frac{\leftCoord{}i}{\rightCoord{}2}(H_0-(H_1+\sqrt{3}H_2))}
\leftCoord{}-G^-_{\rightCoord{}\cal M}e^{\frac{\leftCoord{}\phi}{\rightCoord{}2}+\frac{\leftCoord{}i}{\rightCoord{}2}(H_0-(H_1+\sqrt{3}H_2))}
\leftCoord{}\Bigg),\nonumber\rightCoord{}\\
%%%%%%%%%%%%%%%%%%%%%%%%%%%%%%%%%
 {\rightCoord{}\leftCoord{}\cal Q}^{+-}_{\rightCoord{}\frac{\leftCoord{}1}{\rightCoord{}2}}&=\oint\dz\left\{Q_{\rightCoord{}BRST},\xi 
\leftCoord{}e^{-\frac{\leftCoord{}\phi}{\rightCoord{}2}}e^{\frac{\leftCoord{}i}{\rightCoord{}2}(-H_0+(H_1-\sqrt{3}H_2))}\right\},
\leftCoord{}\nonumber\rightCoord{}\\
&\leftCoord{}=\oint\dz\Bigg(
\leftCoord{}\eta be^{\frac{\leftCoord{}3}{\rightCoord{}2}\phi+\frac{\leftCoord{}i}{\rightCoord{}2}(-H_0+(H_1-\sqrt{3}H_2))}
\leftCoord{}-i\partial X^-e^{\frac{\leftCoord{}\phi}{\rightCoord{}2}-\frac{\leftCoord{}i}{\rightCoord{}2}(-H_0-(H_1-\sqrt{3}H_2))}
\leftCoord{}\nonumber\rightCoord{}\\
&\leftCoord{}\hspace{4cm}
\leftCoord{}+i\partial Ze^{\frac{\leftCoord{}\phi}{\rightCoord{}2}-\frac{\leftCoord{}i}{\rightCoord{}2}(H_0+(H_1+\sqrt{3}H_2))}
\leftCoord{}-G^-_{\rightCoord{}\cal M}e^{\frac{\leftCoord{}\phi}{\rightCoord{}2}+\frac{\leftCoord{}i}{\rightCoord{}2}(-H_0+(H_1+\sqrt{3}H_2))}
\leftCoord{}\Bigg).\label{susyp}
}{0mm}{22}{18}{theequation}{0}\coordE{}\end{align}
Since we will not refer to supercharges in the other pictures,
we simply denote \myHighlight{$({\cal Q}^{\pm+}_{-\frac{1}{2}},
{\cal Q}^{\pm-}_{\frac{1}{2}})=({\cal Q}^{\pm+},{\cal Q}^{\pm-})$}\coordHE{} 
for the rest of this paper. These picture changed supercharges
together with (\ref{glbnwgen}) generate
the supersymmetry algebra in the NS-NS plane waves\cite{HS2}
\begin{alignat}{2}
\left[{\cal J},{\cal P}\right]&={\cal P},&\qquad
\left[{\cal J},{\cal P}^*\right]&=-{\cal P}^*,\nonumber\\
%%%%%%%%%%%%%%%%%%%%%%
\left[{\cal P},{\cal P}^*\right]&={\cal F},& &\nonumber\\
%%%%%%%%%%%%%%%%%%%%%%
\left[{\cal J},{\cal Q}^{\pm\pm}\right]&=\pm{\cal Q}^{\pm\pm},
&\qquad
\left[{\cal J},{\cal Q}^{\pm\mp}\right]&=0,\nonumber\\
%%%%%%%%%%%%%%%%%%%%%%%
\left[{\cal Q}^{-+},{\cal P}\right]&=-{\cal Q}^{++},&\qquad
\left[{\cal Q}^{+-},{\cal P}^*\right]&={\cal Q}^{--},\nonumber\\
%%%%%%%%%%%%%%%%%%%%%%%
 \left\{{\cal Q}^{++},{\cal Q}^{--}\right\}&={\cal F},&\qquad
 \left\{{\cal Q}^{-+},{\cal Q}^{+-}\right\}&=-{\cal J},\nonumber\\
%%%%%%%%%%%%%%%%%%%%%%%
 \left\{{\cal Q}^{++},{\cal Q}^{+-}\right\}&={\cal P},&\qquad
 \left\{{\cal Q}^{-+},{\cal Q}^{--}\right\}&={\cal P}^*.\label{alg}
\end{alignat}

Before closing this section, it is useful to reconsider physical state
conditions in the RNS formalism. 
Although the physical states is defined by the BRST cohomology
in \myHighlight{${\cal H}_{small}$}\coordHE{}, we must extend it to \myHighlight{${\cal H}_{large}$}\coordHE{} to perform
a field redefinition to hybrid fields since, as mentioned above, 
\myHighlight{${\cal H}_{small}$}\coordHE{} is not enough to realize the space-time
supersymmetry.
Therefore, we must generalize the physical state conditions to
\begin{subequations}\label{pscorg}
\begin{eqnarray}\coord{}\boxAlignEqnarray{\leftCoord{}
 Q_{BRST}|\psi\rangle&=&0,\nonumber\rightCoord{}\\\leftCoord{}
\leftCoord{}|\psi\rangle&\sim&|\psi\rangle+\delta|\psi\rangle,\qquad
\delta|\psi\rangle\ =\ Q_{BRST}|\Lambda\rangle,\nonumber\rightCoord{}\\\leftCoord{}
\eta_0|\psi\rangle&=&\eta_0|\Lambda\rangle\ =\ 0,\rightCoord{}\label{psc}
\rightCoord{}}{0mm}{4}{5}{
 Q_{BRST}|\psi\rangle&=&0,\\
|\psi\rangle&\sim&|\psi\rangle+\delta|\psi\rangle,\qquad
\delta|\psi\rangle\ =\ Q_{BRST}|\Lambda\rangle,\\
\eta_0|\psi\rangle&=&\eta_0|\Lambda\rangle\ =\ 0,}{1}\coordE{}\end{eqnarray}
where \myHighlight{$|\psi\rangle, |\Lambda\rangle\in{\cal H}_{large}$}\coordHE{}.
In addition to these cohomology conditions, 
we require that the physical states have ghost number one
\begin{equation}\coord{}\boxEquation{
  Q_{gh}|\psi\rangle=|\psi\rangle,\label{gncond}
}{
  Q_{gh}|\psi\rangle=|\psi\rangle,}{ecuacion}\coordE{}\end{equation}
\end{subequations}
counted by the charge\footnote{
This definition of the ghost number is related to 
the familiar one \myHighlight{$N_c=\oint\dz(cb-\gamma\beta)$}\coordHE{} 
by \myHighlight{$Q_{gh}=N_c-{\cal R}$}\coordHE{}. The difference is a constant in a fixed picture.}
\begin{equation}\coord{}\boxEquation{
 Q_{gh}=\oint\dz\left(cb-\xi\eta\right).
}{
 Q_{gh}=\oint\dz\left(cb-\xi\eta\right).
}{ecuacion}\coordE{}\end{equation}
These conditions (\ref{pscorg}) have following natural 
interpretation as a topological \myHighlight{$N=4$}\coordHE{} string theory.

Let us note that there is the hidden twisted \myHighlight{$N=4$}\coordHE{} superconformal
symmetry generated by
\begin{align}\coord{}\boxAlignEqnarray{
 \leftCoord{}T&=T_m+T_{\rightCoord{}gh},\nonumber\rightCoord{}\\
%%%%%%%%%%%%%%%%%%%%%%%%%%%%%%
 \leftCoord{}G^+&=J_{\rightCoord{}BRST},\nonumber\rightCoord{}\\
%%%%%%%%%%%%%%%%%%%%%%%%%%%%%%
&\leftCoord{}= c\left(T_m+T_{\rightCoord{}\beta\gamma}\right)
\leftCoord{}+\gamma G_m-\gamma^{\leftCoord{}2b}+c\partial cb+\partial(c\xi\eta)+\partial^{\leftCoord{}2c},
\leftCoord{}\nonumber\rightCoord{}\\
%%%%%%%%%%%%%%%%%%%%%%%%%%%%%%
\leftCoord{}G^-&=b,\qquad
 \leftCoord{}\widetilde G^+\ =\ \eta,\qquad
 \leftCoord{}\widetilde G^-\ =\ \xi T-b\left\{Q_{\rightCoord{}BRST},\xi\right\}+\partial^{\leftCoord{}2}\xi,
\leftCoord{}\nonumber\rightCoord{}\\
%%%%%%%%%%%%%%%%%%%%%%%%%%%%%%
 \leftCoord{}I^{++}&=\eta c,\qquad
 \leftCoord{}I^{--}\ =\ b\xi,\qquad
 \leftCoord{}I\ =\ cb-\xi\eta,\label{topn4}
}{0mm}{3}{1}{theequation}{0}\coordE{}\end{align} 
in \myHighlight{${\cal H}_{large}$}\coordHE{}. Physical state conditions (\ref{psc}) 
and ghost number condition (\ref{gncond}) can be written 
in terms of these \myHighlight{$N=4$}\coordHE{} generators as
\begin{subequations}\label{psc1}
\begin{align}\coord{}\boxAlignEqnarray{
 \leftCoord{}G^+_0|\psi\rangle&=0,\qquad 
\leftCoord{}\delta|\psi\rangle=G^+_0|\Lambda\rangle,\rightCoord{}\\
%%%%%%%%%%%%%%%%%%%%%%%%
\leftCoord{}I_0|\psi\rangle&=|\psi\rangle,\rightCoord{}\\
%%%%%%%%%%%%%%%%%%%%%%%%
\leftCoord{}\widetilde G^+_0|\psi\rangle&=\widetilde G^+_0|\Lambda\rangle=0,
\label{eta}
}{0mm}{1}{1}{theequation}{0}\coordE{}\end{align}
\end{subequations}
which can be naturally interpreted as physical state conditions
in the topological \myHighlight{$N=4$}\coordHE{} string theory.\cite{BV}
Since \myHighlight{$\eta_0$}\coordHE{}-cohomology is trivial we can always
solve Eqs.~(\ref{eta}) by \myHighlight{$|\psi\rangle=\widetilde G^+_0|V\rangle$}\coordHE{} and
\myHighlight{$|\Lambda\rangle=\widetilde G^+_0|\Lambda^-\rangle$}\coordHE{} 
to rewrite (\ref{psc1}) in the more symmetric forms:
\begin{subequations}\label{phys}
\begin{align}\coord{}\boxAlignEqnarray{
 \leftCoord{}G^+_0\widetilde G^+_0|V\rangle&=0,\label{eom}\rightCoord{}\\
%%%%%%%%%%%%%%%%%%%%%%%%%%%%%%%%%
\leftCoord{}\delta|V\rangle&=G^+_0|\Lambda^-\rangle
\leftCoord{}+ \widetilde G^+_0|\widetilde\Lambda^-\rangle,\label{gauge}\rightCoord{}\\
%%%%%%%%%%%%%%%%%%%%%%%%%%%%%%%%%
\leftCoord{}I_0|V\rangle&=0.\label{u1}
}{0mm}{1}{1}{theequation}{0}\coordE{}\end{align}
\end{subequations}
In this paper, we call the first condition (\ref{eom}) 
the equation of motion and the second (\ref{gauge})
the gauge transformation following terminologies in 
string field theory.
These conditions will be used to study physical spectrum 
in section \ref{physspec}.

\section{Hybrid superstrings in the NS-NS plane waves}\label{hyb}

In this section, we develop the hybrid formalism 
for superstrings in the NS-NS plane waves.
We first introduce hybrid fields by finding
a field redefinition from RNS fields, which allows 
whole space-time supersymmetry to be manifest.
We rewrite worldsheet superconformal generators using
these new fields to formulate the model as a topological
\myHighlight{$N=4$}\coordHE{} string theory.

As explained in the previous section,
the basic fields of the super NW model are free fields 
\myHighlight{$(X^\pm,Z,Z^*,\psi^\pm,\psi,\psi^*)$}\coordHE{} and superconformal 
ghosts \myHighlight{$(b,c,\beta,\gamma)$}\coordHE{}. We also add them the boson \myHighlight{$H_2$}\coordHE{} 
come from the \myHighlight{$U(1)$}\coordHE{} current (\ref{h2}) in the \myHighlight{${\cal M}$}\coordHE{} sector,
which we need to define supercharges (\ref{susy}).
Although bosonic fields \myHighlight{$(X^\pm,Z,Z^*)$}\coordHE{} are common to 
the hybrid formalism,\footnote{
These bosons are not exactly the same in two formalisms
but related by the similarity transformation (\ref{similar}).} 
remaining fields must be rearranged to obtain 
basic fields in the hybrid formalism.
Describing these fields in terms of six free bosons 
\myHighlight{$(H_0, H_1, H_2, \phi, \chi, \sigma)$}\coordHE{} with the help of 
bosonization formulas (\ref{bp}) and (\ref{bg}), 
we perform a linear transformation
\begin{align}\coord{}\boxAlignEqnarray{
 \leftCoord{}\phi_{\rightCoord{}--}&=-\frac{\leftCoord{}i}{\rightCoord{}2}H_0-\frac{\leftCoord{}i}{\rightCoord{}2}H_1
\leftCoord{}-\frac{\leftCoord{}i}{\rightCoord{}2}\sqrt{3}H_2+\frac{\leftCoord{}1}{\rightCoord{}2}\phi,
\leftCoord{}\nonumber\rightCoord{}\\
%%%%%%%%%%%%%%%%%%%%
 \leftCoord{}\phi_{\rightCoord{}+-}&=\frac{\leftCoord{}i}{\rightCoord{}2}H_0+\frac{\leftCoord{}i}{\rightCoord{}2}H_1
\leftCoord{}-\frac{\leftCoord{}i}{\rightCoord{}2}\sqrt{3}H_2+\frac{\leftCoord{}1}{\rightCoord{}2}\phi,
\leftCoord{}\nonumber\rightCoord{}\\
%%%%%%%%%%%%%%%%%%%%
 \leftCoord{}\phi_{\rightCoord{}++}&=-\frac{\leftCoord{}i}{\rightCoord{}2}H_0+\frac{\leftCoord{}i}{\rightCoord{}2}H_1
\leftCoord{}+\frac{\leftCoord{}i}{\rightCoord{}2}\sqrt{3}H_2-\frac{\leftCoord{}3}{\rightCoord{}2}\phi+\chi+\sigma,
\leftCoord{}\nonumber\rightCoord{}\\
%%%%%%%%%%%%%%%%%%%%
 \leftCoord{}\phi_{\rightCoord{}-+}&=\frac{\leftCoord{}i}{\rightCoord{}2}H_0-\frac{\leftCoord{}i}{\rightCoord{}2}H_1
\leftCoord{}+\frac{\leftCoord{}i}{\rightCoord{}2}\sqrt{3}H_2-\frac{\leftCoord{}3}{\rightCoord{}2}\phi+\chi+\sigma,
\leftCoord{}\nonumber\rightCoord{}\\
%%%%%%%%%%%%%%%%%%%%
 \leftCoord{}\rho&=\sqrt{3}H_2+3i\phi-2i\chi-i\sigma,
\leftCoord{}\nonumber\rightCoord{}\\
%%%%%%%%%%%%%%%%%%%%
 \leftCoord{}\widehat H_2&=H_2+\sqrt{3}i\phi-\sqrt{3}i\chi,
}{0mm}{1}{1}{theequation}{0}\coordE{}\end{align}
and then define fermionic fields as
\begin{equation}\coord{}\boxEquation{
\theta^{\alpha\alpha'}=e^{\phi_{\alpha\alpha'}},\qquad
p_{\alpha\alpha'}=e^{-\phi_{\alpha\alpha'}},\qquad
(\alpha,\alpha'=\pm)
\label{bo}
}{
\theta^{\alpha\alpha'}=e^{\phi_{\alpha\alpha'}},\qquad
p_{\alpha\alpha'}=e^{-\phi_{\alpha\alpha'}},\qquad
(\alpha,\alpha'=\pm)
}{ecuacion}\coordE{}\end{equation}
which satisfy
\begin{eqnarray}\coord{}\boxAlignEqnarray{\leftCoord{}
 \theta^{\alpha\alpha'}(z)p_{\beta\beta'}(w)&\sim&
\frac{\leftCoord{}\delta^\alpha_\beta\delta^{\alpha'}_{\beta'}}{\rightCoord{}z-w}.\rightCoord{}
\rightCoord{}}{0mm}{2}{4}{
 \theta^{\alpha\alpha'}(z)p_{\beta\beta'}(w)&\sim&
\frac{\delta^\alpha_\beta\delta^{\alpha'}_{\beta'}}{z-w}.
}{1}\coordE{}\end{eqnarray}
The basic fields of the hybrid superstrings are 
finally defined by Green-Schwarz-like fields with 
an additional boson  \myHighlight{$(X^\pm,Z,Z^*,\theta^{\alpha\alpha'},
p_{\alpha\alpha'},\rho)$}\coordHE{}. 
The \myHighlight{$U(1)$}\coordHE{} boson in the \myHighlight{${\cal M}$}\coordHE{} sector is also 
modified to \myHighlight{$\widehat H_2$}\coordHE{}, which requires modifications of 
superconformal generators to 
\myHighlight{$(\widehat T_{\cal M}, \widehat G^\pm_{\cal M}, 
\widehat I_{\cal M})$}\coordHE{} uniquely determined by change of 
the \myHighlight{$U(1)$}\coordHE{} current
\begin{equation}\coord{}\boxEquation{
 \widehat I_{\cal M}=-\sqrt{3}i\partial\widehat H_2.
}{
 \widehat I_{\cal M}=-\sqrt{3}i\partial\widehat H_2.
}{ecuacion}\coordE{}\end{equation}
We note that these new generators completely (anti-)commute 
with the hybrid fields.

The space-time supercharges (\ref{susy}) are written 
by using these hybrid fields: 
\begin{align}\coord{}\boxAlignEqnarray{
 \leftCoord{}{\cal Q}^{++}&=\oint\dz e^{iX^+}p_{\rightCoord{}--},\nonumber\rightCoord{}\\
%%%%%%%%%%%%%%%%%%%%
 {\rightCoord{}\leftCoord{}\cal Q}^{--}&=\oint\dz e^{-iX^+}\Big(
\leftCoord{}p_{\rightCoord{}++}+i\partial X^+\theta^{--}
\leftCoord{}+(i\partial Z^*+\theta^{-+}p_{\rightCoord{}++})\theta^{+-}
\leftCoord{}+e^{-i\rho}\theta^{-+}\widehat G^-_{\rightCoord{}\cal M}\Big),
\leftCoord{}\nonumber\rightCoord{}\\
%%%%%%%%%%%%%%%%%%%
 \leftCoord{}{\cal Q}^{-+}&=\oint\dz p_{\rightCoord{}+-},\nonumber\rightCoord{}\\
%%%%%%%%%%%%%%%%%%%
 {\rightCoord{}\leftCoord{}\cal Q}^{+-}&=\oint\dz\Big(
\leftCoord{}p_{\rightCoord{}-+}-i\partial X^-\theta^{+-}
\leftCoord{}+i\partial Z\theta^{--}
\leftCoord{}-e^{-i\rho}\theta^{++}\widehat G^-_{\rightCoord{}\cal M}\Big).\label{hsusy}
}{0mm}{4}{4}{theequation}{0}\coordE{}\end{align}
However, these supercharges are not symmetric,
which leads a complicated hermiticity property
to hybrid fields.\cite{Bherm} These fields are
chiral coordinates in the sense that a half of supercharges 
\myHighlight{${\cal Q}^{++}$}\coordHE{} and \myHighlight{${\cal Q}^{-+}$}\coordHE{} are simple 
superderivatives \myHighlight{$p_{--}$}\coordHE{} and \myHighlight{$p_{+-}$}\coordHE{} (except for 
factors \myHighlight{$e^{\pm iX^+}$}\coordHE{}). 
In order to obtain symmetric supercharges and hybrid fields
with proper hermiticity, we must further perform a similarity 
transformation generated by
\begin{align}\coord{}\boxAlignEqnarray{
  \leftCoord{}U&=\oint\dz\Bigg(-e^{-i\rho}\theta^{++}\theta^{-+}\widehat G^-_M
\leftCoord{}+\frac{\leftCoord{}1}{\rightCoord{}2}i\partial X^+\theta^{++}\theta^{--}
\leftCoord{}+\frac{\leftCoord{}1}{\rightCoord{}2}i\partial Z^*\theta^{++}\theta^{+-}
\leftCoord{}\nonumber\rightCoord{}\\
&\leftCoord{}\hspace{2cm}
\leftCoord{}+\frac{\leftCoord{}1}{\rightCoord{}2}i\partial X^-\theta^{+-}\theta^{-+}
\leftCoord{}+\frac{\leftCoord{}1}{\rightCoord{}2}i\partial Z\theta^{-+}\theta^{--}
\leftCoord{}+\frac{\leftCoord{}1}{\rightCoord{}4}\theta^{-+}\theta^{++}\partial(\theta^{--}\theta^{+-})
\leftCoord{}\Bigg).\label{similar}
}{0mm}{14}{7}{
  U&=\oint\dz\Bigg(-e^{-i\rho}\theta^{++}\theta^{-+}\widehat G^-_M
+\frac{1}{2}i\partial X^+\theta^{++}\theta^{--}
+\frac{1}{2}i\partial Z^*\theta^{++}\theta^{+-}
\nonumber\\
&\hspace{2cm}
+\frac{1}{2}i\partial X^-\theta^{+-}\theta^{-+}
+\frac{1}{2}i\partial Z\theta^{-+}\theta^{--}
+\frac{1}{4}\theta^{-+}\theta^{++}\partial(\theta^{--}\theta^{+-})
\Bigg).}{0}\coordE{}\end{align}
In fact, the space-time supercharges (\ref{hsusy})
have symmetric forms
\begin{align}\coord{}\boxAlignEqnarray{
{\rightCoord{}\leftCoord{}\cal Q}^{++}&=\oint\dz e^{iX^+}\left(
\leftCoord{}p_{\rightCoord{}--}+\frac{\leftCoord{}1}{\rightCoord{}2}i\partial X^+\theta^{++}+
\leftCoord{}\frac{\leftCoord{}1}{\rightCoord{}2}(i\partial Z-\theta^{+-}p_{\rightCoord{}--})\theta^{-+}
\leftCoord{}+\frac{\leftCoord{}1}{\rightCoord{}8}\partial(\theta^{-+}\theta^{++})
\leftCoord{}\theta^{+-}\right),\nonumber\rightCoord{}\\
%%%%%%%%%%%%%%%%%%%
{\rightCoord{}\leftCoord{}\cal Q}^{--}&=\oint\dz e^{-iX^+}\left(
\leftCoord{}p_{\rightCoord{}++}+\frac{\leftCoord{}1}{\rightCoord{}2}i\partial X^+\theta^{--}+
\leftCoord{}\frac{\leftCoord{}1}{\rightCoord{}2}(i\partial Z^*+\theta^{-+}p_{\rightCoord{}++})\theta^{+-}
\leftCoord{}-\frac{\leftCoord{}1}{\rightCoord{}8}\partial(\theta^{--}\theta^{+-})\theta^{-+}\right),
\leftCoord{}\nonumber\rightCoord{}\\
%%%%%%%%%%%%%%%%%%%
{\rightCoord{}\leftCoord{}\cal Q}^{-+}&=\oint\dz\left(
\leftCoord{}p_{\rightCoord{}+-}-\frac{\leftCoord{}1}{\rightCoord{}2}i\partial X^-\theta^{-+}+
\leftCoord{}\frac{\leftCoord{}1}{\rightCoord{}2}i\partial Z^*\theta^{++}
\leftCoord{}-\frac{\leftCoord{}1}{\rightCoord{}8}\partial(\theta^{-+}\theta^{++})\theta^{--}\right),
\leftCoord{}\nonumber\rightCoord{}\\
%%%%%%%%%%%%%%%%%%%
{\rightCoord{}\leftCoord{}\cal Q}^{+-}&=\oint\dz\left(
\leftCoord{}p_{\rightCoord{}-+}-\frac{\leftCoord{}1}{\rightCoord{}2}i\partial X^-\theta^{+-}+
\leftCoord{}\frac{\leftCoord{}1}{\rightCoord{}2}i\partial Z\theta^{--}
\leftCoord{}+\frac{\leftCoord{}1}{\rightCoord{}8}\partial(\theta^{--}\theta^{+-})\theta^{++}\right),
\label{susyh}
}{0mm}{7}{6}{theequation}{0}\coordE{}\end{align}
after the similarity transformation.

We can also provide the topological \myHighlight{$N=4$}\coordHE{} superconformal generators 
(\ref{topn4}) using the hybrid fields. 
The \myHighlight{$N=2$}\coordHE{} subalgebra is first given by 
\begin{align}\coord{}\boxAlignEqnarray{
  \leftCoord{}T&=-\partial X^+\partial X^-
\leftCoord{}-\partial Z\partial Z^*
\leftCoord{}-p_{\rightCoord{}\alpha\alpha'}\partial\theta^{\alpha\alpha'}
\leftCoord{}+\frac{\leftCoord{}1}{\rightCoord{}2}\partial\rho\partial\rho
\leftCoord{}+\frac{\leftCoord{}1}{\rightCoord{}2}i\partial^{\leftCoord{}2}\rho
\leftCoord{}+\widehat T_{\rightCoord{}\cal M}
\leftCoord{}+\frac{\leftCoord{}1}{\rightCoord{}2}\partial \widehat I_{\rightCoord{}\cal M},\nonumber\rightCoord{}\\
%%%%%%%%%%%%%%%%%%%%%%%%%%%%%%%%%%%%%%%%%%%%%%%%%%
\leftCoord{}G^+&=e^{-i\rho}\left(
\leftCoord{}d_{\rightCoord{}--}d_{\rightCoord{}+-}
\leftCoord{}+\frac{\leftCoord{}1}{\rightCoord{}8}\partial^{\leftCoord{}2}\theta^{-+}\theta^{++}
\leftCoord{}+\frac{\leftCoord{}1}{\rightCoord{}8}\theta^{-+}\partial^{\leftCoord{}2}\theta^{++}
\leftCoord{}-\frac{\leftCoord{}1}{\rightCoord{}4}\partial^{\leftCoord{}2}(\theta^{-+}\theta^{++})\right)
\leftCoord{}+\widehat G^+_M,\nonumber\rightCoord{}\\
%%%%%%%%%%%%%%%%%%%%%%%%%%%%%%%%%%%%%%%%%%%%%%%%%%
\leftCoord{}G^-&=e^{i\rho}\left(
\leftCoord{}d_{\rightCoord{}-+}d_{\rightCoord{}++}+\frac{\leftCoord{}1}{\rightCoord{}8}\partial^{\leftCoord{}2}\theta^{--}\theta^{+-}
\leftCoord{}+\frac{\leftCoord{}1}{\rightCoord{}8}\theta^{--}\partial^{\leftCoord{}2}\theta^{+-}
\leftCoord{}-\frac{\leftCoord{}1}{\rightCoord{}4}\partial^{\leftCoord{}2}(\theta^{--}\theta^{+-})\right)
\leftCoord{}+\widehat G^-_M,\nonumber\rightCoord{}\\
%%%%%%%%%%%%%%%%%%%%%%%%%%%%%%%%%%%%%%%%%%%%%%%%%%
\leftCoord{}I&=i\partial\rho-\sqrt{3}i\partial\widehat H_2,\label{uone}
}{0mm}{1}{1}{theequation}{0}\coordE{}\end{align}
where
\begin{align}\coord{}\boxAlignEqnarray{
 \leftCoord{}d_{\rightCoord{}--}&=p_{\rightCoord{}--}-\frac{\leftCoord{}1}{\rightCoord{}2}i\partial X^+\theta^{++}
\leftCoord{}-\frac{\leftCoord{}1}{\rightCoord{}2}i\partial Z\theta^{-+}
\leftCoord{}+\frac{\leftCoord{}1}{\rightCoord{}4}\theta^{-+}\theta^{++}\partial\theta^{+-}
\leftCoord{}-\frac{\leftCoord{}1}{\rightCoord{}8}\partial(\theta^{-+}\theta^{++})\theta^{+-},
\leftCoord{}\nonumber\rightCoord{}\\
%%%%%%%%%%%%%%%%%%%%%%%%%%%%%%%%%
 \leftCoord{}d_{\rightCoord{}+-}&=p_{\rightCoord{}+-}+\frac{\leftCoord{}1}{\rightCoord{}2}i\partial X^-\theta^{-+}
\leftCoord{}-\frac{\leftCoord{}1}{\rightCoord{}2}i\partial Z^*\theta^{++}
\leftCoord{}-\frac{\leftCoord{}1}{\rightCoord{}4}\theta^{-+}\theta^{++}\partial\theta^{--}
\leftCoord{}+\frac{\leftCoord{}1}{\rightCoord{}8}\partial(\theta^{-+}\theta^{++})\theta^{--},
\leftCoord{}\nonumber\rightCoord{}\\
%%%%%%%%%%%%%%%%%%%%%%%%%%%%%%%%%
 \leftCoord{}d_{\rightCoord{}++}&=p_{\rightCoord{}++}-\frac{\leftCoord{}1}{\rightCoord{}2}i\partial X^+\theta^{--}
\leftCoord{}-\frac{\leftCoord{}1}{\rightCoord{}2}i\partial Z^*\theta^{+-}
\leftCoord{}-\frac{\leftCoord{}1}{\rightCoord{}4}\theta^{--}\theta^{+-}\partial\theta^{-+}
\leftCoord{}+\frac{\leftCoord{}1}{\rightCoord{}8}\partial(\theta^{--}\theta^{+-})\theta^{-+},
\leftCoord{}\nonumber\rightCoord{}\\
%%%%%%%%%%%%%%%%%%%%%%%%%%%%%%%%%
 \leftCoord{}d_{\rightCoord{}-+}&=p_{\rightCoord{}-+}+\frac{\leftCoord{}1}{\rightCoord{}2}i\partial X^-\theta^{+-}
\leftCoord{}-\frac{\leftCoord{}1}{\rightCoord{}2}i\partial Z\theta^{--}
\leftCoord{}+\frac{\leftCoord{}1}{\rightCoord{}4}\theta^{--}\theta^{+-}\partial\theta^{++}
\leftCoord{}-\frac{\leftCoord{}1}{\rightCoord{}8}\partial(\theta^{--}\theta^{+-})\theta^{++},
\label{scd}
}{0mm}{8}{7}{theequation}{0}\coordE{}\end{align}
are local currents of supercovariant derivatives.
It is also useful to introduce bosonic supercovariant
derivatives as
\begin{align}\coord{}\boxAlignEqnarray{
 \leftCoord{}\Pi^+&=i\partial X^++\frac{\leftCoord{}1}{\rightCoord{}2}\theta^{+-}\partial\theta^{-+}
\leftCoord{}-\frac{\leftCoord{}1}{\rightCoord{}2}\partial\theta^{+-}\theta^{-+},\nonumber\rightCoord{}\\
%%%%%%%%%%%%%%%%%%%%%%%%%%%
 \leftCoord{}\Pi^-&=i\partial X^--\frac{\leftCoord{}1}{\rightCoord{}2}\theta^{++}\partial\theta^{--}
\leftCoord{}+\frac{\leftCoord{}1}{\rightCoord{}2}\partial\theta^{++}\theta^{--},\nonumber\rightCoord{}\\
%%%%%%%%%%%%%%%%%%%%%%%%%%%
 \leftCoord{}\Pi_Z&=i\partial Z-\frac{\leftCoord{}1}{\rightCoord{}2}\theta^{++}\partial\theta^{+-}
\leftCoord{}+\frac{\leftCoord{}1}{\rightCoord{}2}\partial\theta^{++}\theta^{+-},\nonumber\rightCoord{}\\
%%%%%%%%%%%%%%%%%%%%%%%%%%%
 \leftCoord{}\Pi^*_Z&=i\partial Z^*-\frac{\leftCoord{}1}{\rightCoord{}2}\theta^{-+}\partial\theta^{--}
\leftCoord{}+\frac{\leftCoord{}1}{\rightCoord{}2}\partial\theta^{-+}\theta^{--}.\label{scmom}
}{0mm}{4}{3}{theequation}{0}\coordE{}\end{align}
These supercovariant derivatives and 
\myHighlight{$\partial\theta^{\alpha\alpha'}$}\coordHE{} form 
a closed superalgebra
\begin{alignat}{2}
 d_{--}(z)d_{++}(w)&\sim
-\frac{\Pi^+(w)}{z-w},&\qquad
 d_{--}(z)d_{-+}(w)&\sim 
-\frac{\Pi_Z(w)}{z-w},\nonumber\\
%%%%%%%%%%%%%%%%%%%
d_{++}(z)d_{+-}(w)&\sim
-\frac{\Pi^*_Z(w)}{z-w},&\qquad
d_{+-}(z)d_{-+}(w)&\sim
\frac{\Pi^-(w)}{z-w},\nonumber\\
%%%%%%%%%%%%%%%%%%%
\Pi^+(z)\Pi^-(w)&\sim
\frac{1}{(z-w)^2},&\qquad
\Pi_Z(z)\Pi^*_Z(w)&\sim
\frac{1}{(z-w)^2},\nonumber\\
%%%%%%%%%%%%%%%%%%%
d_{--}(z)\Pi^-(w)&\sim
-\frac{\partial\theta^{++}(w)}{z-w},&\qquad
d_{--}(z)\Pi^*_Z(w)&\sim
-\frac{\partial\theta^{-+}(w)}{z-w},\nonumber\\
%%%%%%%%%%%%%%%%%%%
d_{++}(z)\Pi^-(w)&\sim
-\frac{\partial\theta^{--}(w)}{z-w},&\qquad
d_{++}(z)\Pi_Z(w)&\sim
-\frac{\partial\theta^{+-}(w)}{z-w},\nonumber\\
%%%%%%%%%%%%%%%%%%%
d_{+-}(z)\Pi^+(w)&\sim
\frac{\partial\theta^{-+}(w)}{z-w},&\qquad
d_{+-}(z)\Pi_Z(w)&\sim
-\frac{\partial\theta^{++}(w)}{z-w},\nonumber\\
%%%%%%%%%%%%%%%%%%%
d_{-+}(z)\Pi^+(w)&\sim
\frac{\partial\theta^{+-}(w)}{z-w},&\qquad
d_{-+}(z)\Pi^*_Z(w)&\sim
-\frac{\partial\theta^{--}(w)}{z-w}.
\end{alignat}
We note here that these supercovariant derivatives 
are {\it supercovariant} only in the sense that they 
(anti-)commute with a half of supercharges \myHighlight{${\cal Q}^{\pm\mp}$}\coordHE{}.

One can show that complicated forms of \myHighlight{$G^\pm$}\coordHE{} in (\ref{uone}) are 
rewritten as
\begin{align}\coord{}\boxAlignEqnarray{
  \leftCoord{}G^+&=e^{-i\rho}\norm{d_{\rightCoord{}--}d_{\rightCoord{}+-}}
\leftCoord{}+\widehat G^+_M,\nonumber\rightCoord{}\\
 \leftCoord{}G^-&=e^{i\rho}\norm{d_{\rightCoord{}-+}d_{\rightCoord{}++}}
\leftCoord{}+\widehat G^-_M,
}{0mm}{4}{6}{
  G^+&=e^{-i\rho}\norm{d_{--}d_{+-}}
+\widehat G^+_M,\nonumber\\
 G^-&=e^{i\rho}\norm{d_{-+}d_{++}}
+\widehat G^-_M,
}{0}\coordE{}\end{align}
by introducing the new normal ordering \myHighlight{${}^\times_\times$}\coordHE{} 
with respect to the currents \myHighlight{$d_{\alpha\alpha'}$}\coordHE{}.
Whole generators (\ref{topn4}) of the topological 
\myHighlight{$N=4$}\coordHE{} superconformal symmetry are then provided by
\begin{align}\coord{}\boxAlignEqnarray{
  \leftCoord{}T&=-\partial X^+\partial X^-
\leftCoord{}-\partial Z\partial Z^*
\leftCoord{}-p_{\rightCoord{}\alpha\alpha'}\partial\theta^{\alpha\alpha'}
\leftCoord{}+\frac{\leftCoord{}1}{\rightCoord{}2}\partial\rho\partial\rho
\leftCoord{}+\frac{\leftCoord{}1}{\rightCoord{}2}i\partial^{\leftCoord{}2}\rho
\leftCoord{}+\widehat T_{\rightCoord{}\cal M}
\leftCoord{}+\frac{\leftCoord{}1}{\rightCoord{}2}\partial \widehat I_{\rightCoord{}\cal M},\nonumber\rightCoord{}\\
%%%%%%%%%%%%%%%%%%%%%%%%%%%%%%%%%%%%%%%%%%%%%%%%%%
\leftCoord{}G^+&=e^{-i\rho}\norm{d_{\rightCoord{}--}d_{\rightCoord{}+-}}
\leftCoord{}+\widehat G^+_M,\nonumber\rightCoord{}\\
%%%%%%%%%%%%%%%%%%%%%%%%%%%%%%%%%%%%%%%%%%%%%%%%%%
\leftCoord{}G^-&=e^{i\rho}\norm{d_{\rightCoord{}-+}d_{\rightCoord{}++}}
\leftCoord{}+\widehat G^-_M,\nonumber\rightCoord{}\\
%%%%%%%%%%%%%%%%%%%%%%%%%%%%%%%%%%%%%%%%%%%%%%%%%%
\leftCoord{}\widetilde G^+&=
\leftCoord{}e^{2i\rho-\sqrt{3}i\widehat H_2}
\leftCoord{}\norm{d_{\rightCoord{}-+}d_{\rightCoord{}++}}
\leftCoord{}+e^{i\rho-\sqrt{3}i\widehat H_2}\widehat G^-_M,
\leftCoord{}\nonumber\rightCoord{}\\
%%%%%%%%%%%%%%%%%%%%%%%%%%%%%%%%%%%%%%%%%%%%%%%%%%
\leftCoord{}\widetilde G^-&=
\leftCoord{}e^{-2i\rho+\sqrt{3}i\widehat H_2}
\leftCoord{}\norm{d_{\rightCoord{}--}d_{\rightCoord{}+-}}
\leftCoord{}+e^{-i\rho+\sqrt{3}i\widehat H_2}\widehat G^+_M,
\leftCoord{}\nonumber\rightCoord{}\\
%%%%%%%%%%%%%%%%%%%%%%%%%%%%%%%%%%%%%%%%%%%%%%%%%%
\leftCoord{}I^{++}&=e^{i\rho-\sqrt{3}i\widehat H_2},\qquad
%%%%%%%%%%%%%%%%%%%%%%%%%%%%%%%%%%%%%%%%%%%%%%%%%%
\leftCoord{}I^{--}=e^{-i\rho+\sqrt{3}i\widehat H_2},\nonumber\rightCoord{}\\
%%%%%%%%%%%%%%%%%%%%%%%%%%%%%%%%%%%%%%%%%%%%%%%%%%
\leftCoord{}I&=i\partial\rho-\sqrt{3}i\partial\widehat H_2.\label{n4}
}{0mm}{1}{1}{theequation}{0}\coordE{}\end{align}
Physical states are defined by conditions (\ref{phys}) 
using zero modes \myHighlight{$G_0^+$}\coordHE{}, \myHighlight{$\widetilde G_0^+$}\coordHE{} and \myHighlight{$I_0$}\coordHE{} of
these generators.
The supercharges (\ref{susyh}) (anti-)commute with them,
which guarantees the physical spectrum to be supersymmetric.

Finally we rewrite the picture counting operator 
(\ref{picture}) interpreted in the hybrid formalism 
as the R-charge operator:
\begin{equation}\coord{}\boxEquation{
 {\cal R}=\oint\dz\left(
i\partial\rho-\frac{1}{2}(
\theta^{++}p_{++}+\theta^{-+}p_{-+}-
\theta^{--}p_{--}-\theta^{+-}p_{+-})\right).\label{Rch}
}{
 {\cal R}=\oint\dz\left(
i\partial\rho-\frac{1}{2}(
\theta^{++}p_{++}+\theta^{-+}p_{-+}-
\theta^{--}p_{--}-\theta^{+-}p_{+-})\right).}{ecuacion}\coordE{}\end{equation}
This is useful to identify whether each component of superfields
is space-time boson or fermion. The field having
(half-)integral R-charge is space-time boson (fermion) since
it comes from the NS-(R-)sector in the RNS formalism.

\section{Spectral flow and the Hilbert space of 
the hybrid superstring}\label{string}

Now we study the Hilbert space of the hybrid superstring.
Using the hybrid fields, the \myHighlight{$H_4$}\coordHE{} currents are 
realized as
\begin{subequations}\label{hybfree}
\begin{alignat}{2}
 J&=i\partial X^-,&\qquad
 F&=i\partial X^+,\nonumber\\
%%%%%%%%%%%%%%%%%%%%%%%%%%%%%%%%
 P&=e^{iX^+}(i\partial Z-\theta^{+-}p_{--}),&\qquad
 P^*&=e^{-iX^+}(i\partial Z^*+\theta^{-+}p_{++}).\label{nwhyb}
\end{alignat}
We can extend this \myHighlight{$H_4$}\coordHE{} current algebra to a superalgebra, 
which is an analog of the super current algebra (\ref{h4sca}), 
by introducing space-time supercoordinates (and their conjugates)
\begin{alignat}{2}
 \Theta^{\pm\mp}&=\theta^{\pm\mp},&\qquad
 {\cal P}_{\pm\mp}&=p_{\pm\mp},\nonumber\\
%%%%%%%%%%%%
 \Theta^{\pm\pm}&=e^{\pm iX^+}\theta^{\pm\pm},&\qquad 
{\cal P}_{\pm\pm}&=e^{\mp iX^+}p_{\pm\pm},\label{scoo}
\end{alignat}
\end{subequations}
together with an extra \myHighlight{$U(1)$}\coordHE{} current \myHighlight{$i\partial\rho$}\coordHE{}.
The Hilbert space of the hybrid superstring is constructed
by representations of this current superalgebra.
We can expand these currents as
\begin{alignat}{2}\label{currentmodes}
  J(z)&=\sum_nJ_nz^{-n-1},&\qquad
  F(z)&=\sum_nF_nz^{-n-1},\nonumber\\
%%%%%%%%%%%%%%%%%%%%%%%%%%%
  P(z)&=\sum_nP_nz^{-n-1},&\qquad
  P^*(z)&=\sum_nP^*_nz^{-n-1},\nonumber\\
%%%%%%%%%%%%%%%%%%%%%%%%%%%
  \Theta^{\pm\mp}(z)&=\sum_n\Theta^{\pm\mp}_nz^{-n},&\qquad
  {\cal P}_{\pm\mp}(z)&=\sum_n({\cal P}_{\pm\mp})_nz^{-n-1},\nonumber\\
%%%%%%%%%%%%%%%%%%%%%%%%%%%
  \Theta^{\pm\pm}(z)&=\sum_n\Theta^{\pm\pm}_nz^{-n},&\qquad
  {\cal P}_{\pm\pm}(z)&=\sum_n({\cal P}_{\pm\pm})_nz^{-n-1},
\nonumber\\
%%%%%%%%%%%%%%%%%%%%%%%%
  i\partial\rho(z)&=\sum_n\rho_nz^{-n-1},&&
\end{alignat} 
where mode operators satisfy the superalgebra
\begin{alignat}{2}
  \left[J_n,P_m\right]&=P_{n+m},&\qquad
  \left[J_n,P^*_m\right]&=-P_{n+m},\nonumber\\
%%%%%%%%%%%%%%%%%%%%%%%%
  \left[J_n,F_m\right]&=n\delta_{n+m,0},&\qquad 
  \left[P_n,P^*_m\right]&=F_{n+m}+n\delta_{n+m,0},\nonumber\\
%%%%%%%%%%%%%%%%%%%%%%%%
  \left[J_n,\Theta^{\pm\pm}_m\right]&=\pm\Theta^{\pm\pm}_{n+m},&\qquad
  \left[J_n,({\cal P}_{\pm\pm})_m\right]&=\mp({\cal
  P}_{\pm\pm})_{n+m},\nonumber\\
%%%%%%%%%%%%%%%%%%%%%%%%
  \left[P_n,\Theta^{--}_m\right]&=-\Theta^{+-}_{n+m},&\qquad 
  \left[P_n,({\cal P}_{+-})_m\right]&=({\cal P}_{--})_{n+m},\nonumber\\
%%%%%%%%%%%%%%%%%%%%%%%%
  \left[P^*_n,\Theta^{++}_m\right]&=\Theta^{-+}_{n+m},&\qquad
  \left[P^*_n,({\cal P}_{-+})_m\right]&=-({\cal P}_{++})_{n+m},\nonumber\\
%%%%%%%%%%%%%%%%%%%%%%%%
  \left\{\Theta^{\pm\pm}_n,({\cal P}_{\pm\pm})_m\right\}&=\delta_{n+m,0},
&\qquad
  \left\{\Theta^{\pm\mp}_n,({\cal P}_{\pm\mp})_m\right\}&=\delta_{n+m,0},
\nonumber\\
%%%%%%%%%%%%%%%%%%%%%%%%
\left[\rho_n,\rho_m\right]&=-n\delta_{n+m,0}.&&\label{hybcur}
\end{alignat}
Since the hybrid fields already provide a free field 
realization (\ref{hybfree}), we can easily obtain 
representations of this superalgebra (\ref{hybcur}).
As in the \myHighlight{$H_4$}\coordHE{} (super) current algebra,\cite{KK,KP}\cite{HS2}
only the non-trivial point is the existence of the spectral flow
symmetry, {\it i.e.} the superalgebra (\ref{hybcur}) is 
preserved by replacement
\begin{alignat}{2}
  J_n&\longrightarrow J_n,&\qquad
  F_n&\longrightarrow F_n+p\delta_{n,0},\nonumber\\
%%%%%%%%%%%%%%%%%%%%%
  P_n&\longrightarrow P_{n+p},&\qquad
  P^*_n&\longrightarrow P^*_{n-p},\nonumber\\
%%%%%%%%%%%%%%%%%%%%%
  \Theta^{\pm\mp}_n&\longrightarrow\Theta^{\pm\mp}_n,&\qquad
  ({\cal P}_{\pm\mp})_n&\longrightarrow ({\cal P}_{\pm\mp})_n,\nonumber\\
%%%%%%%%%%%%%%%%%%%%%
  \Theta^{\pm\pm}_n&\longrightarrow\Theta^{\pm\pm}_{n\pm p},&\qquad
  ({\cal P}_{\pm\pm})_n&\longrightarrow ({\cal P}_{\pm\pm})_{n\mp p},
\nonumber\\
%%%%%%%%%%%%%%%%%%%%%
 \rho_n&\longrightarrow\rho_n,& &
\label{spcf}
\end{alignat}
for any integer \myHighlight{$p\in\mathbf{Z}$}\coordHE{}.
The Hilbert space contains all spectrally flowed representations 
classified into two types describing short and long strings.\cite{KP}

\subsection{The Hilbert space of short strings}\label{short}

The Hilbert space of short strings in the hybrid formalism include
all spectrally flowed type II representations\cite{KK} (\myHighlight{$0<\eta<1$}\coordHE{})
defined by
\begin{alignat}{2}
 J_0|j,\eta,p,l\rangle&=j|j,\eta,p,l\rangle,&\qquad
 F_0|j,\eta,p,l\rangle&=(\eta+p)|j,\eta,p,l\rangle,\nonumber\\
%%%%%%%%%%%%%%%%%%%%%%
 J_n|j,\eta,p,l\rangle&=0,\quad (n>0),&\quad
 F_n|j,\eta,p,l\rangle&=0,\quad (n>0),\nonumber\\
%%%%%%%%%%%%%%%%%%%%%%
 P_n|j,\eta,p,l\rangle&=0,\quad (n\ge-p),&\quad
 P^*_n|j,\eta,p,l\rangle&=0,\quad (n>p),\nonumber\\
%%%%%%%%%%%%%%%%%%%%%%
 ({\cal P}_{\pm\mp})_n|j,\eta,p,l\rangle&=0,\quad (n\ge0),&\quad
 \Theta^{\pm\mp}_n|j,\eta,p,l\rangle&=0,\quad (n>0),\nonumber\\
%%%%%%%%%%%%%%%%%%%%%%
 ({\cal P}_{++})_n|j,\eta,p,l\rangle&=0,\quad (n\ge p+1),&\quad
 \Theta^{++}_n|j,\eta,p,l\rangle&=0,\quad (n>-p-1),\nonumber\\
%%%%%%%%%%%%%%%%%%%%%%
 ({\cal P}_{--})_n|j,\eta,p,l\rangle&=0,\quad (n\ge-p),&\quad
 \Theta^{--}_n|j,\eta,p,l\rangle&=0,\quad (n>p),\nonumber\\
%%%%%%%%%%%%%%%%%%%%%%
 \rho_0|j,\eta,p,l\rangle&=(l-\eta)|j,\eta,p,l\rangle,&\quad
 \rho_n|j,\eta,p,l\rangle&=0,\quad (n>0),
\label{typeii}
\end{alignat}
where \myHighlight{$l=0,\pm 1$}\coordHE{}. The \myHighlight{$\rho_0$}\coordHE{} eigenvalue is fixed so that 
it make the supercurrents \myHighlight{$G^\pm$}\coordHE{} in (\ref{n4}) periodic
and select a unique representative from infinitely 
degenerated states due to the pictures. 

The explicit representations are easily constructed in terms of 
the hybrid fields by noting that
the transverse fields \myHighlight{$(Z,Z^*,\theta^{\pm\pm},p_{\pm\pm})$}\coordHE{}
obey the twisted boundary condition:
\begin{alignat}{2}
   i\partial Z(e^{2\pi i}z)&=e^{-2\pi i\eta}i\partial Z(z),&\qquad
   i\partial Z^*(e^{2\pi i}z)&=e^{2\pi i\eta}i\partial Z^*(z),
\nonumber\\
%%%%%%%%%%%%%%%%%%%%%
   \theta^{\pm\pm}(e^{2\pi i}z)&=e^{\mp 2\pi i\eta}\theta^{\pm\pm}(z),
&\qquad
   p_{\pm\pm}(e^{2\pi i}z)&=e^{\pm 2\pi i\eta}p_{\pm\pm}(z).
\label{twstbc}
\end{alignat}
Then, the hybrid fields can be expanded as
\begin{alignat}{2}
   i\partial X^\pm(z)&=\sum_n\alpha^\pm_nz^{-n-1},&&\nonumber\\
%%%%%%%%%%%%%%%%%%%%%
  i\partial Z(z)&=\sum_nZ_{n+\eta}z^{-n-\eta-1},&\qquad
  i\partial Z^*(z)&=\sum_nZ^*_{n-\eta}z^{-n+\eta-1},
\nonumber\\
%%%%%%%%%%%%%%%%%%%%%%%%
  \theta^{\pm\mp}(z)&=\sum_n\theta^{\pm\mp}_nz^{-n},&\qquad
  p_{\pm\mp}(z)&=\sum_n(p_{\pm\mp})_nz^{-n-1},\nonumber\\
%%%%%%%%%%%%%%%%%%%%%%%%%%%
  \theta^{\pm\pm}(z)&=\sum_n\theta^{\pm\pm}_{n\pm\eta}z^{-n\mp\eta},
&\qquad
  p_{\pm\pm}(z)&=\sum_n(p_{\pm\pm})_{n\mp\eta}z^{-n\pm\eta-1},
\label{ffr}
\end{alignat}
where the oscillators satisfy the canonical 
(anti-)commutation relations
\begin{alignat}{2}
  \left[\alpha^+_n,\alpha^-_m\right]&=n\delta_{n+m,0},&\qquad
  \left[Z_{n+\eta},Z^*_{m-\eta}\right]&=(n+\eta)\delta_{n+m,0},
\nonumber\\
%%%%%%%%%%%%%%%%%%%%%%
  \left\{\theta^{\pm\mp}_n,(p_{\pm\mp})_m\right\}&=\delta_{n+m,0},&\qquad
  \left\{\theta^{\pm\pm}_{n\mp\eta},(p_{\pm\pm})_{m\pm\eta}\right\}&
=\delta_{n+m,0},
\label{oscccr}
\end{alignat}

The flowed type II representations are simply realized as 
Fock states of these oscillators (and \myHighlight{$\rho_n$}\coordHE{}) on the ground state
 \begin{alignat}{2}  
 \alpha^-_0|\eta,\boldsymbol p,\boldsymbol\theta,l\rangle&=
j|\eta,\boldsymbol p,\boldsymbol\theta,l\rangle,&\qquad
 \alpha^+_0|\eta,\boldsymbol p,\boldsymbol\theta,l\rangle&=
(\eta+p)|\eta,\boldsymbol p,\boldsymbol\theta,l\rangle,\nonumber\\
%%%%%%%%%%%%%%%%%%%%%%%%
 \alpha^\pm_n|\eta,\boldsymbol p,\boldsymbol\theta,l\rangle&=0,\quad
  (n>0),&&
\nonumber\\
%%%%%%%%%%%%%%%%%%%%%%%%
 Z_{n+\eta}|\eta,\boldsymbol p,\boldsymbol\theta,l\rangle&=0,\quad (n\ge 0),&\qquad
 Z^*_{n-\eta}|\eta,\boldsymbol p,\boldsymbol\theta,l\rangle&=0,\quad
  (n>0),
\nonumber\\
%%%%%%%%%%%%%%%%%%%%%%%%
 (p_{+-})_0|\eta,\boldsymbol p,\boldsymbol\theta,l\rangle&=
|\eta,\boldsymbol p,\boldsymbol\theta,l\rangle\frac{\partial}{\partial\theta},&\qquad
 \theta^{+-}_0|\eta,\boldsymbol p,\boldsymbol\theta,l\rangle&=
|\eta,\boldsymbol p,\boldsymbol\theta,l\rangle\theta,\nonumber\\
%%%%%%%%%%%%%%%%%%%%%%%%
 (p_{-+})_0|\eta,\boldsymbol p,\boldsymbol\theta,l\rangle&=|\eta,\boldsymbol
  p,\boldsymbol\theta,l\rangle\frac{\partial}{\partial\bar\theta},
&\qquad
 \theta^{-+}_0|\eta,\boldsymbol p,\boldsymbol\theta,l\rangle&=
|\eta,\boldsymbol p,\boldsymbol\theta,l\rangle\bar\theta,\nonumber\\
%%%%%%%%%%%%%%%%%%%%%%%%
 (p_{\pm\mp})_n|\eta,\boldsymbol p,\boldsymbol\theta,l\rangle&=0,\quad (n> 0),&\qquad
 \theta^{\pm\mp}_n|\eta,\boldsymbol
  p,\boldsymbol\theta,l\rangle&=0,\quad (n>0),
\nonumber\\
%%%%%%%%%%%%%%%%%%%%%%%%
 (p_{++})_{n-\eta}|\eta,\boldsymbol p,\boldsymbol\theta,l\rangle&=0,
\quad (n>0),
&\qquad
 \theta^{++}_{n+\eta}|\eta,\boldsymbol p,\boldsymbol\theta,l\rangle&=0,
\quad (n\ge 0),\nonumber\\
%%%%%%%%%%%%%%%%%%%%%%%
 (p_{--})_{n+\eta}|\eta,\boldsymbol p,\boldsymbol\theta,l\rangle&=0,
\quad (n\ge 0),&\qquad
 \theta^{--}_{n-\eta}|\eta,\boldsymbol
  p,\boldsymbol\theta,l\rangle&=0,\quad (n>0),\nonumber\\
%%%%%%%%%%%%%%%%%%%%%%%
 \rho_0|\eta,\boldsymbol p,\boldsymbol\theta,l\rangle&=
(l-\eta)|\eta,\boldsymbol p,\boldsymbol\theta,l\rangle,&\qquad
 \rho_n|\eta,\boldsymbol p,\boldsymbol\theta,l\rangle&=0,\quad (n\ge0),
\label{ffgs}
 \end{alignat}
where we have diagonalized zero modes
\myHighlight{$\alpha^\pm_0(=p^\pm)$}\coordHE{} and \myHighlight{$\theta^{\pm\pm}_0$}\coordHE{}, 
and denoted their eigenvalues by \myHighlight{$\boldsymbol p=(j,p)$}\coordHE{} and 
\myHighlight{$\boldsymbol\theta=(\theta,\bar\theta)$}\coordHE{}. 
The short string states are obtained by multiplying the Fock states
by a superfield \myHighlight{$\Psi(\boldsymbol p,\boldsymbol\theta)$}\coordHE{} on which
\myHighlight{$\frac{\partial}{\partial\theta}$}\coordHE{} and
\myHighlight{$\frac{\partial}{\partial\bar\theta}$}\coordHE{} act.
Since \myHighlight{$Z$}\coordHE{} and \myHighlight{$Z^*$}\coordHE{} do not have zero modes, the short string 
is localized and cannot reach infinity in the transverse 
space.

The total Hilbert space is obtained by the tensor product of 
this Fock space and unitary representations 
describing the \myHighlight{${\cal M}$}\coordHE{} sector. Since an arbitrary
unitary representation of the \myHighlight{$N=2$}\coordHE{} superconformal field theory is 
characterized by dimension \myHighlight{$\Delta$}\coordHE{} and \myHighlight{$U(1)$}\coordHE{} charge \myHighlight{$Q$}\coordHE{}, 
we can formally define a unitary representation by
 \begin{align}\coord{}\boxAlignEqnarray{
  \leftCoord{}\widehat L_{\rightCoord{}0,{\cal M}}|\Delta,Q\rangle&=
\leftCoord{}\Delta|\Delta,Q\rangle,\nonumber\rightCoord{}\\
  \leftCoord{}\widehat I_{\rightCoord{}0,{\cal M}}|\Delta,Q\rangle&=
\leftCoord{}Q|\Delta,Q\rangle.
 }{0mm}{4}{4}{
  \widehat L_{0,{\cal M}}|\Delta,Q\rangle&=
\Delta|\Delta,Q\rangle,\nonumber\\
  \widehat I_{0,{\cal M}}|\Delta,Q\rangle&=
Q|\Delta,Q\rangle.
 }{0}\coordE{}\end{align}
The short string is represented by
Fock states on the ground state
\begin{equation}\coord{}\boxEquation{
 |\eta,\boldsymbol p,\boldsymbol\theta,l;\Delta,Q\rangle=
|\eta,\boldsymbol p,\boldsymbol\theta,l\rangle\otimes
|\Delta,Q\rangle.\label{shortground}
}{
 |\eta,\boldsymbol p,\boldsymbol\theta,l;\Delta,Q\rangle=
|\eta,\boldsymbol p,\boldsymbol\theta,l\rangle\otimes
|\Delta,Q\rangle.}{ecuacion}\coordE{}\end{equation}

For later use, we note that this ground state have
eigenvalues
 \begin{align}\coord{}\boxAlignEqnarray{
  \leftCoord{}L_0|\eta,\boldsymbol p,\boldsymbol\theta,l;\Delta,Q\rangle&=
\leftCoord{}\Big((\eta+p)j-\frac{\leftCoord{}1}{\rightCoord{}2}(l-\eta)(l-\eta+1)\nonumber\rightCoord{}\\
&\leftCoord{}\hspace{1.8cm}
\leftCoord{}+\Delta-\frac{\leftCoord{}1}{\rightCoord{}2}Q-\frac{\leftCoord{}1}{\rightCoord{}2}\eta(1-\eta)\Big)
\leftCoord{}|\eta,\boldsymbol p,\boldsymbol\theta,l;\Delta,Q\rangle,
\leftCoord{}\label{totene}\rightCoord{}\\
%%%%%%%%%%%%%%%%%%%%%%
  \leftCoord{}I_0|\eta,\boldsymbol p,\boldsymbol\theta,l;\Delta,Q\rangle&=
\leftCoord{}(l-\eta+Q)|\eta,\boldsymbol p,\boldsymbol\theta,l;\Delta,Q\rangle,
\leftCoord{}\label{toti}\rightCoord{}\\
%%%%%%%%%%%%%%%%%%%%%%
 {\rightCoord{}\leftCoord{}\cal R}|\eta,\boldsymbol p,\boldsymbol\theta,l;\Delta,Q\rangle
&\leftCoord{}=|\eta,\boldsymbol p,\boldsymbol\theta,l;\Delta,Q\rangle\left(
\leftCoord{}(l-\frac{\leftCoord{}1}{\rightCoord{}2})+\frac{\leftCoord{}1}{\rightCoord{}2}(\theta\frac{\leftCoord{}\partial}{\rightCoord{}\partial\theta}
\leftCoord{}-\bar\theta\frac{\leftCoord{}\partial}{\rightCoord{}\partial\bar\theta})\right),\label{t2r}
 }{0mm}{8}{6}{theequation}{0}\coordE{}\end{align}
where constant terms in (\ref{totene}) and (\ref{t2r})
are easily derived by using the bosonization (\ref{bo}).
The \myHighlight{$U(1)$}\coordHE{} charge condition (\ref{u1}) together with 
the \myHighlight{$I_0$}\coordHE{} eigenvalue (\ref{toti}) imposes that the charge \myHighlight{$Q$}\coordHE{} 
of the short string must be fractional.

\subsection{The Hilbert space of long strings}\label{long}

The long-string Hilbert space is given
by spectrally flowed type I representations 
\myHighlight{$(\eta=0)$}\coordHE{}.\cite{KK} Mode expansions are 
easily obtained by setting \myHighlight{$\eta=0$}\coordHE{} in the previous 
expressions except for the transverse coordinates 
\myHighlight{$(Z,Z^*,\theta^{\pm\pm},p_{\pm\pm})$}\coordHE{} 
having additional zero-modes. 
The Fock vacuum is defined by
 \begin{alignat}{2}
   \alpha^-_0|\boldsymbol p,\boldsymbol q,\boldsymbol\theta,
\boldsymbol{\tilde\theta},l\rangle&=
j|\boldsymbol p,\boldsymbol q,\boldsymbol\theta,\boldsymbol{\tilde\theta}
,l\rangle,&\qquad
 \alpha^+_0|\boldsymbol p,\boldsymbol q,\boldsymbol\theta,
\boldsymbol{\tilde\theta},l\rangle&=p|\boldsymbol p,
\boldsymbol q,\boldsymbol\theta,\boldsymbol{\tilde\theta},l\rangle,
\nonumber\\
%%%%%%%%%%%%%%%%%%%%%%%%
 \alpha^\pm_n|\boldsymbol p,\boldsymbol q,
\boldsymbol\theta,\boldsymbol{\tilde\theta}
,l\rangle&=0,\quad (n>0),&&\nonumber\\
%%%%%%%%%%%%%%%%%%%%%%%%
 Z_0|\boldsymbol p,\boldsymbol q,
\boldsymbol\theta,\boldsymbol{\tilde\theta}
,l\rangle&=q|\boldsymbol p,\boldsymbol q,
\boldsymbol\theta,\boldsymbol{\tilde\theta},l\rangle,&\qquad
 Z^*_0|\boldsymbol p,\boldsymbol q,
\boldsymbol\theta,\boldsymbol{\tilde\theta}
,l\rangle&=q^*|\boldsymbol p,
\boldsymbol q,\boldsymbol\theta,\boldsymbol{\tilde\theta},l\rangle,
\nonumber\\
%%%%%%%%%%%%%%%%%%%%%%%%
 Z_n|\boldsymbol p,\boldsymbol q,
\boldsymbol\theta,\boldsymbol{\tilde\theta}
,l\rangle&=0,\quad (n>0),&\qquad
 Z^*_n|\boldsymbol p,\boldsymbol q,
\boldsymbol\theta,\boldsymbol{\tilde\theta},l\rangle&=0,\quad (n>0),
\nonumber\\
%%%%%%%%%%%%%%%%%%%%%%%%
 (p_{+-})_0|\boldsymbol p,\boldsymbol q,
\boldsymbol\theta,\boldsymbol{\tilde\theta},l\rangle&=
|\boldsymbol p,\boldsymbol q,
\boldsymbol\theta,\boldsymbol{\tilde\theta}
,l\rangle\frac{\partial}{\partial\theta},&\qquad
 \theta^{+-}_0|\boldsymbol p,\boldsymbol q,
\boldsymbol\theta,\boldsymbol{\tilde\theta}
,l\rangle&=|\boldsymbol p,\boldsymbol q,
\boldsymbol\theta,\boldsymbol{\tilde\theta},l\rangle\theta,
\nonumber\\
%%%%%%%%%%%%%%%%%%%%%%%%
 (p_{-+})_0|\boldsymbol p,\boldsymbol q,
\boldsymbol\theta,\boldsymbol{\tilde\theta},l\rangle&=
|\boldsymbol p,\boldsymbol q,
\boldsymbol\theta,\boldsymbol{\tilde\theta}
,l\rangle\frac{\partial}{\partial\bar\theta},&\qquad
 \theta^{-+}_0|\boldsymbol p,\boldsymbol q,
\boldsymbol\theta,\boldsymbol{\tilde\theta}
,l\rangle&=|\boldsymbol p,\boldsymbol q,
\boldsymbol\theta,\boldsymbol{\tilde\theta},l\rangle\bar\theta,
\nonumber\\
%%%%%%%%%%%%%%%%%%%%%%%%
 (p_{\pm\mp})_n|\boldsymbol p,\boldsymbol q,
\boldsymbol\theta,\boldsymbol{\tilde\theta}
,l\rangle&=0,\quad (n> 0),&\qquad
 \theta^{\pm\mp}_n|\boldsymbol p,\boldsymbol q,
\boldsymbol\theta,\boldsymbol{\tilde\theta},l\rangle&=0,\quad (n>0),
\nonumber\\
%%%%%%%%%%%%%%%%%%%%%%%%
 (p_{--})_0|\boldsymbol p,\boldsymbol q,
\boldsymbol\theta,\boldsymbol{\tilde\theta},l\rangle&=
|\boldsymbol p,\boldsymbol q,
\boldsymbol\theta,\boldsymbol{\tilde\theta}
,l\rangle\frac{\partial}{\partial\tilde\theta},&\qquad
 \theta^{--}_0|\boldsymbol p,\boldsymbol q,
\boldsymbol\theta,\boldsymbol{\tilde\theta}
,l\rangle&=|\boldsymbol p,\boldsymbol q,
\boldsymbol\theta,\boldsymbol{\tilde\theta},l\rangle
\tilde\theta,\nonumber\\
%%%%%%%%%%%%%%%%%%%%%%%%
 (p_{++})_0|\boldsymbol p,\boldsymbol q,
\boldsymbol\theta,\boldsymbol{\tilde\theta},l\rangle&=
|\boldsymbol p,\boldsymbol q,\boldsymbol\theta,
\boldsymbol{\tilde\theta},l\rangle\frac{\partial}
{\partial\bar{\tilde\theta}},&\qquad
 \theta^{++}_0|\boldsymbol p,\boldsymbol q,
\boldsymbol\theta,\boldsymbol{\tilde\theta}
,l\rangle&=|\boldsymbol p,\boldsymbol q,
\boldsymbol\theta,\boldsymbol{\tilde\theta},l\rangle
 \bar{\tilde\theta},\nonumber\\
%%%%%%%%%%%%%%%%%%%%%%%%
 (p_{--})_n|\boldsymbol p,\boldsymbol q,
\boldsymbol\theta,\boldsymbol{\tilde\theta}
,l\rangle&=0,\quad (n>0),&\qquad
 \theta^{--}_n|\boldsymbol p,\boldsymbol q,
\boldsymbol\theta,\boldsymbol{\tilde\theta},l\rangle&=0,\quad (n>0),
\nonumber\\
%%%%%%%%%%%%%%%%%%%%%%%
 (p_{++})_n|\boldsymbol p,\boldsymbol q,
\boldsymbol\theta,\boldsymbol{\tilde\theta}
,l\rangle&=0,\quad (n>0),&\qquad
 \theta^{++}_n|\boldsymbol p,\boldsymbol q,
\boldsymbol\theta,\boldsymbol{\tilde\theta},l\rangle&=0,\quad (n>0),
\nonumber\\
%%%%%%%%%%%%%%%%%%%%%%%
 \rho_0|\boldsymbol p,\boldsymbol q,
\boldsymbol\theta,\boldsymbol{\tilde\theta}
,l\rangle&=l|\boldsymbol p,\boldsymbol q,
\boldsymbol\theta,\boldsymbol{\tilde\theta}
,l\rangle,&\qquad
\rho_n|\boldsymbol p,\boldsymbol q,
\boldsymbol\theta,\boldsymbol{\tilde\theta}
,l\rangle&=0,\quad (n>0),
 \end{alignat} 
where \myHighlight{$\boldsymbol q=(q,q^*)$}\coordHE{} and \myHighlight{$\boldsymbol{\tilde\theta}=
(\tilde\theta,\bar{\tilde\theta})$}\coordHE{} are the additional zero modes.
The coefficient superfield in this sector is a function of 
zero modes \myHighlight{$(\boldsymbol p,\boldsymbol q,
\boldsymbol\theta,\boldsymbol{\tilde\theta})$}\coordHE{}.
The long strings can freely propagate in the four-dimensional
space  \myHighlight{$(X^\pm,Z,Z^*)$}\coordHE{}.

The long string is represented by Fock states on the ground state
\begin{equation}\coord{}\boxEquation{
 |\boldsymbol p,\boldsymbol q,
\boldsymbol\theta,\boldsymbol{\tilde\theta},l;\Delta,Q\rangle=
 |\boldsymbol p,\boldsymbol q,
\boldsymbol\theta,\boldsymbol{\tilde\theta},l\rangle
\otimes|\Delta,Q\rangle,
}{
 |\boldsymbol p,\boldsymbol q,
\boldsymbol\theta,\boldsymbol{\tilde\theta},l;\Delta,Q\rangle=
 |\boldsymbol p,\boldsymbol q,
\boldsymbol\theta,\boldsymbol{\tilde\theta},l\rangle
\otimes|\Delta,Q\rangle,
}{ecuacion}\coordE{}\end{equation}
having eigenvalues
\begin{align}\coord{}\boxAlignEqnarray{ 
 \leftCoord{}L_0|\boldsymbol p,\boldsymbol q,
\leftCoord{}\boldsymbol\theta,\boldsymbol{\tilde\theta},l;\Delta,Q\rangle&=
 \leftCoord{}(pj+qq^*-\frac{\leftCoord{}1}{\rightCoord{}2}l(l+1)+\Delta-\frac{\leftCoord{}1}{\rightCoord{}2}Q)
\leftCoord{}|\boldsymbol p,\boldsymbol q,
\leftCoord{}\boldsymbol\theta,\boldsymbol{\tilde\theta},l;\Delta,Q\rangle,\rightCoord{}\\
%%%%%%%%%%%%%%%%%%%%%%%%%%%%%%
 \leftCoord{}I_0|\boldsymbol p,\boldsymbol q,
\leftCoord{}\boldsymbol\theta,\boldsymbol{\tilde\theta},l;\Delta,Q\rangle&=
 \leftCoord{}(l+Q)|\boldsymbol p,\boldsymbol q,
\leftCoord{}\boldsymbol\theta,\boldsymbol{\tilde\theta},l;\Delta,Q\rangle,
\leftCoord{}\label{contu1}\rightCoord{}\\
%%%%%%%%%%%%%%%%%%%%%%%%%%%%%%
 {\rightCoord{}\leftCoord{}\cal R}|\boldsymbol p,\boldsymbol q,
\leftCoord{}\boldsymbol\theta,\boldsymbol{\tilde\theta},l;\Delta,Q\rangle&=
 \leftCoord{}|\boldsymbol p,\boldsymbol q,
\leftCoord{}\boldsymbol\theta,\boldsymbol{\tilde\theta},l;\Delta,Q\rangle
 \leftCoord{}\left(l+\frac{\leftCoord{}1}{\rightCoord{}2}\left(\theta\frac{\leftCoord{}\partial}{\rightCoord{}\partial\theta}
 \leftCoord{}+\tilde\theta\frac{\leftCoord{}\partial}{\rightCoord{}\partial\tilde\theta}
\leftCoord{}-\bar\theta\frac{\leftCoord{}\partial}{\rightCoord{}\partial\bar\theta}
\leftCoord{}-\bar{\tilde\theta}\frac{\leftCoord{}\partial}{\rightCoord{}\partial\bar{\tilde\theta}}\right)\right).
\label{utotlong}
}{0mm}{13}{7}{theequation}{0}\coordE{}\end{align}
The \myHighlight{$U(1)$}\coordHE{} charge condition (\ref{u1}) together with 
the \myHighlight{$I_0$}\coordHE{} eigenvalue (\ref{contu1}) leads that the long string
must have integral \myHighlight{$Q$}\coordHE{}.

\section{Physical spectrum}\label{physspec}

In this section
we study the physical spectrum at lower mass levels explicitly.
We concentrate on the states whose \myHighlight{${\cal M}$}\coordHE{} sector is 
provided by (anti-)chiral primary states characterized 
by \myHighlight{$\Delta=\frac{|Q|}{2}$}\coordHE{}, then solve the physical state 
conditions (\ref{phys}).

\subsection{physical states in the short string sector}

We first examine physical states at mass levels 
\myHighlight{$N=0,\eta,1-\eta$}\coordHE{} in the short string sector.
One can easily show that it is enough to study the \myHighlight{$l=1,0$}\coordHE{} 
cases since there is no physical state having 
\myHighlight{$\rho_0$}\coordHE{}-momentum \myHighlight{$l=-1$}\coordHE{} at these levels.
The \myHighlight{$U(1)$}\coordHE{} charge condition (\ref{u1}) and the chirality 
condition \myHighlight{$\Delta=\frac{|Q|}{2}$}\coordHE{} lead 
\myHighlight{$\Delta=-\frac{Q}{2}=\frac{1}{2}(1-\eta)$}\coordHE{} for the \myHighlight{$l=1$}\coordHE{} case
and \myHighlight{$\Delta=\frac{Q}{2}=\frac{\eta}{2}$}\coordHE{} for the \myHighlight{$l=0$}\coordHE{} case. 

Let us start to consider the oscillator ground state \myHighlight{$N=0$}\coordHE{} 
with \myHighlight{$l=1$}\coordHE{} provided by
\begin{equation}\coord{}\boxEquation{
|V\rangle=|1\rangle\Psi^{(\frac{1}{2})}
(\boldsymbol p,\boldsymbol\theta).
}{
|V\rangle=|1\rangle\Psi^{(\frac{1}{2})}
(\boldsymbol p,\boldsymbol\theta).
}{ecuacion}\coordE{}\end{equation}
We denote here the state (\ref{shortground}) with \myHighlight{$l=1$}\coordHE{} by 
\myHighlight{$|1\rangle$}\coordHE{} and use this abbreviation in this subsection,
for simplicity. A half of supersymmetry is realized on 
the coefficient superfield \myHighlight{$\Psi^{(\frac{1}{2})}$}\coordHE{} by\footnote{
Another half relates different mass states as a part of the DDF
operators discussed in section \ref{summary}.
}
 \begin{equation}\coord{}\boxEquation{
  Q^{-+}=\frac{\partial}{\partial\theta}-\frac{1}{2}j\bar\theta,\qquad
  Q^{+-}=\frac{\partial}{\partial\bar\theta}-\frac{1}{2}j\theta.
\label{susyforcomp}
 }{
  Q^{-+}=\frac{\partial}{\partial\theta}-\frac{1}{2}j\bar\theta,\qquad
  Q^{+-}=\frac{\partial}{\partial\bar\theta}-\frac{1}{2}j\theta.
}{ecuacion}\coordE{}\end{equation}
Superscript \myHighlight{$(\frac{1}{2})$}\coordHE{} of the coefficient superfield indicates 
that its first component has R-charge \myHighlight{$\frac{1}{2}$}\coordHE{}, 
which can be read from Eq.~(\ref{t2r}). 
The physical state conditions (\ref{phys}) 
lead manifestly supersymmetric conditions on the superfield:
\begin{subequations}\label{tac}
 \begin{eqnarray}\coord{}\boxAlignEqnarray{\leftCoord{}
  D\bar D\Psi^{(\frac{\leftCoord{}1}{\rightCoord{}2})}&=&0,\rightCoord{}\label{taceom}\rightCoord{}\\\leftCoord{}
  \delta\Psi^{(\frac{\leftCoord{}1}{\rightCoord{}2})}&=&\bar D\Lambda^{(1)},\rightCoord{}\label{tacgau}
\rightCoord{}}{0mm}{4}{7}{
  D\bar D\Psi^{(\frac{1}{2})}&=&0,\\
  \delta\Psi^{(\frac{1}{2})}&=&\bar D\Lambda^{(1)},}{1}\coordE{}\end{eqnarray}
\end{subequations}
where \myHighlight{$\Lambda^{(1)}$}\coordHE{} is an arbitrary gauge parameter superfield
and supercovariant derivatives are defined by
 \begin{equation}\coord{}\boxEquation{
  D=\frac{\partial}{\partial\theta}+\frac{1}{2}j\bar\theta,\qquad
 \bar D=\frac{\partial}{\partial\bar\theta}+\frac{1}{2}j\theta.
\label{supercov}
 }{
  D=\frac{\partial}{\partial\theta}+\frac{1}{2}j\bar\theta,\qquad
 \bar D=\frac{\partial}{\partial\bar\theta}+\frac{1}{2}j\theta.
}{ecuacion}\coordE{}\end{equation}

These conditions (\ref{tac}) can be easily solved by taking 
an appropriate gauge as 
\begin{equation}\coord{}\boxEquation{
 \Psi^{(\frac{1}{2})}=\bar\theta\bar\phi^{(0)}(p,j=0).\label{tachyon}
}{
 \Psi^{(\frac{1}{2})}=\bar\theta\bar\phi^{(0)}(p,j=0).}{ecuacion}\coordE{}\end{equation}
The physical component \myHighlight{$\bar\phi^{(0)}$}\coordHE{} is a space-time boson 
and identified with the {\it tachyon like} state obtained 
in Ref.~\citen{HS2}. The solution (\ref{tachyon}) also shows 
that there is no fermionic massless physical state, 
{\it i.e.} the physical spectrum has boson-fermion asymmetry. 
This is only possible for the massless \myHighlight{$(j=0)$}\coordHE{} state
on which the supercharges (\ref{susyforcomp}) anti-commute.

For the oscillator ground state with \myHighlight{$l=0$}\coordHE{}
\begin{equation}\coord{}\boxEquation{
  |V\rangle=
|0\rangle\Psi^{(-\frac{1}{2})}(\boldsymbol p,\boldsymbol\theta),
}{
  |V\rangle=
|0\rangle\Psi^{(-\frac{1}{2})}(\boldsymbol p,\boldsymbol\theta),
}{ecuacion}\coordE{}\end{equation}
physical state conditions are provided by
 \begin{eqnarray}\coord{}\boxAlignEqnarray{\leftCoord{}
  \bar D D\Psi^{(-\frac{\leftCoord{}1}{\rightCoord{}2})}&=&0,\nonumber\rightCoord{}\\\leftCoord{}
  \delta\Psi^{(-\frac{\leftCoord{}1}{\rightCoord{}2})}&=&D\Lambda^{(-1)}, \rightCoord{}
\rightCoord{}}{0mm}{4}{6}{
  \bar D D\Psi^{(-\frac{1}{2})}&=&0,\\
  \delta\Psi^{(-\frac{1}{2})}&=&D\Lambda^{(-1)}, 
}{1}\coordE{}\end{eqnarray}
and the solution has a similar form to \myHighlight{$(\ref{tachyon})$}\coordHE{} as
\begin{equation}\coord{}\boxEquation{
 \Psi^{(-\frac{1}{2})}=\theta\phi^{(0)}(p,j=0).\label{graviton}
}{
 \Psi^{(-\frac{1}{2})}=\theta\phi^{(0)}(p,j=0).}{ecuacion}\coordE{}\end{equation}
The massless boson \myHighlight{$\phi^{(0)}$}\coordHE{} has no fermionic partner
and is identified with {\it graviton like} state in 
Ref.~\citen{HS2}.

Next we consider two massive cases \myHighlight{$N=\eta, 1-\eta$}\coordHE{}.
General states at the level \myHighlight{$N=\eta$}\coordHE{} are expanded
by three Fock states as
\begin{equation}\coord{}\boxEquation{
 |V\rangle=
(\Pi^*_Z)_{-\eta}|l\rangle\Psi^{(l-\frac{1}{2})}(\boldsymbol p,\boldsymbol\theta)
+(d_{++})_{-\eta}|l\rangle\Phi^{(l)}(\boldsymbol p,\boldsymbol\theta)
+\theta^{--}_{-\eta}|l\rangle\Xi^{(l)}(\boldsymbol p,\boldsymbol\theta).
}{
 |V\rangle=
(\Pi^*_Z)_{-\eta}|l\rangle\Psi^{(l-\frac{1}{2})}(\boldsymbol p,\boldsymbol\theta)
+(d_{++})_{-\eta}|l\rangle\Phi^{(l)}(\boldsymbol p,\boldsymbol\theta)
+\theta^{--}_{-\eta}|l\rangle\Xi^{(l)}(\boldsymbol p,\boldsymbol\theta).
}{ecuacion}\coordE{}\end{equation}
Since we take a supercovariant basis created by 
\myHighlight{$((\Pi^*_Z)_{-\eta},(d_{++})_{-\eta},\theta^{--}_{-\eta})$}\coordHE{}, 
the coefficient fields are superfields, {\it i.e.} their 
supersymmetry transformations are generated by 
supercharges (\ref{susyforcomp}).
The equations of motion for the \myHighlight{$l=1$}\coordHE{} case can be written as
 \begin{align}\coord{}\boxAlignEqnarray{
  \leftCoord{}D\left(\bar D\Xi^{(1)}+\eta\Psi^{(\frac{\leftCoord{}1}{\rightCoord{}2})}\right)&=0,
\leftCoord{}\nonumber\rightCoord{}\\
  \leftCoord{}\bar D\left(\Xi^{(1)}-(p+\eta)D\Psi^{(\frac{\leftCoord{}1}{\rightCoord{}2})}\right)+
\leftCoord{}\left((p+\eta)j+\eta\right)\Psi^{(\frac{\leftCoord{}1}{\rightCoord{}2})}&=0,
 }{0mm}{7}{5}{
  D\left(\bar D\Xi^{(1)}+\eta\Psi^{(\frac{1}{2})}\right)&=0,
\nonumber\\
  \bar D\left(\Xi^{(1)}-(p+\eta)D\Psi^{(\frac{1}{2})}\right)+
\left((p+\eta)j+\eta\right)\Psi^{(\frac{1}{2})}&=0,
 }{0}\coordE{}\end{align}
with the gauge transformations
 \begin{align}\coord{}\boxAlignEqnarray{
  \leftCoord{}\delta\Psi^{(\frac{\leftCoord{}1}{\rightCoord{}2})}&=\bar D\Lambda^{(1)},\nonumber\rightCoord{}\\
  \leftCoord{}\delta\Phi^{(1)}&=\Sigma^{(1)},\nonumber\rightCoord{}\\
  \leftCoord{}\delta\Xi^{(1)}&=-\eta\Lambda^{(1)}.
}{0mm}{4}{4}{
  \delta\Psi^{(\frac{1}{2})}&=\bar D\Lambda^{(1)},\nonumber\\
  \delta\Phi^{(1)}&=\Sigma^{(1)},\nonumber\\
  \delta\Xi^{(1)}&=-\eta\Lambda^{(1)}.
}{0}\coordE{}\end{align}
Taking \myHighlight{$\Phi^{(1)}=\Xi^{(1)}=0$}\coordHE{} gauge, 
the physical state is described by an anti-chiral superfield 
obeying \myHighlight{$D\Psi^{(\frac{1}{2})}=0$}\coordHE{} and the on-shell condition
\myHighlight{$((p+\eta)j+\eta)\Psi^{(\frac{1}{2})}=0$}\coordHE{}.
The anti-chiral superfield has the explicit form
\begin{equation}\coord{}\boxEquation{
 \Psi^{(\frac{1}{2})}=\psi^{(\frac{1}{2})}
+\bar\theta\bar\phi^{(0)}
-\frac{1}{2}\theta\bar\theta j\psi^{(\frac{1}{2})},
}{
 \Psi^{(\frac{1}{2})}=\psi^{(\frac{1}{2})}
+\bar\theta\bar\phi^{(0)}
-\frac{1}{2}\theta\bar\theta j\psi^{(\frac{1}{2})},
}{ecuacion}\coordE{}\end{equation}
containing one boson \myHighlight{$\bar\phi^{(0)}$}\coordHE{} and 
one fermion \myHighlight{$\psi^{(\frac{1}{2})}$}\coordHE{}.

For the case of \myHighlight{$l=0$}\coordHE{}, the equations of motion
\begin{align}\coord{}\boxAlignEqnarray{
 \leftCoord{}\bar D\left(\Xi^{(0)}-(p+\eta)D\Psi^{(-\frac{\leftCoord{}1}{\rightCoord{}2})}\right)&=0,
\leftCoord{}\nonumber\rightCoord{}\\
%%%%%%%%%%%%%%%%%%%%%%%%%%
\leftCoord{}D\left(\Xi^{(0)}-(p+\eta)\Phi^{(0)}\right)&=0,\nonumber\rightCoord{}\\
%%%%%%%%%%%%%%%%%%%%%%%%%%
\leftCoord{}(p+\eta)\bar DD\Phi^{(0)}+\eta D\Psi^{(-\frac{\leftCoord{}1}{\rightCoord{}2})}
\leftCoord{}+\left[D,\bar D\right]\Xi^{(0)}&=\left((p+\eta)j+\eta\right)\Phi^{(0)},
\leftCoord{}\nonumber\rightCoord{}\\
%%%%%%%%%%%%%%%%%%%%%%%%%%
\leftCoord{}(p+\eta)D\left(\bar D\Xi^{(0)}+\eta\Psi^{(-\frac{\leftCoord{}1}{\rightCoord{}2})}\right)&=
\leftCoord{}\left((p+\eta)j+\eta\right)\Xi^{(0)},
}{0mm}{3}{2}{theequation}{0}\coordE{}\end{align}
and the gauge transformations
\begin{align}\coord{}\boxAlignEqnarray{
  \leftCoord{}\delta\Psi^{(-\frac{\leftCoord{}1}{\rightCoord{}2})}&=\Sigma^{(-\frac{\leftCoord{}1}{\rightCoord{}2})}
\leftCoord{}+D\Lambda^{(-1)},\nonumber\rightCoord{}\\
%%%%%%%%%%%%%%%%%%%%%%%%%%%%%%%%%%
\leftCoord{}\delta\Phi^{(0)}&=D\Sigma^{(-\frac{\leftCoord{}1}{\rightCoord{}2})},\nonumber\rightCoord{}\\
%%%%%%%%%%%%%%%%%%%%%%%%%%%%%%%%%%
\leftCoord{}\delta\Xi^{(0)}&=(p+\eta)D\Sigma^{(-\frac{\leftCoord{}1}{\rightCoord{}2})},
}{0mm}{2}{2}{theequation}{0}\coordE{}\end{align}
can be solved by taking \myHighlight{$\Psi^{(-\frac{1}{2})}=0$}\coordHE{} gauge as
\begin{equation}\coord{}\boxEquation{
 \Xi^{(0)}=(p+\eta)\frac{1}{j}\bar DD\Phi^{(0)}.
}{
 \Xi^{(0)}=(p+\eta)\frac{1}{j}\bar DD\Phi^{(0)}.
}{ecuacion}\coordE{}\end{equation}
The physical state is identified with an unconstrained 
superfield \myHighlight{$\Phi^{(0)}$}\coordHE{} obeying the on-shell condition
\myHighlight{$\left((p+\eta)j+\eta\right)\Phi^{(0)}=0$}\coordHE{}. 

We can easily repeat the analysis to study physical spectrum 
at level \myHighlight{$N=1-\eta$}\coordHE{}. The states at this level are generally
\begin{equation}\coord{}\boxEquation{
 |V\rangle=(\Pi_Z)_{-1+\eta}|l\rangle
\Psi^{(l-\frac{1}{2})}(\boldsymbol p,\boldsymbol\theta)
+(d_{--})_{-1+\eta}|l\rangle\Phi^{(l-1)}(\boldsymbol p,\boldsymbol\theta)
+\theta^{++}_{-1+\eta}|l\rangle\Xi^{(l-1)}(\boldsymbol p,\boldsymbol\theta).
}{
 |V\rangle=(\Pi_Z)_{-1+\eta}|l\rangle
\Psi^{(l-\frac{1}{2})}(\boldsymbol p,\boldsymbol\theta)
+(d_{--})_{-1+\eta}|l\rangle\Phi^{(l-1)}(\boldsymbol p,\boldsymbol\theta)
+\theta^{++}_{-1+\eta}|l\rangle\Xi^{(l-1)}(\boldsymbol p,\boldsymbol\theta).
}{ecuacion}\coordE{}\end{equation}
For \myHighlight{$l=1$}\coordHE{}, the equations of motion and the gauge
transformations are given by
 \begin{align}\coord{}\boxAlignEqnarray{
  \leftCoord{}D\left(\Xi^{(0)}-(p+\eta)\bar D\Psi^{(\frac{\leftCoord{}1}{\rightCoord{}2})}\right)&=0,
\leftCoord{}\nonumber\rightCoord{}\\
%%%%%%%%%%%%%%%%%%
  \leftCoord{}(p+\eta)D\bar D\Phi^{(0)}
\leftCoord{}+(1-\eta)\bar D\Psi^{(\frac{\leftCoord{}1}{\rightCoord{}2})}
\leftCoord{}-\left[D,\bar D\right]\Xi^{(0)}&=
\leftCoord{}\left((p+\eta)j+1-\eta\right)\Phi^{(0)},\nonumber\rightCoord{}\\
%%%%%%%%%%%%%%%%%%
  \leftCoord{}(p+\eta)\bar D\left(D\Xi^{(0)}+(1-\eta)\Psi^{(\frac{\leftCoord{}1}{\rightCoord{}2})}\right)
&\leftCoord{}=\left((p+\eta)j+1-\eta\right)\Xi^{(0)},\nonumber\rightCoord{}\\
%%%%%%%%%%%%%%%%%%
\leftCoord{}\bar D\left(\Xi^{(0)}-(p+\eta)\Phi^{(0)}\right)&=0,
 }{0mm}{1}{1}{theequation}{0}\coordE{}\end{align}
and
 \begin{align}\coord{}\boxAlignEqnarray{
  \leftCoord{}\delta\Psi^{(\frac{\leftCoord{}1}{\rightCoord{}2})}&=
     \leftCoord{}\Sigma^{(\frac{\leftCoord{}1}{\rightCoord{}2})}+\bar D\Lambda^{(1)},\nonumber\rightCoord{}\\
%%%%%%%%%%%%%%%%%%
  \leftCoord{}\delta\Phi^{(0)}&=
     \leftCoord{}\bar D\Sigma^{(\frac{\leftCoord{}1}{\rightCoord{}2})},\nonumber\rightCoord{}\\
%%%%%%%%%%%%%%%%%%
  \leftCoord{}\delta\Xi^{(0)}&=
     \leftCoord{}(p+\eta)\bar D\Sigma^{(\frac{\leftCoord{}1}{\rightCoord{}2})},
 }{0mm}{3}{2}{theequation}{0}\coordE{}\end{align}
respectively. The physical state is given by an unconstrained
superfield \myHighlight{$\Phi^{(0)}$}\coordHE{} obeying
\myHighlight{$\left((p+\eta)j+1-\eta\right)\Phi^{(0)}=0$}\coordHE{}.
The superfield \myHighlight{$\Psi^{(\frac{1}{2})}$}\coordHE{} can be gauged away and 
\myHighlight{$\Xi^{(0)}$}\coordHE{} is solved by \myHighlight{$\Phi^{(0)}$}\coordHE{}.

We can also solve the physical state conditions for \myHighlight{$l=0$}\coordHE{}
 \begin{align}\coord{}\boxAlignEqnarray{
  \leftCoord{}D\left(\Xi^{(-1)}-(p+\eta)\bar D\Psi^{(-\frac{\leftCoord{}1}{\rightCoord{}2})}\right)
   \leftCoord{}+\left((p+\eta)j+1-\eta\right)\Psi^{(-\frac{\leftCoord{}1}{\rightCoord{}2})}&=0,\nonumber\rightCoord{}\\
%%%%%%%%%%%%%%%%%%%%%%%%
  \leftCoord{}\bar D\left(D\Xi^{(-1)}+(1-\eta)\Psi^{(-\frac{\leftCoord{}1}{\rightCoord{}2})}\right)&=0,
}{0mm}{2}{2}{theequation}{0}\coordE{}\end{align}
and
\begin{align}\coord{}\boxAlignEqnarray{
  \leftCoord{}\delta\Psi^{(-\frac{\leftCoord{}1}{\rightCoord{}2})}&=D\Lambda^{(-1)},\nonumber\rightCoord{}\\
%%%%%%%%%%%%%%%%%%%%%%%
  \leftCoord{}\delta\Phi^{(-1)}&=\Sigma^{(-1)},\nonumber\rightCoord{}\\
%%%%%%%%%%%%%%%%%%%%%%%
  \leftCoord{}\delta\Xi^{(-1)}&=-(1-\eta)\Lambda^{(-1)},
}{0mm}{1}{1}{theequation}{0}\coordE{}\end{align}
by a chiral superfield \myHighlight{$\Psi^{(-\frac{1}{2})}$}\coordHE{} obeying
\begin{align}\coord{}\boxAlignEqnarray{
  \leftCoord{}\bar D\Psi^{(-\frac{\leftCoord{}1}{\rightCoord{}2})}&=0,\nonumber\rightCoord{}\\
%%%%%%%%%%%%%%%%%%%%%%%
  \leftCoord{}\left((p+\eta)j+1-\eta\right)\Psi^{(-\frac{\leftCoord{}1}{\rightCoord{}2})}&=0. 
}{0mm}{2}{2}{theequation}{0}\coordE{}\end{align}
The explicit form of the chiral superfield is 
\begin{equation}\coord{}\boxEquation{
 \Psi^{(-\frac{1}{2})}=\psi^{(-\frac{1}{2})}+\theta\phi^{(0)}+
\frac{1}{2}\theta\bar\theta j\psi^{(-\frac{1}{2})}.
}{
 \Psi^{(-\frac{1}{2})}=\psi^{(-\frac{1}{2})}+\theta\phi^{(0)}+
\frac{1}{2}\theta\bar\theta j\psi^{(-\frac{1}{2})}.
}{ecuacion}\coordE{}\end{equation}

In short, the physical spectrum at these massive levels
contains two types of multiplets, (anti-)chiral and
unconstrained. The latter is reducible and
decomposes into two (chiral and anti-chiral) multiplets.


\subsection{physical states in the long string sector}

In the long string sector, we examine only the ground state 
with \myHighlight{$\Delta=Q=0$}\coordHE{}. The \myHighlight{$U(1)$}\coordHE{} charge condition (\ref{u1}) 
leads \myHighlight{$l=0$}\coordHE{} and then the state is given by
\begin{equation}\coord{}\boxEquation{
 |V\rangle=|\boldsymbol p,\boldsymbol q,
\boldsymbol\theta,\boldsymbol{\tilde\theta},0;0,0\rangle 
V^{(0)}(\boldsymbol p,\boldsymbol q,
\boldsymbol\theta,\boldsymbol{\tilde\theta}).
}{
 |V\rangle=|\boldsymbol p,\boldsymbol q,
\boldsymbol\theta,\boldsymbol{\tilde\theta},0;0,0\rangle 
V^{(0)}(\boldsymbol p,\boldsymbol q,
\boldsymbol\theta,\boldsymbol{\tilde\theta}).
}{ecuacion}\coordE{}\end{equation}
The physical state conditions become
\begin{subequations}\label{longcondition}
 \begin{align}\coord{}\boxAlignEqnarray{
&\leftCoord{}\left(D\bar D\bar{\tilde D}\tilde D-
\leftCoord{}\tilde D\bar D\bar{\tilde D}D\right)V^{(0)}=0,\label{max}\rightCoord{}\\
&\leftCoord{}\delta V^{(0)}=\tilde DD\Lambda^{(-1)}+\bar D\bar{\tilde D}\bar\Lambda^{(1)},
 }{0mm}{3}{2}{
&\left(D\bar D\bar{\tilde D}\tilde D-
\tilde D\bar D\bar{\tilde D}D\right)V^{(0)}=0,\\
&\delta V^{(0)}=\tilde DD\Lambda^{(-1)}+\bar D\bar{\tilde D}\bar\Lambda^{(1)},
 }{0}\coordE{}\end{align}
\end{subequations}
where the supercovariant derivatives in this sector have the form
\begin{alignat}{2}
 D&=\frac{\partial}{\partial\theta}+\frac{1}{2}j\bar\theta
-\frac{1}{2}q^*\bar{\tilde\theta},&\qquad
 \bar D&=\frac{\partial}{\partial\bar\theta}+\frac{1}{2}j\theta
-\frac{1}{2}q\tilde\theta,\nonumber\\
%%%%%%%%%%%%%%%
 \tilde D&=\frac{\partial}{\partial\tilde\theta}
-\frac{1}{2}p\bar{\tilde\theta}
-\frac{1}{2}q\bar\theta,&\qquad
 \bar{\tilde D}&=\frac{\partial}{\partial\bar{\tilde\theta}}
-\frac{1}{2}p\tilde\theta
-\frac{1}{2}q^*\theta.
\end{alignat}
These conditions (\ref{longcondition}) are essentially
the same with those for the four-dimensional vector
multiplet. If we take the WZ gauge
 \begin{align}\coord{}\boxAlignEqnarray{
  \leftCoord{}V=&\frac{\leftCoord{}1}{\rightCoord{}2}\tilde\theta\bar{\tilde\theta}A_+
   \leftCoord{}-\frac{\leftCoord{}1}{\rightCoord{}2}\theta\bar\theta A_-
   \leftCoord{}+\frac{\leftCoord{}1}{\rightCoord{}2}\tilde\theta\bar\theta A
   \leftCoord{}+\frac{\leftCoord{}1}{\rightCoord{}2}\theta\bar{\tilde\theta}A^*\nonumber\rightCoord{}\\
  \leftCoord{}&
   \leftCoord{}+\theta\bar\theta\bar{\tilde\theta}\tilde\lambda
   \leftCoord{}-\tilde\theta\bar\theta\bar{\tilde\theta}\lambda
   \leftCoord{}+\theta\tilde\theta\bar\theta\bar{\tilde\lambda}
   \leftCoord{}-\theta\tilde\theta\bar{\tilde\theta}\bar\lambda
   \leftCoord{}+\theta\tilde\theta\bar\theta\bar{\tilde\theta}{\cal D},
 }{0mm}{14}{6}{
  V=&\frac{1}{2}\tilde\theta\bar{\tilde\theta}A_+
   -\frac{1}{2}\theta\bar\theta A_-
   +\frac{1}{2}\tilde\theta\bar\theta A
   +\frac{1}{2}\theta\bar{\tilde\theta}A^*\nonumber\\
  &
   +\theta\bar\theta\bar{\tilde\theta}\tilde\lambda
   -\tilde\theta\bar\theta\bar{\tilde\theta}\lambda
   +\theta\tilde\theta\bar\theta\bar{\tilde\lambda}
   -\theta\tilde\theta\bar{\tilde\theta}\bar\lambda
   +\theta\tilde\theta\bar\theta\bar{\tilde\theta}{\cal D},
 }{0}\coordE{}\end{align}
the equations of motion (\ref{max}) leads 
the Maxwell equations
 \begin{align}\coord{}\boxAlignEqnarray{
&\leftCoord{}\frac{\leftCoord{}1}{\rightCoord{}4}j(pA_-+jA_++qA^*+q^*A)
\leftCoord{}-\frac{\leftCoord{}1}{\rightCoord{}2}(pj+qq^*)A_-=0,\nonumber\rightCoord{}\\
%%%%%%%%%%%%%%%%%%
&\leftCoord{}\frac{\leftCoord{}1}{\rightCoord{}4}p(pA_-+jA_++qA^*+q^*A)
\leftCoord{}-\frac{\leftCoord{}1}{\rightCoord{}2}(pj+qq^*)A_+=0,\nonumber\rightCoord{}\\
%%%%%%%%%%%%%%%%%%
&\leftCoord{}\frac{\leftCoord{}1}{\rightCoord{}4}q^*(pA_-+jA_++qA^*+q^*A)
\leftCoord{}-\frac{\leftCoord{}1}{\rightCoord{}2}(pj+qq^*)A^*=0,\nonumber\rightCoord{}\\
%%%%%%%%%%%%%%%%%%
&\leftCoord{}\frac{\leftCoord{}1}{\rightCoord{}4}q(pA_-+jA_++qA^*+q^*A)
\leftCoord{}-\frac{\leftCoord{}1}{\rightCoord{}2}(pj+qq^*)A=0,
}{0mm}{4}{3}{theequation}{0}\coordE{}\end{align}
the massless Dirac equations
\begin{align}\coord{}\boxAlignEqnarray{
%%%%%%%%%%%%%%%%%%
\leftCoord{}(q^*\bar{\tilde\lambda}-j\bar\lambda)&=0,\qquad
\leftCoord{}(p\bar{\tilde\lambda}+q\bar\lambda)=0,\nonumber\rightCoord{}\\
%%%%%%%%%%%%%%%%%%
\leftCoord{}(q\tilde\lambda-j\lambda)&=0,\qquad
\leftCoord{}(p\tilde\lambda+q^*\lambda)=0,
}{0mm}{2}{1}{theequation}{0}\coordE{}\end{align}
and \myHighlight{${\cal D}=0$}\coordHE{} for auxiliary field.
The massless spectrum of the long string is thus 
the vector multiplet in the four-dimensional 
{\it free-field space} \myHighlight{$(X^\pm,Z,Z^*)$}\coordHE{}.

\section{Summary and Discussions}\label{summary}

In this paper, we have studied four-dimensional superstrings
in the NS-NS plane-wave backgrounds by using the hybrid 
formalism. This description of the superstring has been 
obtained by a field redefinition of the worldsheet 
fields in the super-NW model.\cite{HS2}
Since we have adopted a weak GSO projection restricting only
the total \myHighlight{$U(1)_R$}\coordHE{} charge to be integer,
the model has enhanced supersymmetry which is manifest 
in the hybrid formalism. The Hilbert space consists of 
two sectors describing the short and the long strings,
and including all the spectrally flowed representations of 
type II and I, respectively.\cite{KK,KP} 
Then we have studied physical states to find boson-fermion 
asymmetry in the massless spectrum of the short string. 
There are two massless bosons, called tachyon like 
and graviton like in Ref.~\citen{HS2}, but no fermionic 
partners. We have also identified massive physical 
states at level \myHighlight{$N=\eta, 1-\eta$}\coordHE{} in the short string sector
and massless physical states in the long string sector. 
The massless physical spectrum of the long string is 
the vector multiplet freely propagating 
in the four-dimensional space \myHighlight{$(X^\pm,Z,Z^*)$}\coordHE{}.

The massive physical states obtained by solving physical
state conditions are also created by acting the DDF operators
\begin{align}\coord{}\boxAlignEqnarray{
   {\rightCoord{}\leftCoord{}\cal P}_n=&\oint\dz 
\leftCoord{}e^{i\left(\frac{\leftCoord{}n+\eta}{\rightCoord{}p+\eta}\right)X^+}
\leftCoord{}\left(i\partial Z-\left(\frac{\leftCoord{}n+\eta}{\rightCoord{}p+\eta}\right)
\leftCoord{}\theta^{+-}p_{\rightCoord{}--}\right),\nonumber\rightCoord{}\\
%%%%%%%%%%%%%%%%%%%%%%%%%%
 {\rightCoord{}\leftCoord{}\cal P}^*_n=&\oint\dz 
\leftCoord{}e^{i\left(\frac{\leftCoord{}n-\eta}{\rightCoord{}p+\eta}\right)X^+}
\leftCoord{}\left(i\partial Z^*-\left(\frac{\leftCoord{}n-\eta}{\rightCoord{}p+\eta}\right)
\leftCoord{}\theta^{-+}p_{\rightCoord{}++}\right),\nonumber\rightCoord{}\\
%%%%%%%%%%%%%%%%%%%%%%%%%%
{\rightCoord{}\leftCoord{}\cal Q}^{++}_n=&\oint\dz 
\leftCoord{}e^{i\left(\frac{\leftCoord{}n+\eta}{\rightCoord{}p+\eta}\right)X^+}
\leftCoord{}\Bigg(p_{\rightCoord{}--}+\frac{\leftCoord{}1}{\rightCoord{}2}i\partial X^+\theta^{++}\nonumber\rightCoord{}\\
&\leftCoord{}\hspace{2cm}
\leftCoord{}+\frac{\leftCoord{}1}{\rightCoord{}2}\left(i\partial Z
\leftCoord{}-\left(\frac{\leftCoord{}n+\eta}{\rightCoord{}p+\eta}\right)\theta^{+-}p_{\rightCoord{}--}\right)\theta^{-+}
\leftCoord{}+\frac{\leftCoord{}1}{\rightCoord{}8}\partial(\theta^{-+}\theta^{++})\theta^{+-}\Bigg),
\leftCoord{}\nonumber\rightCoord{}\\
%%%%%%%%%%%%%%%%%%%%%%%%%%
{\rightCoord{}\leftCoord{}\cal Q}^{--}_n=&\oint\dz 
\leftCoord{}e^{i\left(\frac{\leftCoord{}n-\eta}{\rightCoord{}p+\eta}\right)X^+}
\leftCoord{}\Bigg(p_{\rightCoord{}++}+\frac{\leftCoord{}1}{\rightCoord{}2}i\partial X^+\theta^{--}\nonumber\rightCoord{}\\
&\leftCoord{}\hspace{2cm}
\leftCoord{}+\frac{\leftCoord{}1}{\rightCoord{}2}\left(i\partial Z^*
\leftCoord{}-\left(\frac{\leftCoord{}n-\eta}{\rightCoord{}p+\eta}\right)\theta^{-+}p_{\rightCoord{}++}\right)\theta^{+-}
\leftCoord{}-\frac{\leftCoord{}1}{\rightCoord{}8}\partial(\theta^{--}\theta^{+-})\theta^{-+}\Bigg),
}{0mm}{12}{10}{theequation}{0}\coordE{}\end{align}
on the massless physical states.\cite{HS2} 
They contain \myHighlight{${\cal P}={\cal P}_p,\ {\cal P}^*={\cal P}^*_{-p},\
{\cal Q}^{\pm\pm}={\cal Q}^{\pm\pm}_{\pm p}$}\coordHE{}
and generate an affine extension of the supersymmetry algebra (\ref{alg}):
\begin{alignat}{2}
\left[{\cal J},{\cal P}_n\right]
&=\left(\frac{n+\eta}{p+\eta}\right){\cal P}_n,&\qquad
\left[{\cal J},{\cal P}^*_n\right]
&=\left(\frac{n-\eta}{p+\eta}\right){\cal P}^*_n,\nonumber\\
%%%%%%%%%%%%%%%%%%%%%%%%%%%%%%%%%%%%%%%%%%%%%%%
 \left[{\cal P}_n,{\cal P}^*_m\right]
&=\left(\frac{n+\eta}{p+\eta}\right){\cal F}\delta_{n+m,0},&\qquad
\left[{\cal J},{\cal Q}^{\pm\mp}\right]&=0,
\nonumber\\
%%%%%%%%%%%%%%%%%%%%%%%%%%%%%%%%%%%%%%%%%%%%%%%
\left[{\cal J},{\cal Q}^{++}_n\right]
&=\left(\frac{n+\eta}{p+\eta}\right){\cal Q}^{++}_n,&\qquad
\left[{\cal J},{\cal Q}^{--}_n\right]
&=\left(\frac{n-\eta}{p+\eta}\right){\cal Q}^{--}_n,\nonumber\\
%%%%%%%%%%%%%%%%%%%%%%%%%%%%%%%%%%%%%%%%%%%%%%%%
\left[{\cal Q}^{-+},{\cal P}_n\right]
&=-\left(\frac{n+\eta}{p+\eta}\right){\cal Q}^{++}_n,&\qquad
\left[{\cal Q}^{+-},{\cal P}^*_n\right]
&=-\left(\frac{n-\eta}{p+\eta}\right){\cal Q}^{--}_n,\nonumber\\
%%%%%%%%%%%%%%%%%%%%%%%%%%%%%%%%%%%%%%%%%%%%%%%%
 \left\{{\cal Q}^{++}_n,{\cal Q}^{--}_m\right\}&=
{\cal F}\delta_{n+m,0},&\qquad
 \left\{{\cal Q^{+-}},{\cal Q}^{-+}\right\}&={\cal J},\nonumber\\
%%%%%%%%%%%%%%%%%%%%%%%%%%%%%%%%%%%%%%%%%%%%%%%%
\left\{{\cal Q}^{+-},{\cal Q}^{++}_n\right\}&={\cal P}_n,&\qquad
\left\{{\cal Q}^{-+},{\cal Q}^{--}_n\right\}&={\cal P}^*_n,
\end{alignat}
which provides the enhanced space-time symmetry. 
This symmetry is obtained by taking the Penrose limit of 
\myHighlight{${\cal N}=2$}\coordHE{} superconformal symmetry being the isometry of the
\myHighlight{$AdS_3\times S^1$}\coordHE{} background.\cite{HS2}
It is interesting to trace this limit by studying
hybrid superstrings in the \myHighlight{$AdS_3\times S^1$}\coordHE{},
which is under investigation and will be reported elsewhere.\cite{K}

It is also interesting to understand general aspects of 
the holographic duality in the plane wave backgrounds,\cite{HS2,KP} 
which requires further consideration. We hope that the manifest 
supersymmetry in the hybrid formalism shed new light on 
future development.

\section*{Acknowledgements}
The author would like to thank Y.~Hikida and Y.~Sugawara
for valuable discussions. The work is supported in part by 
the Grant-in-Aid for Scientific Research No.11640276
from Japan Society for Promotion of Science and
No.13135213 from the Ministry of Education, Science, 
Sports and Culture of Japan.

\begin{thebibliography}{99}
%%%%%%%%%%%%%%%%%%%%%%%%%%%%%%%%%%%%%%%%%%%%%%%%%%%%%%%%%%%%%
%%%%%%%%%%%%%%%%%%%%%%%%%%%%%%%%%%%%%%%%%%%%%%%%%%%%%%%%%%%%%
\bibitem{BMN}
D.~Berenstein, J.~Maldacena and H.~Nastase, \JHEP{0240,2002,013}, 
hep-th/0202021.

\bibitem{M}
R.R~Metsaev, \NPB{625,2002,70}, hep-th/0112044. 

\bibitem{BM}
N.~Berkovits and J.~Maldacena, \JHEP{0210,2002,059}, hep-th/0208092. 

\bibitem{HS1}
Y.~Hikida and Y.~Sugawara, \JHEP{0206,2002,037}, hep-th/0205200.

\bibitem{HS2}
Y.~Hikida and Y.~Sugawara, \JHEP{0210,2002,067}, hep-th/0207124.

\bibitem{B}
N.~Berkovits, \NPB{431,1994,258}, hep-th/9404162;\\
  \lq\lq\textsl{ A New Description of the Superstring}'', 
in Proceedings of Jorge Swieca Summer School (1995) 418, hep-th/9604123.

\bibitem{BV}
N.~Berkovits and C.~Vafa, \NPB{433,1995,123}, hep-th/9407190.

\bibitem{compact}
N.~Berkovits, C.~Vafa and E.~Witten, \JHEP{9903,1999,018},
hep-th/9902098.\\
N.~Berkovits, M.~Bershadsky, T.~Hauer, S.~Zhukov and B.~Zwiebach,
\NPB{567,2000,61},hep-th/9907200.\\
N.~Berkovits, S.~Gukov and B.C.~Vallilo, \NPB{614,2001,195}, 
hep-th/0107140.\\
K.~Ito, \JL{Mod. Phys. Lett.,A14,1999,2379}, hep-th/9910047.\\
K.~Ito and H.~Kunitomo, \PLB{536,2002,327}, hep-th/0204009.

\bibitem{NW}
C.R.~Nappi and E.~Witten, \PRL{71,1993,3751}, hep-th/9310112.

\bibitem{KK}
E.~Kiritsis and C.~Kounnas, \PLB{320,1994,264}, hep-th/9310202.\\
E.~Kiritsis, C.~Kounnas and D.~L\"ust, \PLB{331,1994,321},
hep-th/9404114.

\bibitem{KP}
E.~Kiritsis and B.~Pioline, \JHEP{0208,2002,048}, hep-th/0204004.

\bibitem{MO}
J.M.~Maldacena and H.~Ooguri, \IJMP{42,2001,2929}, hep-th/0001053.

\bibitem{FMS}
D.~Friedan, E.J.~Martinec and S.H. Shenker, \NPB{271,1986,93}.

\bibitem{Bherm}
N.~Berkovits, \PRL{77,1996,2891}, hep-th/9604121.

\bibitem{K}
H.~Kunitomo, work in progress.

\end{thebibliography}

\end{document}






\bye
