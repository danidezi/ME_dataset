
\documentclass[a4paper,12pt,a4]{article}
\usepackage{useful_macros}
\begin{document}
\setlength{\textheight}{9.4in}
\setlength{\topmargin}{-0.6in}
\setlength{\oddsidemargin}{0.2in}
\setlength{\evensidemargin}{0.5in}
\renewcommand{\thefootnote}{\fnsymbol{footnote}}
\begin{center}
{\Large {\bf Supersymmetry in Quantum Mechanics of Colored Particles}}
\vspace{0.8cm}

{Sh. Mamedov,\raisebox{0.8ex}{\small a,b}\footnote[1]
{ Email:shahin@physik.uni-kl.de}
Jian--zu Zhang\raisebox{0.8ex}{\small a,c}\footnote[2]
{Email:jzzhang@physik.uni-kl.de}
and V. Zhukovskii\raisebox{0.8ex}{\small d}\footnote[3]
{Email:th180@phys.msu.su}\\

\raisebox{0.8 ex}{\small a)}{\it Department of Physics, University
of Kaiserslautern, 67653 Kaiserslautern, Germany}

\raisebox{0.8 ex}{\small b)}{\it  High Energy Physics Laboratory, Baku State
University, 23 Z. Khalilov, 370148 Baku, Azerbaijan}


\raisebox{0.8 ex}{\small c)}{\it Institute for Theoretical Physics, Box 316,
East China University of Science and Technology, Shanghai 200237, P.R. China}


\raisebox{0.8 ex}{\small d)}{\it Faculty of Physics,
Department of Theoretical Physics, Moscow
State University, 119899 Moscow, Russia}}
\end{center}
\vspace{0.3cm}

{\centerline {\bf Abstract}}
\noindent

The role of supercharge operators is studied in the case
of a Dirac particle moving in a constant
chromomagnetic field. The Hamiltonian is factorised and the ground state
wave function in the case of unbroken supersymmetry
is determined.






\vspace{2cm}
The study of supersymmetric quantum mechanics began with
 ref.\cite{1} and has been  applied to
various problems, including in particular the case of
charged abelian particles\cite{2,3,4,5,6,7}.
In refs.\cite{2} and \cite{3} the supersymmetry of
the equation of motion of the electron in
a magnetic field was investigated. The related
problem of supersymmetry of the Dirac equation
for a colored particle in an external
chromonagnetic field has been considered in
ref.\cite{8}. In the following the investigation of the
latter is continued with the aim to explore the significance of
the supercharge operators in this case of
a chromomagnetic field, to factorise the Hamiltonian
and to determine the ground state
wave function.


It is known \cite{9,10} that a constant color background
can be obtained from two types of vector potentials
\myHighlight{$A_{\mu}$}\coordHE{}, i.e. linearly growing abelian potentials and non-commuting
vector potentials.  Here we consider the
simplest case of the latter type which implies
the non-commuting potentials
\begin{eqnarray}\coord{}\boxAlignEqnarray{\leftCoord{}
A_{\mu}&=&A^a_{\mu}T^a=A^a_{\mu}\tau^a/2, 
\leftCoord{}\;\;a=1,2,3, \;\; \mu = 0,1,2,3,\nonumber\rightCoord{}\\\leftCoord{}
 A^1_{\mu}&=&(0,\sqrt{\lambda_1},0,0), A^2_{\mu}=(0,0,\sqrt{\lambda_2},0),
A^3_{\mu}=0.\rightCoord{}
\label{1}
\rightCoord{}}{0mm}{3}{4}{
A_{\mu}&=&A^a_{\mu}T^a=A^a_{\mu}\tau^a/2, 
\;\;a=1,2,3, \;\; \mu = 0,1,2,3,\\
 A^1_{\mu}&=&(0,\sqrt{\lambda_1},0,0), A^2_{\mu}=(0,0,\sqrt{\lambda_2},0),
A^3_{\mu}=0.
}{1}\coordE{}\end{eqnarray}
with \myHighlight{$\sqrt{\lambda_1},\sqrt{\lambda_2}$}\coordHE{} constant.
Latin indices are \myHighlight{$SU(2)$}\coordHE{} color indices and Greek indices
are Lorentz ones. The Pauli matrices \myHighlight{$\tau_i$}\coordHE{} are the generators of the
\myHighlight{$SU(2)$}\coordHE{} color group. The potentials (\ref{1})
determine a constant homogeneous chromomagnetic field
along the third axes of both ordinary and color spaces, 
i.e.
\begin{equation}\coord{}\boxEquation{
F^3_{12} = g\epsilon^{ab3}A^a_1A^b_2 = g\sqrt{\lambda_1
\lambda_2} = H^3_z
\label{2}
}{
F^3_{12} = g\epsilon^{ab3}A^a_1A^b_2 = g\sqrt{\lambda_1
\lambda_2} = H^3_z
}{ecuacion}\coordE{}\end{equation}
where \myHighlight{$g$}\coordHE{} is the constant of color interaction.  
With analogous non-commuting
potentials a constant field could be
given for any chromomagnetic or chromoelectric
component.  

The Dirac equation for a colored
particle in the
chromomagnetic field (\ref{1}) is 
\begin{equation}\coord{}\boxEquation{
\bigg[\gamma_{\mu}\bigg(p_{\mu}+gA^a_{\mu}\frac{{\tau}^a}
{2}\bigg)-m\bigg]\psi=0.
\label{3}
}{
\bigg[\gamma_{\mu}\bigg(p_{\mu}+gA^a_{\mu}\frac{{\tau}^a}
{2}\bigg)-m\bigg]\psi=0.
}{ecuacion}\coordE{}\end{equation}
In terms of
Majorana spinors \myHighlight{$\phi$}\coordHE{} and \myHighlight{$\chi$}\coordHE{}  \myHighlight{$\psi =\left(
\begin{array}{c}\phi\\\chi \end{array}\right)$}\coordHE{} and
with
\myHighlight{$P_{\mu}=p_{\mu}+gA^a_{\mu}{\tau}^a/2$}\coordHE{}
we rewrite eq.(\ref{3}) as
\begin{eqnarray}\coord{}\boxAlignEqnarray{\leftCoord{}
\sigma_i P_i\chi
&\leftCoord{}=& \bigg(i\frac{\leftCoord{}\partial}{\rightCoord{}\partial t}-m \rightCoord{}
\bigg)\phi,\nonumber\rightCoord{}\\\leftCoord{}
\sigma_i P_i\phi &=& 
\bigg(i\frac{\leftCoord{}\partial}{\rightCoord{}\partial t}+m \rightCoord{}
\bigg)\chi
\label{4}
\rightCoord{}}{0mm}{5}{7}{
\sigma_i P_i\chi
&=& \bigg(i\frac{\partial}{\partial t}-m 
\bigg)\phi,\\
\sigma_i P_i\phi &=& 
\bigg(i\frac{\partial}{\partial t}+m 
\bigg)\chi
}{1}\coordE{}\end{eqnarray}
Here the Pauli matrices \myHighlight{$ \sigma_i$}\coordHE{} describe the particle's spin.
The spinors \myHighlight{$\phi$}\coordHE{} and \myHighlight{$\chi$}\coordHE{}
can easily be seen to satisfy the equations
\begin{eqnarray}\coord{}\boxAlignEqnarray{\leftCoord{}
\bigg({\bf \sigma}\cdot{\bf P}\bigg)^2\phi &=& \rightCoord{}
\leftCoord{}-\bigg(\frac{\leftCoord{}\partial^2}{\rightCoord{}\partial t^2}+m^2\bigg)\phi,\nonumber\rightCoord{}\\\leftCoord{}
\bigg({\bf \sigma}\cdot{\bf P}\bigg)^2\chi &=& \rightCoord{}
\leftCoord{}-\bigg(\frac{\leftCoord{}\partial^2}{\rightCoord{}\partial t^2}+m^2\bigg)\chi
\label{5}
\rightCoord{}}{0mm}{6}{7}{
\bigg({\bf \sigma}\cdot{\bf P}\bigg)^2\phi &=& 
-\bigg(\frac{\partial^2}{\partial t^2}+m^2\bigg)\phi,\\
\bigg({\bf \sigma}\cdot{\bf P}\bigg)^2\chi &=& 
-\bigg(\frac{\partial^2}{\partial t^2}+m^2\bigg)\chi
}{1}\coordE{}\end{eqnarray}
Setting
\begin{equation}\coord{}\boxEquation{
Q={\bf \sigma}\cdot{\bf P}
\label{6}
}{
Q={\bf \sigma}\cdot{\bf P}
}{ecuacion}\coordE{}\end{equation}
and
\begin{equation}\coord{}\boxEquation{
H=-\bigg(\frac{\partial^2}{\partial t^2} + m^2\bigg)
\label{7}
}{
H=-\bigg(\frac{\partial^2}{\partial t^2} + m^2\bigg)
}{ecuacion}\coordE{}\end{equation}
eqs. (\ref{5}) assume the form
\begin{equation}\coord{}\boxEquation{
Q^2\psi = H\psi.
\label{8}
}{
Q^2\psi = H\psi.
}{ecuacion}\coordE{}\end{equation}
The ``Hamiltonian'' (\ref{7}) (cf. ref.\cite{2})
 has eigenvalues \myHighlight{$E^2-m^2$}\coordHE{}, i.e.
\begin{equation}\coord{}\boxEquation{
H\psi = (E^2-m^2)\psi.
\label{9}
}{
H\psi = (E^2-m^2)\psi.
}{ecuacion}\coordE{}\end{equation}
Eqs. (\ref{8}) are reminiscent of Witten's one--measure
supersymmetric quantum mechanics \cite{1}
for \myHighlight{$N$}\coordHE{} supercharges. In
this the supersymmetry algebra
has the form
\begin{equation}\coord{}\boxEquation{
\{Q_i, Q_j\}= 2\delta_{ij}H, \;\;\;
 [H, Q_i]=0, \;\;\; i=1, 2, 3, \cdot\cdot\cdot N
\label{10}
}{
\{Q_i, Q_j\}= 2\delta_{ij}H, \;\;\;
 [H, Q_i]=0, \;\;\; i=1, 2, 3, \cdot\cdot\cdot N
}{ecuacion}\coordE{}\end{equation}
where the curly bracket denotes an anticommutator.
In view of its commutation
with \myHighlight{$H$}\coordHE{} the quantity \myHighlight{${\bf \sigma}\cdot{\bf P}$}\coordHE{} is conserved on
the classical level.
It is known in supersymmetric theories that supercharge operators
\myHighlight{$Q_i$}\coordHE{} obeying (\ref{10}) lead to a degeneracy of the energy.
The relation between the
degeneracy \myHighlight{$n$}\coordHE{} and the number of supercharges \myHighlight{$N$}\coordHE{}
is given by the formula
\begin{equation}\coord{}\boxEquation{
n=2^{[N/2]} 
\label{11}
}{
n=2^{[N/2]} 
}{ecuacion}\coordE{}\end{equation}
where \myHighlight{$[N/2]$}\coordHE{} means {\it integer part of} \myHighlight{$N/2$}\coordHE{}.  
The energy spectrum of a colored particle in the field
(\ref{1}) was  found in \cite{10} and is given by
\begin{equation}\coord{}\boxEquation{
E^2_{1,2}={\bf p}^2 + m^2 +\frac{g^2(\lambda_1+\lambda_2)}{4} 
\pm g\sqrt{\lambda_1 p^2_1+\lambda_2p^2_2+\frac{(H^3_z)^2}{4}}
\label{12}
}{
E^2_{1,2}={\bf p}^2 + m^2 +\frac{g^2(\lambda_1+\lambda_2)}{4} 
\pm g\sqrt{\lambda_1 p^2_1+\lambda_2p^2_2+\frac{(H^3_z)^2}{4}}
}{ecuacion}\coordE{}\end{equation}
This spectrum is doubly degenerate so that two
energy levels correspond to four states
having different quantum numbers of \myHighlight{$(\sigma, \tau)$}\coordHE{}.

From the explicit expression of the operator \myHighlight{$Q$}\coordHE{}  of eq.(\ref{6})
it is seen that supersymmetry transformations
transform a state with quantum numbers
\myHighlight{$(p, \sigma, \tau)$}\coordHE{} into another state with
quantum numbers \myHighlight{$(p^{\prime\prime}, \sigma^{\prime\prime},
 \tau^{\prime\prime})$}\coordHE{}. There are states with the same energy among these.
The states with the same energy
are called superpartners.
There are therefore superpartner states on the field of
eq.(\ref{1}).
According to eq.(\ref{11}) there is a second
supercharge operator in the theory.
Since \myHighlight{$A^a_3=0$}\coordHE{} the z--dependence of the quark wave
function is of the plane wave type. Considering
the case \myHighlight{$p_z = 0$}\coordHE{} at initial time \myHighlight{$t=0$}\coordHE{}, 
 it is reasonable to consider the supercharge \myHighlight{$Q$}\coordHE{}
in the \myHighlight{$(x, y)$}\coordHE{} plane and set
\begin{equation}\coord{}\boxEquation{
Q_1=\sigma_1P_1 + \sigma_2 P_2
\label{13}
}{
Q_1=\sigma_1P_1 + \sigma_2 P_2
}{ecuacion}\coordE{}\end{equation}
The second supercharge operator can be constructed with the
help of the prescription suggested in ref.\cite{2}:
\begin{equation}\coord{}\boxEquation{
Q_2=iQ_1\sigma_3=\sigma_2P_1-\sigma_1P_2
\label{14}
}{
Q_2=iQ_1\sigma_3=\sigma_2P_1-\sigma_1P_2
}{ecuacion}\coordE{}\end{equation}
The supercharge \myHighlight{$Q_2$}\coordHE{} is hermitian and
together with \myHighlight{$Q_1$}\coordHE{} obeys Witten's
supercharge algebra eq. (\ref{10}).
With the help of \myHighlight{$Q_1$}\coordHE{} and \myHighlight{$Q_2$}\coordHE{} we can construct 
hermitian mutually conjugate supercharge
operators \myHighlight{$Q_{\pm}$}\coordHE{} by setting
\begin{equation}\coord{}\boxEquation{
Q_{\pm}=\frac{1}{2}(Q_1\pm iQ_2)=Q_1\frac{1\mp \sigma_3}{2}.
\label{15}
}{
Q_{\pm}=\frac{1}{2}(Q_1\pm iQ_2)=Q_1\frac{1\mp \sigma_3}{2}.
}{ecuacion}\coordE{}\end{equation}
These operators obey the field theory supersymmetry
algebra\cite{11} and the Jacobi identity:
\begin{eqnarray}\coord{}\boxAlignEqnarray{\leftCoord{}
Q^2_+ &=& Q^2_- = 0, \;\;\; \{Q_+, Q_-\}=H,\;\;\; [Q_{\pm}, H] = 0,\nonumber\rightCoord{}\\\leftCoord{}
\leftCoord{}0&\equiv &\{[Q_+,H], Q_-\} +\{[Q_-,H], Q_+\} +[H, \{Q_+,Q_-\}]
\label{16}
\rightCoord{}}{0mm}{3}{3}{
Q^2_+ &=& Q^2_- = 0, \;\;\; \{Q_+, Q_-\}=H,\;\;\; [Q_{\pm}, H] = 0,\\
0&\equiv &\{[Q_+,H], Q_-\} +\{[Q_-,H], Q_+\} +[H, \{Q_+,Q_-\}]
}{1}\coordE{}\end{eqnarray}
Setting
$$\coord{}\boxMath{
P_{\pm} = P_1\pm iP_2, \;\;\;a_{\pm}=\frac{1}{2}(\sigma_1 \pm i\sigma_2),
}{dollar}{0pt}\coordE{}$$
which satisfy \myHighlight{$P^{\dagger}_{\pm}=P_{\mp}, a^{\dagger}_{\pm}=a_{\mp}$}\coordHE{}, 
the operators \myHighlight{$Q_{\pm}$}\coordHE{} can be expressed as
\begin{equation}\coord{}\boxEquation{
Q_+=P_-a_+, \;\;\; Q_-=P_+a_-,
\label{17}
}{
Q_+=P_-a_+, \;\;\; Q_-=P_+a_-,
}{ecuacion}\coordE{}\end{equation}
and the supersymmetry algebra (\ref{15}) is
fulfilled by the relations
\begin{equation}\coord{}\boxEquation{
[Q_+, P_-]=0, \;\;[Q_-, P_+]=0, \;\;\{Q_+, a_-\}=P_-,
\;\;\{Q_-, a_+\}=-P_+,
\label{18}
}{
[Q_+, P_-]=0, \;\;[Q_-, P_+]=0, \;\;\{Q_+, a_-\}=P_-,
\;\;\{Q_-, a_+\}=-P_+,
}{ecuacion}\coordE{}\end{equation}
which is in full agreement with ref.\cite{11}.
Only four of the six operators are independent.

It is easy to see from the definition of the operators \myHighlight{$a_{\pm}$}\coordHE{}
that they obey the algebra of creation and annihilation operators
for fermion degrees of freedom, i.e.
$$\coord{}\boxMath{
\{a_+,a_-\}=1, \;\;\; (a_+)^2 = (a_-)^2 = 0.
}{dollar}{0pt}\coordE{}$$
This implies that the fermion degree of freedom in the
quantum mechanics
of the nonabelian charged particle in a chromomagnetic field is
the particle's spin. The commutator of the operators
\myHighlight{$P_{\pm}$}\coordHE{} gives
\begin{equation}\coord{}\boxEquation{
[P_+, P_-]=gH^3_z\tau_3
\label{19}
}{
[P_+, P_-]=gH^3_z\tau_3
}{ecuacion}\coordE{}\end{equation}
 The spinors \myHighlight{$\phi$}\coordHE{}
and \myHighlight{$\chi$}\coordHE{} of eqs. (\ref{4})  and (\ref{5})
 have two components corresponding to different
projections of spin which we write \myHighlight{$\psi =\left(\begin{array}{c} \psi_+\\
\psi_-\end{array}\right)$}\coordHE{}.
 The components \myHighlight{$\psi_+$}\coordHE{} and \myHighlight{$ \psi_-$}\coordHE{}
are spinors in
color space which  we write  
\myHighlight{$\psi_+ =\left(\begin{array}{c} \psi^{(1)}_+\\
\psi^{(2)}_+\end{array}\right)
, \psi_- =\left(\begin{array}{c} \psi^{(1)}_-\\
\psi^{(2)}_-\end{array}\right)$}\coordHE{}.
Since the color operator \myHighlight{$\tau_3$}\coordHE{} has its own two eigenvalues, the commutator
(\ref{19}) splits into two commutation relations
depending on the color state chosen, i.e.
\begin{eqnarray}\coord{}\boxAlignEqnarray{
&\leftCoord{}a)&\;\; [P_+,P_-]=gH^3_z \;\;\;
 for \; state \; \psi^{(1)}_{\pm} \; with \;\;\;
 {\rightCoord{}\leftCoord{}\hat \tau}_3\psi^{(1)}_{\pm}=\psi^{(1)}_{\pm},
\nonumber\rightCoord{}\\
&\leftCoord{}b)& \;\;[P_+,P_-]=-gH^3_z \;\;\; for \; state 
\leftCoord{}\; \psi^{(2)}_{\pm} \;  with \;\;\;
  {\rightCoord{}\leftCoord{}\hat \tau}_3\psi^{(2)}_{\pm}=-\psi^{(2)}_{\pm}
\label{20}
\rightCoord{}}{0mm}{5}{5}{
&a)&\;\; [P_+,P_-]=gH^3_z \;\;\;
 for \; state \; \psi^{(1)}_{\pm} \; with \;\;\;
 {\hat \tau}_3\psi^{(1)}_{\pm}=\psi^{(1)}_{\pm},
\\
&b)& \;\;[P_+,P_-]=-gH^3_z \;\;\; for \; state 
\; \psi^{(2)}_{\pm} \;  with \;\;\;
  {\hat \tau}_3\psi^{(2)}_{\pm}=-\psi^{(2)}_{\pm}
}{1}\coordE{}\end{eqnarray}
We run into a situation which is analogous to one
in ref.\cite{12} where the quantum mechanical motion of
a wave packet was shown to be made up of a mixture of
states with \myHighlight{$\tau_3$}\coordHE{}--eigenvalues = +1 and -1. 
In a chromomagnetic field  such
a packet breaks up into modes with different values
of \myHighlight{$\tau_3$}\coordHE{} moving in opposite directions.
We introduce operators \myHighlight{$b_{\pm}$}\coordHE{} as
creation and annihilation operators
of bosonic states respectively by defining 
$$\coord{}\boxMath{
b_{\pm} = \frac{P_{\pm}}{\sqrt{gH^3_z}}
}{dollar}{0pt}\coordE{}$$
which obey
correspondingly  the Heisenberg--Weyl algebra
\begin{equation}\coord{}\boxEquation{
a) \;\;\; [b_+,b_-] = 1 \;\; for \; \psi^{(1)}_{\pm},
 \;\;\;\; b)\;\;\; [b_-, b_+]= 1
\;\; for \; \psi^{(2)}_{\pm}.
\label{21}
}{
a) \;\;\; [b_+,b_-] = 1 \;\; for \; \psi^{(1)}_{\pm},
 \;\;\;\; b)\;\;\; [b_-, b_+]= 1
\;\; for \; \psi^{(2)}_{\pm}.
}{ecuacion}\coordE{}\end{equation}
>From this we see, that the operators \myHighlight{$b_+$}\coordHE{}
and \myHighlight{$b_-$}\coordHE{} interchange their roles
for  state \myHighlight{$\psi^{(1)}$}\coordHE{}
and  state \myHighlight{$\psi^{(2)}$}\coordHE{}.
The meaning of the operators \myHighlight{$Q_{\pm}$}\coordHE{}
of eqs.(\ref{16}) becomes
clear \footnote{Here the operators
\myHighlight{$a_{\pm}$}\coordHE{} do not have the same meaning as in field
theory where the fermion operators
imply changes of spin by \myHighlight{$\pm 1/2$}\coordHE{}.}:
If \myHighlight{$n_f$}\coordHE{} and \myHighlight{$n_b$}\coordHE{} denote the number of fermions
 and
bosons respectively,
the operator \myHighlight{$Q_+$}\coordHE{} transforms a state \myHighlight{$\psi^{(1)}$}\coordHE{}
with \myHighlight{$(n_f, n_b)$}\coordHE{} into the state
\myHighlight{$\psi^{(2)}$}\coordHE{} with \myHighlight{$(n_f+1, n_b-1)$}\coordHE{}
and the state \myHighlight{$\psi^{(2)}$}\coordHE{} with
\myHighlight{$(n_f+1, n_b-1)$}\coordHE{} into the state
\myHighlight{$\psi^{(1)}$}\coordHE{} with \myHighlight{$(n_f, n_b)$}\coordHE{}, and
the operator \myHighlight{$Q_-$}\coordHE{} transforms the state
\myHighlight{$\psi^{(2)}$}\coordHE{} with \myHighlight{$(n_f+1, n_b-1)$}\coordHE{} into the state \myHighlight{$\psi^{(1)}$}\coordHE{}
with \myHighlight{$(n_f, n_b)$}\coordHE{} and the state \myHighlight{$\psi^{(1)}$}\coordHE{} with
\myHighlight{$(n_f, n_b)$}\coordHE{} into the state \myHighlight{$\psi^{(2)}$}\coordHE{}
with \myHighlight{$(n_f+1, n_b-1)$}\coordHE{}.  In any case the sum
of fermion and boson numbers is conserved, \myHighlight{$n_f+n_b=const.$}\coordHE{}

We now rewrite the anticommutator (\ref{16}) using (\ref{17}):
\begin{equation}\coord{}\boxEquation{
H=Q_+Q_-+Q_-Q_+=\frac{1}{2}\bigg\{P_-,P_+\bigg\}
+\frac{1}{2}\bigg[P_-,P_+\bigg]\sigma_3
=\left(\begin{array}{cc}P_-P_+&0\\
0 & P_+P_-\end{array}\right)
\label{22}
}{
H=Q_+Q_-+Q_-Q_+=\frac{1}{2}\bigg\{P_-,P_+\bigg\}
+\frac{1}{2}\bigg[P_-,P_+\bigg]\sigma_3
=\left(\begin{array}{cc}P_-P_+&0\\
0 & P_+P_-\end{array}\right)
}{ecuacion}\coordE{}\end{equation}
Thus the Hamiltonian of (\ref{8}) is split into two
parts corresponding to different projections of spin.
According to eq. (\ref{9}) we have
\begin{equation}\coord{}\boxEquation{
H\psi=\left(\begin{array}{cc}P_-P_+ & 0\\
0 & P_+P_-\end{array}\right)\psi = (E^2-m^2)\left(\begin{array}{c}\psi_+ \\
\psi_-\end{array}\right)
\label{23}
}{
H\psi=\left(\begin{array}{cc}P_-P_+ & 0\\
0 & P_+P_-\end{array}\right)\psi = (E^2-m^2)\left(\begin{array}{c}\psi_+ \\
\psi_-\end{array}\right)
}{ecuacion}\coordE{}\end{equation}
Thus we have two independent equations:
\begin{eqnarray}\coord{}\boxAlignEqnarray{\leftCoord{}
P_-P_+\psi_+&=&(E^2-m^2)\psi_+,\nonumber\rightCoord{}\\\leftCoord{}
P_+P_-\psi_-&=&(E^2-m^2)\psi_-
\label{24}
\rightCoord{}}{0mm}{2}{3}{
P_-P_+\psi_+&=&(E^2-m^2)\psi_+,\\
P_+P_-\psi_-&=&(E^2-m^2)\psi_-
}{1}\coordE{}\end{eqnarray}
These equations could be used for finding the ground state wave functions
of quarks in the field (\ref{1}).
It is well known that with unbroken supersymmetry
the ground state has zero energy eigenvalue. As we see from
(\ref{12}) the energy
spectrum \myHighlight{$E^2_1$}\coordHE{} has its minimal eigenvalue
at \myHighlight{${\bf p}=0$}\coordHE{},
\begin{equation}\coord{}\boxEquation{
\bigg(E^2_1\bigg)_{min}=\bigg( m^2+\frac{g^2({\lambda_1}^{1/2}+{\lambda _2}
^{1/2})^2}{4}\bigg),
\label{25}
}{
\bigg(E^2_1\bigg)_{min}=\bigg( m^2+\frac{g^2({\lambda_1}^{1/2}+{\lambda _2}
^{1/2})^2}{4}\bigg),
}{ecuacion}\coordE{}\end{equation}
and 
 the spectrum \myHighlight{$E^2_2$}\coordHE{} has its minimal eigenvalue at
\myHighlight{${\bf p} = (0, g\sqrt{(\lambda_2-\lambda_1)}/2, 0)$}\coordHE{}
for \myHighlight{$\lambda_2 > \lambda_1$}\coordHE{} 
and at \myHighlight{${\bf p} = (g\sqrt{(\lambda_1-\lambda_2)}/2, 0, 0)$}\coordHE{}
for \myHighlight{$\lambda_1> \lambda_2$}\coordHE{} with
\begin{equation}\coord{}\boxEquation{
 \bigg(E^2_2\bigg)_{min} = m^2.
\label{26}
}{
 \bigg(E^2_2\bigg)_{min} = m^2.
}{ecuacion}\coordE{}\end{equation}
Hence 
only for the spectrum \myHighlight{$E^2_2$}\coordHE{} supersymmetry is
unbroken for massless quarks (\myHighlight{$m=0$}\coordHE{}).
Then for the ground state with unbroken supersymmetry
eqs.(\ref{24}) assume the form
\begin{equation}\coord{}\boxEquation{
P_-P_+\psi_+ = 0, \;\;\; P_+P_-\psi_- = 0. 
\label{27}
}{
P_-P_+\psi_+ = 0, \;\;\; P_+P_-\psi_- = 0. 
}{ecuacion}\coordE{}\end{equation}
These equations show that the Hamiltonian for the
ground state is reduced to a product of two
linear, i.e. factorised, operators.
According to (\ref{27}) the expectation values of 
the operators \myHighlight{$P_{\mp}P_{\pm}$}\coordHE{} vanish, i.e.
\begin{equation}\coord{}\boxEquation{
<\psi_+|P_-P_+|\psi_+> = 0, \;\;
<\psi_-|P_+P_-|\psi_-> = 0.
\label{28}
}{
<\psi_+|P_-P_+|\psi_+> = 0, \;\;
<\psi_-|P_+P_-|\psi_-> = 0.
}{ecuacion}\coordE{}\end{equation}
Since \myHighlight{$P_{\pm}$}\coordHE{} are mutually hermitian  conjugate operators
these equations can be written as 
\begin{equation}\coord{}\boxEquation{
\bigg|P_+|\psi_+>\bigg|^2=0, \;\;
\bigg|P_-|\psi_->\bigg|^2=0.
\label{29}
}{
\bigg|P_+|\psi_+>\bigg|^2=0, \;\;
\bigg|P_-|\psi_->\bigg|^2=0.
}{ecuacion}\coordE{}\end{equation}
Thus to find the ground state wave function we have to solve
the two linear equations obtained from (\ref{29}). Taking
into account the explicit expressions of \myHighlight{$P_{\pm}$}\coordHE{} and
the color components \myHighlight{$\psi^{(1),(2)}_{\pm}$}\coordHE{} of the spinors
\myHighlight{$\psi_{\pm}$}\coordHE{}, we obtain
\begin{eqnarray}\coord{}\boxAlignEqnarray{\leftCoord{} 
P_+|\psi_+> &=&\left(\begin{array}{cc}
p_1+ip_2 & \frac{\leftCoord{}ig}{\rightCoord{}2}(A^1_1+A^2_2)\rightCoord{}\\\leftCoord{}
\frac{\leftCoord{}ig}{\rightCoord{}2}(A^1_1-A^2_2) & p_1+ip_2\rightCoord{}
\end{array}\right)\left(\begin{array}{c} \psi^{(1)}_+\rightCoord{}\\\leftCoord{}
\psi^{(2)}_+\end{array}\right) = 0, \nonumber\rightCoord{}\\\leftCoord{}
P_-|\psi_+> &=&\left(\begin{array}{cc}
p_1-ip_2 & \frac{\leftCoord{}ig}{\rightCoord{}2}(A^1_1-A^2_2)\rightCoord{}\\\leftCoord{}
\frac{\leftCoord{}ig}{\rightCoord{}2}(A^1_1+A^2_2) & p_1-ip_2\rightCoord{}
\end{array}\right)\left(\begin{array}{c} \psi^{(1)}_-\rightCoord{}\\\leftCoord{}
\psi^{(2)}_-\end{array}\right) = 0.\rightCoord{}
\label{30}
\rightCoord{}}{0mm}{10}{14}{ 
P_+|\psi_+> &=&\left(\begin{array}{cc}
p_1+ip_2 & \frac{ig}{2}(A^1_1+A^2_2)\\
\frac{ig}{2}(A^1_1-A^2_2) & p_1+ip_2
\end{array}\right)\left(\begin{array}{c} \psi^{(1)}_+\\
\psi^{(2)}_+\end{array}\right) = 0, \\
P_-|\psi_+> &=&\left(\begin{array}{cc}
p_1-ip_2 & \frac{ig}{2}(A^1_1-A^2_2)\\
\frac{ig}{2}(A^1_1+A^2_2) & p_1-ip_2
\end{array}\right)\left(\begin{array}{c} \psi^{(1)}_-\\
\psi^{(2)}_-\end{array}\right) = 0.
}{1}\coordE{}\end{eqnarray}
These equations are easily solved in polar coordinates
with \myHighlight{$ x=r\cos\theta, y= r\sin\theta$}\coordHE{}, in view of the cylindrical
symmetry of the external field (\ref{1}).
In these coordinates the operators \myHighlight{$p_1\pm ip_2$}\coordHE{}
assume the form
\begin{equation}\coord{}\boxEquation{
p_1\pm ip_2 = e^{\pm i\theta}\bigg(\frac{\partial}{\partial r}\mp\frac{i}
{r}\frac{\partial}{\partial\theta}\bigg)
\label{31}
}{
p_1\pm ip_2 = e^{\pm i\theta}\bigg(\frac{\partial}{\partial r}\mp\frac{i}
{r}\frac{\partial}{\partial\theta}\bigg)
}{ecuacion}\coordE{}\end{equation}
We thus have two independent variables \myHighlight{$r, \theta$}\coordHE{}
and one constraint resulting from conservation of
\myHighlight{$Q={\bf \sigma}\cdot{\bf P}$}\coordHE{}. 
We choose the reference frame so that \myHighlight{$\theta$}\coordHE{} is the
angle between \myHighlight{${\bf \sigma} $}\coordHE{} and \myHighlight{${\bf P}$}\coordHE{}, and
we assume that the external chromomagnetic field given
by potentials (\ref{1}) is given in this
reference frame, since otherwise, 
on passing to any other moving frame, the external
field (\ref{1}) will also possess chromoelectric
components.  Conservation of \myHighlight{${\bf \sigma}\cdot{\bf P}$}\coordHE{}
means \myHighlight{$\cos\theta = const.$}\coordHE{} Consequently
\myHighlight{$\theta = const.$}\coordHE{} and \myHighlight{$\partial\psi^{(1),(2)}_{\pm}/\partial\theta =0$}\coordHE{}.
With this eqs.(\ref{30}) assume the well known form
\begin{equation}\coord{}\boxEquation{
\bigg(\frac{\partial^2}{\partial r^2}+\frac{g^2}{4}e^{\mp2i\theta}
(\lambda_2-\lambda_1)\bigg)\psi^{(1)}_{\pm} =0, \;\;
\psi^{(2)}_{\pm}=\frac{2e^{\pm i\theta}}{g(\sqrt{\lambda_1}+\sqrt{\lambda_2})}
\frac{\partial}{\partial r}\psi^{(1)}_{\pm}.
\label{32}
}{
\bigg(\frac{\partial^2}{\partial r^2}+\frac{g^2}{4}e^{\mp2i\theta}
(\lambda_2-\lambda_1)\bigg)\psi^{(1)}_{\pm} =0, \;\;
\psi^{(2)}_{\pm}=\frac{2e^{\pm i\theta}}{g(\sqrt{\lambda_1}+\sqrt{\lambda_2})}
\frac{\partial}{\partial r}\psi^{(1)}_{\pm}.
}{ecuacion}\coordE{}\end{equation}
Setting
$$\coord{}\boxMath{
\xi=\frac{g}{2}\sqrt{\lambda_2-\lambda_1}\sin\theta, \;\;
\eta=\frac{g}{2}\sqrt{\lambda_2-\lambda_1}\cos\theta,
}{dollar}{0pt}\coordE{}$$
the general solution of eqs. (\ref{32}) can be written
\begin{eqnarray}\coord{}\boxAlignEqnarray{\leftCoord{}
\psi^{(1)}_+&=& C_1e^{\xi r}e^{i\eta r}+C_2e^{-\xi r}e^{-i\eta r},\nonumber\rightCoord{}\\\leftCoord{}
\psi^{(2)}_+&=&i\sqrt{\frac{\leftCoord{}\sqrt{\lambda_2}-\sqrt{\lambda_1}}
{\rightCoord{}\leftCoord{}\sqrt{\lambda_1}+\sqrt{\lambda_2}}}
\bigg( C_1e^{\xi r}e^{i\eta r}-C_2e^{-\xi r}e^{-i\eta r}\bigg),\nonumber\rightCoord{}\\\leftCoord{}
\psi^{(1)}_-&=&i\sqrt{\frac{\leftCoord{}\sqrt{\lambda_2}-\sqrt{\lambda_1}}
{\rightCoord{}\leftCoord{}\sqrt{\lambda_1}+\sqrt{\lambda_2}}}
\bigg( C^{\prime}_1e^{-\xi r}e^{i\eta r}
\leftCoord{}-C^{\prime}_2e^{\xi r}e^{-i\eta r}\bigg),\nonumber\rightCoord{}\\\leftCoord{}
\psi^{(2)}_-&=& C^{\prime}_1e^{-\xi r}
e^{i\eta r}+C^{\prime}_2e^{\xi r}e^{-i\eta r}.\rightCoord{}
\label{33}
\rightCoord{}}{0mm}{9}{8}{
\psi^{(1)}_+&=& C_1e^{\xi r}e^{i\eta r}+C_2e^{-\xi r}e^{-i\eta r},\\
\psi^{(2)}_+&=&i\sqrt{\frac{\sqrt{\lambda_2}-\sqrt{\lambda_1}}
{\sqrt{\lambda_1}+\sqrt{\lambda_2}}}
\bigg( C_1e^{\xi r}e^{i\eta r}-C_2e^{-\xi r}e^{-i\eta r}\bigg),\\
\psi^{(1)}_-&=&i\sqrt{\frac{\sqrt{\lambda_2}-\sqrt{\lambda_1}}
{\sqrt{\lambda_1}+\sqrt{\lambda_2}}}
\bigg( C^{\prime}_1e^{-\xi r}e^{i\eta r}
-C^{\prime}_2e^{\xi r}e^{-i\eta r}\bigg),\\
\psi^{(2)}_-&=& C^{\prime}_1e^{-\xi r}
e^{i\eta r}+C^{\prime}_2e^{\xi r}e^{-i\eta r}.
}{1}\coordE{}\end{eqnarray}
Selection of the normalizable parts depends on
whether \myHighlight{$\lambda_2>$}\coordHE{} or \myHighlight{$<\lambda_1$}\coordHE{} and on the value of
the angle \myHighlight{$\theta$}\coordHE{}. For instance when \myHighlight{$\lambda_2>\lambda_1$}\coordHE{}
and \myHighlight{$0< \theta\leq \pi/2$}\coordHE{}, we have
\begin{eqnarray}\coord{}\boxAlignEqnarray{\leftCoord{}
\psi^{(1)}_+&=&C_2e^{-\xi r}e^{-i\eta r},\nonumber\rightCoord{}\\\leftCoord{}
\psi^{(2)}_+&=&-iC_2\sqrt{\frac{\leftCoord{}\sqrt{\lambda_2}-\sqrt{\lambda_1}}
{\rightCoord{}\leftCoord{}\sqrt{\lambda_1}+\sqrt{\lambda_2}}}
 e^{-\xi r}e^{-i\eta r},\nonumber\rightCoord{}\\\leftCoord{}
\psi^{(1)}_-&=&iC^{\prime}_1\sqrt{\frac{\leftCoord{}\sqrt{\lambda_2}-\sqrt{\lambda_1}}
{\rightCoord{}\leftCoord{}\sqrt{\lambda_1}+\sqrt{\lambda_2}}}
 e^{-\xi r}e^{i\eta r},\nonumber\rightCoord{}\\\leftCoord{}
\psi^{(2)}_-&=& C^{\prime}_1e^{-\xi r}
e^{i\eta r}. \rightCoord{}
\label{34}
\rightCoord{}}{0mm}{8}{8}{
\psi^{(1)}_+&=&C_2e^{-\xi r}e^{-i\eta r},\\
\psi^{(2)}_+&=&-iC_2\sqrt{\frac{\sqrt{\lambda_2}-\sqrt{\lambda_1}}
{\sqrt{\lambda_1}+\sqrt{\lambda_2}}}
 e^{-\xi r}e^{-i\eta r},\\
\psi^{(1)}_-&=&iC^{\prime}_1\sqrt{\frac{\sqrt{\lambda_2}-\sqrt{\lambda_1}}
{\sqrt{\lambda_1}+\sqrt{\lambda_2}}}
 e^{-\xi r}e^{i\eta r},\\
\psi^{(2)}_-&=& C^{\prime}_1e^{-\xi r}
e^{i\eta r}. 
}{1}\coordE{}\end{eqnarray}

Further analysis of the Dirac equation shows that there is no
supersymmetry in the chromoelectric case. Also supersymmetry is
broken in a chromomagnetic field with spherically
symmetric components.


\vspace{0.4cm}



\noindent
{\bf Acknowledgments}

Sh. M. and J.--z. Z. acknowledge
 support by DAAD 
and discussions with H. J. W. M\"uller--Kirsten. JZZ's work has also been 
supported by the National Natural Science 
Foundation of China under the grant number 10074014 and by the Shanghai 
Education Development Foundation.

\vspace{0.5cm}
\clearpage
\begin{thebibliography}{99}
\bibitem{1} E. Witten, Nucl. Phys. {\bf B188} (1981) 513. 
\bibitem{2} L. Gendenshtein and I. Krive, Uspekhi Fizicheskikh Nauk
{\bf 146} (1985) 552 [Sov. Phys. Usp. {\bf 28} (1985)645].
\bibitem{3} L. Gendenshtein, J. of Nucl. Phys. {\bf 41}(1985) 261.
\bibitem{4} Yu. R. Musin, Sov. J. Phys.{\bf 5} (1990) 28.
\bibitem{5} A.V. Smilga,
Nucl. Phys. {\bf B249} (1985) 413.
\bibitem{6} P. Solomonson and I.W. van Holten, Nucl. Phys. {\bf B196} (1982)
509.
\bibitem{7} F. Cooper, A. Khare and U. Sukhatme, Phys. Rept.
{\bf 251} (1995) 267, hep--th/
9405029;
A.Das and S.A. Pernice,
Mod. Phys. Lett. {\bf A12} (1997) 581, hep--th/9612125.
\bibitem{8} V. C. Zhukovskii, JETP {\bf 90} (1986) 1137.
\bibitem{9} L.S. Brown and W. J. Weisberger, Nucl. Phys. {\bf B157}
(1979) 285. 
\bibitem{10} M. Reuter and C. Wetterich,
Phys. Lett. {\bf B334} (1994) 412, 
hep--ph/9405300.
\bibitem{11} I. M. Ternov, V. C.  Zhukovskii and A. V. Borisov,
{\it Quantum Processes in Strong External Fields}, M. Nauka (1981). 
\bibitem{12} V. Ogievetskii and
L. Mezinchesku, Uspekhi Fizicheskikh Nauk {\bf 117} 
(1975) 673 [Sov. Phys. Usp. {\bf 18} (1975) 960].
\bibitem{13} V.C. Zhukovskii and S. A. Mamedov,
Soviet J. Phys. {\bf 1} (1990) 106.


\end{thebibliography}

\end{document}





\bye
