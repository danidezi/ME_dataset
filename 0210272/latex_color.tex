
\documentclass[a4paper,12pt]{article}
\renewcommand{\baselinestretch}{1.2}
\providecommand{\sect}[1]{\setcounter{equation}{0}\section{#1}}
\renewcommand{\theequation}{\thesection.\arabic{equation}}
\textwidth 160mm \textheight 220mm

\usepackage{useful_macros}
\begin{document}
\topmargin 0pt \oddsidemargin 0mm

\renewcommand{\thefootnote}{\fnsymbol{footnote}}
\begin{titlepage}
\begin{flushright}
INJE-TP-02-05\\
hep-th/0210272
\end{flushright}

\vspace{5mm}
\begin{center}
{\Large \bf Holography in Radiation-dominated  Universe with
a Positive Cosmological Constant} \vspace{12mm}

{\large Rong-Gen Cai\footnote{e-mail address: cairg@itp.ac.cn}\myHighlight{$^1$}\coordHE{}
 and Yun Soo Myung\footnote{e-mail
 address: ysmyung@physics.inje.ac.kr}\myHighlight{$^2$}\coordHE{}}
 \\
\vspace{10mm} {\em \myHighlight{$^1$}\coordHE{} Institute of Theoretical Physics, Chinese
Academy of Sciences, P.O. Box 2735, Beijing 100080, China \\
\myHighlight{$^2$}\coordHE{} Relativity Research Center and School of Computer Aided
Science, Inje University, Gimhae 621-749, Korea}
\end{center}

\vspace{5mm} \centerline{{\bf{Abstract}}}
 \vspace{5mm}
We discuss the holographic principle in a radiation-dominated,
closed Friedmann-Robertson-Walker (FRW) universe with a positive
cosmological constant. By introducing a cosmological D-bound on
entropy of matter in the universe, we can write the Friedmann equation 
governing evolution of the universe in the form of Cardy formula. When 
the cosmological D-bound is saturated, the Friedmann equation coincides 
with the Cardy-Verlinde formula describing entropy of radiation in the
universe.  As a concrete model, we consider the brane cosmology in the 
background of Schwarzschild-de Sitter black holes. It is found that the 
cosmological D-bound is saturated when the brane crosses the black hole 
horizon of the background. At that moment, the Friedmann equation
coincides with the Cardy-Verlinde formula describing the entropy
of radiation matter on the brane.



\end{titlepage}

\newpage
\renewcommand{\thefootnote}{\arabic{footnote}}
\setcounter{footnote}{0} \setcounter{page}{2}

%%===================section 1 ====================
\sect{Introduction}

The holographic principle is perhaps one of fundamental principles
of nature, which relates a theory with gravity in \myHighlight{$D$}\coordHE{} dimensions
to a theory without gravity in lower dimensions~\cite{Hooft}.
Although we do not yet completely understand how the hologram of
gravity is realized, some beautiful examples as the realization of
the principle have been found, through the AdS/CFT
correspondence~\cite{AdS}.


In a seminar paper~\cite{Verl},  E. Verlinde found a quite
interesting holographic relation between the Friedmann equation
describing  a radiation-dominated, closed
Friedmann-Robertson-Walker (FRW) universe and Cardy
formula~\cite{Cardy} describing the entropy of matter filling the
universe. The radiation can be represented by a conformal field
theory (CFT) with a large central charge, while the entropy for the
latter can be expressed in terms of the so-called Cardy-Verlinde
formula~\cite{Verl}, a generalized form of Cardy formula in any
dimension. The Cardy-Verlinde formula was proved to hold at least
for CFTs with AdS gravity dual, for instance, for CFTs dual to
Schwarzschild-AdS black holes~\cite{Verl}, Kerr-AdS black
holes~\cite{Klemm}, hyperbolic AdS black holes and charged AdS
black holes~\cite{Cai1}, Taub-Bolt AdS instanton
solutions~\cite{Birm}, and Kerr-Newman-AdS black
holes~\cite{Jing}. Verlinde found that the Friedmann equation can
be rewritten in the form of Cardy-Verlinde formula with the help
of three cosmological entropy bounds, and that when the Hubble entropy
bound is saturated, the Friedmann equation coincides with the
Cardy-Verlinde formula.  This observation is very interesting in the 
sense that the Friedmann equation is a dynamic one describing geometry
evolution of the universe while the Cardy-Verlinde formula is just a formula
describing the number of degrees of freedom of matter in the universe.
Therefore, the Verlinde's observation is indeed an interesting 
manifestation of holographic principle in the cosmological setting.

By considering a moving brane universe in the background of
Schwarzschild-AdS black holes in arbitrary dimensions, the
holographic connection between the geometry and matter can be
realized. A radiation-dominated closed FRW universe appears as an induced
metric on the brane embedded in the bulk background. Savonije and 
Verlinde~\cite{SV} interpreted the radiation
as the thermal CFT dual to the bulk Schwarzschild-AdS black hole. 
Further they observed that when the
brane crosses the bulk black hole horizon, (i) the entropy and
temperature of the universe can be simply expressed in terms of
the Hubble parameter and its time derivative; (ii) the entropy
formula (Cardy-Verlinde formula) of CFTs in any dimension
coincides with the Friedmann equation; and (iii) the Hubble entropy
bound is just saturated by the entropy of bulk black holes.

Since then a lot of studies have been done focusing on the
generalization of \cite{Verl,SV} to various bulk geometries. In
this paper, we are interested in the case where a positive cosmological
constant is present in the universe. Namely we will discuss the holography 
for a radiation-dominated closed FRW universe with a positive
cosmological constant. This is motivated partially by the de 
Sitter (dS)/CFT correspondence~\cite{Strom}, and by recent astronomy 
observations on supernova indicating that our universe is accelerating~\cite{Super}, 
which can be 
interpreted that there might be a positive cosmological constant in our 
universe. We will start with a brief review on the case 
without the cosmological constant~\cite{Verl} in the next section, and then
discuss the holography in the case with a positive cosmological constant 
by introducing a cosmological D-bound on entropy of matter in the universe. 
As an example, in Sec.~3 we investigate the holographic connection between the 
Friedmann equation and Cardy-Verlinde formula in the brane cosmology in the
background of Schwarzschild-de Sitter black holes. We end this paper in 
Sec.~4 with some conclusions and discussions.  



%%====================section 2=====================
\section{Holography in a radiation-dominated closed universe
with a positive cosmological constant}

Let us consider a \myHighlight{$(n+1)$}\coordHE{}-dimensional closed FRW universe
\begin{equation}\coord{}\boxEquation{
\label{2eq1} ds^2 =-d\tau^2 +R^2 d\Omega^2_n,
}{
ds^2 =-d\tau^2 +R^2 d\Omega^2_n,
}{ecuacion}\coordE{}\end{equation}
where \myHighlight{$R$}\coordHE{} is the  scale factor of the universe and \myHighlight{$d\Omega^2_n$}\coordHE{}
denotes the line element of a \myHighlight{$n$}\coordHE{}-dimensional unit sphere.
The evolution of the universe is determined by the FRW equations
\begin{eqnarray}\coord{}\boxAlignEqnarray{\leftCoord{}
\label{2eq2}
&&\leftCoord{} H^2 =\frac{\leftCoord{}16\pi G_{n+1}}{n(n-1)}\frac{\leftCoord{}E}{\rightCoord{}V}
\leftCoord{}-\frac{\leftCoord{}1}{\rightCoord{}R^2} \rightCoord{}
     \leftCoord{}+\frac{\leftCoord{}1}{\rightCoord{}l^2_{n+1}}, \nonumber \rightCoord{}\\
&&\leftCoord{} \dot H =-\frac{\leftCoord{}8\pi G_{n+1}}{n-1}\left (\frac{\leftCoord{}E}{\rightCoord{}V} +p\right)
    \leftCoord{}+\frac{\leftCoord{}1}{\rightCoord{}R^2}, \rightCoord{}
\rightCoord{}}{0mm}{13}{10}{
&& H^2 =\frac{16\pi G_{n+1}}{n(n-1)}\frac{E}{V}
-\frac{1}{R^2} 
     +\frac{1}{l^2_{n+1}}, \\
&& \dot H =-\frac{8\pi G_{n+1}}{n-1}\left (\frac{E}{V} +p\right)
    +\frac{1}{R^2}, 
}{1}\coordE{}\end{eqnarray}
where \myHighlight{$H$}\coordHE{} represents the Hubble parameter with the definition
\myHighlight{$H=\dot R/R$}\coordHE{} and the overdot stands for  derivative with
respect to the cosmic time \myHighlight{$\tau$}\coordHE{},  \myHighlight{$E$}\coordHE{} is the energy of matter
filling the universe, \myHighlight{$p$}\coordHE{} is the pressure, \myHighlight{$V$}\coordHE{} is the
volume of the universe, \myHighlight{$V=R^n \Omega_n$}\coordHE{} with \myHighlight{$\Omega_n$}\coordHE{} being the
volume of a \myHighlight{$n$}\coordHE{}-dimensional unit sphere, and \myHighlight{$G_{n+1}$}\coordHE{} is the
Newton constant in (\myHighlight{$n+1$}\coordHE{}) dimensions. In addition,
\myHighlight{$l^2_{n+1}$}\coordHE{} is related to the cosmological constant
\myHighlight{$\Lambda_{n+1}= n(n-1)/2l^2_{n+1}$}\coordHE{} in (\myHighlight{$n+1$}\coordHE{}) dimensions.


%%==========================section 2.1 =====================

\subsection{The case without cosmological constant}

In \cite{Verl} Verlinde introduced three entropy
bounds\footnote{In \cite{Verl} the first bound is called the
Bekenstein bound. In fact this bound is slightly different from
the original Bekenstein bound~\cite{Beke} by a numerical factor
\myHighlight{$1/n$}\coordHE{}. So we call this the Bekenstein-Verlinde bound. This bound
could be viewed as the counterpart of the Bekenstein bound in the
cosmological setting~\cite{CMO}.} :
\begin{eqnarray}\coord{}\boxAlignEqnarray{\leftCoord{}
\label{2eq3}
 {\rightCoord{}\leftCoord{}\rm Bekenstein-Verlinde\ bound}:&& S_{\rm BV}=\frac{\leftCoord{}2\pi}{\rightCoord{}n}ER
   \nonumber \rightCoord{}\\\leftCoord{}
 {\rightCoord{}\leftCoord{}\rm Bekenstein-Hawking\ bound}:&& S_{\rm BH}=(n-1)\frac{\leftCoord{}V}{\rightCoord{}4G_{n+1}R}
    \nonumber \rightCoord{}\\\leftCoord{}
  {\rightCoord{}\leftCoord{}\rm Hubble\ bound}:&& S_{\rm H}=(n-1)\frac{\leftCoord{}HV}{\rightCoord{}4G_{n+1}}.
\rightCoord{}
\rightCoord{}}{0mm}{9}{11}{
{\rm Bekenstein-Verlinde\ bound}:&& S_{\rm BV}=\frac{2\pi}{n}ER
   \\
 {\rm Bekenstein-Hawking\ bound}:&& S_{\rm BH}=(n-1)\frac{V}{4G_{n+1}R}
    \\
  {\rm Hubble\ bound}:&& S_{\rm H}=(n-1)\frac{HV}{4G_{n+1}}.
}{1}\coordE{}\end{eqnarray}
The Bekenstein-Verlinde bound is supposed to hold for a weakly 
self-gravitating universe (\myHighlight{$HR \le 1$}\coordHE{}), while the Hubble entropy bound
works when the universe is in the strongly self-gravitating 
phase (\myHighlight{$HR \ge 1$}\coordHE{}). In the case without the cosmological constant, the Friedmann
equation [the first equation in (\ref{2eq2})] can be rewritten as
\begin{equation}\coord{}\boxEquation{
\label{2eq4} 
S_{\rm H}=\sqrt{S_{\rm BH}(2S_{\rm BV}-S_{\rm BH})},
}{
S_{\rm H}=\sqrt{S_{\rm BH}(2S_{\rm BV}-S_{\rm BH})},
}{ecuacion}\coordE{}\end{equation}
in terms of three entropy bounds. Expression (\ref{2eq4}) 
is similar to the Cardy formula~\cite{Cardy}, an
entropy formula of CFTs in two dimensions. It is interesting to
note that when \myHighlight{$HR=1$}\coordHE{}, one has \myHighlight{$S_{\rm BV}= S_{\rm BH}=S_{\rm H}$}\coordHE{}.



Let us  define a quantity \myHighlight{$E_{\rm BH}$}\coordHE{} which corresponds to energy
needed to form a black hole with size of the whole universe
through the relation: \myHighlight{$ S_{\rm BH}=(n-1)V/4G_{n+1}R \equiv 2\pi E
R/n $}\coordHE{}. With this quantity, the Friedmann equation (\ref{2eq4}) can
be further cast to
\begin{equation}\coord{}\boxEquation{
\label{2eq5}
 S_{\rm H}=\frac{2\pi R}{n}\sqrt{E_{\rm BH}(2E-E_{\rm BH})},
}{
S_{\rm H}=\frac{2\pi R}{n}\sqrt{E_{\rm BH}(2E-E_{\rm BH})},
}{ecuacion}\coordE{}\end{equation}
which takes the same form as the Cardy-Verlinde formula~\cite{Verl}
\begin{equation}\coord{}\boxEquation{
\label{2eq6}
  S=\frac{2\pi R}{n}\sqrt{E_c(2E-E_c)}.
}{
S=\frac{2\pi R}{n}\sqrt{E_c(2E-E_c)}.
}{ecuacion}\coordE{}\end{equation}
This formula is supposed to describe the entropy \myHighlight{$S$}\coordHE{} of a CFT
living on a \myHighlight{$n$}\coordHE{}-dimensional sphere with radius \myHighlight{$R$}\coordHE{}. Here \myHighlight{$E$}\coordHE{} is
the total energy of the CFT and \myHighlight{$E_c$}\coordHE{} stands for the Casimir
energy of the system, the non-extensive part of the total energy.
Now we suppose that the entropy of the radiation matter in the FRW
universe can be described by the Cardy-Verlinde formula. Comparing
(\ref{2eq5}) with (\ref{2eq6}), one can easily see that if
\myHighlight{$E_{\rm BH}=E_c$}\coordHE{}, \myHighlight{$S_{\rm H}$}\coordHE{} and \myHighlight{$S$}\coordHE{} must be equal. In other
words, the Hubble entropy bound is saturated by the entropy
of radiation matter in the universe if the Casimir energy \myHighlight{$E_c$}\coordHE{} is
just enough to form a black hole with the size of the universe.
At that moment, equations (\ref{2eq5}) and (\ref{2eq6}) coincide with
each other.  This implies that the Friedmann equation
somehow knows the entropy formula of radiation-matter filling the
universe~\cite{Verl}. Considering a moving brane universe in the
background of Schwarzschild-AdS black holes, Savonije and
Verlinde~\cite{SV} found that when the brane crosses the black
hole horizon, the Hubble entropy bound is saturated by the entropy of
black holes in the bulk.

In (\ref{2eq6}) the Casimir energy \myHighlight{$E_c$}\coordHE{} is defined as~\cite{Verl}
\begin{equation}\coord{}\boxEquation{
\label{2eq7}
 E_c=n(E +pV -TS),
 }{
E_c=n(E +pV -TS),
 }{ecuacion}\coordE{}\end{equation}
 where \myHighlight{$T$}\coordHE{} stands for the temperature of the thermal CFT. Further
 Verlinde found that except for the similarity between the
 Friedmann equation (\ref{2eq5}) and the Cardy-Verlinde formula
 (\ref{2eq6}), there is also a similarity between the second
 equation in (\ref{2eq2}) concerning the time derivative of Hubble
 parameter and the equation (\ref{2eq7}) about the Casimir energy
 of CFTs. Let us define a (limiting) temperature
 \begin{equation}\coord{}\boxEquation{
 \label{2eq8}
 T_{\rm H}=-\frac{\dot H}{2\pi H}.
 }{
 T_{\rm H}=-\frac{\dot H}{2\pi H}.
 }{ecuacion}\coordE{}\end{equation}
 Here the minus sign is necessary to get a positive result. In
 addition, it is assumed that we are in the strongly
 self-gravitating phase with \myHighlight{$HR \ge 1$}\coordHE{} so that \myHighlight{$H\ne 0$}\coordHE{} and 
 \myHighlight{$T_{\rm H}$}\coordHE{} is well-defined.  With this temperature, the second
 equation in (\ref{2eq2}) can be rewritten as
 \begin{equation}\coord{}\boxEquation{
 \label{2eq9}
 E_{\rm BH}=n(E+pV -T_{\rm H}S_{\rm H}).
 }{
 E_{\rm BH}=n(E+pV -T_{\rm H}S_{\rm H}).
 }{ecuacion}\coordE{}\end{equation}
 Thus we see that when the Hubble bound \myHighlight{$S_{\rm H}$}\coordHE{} is saturated by the
 matter entropy \myHighlight{$S$}\coordHE{}, the (limiting) temperature \myHighlight{$T_{\rm H}$}\coordHE{} equals
 to the thermodynamic temperature \myHighlight{$T$}\coordHE{} of matter filling the
 universe. Note that like the Hubble entropy bound,  the (limiting) temperature
 \myHighlight{$T_{\rm H}$}\coordHE{} is a geometric quantity determined by the Hubble parameter 
and its time derivative.


\subsection{The case with a positive cosmological constant}


Now we turn to the case with a nonvanishing cosmological constant,
and generalize those interesting observations to a
radiation-dominated closed FRW universe with a positive
cosmological constant. We will argue that three cosmological entropy
bounds in (\ref{2eq3}) keep the same forms even though a cosmological 
constant is present.

To write down a formula like (\ref{2eq4}), let us first
discuss three entropy bounds in (\ref{2eq3}). The
Bekenstein-Verlinde bound \myHighlight{$S_{\rm BV}$}\coordHE{} is the counterpart of the
Bekenstein entropy bound~\cite{Beke} in the cosmological
setting~\cite{CMO}. It is supposed to hold for systems with
limited self-gravity, which means that the gravitational
self-energy of the system is small compared to the total energy
\myHighlight{$E$}\coordHE{}. Namely, the gravitational effect on the bound can be
neglected. Therefore this bound is independent of gravity
theories. It is also independent of
whether the gravity theory under consideration includes or not a 
cosmological constant. In other words, the form of the Bekenstein-Verlinde
bound should keep unchanged in any gravitational theory\footnote{
In ~\cite{CM} we show that the Bekenstein entropy bound always
has the form \myHighlight{$S_{\rm B}=2\pi E R$}\coordHE{} independent of gravity
theories by applying a Geroch process to an arbitrary black hole.}. 
Hence even when a positive cosmological constant is present, the
Bekenstein-Verlinde bound still takes the form in
(\ref{2eq3}). As a result, for a radiation-dominated FRW universe with a
cosmological constant, the Bekenstein-Verlinde bound is a constant 
because of \myHighlight{$E\sim R^{-1}$}\coordHE{}. Thus once this bound is satisfied at one time,
 it will be always satisfied at all times if the entropy
 \myHighlight{$S$}\coordHE{} of matter does not change.

 As for the Bekenstein-Hawking bound in (\ref{2eq3}), it can be
 viewed as the holographic Bekenstein-Hawking entropy of a black
 hole with the size of the universe~\cite{Verl}. Indeed, it varies
 like an area instead of the volume. And for a closed universe it
 is the closest one that can lead to the usual area formula of black
 hole entropy \myHighlight{$A/4G$}\coordHE{}. We know from the thermodynamics of black
 holes that in Einstein gravity  the entropy of a black hole
 is always proportional to its horizon area in spite of whether the
 gravitational theory includes or not a cosmological
 constant~\cite{TM}. Further, as argued by Verlinde~\cite{Verl},
 the role of \myHighlight{$S_{\rm BH}$}\coordHE{} is not to serve as a bound on the total
 entropy, but rather on a sub-extensive component of the entropy
 that is associated with the Casimir energy of CFTs. The above
 leads to the conclusion that the Bekenstein-Hawking bound should
 remain unchanged in its form and implication as in the case without
 the cosmological constant.

 Finally we consider the Hubble entropy bound, which is an entropy
 bound for matter in a strongly self-gravitating universe \myHighlight{$(HR\ge1$}\coordHE{}). In such
 a strongly self-gravitating universe,  black holes might occur. 
 As argued in \cite{FS,Hubb}, the maximal
 entropy inside the universe is produced by black holes with
 size of the Hubble horizon. The usual holographic argument shows
 that the total entropy should be less than or equal to the
 Bekenstein-Hawking entropy of a Hubble-horizon-sized black hole
 times the number of Hubble regions in the universe. In
 \cite{Verl}, by using a local holographic bound due to
 Fischler and Susskind~\cite{FS} and Bousso~\cite{Bous}, see
 also \cite{Wald}, Verlinde ``derived" the Hubble entropy bound
 in (\ref{2eq3}). It is worth noting that in the ``derivation" of the Hubble 
bound, Verlinde used mainly the idea that the entropy flow \myHighlight{$S$}\coordHE{} through 
a contracting light sheet is less than or equal to \myHighlight{$A/4G$}\coordHE{}, where \myHighlight{$A$}\coordHE{} is the 
area of the surface from which the light sheet originates.
Hence we insist that the cosmological constant will not affect the form of 
the Hubble bound (see also \cite{CM}). This conclusion is based on the fact that
even if a cosmological constant is present, it will not 
occurs explicitly in the ``derivation" of Hubble entropy bound.  



We conclude that in a closed FRW universe with a cosmological
constant, three bounds introduced in (\ref{2eq3}) are still 
applicable.  That is, their forms and implications keep 
unchanged even if  the cosmological constant is present. 
However, we see that the cosmological constant indeed affects the
evolution of the universe. Is there a similar relation between the
Friedmann equation in (\ref{2eq2}) and the Cardy-Verlinde formula
(\ref{2eq6}) as in the case without the cosmological constant?  
 Our key observation is that the positive  cosmological constant provides an
additional entropy measure. When the cosmological constant
occurs, not the Hubble bound, but a new cosmological D-bound
plays the role as the Hubble bound does in the case without the
cosmological constant.

Let us go to the details. We know that in a de Sitter universe,  there is 
a cosmological horizon for an inertial observer. Like a black hole horizon, the 
cosmological horizon has a Hawking temperature and an associated entropy~\cite{GH}. 
The entropy is proportional to the area of the cosmological horizon.
 It is a geometric quantity although it has a 
statistical origin in quantum gravity.
In an asymptotically de Sitter space, the cosmological horizon
shrinks. According to the second law of thermodynamics, {\it
entropy of matter in de Sitter space} is  bounded by the
difference (D) between the entropy of de Sitter space and that of
the asymptotically de Sitter space:
\begin{equation}\coord{}\boxEquation{
\label{2eq10}
  S_{\rm m} \le \frac{1}{4 G}\left( A_0-A\right),
}{
S_{\rm m} \le \frac{1}{4 G}\left( A_0-A\right),
}{ecuacion}\coordE{}\end{equation}
where \myHighlight{$A_0$}\coordHE{} and \myHighlight{$A$}\coordHE{} are areas of cosmological horizons for  de
Sitter and asymptotically de Sitter spaces, respectively. This is
the so-called D-bound proposed by Bousso in \cite{Bousso}. The
D-bound is closely related to the Bekenstein bound which applies in
flat backgrounds~\cite{Bousso,CMO}.

In our present context, the occurrence of the cosmological
constant does not guarantee that the universe approaches to a de
Sitter phase. Like the case without the cosmological constant,
 in general the universe starts from a big bang, reaches a
maximal radius and then recollapses with a big crunch. From
(\ref{2eq2}), however, we see that for an empty flat
universe\footnote{In that case, \myHighlight{$E=p=0$}\coordHE{}, and the term \myHighlight{$1/R^2$}\coordHE{} will be
also absent.}, the Hubble radius is just the cosmological horizon size
\myHighlight{$l_{n+1}$}\coordHE{} of de Sitter space. It implies that the cosmological constant
provides a new entropy measure in the universe. By analogy with the
Hubble entropy bound~\cite{FS,Hubb,Verl}, we define a quantity 
\begin{equation}\coord{}\boxEquation{
\label{2eq11} S_{\rm \Lambda} =(n-1)\frac{V}{4G_{n+1}l_{n+1}}.
}{
S_{\rm \Lambda} =(n-1)\frac{V}{4G_{n+1}l_{n+1}}.
}{ecuacion}\coordE{}\end{equation}
which  is the entropy of a de Sitter horizon times the number of the regions 
with the size of the de Sitter horizon  in
the universe. Like the Hubble entropy bound, it is also a geometric 
quantity.  Together with the  three entropy
bounds in (\ref{2eq3}),  the Friedmann equation in
(\ref{2eq2}) can be rewritten as
\begin{equation}\coord{}\boxEquation{
\label{2eq12}
  S_{\rm H}^2-S_{\rm \Lambda}^2=S_{\rm BH}(2S_{\rm BV}-S_{\rm
  BH}).
}{
S_{\rm H}^2-S_{\rm \Lambda}^2=S_{\rm BH}(2S_{\rm BV}-S_{\rm
  BH}).
}{ecuacion}\coordE{}\end{equation}
Further we note that the cosmological horizon in the
asymptotically de Sitter spaces is always less that of corresponding de
Sitter spaces, but one can see from (\ref{2eq2}) that the Hubble radius \myHighlight{$H^{-1}$}\coordHE{} 
is not always less than the cosmological horizon \myHighlight{$l_{n+1}$}\coordHE{} of de Sitter spaces. 
As a result, the left-hand side of equation (\ref{2eq12}) is not always
positive. In addition, we stress that in the case without the
cosmological constant, the Hubble bound in (\ref{2eq5}) is a
geometry quantity, which gives an entropy bound of matter in the
universe when the universe is in the strongly self-gravitating
phase. Considering the D-bound (\ref{2eq10}) of matter in de
Sitter spaces, we can define a cosmological entropy
bound in the universe with a positive cosmological
constant\footnote{Since a black hole larger than the cosmological
horizon cannot form, one therefore should have \myHighlight{$S_{\rm H} \ge
S_{\rm \Lambda}$}\coordHE{}. As a result, if \myHighlight{$S_{\rm H} <S_{\rm \Lambda}$}\coordHE{}, a
cosmological singularity might occur during the evolution of the
universe.}
\begin{equation}\coord{}\boxEquation{
\label{2eq13}
 S_{\rm D}=\sqrt{|S_{\rm H}^2-S_{\rm
 \Lambda}^2|}.
}{
S_{\rm D}=\sqrt{|S_{\rm H}^2-S_{\rm
 \Lambda}^2|}.
}{ecuacion}\coordE{}\end{equation}
We call it the cosmological D-bound, which can be viewed as the
counterpart of the D-bound in the cosmology setting. Note that the cosmological 
D-bound is a square root of the difference between two geometric quantity 
squares, while  the D-bound in de Sitter spaces is the difference between two geometric
quantities. 

   

On the analogy of the (limiting) temperature \myHighlight{$T_{\rm H}$}\coordHE{} in (\ref{2eq8}),
we  further  define a new geometric temperature in  our case
\begin{equation}\coord{}\boxEquation{
\label{2eq14}
 T_{\rm D}= -\frac{\dot H}{2\pi
  \sqrt{|1/l_{n+1}^2-H^2|}}.
}{
T_{\rm D}= -\frac{\dot H}{2\pi
  \sqrt{|1/l_{n+1}^2-H^2|}}.
}{ecuacion}\coordE{}\end{equation}
Note that this is also a geometric quantity like \myHighlight{$T_{\rm H}$}\coordHE{} for the case without 
the cosmological constant.  With this,  the second equation 
in (\ref{2eq2}) can be expressed as
\begin{equation}\coord{}\boxEquation{
\label{2eq15}
 E_{\rm BH}=n(E +pV -T_{\rm D}S_{\rm D}).
}{
E_{\rm BH}=n(E +pV -T_{\rm D}S_{\rm D}).
}{ecuacion}\coordE{}\end{equation}
Here the definition of \myHighlight{$E_{\rm BH}$}\coordHE{} is the same as the one in
(\ref{2eq5}).

Now we turn to the Cardy-Verlinde formula (\ref{2eq6}). In the
form (\ref{2eq6}) it is implicitly assumed that one has \myHighlight{$2E-E_c \ge
0$}\coordHE{} for any CFT. In fact, in some circumstances, this condition
does not get satisfied. For example, in the CFT description dual
to the thermodynamics of Schwarzschild-de Sitter black hole
horizon, this quantity is negative~\cite{Cai2}. In that case,
the Cardy-Verlinde formula should be changed to
\begin{equation}\coord{}\boxEquation{
\label{2eq16} S=\frac{2\pi R}{n}\sqrt{E_c(E_c-2E)},
}{
S=\frac{2\pi R}{n}\sqrt{E_c(E_c-2E)},
}{ecuacion}\coordE{}\end{equation}
where the definition of \myHighlight{$E_c$}\coordHE{} is still the same as the one
(\ref{2eq7}).

In summary, for a radiation-dominated closed FRW universe with a
positive cosmological constant the dynamic equations can be
rewritten as
\begin{eqnarray}\coord{}\boxAlignEqnarray{\leftCoord{}
\label{2eq17}
&&\leftCoord{} S_{\rm D}=\frac{\leftCoord{}2\pi R}{\rightCoord{}n}\sqrt{E_{\rm
BH}(2E-E_{\rm
BH})}, \nonumber \rightCoord{}\\
&&\leftCoord{} E_{\rm BH}=n(E+pV -T_{\rm D} S_{\rm D}),
\rightCoord{}}{0mm}{4}{4}{
&& S_{\rm D}=\frac{2\pi R}{n}\sqrt{E_{\rm
BH}(2E-E_{\rm
BH})}, \\
&& E_{\rm BH}=n(E+pV -T_{\rm D} S_{\rm D}),
}{1}\coordE{}\end{eqnarray}
when \myHighlight{$S_{\rm H} \ge S_{\Lambda}$}\coordHE{}, while the entropy of the
radiation can be expressed as
\begin{eqnarray}\coord{}\boxAlignEqnarray{\leftCoord{}
\label{2eq18}
&&\leftCoord{} S =\frac{\leftCoord{}2\pi R}{\rightCoord{}n}\sqrt{E_c(2E-E_c)},
  \nonumber \rightCoord{}\\
&&\leftCoord{} E_c=n(E+pV -T S).
\rightCoord{}
\rightCoord{}}{0mm}{4}{5}{
&& S =\frac{2\pi R}{n}\sqrt{E_c(2E-E_c)},
  \\
&& E_c=n(E+pV -T S).
}{1}\coordE{}\end{eqnarray}
On the other hand, when \myHighlight{$S_{\rm H} \le S_{\rm \Lambda}$}\coordHE{}, the
dynamic equations can be rewritten as
\begin{eqnarray}\coord{}\boxAlignEqnarray{\leftCoord{}
\label{2eq19}
&&\leftCoord{} S_{\rm D}=\frac{\leftCoord{}2\pi R}{\rightCoord{}n}\sqrt{E_{\rm
BH}(E_{\rm
BH}-2E )}, \nonumber \rightCoord{}\\
&&\leftCoord{} E_{\rm BH}=n(E+pV -T_{\rm D} S_{\rm D}),
\rightCoord{}}{0mm}{4}{4}{
&& S_{\rm D}=\frac{2\pi R}{n}\sqrt{E_{\rm
BH}(E_{\rm
BH}-2E )}, \\
&& E_{\rm BH}=n(E+pV -T_{\rm D} S_{\rm D}),
}{1}\coordE{}\end{eqnarray}
and the entropy expressions are
\begin{eqnarray}\coord{}\boxAlignEqnarray{\leftCoord{}
\label{2eq20}
&&\leftCoord{} S =\frac{\leftCoord{}2\pi R}{\rightCoord{}n}\sqrt{E_c(E_c-2E)},
  \nonumber \rightCoord{}\\
&&\leftCoord{} E_c=n(E+pV -T S).
\rightCoord{}
\rightCoord{}}{0mm}{4}{5}{
&& S =\frac{2\pi R}{n}\sqrt{E_c(E_c-2E)},
  \\
&& E_c=n(E+pV -T S).
}{1}\coordE{}\end{eqnarray}
When the cosmological D-bound is saturated by the
entropy \myHighlight{$S$}\coordHE{} of radiation matter, both sets of equations (\ref{2eq17}) [or
(\ref{2eq19})] and (\ref{2eq18}) [or (\ref{2eq20})] coincide with
each other, just like the case without the cosmological constant. Further,
as the D-bound in (\ref{2eq10}) does, the cosmological D-bound in (\ref{2eq13}) 
provides an entropy bound for matter filling the universe when the universe
is in the strongly self-gravitating phase (\myHighlight{$HR\ge 1$}\coordHE{})\footnote{When the universe is in
the weakly self-gravitating phase (\myHighlight{$HR \le 1$}\coordHE{}), as argued in the above, the 
Bekenstein-Verlinde bound still works well.}. Namely, the cosmological D-bound
plays the same role as the Hubble bound does in the case without cosmological 
constant. Furthermore we note from (\ref{2eq2}) that one has \myHighlight{$S_{\rm D}=
S_{\rm BV}=S_{\rm BH}$}\coordHE{} when \myHighlight{$HR=1$}\coordHE{}.



%%=====================section 3=======================
\sect{Brane cosmology in the background of Schwarzschild-de Sitter
black holes}

\subsection{Thermodynamics of Schwarzschild-de Sitter black holes}

Consider a (\myHighlight{$n+2$}\coordHE{})-dimensional Schwarzschild-de Sitter black hole,
whose line element is
\begin{equation}\coord{}\boxEquation{
\label{3eq1} ds^2 =- f(r) dt^2 +f(r)^{-1}dr^2 +r^2 d\Omega_{n}^2.
}{
ds^2 =- f(r) dt^2 +f(r)^{-1}dr^2 +r^2 d\Omega_{n}^2.
}{ecuacion}\coordE{}\end{equation}
Here
$$\coord{}\boxMath{ f(r) =1 -\frac{\omega_n M}{r^{n-1}} -\frac{r^2}{l_{n+2}^2}, \ \ \
 \omega_n=\frac{16\pi G_{n+2}}{n \Omega_n},}{dollar}{0pt}\coordE{}$$
\myHighlight{$M$}\coordHE{} stands for the mass of the Schwarzschild-de Sitter black hole
in the definition due to Abbott and Deser~\cite{AD}, \myHighlight{$G_{n+2}$}\coordHE{}
denotes the \myHighlight{$(n+2)$}\coordHE{}-dimensional Newton constant, and
\myHighlight{$l_{n+2}$}\coordHE{} represents the cosmological radius of the
(\myHighlight{$n+2$}\coordHE{})-dimensional de Sitter universe. When \myHighlight{$M=0$}\coordHE{}, the solution
(\ref{3eq1}) reduces to a  de Sitter space with a cosmological
horizon at \myHighlight{$r_c=l_{n+2}$}\coordHE{}. When \myHighlight{$M$}\coordHE{} increases from \myHighlight{$M=0$}\coordHE{}, a black
hole horizon appears and grows, while the cosmological horizon shrinks. 
Finally the black hole horizon \myHighlight{$r_{\rm BH}$}\coordHE{} touches the cosmological 
horizon \myHighlight{$r_{\rm CH}$}\coordHE{} when
$$\coord{}\boxMath{ M =M_N \equiv \frac{2}{\omega_n (n+1)}\left
(\frac{n-1}{n+1}l^2_{n+2}\right)^{(n-1)/2}.}{dollar}{0pt}\coordE{}$$
 This is the Nariai black hole, the maximal black hole in de
 Sitter space. When \myHighlight{$M >M_N$}\coordHE{}, both the two horizons disappear and
 the solution describes a naked singularity. When \myHighlight{$M <M_N$}\coordHE{}, the
 equation \myHighlight{$f(r)=0$}\coordHE{} has two real roots, the larger one is the
 cosmological horizon, while the smaller one is the black hole
 horizon.

  
The Hawking temperature \myHighlight{$T_{\rm HK}$}\coordHE{} and entropy \myHighlight{$S$}\coordHE{} associated with the
  black hole horizon are~\cite{Cai2}
  \begin{equation}\coord{}\boxEquation{
  \label{3eq2}
  T_{\rm HK} =\frac{1}{4\pi
  r_{\rm BH}}\left((n-1)-(n+1)\frac{r_{\rm BH}^2}{l^2_{n+2}}\right), \ \ \ \
 S =\frac{r_{\rm BH}^n \Omega_n}{4G_{n+2}}.
}{
  T_{\rm HK} =\frac{1}{4\pi
  r_{\rm BH}}\left((n-1)-(n+1)\frac{r_{\rm BH}^2}{l^2_{n+2}}\right), \ \ \ \
 S =\frac{r_{\rm BH}^n \Omega_n}{4G_{n+2}}.
}{ecuacion}\coordE{}\end{equation}
With the identification \myHighlight{$E=M$}\coordHE{} and the definition
\myHighlight{$E_c=(n+1)E-nT_{\rm HK}S~$}\coordHE{}\footnote{Here it is assumed that the
thermodynamics of the Schwarzschild-de Sitter black hole can be
described in terms of a CFT.}, one can easily obtain
\begin{equation}\coord{}\boxEquation{
 E_c =\frac{2nr_{\rm BH}^{n-1}\Omega_n}{16\pi G_{n+2}}, \ \ \
 2E-E_c =-\frac{2n r_{\rm BH}^{n+1} \Omega_n}{16\pi G_{n+2}l^2_{n+2}}.
 }{
 E_c =\frac{2nr_{\rm BH}^{n-1}\Omega_n}{16\pi G_{n+2}}, \ \ \
 2E-E_c =-\frac{2n r_{\rm BH}^{n+1} \Omega_n}{16\pi G_{n+2}l^2_{n+2}}.
 }{ecuacion}\coordE{}\end{equation}
Clearly the entropy (\ref{3eq2}) can be expressed by the
Cardy-Verlinde formula~\cite{Cai2}
\begin{equation}\coord{}\boxEquation{
S =\frac{2\pi l_{n+2}}{n} \sqrt{E_c (E_c-2E)}.
}{
S =\frac{2\pi l_{n+2}}{n} \sqrt{E_c (E_c-2E)}.
}{ecuacion}\coordE{}\end{equation}
If one rescales the energies by a factor \myHighlight{$R/l_{n+2}$}\coordHE{}, the above equation
is changed to
\begin{equation}\coord{}\boxEquation{
\label{3eq5} S =\frac{2\pi R}{n}\sqrt{E_c(E_c-2E)}.
}{
S =\frac{2\pi R}{n}\sqrt{E_c(E_c-2E)}.
}{ecuacion}\coordE{}\end{equation}
Note that this expression is  exactly the same as the entropy formula in
(\ref{2eq20}).


\subsection{Brane dynamics in the background of Schwarzschild-de
Sitter black holes}

Let us introduce a \myHighlight{$(n+1)$}\coordHE{}-dimensional brane with tension \myHighlight{$\sigma $}\coordHE{} moving
in the background of the Schwarzschild-de Sitter black holes
(\ref{3eq1}). Its dynamics is determined by the following
action~\cite{SV,Wall}
\begin{equation}\coord{}\boxEquation{
\label{3eq6} S_{\rm brane}=\frac{1}{8\pi G_{n+2}}\int_{\partial
M}d^{n+1}x \sqrt{-h}K +\frac{1}{8\pi G_{n+2}}\int_{\partial
M}d^{n+1}x\sqrt{-h} \sigma.
}{
S_{\rm brane}=\frac{1}{8\pi G_{n+2}}\int_{\partial
M}d^{n+1}x \sqrt{-h}K +\frac{1}{8\pi G_{n+2}}\int_{\partial
M}d^{n+1}x\sqrt{-h} \sigma.
}{ecuacion}\coordE{}\end{equation}
Here the brane is viewed as boundary of the bulk spacetime
(\ref{3eq1}), \myHighlight{$K$}\coordHE{} is the extrinsic curvature for the boundary with
the induced metric \myHighlight{$h_{ab}$}\coordHE{}. The equation of motion for the brane is
\begin{equation}\coord{}\boxEquation{
\label{3eq7}
 K_{ab}=\frac{\sigma}{n}h_{ab}.
}{
K_{ab}=\frac{\sigma}{n}h_{ab}.
}{ecuacion}\coordE{}\end{equation}
The brane cosmology in the Schwarzschild-de Sitter black holes has
been first considered in \cite{Ogush}. The holography in brane
cosmology in various asymptotically de Sitter spaces has  also
been discussed in~\cite{Ogush,deSitter}\footnote{However, our philosophy 
in understanding the holography in the case with a cosmological 
constant is different from those in \cite{Ogush,deSitter}. We will discuss this point
at the end of this paper.}.
 Now let us specify the location
of the brane as \myHighlight{$r=r(t)$}\coordHE{}. We introduce a cosmic time \myHighlight{$\tau$}\coordHE{} so
that \myHighlight{$t=t(\tau)$}\coordHE{} and \myHighlight{$r=r(\tau)$}\coordHE{} and require
\begin{equation}\coord{}\boxEquation{
\label{3eq8}
 f(r)\left(\frac{dt}{d\tau}\right)^2
 -\frac{1}{f(r)}\left(\frac{dr}{d\tau}\right)^2 =1,
 }{
f(r)\left(\frac{dt}{d\tau}\right)^2
 -\frac{1}{f(r)}\left(\frac{dr}{d\tau}\right)^2 =1,
 }{ecuacion}\coordE{}\end{equation}
 which implies that the brane moves along a radial time-like geodesic
in the background (\ref{3eq1})\footnote{The dynamics of brane
along a radial space-like geodesic in various asymptotically de Sitter 
backgrounds has also been discussed in \cite{Ogush,deSitter}.}. In that case, the
induced metric \myHighlight{$h_{ab}$}\coordHE{} on the brane becomes
\begin{equation}\coord{}\boxEquation{
\label{3eq9} ds^2 =-d\tau^2 +R^2(\tau)d\Omega_n^2,
}{
ds^2 =-d\tau^2 +R^2(\tau)d\Omega_n^2,
}{ecuacion}\coordE{}\end{equation}
which is just a (\myHighlight{$n+1)$}\coordHE{}-dimensional closed FRW universe metric
(\ref{2eq1}) with  scale factor \myHighlight{$R(\tau)=r(\tau)$}\coordHE{}.

Calculating the extrinsic curvature for the brane and then from the
equation (\ref{3eq7}), we have
\begin{equation}\coord{}\boxEquation{
\label{3eq10}
 \frac{dt}{d\tau}= \frac{\sigma R}{nf(R)}.
}{
\frac{dt}{d\tau}= \frac{\sigma R}{nf(R)}.
}{ecuacion}\coordE{}\end{equation}
Substituting into (\ref{3eq8}) yields
\begin{equation}\coord{}\boxEquation{
\label{3eq11}
H^2 = \frac{\omega_n M}{R^{n+1}}
 -\frac{1}{R^2}+\frac{1}{l_{n+2}^2} +\frac{\sigma^2}{n^2},
 }{
H^2 = \frac{\omega_n M}{R^{n+1}}
 -\frac{1}{R^2}+\frac{1}{l_{n+2}^2} +\frac{\sigma^2}{n^2},
 }{ecuacion}\coordE{}\end{equation}
 The time derivative of the Hubble parameter is
 \begin{equation}\coord{}\boxEquation{
 \label{3eq12}
\dot H =-\frac{(n+1)\omega_n M}{2R^{n+1}} +\frac{1}{R^2}.
}{
 \dot H =-\frac{(n+1)\omega_n M}{2R^{n+1}} +\frac{1}{R^2}.
}{ecuacion}\coordE{}\end{equation}
These equations just describe a radiation-dominated closed FRW
universe with a positive cosmological constant \myHighlight{$\Lambda_{n+1}=
n(n-1)/2l^2_{n+1}$}\coordHE{} with
\begin{equation}\coord{}\boxEquation{
 \label{3eq13}
 \frac{1}{l_{n+1}^2}=
    \frac{1}{l_{n+2}^2} +
       \frac{\sigma^2}{n^2},
 }{
 \frac{1}{l_{n+1}^2}=
    \frac{1}{l_{n+2}^2} +
       \frac{\sigma^2}{n^2},
 }{ecuacion}\coordE{}\end{equation}
 from which we see \myHighlight{$l^2_{n+1} < l^2_{n+2}$}\coordHE{}. Now we consider the solution
  of (\ref{3eq11}). As an example, let us discuss the special case of  
\myHighlight{$n=3$}\coordHE{}. The generalization to other dimensions is straightforward. In that 
case, the solution has been found in three different cases depending 
on the parameter \myHighlight{$\omega_4 M/l^2_{n+1}$}\coordHE{} in
 \cite{PS}, where the authors discussed the dynamics of  a noncritical
 brane in the Schwarzschild-AdS black hole. Defining \myHighlight{$x=R^2$}\coordHE{}, we
 can rewrite (\ref{3eq11}) as
 \begin{equation}\coord{}\boxEquation{
 \label{3eq14}
 \dot x^2 =\frac{4}{l_4^2}(x-x_+)(x-x_-),
 }{
 \dot x^2 =\frac{4}{l_4^2}(x-x_+)(x-x_-),
 }{ecuacion}\coordE{}\end{equation}
where
\begin{equation}\coord{}\boxEquation{
x_{\pm} =\frac{l^2_4}{2}\left(1 \pm
\sqrt{1-4\omega_4M/l_4^2}\right).
}{
x_{\pm} =\frac{l^2_4}{2}\left(1 \pm
\sqrt{1-4\omega_4M/l_4^2}\right).
}{ecuacion}\coordE{}\end{equation}
Note that when \myHighlight{$x=x_{\pm}$}\coordHE{}, one has \myHighlight{$H=0$}\coordHE{}. Actually, \myHighlight{$x_{\pm}$}\coordHE{} are turning 
points of the brane.

(1) When \myHighlight{$4\omega_4 M =l^2_4$}\coordHE{}, one has \myHighlight{$x_+=x_-$}\coordHE{}.  In this case, the brane 
has only one turning point. However, the solution has
two branches:
\begin{itemize}
\item  \myHighlight{$ x \in [x_+,\infty)$}\coordHE{}. In this case, the solution is given by
   \begin{equation}\coord{}\boxEquation{
   \label{3eq16}
   R^2 =x(\tau)= \frac{l^2_4}{2}\left(1 +e^{2\tau/l_4}\right), \ \
    \ \tau \in (-\infty, \infty).
   }{
   R^2 =x(\tau)= \frac{l^2_4}{2}\left(1 +e^{2\tau/l_4}\right), \ \
    \ \tau \in (-\infty, \infty).
   }{ecuacion}\coordE{}\end{equation}
   And the Hubble parameter takes the expression
   \begin{equation}\coord{}\boxEquation{
   H =\frac{1}{l_4}\frac{e^{2\tau/l_4}}{1+e^{2\tau/l_4}}.
   }{
   H =\frac{1}{l_4}\frac{e^{2\tau/l_4}}{1+e^{2\tau/l_4}}.
   }{ecuacion}\coordE{}\end{equation}
   Clearly in this case one has \myHighlight{$0< H < 1/l_4$}\coordHE{}.

\item  \myHighlight{$x \in (0, x_+]$}\coordHE{}. In this case the solution is
\begin{equation}\coord{}\boxEquation{
   R^2(\tau) = \frac{l_4^2}{2}\left(1-e^{-2\tau/l_4}\right),
     \ \ \  \tau \in [0, \infty).
}{
   R^2(\tau) = \frac{l_4^2}{2}\left(1-e^{-2\tau/l_4}\right),
     \ \ \  \tau \in [0, \infty).
}{ecuacion}\coordE{}\end{equation}
Accordingly the Hubble parameter reads
\begin{equation}\coord{}\boxEquation{
H= \frac{1}{l_4}\frac{e^{-2\tau/l_4}}{1-e^{-2\tau/l_4}}.
}{
H= \frac{1}{l_4}\frac{e^{-2\tau/l_4}}{1-e^{-2\tau/l_4}}.
}{ecuacion}\coordE{}\end{equation}
In this branch, \myHighlight{$ 0\le H < \infty$}\coordHE{}.  Since  the
black hole horizon \myHighlight{$x_{\rm BH}$}\coordHE{} falls in the range of \myHighlight{$0 <x_{\rm BH}
< x_+$}\coordHE{},  we  find \myHighlight{$ H <1/l_4$}\coordHE{} in the range of \myHighlight{$ x_{\rm BH} < x < x_+$}\coordHE{}.

\end{itemize}

(2) When \myHighlight{$4\omega_4 M > l_4^2$}\coordHE{}, the brane has no turning point. 
Namely, there is no point which has \myHighlight{$H=0$}\coordHE{} along the geodesic of 
the brane. In this case one has the solution
    \begin{equation}\coord{}\boxEquation{
   \label{3eq20}
    R^2(\tau)= \frac{l_4^2}{2}\left (1 +\sqrt{4\omega_4M/l_4^2
    -1}\sinh( 2\tau /l_4)\right),
    }{
   R^2(\tau)= \frac{l_4^2}{2}\left (1 +\sqrt{4\omega_4M/l_4^2
    -1}\sinh( 2\tau /l_4)\right),
    }{ecuacion}\coordE{}\end{equation}
    where \myHighlight{$\tau_0 \le \tau < \infty$}\coordHE{} with
    $$\coord{}\boxMath{ \sinh(2\tau_0/l_4) =-\left(4\omega_4 M/l_4^2-1
      \right)^{-1/2}.}{dollar}{0pt}\coordE{}$$
   The evolution of the Hubble parameter is given by
   \begin{equation}\coord{}\boxEquation{
   H
   =\frac{1}{l_4}\frac{\sqrt{4\omega_4M/l_4^2-1}\cosh(2\tau/l_4)}
    {1+\sqrt{4\omega_4M/l_4-1}\sinh(2\tau/l_4)}.
    }{
   H
   =\frac{1}{l_4}\frac{\sqrt{4\omega_4M/l_4^2-1}\cosh(2\tau/l_4)}
    {1+\sqrt{4\omega_4M/l_4-1}\sinh(2\tau/l_4)}.
    }{ecuacion}\coordE{}\end{equation}
 We find that outside the black hole horizon, \myHighlight{$H<1/l_4$}\coordHE{}. Actually, there 
exists another solution for the equation (\ref{3eq14}). But this solution 
describes the same movement of the brane as the solution (\ref{3eq20}) 
does. So we do not present it here. 


    (3) When \myHighlight{$4\omega_4 M <l_4^2$}\coordHE{}, the brane has two turning 
 points \myHighlight{$x_{\pm}$}\coordHE{}. The range \myHighlight{$x \in [x_-,x_+]$}\coordHE{} is not allowed  since in which 
\myHighlight{$H^2 \le 0$}\coordHE{}. As a result,  solution of 
  equation (\ref{3eq14}) has   two branches:
  \begin{itemize}
  \item \myHighlight{$x\in (0, x_-]$}\coordHE{}. The solution is
  \begin{equation}\coord{}\boxEquation{
  R^2(\tau)= \frac{l_4^2}{2}\left(1
  -\sqrt{1-4\omega_4M/l_4^2}\cosh(2\tau/l_4)\right),
  }{
  R^2(\tau)= \frac{l_4^2}{2}\left(1
  -\sqrt{1-4\omega_4M/l_4^2}\cosh(2\tau/l_4)\right),
  }{ecuacion}\coordE{}\end{equation}
  where \myHighlight{$\tau$}\coordHE{} takes value in the range \myHighlight{$-\tau_c \le \tau \le
  \tau_c$}\coordHE{} with
\begin{equation}\coord{}\boxEquation{
 \cosh(2\tau_c/l_4)=(1-4\omega_4M/l_4)^{-1/2}.
 }{
 \cosh(2\tau_c/l_4)=(1-4\omega_4M/l_4)^{-1/2}.
 }{ecuacion}\coordE{}\end{equation}
  When \myHighlight{$\tau =\pm \tau_c$}\coordHE{}, one has \myHighlight{$R=0$}\coordHE{} and \myHighlight{$H= \infty$}\coordHE{}.
  The Hubble parameter
  \begin{equation}\coord{}\boxEquation{
  H
  =\frac{1}{l_4}\frac{\sqrt{1-4\omega_4 M/l_4^2}\sinh(2\tau/l_4)}
    {1-\sqrt{1-4\omega_4M/l_4^2}\cosh(2\tau/l_4)},
    }{
  H
  =\frac{1}{l_4}\frac{\sqrt{1-4\omega_4 M/l_4^2}\sinh(2\tau/l_4)}
    {1-\sqrt{1-4\omega_4M/l_4^2}\cosh(2\tau/l_4)},
    }{ecuacion}\coordE{}\end{equation}
    from which we see that \myHighlight{$H=0$}\coordHE{} when \myHighlight{$\tau=0$}\coordHE{}.
Note that because of the relation (\ref{3eq13}),
it is easy to see that the black hole horizon \myHighlight{$x_{\rm BH} \in
 (0, x_-)$}\coordHE{}, while the cosmological horizon 
\myHighlight{$x_{\rm CH} \in (x_-, x_+)$}\coordHE{}.
 Thus we find that in this branch one has \myHighlight{$H <1/l_4$}\coordHE{} as
 the brane stays outside the black hole horizon  of the bulk
 background.

  \item \myHighlight{$x\in[x_+,\infty)$}\coordHE{}. In this branch the solution is
  \begin{equation}\coord{}\boxEquation{
  R^2(\tau)=\frac{l^2_4}{2}\left( 1+\sqrt{1-4\omega_4
  M/l_4^2}\cosh(2\tau/l_4)\right), \ \ \ \tau \in (-\infty,
  \infty).
  }{
  R^2(\tau)=\frac{l^2_4}{2}\left( 1+\sqrt{1-4\omega_4
  M/l_4^2}\cosh(2\tau/l_4)\right), \ \ \ \tau \in (-\infty,
  \infty).
  }{ecuacion}\coordE{}\end{equation}
  And the Hubble parameter is given by
  \begin{equation}\coord{}\boxEquation{
  H =\frac{1}{l_4}\frac{\sqrt{1-4\omega_4 M/l_4^2}\sinh(2\tau/l_4)}
    {1+\sqrt{1-4\omega_4M/l_4^2}\cosh(2\tau/l_4)}.
    }{
  H =\frac{1}{l_4}\frac{\sqrt{1-4\omega_4 M/l_4^2}\sinh(2\tau/l_4)}
    {1+\sqrt{1-4\omega_4M/l_4^2}\cosh(2\tau/l_4)}.
    }{ecuacion}\coordE{}\end{equation}
  At \myHighlight{$\tau=0$}\coordHE{}, one has \myHighlight{$H=0$}\coordHE{}.
\end{itemize}
We conclude that the evolution of the brane depends on value of the
parameter \myHighlight{$4\omega_4M/l_4^2$}\coordHE{} and its initial position. Since we
are interested in a radiation-dominated cosmology beginning with a
big bang, so in addition to the solution in case (2),  
the solutions in the branch \myHighlight{$x \in (0,x_+]$}\coordHE{} of  case (1) and in the branch 
\myHighlight{$x \in (0, x_-]$}\coordHE{}  of case (3) are suitable, respectively, for our purpose. 
Inspecting the three appropriate  solutions, we find that  \myHighlight{$H <1/l_4$}\coordHE{} always 
holds when the brane stays outside the
bulk black hole horizon. Further we mention that in the above discussions, 
the condition \myHighlight{$4\omega_4M <l^2_5$}\coordHE{} is assumed to hold, which implies that the
black hole horizon is always present.

 
\subsection{Holography in the brane cosmology}

In brane world scenario with an AdS bulk, the tension
of the brane can be adjusted to result in a so-called critical
brane on which the effective cosmological constant
vanishes~\cite{SV}. In the present case, one can see from (\ref{3eq13})
that it is impossible to obtain a vanishing cosmological constant on the 
brane. Now we set
\begin{equation}\coord{}\boxEquation{
\label{3eq27}
 \sigma = n/l_{n+2}.
}{
\sigma = n/l_{n+2}.
}{ecuacion}\coordE{}\end{equation}
In that case the Newton constant on the brane has the
relation
\begin{equation}\coord{}\boxEquation{
\label{3eq28}
 G_{n+1} = \frac{n-1}{ l_{n+2}}G_{n+2},
}{
G_{n+1} = \frac{n-1}{ l_{n+2}}G_{n+2},
}{ecuacion}\coordE{}\end{equation}
to the Newton constant in the bulk. This relation is the same
as that for a critical brane in AdS bulk~\cite{SV}.
Furthermore the parameter \myHighlight{$M$}\coordHE{} in the solution (\ref{3eq1}) is the
black hole mass measured in the bulk coordinates~\cite{AD}.
According to the relation (\ref{3eq10}), the holographic energy
\myHighlight{$E$}\coordHE{} measured on the brane is\footnote{Due to existence of the
cosmological horizon in the bulk, this relation is not justified
well as the case for the AdS bulk~\cite{SV}. Even for the latter
case, there exists a different viewpoint, for example, see
\cite{Padi}, which argued that this relation holds only near the
boundary of AdS space. However, we note that the rescaling
(\ref{3eq29}) indeed gives a scale relation for a radiation matter
in universe. Further, the relation (\ref{3eq29}) holds at least for 
small black holes as in the AdS case.}
\begin{equation}\coord{}\boxEquation{
\label{3eq29}
 E =\frac{l_{n+2}}{R}M.
}{
E =\frac{l_{n+2}}{R}M.
}{ecuacion}\coordE{}\end{equation}
Substituting (\ref{3eq27}), (\ref{3eq28}) and (\ref{3eq29}) into
(\ref{3eq11}) and (\ref{3eq12}), we have
\begin{eqnarray}\coord{}\boxAlignEqnarray{\leftCoord{}
\label{3eq30}
&&\leftCoord{} H^2 =\frac{\leftCoord{}16\pi G_{n+1}}{n(n-1)}\frac{\leftCoord{}E}{\rightCoord{}V}
\leftCoord{}-\frac{\leftCoord{}1}{\rightCoord{}R^2} \rightCoord{}
     \leftCoord{}+\frac{\leftCoord{}1}{\rightCoord{}l^2_{n+1}}, \nonumber \rightCoord{}\\
&&\leftCoord{} \dot H =-\frac{\leftCoord{}8\pi G_{n+1}}{n-1}\left (\frac{\leftCoord{}E}{\rightCoord{}V} +p\right)
    \leftCoord{}+\frac{\leftCoord{}1}{\rightCoord{}R^2}, \rightCoord{}
\rightCoord{}}{0mm}{13}{10}{
&& H^2 =\frac{16\pi G_{n+1}}{n(n-1)}\frac{E}{V}
-\frac{1}{R^2} 
     +\frac{1}{l^2_{n+1}}, \\
&& \dot H =-\frac{8\pi G_{n+1}}{n-1}\left (\frac{E}{V} +p\right)
    +\frac{1}{R^2}, 
}{1}\coordE{}\end{eqnarray}
with \myHighlight{$l^2_{n+1}=l^2_{n+2}/2$}\coordHE{} and \myHighlight{$p=E/nV$}\coordHE{}. These two equations are the same 
as the ones in
(\ref{2eq2}).  The equation of state \myHighlight{$p=E/nV$}\coordHE{} is just the one for radiation 
matter (or more general CFTs). As a result, the discussions on holography 
in Sec.~2 are applicable here.

Suppose the brane moves between the bulk black hole horizon and
cosmological horizon\footnote{In case (2) the brane can cross the
cosmological horizon. We have not yet well understood the
holographic connection when the brane crosses the bulk cosmological horizon. 
In case (1) and (3) the
brane is always inside the cosmological horizon for branches of
interest.}. Since the brane is viewed as the boundary of the bulk
spacetime, the entropy of holographic matter (radiation)
on the brane is just the entropy of black hole horizon, which is a
constant during the evolution of the brane universe. However, the
entropy density varies with time as
\begin{equation}\coord{}\boxEquation{
\label{3eq31}
 s \equiv \frac{S}{V}=\frac{r_{\rm BH}^n}{4G_{n+2}R^n}
    =\frac{(n-1)r_{\rm BH}^n}{4G_{n+1}l_{n+2}R^n},
}{
s \equiv \frac{S}{V}=\frac{r_{\rm BH}^n}{4G_{n+2}R^n}
    =\frac{(n-1)r_{\rm BH}^n}{4G_{n+1}l_{n+2}R^n},
}{ecuacion}\coordE{}\end{equation}
and the energy density of radiation-matter
\begin{equation}\coord{}\boxEquation{
\rho \equiv \frac{E}{V}=\frac{n r_{\rm BH}^{n-1}l_{n+2}}{16\pi G_{n+2}
R^{n+1}} \left (1-\frac{r_{\rm BH}^2}{l^2_{n+2}}\right),
}{
\rho \equiv \frac{E}{V}=\frac{n r_{\rm BH}^{n-1}l_{n+2}}{16\pi G_{n+2}
R^{n+1}} \left (1-\frac{r_{\rm BH}^2}{l^2_{n+2}}\right),
}{ecuacion}\coordE{}\end{equation}
in terms of the black hole horizon radius \myHighlight{$r_{\rm BH}$}\coordHE{}. Further from the
scaling relation (\ref{3eq29}), the temperature \myHighlight{$T$}\coordHE{} on
the brane is given by
\begin{equation}\coord{}\boxEquation{
\label{3eq33}
 T=\frac{l_{n+2}}{R}T_{\rm HK}=\frac{l_{n+2}}{4\pi
  r_{\rm BH}R}\left((n-1)-(n+1)\frac{r_{\rm BH}^2}{l^2_{n+2}}\right).
}{
T=\frac{l_{n+2}}{R}T_{\rm HK}=\frac{l_{n+2}}{4\pi
  r_{\rm BH}R}\left((n-1)-(n+1)\frac{r_{\rm BH}^2}{l^2_{n+2}}\right).
}{ecuacion}\coordE{}\end{equation}
Applying the first law of thermodynamics to the radiation matter
in the brane universe, one has
\begin{equation}\coord{}\boxEquation{
Tds = d\rho +n(\rho +p -T s)\frac{dR}{R}.
}{
Tds = d\rho +n(\rho +p -T s)\frac{dR}{R}.
}{ecuacion}\coordE{}\end{equation}
Following \cite{SV} and defining
\begin{equation}\coord{}\boxEquation{
\gamma = \frac{n}{2}(\rho +p -Ts)R^2,
}{
\gamma = \frac{n}{2}(\rho +p -Ts)R^2,
}{ecuacion}\coordE{}\end{equation}
we have
\begin{equation}\coord{}\boxEquation{
\label{3eq36}
 \gamma =\frac{nr_{\rm BH}^{n-1}l_{n+2}}{16\pi
G_{n+2}R^{n-1}}
       =\frac{n(n-1)r_{\rm BH}^{n-1}}{16\pi G_{n+1} R^{n-1}}.
 }{
\gamma =\frac{nr_{\rm BH}^{n-1}l_{n+2}}{16\pi
G_{n+2}R^{n-1}}
       =\frac{n(n-1)r_{\rm BH}^{n-1}}{16\pi G_{n+1} R^{n-1}}.
 }{ecuacion}\coordE{}\end{equation}
Then the entropy density (\ref{3eq31}) can be
expressed as
\begin{equation}\coord{}\boxEquation{
\label{3eq37}
 s=\frac{4\pi }{n}\sqrt{\gamma
\left(\frac{\gamma}{R^2}-\rho \right)}.
}{
s=\frac{4\pi }{n}\sqrt{\gamma
\left(\frac{\gamma}{R^2}-\rho \right)}.
}{ecuacion}\coordE{}\end{equation}
Now we consider a special moment that the brane crosses the bulk
black hole horizon. In that time, one has \myHighlight{$R=r_{\rm BH}$}\coordHE{}. From
(\ref{3eq11}) we see
\begin{equation}\coord{}\boxEquation{
H^2 =\frac{1}{l_{n+2}^2}.
}{
H^2 =\frac{1}{l_{n+2}^2}.
}{ecuacion}\coordE{}\end{equation}
Note that we have taken \myHighlight{$\sigma^2/n^2 =1/l_{n+2}^2$}\coordHE{}. At that moment
the cosmological D-bound in (\ref{2eq13}) turns out to be
\begin{equation}\coord{}\boxEquation{
S_{\rm D}= \frac{( n-1)\Omega_n}{4G_{n+1}l_{n+2}}
             =\frac{ r_{\rm BH}^n \Omega_n}{4G_{n+2}},
}{
S_{\rm D}= \frac{( n-1)\Omega_n}{4G_{n+1}l_{n+2}}
             =\frac{ r_{\rm BH}^n \Omega_n}{4G_{n+2}},
}{ecuacion}\coordE{}\end{equation}
which is just the black hole horizon entropy (\ref{3eq2}). That
is, when the brane  crosses the bulk black hole horizon,
the cosmological D-bound is saturated by the entropy of the bulk
black hole. At that time, the geometric temperature in
(\ref{2eq14}) is given by
\begin{equation}\coord{}\boxEquation{
T_{\rm D} =\frac{l_{n+2}}{4\pi r_{\rm BH}^2}\left ((n-1) -(n+1)
\frac{r_{\rm BH}^2}{l_{n+2}^2}\right),
}{
T_{\rm D} =\frac{l_{n+2}}{4\pi r_{\rm BH}^2}\left ((n-1) -(n+1)
\frac{r_{\rm BH}^2}{l_{n+2}^2}\right),
}{ecuacion}\coordE{}\end{equation}
which equals  the temperature (\ref{3eq33}) of radiation filling
the universe. Furthermore,  at that moment the
FRW equations (\ref{2eq19}) coincide with the equations
(\ref{3eq37}) and (\ref{3eq36}) which describe the entropy of
radiation matter in the brane universe with a positive
cosmological constant. Thus we reach the same conclusion as in the case
without the cosmological constant~\cite{SV}.



%%=========================section 4==================
\section{Conclusion and discussion}

We  have discussed the holography in a radiation-dominated, closed
FRW universe with a positive cosmological constant. By introducing the
cosmological D-bound (\ref{2eq13}) on entropy of matter in the universe,
the Friedmann equation describing the evolution of the universe can be 
rewritten in the form of Cardy-Verlinde formula which describes the degree
of freedom of radiation matter filling the universe. When the cosmological
D-bound is saturated by the entropy of matter, these two equations coincide
with each other. Thus we have successfully generalized  
interesting observations by Verlinde~\cite{Verl,SV} on the holographic 
connection between the Friedmann equation and Cardy-Verlinde formula
to the case with a positive cosmological constant. By considering brane 
cosmology in the background of Schwarzschild-de Sitter black holes, we have 
found that the cosmological D-bound is saturated when the brane crosses 
the black hole horizon in the bulk background. At that moment, the Friedmann
equation and Cardy-Verlinde formula coincide with each other, and the 
introduced geometric temperature \myHighlight{$T_{\rm D}$}\coordHE{} in (\ref{2eq14}) equals  the 
thermodynamic temperature \myHighlight{$T$}\coordHE{} of the radiation matter.

We stress that when discussing the holographic connection in the brane cosmology 
in the Schwarzschild-de Sitter black holes, we have taken a special 
tension (\ref{3eq27}), which is the same as in the case for critical brane 
in the AdS bulk. Only in that case, the Friedmann equation and Cardy-Verlinde
formula coincide with each other very well. We point out here that if the brane 
tension is arbitrary, the Friedmann equation still has a  form as the
Cardy-Verlinde formula,  but a factor \myHighlight{$\sigma l_{n+2}/n$}\coordHE{} will
appear in (\ref{3eq37}). Furthermore, for the radiation matter dual to the black holes
in de Sitter spaces, we see from (\ref{3eq37}) that \myHighlight{$\rho - \gamma/R^2 <0$}\coordHE{}. If 
 considering a noncritical brane cosmology in the Schwarzschild-AdS black holes,  
 one will see that in that case \myHighlight{$\rho-\gamma/R^2>0$}\coordHE{}. This case corresponds to the 
holographic connection described by equations (\ref{2eq17}) and (\ref{2eq18}).
In addition, if the cosmological constant becomes negative, the minus sign in front of
\myHighlight{$S_{\rm \Lambda}$}\coordHE{} has to be changed to plus. The quantity (\ref{2eq11})
then will lose its interpretation, but all formulas will still work well.


We have noticed that the holography was discussed in many  literatures in the
case with a cosmological constant, for example, 
see \cite{Ogush,deSitter,Padi,Wang,Nojiri,Youm,Medved}. However, our understanding
is different from those in existing literatures: in some papers the cosmological
constant term is incorporated to the Bekenstein-Verlinde entropy bound; in some
papers this term is kept as an independent term. In those literatures the Friedmann
equation and the Cardy-Verlinde formula have not a same form, and when the Hubble 
bound is saturated, these two formulas do not get matched.  Finally we point out
that at the end of paper~\cite{Verl}, Verlinde mentioned that when the cosmological constant
does not vanish, the Hubble entropy bound needs to be modified by replacing \myHighlight{$H$}\coordHE{} with
the square root of \myHighlight{$H^2-1/l_{n+1}^2$}\coordHE{}. But, as argued in this paper, three entropy 
bounds: Bekenstein-Verlinde bound, Bekenstein-Hawking bound and Hubble bound 
in (\ref{2eq3}) still have the same forms as the case without the cosmological constant,
even when the cosmological constant is present. The cosmological D-bound introduced
in this work provides a new entropy bound of matter in the strongly self-gravitating 
universe (\myHighlight{$HR>1$}\coordHE{}) with a positive cosmological constant and makes all formulas work so 
nicely as the case without the cosmological constant.



\section*{Acknowledgment}

The work  of R.G.C. was supported in part by a grant from Chinese
Academy of Sciences and a grant from Ministry of Education, PRC.  
Y.S.M. acknowledges partial support from the KOSEF grant, Project 
Number: R02-2002-000-00028-0. R.G.C. is grateful to Relativity Research Center 
and School of Computer Aided Science, Inje University for warm hospitality 
during his visit.

\begin{thebibliography}{99}
\bibitem{Hooft}G.~'t Hooft,
arXiv:gr-qc/9310026;
%%CITATION = GR-QC 9310026;%%
L.~Susskind,
J.\ Math.\ Phys.\  {\bf 36}, 6377 (1995) [arXiv:hep-th/9409089].
%%CITATION = HEP-TH 9409089;%%

\bibitem{AdS}J.~M.~Maldacena,
Adv.\ Theor.\ Math.\ Phys.\  {\bf 2}, 231 (1998) [Int.\ J.\
Theor.\ Phys.\  {\bf 38}, 1113 (1999)] [arXiv:hep-th/9711200];
%%CITATION = HEP-TH 9711200;%%
S.~S.~Gubser, I.~R.~Klebanov and A.~M.~Polyakov,
Phys.\ Lett.\ B {\bf 428}, 105 (1998) [arXiv:hep-th/9802109];
%%CITATION = HEP-TH 9802109;%%
E.~Witten,
Adv.\ Theor.\ Math.\ Phys.\  {\bf 2}, 253 (1998)
[arXiv:hep-th/9802150].
%%CITATION = HEP-TH 9802150;%%

\bibitem{Verl}E.~Verlinde,
arXiv:hep-th/0008140.
%%CITATION = HEP-TH 0008140;%%

\bibitem{Cardy}J.~L.~Cardy,
Nucl.\ Phys.\ B {\bf 270}, 186 (1986).
%%CITATION = NUPHA,B270,186;%%

\bibitem{Klemm}D.~Klemm, A.~C.~Petkou and G.~Siopsis,
Nucl.\ Phys.\ B {\bf 601}, 380 (2001) [arXiv:hep-th/0101076].
%%CITATION = HEP-TH 0101076;%%

\bibitem{Cai1}R.~G.~Cai,
Phys.\ Rev.\ D {\bf 63}, 124018 (2001) [arXiv:hep-th/0102113].
%%CITATION = HEP-TH 0102113;%%

\bibitem{Birm}D.~Birmingham and S.~Mokhtari,
Phys.\ Lett.\ B {\bf 508}, 365 (2001) [arXiv:hep-th/0103108].
%%CITATION = HEP-TH 0103108;%%

\bibitem{Jing}J.~l.~Jing,
Phys.\ Rev.\ D {\bf 66}, 024002 (2002) [arXiv:hep-th/0201247].
%%CITATION = HEP-TH 0201247;%%

\bibitem{SV}I.~Savonije and E.~Verlinde,
Phys.\ Lett.\ B {\bf 507}, 305 (2001) [arXiv:hep-th/0102042].
%%CITATION = HEP-TH 0102042;%%

\bibitem{Strom}A.~Strominger,
JHEP {\bf 0110}, 034 (2001) [arXiv:hep-th/0106113].
%%CITATION = HEP-TH 0106113;%%

\bibitem{Super}A.~G.~Riess {\it et al.}  [Supernova Search Team Collaboration],
Astron.\ J.\  {\bf 116}, 1009 (1998) [arXiv:astro-ph/9805201];
%%CITATION = ASTRO-PH 9805201;%%
S.~Perlmutter {\it et al.}  [Supernova Cosmology Project
Collaboration],
Astrophys.\ J.\  {\bf 483}, 565 (1997) [arXiv:astro-ph/9608192];
%%CITATION = ASTRO-PH 9608192;%%
R.~R.~Caldwell, R.~Dave and P.~J.~Steinhardt,
Phys.\ Rev.\ Lett.\  {\bf 80}, 1582 (1998)
[arXiv:astro-ph/9708069];
%%CITATION = ASTRO-PH 9708069;%%
P.~M.~Garnavich {\it et al.},
Astrophys.\ J.\  {\bf 509}, 74 (1998) [arXiv:astro-ph/9806396].
%%CITATION = ASTRO-PH 9806396;%%}

\bibitem{Beke}J.~D.~Bekenstein,
Phys.\ Rev.\ D {\bf 23}, 287 (1981).
%%CITATION = PHRVA,D23,287;%%

\bibitem{CMO}R.~G.~Cai, Y.~S.~Myung and N.~Ohta,
Class.\ Quant.\ Grav.\  {\bf 18}, 5429 (2001)
[arXiv:hep-th/0105070].
%%CITATION = HEP-TH 0105070;%%

\bibitem{CM} R.~G.~Cai and Y.~S.~Myung,
arXiv:hep-th/0210300.
%%CITATION = HEP-TH 0210300;%%

\bibitem{TM}R.~M.~Wald,
Phys.\ Rev.\ D {\bf 48}, 3427 (1993)
[arXiv:gr-qc/9307038].
%%CITATION = GR-QC 9307038;%%
In higher derivative gravitational theories, the so-called area theorem 
no longer holds, for example, see
T.~Jacobson and R.~C.~Myers,
Phys.\ Rev.\ Lett.\  {\bf 70}, 3684 (1993) [arXiv:hep-th/9305016];
%%CITATION = HEP-TH 9305016;%%
R.~G.~Cai,
Phys.\ Rev.\ D {\bf 65}, 084014 (2002)
[arXiv:hep-th/0109133];
%%CITATION = HEP-TH 0109133;%%
and references therein. 

\bibitem{FS}W.~Fischler and L.~Susskind,
arXiv:hep-th/9806039.
%%CITATION = HEP-TH 9806039;%%
\bibitem{Hubb}
R.~Easther and D.~A.~Lowe,
Phys.\ Rev.\ Lett.\  {\bf 82}, 4967 (1999) [arXiv:hep-th/9902088];
%%CITATION = HEP-TH 9902088;%%
G.~Veneziano,
Phys.\ Lett.\ B {\bf 454}, 22 (1999) [arXiv:hep-th/9902126];
%%CITATION = HEP-TH 9902126;%%
G.~Veneziano,
arXiv:hep-th/9907012;
%%CITATION = HEP-TH 9907012;%%
R.~Brustein and G.~Veneziano,
Phys.\ Rev.\ Lett.\  {\bf 84}, 5695 (2000) [arXiv:hep-th/9912055];
%%CITATION = HEP-TH 9912055;%%
D.~Bak and S.~J.~Rey,
Class.\ Quant.\ Grav.\  {\bf 17}, L83 (2000)
[arXiv:hep-th/9902173];
%%CITATION = HEP-TH 9902173;%%
N.~Kaloper and A.~D.~Linde,
Phys.\ Rev.\ D {\bf 60}, 103509 (1999) [arXiv:hep-th/9904120].
%%CITATION = HEP-TH 9904120;%%

\bibitem{Bous}R.~Bousso,
JHEP {\bf 9907}, 004 (1999) [arXiv:hep-th/9905177];
%%CITATION = HEP-TH 9905177;%%
R.~Bousso,
JHEP {\bf 9906}, 028 (1999) [arXiv:hep-th/9906022].
%%CITATION = HEP-TH 9906022;%%

\bibitem{Wald}E.~E.~Flanagan, D.~Marolf and R.~M.~Wald,
Phys.\ Rev.\ D {\bf 62}, 084035 (2000) [arXiv:hep-th/9908070].
%%CITATION = HEP-TH 9908070;%%

\bibitem{GH}G.~W.~Gibbons and S.~W.~Hawking,
Phys.\ Rev.\ D {\bf 15}, 2738 (1977).
%%CITATION = PHRVA,D15,2738;%%

\bibitem{Bousso}R.~Bousso,
JHEP {\bf 0104}, 035 (2001) [arXiv:hep-th/0012052].
%%CITATION = HEP-TH 0012052;%%

\bibitem{Cai2}R.~G.~Cai,
Nucl.\ Phys.\ B {\bf 628}, 375 (2002) [arXiv:hep-th/0112253];
%%CITATION = HEP-TH 0112253;%%
R.~G.~Cai,
Phys.\ Lett.\ B {\bf 525}, 331 (2002) [arXiv:hep-th/0111093].
%%CITATION = HEP-TH 0111093;%%

\bibitem{AD}L.~F.~Abbott and S.~Deser,
Nucl.\ Phys.\ B {\bf 195}, 76 (1982).
%%CITATION = NUPHA,B195,76;%%

\bibitem{Wall}P.~Kraus,
JHEP {\bf 9912}, 011 (1999)
[arXiv:hep-th/9910149];
%%CITATION = HEP-TH 9910149;%%
D.~Ida,
JHEP {\bf 0009}, 014 (2000)
[arXiv:gr-qc/9912002].
%%CITATION = GR-QC 9912002;%%


\bibitem{Ogush}S.~Ogushi,
Mod.\ Phys.\ Lett.\ A {\bf 17}, 51 (2002) [arXiv:hep-th/0111008];
%%CITATION = HEP-TH 0111008;%%
S.~Nojiri and S.~D.~Odintsov,
JHEP {\bf 0112}, 033 (2001)
[arXiv:hep-th/0107134].
%%CITATION = HEP-TH 0107134;%%


\bibitem{deSitter}A.~J.~Medved,
arXiv:hep-th/0111182;
%%CITATION = HEP-TH 0111182;%%
A.~J.~Medved,
Class.\ Quant.\ Grav.\  {\bf 19}, 919 (2002)
[arXiv:hep-th/0111238];
%%CITATION = HEP-TH 0111238;%%
Y.~S.~Myung,
Phys.\ Lett.\ B {\bf 531}, 1 (2002) [arXiv:hep-th/0112140];
%%CITATION = HEP-TH 0112140;%%
S.~Nojiri, S.~D.~Odintsov and S.~Ogushi,
arXiv:hep-th/0205187.
%%CITATION = HEP-TH 0205187;%%

\bibitem{PS}A.~C.~Petkou and G.~Siopsis,
JHEP {\bf 0202}, 045 (2002) [arXiv:hep-th/0111085].
%%CITATION = HEP-TH 0111085;%%

\bibitem{Padi}A.~Padilla,
Phys.\ Lett.\ B {\bf 528}, 274 (2002) [arXiv:hep-th/0111247];
%%CITATION = HEP-TH 0111247;%%
J.~P.~Gregory and A.~Padilla,
Class.\ Quant.\ Grav.\  {\bf 19}, 4071 (2002)
[arXiv:hep-th/0204218];
%%CITATION = HEP-TH 0204218;%%
Y.~S.~Myung,
arXiv:hep-th/0208086.
%%CITATION = HEP-TH 0208086;%%


\bibitem{Wang}B.~Wang, E.~Abdalla and R.~K.~Su,
Phys.\ Lett.\ B {\bf 503}, 394 (2001)
[arXiv:hep-th/0101073];
%%CITATION = HEP-TH 0101073;%%
B.~Wang, E.~Abdalla and R.~K.~Su,
Mod.\ Phys.\ Lett.\ A {\bf 17}, 23 (2002)
[arXiv:hep-th/0106086].
%%CITATION = HEP-TH 0106086;%%

\bibitem{Nojiri}S.~Nojiri, O.~Obregon, S.~D.~Odintsov, H.~Quevedo and M.~P.~Ryan,
Mod.\ Phys.\ Lett.\ A {\bf 16}, 1181 (2001)
[arXiv:hep-th/0105052].
%%CITATION = HEP-TH 0105052;%%

\bibitem{Youm}D.~Youm,
arXiv:hep-th/0111276.
%%CITATION = HEP-TH 0111276;%%

\bibitem{Medved}A.~J.~Medved,
arXiv:hep-th/0112009.
%%CITATION = HEP-TH 0112009;%%











\end{thebibliography}

\end{document}

\bye
