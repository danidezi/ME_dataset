\documentclass[a4paper,10pt]{article}
\begin{document}
\begin{flushright}
{\sf SU-4252-747}\\
{\sf DIAS-02-12}
\end{flushright}
\begin{center}
{\bf Noncommutative Chiral Anomaly and The Dirac-Ginsparg-Wilson Operator}\\
\bigskip
Badis Ydri\\
\bigskip
{\it School of Theoretical Physics , Dublin Institute for Advanced Studies ,\\
10 Burlington Road , Dublin 4 , Ireland .}\\
\end{center}

\vskip.5cm
\begin{abstract}
It is  shown that local axial anomaly in $2-$dimensions emerges
naturally in the {\it gauge-invariant quantized } Ginsparg-Wilson
relation if one postulates an underlying noncommutative structure
of spacetime . Indeed if one first regularizes the $2-$d  plane
with a fuzzy sphere, i.e with a $(2l+1){\times}(2l+1)$ matrix
model , then one immediately finds that the gauge-invariant Dirac
operator $D_{GF}$ is different from the chiral-invariant Dirac
operator $D_{CF}$ 
%%%%%, i.e gauge and chiral symmetries do not
%commute at very short distances 
. The fact that gauge states are
different from chiral states stems from the free
noncommutative Dirac operator $D_F$ which is found to be
inconsistent with chiral symmetry at high frequencies . Furthermore
it splits such that
$\frac{1}{D_F}={\theta}{\Gamma}^R+\frac{1}{D_{\Lambda}}$ where
$\theta=\frac{1}{2l+1}$ is the noncommutativity parameter , ${\Gamma}^R$ is the noncommutative chirality and
$D_{\Lambda}$ is identified as the correct Dirac operator on this
matrix model . $D_{\Lambda}$ has essentially the same IR spectrum
as $D_{F}$ , both operators share the same  continuum limit yet
$D_{\Lambda}$ is consistent with chiral symmetry on the UV modes
. The second step of this fuzzy regularization  consists simply
of approximating the exact Dirac operator $D_F$ with
$D_{\Lambda}$ . This regularization is shown to be
gauge-invariant and yields to an anomaly which in terms of Dirac operators depends
only on the difference $\frac{1}{D_F}-\frac{1}{D_{\Lambda}}$. Although the
difference ${\theta}{\Gamma}^R$ drops from the spectrum
in the limit , its contribution to the variation of the measure
is shown to be not zero , it gives exactly the canonical
anomaly. 


\end{abstract}


\setcounter{page}{1}

\section{Introduction}

The two key concepts in the discussion of anomalies are always ,
gauge invariance and regularization. Indeed Fujikawa
\cite{fujikawa} in his original derivation of the anomaly showed
that quantum chiral symmetry becomes anomalous because the
fermion measure is not actually invariant under chiral
transformations , and the anomaly is obtained from regularizing
large eigenvalues of the underlying Dirac operator. The
regularization prescription which is thus to be adopted must
always be gauge invariant. A simple reason is a physical one
which has to do with the fact that the anomaly has direct
observable consequences ( for example the abelian $U(1)$ axial
anomaly provides precisely the rate of the neutral
electromagnetic pion decay ${\pi}{\longrightarrow}2{\gamma}$ ) .



This quantum anomalous behaviour of chiral symmetry does not
necessarily mean that this symmetry is broken , it only means that
one can not regulate in a gauge-invariant , chiral-invariant
manner . In particular , the famous Nielsen-Ninomiya theorem
\cite{nielsen-ninomiya} states that if one maintains chiral
symmetry at all stages, then one cannot avoid the doubling of
fermions in the usual lattice formulation and hence the anomaly
is zero [ see \cite{creutz} and references therein ]. The anomaly
here vanishes only because the effect of the different doublers
cancel among themselves and not because the chiral symmetry is
being maintained explicitly throughout.




Fuzzy physics [\cite{madore,GKP} , see also \cite{ydri} and
references therein ] , like lattice gauge theories, is aiming for
a nonperturbative regularization of chiral gauge theories.
Discretization in fuzzy physics is achieved by treating the
underlying  spacetimes as phase spaces then quantizing them in a
canonical fashion which means in particular that we are
effectively replacing the underlying spacetimes by non-commutative
matrix models or fuzzy manifolds \cite{ydri,cmlv} . As a
consequence , this regularization will preserve all symmetries
and topological features of the problem . Indeed a fuzzy space is
by construction a discrete lattice-like structure which serves to
regularize , it allows for an exact chiral invariance to be be
formulated  , but still the fermion-doubling problem is
completely avoided \cite{trg} . The same thing was also shown to
hold for fermions on fuzzy ${\bf C}{\bf P}^2$ despite the fact
that these latter do not carry spin quantum numbers but instead
carry spin$_c$ quantum numbers , in other words the left-handed
fermions and the right-handed fermions on ${\bf C}{\bf P}^2$
transform differently under the spin group $SU(2){\times}U(1)$ of
${\bf C}{\bf P}^2$ [ \cite{giorgio} and the last reference of
\cite{GKP} ] . All this seems to suggest that  fuzzy methods do
not suffer from any of the limitations put by the above
Nielsen-Ninomiya theorem on lattice schemes .




Global chiral anomalies on these models were  , along with other
topological non-trivial field configurations , formulated in
\cite{bal,grosse,balsach} , while local anomaly on fuzzy ${\bf
S}^2_F$ was  treated first in \cite{presnajder} then in \cite{giorgio1,nagao2} . The relevance of Ginsparg-Wilson relations in noncommutative matrix models was noted first in \cite{trg} and \cite{miguel} then in \cite{nagao1} . In this article
we will show that despite the fact that the concept of point is
lacking on fuzzy ${\bf S}^2$, we can go beyond global
considerations and define a "fuzzy" axial anomaly associated with
a "fuzzy" $U(1)$ global chiral symmetry. This "fuzzy" axial
anomaly is however shown to be identically zero for each finite
approximation and only by requiring physical states in the
continuum limit to be gauge- and chiral-symmetric in the sense we explain that we
reproduce the canonical theta term .

The plan of the paper is as follows . Section $2$ contains a
brief description of fuzzy ${\bf S}^2$ . Noncommutative $U(1)$
gauge theory and fermion action on ${\bf S}^2_F$ are introduced
in section $3$ . In particular we show in this section that the
gauge-invariant Dirac operator $D_{GF}$ is different from the
chiral-invariant Dirac operator $D_{CF}$ . Section $4$ contains
the main results of the paper , we show that $a)$ The measure is
completely symmetric under chiral transformations for all finite
matrix models , $b)$ The quantized gauge-invariant Ginsparg-Wilson
relation contains already the anomaly and $c)$ The free Dirac
Operator $D_F$ provides a gauge-invariant regularization
consistent with chiral symmetry which is shown to reproduce the
anomaly in the continuum limit. In section $5$ we derive the
divergence of the current using the star product and the Dixmier
trace and show that the two methods are equivalent.  We conclude
in section $6$ with a summary.








\section{The Noncommutative Fuzzy Sphere ${\bf S}^2_F$}

One practical way of thinking about the fuzzy sphere is as a
regulator . In other words , instead of replacing Euclidean
space-time ${\bf R}^2$ by a lattice , we implement the
regularization prescription given by the substitution ${\bf
R}^2{\longrightarrow}{\bf S}^2_F$ where ${\bf S}^2_F$ is the
fuzzy sphere [ see \cite{madore,GKP} and the extensive list of
publications in \cite{ydri}  ]. We now briefly review it .

The fuzzy sphere is a noncommutative space and as such it must be
described by a spectral triple $({\bf A},{\bf H},D_F)$ \cite{cmlv}
. Beside the Dirac operator $D_F$ which will be described shortly
, the most important other ingredient in the description of the
fuzzy sphere , in terms of Connes' noncommutative geometry
\cite{cmlv} , is the algebra ${\bf A}$ of $(2l+1){\times}(2l+1)$
matrices $Mat_{2l+1}$ . This algebra can be obtained as follows .

As embedded in ${\bf R}^3$ the commutative unit sphere is
described by three commuting coordinates ${n}_a=x_a$  , $a=1,2,3$
satisfying the constraint $\sum_{a}n_a^2=1$ . In terms of these
commutative coordinates any function on ${\bf S}^2$ can be
expanded as follows
\begin{equation}
{f}(\vec{n})=\sum_{i_1,...,i_k}f_{i_1,...,i_k}n_{i_1}...n_{i_k}{\equiv}\sum_{km}f_{km}Y_{km}(\theta,\phi),
\end{equation}
where the $Y_{km}$'s are the usual spherical harmonics and the
sum over $k$ goes from $0$ to $\infty$ . The claim is that the
geometry of ${\bf S}^2$ can be completely reconstructed in terms
of the algebra ${\cal A}$ of all functions $f$ on ${\bf S}^2$ .

To obtain the noncommutative fuzzy ${\bf S}^2$ we simply need to
replace consistently the algebra ${\cal A}$ by a suitable algebra
of operators the elements of which will fail to  commute in general .
In the case of ${\bf S}^2$ which is a co-adjoint orbit, i.e ${\bf
S}^2=SU(2)/U(1)$ this process is not arbitrary as one can simply
quantize the underlying symplectic structure to get the algebra
of operators ${\bf A}$ . One finds that any operator $\hat{f}$ of
 ${\bf A}$ has now the expansion
\begin{equation}
\hat{f}(\vec{n}^F)=\sum_{i_1,...,i_k}f_{i_1,...,i_k}n_{i_1}^F...n_{i_k}^F,\label{1}
\end{equation}
where $n_{i}^F$'s are defined by
\begin{equation}
n_i^F=\frac{L_i}{\sqrt{l(l+1)}}.\label{22}
\end{equation}
$L_i$'s satisfy $[L_i,L_j]=i{\epsilon}_{ijk}L_k$ and
$\sum_{i=1}^3L_i^2=l(l+1)$ respectively , in other words $L_i$'s
are the generators of the IRR $l$ of $SU(2)$ , and therefore all
the sums in (\ref{1}) terminate \cite{madore,GKP,ydri} . This
fact can be better seen by rewriting (\ref{1}) in terms of
Polarization tensors , namely
\begin{eqnarray}
\hat{f}=\sum_{k=0}^{2l}\sum_{m=-k}^{k}f_{km}T_{km}(l).\label{scalar}
\end{eqnarray}
[ For an extensive analysis of the properties of $T_{km}(l)$'s see
\cite{VKM} ] . Remark that $n_i^F$'s clearly satisfy the
requirements
\begin{equation}
(n_1^{F})^2+(n_2^{F})^2+(n_3^{F})^2=1,\label{33}
\end{equation}
and
\begin{equation}
[n_i^F,n_j^F]=\frac{i}{{\sqrt{l(l+1)}}}{\epsilon}_{ijk}n_k^F.\label{44}
\end{equation}
(\ref{1}),(\ref{22}),(\ref{33}) and (\ref{44}) define what we call
fuzzy ${\bf S}^2$ or ${\bf S}^2_F$. Loosely speaking , ${\bf
S}^2_F$ is the algebra ${\bf A}$ above which is nothing else but
the algebra $Mat_{2l+1}$ of $(2l+1){\times}(2l+1)$ matrices .
${\bf H}$ in the spectral triple $({\bf A},{\bf H},D_F)$ is simply
the Hilbert space on which the above IRR $l$ of $SU(2)$ is acting
.

Fuzzy Derivations are by definition the generators of the adjoint
action of $SU(2)$ , in other words the derivative of the fuzzy
function $\hat{f}$ in the space-time direction $a$ is
$adL_a\hat{f}$ defined by
\begin{eqnarray}
adL_a(\hat{f})=[L_a,\hat{f}].
\end{eqnarray}
This is motivated by the fact that continuum derivations ${\cal
L}_a=-i{\epsilon}_{abc}n_b{\partial}_c$ generate rotations on
${\bf S}^2$ , which also provide the adjoint IRR of $SU(2)$ , and
hence it is only natural to stress this property once we go to
the fuzzy .

As one can already notice all these definitions are in a very close analogy with the case of continuum ${\bf S}^2$
where the algebra of functions ${\cal A}$ plays the same role played here by ${\bf A}$. In fact the continuum limit
is defined by $l{\longrightarrow}{\infty}$ where the fuzzy coordinates $n_i^F$'s approach the ordinary coordinates
$n_i$'s , i.e

\begin{eqnarray}
&&{n}^F_{i}{\longrightarrow}n_i~~~~{\rm with}~ \vec{n}^2=1~~~{\rm and}~~ [n_i,n_j]=0,\nonumber\\
\end{eqnarray}
and where the algebra ${\bf A}$ tends to the algebra ${\cal A}$ in the sense that
\begin{eqnarray}
&&\hat{f}{\longrightarrow}f(\vec{n})=\sum_{i_1,...,i_k}f_{i_1,...,i_k}n_{i_1}...n_{i_k},\label{5}
\end{eqnarray}
and
\begin{eqnarray}
~~adL_i(\hat{f}){\longrightarrow}{\cal L}_i(f)(\vec{n}).
\end{eqnarray}
Now the sums in  (\ref{5}) are not cut-off and one can have in the expansion polynomials of arbitrary degrees in
$n_i$'s . Formally one writes ${\cal A}=Mat_{\infty}$ and think of the fuzzy sphere as having a finite number of
points equal to $2l+1$ which will diverge in the continuum limit $l{\longrightarrow}{\infty}$.

\section{Noncommutative Fuzzy Actions}
\subsection{Noncommutative Fuzzy YM Action}
A scalar field on the noncommutative space ${\bf S}^2_F$ is a
general matrix $\hat{f}$ of the algebra $Mat_{2l+1}$ which admits
the momentum-space expansion (\ref{scalar}) . Classical actions
for these scalar fields and their quantum theories were put
forward in \cite{ydri2} . A vector field $A_a^F$ can be similarly
defined by an expansion of the form
\begin{eqnarray}
{A}_a^F=\sum_{k=0}^{2l}\sum_{m=-k}^{k}A_a(km)T_{km}(l)~,~A_a^{F+}=A_a^F,
\label{vector}
\end{eqnarray}
i.e ~each component $A_a^F$ is a $(2l+1){\times}(2l+1)$ matrix .
The modes $A_a(km)$ are complex numbers satisfying
$A_a(km)^{*}=(-1)^mA_a(k-m)$
 where for each momentum $(km)$ the corresponding triple
$(A_1^F(km),A_2^F(km),A_3^F(km))$ obviously transforms as an
$SO(3)-$vector . Writing a gauge principle for this matrix vector
field is not difficult , indeed the action takes the usual form
\begin{eqnarray}
S_{YMF}&=&\frac{1}{4e^2}\frac{1}{2l+1}Tr_{l}F_{ab}^FF_{ab}^F,\label{action2}
\end{eqnarray}
where the trace $Tr_l$ is defined on the Hilbert space ${\bf H}$
 whereas the curvature $F_{ab}^F$ is given by
\begin{eqnarray}
F_{ab}^F&{\equiv}&[D_a^F,D_b^F]-i{\epsilon}_{abc}D_c^F~,~D_a^F=L_a+A_a^F\nonumber\\
&=&[L_a,A_b^F]-[L_b,A_a^F]+[A_a^F,A_b^F]-i{\epsilon}_{abc}A_c^F.
\end{eqnarray}
Gauge transformations are implemented by unitary transformations
acting on the $(2l+1)-$dimensional Hilbert space of the
irreducible representation $l$ of $SU(2)$ . These transformations
are $D_a^F{\longrightarrow}D_a^{F'}=U^LD_a^FU^{L+}$ ,
$A_a^F{\longrightarrow}A_a^{F'}=U^LA_a^FU^{L+}+U^L[L_a,U^{L+}]$
and $ F_{ab}^F{\longrightarrow}F_{ab}^{F'}=U^LF_{ab}^FU^{L+}$
where $U^L=e^{i{\Lambda}^L}$ and ${\Lambda}^L={\Lambda}^{L+}$ is
an element of the algebra ${\bf A}$ of $(2l+1){\times}(2l+1)$
matrices acting on the left . $U^L$'s define then fuzzy $U(1)$
gauge theory .

A final remark is to note that the gauge field $\vec{A}^F$ has
three components and hence an extra condition is needed in order
to project this gauge field onto two dimensions . One adopts here
the prescription of \cite{nair} , i.e we impose on the gauge
filed $\vec{A}^F$ the gauge-invariant condition
\begin{eqnarray}
D_a^FD_a^F=l(l+1).\label{local}
\end{eqnarray}
The full quantization of this model and its continuum planar
limit will be reported elsewhere \cite{badis} .

\subsection{Gauge-Invariant Dirac Operator and Chiral Fermion in
NCG}

The Dirac operator is a key ingredient in Connes noncommutative
geometry ( NCG ) especially in connection with metric spaces and
spin structures \cite{cmlv} . In this section we will show
explicitly that the properties of the Dirac operator at very
short distances are indeed at the origin of the axial anomaly .
More precisely we will show that the gauge-invariant Dirac
operator $D_{GF}$ on the matrix model (\ref{44}) is incompatible
with chiral fermion in the sense that chiral symmetry in the
presence of gauge fields is governed by a completely different
Dirac operator $D_{CF}$ . The consequence of this fact will on
the other hand be analyzed in great detail in the next section .



First we find the gauge-invariant Dirac operator $D_{GF}$ on fuzzy
${\bf S}^2$ , i.e a finite dimensional gauge-invariant matrix
action for fermion to be added to (\ref{action2}). In close
analogy with the free fermion action on continuum ${\bf S}^2$
given by \cite{presnajder,denjoe}
\begin{eqnarray}
S=\int_{{\bf S}^2} \frac{d{\Omega}}{4{\pi}} \bar{\chi}{\cal
D}{\chi}~,~{\cal D}={\sigma}_a{\cal L}_a+1~,~{\cal
L}_a=-i{\epsilon}_{abc}x_b{\partial}_c,\label{classiii}
\end{eqnarray}
the free fermion action on fuzzy ${\bf S}^2$ is defined by
\begin{eqnarray}
S_F&=&\frac{1}{2l+1}Tr_{l}\bar{{\psi}}_F{D}_F{\psi}_F~,~ {D}_{
F}={\sigma}_a[L_a,...]+1.\label{fuzziii}
\end{eqnarray}
${\cal D}$ and $D_F$ are precisely the Dirac operators on
continuum ${\bf S}^2$ and fuzzy ${\bf
 S}^2$ respectively
\cite{trg,ydri,bal,grosse} . The continuum spinor ${\chi}$ is an
element of ${\cal A}{\otimes}{\bf C}^2$ where ${\cal A}$ is the
algebra of function on continuum ${\bf S}^2$ , while the fuzzy
spinor ${\psi}_F$ is an element of ${\bf A}{\otimes}{\bf
C}^2{\equiv}Mat_{2l+1}{\otimes}{\bf C}^2$ . Both spinors are of
dimension $({\rm mass})^{\frac{1}{2}}$, $\bar{\chi}={\chi}^{+}$ ,
$\bar{\psi}_F={\psi}_F^{+}$ and ${\sigma}_i$'s are Pauli matrices
.


It is a trivial fact that the above continuum Dirac operator
${\cal D}$ admits a chirality structure , i.e we can define a
chirality operator ${\gamma}$ such that $
{\gamma}^2=1~,~\{{\gamma},{\cal D}\}=0~$ , i.e
${\gamma}={\sigma}_a{n}_a$ . The
Grosse-Klim\v{c}\'{i}k-Pre\v{s}najder Dirac operator $D_F$ on
fuzzy ${\bf S}^2$ admits also a chirality operator which can be
seen as follows , first we rewrite $D_F$ in the form
\cite{trg,ydri, bal,grosse}
\begin{eqnarray}
\frac{1}{2l+1}D_{F}&=&\frac{1}{2}({\Gamma}^R+{\Gamma}^L),\label{key}
\end{eqnarray}
where $L_a^L{\equiv}L_a$'s and $-L_a^R$'s are the generators of
the IRR $l$ of $SU(2)$ which act on the right of the algebra
${\bf A}$ . ${\Gamma}^R$ and ${\Gamma}^L$ , on the other hand ,
are the operators
\begin{eqnarray}
{\Gamma}^L&=&\frac{1}{l+\frac{1}{2}}[\vec{\sigma}.\vec{L}^L+\frac{1}{2}]~,~
{\Gamma}^R=\frac{1}{l+\frac{1}{2}}[-\vec{\sigma}.\vec{L}^R+\frac{1}{2}].\label{chiralities}
\end{eqnarray}
${\Gamma}^R$ is the chirality operator as was shown originally in
\cite{watamuras} , this choice is also motivated by the fact that
${\Gamma}^R{\longrightarrow}-{\gamma}~,~{\rm
when}~l{\longrightarrow}{\infty}$ ( see latter for the minus sign
),
$({\Gamma}^R)^2={\Gamma}^{R}~,~({\Gamma}^{R})^{+}={\Gamma}^{R}~,~{\rm
and}~[{\Gamma}^R,\hat{f}]=0$ for any $\hat{f}{\in}{\bf A}$ .
However this ${\Gamma}^R$ does not exactly anticommute with the
Dirac operator since
\begin{eqnarray}
{\Gamma}^R{D}_{F}+{D}_{F}{\Gamma}^R&=&\frac{1}{l+\frac{1}{2}}{D}^2_{F}.\label{GWrelation}
\end{eqnarray}
Despite this problem , one can show that $({D}_{F},{\Gamma}^R)$
defines a chiral structure on fuzzy ${\bf S}^2$ which satisfies
$a)$ the Ginsparg-Wilson relation , $b)$ is without fermion
doubling and $c)$ has the correct continuum limit
\cite{trg,ydri}. The absence of fermion doubling for example can
be easily seen by comparing the spectrum of ${D}_{F}$ given by $
{D}_{F}(j)=\{{\pm}(j+\frac{1}{2}) ,
j=\frac{1}{2},\frac{3}{2},...,2l-\frac{1}{2}\}\cup
\{j+\frac{1}{2}~~,~~~j=2l+\frac{1}{2}\}$ with the spectrum of
${\cal D}$ given by ${\cal D}(j)=\{{\pm}(j+\frac{1}{2}) ,
j=\frac{1}{2},\frac{3}{2},...,\infty \}$ \cite{grosse} . As one
can immediately see there is no fermion doubling and the spectrum
of ${D}_F$ is simply cut-off at the top eigenvalue
$j=2l+\frac{1}{2}$ if compared with the continuum spectrum
\cite{trg,ydri}  .



However the situation when one includes gauge field is quite
different. Gauging the continuum Dirac operator ${\cal D}$ means
invoking the minimal replacement ${\cal
L}_a{\longrightarrow}{\cal L}_a+{A}_a$ , and therefore we end up
with the Dirac operator $ {\cal D}_{G}={\cal
D}+{{\sigma}}_a{A}_a$. Correspondingly the gauge-invariant
fermion action in the continuum is given by
\begin{eqnarray}
S_{G}&=&\int
\frac{d{\Omega}}{4{\pi}}\bigg[\bar{\chi}{\sigma}_a{\cal
L}_a({\chi})+\bar{\chi}{\chi}+\bar{\chi}{\sigma}_aA_a{\chi}\bigg].\label{contaction1}
\end{eqnarray}
As usual , gauge transformations act on  ${\chi}$ , $\bar{\chi}$
as follows ${\chi}{\longrightarrow}{\chi}^{'}=U{\chi}$ ,
$\bar{\chi}{\longrightarrow}\bar{\chi}^{'}=\bar{\chi}U^{+}$.
Similarly , the fuzzy gauged Dirac operator is defined by
\begin{eqnarray}
{D}_{GF}={D}_{F}+{\sigma}_aA_a^F,\label{GFsummary}
\end{eqnarray}
while the fuzzy analogue of the action (\ref{contaction1}) is
given by
\begin{eqnarray}
S_{GF}&=&\frac{1}{2l+1}Tr_{l}\bigg[\bar{{\psi}}_F{\sigma}_{a}[L_a,{\psi}_F]+\bar{\psi}_F{\psi}_F+\bar{{\psi}}_F{\sigma}_a{A}_a^F{{\psi}}_F\bigg],\label{gaugeFaction}
\end{eqnarray}
This gauge-invariant fermion fuzzy action is invariant under
gauge transformations provided the spinor ${\psi}_F$ is
simultaneously transformed as
${\psi}_F{\longrightarrow}{\psi}^{'}_{F}=U^L{\psi}_{F}$,
$\bar{\psi}_F{\longrightarrow}
\bar{\psi}^{'}_{F}=\bar{\psi}_{F}U^{L+}$ .


The quantum theory of the noncommutative Schwinger model given by
the action (\ref{action2})+(\ref{gaugeFaction}) will be
considered in great detail elsewhere \cite{badis} , in here we
will only touch on the corresponding quantum chiral structure and
its anomalous behaviour . The claim is that (\ref{gaugeFaction})
is not chiral-invariant , nevertheless chiral symmetry can be
exactly defined for all finite values of $l$ and yet an anomalous
behaviour remains in the continuum limit . It is obtainable as we
will show from a combination of fuzzy path integrals and
noncommutative Ginsparg-Wilson relations and the requirement of
gauge invariance .



\subsection{The Chiral-invariant Dirac Operator $D_C$ and its Complex Action}
On continuum ${\bf S}^2$ exact chiral invariance of the classical
action is expressed by the anticommutation relation
${\gamma}{\cal D}+{\cal D}{\gamma}=0$ which is in fact the limit
of the Ginsparg-Wilson relation (\ref{GWrelation}). We now show
that even with (\ref{GWrelation}) exact chiral invariance can be
constructed consistently on fuzzy ${\bf S}^2$. The fermion action
(\ref{contaction1}) was already shown to be gauge invariant , but
under the canonical continuum chiral transformations
${\chi}{\longrightarrow}{\chi}^{'}={\chi}+{\lambda}{\gamma}{\chi}$
, $
\bar{\chi}{\longrightarrow}\bar{\chi}^{'}=\bar{\chi}+{\lambda}\bar{\chi}{\gamma}$
, one can show that it is invariant only if the gauge field is
constrained to satisfy $x_aA_a=0$ which is a consequence of the
identity ${\cal D}_G{\gamma}+{\gamma}{\cal D}_G=2\vec{A}.\vec{n}
=2{\phi}$. From the continuum limit of (\ref{local}) it is
obvious that this constraint is satisfied and hence chiral
symmetry is maintained. However one wants also to formulate
chiral symmetry without the need to use any constraint on the
gauge field , indeed the action

\begin{eqnarray}
\int \frac{d{\Omega}}{4{\pi}}\bigg[\bar{{\chi}}{\cal D}
{\chi}+\bar{{\chi}}\hat{\sigma}_a{A}_a{{\chi}}\bigg].\label{2}
\end{eqnarray}
is strictly chiral invariant for arbitrary gauge configurations .
$\hat{\sigma}_a$ is the Clifford algebra projected onto the sphere
, i.e $\hat{\sigma}_a={\cal P}_{ab}{\sigma}_{b}$ , ${\cal
P}_{ab}={\delta}_{ab}-n_an_b$ . The action (\ref{2}) is however
still gauge invariant because of the identity $n_a{\cal L}_a=0$ .
Action (\ref{2}) can be rewritten as follows
\begin{eqnarray}
S_{C}&=&\int \frac{d{\Omega}}{4{\pi}}\bigg[\bar{{\chi}}{\cal D}
{\chi}+{\epsilon}_{abc}\bar{{\chi}}Z_bn_c{A}_a{{\chi}}\bigg]~,~Z_a=\frac{i}{2}[{\gamma},{\sigma}_a]\label{3}
\end{eqnarray}
from which we can define the new chiral-invariant Dirac operator
${\cal D}_C={\cal D}+{\epsilon}_{abc}Z_bn_cA_a$. The difference
between the continuum gauge-invariant Dirac operator ${\cal D}_G$
and the continuum chiral-invariant Dirac operator ${\cal D}_C$ is
proportional to the normal component ${\phi}$ of the gauge field
, i.e ${\cal D}_{C}={\cal D}_G-{\gamma}{\phi}$ , and hence they
are essentially identical ( by virtue of the constraint $n_aA_a=0$
) and the spectrum of the theory seems to be both gauge invariant
as well as chiral invariant . The Fuzzy analogue of (\ref{3}) is
the action
\begin{eqnarray}
S_{CF} =\frac{1}{2l+1}Tr_{l}
\bigg[\bar{{\psi}_F}D_F{\psi}_F+{\epsilon}_{abc}\bar{{\psi}_F}Z_b^Fn^F_c{A}_{a}^F{{\psi}_F}
\bigg]~,~Z_a^F=\frac{i}{2}[{\Gamma}^L{\sigma}_a+{\sigma}_a{\Gamma}^R],\label{3F}
\end{eqnarray}
which is invariant under the fuzzy chiral transformations
\begin{eqnarray}
&&{\psi}_F{\longrightarrow}{\psi}_F^{'}={\psi}_F+{\Gamma}^R{\psi}_F{\lambda}^L+O({\lambda}^{L})\nonumber\\
&&\bar{\psi}_F{\longrightarrow}\bar{\psi}_F^{'}=\bar{\psi}_F-{\lambda}^L\bar{\psi}_F{\Gamma}^L+O({\lambda}^L).\label{Fchiral}
\end{eqnarray}
Remark that the chiral parameter ${\lambda}^L$ in (\ref{Fchiral})
is a general $(2l+1){\times}(2l+1)$ matrix which is small in the
sense of coherent states \cite{lee,ref21} . Furthermore in order
for this parameter to correspond to a global transformation it
must only be a function of the Casimir $(\vec{L}^{L})^2$ . The
structure of the above chiral transformations (\ref{Fchiral}) is
of course uniquely dictated by the Ginsparg-Wilson relation
(\ref{GWrelation}) which can also be put in the form
${\Gamma}^LD_F-D_F{\Gamma}^R=0$ \cite{leonardo}. Remark also that
the spinor $\bar{\psi}$ can not now be identified with
${\psi}^{+}$ although the theory is still classical and despite
the fact that in gauge theory $\bar{\psi}$ is
${\equiv}{\psi}^{+}$ . This result can be interpreted as the
statement that on the finite dimensional matrix model (\ref{44})
chiral and gauge symmetries are not commuting symmetries  , i.e
chiral states are different from gauge states at very short
distances ( corresponding to the UV part of the spectrum ) .
Indeed we can define from (\ref{3F}) a new Dirac operator
\begin{eqnarray}
D_{CF}=D_F+{\epsilon}_{abc}Z_b^Fxn^F_c{A}_{a}^F~\label{3.38}
\end{eqnarray}
which is very different from $D_{GF}$ and hence the two operators
do not commute. Now , the operator $D_{CF}$ tends in the large
$l$ limit to ${\cal D}_C$ so that both operators $D_{CF}$ and
$D_{GF}$ have the same continuum limit [ Recall that in this
limit we have ${\Gamma}^R{\longrightarrow}-{\gamma}$ ,
${\Gamma}^L{\longrightarrow}{\gamma}$ which are consequences of
the requirements
${\Gamma}^LD_F-D_F{\Gamma}^R=0{\longrightarrow}{\gamma}{\cal
D}+{\cal D}{\gamma}=0$ , $Z_{k}^{F}{\longrightarrow}Z_{k}$  and
hence $D_{CF}{\longrightarrow}{\cal D}_C$ ] .

Finally let us say that the action (\ref{gaugeFaction}) is not
chiral invariant while the action (\ref{3F}) is complex and not
gauge invariant ( in fact the above $D_{CF}$ is not self-adjoint
but as we have just said it is the correct fuzzy analogue of
${\cal D}_C$ from the point of view of chiral invariance and if
we insist that chiral transformations take the canonical form
(\ref{Fchiral}) ) . The kinetic term in both actions
(\ref{gaugeFaction}) and (\ref{3F}) is however the same and hence
we do not loose the nice feature of having in the noncommutative
case a free spectrum which is identical to the continuum spectrum
with a natural cut-off .
\section{Chiral Anomaly}
\subsection{Quantum measure and The Path Integral} The quantum
theory of interest is defined through the following path integral
\begin{eqnarray}
\int{\cal D}A_i^F\int {\cal D}{{\psi}_{F}}{\cal
D}\bar{{\psi}}_{F}e^{-S_{GF}}.\label{par}
\end{eqnarray}
Despite the fact that $S_{GF}$ is not chiral-invariant , it is
the correct starting point as one wants to maintain
gauge-invariance throughout . The anomaly anyway arises from the
non-invariance of the measure under chiral transformations
\cite{fujikawa} and hence we will focus on  this measure and show
explicitly that for all finite approximations of the
noncommutative Schwinger model this measure is in fact invariant
unless the properties of the Dirac operator $D_{GF}$ are also
taken into account in the evaluation of the trace .

In a matrix model such as (\ref{gaugeFaction}) manipulations on
the quantum measure have a precise meaning . Indeed and by
following \cite{fujikawa} we first expand the fuzzy spinors
${\psi}_{F}$ , $\bar{\psi}_{F}$ in terms of the eigentensors
${\phi}(\mu,A)$ of the Dirac operator ${D}_{GF}$, write $
{\psi}_{F}=\sum_{\mu}{\theta}_{\mu}{\phi}(\mu,A)$ , $
\bar{\psi}_{F}=\sum_{\mu}\bar{\theta}_{\mu}{\phi}^{+}(\mu,A)$
where ${\theta}_{\mu}$'s , $\bar{\theta}_{\mu}$'s are independent
sets of Grassmanian variables, and ${\phi}(\mu,A)$'s are defined
by ${D}_{GF}{\phi}(\mu,A)={\lambda}_{\mu}(A){\phi}(\mu,A)$ ,
 and
normalized such that $
\frac{1}{2l+1}Tr_{l}{\phi}^{+}(\mu,A){\phi}(\nu,A)={\delta}_{\mu
\nu }$ . For weak fuzzy gauge fields $\mu$ stands for $j$ , $k$
and $m$ which are the eigenvalues of
$\vec{J}^2=(\vec{K}+\frac{\vec{\sigma}}{2})^2$ ,
$\vec{K}^2=(\vec{L}^L-\vec{L}^R)^2$ and $J_3$ respectively . As
we will show, and similarly to \cite{fujikawa} , what really
matters in the calculation is the asymptotic behaviour when
$A^F_i{\longrightarrow}0$ of ${\phi}(\mu,A)$'s and
${\lambda}_{\mu}(A)$'s given by
${\lambda}_{\mu}(A){\longrightarrow}j(j+1)-k(k+1)+\frac{1}{4}$ and
$
{\phi}(\mu,A){\longrightarrow}\sqrt{2l+1}\sum_{k_3,\sigma}C^{jm}_{kk_3\frac{1}{2}\sigma}T_{kk_3}(l)
{\chi}_{\frac{1}{2}\sigma}$ \cite{ydri,grosse,VKM} . The quantum
measure is therefore well defined  and it is given by
\begin{eqnarray}
&&{\cal D}{\psi}_{F}{\cal
D}\bar{\psi}_{F}=\prod_{\mu}d{\theta}_{\mu}d{\bar{\theta}_{\mu}}{\longrightarrow}\prod_{k=0}^{2l}\prod_{j=k-\frac{1}{2}}^{k+\frac{1}{2}}\prod_{m=-j}^{j}d{\theta}_{kjm}d{\bar{\theta}_{kjm}}
\end{eqnarray}
A canonical calculation shows that the above quantum measure
changes under the fuzzy chiral transformations (\ref{Fchiral}) as
follows
\begin{eqnarray}
\int {\cal D}{A}_i^F\int {\cal D}{{\psi}_{F}}^{'}{\cal
D}\bar{{\psi}}_{F}^{'}e^{-S^{'}_{GF}}=\int {\cal D}{ A}_i^F\int
{\cal D}{{\psi}_{F}}{\cal
D}\bar{{\psi}_{F}}e^{{S}_{{\theta}F}}e^{-S_{GF}-{\Delta}S_{GF}},\label{partition0}
\end{eqnarray}
with
\begin{eqnarray}
{\Delta}S_{GF}=-\frac{1}{2l+1}Tr_{l}{\lambda}^L[L_a,\bar{{\psi}_F}{\sigma}_a{\Gamma}^R{{\psi}_F}]+\frac{1}{2l+1}Tr_{l}{\lambda}^L\bar{\psi}\big[{\sigma}_a{\Gamma}^R-{\Gamma}^L{\sigma}_a\big]A_a^F{\psi}_F,\label{delta}
\end{eqnarray}
where we can see explicitly in the second term in ${\Delta}S_{GF}$
why (\ref{gaugeFaction}) is not chiral-invariant . This extra piece is zero for zero gauge fields and is generically
of the order of $1/l$. It is due to  edge effects introduced by
fuzzification , more precisely it is due to the fact that in the
discrete the chirality operator ${\Gamma}^R$ does not exactly
anti-commute with the tangent Clifford algebra
${\sigma}_{a}P_{ab}$ in the sense that we have $
[{{\sigma}}_a{\Gamma}^R-{\Gamma}^L{{\sigma}}_a]P_{ab}{\neq}0$ .
In the continuum this is an identity , since the normal component of $\vec{A}^F$ is identically zero by virtue of (\ref{local}) , and therefore
${\Delta}{\Gamma}^L$ is not needed , i.e $[{{\sigma}}_a{\gamma}+{\gamma}{{\sigma}}_a]{\cal P}_{ab}=0$ . $P_{ab}$ and ${\cal P}_{ab}$ are simply the projectors ( in the fuzzy and in the continuum respectively ) on the sphere \cite{giorgio1,unp}.



The theta term is on the other hand given by
\begin{eqnarray}
S_{{\theta}F}&=&-\frac{1}{2l+1}\sum_{\mu}Tr_{l}{\lambda}^L{\phi}^{+}(\mu,A)({\Gamma}^R-{\Gamma}^L){\phi}(\mu,A).\label{anomaly-1}
\end{eqnarray}
The problem with (\ref{delta}) and (\ref{anomaly-1}) is in fact not edge effects but rather gauge-invariance ( for example ${\phi}(\mu , A)$ transforms as $U^L{\phi}(\mu,A)$ for the Dirac equation to be gauge-invariant). In here we adopt a different , more economical , route in
defining chiral invariance which is consistent with gauge symmetry . We start by simply changing the
chiral transformations (\ref{Fchiral}) to
\begin{eqnarray}
&&{\psi}_F{\longrightarrow}{\psi}^{'}_F={\psi}_F+{\Gamma}^{R}{\psi}_F{\lambda}^L\nonumber\\
&&\bar{\psi}_F{\longrightarrow}\bar{\psi}^{'}_F=\bar{\psi}_F-{\lambda}^L\bar{\psi}_F\hat{\Gamma}^L,\label{fuzzytrans2}
\end{eqnarray}
where
\begin{eqnarray}
\hat{\Gamma}^L=\frac{1}{l+\frac{1}{2}}({\sigma}_a(L_a+A_a^F)+\frac{1}{2})={\Gamma}^L+\frac{1}{l+\frac{1}{2}}\vec{\sigma}.\vec{A}^F.
\end{eqnarray}
(\ref{fuzzytrans2}) reduces in the limit to the usual transformations yet it gurantees gauge invariance in the noncommutative fuzzy setting since $\hat{\Gamma}^L$ transforms as $U^L\hat{\Gamma}^LU^{L+}$ under gauge transformations. Indeed the  change of the action under these transformations is now given
by
\begin{eqnarray}
{\Delta}S_{GF}&=&-\frac{1}{2l+1}Tr_{l}{\lambda}^L[L_a,\bar{{\psi}_F}{\sigma}_a{\Gamma}^R{{\psi}_F}]+\frac{1}{2l+1}Tr_l{\lambda}^L\bar{\psi}_F\bigg[({\sigma}_a{\Gamma}^R-{\Gamma}^L{\sigma}_a)A_a^F-\frac{1}{l+\frac{1}{2}}\vec{\sigma}.\vec{A}^FD_{GF}\bigg]{\psi}_F\nonumber\\
&=&-\frac{1}{2l+1}Tr_{l}{\lambda}^L[L_a,\bar{{\psi}_F}{\sigma}_a{\Gamma}^R{{\psi}_F}]-\frac{i}{(2l+1)^2}{\epsilon}_{abc}Tr_l{\lambda}^L\bar{\psi}_F{\sigma}_cF_{ab}^F{\psi}_F,
\label{divergence}
\end{eqnarray}
where in particular we have used equation (\ref{local}) . We still have an edge effect, i.e the extra piece above vanishes in the continuum limit but now it is exactly gauge-invariant . The theta term is now also gauge-invariant and is given by
\begin{eqnarray}
S_{{\theta}F}&=&-\frac{1}{2l+1}\sum_{\mu}Tr_{l}{\lambda}^L{\phi}^{+}(\mu,A)({\Gamma}^R-\hat{\Gamma}^L){\phi}(\mu,A).\label{anomaly}
\end{eqnarray}
Next because the Dirac operator $D_{GF}$ is self-adjoint on
$Mat_{2l+1}{\otimes}{\bf C}^2$ , the states ${\phi}(\mu,A)$'s form
a complete set and hence one must have the identity
\begin{eqnarray}
\frac{1}{2l+1}\sum_{\mu}{\phi}^{AB\alpha}(\mu,A){\phi}^{+CD\beta}(\mu,A)={\delta}^{\alpha
\beta}{\delta}^{AD}{\delta}^{BC}.\label{complete}
\end{eqnarray}
Due to the finiteness of the matrix model , it is an identity
easy to check   that the theta term ${S}_{\theta F}$ is zero ,
indeed we have
\begin{eqnarray}
S_{\theta
F}=-\bigg(Tr_l{\lambda}^L\bigg)\bigg(Tr_ltr_2({\Gamma}^R-\hat{\Gamma}^L)\bigg){\equiv}0.
\end{eqnarray}
$tr_2$ is the $2-$dimensional spin trace , i.e $tr_2{\bf 1}=2$ ,
$tr_2{\sigma}_a=0$ , etc . In fact it is this trace which actually
vanishes , i.e $tr_2({\Gamma}^R-\hat{\Gamma}^L)=0$ , and hence the
correct axial anomaly requires more than the naive evaluation of
the trace .




\subsection{Noncommutative Ginsparg-Wilson Relation }






We will now undertake the task of carefully analyzing the
spectrum of the theory and consequently derive the continuum
limit of the axial anomaly . We first start with the free theory
and rewrite the Ginsparg-Wison relation (\ref{GWrelation}) as $
({\Gamma}^R-{\Gamma}^L)D_F+D_F({\Gamma}^R-{\Gamma}^L)=0$ which
means that in the absence of gauge field we must have $
tr[{\Gamma}^R-{\Gamma}^L]=0$ where the trace is taken in the space
of spinors . Chiral invariance of $D_{CF}$ can be expressed by the
Ginsparg-Wilson relation ${\Gamma}^LD_{CF}-D_{CF}{\Gamma}^R=0$
and hence $tr[{\Gamma}^R-{\Gamma}^L]$ is also $=0$ if the effect
of the gauge field is taken into account through $D_{CF}$ .
However if we include the gauge field through the gauge-invariant
Dirac operator $D_{GF}$ we compute instead
\begin{eqnarray}
\{{\Gamma}^R-{\Gamma}^L,D_{GF}\}&=&\frac{2}{2l+1}\bigg[N_aA_a^F-\frac{i}{2}{\epsilon}_{abc}{\sigma}_cF_{ab}+{\sigma}_a{A}^F_a
+[L_a,A_a^F]\bigg], \label{ginsparg-wilson}\nonumber\\
\end{eqnarray}
where $N_a=-2(L_a^R+L_a)$ and where $F_{ij}$ is the {\it
would-be} continuum curvature $F_{ab}^F=F_{ab}+[A_a^F,A_b^F]$ .
The naive limit of this equation is $\{{\gamma},{\cal
D}_G\}=2{\phi}=0$. In other words , in the continuum interacting
theory one might be tempted to conclude that $Tr{\gamma}=0$ which
we know is wrong in the presence of gauge fields. Noncommutative
geometry , as it is already obvious from equation
(\ref{ginsparg-wilson}), already gives us the structure of  the
chiral anomaly , indeed (\ref{ginsparg-wilson}) can furthermore be
rewritten in the form
\begin{eqnarray}
\big[{\Gamma}^R-{\Gamma}^L-\frac{2}{2l+1}D_{GF}\big]^2-4=\frac{2i}{(2l+1)^2}{\epsilon}_{abc}{\sigma}_c(F_{ab}+F_{ab}^F)-\frac{4}{(2l+1)^2}({A}_a^{F})^2,\nonumber
\end{eqnarray}
or equivalently in the gauge-covariant form
\begin{eqnarray}
\{{\Gamma}^R-\hat{\Gamma}^L,D_{GF}\}=-\frac{2i}{2l+1}{\epsilon}_{abc}{\sigma}_cF_{ab}^F,
\end{eqnarray}
which we find remarkable . We have used above the identity $\{D_{GF},{\sigma}_aA_a^F\}=\frac{1}{2}\{N_a,A_a^F\}+i{\epsilon}_{abc}{\sigma}_cF_{ab}^F+{\sigma}_aA_a^F-\frac{i}{2}{\epsilon}_{abc}{\sigma}_cF_{ab}$ . We have also used extensively the identities
$({\Gamma}^R-{\Gamma}^L)^2-4=-\frac{4D^2_F}{(2l+1)^2}$ ,
$D_{GF}^2=D_{GF}+({
L}_a-L_a^R+{A}_a^F)^2+\frac{i}{2}{\epsilon}_{abc}{\sigma}_cF_{ab}^F$
as well as the constraint (\ref{local}) . Now if we trace both
sides of the above equation we obtain the exact answer
\begin{eqnarray}
&&-tr({\Gamma}^R-{\Gamma}^L)=
\frac{i}{2l+1}{\epsilon}_{abc}tr\big(\frac{1}{D_{GF}}{\sigma}_cF_{ab}^F\big)+\frac{1}{2l+1}tr\big(\frac{1}{D_{GF}}J\big)\nonumber
\end{eqnarray}
where
\begin{eqnarray}
J&=&-A_a^FN_a-i{\epsilon}_{abc}{\sigma}_c[A_a^F,A_b^F]-4i{\epsilon}_{abc}{\sigma}_cL_a^RL_b-2{\sigma}_aA_a^F-i{\epsilon}_{abc}{\sigma}_cN_bA_a^F\nonumber\\
&=&-2D_{GF}{\sigma}_a{A}_a^F+4i\sqrt{l(l+1)}D_{Gw}~,
\end{eqnarray}
where $D_{Gw}$ is now obviously gauge-covariant , in fact it is exactly the gauged Watamuras Dirac operator \cite{watamuras} , i.e
\begin{eqnarray}
D_{Gw}={\epsilon}_{abc}{\sigma}_a\frac{D_b^F}{\sqrt{l(l+1)}}L_c^R.
\end{eqnarray}
This operator has the continuum limit ${\cal D}_{Gw}=i{\gamma}{\cal D}_G-i{\phi}=i{\gamma}{\cal D}_G$ where the normal component $\phi$ vanishes by virtue of the constraint (\ref{local}) . Recall that ${\cal D}_G={\cal D}+{\sigma}_aA_a$ is the continuum limit of $D_{GF}$ ( more on this below ). Hence we obtain
\begin{eqnarray}
&&-tr({\Gamma}^R-\hat{\Gamma}^L)=
\frac{i}{2l+1}{\epsilon}_{abc}tr\big(\frac{1}{D_{GF}}{\sigma}_cF_{ab}^F\big)+\frac{4i\sqrt{l(l+1)}}{2l+1}tr\big(\frac{1}{D_{GF}}D_{Gw}\big).\nonumber\\
\end{eqnarray}
The symbol $tr$ denotes the trace in the space of fuzzy spinors ,
i.e $ tr(X)=\sum_{\mu}{\phi}^{+}(\mu,A)X{\phi}(\mu,A)$. In
particular the cyclic property of the trace does not hold because
of the non-commutativity of the different ingredients , in other
words the result of this trace is still an element in the algebra
${\bf A}=Mat_{2l+1}$ . The full gauge-invariant chiral anomaly (\ref{anomaly})
takes now the form
\begin{eqnarray}
S_{\theta F}
=\frac{i}{(2l+1)^2}{\epsilon}_{abc}Tr_{l}{\lambda}^Ltr\big(\frac{1}{D_{GF}}{\sigma}_cF_{ab}^F\big)+\frac{4i\sqrt{l(l+1)}}{(2l+1)^2}Tr_l{\lambda}^Ltr\big(\frac{1}{D_{GF}}D_{Gw}\big),\nonumber
\end{eqnarray}
or by using (\ref{complete})
\begin{eqnarray}
S_{\theta
F}=\bigg[\frac{1}{2l+1}Tr_l{\lambda}^L\bigg]\bigg[i{\epsilon}_{abc}Tr_l\bigg(tr_2(\frac{1}{D_{GF}}{\sigma}_c)F_{ab}^F\bigg)+4i\sqrt{l(l+1)}Tr_ltr_2(\frac{1}{D_{GF}}D_{Gw})\bigg].
\end{eqnarray}


\subsection{Noncommutative Geometry As A Regulator }
It is not difficult to observe that the naive continuum limit $l{\longrightarrow}{\infty}$ of the above equation is
\begin{eqnarray}
(8l^2)\bigg(\int \frac{d{\Omega}}{4{\pi}}{\lambda}\bigg)\bigg(\int \frac{d{\Omega}}{4{\pi}}tr_2({\gamma})\bigg)=\bigg[\int \frac{d{\Omega}}{4{\pi}}{\lambda}\bigg]\bigg[(2il){\epsilon}_{abc}\int \frac{d{\Omega}}{4{\pi}}\bigg(tr_2(\frac{1}{{\cal D}_{G}}{\sigma}_c)F_{ab}\bigg)+(8l^2)\int \frac{d{\Omega}}{4{\pi}}tr_2({\gamma})\bigg].\nonumber
\end{eqnarray}
By dividing across by $l^2$ this becomes an identity $0=0$ . In other words the fact that $S_{\theta F}=0$ in the fuzzy does not necessarily mean that the anomaly vanishes in the continuum . The claim (which we will explain shortly in detail) is that the Dirac operator $D_F$ is inconsistent {\it in the continuum limit with chiral symmetry on top modes } which is exactly the source of the anomaly.
Hence a gauge-invariant regularization of this  behaviour is required to recover the anomaly in the
continuum .  We adopt the prescription

\begin{eqnarray}
S_{\theta
\Lambda}&{\equiv}&\bigg[\frac{1}{2l+1}Tr_l{\lambda}^{L}\bigg]\bigg[i{\epsilon}_{abc}Tr_{l}\bigg(tr_2\big(\frac{1}{D_{G \Lambda }}{\sigma}_c)F_{ab}^F\bigg)+4i\sqrt{l(l+1)}Tr_ltr_2\big(\frac{1}{D_{G \Lambda }}D_{w \Lambda }\big)\bigg]\nonumber\\
&=&\bigg[\frac{1}{2l+1}Tr_l{\lambda}^{L}\bigg]\bigg[i{\epsilon}_{abc}Tr_{l}\bigg(tr_2(\frac{1}{D_{G \Lambda }}-\frac{1}{D_{GF}}){\sigma}_cF_{ab}^F\bigg)\bigg]\nonumber\\
&+&\bigg[\frac{1}{2l+1}Tr_l{\lambda}^{L}\bigg]\bigg[4i\sqrt{l(l+1)}Tr_{l}\bigg(tr_2\bigg(\frac{1}{D_{G\Lambda}}D_{w\Lambda}-\frac{1}{D_{GF}}D_{Gw}\bigg)\bigg],
\label{Lamb}\nonumber
\end{eqnarray}
where we have simply made the replacement
$D_{GF}^{-1}{\longrightarrow}D_{G\Lambda}^{-1}$ , $D_{Gw}{\longrightarrow}D_{w\Lambda }$ in the first line
above, then used again $S_{\theta F}=0$ . The operator $\frac{1}{D_{GF}}D_{Gw}$ as we saw tends in the limit to $-i{\gamma}$ , i.e it behaves as a chirality , whereas the regulator $\Lambda$ is meant only to regularize the free Dirac operator $D_F$ and not  alter chiralities. Therefore a natural choice for $D_{w\Lambda}$ which does not alter chiralities is such that
\begin{eqnarray}
\frac{1}{D_{G\Lambda }}D_{w\Lambda}=\frac{1}{D_{GF}}D_{Gw}.\nonumber
\end{eqnarray}
In perturbation theory
the Dirac operator $D_{GF}$ admits the expansion
\begin{eqnarray}
\frac{1}{D_{GF}}&=&\frac{1}{D_{F}}-\frac{1}{D_F}{\sigma}_aA_a^F\frac{1}{D_F}+\frac{1}{D_F}{\sigma}_aA_a^F\frac{1}{D_F}{\sigma}_aA_a^F\frac{1}{D_F}-...\nonumber\\
&=&\frac{1}{D_F}+\sum_{n=1}(-1)^n\frac{1}{D_F}{\sigma}_aA_a^F\frac{1}{D_F}....{\sigma}_aA_a^F\frac{1}{D_F},\nonumber
\end{eqnarray}
and hence one deduce immediately that to linear terms in the gauge field \cite{nagao1}, the anomaly is given by
\begin{eqnarray}
S_{\theta
\Lambda}
&=&\bigg[\frac{1}{2l+1}Tr_l{\lambda}^{L}\bigg]\bigg[i{\epsilon}_{abc}Tr_{l}\bigg(tr_2(\frac{1}{D_{\Lambda }}-\frac{1}{D_{F}}){\sigma}_cF_{ab}^F\bigg)\bigg].
\label{Lamb}
\end{eqnarray}
The regulator ${\Lambda}$ will now be defined precisely . We
first show that the regulator ${\Lambda}$ is essentially provided
by the chiral properties of the free Dirac operator $D_F$ in the
UV domain and hence most contributions to the chiral anomaly
(\ref{Lamb}) are coming from high frequency modes of the
spectrum. Next recall the continuum limits $
\vec{x}^R{\longrightarrow}\vec{x}~,~\vec{x}^F{\longrightarrow}\vec{x}$
, ${\Gamma}^R{\longrightarrow}-{\gamma}$,
${\Gamma}^L{\longrightarrow}{\gamma}$ , and in particular remark
that because $\vec{x}^R$ tends to $\vec{x}$ in the limit
$l{\longrightarrow}{\infty}$ , the extra minus sign in the
commutation relations of these $x^R_i$'s becomes completely
unobserved in the limit which is indeed expected . But from the
free fuzzy eigenvalues equation
\begin{eqnarray}
\frac{D_F}{2l+1}{\phi}(\mu,0)=\frac{{\Gamma}^R+{\Gamma}^L}{2}{\phi}(\mu,0)=\frac{{\lambda}_{\mu}}{2l+1}{\phi}(\mu,o),\label{side}
\end{eqnarray}
i.e equation (\ref{key}) ( recall that ${\mu}$ stands for
$k=0,...,2l$ , $j=k-\frac{1}{2},k+\frac{1}{2}$ and $m=-j,...,+j$
and ${\lambda}_{\mu}=j(j+1)-k(k+1)+\frac{1}{4}$) we can easily
see that for all the infrared modes $\mu<<2l$, the limits
${\Gamma}^R{\longrightarrow}-{\gamma}$,
${\Gamma}^L{\longrightarrow}{\gamma}$ are indeed satisfied since
both left and right hand sides of (\ref{side}) vanish in the
continuum limit , i.e
\begin{eqnarray}
Lim_{l{\longrightarrow}{\infty}}\big[\frac{D_F}{2l+1}{\phi}(\mu,0)\big]=Lim_{\l{\longrightarrow}{\infty}}\big[\frac{{\Gamma}^R+{\Gamma}^L}{2}{\phi}(\mu,0)\big]=0.
\end{eqnarray}
But for ultraviolet modes $\mu{\sim}2l$ we have instead
\begin{eqnarray}
Lim_{l{\longrightarrow}{\infty}}\big[\frac{D_F}{2l+1}{\phi}(\mu,0)\big]={\pm}1~,~{\rm
whereas}~
Lim_{\l{\longrightarrow}{\infty}}\big[\frac{{\Gamma}^R+{\Gamma}^L}{2}{\phi}(\mu,0)\big]=0.\label{364}
\end{eqnarray}
The sign is $+1$ if ${\lambda}_{\mu}$ is a positive energy
eigenvalue and $-1$ if ${\lambda}_{\mu}$ is a negative energy
eigenvalue . Using the identity
$\frac{2l+1}{D_F}=\frac{(2l+1)^2}{D^2_F}\frac{{\Gamma}^R+{\Gamma}^L}{2}$
first , then taking the limit $l{\longrightarrow}{\infty}$ one can
rewrite (\ref{364}) in the form
\begin{eqnarray}
Lim_{l{\longrightarrow}{\infty}}\big[\frac{2l+1}{D_F}{\phi}(\mu,0)\big]={\pm}1~,~{\rm
whereas}~
Lim_{\l{\longrightarrow}{\infty}}\big[\frac{2}{{\Gamma}^R+{\Gamma}^L}{\phi}(\mu,0)\big]=0.\label{365}
\end{eqnarray}
All this means in particular that either $1)$ the limits
${\Gamma}^R{\longrightarrow}-{\gamma}$ and
${\Gamma}^L{\longrightarrow}{\gamma}$ are not valid in the UV
domain or that $2)$ the free Dirac operator $D_F$ is not
appropriate in describing chiral symmetry in the UV . Since one
wants to maintain chiral symmetry explicitly throughout, the
limits ${\Gamma}^R{\longrightarrow}-{\gamma}$,
${\Gamma}^L{\longrightarrow}{\gamma}$ should always hold true and
one must instead
 replace $D_F^{-1}$ ( when restricted to the UV modes ) with  a new Dirac
operator $D_{\Lambda}^{-1}$ with the key property of having
vanishing eigenvalues for ${\mu}{\sim}2l$ . Such a Dirac operator
$D_{w}$  already exists in the literature , it was found
originally by Watamuras \cite{watamuras} and used extensively in
\cite{trg,bal} to construct , among other topological quantities
, the global anomaly on the noncommutative matrix model (\ref{44})
. In here and as it turns out this Dirac operator is also very
useful in extracting the local anomaly on the continuum limit of
(\ref{44}).

In order to define $D_{\Lambda}^{-1}$ more precisely , we proceed
simply by writing identities . We have
\begin{eqnarray}
\frac{1}{D_F}&=&\frac{1}{D_F^2}\frac{2l+1}{2}({\Gamma}^R+{\Gamma}^L)\nonumber\\
&=&\frac{1}{2l+1}\frac{1}{D_F^2}{\Gamma}^R\bigg(D^2_F+2i\sqrt{l(l+1)}D_{w}\bigg),\nonumber
\end{eqnarray}
or equivalently
\begin{eqnarray}
\frac{1}{D_F}&=&\frac{{\Gamma}^R}{2l+1}+2i\frac{\sqrt{l(l+1)}}{2l+1}\frac{1}{D_F^2}{\Gamma}^RD_{w},\label{result0}
\end{eqnarray}
where we have used the results
${\Gamma}^R{\Gamma}^{L}=-1+\frac{1}{2(l+\frac{1}{2})^2}\big(D^2_F+2i\sqrt{l(l+1)}D_{w}\big)$
, ${\Gamma}^RD_F-D_F{\Gamma}^L=0$ , and $D_{w}$ is exactly the
Watamura Dirac operator given by \cite{watamuras,ydri}
\begin{eqnarray}
D_{w}={\epsilon}_{abc}{\sigma}_a\frac{L_b^L}{\sqrt{l(l+1)}}L_c^R.
\end{eqnarray}
We now recall few facts important to us here about this Dirac
operator $D_{w}$. In the continuum ${\cal D}_{w}$ is related to
${\cal D}$ by ${\cal D}_{w}=i{\gamma}{\cal D}$ and hence both
operators ${\cal D}_{w}$ and ${\cal D}$ have the same spectrum (
that does not mean that they commute in fact $\{{\cal
D}_{w},{\cal D}\}=0$ ) . In the fuzzy , the spectrum of $D_{F}$ (
the fuzzy analogue of ${\cal D}$ ) is simply cut-off at the top
modes $j=2l+\frac{1}{2}$ while the spectrum of $D_{w}$ ( the
fuzzy analogue of ${\cal D}_{w}$ ) is quite deformed , in
particular the eigenvalues of $D_{w}$ when $j=2l+\frac{1}{2}$ are
now exactly zero  while for other large $j$'s these eigenvalues
are very small which is precisely the correct behaviour we want
at the top modes \cite{watamuras,ydri}. In fact as a consequence
of this behaviour we have the exact anticommutation relation
$\{{\Gamma}^R,D_{w}\}=0$ .


The remarkable identity (\ref{result0}) gives essentially the
prescription we want . One can immediately see that the behaviour
of the operator $\frac{2l+1}{D_F}$ as a sign operator in the UV (
see equation (\ref{365}) ) comes entirely from the first term
${\Gamma}^R$ in (\ref{result0}) since the second term clearly
vanishes. Remark also that the operator
$2i\frac{\sqrt{l(l+1)}}{2l+1}\frac{1}{D_F^2}{\Gamma}^RD_{w}$ has
the limit
\begin{eqnarray}
Lim_{l{\longrightarrow}{\infty}}\bigg[2i\frac{\sqrt{l(l+1)}}{2l+1}\frac{1}{D_F^2}{\Gamma}^RD_{w}\bigg]=\frac{1}{{\cal
D}},
\end{eqnarray}
i.e it has the same limit as $\frac{1}{D_F}$ and hence it has
also the same continuum spectrum . In fact even in the
noncommutative fuzzy setting the IR part of the spectrum of the
operator $
2i\frac{\sqrt{l(l+1)}}{2l+1}\frac{1}{D_F^2}{\Gamma}^RD_{w}$
differs from that of $\frac{1}{D_F}$ only by correction of the
order of $1/l$ . The Dirac operator $D_{\Lambda}^{-1}$ can
therefore be identified with
\begin{eqnarray}
\frac{1}{D_{\Lambda}}{\equiv}2i\frac{\sqrt{l(l+1)}}{2l+1}\frac{1}{D_F^2}{\Gamma}^RD_{w},
\end{eqnarray}
or equivalently
\begin{eqnarray}
\frac{1}{D_{\Lambda}}-\frac{1}{D_{F}}=-\frac{{\Gamma}^R}{2l+1}.\label{Dlambda}
\end{eqnarray}
This equation is viewed properly as a {\it noncommutative
regularization prescription} dictated by the requirement that the
free Dirac operator must be consistently chiral-invariant while
taking the continuum limit . The operator $\frac{1}{D_{\Lambda}}$
satisfies all the desired requirement  $a)$ its spectrum in the
IR is essentially identical to the spectrum of $\frac{1}{D_F}$ ,
 $b)$ It vanishes in the UV as required by chiral symmetry , see equation (\ref{365}) .
$c)$ Removing the regulator here is therefore the same as taking
the continuum limit in which case the two operators
$\frac{1}{D_F}$ and $\frac{1}{D_{\Lambda}}$ become identical .
The contribution of the difference ( which is here
$-\frac{{\Gamma}^R}{2l+1}$ ) drops in the continuum limit from the
spectrum yet its contribution to the variation of the quantum
measure (\ref{Lamb}) does not vanish , this is exactly the
anomaly.

To summarize , {\it the regulator ${\Lambda}$ consists simply of
approximating in a gauge-invariant manner the exact Dirac operator
$\frac{1}{D_F}$ with the operator $
2i\frac{\sqrt{l(l+1)}}{2l+1}\frac{1}{D_F^2}{\Gamma}^RD_{w}$} and
one obtains as a result the following anomaly on ${\bf S}^2_F$
\begin{eqnarray}
S_{\theta \Lambda}&=&\bigg[\frac{1}{2l+1}Tr_l{\lambda}^{L}\bigg]\bigg[i{\epsilon}_{abc}Tr_{l}\bigg(tr_2(\frac{1}{D_{\Lambda}}-\frac{1}{D_{F}}){\sigma}_cF_{ab}^F\bigg)\bigg]\nonumber\\
&=&-\bigg[\frac{1}{2l+1}Tr_l{\lambda}^{L}\bigg]\bigg[i{\epsilon}_{abc}\frac{1}{2l+1}Tr_{l}\bigg(tr_2\big({\Gamma}^R{\sigma}_c\big)F_{ab}^F\bigg)\bigg]\nonumber\\
&=&\bigg[\frac{1}{2l+1}Tr_l{\lambda}^{L}\bigg]\bigg[\frac{2\sqrt{l(l+1)}}{l+\frac{1}{2}}i{\epsilon}_{abc}
\frac{1}{2l+1}Tr_{l}\bigg(n_c^RF_{ab}^F\bigg)\bigg],
\end{eqnarray}
where we have used the identity
$tr_2{\Gamma}^R{\sigma}_c=-\frac{2\sqrt{l(l+1)}}{l+\frac{1}{2}}n_c^R$
. Using the $SU(2)$ coherent states $|\vec{n},l>$ or equivalently
the corresponding star product on ${\bf S}^2$ given in
\cite{lee,ref21} , then taking the continuum limit
$l{\longrightarrow}{\infty}$ one recovers the anomaly on
commutative ${\bf S}^2$ to be
\begin{eqnarray}
S_{\theta }=\bigg[\int
\frac{d{\Omega}}{4{\pi}}{\lambda}(\vec{n})\bigg]\bigg[i\frac{e}{2{\pi}}{\epsilon}_{abc}\int
\frac{d{\Omega}}{4{\pi}}n_cF_{ab}\bigg],\label{FA}
\end{eqnarray}
where for convenience we have scaled the electric charge from the
field strength as follows
$<\vec{n},l|F_{ab}^F|\vec{n},l>=\frac{e}{4{\pi}}F_{ab}(\vec{n})$ .



\section{Dixmier Trace As An Alternative Continuum Limit}


The claim of \cite{denjoe} is that the correct approximation of the continuum action (\ref{classiii}) is not the action (\ref{fuzziii}) but rather the action
\begin{eqnarray}
S_F^{'}&=&\frac{1}{2l+1}Tr_{l}\frac{1}{|D_F|^2}\bar{{\chi}}_F{D}_F{\chi}_F~                                .\label{fuzziii1}
\end{eqnarray}
This can be motivated as follows . The continuum action (\ref{classiii}) can be rewritten in terms of the Dixmier tarce in the form\cite{cmlv}
\begin{eqnarray}
S&=&Tr_{\omega}\frac{1}{|{\cal D}|^2}\bar{\chi}{\cal D}{\chi}\nonumber\\
&=&Tr_{\omega}\bar{\psi}{\cal
D}{\psi},\label{contaction}
\end{eqnarray}
where ${\psi}=\frac{1}{|{\cal
D}|}{\chi}$ .
The correct spinor is of course ${\chi}$ and not
${\psi}$ . $Tr_{\omega}$ is the Dixmier trace which is defined "roughly
speaking" as the coefficient of the logarithmic divergence of the
ordinary trace , which turns out to be , for continuum manifolds
, equal to the ordinary integral. In the continuum, its
definition can be given by Connes trace theorem \cite{cmlv}
\begin{eqnarray}
Tr_{\omega}|D|^{-d}f=\int_{M}dx^1{\wedge}dx^2{\wedge}...{\wedge}dx^{d}\sqrt{detg(x)}f(x),\nonumber
\end{eqnarray}
where $d$ and $D$ are the dimension and the Dirac operator of the
Riemannian spin manifold $M$ , and $g_{{\mu}{\nu}}(x)$ is the
metric on $M$.

The free fermionic action (\ref{fuzziii}) on fuzzy ${\bf S}^2$ is essentially
the fuzzy analogue of the second line of (\ref{contaction}) ,
namely
\begin{eqnarray}
S_F=Tr_{\omega,F}\bar{{\psi}}_F{D}_F{\psi}_F~,~Tr_{\omega,F}=\frac{1}{2l+1}Tr_l.\label{fuzzyaction}
\end{eqnarray}
Now the fuzzy analogue of the first line of (\ref{contaction}) ( which is the action (\ref{fuzziii1})  ) is
not equal to (\ref{fuzzyaction}) because the fuzzy Dixmier trace
$Tr_{\omega,F}$ does not satisfy any of the properties satisfied
by $Tr_{\omega}$ and which were needed in proving
(\ref{contaction}). In particular the crucial property of
$Tr_{\omega}$ which is that the Dixmier trace of operators of
order higher than one vanishes identically , is not satisfied by
$Tr_{\omega,F}$. The
correct noncommutative fuzzy spinor when approaching the limit is therefore not  ${\psi}_F$ but rather ${\chi}_F$ defined by $
{\psi}_F=\frac{1}{|{    D}_F|}{\chi}_F$ .




For completeness we now show that even with this assumption both actions (\ref{fuzziii1}) and (\ref{fuzzyaction}) yields the same physics at the limit . In particular we show that the divergence of the current given by equation (\ref{divergence}) has ( upto an overall logarithmic divergence ) the same continuum limit for both (\ref{fuzziii1}) and (\ref{fuzzyaction}) , i.e the star product used in deriving the final result (\ref{FA}) is upto an overall logarithmic divergence the same as the Dixmier trace . Indeed it is a trivial statement that (\ref{divergence}) tends in the continuum limit ( in the sense of the star product ) to the canonical result

\begin{eqnarray}
{\Delta}S_{GF}&=&\int \frac{d{\Omega}}{4{\pi}}{\lambda}(\vec{n}){\cal L}_a\big(\bar{{\psi}}{\sigma}_a{\gamma}{{\psi}}\big)(\vec{n}).
\end{eqnarray}
To show this result in the case of the Dixmier trace is however a more involved exercise .



The first step towards this end is to use the fact that ${\psi}_F$ is
not the correct spinor and that we need to substitute it with
$\frac{1}{|{D}_F|}{\chi}_F$ in
${\Delta}{S}_{GF}$ . The
second step is to write explicitly the trace in the Hilbert space
${\bf H}$ in terms of its $SU(2)-$coherent states's basis
$|\vec{n},l>$ . We then obtain in the large $l$ limit
\begin{eqnarray}
{\Delta}{S}_{GF}{\simeq}g_l\int
\frac{d{\Omega}_1}{4{\pi}}\frac{d{\Omega}_2}{4{\pi}}{\cal
L}_a({\lambda})(\vec{n}_1)\bar{\chi}(\vec{n}_1)[\frac{1}{|{D}_F|}]_{n_1n_2l}{\sigma}_a{\gamma}(\vec{n}_2)[\frac{1}{|{D}_F|}]_{n_2n_1l}{\chi}(\vec{n}_1),\label{FG2}\nonumber
\end{eqnarray}
The notation is $O_{n_1n_2l}=<\vec{n}_1,l|O|\vec{n}_2,l>$ and
where $g_l$ is given by $g_l=-\frac{1}{(2l+1)(4{\pi})^4}$ .
We have also used above the fact that in the continuum limit
$l{\longrightarrow}{\infty}$ the coherent state $|\vec{n},l>$
becomes localized at $\vec{n}$ . The next step is crucial and
involves the use of heat kernel techniques to extract the
continuum answer from the above action \cite{denjoe,heat}. These
considerations, which we will explain next, were first put
forward in \cite{denjoe} and were used to study the continuum
limit of the free scalar as well as the free fermionic fuzzy
actions on ${\bf S}^2_F$ . The assumption here is that these
considerations are still valid in the presence of gauge fields
if these latter can be treated as weak perturbations in the
large $l$ limit . The spectrum of the fuzzy Dirac operator
${D}_F$ was shown to be precisely the spectrum of the continuum
Dirac operator ${\cal D}$ only cut-off at the top eigenvalue
$j=2l+\frac{1}{2}$, and therefore in the limit
$l{\longrightarrow}{\infty}$ , it is reasonable to assume that
the operator $|{D}_F|^{-1}$ will converge to its continuum
counterpart $|{\cal D}_{c}|^{-1}$ whose eigenvalues smaller than
the cut-off eigenvalue
${\Lambda}(l)^{-1}=|{D}_F(j=2l+\frac{1}{2})|^{-1}=|2l+1|^{-1}$
are simply set to zero . The above action takes then the form
\begin{eqnarray}
{\Delta}{S}_{GF}{\simeq}g_l\int
\frac{d{\Omega}_1}{4{\pi}}\frac{d{\Omega}_2}{4{\pi}}{\cal
L}_a({\lambda})(\vec{n}_1)\bar{\chi}(\vec{n}_1)[\frac{1}{|{\cal
D}_{c} |}]_{n_1n_2}{\sigma}_a{\gamma}(\vec{n}_2)[\frac{1}{|{\cal
D}_{c}|}]_{n_2n_1}{\chi}(\vec{n}_1).\label{FG3}
\end{eqnarray}
At the large $l$ limit , the truncated operator $|{\cal
D}_{c}|^{-1}$ converges weakly to the usual continuum operator
$|{\cal D}|^{-1}$ because of its boundedness , and therefore the
high frequency behaviour of ${D}_{F}$ is irrelevant . However in
order to obtain the expected logarithmic divergence at the limit
$l{\longrightarrow}{\infty}$, it is not enough , as one can check
, to simply replace in (\ref{FG3}) all the operators $|{\cal
D}_{c}|^{-1}$ by their weak limit $|{\cal D}|^{-1}$ . To recover
this crucial logarithmic divergence , the high frequency
behaviour , which is irrelevant for individual operators , is
important to take into account in the whole expression . As was
shown in \cite{denjoe}, the proper way to take the continuum
limit is to leave in the action (\ref{FG3}) only any one
truncated operator and substitute the other with its weak limit.
We have then
\begin{eqnarray}
{\Delta}{S}_{GF}{\simeq}g_l\int
\frac{d{\Omega}_1}{4{\pi}}\frac{d{\Omega}_2}{4{\pi}}{\cal
L}_a({\lambda})(\vec{n}_1)\bar{\chi}(\vec{n}_1)[\frac{1}{|{\cal
D}_{c} |}]_{n_1n_2}{\sigma}_a{\gamma}(\vec{n}_2)[\frac{1}{|{\cal
D}|}]_{n_2n_1}{\chi}(\vec{n}_1).\label{FG4}
\end{eqnarray}
We need now to evaluate the heat kernels $[\frac{1}{|{\cal D}_{c}
|}]_{n_1n_2}$ and  $[\frac{1}{|{\cal D}|}]_{n_2n_1}$ . To this
end we first remark that a reliable approximation for ${|{\cal
D}_{c}|}^{-1}$ is ${|{\cal D}_{l}|}^{-1}$ whose modes with
eigenvalues smaller than ${|{\Lambda}(l)|}^{-1}$ are
exponentially suppressed , i.e \cite{denjoe}
\begin{eqnarray}
\frac{1}{|{\cal
D}_{l}|}&{\equiv}&\frac{1}{\Gamma(\frac{1}{2})}\int_{0}^{\infty}\frac{dt}{t}t^{\frac{1}{2}}e^{-{\cal
D}_{l}^2t}
{\simeq}\frac{1}{\Gamma(\frac{1}{2})}\int_{T(l)}^{\infty}\frac{dt}{t}t^{\frac{1}{2}}e^{-{\cal
D}^2t},
\end{eqnarray}
where we have substituted ${\cal D}$ for ${\cal D}_{l}$ but
restricted the integration to
$t{\geq}T(l)=\frac{1}{{\Lambda}(l)^2}$ , the eigenvalues ${\cal
D}^2>>{\Lambda}^2(l)$ will then all be suppressed and that is
precisely ${\cal D}_{l}$ . At $l{\longrightarrow}{\infty}$ , the
logarithmically divergent piece in (\ref{FG4}) will come from the
limit when $\vec{n}_1{\longrightarrow}\vec{n}_2$ , i.e from
contact terms . This can be understood from the fact that the
heat kernel $G_l(\vec{n}_1,\vec{n}_2)=[\frac{1}{|{\cal
D}_{l}|}]_{n_1n_2}$ has a singularity as
$\vec{n}_1{\longrightarrow}\vec{n}_2$ , which is coming from its
short time behavior \cite{heat} , and therefore this heat kernel
is effectively given by the integral
\begin{eqnarray}
G_l(\vec{n}_1,\vec{n}_2)&=&\frac{1}{{\Gamma}(\frac{1}{2})}\int_{T(l)}^{T_0}\frac{dt}{t}t^{\frac{1}{2}}<\vec{n}_1|e^{-{\cal
D}^2t}|\vec{n}_2>,
\end{eqnarray}
where $T_0$ is any small number larger than $T(l)$ . Putting it
differently , the contribution to the heat kernel
$G_l(\vec{n}_1,\vec{n}_2)$ coming from the integration over
$t{\geq}T_0$ is a finite quantity which vanishes at the large $l$
limit . In the expression for the heat kernel above , we are now
allowed to use the asymptotic formula at short times $
<\vec{n}_1|e^{-{\cal
D}^2t}|\vec{n}_2>{\simeq}\frac{1}{4{\pi}t}e^{-\frac{|\vec{n}_1-\vec{n}_2|^2}{4t}}
$ and hence \cite{denjoe,heat}
\begin{eqnarray}
G_l(\vec{n}_1,\vec{n}_2)&=&\frac{1}{{\Gamma}(\frac{1}{2})}\int_{T(l)}^{T_0}\frac{dt}{t}t^{\frac{1}{2}}\frac{1}{4{\pi}t}e^{-\frac{|\vec{n}_1-\vec{n}_2|^2}{4t}}.
\end{eqnarray}
If we let $l$ goes to infinity in the above equation , the left
hand side will give the heat kernel
$G(\vec{n}_1,\vec{n}_2)=[\frac{1}{|{\cal D}|}]_{n_1n_2}$,
whereas in the right hand side $T(l){\longrightarrow}0$. We then
obtain a formula for the other heat kernel, namely
\begin{eqnarray}
G(\vec{n}_1,\vec{n}_2)=\frac{1}{{\Gamma}(\frac{1}{2})}\int_{0}^{\infty}\frac{dt}{t}t^{\frac{1}{2}}\frac{1}{4{\pi}t}e^{-\frac{|\vec{n}_1-\vec{n}_2|^2}{4t}}=\frac{1}{2{\pi}}\frac{1}{|\vec{n}_1-\vec{n}_2|},
\end{eqnarray}
where we have also let $T_0$ goes to infinity as this will only
add regular terms when $\vec{n}_1{\longrightarrow}\vec{n}_2$ but
allows us to evaluate the integral exactly .

The relevant limit , $l{\longrightarrow}{\infty}$ ,
$\vec{n}_1{\longrightarrow}\vec{n}_2$ , suggests the following
change of variables , $2\vec{n}=\vec{n}_1+\vec{n}_2$ and
$\vec{\xi}=\vec{n}_1-\vec{n}_2$ in (\ref{FG4}) . Clearly
$\vec{n}_1$ and $\vec{n}_2$ are normalized such that
$\vec{n}_1^2=\vec{n}_2^2=1$ and therefore
$\vec{n}^2=1-\frac{1}{4}\vec{\xi}^2$ which means that $\vec{n}$
will generate a sphere in the limit of interest
$\vec{n}_1{\longrightarrow}\vec{n}_2$ or
$\vec{\xi}{\longrightarrow}0$, and hence one can take
$d{\Omega}_1d{\Omega}_2=d^2\vec{\xi}d{\Omega}={\xi}d{\xi}d{\Omega}_{\xi}d{\Omega}$
. The statement here is that the continuum limit is the same if
we perform the above change of coordinates and just integrate
$\vec{n}$ over a sphere whereas integrate $\vec{\xi}$ over a
plane, since the short distance behaviour in the
$\vec{\xi}-$variables is really what matters after all . Using
the expansions
$f(\vec{n}_1)==f(\vec{n})+\frac{\xi^i}{2}{\partial}_if(\vec{n})+O(\xi^2)$
and
$f(\vec{n}_2)=f(\vec{n})-\frac{\xi^i}{2}{\partial}_if(\vec{n})+O(\xi^2)$
in (\ref{FG4}) we obtain
\begin{eqnarray}
{\Delta}{S}_{GF}{\simeq}\frac{g_l}{(4{\pi})^2}\int_{{\bf
S}^2}d^2\vec{\xi}d{\Omega}\bigg[{\cal
L}_a({\lambda})(\vec{n})\bar{\chi}(\vec{n})G_l(\xi){\sigma}_a{\gamma}(\vec{n})G(\xi){\chi}(\vec{n})+O({\xi}^2)\bigg],\label{429}
\end{eqnarray}
where linear terms in $\vec{\xi}$ vanish by rotational invariance . Now by using the identity
\begin{eqnarray}
\int d^2\vec{\xi}G_l(\xi)G(\xi)=\int d{\xi}G_l(\xi)=\frac{1}{2\pi}ln l +~{\rm finite}~ {\rm terms},\label{result}
\end{eqnarray}
we obtain the final answer , which is the ordinary total divergence of the axial current , namely
\begin{eqnarray}
{\Delta}{S}_{GF}{\longrightarrow}{\Delta}{S}_{G}=-\frac{2g_llnl}{(4{\pi})^2}\int_{{\bf
S}^2}\frac{d{\Omega}}{4{\pi}}{\lambda}(\vec{n}){\cal
L}_a\big(\bar{\chi}{\sigma}_a{\gamma}{\chi}\big)(\vec{n}).\label{anomaly0}
\end{eqnarray}
The overall normalization $\frac{1}{\cal N}=-\frac{2g_llnl}{(4{\pi})^2}=\frac{1}{(4{\pi})^6}\frac{lnl}{l}$ clearly vanishes in the limit of large $l$ and can be accounted for by simply redefining the Dirac operator $D_{\Lambda}$ in (\ref{Dlambda}) as follows
\begin{eqnarray}
\frac{1}{D_{\Lambda}}-\frac{1}{D_{F}}=-\frac{1}{\cal N}\frac{{\Gamma}^R}{2l+1}.
\end{eqnarray}
This redefinition does not alter any of the properties of the Dirac operator $D_{\Lambda}$ as ${\cal N}{\longrightarrow}{\infty}$ when $l{\longrightarrow}{\infty}$ . Putting together this last result (\ref{anomaly0}) with the result
(\ref{FA}) for global parameters ${\lambda}$ , we obtain the local chiral anomaly
equation
\begin{eqnarray}
{\cal
L}_a(\bar{\chi}{\sigma}_a{\gamma}{\chi})(\vec{n})=i\frac{e}{2{\pi}}{\epsilon}_{abc}n_cF_{ab}.
\end{eqnarray}








\section{Conclusion}
We showed that on the noncommutative matrix model ${\bf S}^2_F$
given by the commutation relations (\ref{44}) the gauge-invariant
Dirac operator $D_{GF}$ ( equation (\ref{GFsummary}) ) is
different from the chiral-invariant Dirac operator $D_{CF}$ (
equation (\ref{3.38}) ) . The kinetic term in the corresponding
actions (\ref{gaugeFaction}) and (\ref{3F}) is however the same
and hence in both cases we have the nice feature of having a free
spectrum which is identical to the continuum spectrum with a
natural cut-off . In particular as we go deep in the UV domain the
gauge-invariant modes do not coincide with the chiral-invariant
modes and the net effect is essentially the source of the anomaly
in two dimensions . $D_{GF}$ and $D_{CF}$ become identical only
at large distances where modes can be both chiral- and
gauge-symmetric simultaneously.

The fact that gauge states are different from chiral states stems
already from the free noncommutative Dirac operator $D_F$ which
was found to be inconsistent with chiral symmetry at high
energies ( see equation (\ref{365}) ) . Furthermore it was found
to split such that
\begin{eqnarray}
\frac{1}{D_F}={\theta}{\Gamma}^R+\frac{1}{D_{\Lambda}}~,~\theta=\frac{1}{2l+1}\nonumber
\end{eqnarray}
where $\theta$ is the noncommutativity parameter or in some sense
the lattice spacing, ${\Gamma}^R$ is the noncommutative chirality
and $D_{\Lambda}$ is identified as the correct Dirac operator on
this matrix model . $D_{\Lambda}$ given by (\ref{Dlambda}) has
essentially the same IR spectrum as $D_{F}$ , both operators
share the same noncommutative continuum limit yet $D_{\Lambda}$
is consistent with chiral symmetry on the UV modes in the sense
of equation (\ref{365}).

The second step of the fuzzy regularization implemented in this
paper consisted simply of approximating the exact Dirac operator
$D_F$ with $D_{\Lambda}$ . This regularization was shown to yield
in the gauge-invariant quantized Ginsparg-Wilson relation
(\ref{Lamb}), i.e
\begin{eqnarray}
S_{\theta \Lambda}
&=&\bigg[\frac{1}{2l+1}Tr_l{\lambda}^{L}\bigg]\bigg[i{\epsilon}_{abc}Tr_{l}\bigg(tr_2(\frac{1}{D_{\Lambda}}-\frac{1}{D_{F}}){\sigma}_cF_{ab}^F\bigg)\bigg]
\end{eqnarray}
to an anomaly which in terms of Dirac operators depends only on
the difference $\frac{1}{D_F}-\frac{1}{D_{\Lambda}}$. The
difference ${\theta}{\Gamma}^R$ clearly drops from the spectrum
in the limit yet its contribution to the variation of the measure
under chiral transformations is not zero , it gives exactly the
anomaly (\ref{FA}).

The divergence of the axial current was also derived using two
methods , the star product and the Dixmier trace , and the theta
term found.

\begin{center}
{\bf\large Acknowledgments}
\end{center}
I would like to thank A.P.Balachandran and G.Immirzi for their
critical comments while the work was in progress. I would also
like to thank  B.P.Dolan, X.Martin , Denjoe O'Connor and
P.Presnajder  for their helpful discussions . This work was
supported in part by the DOE under contract number
DE-FG02-85ER40231.



\bibliographystyle{unsrt}

\begin{thebibliography}{99}

\bibitem{madore}
J.~Madore.
\newblock {\em An Introduction to Noncommutative Differential Geometry
and its Applications}.
\newblock Cambridge University Press, Cambridge, 1995;
{\tt gr-qc/9906059}.

\bibitem{GKP}
H.~Grosse, C.~Klimcik, P.~Presnajder.\newblock{\tt
hep-th/9602115};\newblock{\em Commun.Math.Phys.}{\bf
180}(1996)429-438. H.~Grosse, C.~Klimcik, P.~Presnajder.
\newblock{\tt hep-th/9505175}; \newblock{\em Int.J.Theor.Phys.}{\bf 35}(1996)231-244.
H.~Grosse, A.~Strohmair .
\newblock{\tt hep-th/9902138}; \newblock{\em Lett.Math.Phys.}{\bf 48}(1999)163-175.

\bibitem{ydri}
Badis Ydri , {\it Fuzzy Physics} , {\tt hep-th/0110006} , Ph.D
Thesis .

\bibitem{cmlv}
A.~Connes , {\em Noncommutative Geometry} , Academic Press ,
London, 1994 . G.~Landi , {\em An Introduction to Noncommutative
Spaces and Their Geometries}, Springer-Verlag , Berlin , 1997 ,
{\tt hep-th/9701078}. J.~C. Varilly , {\it An Introduction to
Noncommutative Geometry} , {\em physics/9709045}.

\bibitem{VKM}
D.A.Varshalovich , A.N.Moskalev , V.K.Khersonky , "Quantum Theory
of Angular Momentum : Irreducible Tensors , Spherical Harmonics ,
vector Coupling Coefficients, 3nj Symbols " , Singapore ,
Singapore : World Scientific (1998) .

\bibitem{ydri2}
S . Vaidya , B.Ydri ; "New Scaling Limit for Fuzzy Spheres" ,
hep-th/0209131 . S.Vaidya , {\it hep-th/0102212} and {\em
Phys.Lett.}{\bf B512} (2001) 403-411 . Chong-Sun Chu , John
Madore and Harold Steinacker , {\it hep-th/0106205} . B.P.Dolan ,
D.O'Connor , P.Pre\v{s}najder , {\it hep-th/0109084}. Sachindeo
Vaidya , Badis Ydri , "On The Origin of The UV-IR Mixing And
Noncommutative Matrix Geometry" , in preparation .

\bibitem{badis}
Badis . Ydri ; "The $1-$Loop NC Schwinger Model and Fuzzy QED in $2-$ and $4-$Dimensions" , in preparation .

\bibitem{presnajder}
P.Pre\v{s}najder , {\tt hep-th/9912050} and {\em
J.Math.Phys.}{\bf 41}(2000)2789-2804.




\bibitem{denjoe}
A.~P.~Balachandran , X.~Martin and D.~O'Connor , {\tt
hep-th/0007030} and {\em Int.J.Mod.Phys.}{\bf A16}(2001)2577-2594.

\bibitem{trg}
A.P.Balachandran , T.R.Govindarajan , B.Ydri , {\tt
hep-th/9911087}; A.P.Balachandran , T.R.Govindarajan , B.Ydri ,
{\tt hep-th/0006216} and {\em Mod.Phys.Lett.}{\bf A15} , 1279
(2000).

\bibitem{bal}
S.Baez , A.P. Balachandran , S.Vaidya and B.Ydri , {\tt
hep-th/9811169} and Comm.Math.Phys.{\bf 208},787(2000).


\bibitem{grosse}
H.~Grosse and P.~Pre\v{s}najder , {\em Lett.Math.Phys.}{\bf 33},
171-182(1995). H.~Grosse, C.~Klim\v{c}\'{\i}k and
P.~Pre\v{s}najder ,{\em Commun.Math.Phys.}{\bf 178}, 507-526
(1996). H.~Grosse, C.~Klim\v{c}\'{i}k and P.~Pre\v{s}najder in
{\em Les Houches Summer School on Theoretical Physics}, 1995,
{\tt hep-th/9603071} .


\bibitem{watamuras}
U.Carow-Watamura and S.Watamura ,{\tt hep-th/9605003},{\em
Comm.Math.Phys.}{\bf 183},365 (1997). U.Carow-Watamura and
S.Watamura,{\tt hep-th/9801195},{\em Comm.Math.Phys.} {\bf
212},395(2000).

\bibitem{nair}
D.Karabali , V.P.Nair and A.P.Polychronakos , {\tt
hep-th/0111249} .

\bibitem{lee}
A.P.Balachandran , B.P.Dolan , J.Lee , X.Martin and D.O'Connor ,
{\tt hep-th/0107099} , J.Geom.Phys. 43(2002)184-204.

\bibitem{ref21}
J.~Klauder and B.-S.~Skagerstam , Coherent States : {\em
Application in Physics and Mathematical Physics}, World
Scientific(1985); A.M.~Perelomov : {\em Generalized coherent
states and their applications}, Springer-Verlag,(1986);
M.~Bordemann, M.~Brischle, C.~Emmrich and S.~Waldmann,{\em
J.Math.Phys.}{\bf 37},6311(1996); M.~Bordemann, M.~Brischle,
C.~Emmrich and S.~Waldmann,{\em Lett.Math.Phys.}{\bf
36},357(1996); S.~Waldmann,{\em Lett.Math.Phys.}{\bf
44},331(1998).

\bibitem{fujikawa}
K.Fujikawa ,{\em Phys.Rev.Lett.}{\bf 42},1195(1979) ; {\em
Phys.Rev.}{\bf D21},2848(1980).

\bibitem{leonardo}
Leonardo Ginsti , Christian Hoelbling , Claudio Rebbi , {\tt
hep-lat/0101015}, {\em Phys.Rev.D64} (2001) 054501.


\bibitem{heat}
Peter B.Gilkey , "Invariance Theory , The Heat Equation and The Atiyah-Singer Index Theorem" , Mathematics Lecture
Series , Vol 11, Publish or Perish , 1984 .



\bibitem{nielsen-ninomiya}
N.B.Nielsen , M.Ninomiya ,{\em Phys.Lett.}{\bf
B105},219(1981);{\em Nucl.Phys.}{\bf B185},20(1981).

\bibitem{creutz}
M . Creutz ,{\em Quarks , Gluons and Lattices} , Cambridge
University Press , Cambridge , 1983 .

\bibitem{giorgio}
G.Alexanian , A.P. Balachandran , G.Immirzi and B.Ydri . "Fuzzy
${\bf C}{\bf P}^{2}$", {\tt hep-th/0103023} ,
J.Geom.Phys.42(2002)28-53.

\bibitem{balsach}
A.P.Balachandran and S.Vaidya , Int.J.Mod.Phys.{\bf
A16},17(2001) .

\bibitem{nagao1}
H.~Aoki, S.~Iso and K.~Nagao,
``Ginsparg-Wilson relation, topological invariants and finite 
noncommutative geometry'' , {\tt hep-th/0209223}.


\bibitem{nagao2}
H.~Aoki, S.~Iso and K.~Nagao, ``Chiral anomaly on fuzzy 2-sphere``, {\tt hep-th/0209137}.


\bibitem{giorgio1}
Giorgio Immirzi , Badis Ydri ; " Chiral Symmetry on ${\bf
S}^2_F$" , {\tt hep-th/0203121}. Proceeding of The Theoretical
High-Energy Physics Conference At Utica, New York , 12 Oct 2001 ,
edited by A.H.Fariborz and E.Rusjan.

\bibitem{miguel}
J . Nishimura  , M .A .Vazquez-Mozo ,  {\tt hep-th/0107110}(JHEP 0108 (2001) 033) and  {\tt hep-lat/0210017} .

\bibitem{unp}
Badis Ydri , unpublished notes .


\end{thebibliography}












































\end{document}

