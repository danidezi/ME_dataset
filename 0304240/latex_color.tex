
\documentclass[11pt,a4paper]{article}
\usepackage{amsmath}
\usepackage{cite}
\usepackage{graphicx}
\usepackage{amsfonts}
\usepackage{amssymb}

\setlength {\topmargin}{-10mm} \setlength {\textwidth}{16cm}
\setlength {\textheight}{220mm} \setlength {\footskip}{2cm}
\setlength {\headheight}{10mm}
\setlength {\oddsidemargin}{-4mm} \setlength
{\evensidemargin}{0mm}
\def\bt{
\beta}

\def\bt'{\beta'
}
\usepackage{useful_macros}
\begin{document}
\bigskip
\hfill\hbox{SPhT-T03/052} \vspace{2cm}

\begin{center}
{\Large \textbf{Massless N=1 Super Sinh-Gordon:\\ \vspace{0.5cm}  Form Factors approach}} \\
\vspace{1.2cm} {\Large B\'en\'edicte Ponsot \footnote{\textsf
{ponsot@spht.saclay.cea.fr}}} \\
\vspace{0.7cm} {\it  Service de Physique Th\'eorique, Commissariat
\`a l'\'energie
atomique,\\
L'Orme des Merisiers,
 F-91191 Gif sur Yvette, France.}\\
\vspace{1.2cm}
\end{center}


\begin{abstract}
The \myHighlight{$N=1$}\coordHE{} Super Sinh-Gordon model with spontaneously broken supersymmetry is considered.
Explicit expressions for form-factors of operators \myHighlight{$e^{\alpha \phi}$}\coordHE{} of the Neveu-Schwartz sector
 and operators \myHighlight{$\sigma e^{\alpha \phi},\,\mu e^{\alpha \phi}$}\coordHE{} of the Ramond sector are proposed.
\end{abstract}
\begin{center}
 PACS: 11.25.Hf, 11.55.Ds
\end{center}

\vspace{1cm}
\section{Introduction}
The SShG model can be considered as a perturbed super Liouville
field theory, which lagrangian density is given by
$$\coord{}\boxMath{
\mathcal{L}=\frac{1}{8\pi}(\partial_a
\phi)^2-\frac{1}{2\pi}(\bar{\psi}\partial
\bar{\psi}+\psi\bar{\partial} \psi) +i\mu b^2\psi \bar{\psi}
e^{b\phi}+\frac{\pi \mu^2 b^2}{2}e^{2b\phi}.
}{dollar}{0pt}\coordE{}$$
with the background charge \myHighlight{$Q=b+1/b$}\coordHE{}. This model is a CFT with
central charge
$$\coord{}\boxMath{
c_{SL}=\frac{3}{2}(1+2Q^2).
}{dollar}{0pt}\coordE{}$$
The super Sinh-Gordon model is 1+1 dimensional integrable quantum
field theory with \myHighlight{$N=1$}\coordHE{} supersymmetry. We consider the Lagrangian
$$\coord{}\boxMath{
\mathcal{L}=\frac{1}{8\pi}(\partial_a
\phi)^2-\frac{1}{2\pi}(\bar{\psi}\partial
\bar{\psi}+\psi\bar{\partial} \psi) +2i\mu b^2\psi \bar{\psi}
\sinh b\phi+2\pi \mu^2 b^2\cosh^2 b\phi.
}{dollar}{0pt}\coordE{}$$
In this model the supersymmetry is spontaneously broken
\cite{AKRZ}: the bosonic field becomes massive, but the Majorana fermion
stays massless and plays the role of Goldstino. In the IR limit, the
 effective theory for the Goldstino is to the lowest order
 the Volkov-Akulov lagrangian \cite{VA}
\begin{eqnarray}\coord{}\boxAlignEqnarray{\leftCoord{}
\mathcal{L}_{IR}=(\bar{\psi}\partial \bar{\psi}+\psi\bar{\partial}
\psi)-\frac{\leftCoord{}4}{\rightCoord{}M^2}(\psi\partial \psi)(\bar{\psi}\bar{\partial} \rightCoord{}
\bar{\psi})+\cdots
\label{effectif}
\rightCoord{}}{0mm}{2}{4}{
\mathcal{L}_{IR}=(\bar{\psi}\partial \bar{\psi}+\psi\bar{\partial}
\psi)-\frac{4}{M^2}(\psi\partial \psi)(\bar{\psi}\bar{\partial} 
\bar{\psi})+\cdots
}{1}\coordE{}\end{eqnarray}
where supersymmetry is realized non linearly. The irrelevant
operator along which the Super Liouville theory flows into Ising
is the product of stress-energy tensor \myHighlight{$T\bar{T}=(\psi\partial \psi)(\bar{\psi}\bar{\partial}
\bar{\psi})$}\coordHE{}, which is the
lowest dimension non derivative operator allowed by the
symmetries. The dots include higher dimensional irrelevant
operators.\\
The scattering in the left-left and right-right subchannels is
trivial, but not in the right-left channel. The following
scattering matrices were proposed in \cite{AKRZ}
$$\coord{}\boxMath{
S_{RR}(\theta)=S_{LL}(\theta')=-1,\; S_{RL}(\theta - \theta')=
-\frac{\sinh(\theta-\theta')-i\sin\pi\nu}{\sinh(\theta-\theta')+i\sin\pi\nu}
,\quad \nu\equiv b/Q.
}{dollar}{0pt}\coordE{}$$
For the right (left) movers the energy momentum is parametrized in
terms of the rapidity variable \myHighlight{$\theta$}\coordHE{} (\myHighlight{$\theta'$}\coordHE{}) by
\myHighlight{$p^0=p^1=\frac{M}{2}e^{\theta}$}\coordHE{} (and
\myHighlight{$p^0=-p^1=\frac{M}{2}e^{-\theta'}$}\coordHE{}). The mass scale of the theory
\myHighlight{$M^{-2}$}\coordHE{} is equal to \myHighlight{$2\sin \pi \nu$}\coordHE{}. The form factors\footnote{We
refer the reader to \cite{DMS} for a discussion on form factors in
massless QFT.}
\myHighlight{$F_{r,l}(\theta_1,\theta_2,\ldots,\theta_r;\theta'_1,\theta'_2,\ldots,\theta'_l)$}\coordHE{}
are defined to be matrix elements of an operator between the
vacuum and a set of asymptotics states. The form factor bootstrap
approach \cite{KW,BKW,S} (developed originally for massive
theories, but that turned out to be also an effective tool for massless theories \cite{DMS,MS}) leads to a system of
linear functional relations for the matrix elements \myHighlight{$F_{r,l}$}\coordHE{}; let
us introduce the minimal form factors which have neither poles nor
zeros in the strip \myHighlight{$0<\Im m\theta<\pi$}\coordHE{} and which are solutions of
the equations \myHighlight{$f_{\alpha_1 \alpha_2}(\theta)=f_{\alpha_1
\alpha_2}(\theta+2i\pi)S_{\alpha_1\alpha_2}(\theta),\;
\alpha_{i}=R,L$}\coordHE{}.\\
Then the general form factor is parametrized as follows:
\begin{eqnarray}\coord{}\boxAlignEqnarray{\leftCoord{}
\lefteqn{F_{r,l}^{\alpha}(\theta_1,\theta_2,\ldots,\theta_r;\theta'_1,\theta'_2,\ldots,\theta'_l)=}
\nonumber \rightCoord{}\\
&&\leftCoord{} \rightCoord{}
 \prod_{\rightCoord{}1\leq i<j\leq r}f_{RR}(\theta_i -\theta_j)
\prod_{\rightCoord{}i=1}^{\leftCoord{}r}\prod_{\rightCoord{}j=1}^{\leftCoord{}l}f_{RL}(\theta_i -\theta'_j)
\prod_{\rightCoord{}1\leq i<j\leq l}f_{LL}(\theta'_i -\theta'_j)
Q_{r,l}^{\alpha}(\theta_1,\theta_2,\ldots\theta_r;\theta'_1,\theta'_2,\ldots,\theta'_l),
\nonumber\rightCoord{}
\rightCoord{}}{0mm}{4}{9}{
\lefteqn{F_{r,l}^{\alpha}(\theta_1,\theta_2,\ldots,\theta_r;\theta'_1,\theta'_2,\ldots,\theta'_l)=}
\\
&& 
 \prod_{1\leq i<j\leq r}f_{RR}(\theta_i -\theta_j)
\prod_{i=1}^{r}\prod_{j=1}^{l}f_{RL}(\theta_i -\theta'_j)
\prod_{1\leq i<j\leq l}f_{LL}(\theta'_i -\theta'_j)
Q_{r,l}^{\alpha}(\theta_1,\theta_2,\ldots\theta_r;\theta'_1,\theta'_2,\ldots,\theta'_l),
}{1}\coordE{}\end{eqnarray}
and the function \myHighlight{$Q_{r,l}^{\alpha}$}\coordHE{} depends on the operator considered through the parameter \myHighlight{$\alpha$}\coordHE{}
\footnote{The meaning of \myHighlight{$\alpha$}\coordHE{} will be clear in the next sections.}.\\
The \myHighlight{$RR$}\coordHE{} and \myHighlight{$LL$}\coordHE{} scattering formally behave as in the massive
case, so annihilation poles occur
 {\it only} in the \myHighlight{$RR$}\coordHE{} and \myHighlight{$LL$}\coordHE{} subchannel. This leads to the residue formula
\begin{eqnarray}\coord{}\boxAlignEqnarray{\leftCoord{}
\lefteqn{\mathrm{Res}_{\theta_{12}=i\pi}F_{r,l}^{\alpha}(\theta_1,\theta_2,\ldots,\theta_r;\theta'_1,\theta'_2,\ldots,\theta'_l)=}
\nonumber \rightCoord{}\\
&&\leftCoord{} \rightCoord{}
\leftCoord{}2F_{r-2,l}^{\alpha}(\theta_3,\ldots,\theta_r;\theta'_1,\theta'_2,\ldots,\theta'_l)\left(1-\prod_{\rightCoord{}j=3}^{\leftCoord{}r}
S_{RR}(\theta_{2i})\prod_{\rightCoord{}k=1}^{\leftCoord{}l}S_{RL}(\theta_2-\theta'_k)\right),
\label{residue}
\rightCoord{}}{0mm}{5}{6}{
\lefteqn{\mathrm{Res}_{\theta_{12}=i\pi}F_{r,l}^{\alpha}(\theta_1,\theta_2,\ldots,\theta_r;\theta'_1,\theta'_2,\ldots,\theta'_l)=}
\\
&& 
2F_{r-2,l}^{\alpha}(\theta_3,\ldots,\theta_r;\theta'_1,\theta'_2,\ldots,\theta'_l)\left(1-\prod_{j=3}^{r}
S_{RR}(\theta_{2i})\prod_{k=1}^{l}S_{RL}(\theta_2-\theta'_k)\right),
}{1}\coordE{}\end{eqnarray}
and a similar expression in the \myHighlight{$LL$}\coordHE{} subchannel.
 It
is important to note that these equations {\it do not} refer to
any specific operator.


\section{Expression for form factors}
 The minimal form factors read explicitly:
$$\coord{}\boxMath{
f_{RR}(\theta)=\sinh\frac{\theta}{2}, \ \
f_{LL}(\theta')=\sinh\frac{\theta'}{2},
}{dollar}{0pt}\coordE{}$$
and
$$\coord{}\boxMath{
f_{RL}(\theta) = \frac{1}{2\cosh\frac{\theta}{2}}\ \ \exp
\int_{0}^{\infty}
\frac{dt}{t}\frac{\cosh(\frac{1}{2}-\nu)t-\cosh\frac{1}{2}t}{\sinh
t \cosh t/2}\cosh t\left(1-\frac{\theta}{i\pi}\right).
}{dollar}{0pt}\coordE{}$$
The latter form factor has asymptotic behaviour when \myHighlight{$\theta \to
-\infty$}\coordHE{}
\begin{eqnarray}\coord{}\boxAlignEqnarray{\leftCoord{}
 f(\theta) \sim
e^{\theta/2}\left(1+\left(A + A'\theta\right) e^{\theta}+ \rightCoord{}
\left(\frac{\leftCoord{}A^2}{\rightCoord{}2}+B+ \rightCoord{}
AA'\theta+\frac{\leftCoord{}(A')^2\theta^2}{\rightCoord{}2}\right)e^{2\theta}\right). \rightCoord{}
\label{IR}
\rightCoord{}}{0mm}{3}{7}{
 f(\theta) \sim
e^{\theta/2}\left(1+\left(A + A'\theta\right) e^{\theta}+ 
\left(\frac{A^2}{2}+B+ 
AA'\theta+\frac{(A')^2\theta^2}{2}\right)e^{2\theta}\right). 
}{1}\coordE{}\end{eqnarray}
where \myHighlight{$A=(1-2\nu)\cos\pi\nu -1 + 2i\sin\pi\nu, \;
A'=-\frac{2}{\pi}\sin\pi\nu, \; B=\frac{1}{2}(\cos 2\pi\nu -1)$}\coordHE{}.
The logarithmic contributions come from resonances.\\
The residue condition (\ref{residue}) written in terms of the
function \myHighlight{$Q_{r,l}^{\alpha}$}\coordHE{} reads
\begin{eqnarray}\coord{}\boxAlignEqnarray{\leftCoord{}
\lefteqn{\mathrm{Res}_{\theta_{12}=i\pi}Q_{r,l}^{\alpha}(\theta_1,\theta_2,\ldots\theta_r;
\theta'_1,\theta'_2,\ldots,\theta'_l)
\leftCoord{}=Q_{r-2,l}^{\alpha}(\theta_3,\ldots\theta_r;\theta'_1,\theta'_2,\ldots,\theta'_l)\times
\leftCoord{}(-)^{r-1} \rightCoord{}
\leftCoord{}(2i)^{l+r-1}\times} \nonumber \rightCoord{}\\
&&\leftCoord{} \times\prod_{\rightCoord{}j=3}^{\leftCoord{}r}\frac{\leftCoord{}1}{\rightCoord{}\sinh\theta_{2j}} \left(
\prod_{\rightCoord{}k=1}^{\leftCoord{}l}(\sinh (\theta_2-\theta'_k)+i\sin\pi \nu)
\leftCoord{}-(-1)^{r+l}\prod_{\rightCoord{}k=1}^{\leftCoord{}l}(\sinh (\theta_2-\theta'_k)-i\sin\pi
\nu) \right). \rightCoord{}
\label{residu2}
\rightCoord{}}{0mm}{10}{9}{
\lefteqn{\mathrm{Res}_{\theta_{12}=i\pi}Q_{r,l}^{\alpha}(\theta_1,\theta_2,\ldots\theta_r;
\theta'_1,\theta'_2,\ldots,\theta'_l)
=Q_{r-2,l}^{\alpha}(\theta_3,\ldots\theta_r;\theta'_1,\theta'_2,\ldots,\theta'_l)\times
(-)^{r-1} 
(2i)^{l+r-1}\times} \\
&& \times\prod_{j=3}^{r}\frac{1}{\sinh\theta_{2j}} \left(
\prod_{k=1}^{l}(\sinh (\theta_2-\theta'_k)+i\sin\pi \nu)
-(-1)^{r+l}\prod_{k=1}^{l}(\sinh (\theta_2-\theta'_k)-i\sin\pi
\nu) \right). 
}{1}\coordE{}\end{eqnarray}
Let us introduce now the functions
\begin{eqnarray}\coord{}\boxAlignEqnarray{\leftCoord{}
\phi(\theta_{ij}) \equiv
\frac{\leftCoord{}S_{RR}}{\rightCoord{}f_{RR}(\theta_{ij})f_{RR}(\theta_{ij}+i\pi)}
\leftCoord{}=\frac{\leftCoord{}2i}{\rightCoord{}\sinh \theta_{ij}}\; ,\quad \phi(\theta'_{ij}) \equiv
\frac{\leftCoord{}S_{LL}}{\rightCoord{}f_{LL}(\theta'_{ij})f_{LL}(\theta'_{ij}+i\pi)} =
\frac{\leftCoord{}2i}{\rightCoord{}\sinh \theta'_{ij}}\; . \nonumber\rightCoord{}
\rightCoord{}}{0mm}{6}{7}{
\phi(\theta_{ij}) \equiv
\frac{S_{RR}}{f_{RR}(\theta_{ij})f_{RR}(\theta_{ij}+i\pi)}
=\frac{2i}{\sinh \theta_{ij}}\; ,\quad \phi(\theta'_{ij}) \equiv
\frac{S_{LL}}{f_{LL}(\theta'_{ij})f_{LL}(\theta'_{ij}+i\pi)} =
\frac{2i}{\sinh \theta'_{ij}}\; . }{1}\coordE{}\end{eqnarray}
as well as
\begin{eqnarray}\coord{}\boxAlignEqnarray{\leftCoord{}
\Phi(\theta_i-\theta'_j) &&\equiv
\frac{\leftCoord{}S_{RL}(\theta_i-\theta'_j)}{\rightCoord{}f_{RL}(\theta_i-\theta'_j)f_{RL}(\theta_i-\theta'_j+i\pi)}
\leftCoord{}= -\frac{\leftCoord{}\sinh(\theta_i-\theta'_j)-i\sin \pi \nu}{\rightCoord{}2i}\; ,
\nonumber\rightCoord{}
\rightCoord{}}{0mm}{4}{5}{
\Phi(\theta_i-\theta'_j) &&\equiv
\frac{S_{RL}(\theta_i-\theta'_j)}{f_{RL}(\theta_i-\theta'_j)f_{RL}(\theta_i-\theta'_j+i\pi)}
= -\frac{\sinh(\theta_i-\theta'_j)-i\sin \pi \nu}{2i}\; ,
}{1}\coordE{}\end{eqnarray}
and
\begin{eqnarray}\coord{}\boxAlignEqnarray{\leftCoord{}
\tilde{\Phi}(\theta_i-\theta'_j) \equiv
\frac{\leftCoord{}\sinh(\theta_i-\theta'_j)+i\sin \pi \nu}{\rightCoord{}2i}\; . \nonumber\rightCoord{}
\rightCoord{}}{0mm}{2}{4}{
\tilde{\Phi}(\theta_i-\theta'_j) \equiv
\frac{\sinh(\theta_i-\theta'_j)+i\sin \pi \nu}{2i}\; . }{1}\coordE{}\end{eqnarray}
We assign odd \myHighlight{$Z_2$}\coordHE{}-parity to both right and left-movers (\myHighlight{$\psi_R
\to -\psi_R, \ \ \psi_L \to -\psi_L, \ \ \phi \to \phi)$}\coordHE{}  and even
(odd) parity to right (left) movers under duality transformations
(\myHighlight{$\psi_R \to \psi_R, \ \ \psi_L \to -\psi_L, \ \ \phi \to -\phi).$}\coordHE{}
 In the
Neveu-Schwartz sector (which consists of local operators), the operators \myHighlight{$\cosh \alpha \phi$}\coordHE{} and \myHighlight{$\psi
\bar{\psi}\sinh \alpha \phi$}\coordHE{} have non zero matrix elements on
(even,even) number of particles, whereas \myHighlight{$\sinh \alpha \phi$}\coordHE{} and
\myHighlight{$\psi \bar{\psi}\cosh \alpha \phi$}\coordHE{} on (odd,odd) number of
particles. Let us remind \cite{Ro} that the Volkov-Akulov
formalism is expressed in terms of a constraint superfield
\myHighlight{$\Phi(z,\bar{z})=\phi(z,\bar{z})+\theta\psi(z)+\bar{\theta}\bar{\psi}(\bar{z})
+\theta\bar{\theta}F(z,\bar{z})$}\coordHE{}, satisfying \myHighlight{$\Phi^2=0$}\coordHE{}, and for
which a solution is \myHighlight{$\phi =(1/F) \bar{\psi}{\psi}$}\coordHE{}. Hence, from
the fusion rules of the critical Ising model \cite{BPZ} we deduce
that the first set of operators renormalizes on the family of the
identity and the second set on the family of the energy \myHighlight{$\epsilon$}\coordHE{}. As
\myHighlight{$\psi\bar{\psi}e^{\alpha \phi}$}\coordHE{} is the descendant field by
supersymmetry of the primary \myHighlight{$e^{\alpha \phi}$}\coordHE{}, its form factors
differ from those of \myHighlight{$e^{\alpha \phi}$}\coordHE{} by a multiplicative factor
that does not affect the bootstrap equations and that has correct
behaviour under Lorentz transformations:
\myHighlight{$F^{\psi\bar{\psi}e^{\alpha\phi}}=\left(\sum
e^{\theta_i}\right)^{1/2}\left(\sum e^{-\theta'_j}\right)^{1/2}
F^{e^{\alpha\phi}}$}\coordHE{}. In the Ramond sector, the non local operators \myHighlight{$\sigma
e^{\alpha \phi}$}\coordHE{} and \myHighlight{$\mu e^{\alpha \phi}$}\coordHE{} renormalize on the
family of the spin field and disorder field respectively.




\subsection{Neveu-Schwarz sector: operators \myHighlight{$e^{\alpha\phi}$}\coordHE{}}
\subsubsection{Form factors of the operator \myHighlight{$\cosh\alpha\phi$}\coordHE{}.}
Because of symmetry under spin and parity, \myHighlight{$\cosh\alpha\phi$}\coordHE{} has
non vanishing matrix elements for an even number of left and right
movers. We introduce the sets \myHighlight{$S=(1,\dots,2r), S'=(1,\dots,2l)$}\coordHE{}
and the first form factor\footnote{As it is mentionned above, the
bootstrap equations do not refer to any particular operator, so
the dependence with respect to the parameter \myHighlight{$\alpha$}\coordHE{} is
introduced by hand. We do not have so far any compelling argument
to justify our choice for the \myHighlight{$\alpha$}\coordHE{} dependence of the lowest
form factor, neither in the Neveu-Schwartz sector nor in the
Ramond sector considered in the next section. They are simply
similar to some other known cases (see {\it e.g.} \cite{KM}).}
\myHighlight{$Q_{2,2}^{\alpha}=\left(\frac{\sin\frac{\pi\nu\alpha}{b}}{\sin\pi\nu}\right)^{2}$}\coordHE{}.
We propose
\begin{eqnarray}\coord{}\boxAlignEqnarray{\leftCoord{}
\lefteqn{Q_{2r,2l}^{\alpha}(\theta_1,\theta_2,\ldots
\theta_{2r};\theta'_1,\theta'_2,\ldots,\theta'_{2l})=
\left(\frac{\leftCoord{}\sin\frac{\leftCoord{}\pi\nu\alpha}{\rightCoord{}b}}{\rightCoord{}\sin\pi\nu}\right)^{r+l}}\nonumber \rightCoord{}\\
&&\leftCoord{} \sum_{\rightCoord{}T \in S, \atop \#T=r-1}\sum_{\rightCoord{}T' \in S',\atop
\leftCoord{}\#T'=l-1}\prod_{\rightCoord{}i \in T, \atop k\in
\bar{T}}\phi(\theta_{ik})\prod_{\rightCoord{}i \in T', \atop  k\in
\bar{T}'}\phi(\theta'_{ik}) \prod_{\rightCoord{}i \in T, \atop k\in
\bar{T}'}\Phi(\theta_i-\theta'_k) \prod_{\rightCoord{}i \in T', \atop k\in
\bar{T}}\tilde{\Phi}(\theta_k-\theta'_i)\nonumber\rightCoord{}
\rightCoord{}}{0mm}{5}{12}{
\lefteqn{Q_{2r,2l}^{\alpha}(\theta_1,\theta_2,\ldots
\theta_{2r};\theta'_1,\theta'_2,\ldots,\theta'_{2l})=
\left(\frac{\sin\frac{\pi\nu\alpha}{b}}{\sin\pi\nu}\right)^{r+l}}\\
&& \sum_{T \in S, \atop \#T=r-1}\sum_{T' \in S',\atop
\#T'=l-1}\prod_{i \in T, \atop k\in
\bar{T}}\phi(\theta_{ik})\prod_{i \in T', \atop  k\in
\bar{T}'}\phi(\theta'_{ik}) \prod_{i \in T, \atop k\in
\bar{T}'}\Phi(\theta_i-\theta'_k) \prod_{i \in T', \atop k\in
\bar{T}}\tilde{\Phi}(\theta_k-\theta'_i)}{1}\coordE{}\end{eqnarray}
Such an expression satisfies the residue condition
(\ref{residu2}).



\subsubsection{Form factors of the trace of the stress energy
tensor.} The first form factor is determined by using the
Lagrangian: \myHighlight{$Q_{2,2}=-4\pi M^2$}\coordHE{}.
\begin{eqnarray}\coord{}\boxAlignEqnarray{\leftCoord{}
\lefteqn{Q_{2r,2l}(\theta_1,\theta_2,\ldots
\theta_{2r};\theta'_1,\theta'_2,\ldots,\theta'_{2l})=}\nonumber \rightCoord{}\\
&&\leftCoord{} -4\pi M^2\sum_{\rightCoord{}T \in S, \atop \#T=r-1}\sum_{\rightCoord{}T' \in S',\atop
\leftCoord{}\#T'=l-1}\prod_{\rightCoord{}i \in T, \atop k\in
\bar{T}}\phi(\theta_{ik})\prod_{\rightCoord{}i \in T', \atop  k\in
\bar{T}'}\phi(\theta'_{ik}) \prod_{\rightCoord{}i \in T, \atop k\in
\bar{T}'}\Phi(\theta_i-\theta'_k) \prod_{\rightCoord{}i \in T', \atop k\in
\bar{T}}\tilde{\Phi}(\theta_k-\theta'_i)\nonumber\rightCoord{}
\rightCoord{}}{0mm}{3}{10}{
\lefteqn{Q_{2r,2l}(\theta_1,\theta_2,\ldots
\theta_{2r};\theta'_1,\theta'_2,\ldots,\theta'_{2l})=}\\
&& -4\pi M^2\sum_{T \in S, \atop \#T=r-1}\sum_{T' \in S',\atop
\#T'=l-1}\prod_{i \in T, \atop k\in
\bar{T}}\phi(\theta_{ik})\prod_{i \in T', \atop  k\in
\bar{T}'}\phi(\theta'_{ik}) \prod_{i \in T, \atop k\in
\bar{T}'}\Phi(\theta_i-\theta'_k) \prod_{i \in T', \atop k\in
\bar{T}}\tilde{\Phi}(\theta_k-\theta'_i)}{1}\coordE{}\end{eqnarray}
Let us note that the leading infrared behavior of \myHighlight{$F_{2,2}$}\coordHE{} is given by
\myHighlight{$T\bar{T}$}\coordHE{}, which defines the direction of the flow in
the IR region. To determine the subleading IR terms that appear
in the expansion (\ref{effectif}),
one uses the asymptotic developpement for \myHighlight{$f_{RL}$}\coordHE{} given by
equation (\ref{IR}). For example (up to the logarithmic
terms):
\begin{eqnarray}\coord{}\boxAlignEqnarray{
&&\leftCoord{}\lefteqn{f_{RL}(\theta_1-\theta'_1)f_{RL}(\theta_1-\theta'_2)f_{RL}(\theta_2-\theta'_1)f_{RL}(\theta_2-\theta'_2)
\sim e^{\theta_1+\theta_2-\theta'_1-\theta'_2}\times} \nonumber \rightCoord{}\\
&&\leftCoord{}\left[1+Ae^{\theta_1-\theta'_1}+\left(\frac{\leftCoord{}A^2}{\rightCoord{}2}+B\right)e^{2\theta_1-2\theta'_1}\right]\times
\left[1+Ae^{\theta_1-\theta'_2}+\left(\frac{\leftCoord{}A^2}{\rightCoord{}2}+B\right)e^{2\theta_1-2\theta'_2}\right]\times
\nonumber \rightCoord{}\\
&&\leftCoord{} \rightCoord{}
\left[1+Ae^{\theta_2-\theta'_1}+\left(\frac{\leftCoord{}A^2}{\rightCoord{}2}+B\right)e^{2\theta_2-2\theta'_1}\right]\times
\left[1+Ae^{\theta_2-\theta'_2}+\left(\frac{\leftCoord{}A^2}{\rightCoord{}2}+B\right)e^{2\theta_2-2\theta'_2}\right]\;
\leftCoord{}. \nonumber\rightCoord{}
\rightCoord{}}{0mm}{8}{10}{
&&\lefteqn{f_{RL}(\theta_1-\theta'_1)f_{RL}(\theta_1-\theta'_2)f_{RL}(\theta_2-\theta'_1)f_{RL}(\theta_2-\theta'_2)
\sim e^{\theta_1+\theta_2-\theta'_1-\theta'_2}\times} \\
&&\left[1+Ae^{\theta_1-\theta'_1}+\left(\frac{A^2}{2}+B\right)e^{2\theta_1-2\theta'_1}\right]\times
\left[1+Ae^{\theta_1-\theta'_2}+\left(\frac{A^2}{2}+B\right)e^{2\theta_1-2\theta'_2}\right]\times
\\
&& 
\left[1+Ae^{\theta_2-\theta'_1}+\left(\frac{A^2}{2}+B\right)e^{2\theta_2-2\theta'_1}\right]\times
\left[1+Ae^{\theta_2-\theta'_2}+\left(\frac{A^2}{2}+B\right)e^{2\theta_2-2\theta'_2}\right]\;
. }{1}\coordE{}\end{eqnarray}
The terms into brackets give
\begin{eqnarray}\coord{}\boxAlignEqnarray{
&&\leftCoord{}1+A(e^{\theta_1}+e^{\theta_2})(e^{-\theta'_1}+e^{-\theta'_2})+
\left(\frac{\leftCoord{}A^2}{\rightCoord{}2}+B\right)(e^{2\theta_1-2\theta'_1}+e^{2\theta_1-2\theta'_2}+e^{2\theta_2-2\theta'_1}
\leftCoord{}+e^{2\theta_2-2\theta'_2})
\nonumber\rightCoord{}\\
&&\leftCoord{}+A^2(e^{2\theta_1-\theta'_1-\theta'_2}+
e^{\theta_1+\theta_2-2\theta'_1}+e^{\theta_1+\theta_2-2\theta'_2}+e^{2\theta_2-\theta'_1-\theta'_2}+
\leftCoord{}2e^{\theta_1+\theta_2-\theta'_1-\theta'_2})
\leftCoord{}+\cdots\nonumber\rightCoord{}\\
&&\leftCoord{}= \rightCoord{}
\leftCoord{}1+\frac{\leftCoord{}A}{\rightCoord{}M^2}L_{-1}\bar{L}_{-1}+\frac{\leftCoord{}A^2}{\rightCoord{}2M^4}L^2_{-1}\bar{L}^2_{-1}+
\frac{\leftCoord{}B}{\rightCoord{}M^4}L_{-2}\bar{L}_{-2}+\cdots\nonumber\rightCoord{}
\rightCoord{}}{0mm}{11}{10}{
&&1+A(e^{\theta_1}+e^{\theta_2})(e^{-\theta'_1}+e^{-\theta'_2})+
\left(\frac{A^2}{2}+B\right)(e^{2\theta_1-2\theta'_1}+e^{2\theta_1-2\theta'_2}+e^{2\theta_2-2\theta'_1}
+e^{2\theta_2-2\theta'_2})
\\
&&+A^2(e^{2\theta_1-\theta'_1-\theta'_2}+
e^{\theta_1+\theta_2-2\theta'_1}+e^{\theta_1+\theta_2-2\theta'_2}+e^{2\theta_2-\theta'_1-\theta'_2}+
2e^{\theta_1+\theta_2-\theta'_1-\theta'_2})
+\cdots\\
&&= 
1+\frac{A}{M^2}L_{-1}\bar{L}_{-1}+\frac{A^2}{2M^4}L^2_{-1}\bar{L}^2_{-1}+
\frac{B}{M^4}L_{-2}\bar{L}_{-2}+\cdots}{1}\coordE{}\end{eqnarray}
where \myHighlight{$L_{-1}=e^{\theta_1}+e^{\theta_2}$}\coordHE{} and
\myHighlight{$L_{-2}=e^{2\theta_1}+e^{2\theta_2}$}\coordHE{}. So the next irrelevant
operator appearing is \myHighlight{$T^2\bar{T}^2$}\coordHE{} (up to derivatives).\\






\subsubsection{Form factors of the operator \myHighlight{$\sinh\alpha\phi$}\coordHE{}.}
The number of left movers and right movers is odd. The form
factors are written in this case Let
\myHighlight{$S=(1,\dots,2r+1),S'=(1,\dots,2l+1)$}\coordHE{}. The lowest form factor is
\myHighlight{$Q_{1,1}^{\alpha}=\left(\frac{\sin\frac{\pi\nu}{b}\alpha}{\sin\pi\nu}\right)$}\coordHE{}.
We propose
\begin{eqnarray}\coord{}\boxAlignEqnarray{\leftCoord{}
\lefteqn{Q_{2r+1,2l+1}^{\alpha}(\theta_1,\theta_2,\ldots
\theta_{2r+1};\theta'_1,\theta'_2,\ldots,\theta'_{2l+1})=
\left(\frac{\leftCoord{}\sin\frac{\leftCoord{}\pi\nu}{\rightCoord{}b}\alpha}{\rightCoord{}\sin\pi\nu}\right)^{r+l+1}} \rightCoord{}
\nonumber \rightCoord{}\\
&&\leftCoord{} \sum_{\rightCoord{}T \in S, \atop \#T=r}\sum_{\rightCoord{}T' \in S',\atop
\leftCoord{}\#T'=l}\prod_{\rightCoord{}i \in T, \atop k\in
\bar{T}}\phi(\theta_{ik})\prod_{\rightCoord{}i \in T', \atop k\in
\bar{T}'}\phi(\theta'_{ik}) \prod_{\rightCoord{}i \in T, \atop k\in
\bar{T}'}\Phi(\theta_i-\theta'_k) \prod_{\rightCoord{}i \in T', \atop k\in
\bar{T}}\tilde{\Phi}(\theta_k-\theta'_i)\nonumber\rightCoord{}.\rightCoord{}
\rightCoord{}}{0mm}{5}{14}{
\lefteqn{Q_{2r+1,2l+1}^{\alpha}(\theta_1,\theta_2,\ldots
\theta_{2r+1};\theta'_1,\theta'_2,\ldots,\theta'_{2l+1})=
\left(\frac{\sin\frac{\pi\nu}{b}\alpha}{\sin\pi\nu}\right)^{r+l+1}} 
\\
&& \sum_{T \in S, \atop \#T=r}\sum_{T' \in S',\atop
\#T'=l}\prod_{i \in T, \atop k\in
\bar{T}}\phi(\theta_{ik})\prod_{i \in T', \atop k\in
\bar{T}'}\phi(\theta'_{ik}) \prod_{i \in T, \atop k\in
\bar{T}'}\Phi(\theta_i-\theta'_k) \prod_{i \in T', \atop k\in
\bar{T}}\tilde{\Phi}(\theta_k-\theta'_i).
}{1}\coordE{}\end{eqnarray}
We suppose that the form factors of the energy operator \myHighlight{$\epsilon$}\coordHE{}
are given by the expression
\begin{eqnarray}\coord{}\boxAlignEqnarray{\leftCoord{}
\lefteqn{Q_{2r+1,2l+1}(\theta_1,\theta_2,\ldots
\theta_{2r+1};\theta'_1,\theta'_2,\ldots,\theta'_{2l+1})=}
\nonumber \rightCoord{}\\
&&\leftCoord{} \sum_{\rightCoord{}T \in S, \atop \#T=r}\sum_{\rightCoord{}T' \in S',\atop
\leftCoord{}\#T'=l}\prod_{\rightCoord{}i \in T, \atop k\in
\bar{T}}\phi(\theta_{ik})\prod_{\rightCoord{}i \in T', \atop k\in
\bar{T}'}\phi(\theta'_{ik}) \prod_{\rightCoord{}i \in T, \atop k\in
\bar{T}'}\Phi(\theta_i-\theta'_k) \prod_{\rightCoord{}i \in T', \atop k\in
\bar{T}}\tilde{\Phi}(\theta_k-\theta'_i)\nonumber\rightCoord{}.\rightCoord{}
\rightCoord{}}{0mm}{3}{11}{
\lefteqn{Q_{2r+1,2l+1}(\theta_1,\theta_2,\ldots
\theta_{2r+1};\theta'_1,\theta'_2,\ldots,\theta'_{2l+1})=}
\\
&& \sum_{T \in S, \atop \#T=r}\sum_{T' \in S',\atop
\#T'=l}\prod_{i \in T, \atop k\in
\bar{T}}\phi(\theta_{ik})\prod_{i \in T', \atop k\in
\bar{T}'}\phi(\theta'_{ik}) \prod_{i \in T, \atop k\in
\bar{T}'}\Phi(\theta_i-\theta'_k) \prod_{i \in T', \atop k\in
\bar{T}}\tilde{\Phi}(\theta_k-\theta'_i).
}{1}\coordE{}\end{eqnarray}


\subsubsection{Remark}
One can check that in the limit
\myHighlight{$\theta_1,\theta'_1 \to \pm \infty$}\coordHE{}, the cluster property is
satisfied\footnote{This type of argument has also been used before, see 
{\it e.g.} \cite{KW,KM}.}:
\myHighlight{$F^{\alpha}_{r,l}(\theta_1,\dots,\theta_{r},\theta'_1,\dots\theta'_{l})
\sim F^{\alpha}_{1,1}(\theta_1,\theta'_1)
F^{\alpha}_{r-1,l-1}(\theta_2,\dots,\theta_{r},\theta'_2,\dots\theta'_{l})$}\coordHE{}.





\subsection{Ramond sector}
\subsubsection{Operator \myHighlight{$\sigma e^{\alpha\phi}$}\coordHE{}.}
It has non vanishing matrix elements when the sum of left movers
and right movers is odd. Let \myHighlight{$S=(1,\dots,2r+1),S'=(1,\dots,2l)$}\coordHE{}.
The lowest form factors are \myHighlight{$Q_{1,0}^{\alpha}=Q_{0,1}^{\alpha}=
\left(\frac{\sin \frac{\pi \nu(\alpha+b)}{2b}\sin \frac{\pi
\nu(-\alpha+b)}{2b}}{\sin^2\frac{\pi\nu}{2}}\right)^{1/2}$}\coordHE{}. We
propose:
\begin{eqnarray}\coord{}\boxAlignEqnarray{\leftCoord{}
\lefteqn{Q_{2r+1,2l}^{\alpha}(\theta_1,\theta_2,\ldots
\theta_{2r+1};\theta'_1,\theta'_2,\ldots,\theta'_{2l})=}\nonumber \rightCoord{}\\
&&\leftCoord{} \left( \frac{\leftCoord{}\sin \frac{\leftCoord{}\pi \nu(\alpha+b)}{\rightCoord{}2b}\sin \frac{\leftCoord{}\pi
\nu(-\alpha+b)}{2b}}{\sin^2\frac{\leftCoord{}\pi\nu}{\rightCoord{}2}} \rightCoord{}
\right)^{r+l+1/2}\sum_{\rightCoord{}T \in S, \atop \#T=r}\sum_{\rightCoord{}T' \in S', \atop
\leftCoord{}\#T'=l}\prod_{\rightCoord{}i \in T, \atop k\in
\bar{T}}\phi(\theta_{ik})\prod_{\rightCoord{}i \in T', \atop k\in
\bar{T}'}\phi(\theta'_{ik}) \prod_{\rightCoord{}i \in T, \atop k\in
\bar{T}'}\Phi(\theta_i-\theta'_k) \prod_{\rightCoord{}i \in T', \atop k\in
\bar{T}}\tilde{\Phi}(\theta_k-\theta'_i)\nonumber\rightCoord{}
\rightCoord{}}{0mm}{7}{13}{
\lefteqn{Q_{2r+1,2l}^{\alpha}(\theta_1,\theta_2,\ldots
\theta_{2r+1};\theta'_1,\theta'_2,\ldots,\theta'_{2l})=}\\
&& \left( \frac{\sin \frac{\pi \nu(\alpha+b)}{2b}\sin \frac{\pi
\nu(-\alpha+b)}{2b}}{\sin^2\frac{\pi\nu}{2}} 
\right)^{r+l+1/2}\sum_{T \in S, \atop \#T=r}\sum_{T' \in S', \atop
\#T'=l}\prod_{i \in T, \atop k\in
\bar{T}}\phi(\theta_{ik})\prod_{i \in T', \atop k\in
\bar{T}'}\phi(\theta'_{ik}) \prod_{i \in T, \atop k\in
\bar{T}'}\Phi(\theta_i-\theta'_k) \prod_{i \in T', \atop k\in
\bar{T}}\tilde{\Phi}(\theta_k-\theta'_i)}{1}\coordE{}\end{eqnarray}





\subsubsection{Operator \myHighlight{$\mu e^{\alpha\phi}$}\coordHE{} (\myHighlight{$r+l$}\coordHE{} even).}
As it is explained in \cite{YZ}, there is an additional minus sign
in front of the product of \myHighlight{$S$}\coordHE{} matrices in the bootstrap equation
(\ref{residue}).
\begin{itemize}
\item \myHighlight{$r,l$}\coordHE{} even\\
Let \myHighlight{$S=(1,\dots,2r), S'=(1,\dots,2l)$}\coordHE{} and the lowest form factor
\myHighlight{$Q_{0,0}^{\alpha}=1$}\coordHE{}. We propose :
\begin{eqnarray}\coord{}\boxAlignEqnarray{\leftCoord{}
\lefteqn{Q_{2r,2l}^{\alpha}(\theta_1,\theta_2,\ldots
\theta_{2r};\theta'_1,\theta'_2,\ldots,\theta'_{2l})=(-i)^{r+l}\left(
\frac{\leftCoord{}\sin \frac{\leftCoord{}\pi \nu(\alpha+b)}{\rightCoord{}2b}\sin \frac{\leftCoord{}\pi
\nu(-\alpha+b)}{2b}}{\sin^2\frac{\leftCoord{}\pi\nu}{\rightCoord{}2}} \right)^{r+l}}\nonumber \rightCoord{}\\
&&\leftCoord{} \sum_{\rightCoord{}T \in S, \atop \#T=r}\sum_{\rightCoord{}T' \in S', \atop
\leftCoord{}\#T'=l}\prod_{\rightCoord{}i \in T, \atop k\in
\bar{T}}\phi(\theta_{ik})e^{\frac{\leftCoord{}1}{\rightCoord{}2}\sum\theta_{ik}}\prod_{\rightCoord{}i
\in T',\atop  k\in \bar{T}'}\phi(\theta'_{ik})
e^{\frac{\leftCoord{}1}{\rightCoord{}2}\sum\theta'_{ki}} \prod_{\rightCoord{}i \in T, \atop k\in
\bar{T}'}\Phi(\theta_i-\theta'_k) \prod_{\rightCoord{}i \in T', \atop k\in
\bar{T}}\tilde{\Phi}(\theta_k-\theta'_i)\nonumber\rightCoord{}
\rightCoord{}}{0mm}{9}{14}{
\lefteqn{Q_{2r,2l}^{\alpha}(\theta_1,\theta_2,\ldots
\theta_{2r};\theta'_1,\theta'_2,\ldots,\theta'_{2l})=(-i)^{r+l}\left(
\frac{\sin \frac{\pi \nu(\alpha+b)}{2b}\sin \frac{\pi
\nu(-\alpha+b)}{2b}}{\sin^2\frac{\pi\nu}{2}} \right)^{r+l}}\\
&& \sum_{T \in S, \atop \#T=r}\sum_{T' \in S', \atop
\#T'=l}\prod_{i \in T, \atop k\in
\bar{T}}\phi(\theta_{ik})e^{\frac{1}{2}\sum\theta_{ik}}\prod_{i
\in T',\atop  k\in \bar{T}'}\phi(\theta'_{ik})
e^{\frac{1}{2}\sum\theta'_{ki}} \prod_{i \in T, \atop k\in
\bar{T}'}\Phi(\theta_i-\theta'_k) \prod_{i \in T', \atop k\in
\bar{T}}\tilde{\Phi}(\theta_k-\theta'_i)}{1}\coordE{}\end{eqnarray}


\item{\myHighlight{$r,l$}\coordHE{} odd}\\
Let \myHighlight{$S=(1,\dots,2r+1),S'=(1,\dots,2l+1)$}\coordHE{}. The lowest form factor
is\\ \myHighlight{$Q_{1,1}^{\alpha}=\left( \frac{\sin \frac{\pi
\nu(\alpha+b)}{2b}\sin \frac{\pi
\nu(-\alpha+b)}{2b}}{\sin^2\frac{\pi\nu}{2}}\right)
e^{\frac{\theta'_1-\theta_1}{2}}$}\coordHE{}. We propose:
\begin{eqnarray}\coord{}\boxAlignEqnarray{\leftCoord{}
\lefteqn{Q_{2r+1,2l+1}^{\alpha}(\theta_1,\theta_2,\ldots
\theta_{2r+1};\theta'_1,\theta'_2,\ldots,\theta'_{2l+1})=
\leftCoord{}(-i)^{r+l}\left( \frac{\leftCoord{}\sin \frac{\leftCoord{}\pi \nu(\alpha+b)}{\rightCoord{}2b}\sin
\frac{\leftCoord{}\pi
\nu(-\alpha+b)}{2b}}{\sin^2\frac{\leftCoord{}\pi\nu}{\rightCoord{}2}}\right)^{r+l+1}}\nonumber \rightCoord{}\\
&&\leftCoord{} \sum_{\rightCoord{}T \in S, \atop \#T=r}\sum_{\rightCoord{}T' \in S', \atop
\leftCoord{}\#T'=l}\prod_{\rightCoord{}i \in T, \atop k\in
\bar{T}}\phi(\theta_{ik})e^{\frac{\leftCoord{}1}{\rightCoord{}2}\sum\theta_{ik}}\prod_{\rightCoord{}i
\in T',\atop  k\in \bar{T}'}\phi(\theta'_{ik})
e^{\frac{\leftCoord{}1}{\rightCoord{}2}\sum\theta'_{ki}}\prod_{\rightCoord{}i \in T, \atop k\in
\bar{T}'}\Phi(\theta_i-\theta'_k) \prod_{\rightCoord{}i \in T', \atop k\in
\bar{T}}\tilde{\Phi}(\theta_k-\theta'_i) \nonumber\rightCoord{}
\rightCoord{}}{0mm}{10}{14}{
\lefteqn{Q_{2r+1,2l+1}^{\alpha}(\theta_1,\theta_2,\ldots
\theta_{2r+1};\theta'_1,\theta'_2,\ldots,\theta'_{2l+1})=
(-i)^{r+l}\left( \frac{\sin \frac{\pi \nu(\alpha+b)}{2b}\sin
\frac{\pi
\nu(-\alpha+b)}{2b}}{\sin^2\frac{\pi\nu}{2}}\right)^{r+l+1}}\\
&& \sum_{T \in S, \atop \#T=r}\sum_{T' \in S', \atop
\#T'=l}\prod_{i \in T, \atop k\in
\bar{T}}\phi(\theta_{ik})e^{\frac{1}{2}\sum\theta_{ik}}\prod_{i
\in T',\atop  k\in \bar{T}'}\phi(\theta'_{ik})
e^{\frac{1}{2}\sum\theta'_{ki}}\prod_{i \in T, \atop k\in
\bar{T}'}\Phi(\theta_i-\theta'_k) \prod_{i \in T', \atop k\in
\bar{T}}\tilde{\Phi}(\theta_k-\theta'_i) }{1}\coordE{}\end{eqnarray}
\end{itemize}
\subsubsection{Remarks}
\begin{itemize}
\item
Let us introduce now the operator \myHighlight{$\mathcal{O}=\sigma
e^{\alpha\phi} +\mu e^{\alpha \phi}$}\coordHE{}. Its form factors satisfy the
cluster
 property like an exponential of a bose field
\begin{eqnarray}\coord{}\boxAlignEqnarray{\leftCoord{}
\mathcal{O}_{r,l}(\theta_1,\theta_2,\ldots
\theta_{r};\theta'_1,\theta'_2,\ldots,\theta'_{l}) \sim
\mathcal{O}_{1,0}(\theta_1)\mathcal{O}_{r-1,l}(\theta_2,\ldots
\theta_{r};\theta'_1,\theta'_2,\ldots,\theta'_{l})\quad
\mathrm{for} \quad \theta_1 \to \infty\; . \nonumber\rightCoord{}
\rightCoord{}}{0mm}{1}{3}{
\mathcal{O}_{r,l}(\theta_1,\theta_2,\ldots
\theta_{r};\theta'_1,\theta'_2,\ldots,\theta'_{l}) \sim
\mathcal{O}_{1,0}(\theta_1)\mathcal{O}_{r-1,l}(\theta_2,\ldots
\theta_{r};\theta'_1,\theta'_2,\ldots,\theta'_{l})\quad
\mathrm{for} \quad \theta_1 \to \infty\; . }{1}\coordE{}\end{eqnarray}
\item
The expressions for the form factors of \myHighlight{$\sigma e^{\alpha\phi}$}\coordHE{}
and \myHighlight{$\mu e^{\alpha\phi}$}\coordHE{} give the expected leading IR behaviour
\cite{BKW,YZ,CM}: \myHighlight{$
F_{r,l}^{IR}(\theta_1,\theta_2,\ldots,\theta_r;\theta'_1,\theta'_2,\ldots,\theta'_l)
\sim
\prod_{i<j}\tanh\frac{\theta_{ij}}{2}\tanh\frac{\theta'_{ij}}{2}$}\coordHE{},
where \myHighlight{$r+l$}\coordHE{} is odd for \myHighlight{$\sigma e^{\alpha\phi}$}\coordHE{} and even for \myHighlight{$\mu
e^{\alpha\phi}$}\coordHE{}.
\end{itemize}


\section{Conclusion}
Finally, we would like to say that the representations proposed
for the functions \myHighlight{$Q_{r,l}^{\alpha}$}\coordHE{} are general enough
\footnote{Their structure is very similar to the one found for the
operator \myHighlight{$e^{\alpha\phi}$}\coordHE{} in the bosonic Sinh-Gordon model in
\cite{BK}, equ. (61).} to obtain immediately the form factors of the
operators \myHighlight{$\Theta, \epsilon,\sigma,\mu$}\coordHE{} in the Tricritical Ising
model perturbed by the subenergy that defines a massless flow
 to the Ising model \cite{KMS,AZ}, simply by replacing \myHighlight{$S_{RL}$}\coordHE{}
 and \myHighlight{$f_{RL}$}\coordHE{} by their
corresponding values \cite{Z}. We checked for a low number of
particles that they indeed reproduce the results of \cite{DMS}
where the first form factors of \myHighlight{$\Theta,\sigma,\mu$}\coordHE{}
 are computed in terms of symmetric polynomials,
as well as with \cite{MS}, where an expression for the form
factors of the operator \myHighlight{$\Theta$}\coordHE{} for an arbitrary number of
intermediate particles is proposed. In principle, it should not be
difficult to extend our results to other massless models flowing
to the Ising model,
 but where the \myHighlight{$S$}\coordHE{}-matrix has a more complicated structure of resonance poles \cite{Z}.
Of course it remains to prove if our choice for the \myHighlight{$\alpha$}\coordHE{} dependence
of the form factors reproduces the correct UV conformal dimension of the operators.

\section*{Acknowledgments}
I am grateful to Al.B.~Zamolodchikov for suggesting the problem.
Discussions with D.~Bernard, V.A.~Fateev, G.~Mussardo, H.~Saleur, F.A.~Smirnov are
acknowledged. In particular, I thank G.~Delfino for his
comments on the manuscript and useful discussions. Work supported
by the CEA and the Euclid Network HPRN-CT-2002-00325.


\begin{thebibliography}{99}
\bibitem{AKRZ} C.~Ahn, C.~Kim, C.~Rim and Al.B.~Zamolodchikov, "RG
flows from Super-Liouville Theory to Critical Ising Model",
Phys.~Lett.~\textbf{B541} (2002) 194, hep-th/0206210





\bibitem{VA} D.V.~Volkov and V.P.~Akulov, "Is the neutrino a
Goldstone particle?", Phys.~Lett.~\textbf{B46} (1973) 109




\bibitem{KW} M.~Karowski and P.~Weisz, Nucl.~Phys.~\textbf{B139}
(1978) 455

\bibitem{BKW} B.~Berg, M.~Karowski and P.~Weisz,
Phys.~Rev.~\textbf{D19} (1979) 2477

\bibitem{S} F.A.~Smirnov, "Form factors in Completely Integrable
Models of Quantum Field Theory", Adv.~Series in Math.~Phys.
\textbf{14}, World Scientific 1992


\bibitem{DMS} G.~Delfino, G.~Mussardo and P.~Simonetti,
"Correlation fonctions along a massless flow",
Phys.~Rev.~\textbf{D51} (1995) 6620, hep-th/9410117



\bibitem{MS} P.~M\'ejean and F.A.~Smirnov, "Form-factors for principal chiral field model
with Wess-Zuminov-Novikov-Witten term", Int.~J.~Mod.~Phys.
\textbf{A12} (1997) 3383, hep-th/9609068



\bibitem{Ro} M.~Rocek, "Linearizing the Volkov-Akulov Model", Phys.~Rev.~Lett.~\textbf{41} (1978) 451

\bibitem{BPZ} A.A.~Belavin, A.M.~Polyakov and A.B.~Zamolodchikov, "Infinite conformal symmetry in 2D
 quantum field theory", Nucl.~Phys.~\textbf{B241} (1984) 333


\bibitem{KM} A.~Koubek and G.~Mussardo, ``On the Operator Content of the Sinh-Gordon Model'', Phys.~Lett.~\textbf{B311} (1993) 193, hep-th/9306044


\bibitem{YZ} V.P.~Yurov and Al.B.~Zamolodchikov, "Correlation functions of integrable 2D models of the
 relativistic field theory; Ising model", Int. J. Mod. Phys.
 \textbf{A6} (1991) 3419






\bibitem{CM} J.L.~Cardy and G.~Mussardo, Nucl.~Phys.~\textbf{B340} (1990) 387


\bibitem{BK} H.~Babujian and M.~Karowski, "Sine-Gordon breather
form factors and quantum field equations", J.~Phys.~\textbf{A35} (2002) 9081, hep-th/0204097

\bibitem{KMS} D.A.~Kastor, E.J.~Martinec and S.H.~Shenker, "RG
flow in \myHighlight{$N=1$}\coordHE{} Discrete Series", Nucl.~Phys.~\textbf{B316} (1989)
590

\bibitem{AZ} A.B.~ Zamolodchikov, Sov.~J.~Nucl.~Phys. \textbf{48} (1987), 1090



\bibitem{Z} Al.B.~Zamolodchikov, "From tricritical Ising to
 critical Ising by thermodynamic Bethe ansatz", Nucl.~Phys.~\textbf{B358} (1991) 524








\end{thebibliography}
\end{document}
\bye
