\documentclass[a4paper,a4paper]{article}

\usepackage{amsmath,amssymb,epsf,epsfig,a4wide}



%\usepackage{showkeys}

\begin{document}


\begin{centering}

\vspace{0in}

{\Large {\bf Israel conditions for the Gauss-Bonnet theory and the Friedmann equation on 
the brane universe}}



\vspace{0.4in}



{ \bf Elias Gravanis, \; Steven Willison} \\
\vspace{0.2in}
Department of Physics, King's College London,\\
Strand, London WC2R 2LS, United Kingdom.



\vspace{0.2in}
 {\bf Abstract}

\end{centering}

\vspace{0.2in}

{\small Assuming an Einstein-Gauss-Bonnet theory of gravitation in a ($D \geq 5$)-dimensional spacetime
with boundary, we consider the problem of the boundary dynamics given the matter Lagrangian on it. The
resulting equation is applied in particular on the derivation of the Friedmann equation of a 3-brane, 
understood as the non-orientable boundary of a 5d spacetime. We briefly discuss the contradictory 
conclusions of the literature.}


\vspace{0.3in}





In the framework of Einstein gravitational theory, the cosmology of the brane-world model
has been first studied in $\cite{Binetruy:1999ut}$. There the domain-wall/brane 
 is described as a 4d world-
volume slice of a 5d spacetime, which is of constant spatial curvature.  In \cite{Mukohyama:1999wi}
it was shown that these solutions and the ones in $\cite{Kraus:1999it}$ 
 describe the same physical system which consists in a 3-surface 
of Robertson-Walker world-volume moving in the background of a 5d AdS-Schwarzchild black hole
\footnote{More precisely, the construction is the following:
 the 3-surface is the boundary of ``half" such a space. Its location
with respect to the black hole is determined by the boundary dynamics.
One introduces a mirror image half, with respect to the
boundary, imposing also a ${\bf Z_2}$ symmetry to avoid overcounting. The 3-surface
becomes the non-orientable boundary of the new space. The evolution/motion of
the brane just
 recovers parts of the bulk spacetime which have
been cut off. This is because they are both of constant 3-curvature.}. Inverting 
this course one obtains a procedure 
for constructing brane cosmological solutions
that can be applied to
other gravitational theories.  Different
matter content on the 3-brane corresponds to a different 4d trajectory in the bulk black hole spacetime.
The dynamics of the 4d trajectory is given by the appropriate Friedmann equation, which
is obtained either by matching the bulk solutions around the brane or by viewing the
3-wall as boundary of the spacetime(s) in the spirit of $\cite{Israel:rt}$, as explained
nicely in $\cite{Chamblin:1999ya}$.

In order to extend these results in the case where 
the gravitational theory is corrected by  higher order terms in the curvature, 
is at least convenient to
study the case where we add to the Einstein theory the Gauss-Bonnet term. That combination
of  $R^2, R^2_{\mu\nu},R^2_{\mu\nu\rho\sigma}$,
 leads to second order 
field equations which is not true for a general 
combination. Its special properties in $D(\geq 5$) dimensions
are not unrelated to its nature as a topological invariant in 
4d $\cite{mathbook}$ or its origin from strings $\cite{Zwiebach:1985uq}$ 
\footnote{See also the discussion in $\cite{Myers:yn}$.}.

The black hole solution studied in $\cite{Boulware:1985wk}$ is the general solution
of the Einstein-Gauss-Bonnet theory in 5d spacetime with constant spatial curvature,
that respects Birkhoff's theorem $\cite{Charmousis:2002rc}$. We would like to derive
the equation of motion of a 3-surface with constant spatial curvature, that determines
its 4d trajectory in the 5d bulk or, in other words the Friedmann eq. of its Robertson-Walker
world-volume, given the energy momentum tensor $T^{\mu}_{\nu}$ of the surface. Equations
(\ref{thing}) and in an explicit form for the brane universe, eq.(\ref{fin3}) and
(\ref{expl}), are the main
results of this paper.

Assume that the gravity is described by Einstein-Gauss-Bonnet theory and
the spacetime ($D \geq 5$) is a manifold $M$ with boundary $\partial M$. The theory
takes an elegant form and calculations are easier if we write everything in terms
of differential forms. 
Let $E^A$ be the normalized basis of 1-forms in terms of which the metric is  $g= \eta_{AB} E^A \otimes E^B$
with $\eta_{AB}=(-+++..)$. Also define
\begin{eqnarray}
e_{A_1..A_m}=\frac{1}{(D-m)!} \, \epsilon_{A_1..A_m A_{m+1}..A_D} E^{A_{m+1}} \wedge .. \wedge E^{A_D}
\end{eqnarray}
 where
$\epsilon_{A_1..A_D}$ is the completely antisymmetric tensor with the normalization $\epsilon_{0..D-1}=1$.
The curvature 2-form $\Omega^{AB}$ in terms of the connection $\omega^{AB}$ and 
the Riemann tensor $R^A_{\; BCD} $ is
$\Omega^A_{\;B}=d\omega^A_{\;B}+\omega^A_{\;C} \wedge \omega^C_{\;B}=\frac{1}{2} R^A_{\; BCD} E^C \wedge E^D$.
The second fundamental form $\theta^{AB}$ is defined as follows:
if we introduce  Gaussian normal coordinates $(x,w)$ chosen so that the (time-like) boundary 
$\partial M$ is given by $w=0$, the
metric is written as
$ds^2=dw^2+ \gamma_{\mu\nu}(x,w) dx^{\mu} dx^{\nu}$.
We define $\omega_0$ as the connection of the product metric that agrees
 with the previous metric at $\partial M$: $ds^2= dw^2+\gamma_{\mu\nu}(x,0) dx^{\mu} dx^{\nu}$.
Clearly this connection has non-zero components only tangentially on the boundary. Then
\begin{eqnarray}
\theta= \omega- \omega_0
\end{eqnarray}
or in terms of the extrinsic curvature $K^{AB}$ 
\begin{eqnarray}
\theta^{AB}=\theta^{AB}_{\quad C} E^C=(N^A K^B_C- N^B K^A_C) E^C= 2 N^{[A} K^{B]}_C E^C
\end{eqnarray}
$N^A$ is the normal vector on the boundary.
Explicitly the extrinsic curvature reads $K^A_B=-(\eta^{AC}-N^AN^C) \nabla_C N_B$ or in
$(x,w)$ coordinates
$K_{\mu\nu}=-\frac{1}{2} \; \partial \gamma_{\mu\nu}/ \partial w$.
Note that $\theta^{AB}$ has only mixed (normal - tangential on the boundary) components
due to its antisymmetry and the property  $N^A K_{AB}=N^B K_{AB}=0$.

Then the action, discussed explicitly in $\cite{Myers:yn}$
(see also $\cite{Eguchi:jx}$) 
, that contains appropriate
boundary terms so that the normal derivatives of the variations of the metric
cancel identically, can be written
\begin{eqnarray}  \label{action}
S=\int_M -2\Lambda \; e+\Omega^{AB} \wedge e_{AB} + \alpha \; \Omega^{AB} \wedge \Omega^{CD} \wedge e_{ABCD}- \\ \nonumber
- \int_{\partial M} \theta^{AB} \wedge e_{AB}+ \alpha \; 2 \theta^{AB} \wedge 
( \Omega^{CD} - \frac{2}{3} \theta^C_E \wedge \theta^{ED}) \wedge e_{ABCD}
\end{eqnarray}
where we have also introduced a bulk cosmological constant.
\footnote{We are mainly interested in Chern class type actions in view of their desirable
properties. For a construction of the boundary term for a bulk Lagrangian formed as
a general polynomial of the Riemann tensor
see $\cite{Barvinsky:1995dp}$. }

Varying with respect to the basis forms while keeping the connection fixed we have
\begin{eqnarray} \label{var}
\delta_E S=\int_M \delta E^F \wedge (-2 \Lambda \; e_F + \Omega^{AB} \wedge e_{ABF} + \alpha \;
\Omega^{AB} \wedge \Omega^{CD} \wedge e_{ABCDF})+ \\ 
+ \int_{\partial M} \delta E^F \wedge (\theta^{AB} \wedge e_{ABF}+\alpha \; 2 \theta^{AB} \wedge 
( \Omega^{CD} - \frac{2}{3} (\theta \wedge \theta)^{CD}) \wedge e_{ABCDF}) \nonumber
\end{eqnarray}
If there is no matter in the bulk, the bulk volume integral vanishes giving the field equation outside the
boundary. Assuming that there is matter on the boundary, the boundary integral above equals the energy momentum
tensor of the matter on it coming from the variation of the matter Lagrangian. 

In order to write this in an explicit form in terms of the extrinsic curvature and the intrinsic curvature
tensors of the boundary, we first note 
that by $\omega=\omega_0+\theta$ we have
\begin{equation}
\Omega^{AB}=\Omega_0^{AB}+d\theta^{AB}+(\omega_0 \wedge \theta)^{AB}+
(\theta \wedge \omega_0)^{AB} +(\theta \wedge \theta)^{AB}
\end{equation}
where $\Omega_0=d\omega_0+ \omega_0 \wedge \omega_0$ is the intrinsic curvature 2-form of the boundary.
Using that in (\ref{var}) we see that the boundary terms can be written as
\begin{eqnarray} \label{var2}
\int_{\partial M} \delta E^F \wedge (\theta^{AB} \wedge e_{ABF}+\alpha \; 2 \theta^{AB} \wedge 
( \Omega_0^{CD} + \frac{1}{3} (\theta \wedge \theta)^{CD}) \wedge e_{ABCDF}) 
\end{eqnarray} 
as the other terms have to have an index on the normal direction that contributes zero
as there is a factor $\theta^{AB}$ already in the expression.
Using the second of the identities
\begin{eqnarray}
E^C \wedge e_{ABF}= \frac{3!}{2!} \delta^C_{[F} e_{AB]}, \quad 
E^G \wedge E^H \wedge E^I \wedge e_{ABCDF}= \frac{5!}{2!} \delta^I_{[F} \delta^H_D \delta^G_C e_{AB]}=
\frac{5!}{2!} \delta^G_{[A} \delta^H_B \delta^I_C e_{DF]},
\end{eqnarray}
 the first Gauss-Bonnet boundary term gives
\begin{eqnarray}
2 \alpha \, \theta^{AB} \wedge \Omega_0^{CD} \wedge e_{ABCDF}= 
\alpha \, \theta^{AB}_{\; G} \; R^{CD}_{0 \quad HI} \,
\frac{5!}{2!} \; \delta^G_{[A} \delta^H_B \delta^I_C e_{DF]}=
\alpha \, N^A 5! K^B_{[A} \; R^{CD}_{0 \quad BC} e_{DF]}
\end{eqnarray}
Given that the first three indices are orthogonal to $N^A$ we can write
\footnote{Our convention is $R^{\mu}_{\; \nu\rho\sigma}= \partial_{\rho} \Gamma^{\mu}_{\nu\sigma}+..$ .}
\begin{eqnarray} 
&& \alpha \, N^A5! K^B_{[A} \; R^{CD}_{0 \quad BC} e_{DF]}=\alpha \, N^A \,
3!\,2 ( K^B_{[B}R^{CD}_{0 \quad CD]} e_{FA}-
K^B_{[F}R^{CD}_{0 \quad BC]} e_{DA} +  \\ && +K^B_{[D}R^{CD}_{0 \quad FB]} e_{CA}-
K^B_{[C}R^{CD}_{0 \quad DF]} e_{BA} ) \nonumber
\end{eqnarray}
which is easy to see performing the contractions that it takes the form
\begin{eqnarray}
-4\alpha \, N^A \, e_{AB} ([4 (KR_0)^B_F+2K^D_C \, R^{CB}_{0 \quad DF}-2K R^B_{0 \,F }- K^B_F R_0 ]+
\delta^B_F [K R_0 -2 Tr(K R_0)])
\end{eqnarray}
Doing the same for the $\theta \wedge \theta$ term, 
using  $(\theta \wedge \theta)^{CD}=- K^C_H K^D_I \, E^H \wedge E^I$, 
we can finally write the integrand of (\ref{var2}) in the form
\begin{eqnarray} \label{thing} 
&&  \delta E^F \wedge N^A e_{AB} \; [2 (K^B_F- \delta^B_F K) 
+4 \, \alpha ( Q^B_F- \frac{1}{3} \, \delta^B_F Q^C_C )], \\
&&Q^B_F= -2 K^D_C R^{CB}_{0\quad DF}-4 (R_0 K)^B_F +2 K R^B_{0F}+ R_0 K^B_F+ \nonumber \\ \nonumber && \qquad+
 K^B_F (TrK^2-K^2)+
2K (K^2)^B_F -2 (K^3)^B_F \nonumber
\end{eqnarray}
The dynamics of the each boundary is described by setting the quantity in square brackets equal to 
$-2 T^B_F$, for $N^A$ oriented outwards the space
 according to Stokes theorem and
 $T^B_F$ is boundary matter energy momentum tensor. The 
normalization is fixed by the 
the bulk part of the (\ref{var}) which reads $-2( \delta^B_F \Lambda + G^B_F+..)$, where
$G^B_F$ is the Einstein tensor and we use a convention such that the field equations have
the form $G^B_F+..=T^B_F$.


Let us now turn to the problem of interest, where in a 5d spacetime we identify symmetrical points
around an arbitrarily chosen origin, in a certain direction $w$.  
We can think of the space  either as a half infinite line or, as it
is the usual practice, as the
whole line bearing in mind that we have identified $w \sim -w$. The normal vector at the
origin has to have both directions and the origin can be understood as
a non-orientable boundary of the spacetime. Practically
 one, integrating over the boundary, should 
integrate over both directions. The point is that
due to the $\mathbf Z_2$ symmetry, both the normal vector and the extrinsic curvature change sign
going from one ``side" of the boundary to the other, leading to a total factor of 2 in the final 
result. If one wishes to work with the half infinite line, one should divide everything except the
boundary part of (\ref{action}) with a 2! and to take the single direction of normal vector into
that half infinite space.  
Also, as it is natural to take the normal vector oriented inwards each half-space,
there is an additional minus sign.
We can finally write the equation of boundary motion or junction condition for the problem of
interest as 
\begin{eqnarray} \label{fin3}
2 (K^B_F- \delta^B_F K) 
+4 \, \alpha ( Q^B_F- \frac{1}{3} \, \delta^B_F Q^C_C )=T^B_{F}
\end{eqnarray}



Let us apply these results to the case of a 3-wall of constant spatial curvature in
the background of the particular black hole solution of the Einstein-Gauss-Bonnet 5d spacetime
of constant spatial curvature
mentioned earlier. 
Its line element can be written in the form
\begin{eqnarray} \label{bh}
&&ds^2=-f(y) \; dt^2+ \frac{dy^2}{f(y)}+y^2 dx^2, \\ \nonumber
&&dx^2=\frac{dr^2}{1-kr^2}+r^2(d\theta^2+\sin^2\theta d\phi^2) \\ \nonumber
&&f(y)=k+\frac{ y^2}{4\alpha} \left( 1 \pm \surd \left(  1+  
\frac{4\alpha \Lambda}{3}+\frac{8 \alpha \mu}{ y^4} \right) \right)  
\end{eqnarray}
where $\mu$ is the gravitational mass of the black hole.
$\Lambda$ is positive for de Sitter
space and should be bounded as $\Lambda \geq - 3/4 \alpha$.
For the sake of the result as well as for its usefulness,  
we use the method of \cite{Mukohyama:1999wi} to transform from the black hole coordinates to
the ones built around the trajectory of the wall, where the coordinate $w$ is the proper-``time" of the
spacelike geodesics that cross the trajectory vertically.
Then one finds that the metric (\ref{bh}) can be also written
in another familiar form that reads
\begin{eqnarray} \label{wall}
ds^2=-\frac{\psi^2(\tau,w)}{ \varphi (\tau ,w)} d\tau^2+ a^2(\tau) \varphi (\tau,w) dx^2+dw^2
\end{eqnarray} 
where the function $\varphi$ is given implicitly by the equation
\begin{eqnarray} \label{phi}
\int^{\varphi}_1 \frac{dx}{V^{1/2}}= \pm 2w
\end{eqnarray}
where $V$ is given by
\begin{eqnarray}
V=x\left(H^2+ \frac{f(a\surd x)}{a^2} \right)=x \left( H^2+ \frac{k}{a^2} \right)+ \frac{x^2}{4 \alpha} \;
\left( 1 \pm \surd \left(1+\frac{4 \alpha \Lambda}{3} +\frac{8 \alpha \mu}{ a^4 x^2} \right) \right) 
\end{eqnarray}
The Hubble parameter is $H=\dot a(\tau) / a(\tau)$ and
$\psi(\tau,w)=\varphi(\tau,w)+\frac{1}{2H} ( \frac{\partial \varphi}{\partial \tau} )_w$,
where the derivative is given implicitly by (\ref{phi}) and
\begin{equation}
\frac{1}{2H} ( \frac{\partial \varphi}{\partial \tau})_w = \frac{ V^{1/2}|_{x=\varphi } }{2}
\int^{\varphi}_1   \frac{dx}{V^{3/2}}   \left( x \left( \dot H-\frac{k}{a^2} \right) 
\mp \frac{2\mu}{a^4}  \left(  1+\frac{4\alpha \Lambda}{3} +\frac{8 \alpha \mu}{ a^4 x^2} \right)^{-3/2} \right)
\end{equation}
Note that $\varphi (\tau,0)=1$ and $ \partial_{\tau} \varphi (\tau,0)=0$ so $\psi(\tau,0)=1$. That is
the induced metric at $w=0$ is Robertson-Walker with scale factor $a(\tau)$.

We define $S^2(\tau,w)=a^2(\tau) \varphi (\tau,w)$
and $N^2(\tau,w)= \psi^2(\tau,w)/  \varphi (\tau ,w)$ and calculate the 00-component of 
(\ref{fin3}) to obtain
\begin{eqnarray} \label{expl}
2\frac{S'_0}{S_0} \left( 3 +12 \alpha \left( \frac{\dot S_0^2}{N_0^2 S_0^2}+\frac{k}{S_0^2} \right)
-4 \alpha \frac{{S'}^2_0}{S_0^2} \right)=  -\rho
\end{eqnarray}
where $S'_0=S'(0^+)=-S'(0^-)$ fixed at that value from the product of the normal vector with the
extrinsic curvature as explained earlier. Note that $S_0=S(\tau,0)=a(\tau)$ and 
$H=\dot S_0/ S_0= \dot a(\tau)/a(\tau)$
with $N_0=N(\tau,0)=1$.


Taking the square of that equation, we see that we only need the quantity ${S'}^2_0/S_0^2$, which is obtained
by using equation (\ref{phi}) and the definition of $S(\tau,w)$. We have
\begin{eqnarray} \label{basicdis}
&&\frac{{S'_0}^2}{S_0^2}=\frac{1}{4} ( \frac{\partial \varphi} { \partial w} )_{w=0}^2=V(x=1)=
 H^2+ \frac{f(a)}{a^2} = \\
&& \quad =H^2+ \frac{k}{a^2} + \frac{1}{4 \alpha} \;  
\left( 1 \pm \surd \left(1+\frac{4 \alpha \Lambda}{3} +\frac{8 \alpha \mu}{ a^4 } \right) \right) \nonumber
\end{eqnarray}
Substituting in (\ref{expl}) we obtain
 \begin{eqnarray} \label{h}
&&4 (H^2+\frac{k}{a^2}- {\phi}) (3+ 8 \alpha  
(H^2+\frac{k}{a^2}) + 4 \alpha {\phi})^2 = {\rho}^2, \\
&& \phi= - \frac{1}{4 \alpha} \;  
\left( 1 \pm \surd \left(1+\frac{4 \alpha \Lambda}{3}
+\frac{8 \alpha \mu}{ a^4 } \right) \right) \nonumber
\end{eqnarray}

The single real solution of the equation above 
with a smooth limit $\alpha \to 0$
(corresponding to
the minus sign choice for the functions $f(y)$ and $\phi$\,),
can be written in the form
\begin{eqnarray} \label{sol}
H^2= -\frac{1}{4 \alpha} + \frac{(\phi+ \frac{1}{4 \alpha})^2}{Q^{2/3}}+ \frac{Q^{2/3}}{4}-\frac{k}{a^2}, \\
Q=\frac{1}{4 \alpha} \left( \rho+ \surd \left( \rho^2+ 128 \alpha^2 (\phi+ 
\frac{1}{4 \alpha} )^3 \right) \right) \nonumber
\end{eqnarray}
The first order corrections to the result of $\cite{Binetruy:1999ut}$ can be read from
\begin{eqnarray}
H^2+\frac{k}{a^2}=\left(1-\frac{4\alpha\Lambda}{3}-\frac{72\alpha\mu}{a^4} \right) \frac{\rho^2}{36}
-\alpha \frac{\rho^4}{243}
+\frac{\Lambda}{6} \left(1-\frac{\alpha\Lambda}{3} \right) 
+ \frac{\mu(1-6\alpha\Lambda)}{a^4}-\alpha \frac{2\mu^2}{a^8}
\end{eqnarray}
Shifting the energy density, $\rho=\eta+\varrho\,$,  and tuning $\eta$ so that to have 
a vanishing 4d cosmological constant, we obtain
\begin{equation}
H^2+\frac{k}{a^2}=\frac{\sqrt{-\Lambda}}{3 \sqrt{6}} 
\left(1+  \frac{\alpha\Lambda}{2}-\frac{8\alpha \mu}{a^4} \right) \varrho+
\frac{\mu(1+\frac{2}{3} \alpha \Lambda)}{a^4}-\frac{2\alpha\mu^2}{a^8} 
+ {\cal O}(\varrho^2)
\end{equation}
The effective 4d coupling constant has become scale factor dependent.
\footnote{For the plus sign choice in the definition of $f(y)$ the solution (\ref{bh})
is classically unstable $\cite{Boulware:1985wk}$. A simple analysis of the equation (\ref{h})
shows that in this case (where $\phi+ \frac{1}{4\alpha}<0$), the condition for a single real
solution is $\rho^2+ 128 \alpha^2 (\phi+ 
\frac{1}{4 \alpha} )^3>0$. Otherwise there are three real solutions.
 For $\phi+\frac{1}{4\alpha}>0$
there is only one real solution, eq. (\ref{sol}), without constraint.}

Now let us see how the same result arises when we treat the 3-wall as a body in the bulk. Calculating the
00-component of the bulk field equations we obtain
\begin{equation}
\frac{3S''}{S}-
12 \alpha \frac{ S'' {S'}^2}{S^3}+ 12 \alpha  \frac{k S''}{S^3}+
12 \alpha \frac{ S'' \dot S^2}{N^2S^3}+..=- \rho \delta(w)
\end{equation}
where dots contain terms involving only first 
$w$-derivatives and $S'=\partial_w S$ and $\dot S=\partial_{\tau} S$. 

We integrate both sides
in an infinitesimal region around zero. The first derivative of the metric functions, such as
$\varphi$, are taken to change sign passing through the point $w=0$ as is actually implied by
(\ref{phi}). With that in mind one can write
\begin{equation}
\partial_w \left(\frac{3S'}{S}- 4 \alpha \frac{  {S'}^3}{S^3}
 +   12 \alpha \frac{ k S'}{S^3}
+ 12 \alpha \frac{S' \dot S^2}{N^2S^3} \right)+..=- \rho \delta(w)
\end{equation}
where dots are first derivative terms that contribute zero to the integral. Then
\begin{equation} \label{dis}
\frac{3[S_0']}{S_0}- 4 \alpha \frac{  [{S'}_0^3]}{S_0^3}+
 12 \alpha  \frac{ k [S_0']}{S^3_0}+ 12 \alpha \frac{ [S_0'] \dot S_0^2}{N_0^2S_0^3}=-\rho
\end{equation}
where  $[S']=S'(0^+)-S'(0^-)$. $\bf Z_2$ symmetry 
implies that $S'(0^+)=-S'(0^-)$ 
that is $[S']=2S'(0^+)=2S'_0$ and
${S'}^2(0^+)={S'}^2(0^-)={S'}^2_0$. Then, we obtain the same Friedmann equation. This result is
agreement with the results of $\cite{Charmousis:2002rc}$ 
and $\cite{Davis:2002gn}$ and the analysis of $\cite{Low:2000pq}$.

In $\cite{Deruelle:2000ge}$ it was argued that the quantities that appear in these formulas
make the expression not well defined from the point of view of distributions. This due to 
the existence of a term involving ${S'}^2 S''$, as ${S'}^2$ is a discontinuous function. On the
other hand, this can be combined to $({S'}^3)'$ which is well defined and
still a delta function, as the derivative of both the sign function and its cube behave as
delta functions multiplied with smooth functions. A real difficulty would arise only 
in the case of product of distributions, as in the loop calculations in quantum field theory,
where this leads to an introduction of a cutoff. This is the conclusion of $\cite{Deruelle:2000ge}$
for the problem of the Friedmann equation on the brane in the Gauss-Bonnet theory, where
based on that, it is argued that the equation does not change, only the coupling constants, as
in the renormalization of loop graphs.

 The same is suggested in $\cite{Kim:2000pz}$, where
finite results have been found, by treating the sign function squared $\epsilon^2(w)$, in products
with the delta function, as a constant function at the value one. On the other hand 
integrating the functions $\epsilon(w)$ and $\epsilon^2(w)$ with the delta function, it is clear that
the usual rules are not obeyed when one assigns prescribed values to them at $w=0$, through
irrelevant limiting procedures. 

Concluding the main discussion, we would like to emphasize that the procedure of obtaining the junction conditions
by the appropriate boundary term makes clear that the domain wall in an Einstein-Gauss-Bonnet theory
is a surface of zero thickness as much as it is in the Einstein theory. This is because 
one is working directly with the surface of the boundary and there is no room
for any regularization in the normal direction.

Finally, we briefly discuss theories with additional bulk fields.
 Assume for example that the gravity
includes the dilaton field and its action is given by 
\begin{eqnarray}
S=\int d^5x \left( -2\Lambda-\frac{1}{2} (\nabla \Phi)^2
+R+ \alpha\, h(\Phi) (R^2-4 R^2_{\mu\nu}+R^2_{\mu\nu\rho\sigma})
 \right)
\end{eqnarray}
This is just one convenient choice so that to obtain a Friedmann equation with a dilaton type
field present. A general solution for the cosmology on the brane in this theory, is
equivalent to this 3-surface of constant spatial curvature, moving
in a general background in the 5d spacetime with the same property.
Assume then, that a certain general static solution of constant spatial
curvature, as in (\ref{bh}) with a different $f(y)$, is known, as well as the form 
of the dilaton $\Phi=\Phi(y)$
\footnote{To our knowledge there are no analytically  known solutions of this kind. 
It will be harder to find cosmological solutions of such a theory 
using a time dependent metric, e.g. in Gaussian normal coordinates
along the lines of $\cite{Binetruy:1999ut}$.
We derive the Friedmann equation in this case out of theoretical interest. It
can of course be applied to numerical solutions. Solutions of the
type we are interested in have been studied for the case of 4d in $\cite{Kanti:1995vq}$.}.
 By integrating the 00-component of the field equations,
calculated
for the line element in the Gaussian normal coordinates form, 
around $w=0$ and following steps explained above, we obtain
\begin{equation}
4 \left( H^2+\frac{f(a)}{a^2} \right)
  \left( -3+ 4 \alpha \left(h+ 3 a \frac{dh}{da} \right)
  \left( 
 H^2+\frac{f(a)}{a^2} -3 \left(  H^2+\frac{k}{a^2} \right)  \right) 
+ 24 \alpha a \frac{dh}{da} \frac{k}{a^2} \right)^2=\rho^2
\end{equation}
where $a=a(\tau)$ as above and $h=h(\Phi(a))$ and we have used the relation
${S'}_0^2/S_0^2=H^2+f(a)/a^2$ as
shown above (eq. (\ref{basicdis}). For $h=1$ this goes over to equation (\ref{h}).
It is still a cubic equation with respect to the Hubble parameter $H^2$ but with 
scale factor dependent coefficients. 


For the case of a bulk form-field, practically only the black hole changes so
that the Hubble parameter
still satisfies a cubic equation similar to $(\ref{h})$. 
\footnote{Such cases has been studied
at least in $\cite{Lidsey:2002zw}$.}

$\mathbf {Note \; added}$: While this work had been at the final stages S. C. Davis
reported similar analysis in $\cite{Davis:2002gn}$. 

$\mathbf{ \qquad Acknowledgements}$

The work of EG was supported by a King's College Research Scholarship (KRS). The
work of SW was supported by EPSRC. This work is also partially supported
by the E.U. (contract ref. HPRN-CT-2000-00152). We thank Nick Mavromatos (King's College London)
and John Rizos (Ioannina Univ.) for useful
discussions.
We also thank 
E. Papantonopoulos (NTU Athens) for
discussions at the early stages of this work.

\begin{thebibliography}{99} 


%\cite{Binetruy:1999ut}
\bibitem{Binetruy:1999ut}
P.~Binetruy, C.~Deffayet and D.~Langlois,
%``Non-conventional cosmology from a brane-universe,''
Nucl.\ Phys.\ B {\bf 565}, 269 (2000)
[arXiv:hep-th/9905012];
%%CITATION = HEP-TH 9905012;%%
%\cite{Binetruy:1999hy}
P.~Binetruy, C.~Deffayet, U.~Ellwanger and D.~Langlois,
%``Brane cosmological evolution in a bulk with cosmological constant,''
Phys.\ Lett.\ B {\bf 477}, 285 (2000)
[arXiv:hep-th/9910219].
%%CITATION = HEP-TH 9910219;%%


%\cite{Mukohyama:1999wi}
\bibitem{Mukohyama:1999wi}
S.~Mukohyama, T.~Shiromizu and K.~i.~Maeda,
%``Global structure of exact cosmological solutions in the brane world,''
Phys.\ Rev.\ D {\bf 62}, 024028 (2000)
[Erratum-ibid.\ D {\bf 63}, 029901 (2001)]
[arXiv:hep-th/9912287].
%%CITATION = HEP-TH 9912287;%%


%%\cite{Kraus:1999it}
\bibitem{Kraus:1999it}
P.~Kraus,
%``Dynamics of anti-de Sitter domain walls,''
JHEP {\bf 9912}, 011 (1999)
[arXiv:hep-th/9910149];
%%CITATION = HEP-TH 9910149;%%
D.~Ida,
%``Brane-world cosmology,''
JHEP {\bf 0009}, 014 (2000)
[arXiv:gr-qc/9912002].
%%CITATION = GR-QC 9912002;%%


%\cite{Israel:rt}
\bibitem{Israel:rt}
W.~Israel,
%``Singular Hypersurfaces And Thin Shells In General Relativity,''
Nuovo Cim.\ B {\bf 44S10}, 1 (1966)
[Erratum-ibid.\ B {\bf 48}, 463 (1967\ NUCIA,B44,1.1966)].
%%CITATION = NUCIA,B44S10,1;%%

%\cite{Chamblin:1999ya}
\bibitem{Chamblin:1999ya}
H.~A.~Chamblin and H.~S.~Reall,
%``Dynamic dilatonic domain walls,''
Nucl.\ Phys.\ B {\bf 562}, 133 (1999)
[arXiv:hep-th/9903225].
%%CITATION = HEP-TH 9903225;%%

%\cite{mathbook}
\bibitem{mathbook}
Y.~Choquet-Bruhat, C.~DeWitt-Morette with M.~Dillard-Bleick,
%``Analysis,Manifolds and Physics",
North-Holland, Revised Edition 1982.


%\cite{Zwiebach:1985uq}
\bibitem{Zwiebach:1985uq}
B.~Zwiebach,
%``Curvature Squared Terms And String Theories,''
Phys.\ Lett.\ B {\bf 156}, 315 (1985).
%%CITATION = PHLTA,B156,315;%%
%\cite{Metsaev:1987zx}
R.~R.~Metsaev and A.~A.~Tseytlin,
%``Order Alpha-Prime (Two Loop) Equivalence Of The String Equations Of Motion And The Sigma Model Weyl Invariance Conditions: Dependence On The
Dilaton And The Antisymmetric Tensor,''
Nucl.\ Phys.\ B {\bf 293}, 385 (1987).
%%CITATION = NUPHA,B293,385;%%

%\cite{Boulware:1985wk}
\bibitem{Boulware:1985wk}
D.~G.~Boulware and S.~Deser,
%``String Generated Gravity Models,''
Phys.\ Rev.\ Lett.\  {\bf 55}, 2656 (1985);
%%CITATION = PRLTA,55,2656;%%
R.~G.~Cai,
%``Gauss-Bonnet black holes in AdS spaces,''
Phys.\ Rev.\ D {\bf 65}, 084014 (2002)
[arXiv:hep-th/0109133].
%%CITATION = HEP-TH 0109133;%%


%\cite{Charmousis:2002rc}
\bibitem{Charmousis:2002rc}
C.~Charmousis and J.~F.~Dufaux,
%``General Gauss-Bonnet brane cosmology,''
arXiv:hep-th/0202107.
%%CITATION = HEP-TH 0202107;%%

%\cite{Myers:yn}
\bibitem{Myers:yn}
R.~C.~Myers,
%``Higher Derivative Gravity, Surface Terms And String Theory,''
Phys.\ Rev.\ D {\bf 36}, 392 (1987).
%%CITATION = PHRVA,D36,392;%%

%\cite{Eguchi:jx}
\bibitem{Eguchi:jx}
T.~Eguchi, P.~B.~Gilkey and A.~J.~Hanson,
%``Gravitation, Gauge Theories And Differential Geometry,''
Phys.\ Rept.\  {\bf 66}, 213 (1980).
%%CITATION = PRPLC,66,213;%%

%\cite{Barvinsky:1995dp}
\bibitem{Barvinsky:1995dp}
A.~D.~Barvinsky and S.~N.~Solodukhin,
%``Non-minimal coupling, boundary terms and renormalization of 
%the Einstein-Hilbert action and black hole entropy,''
Nucl.\ Phys.\ B {\bf 479}, 305 (1996)
[arXiv:gr-qc/9512047].
%%CITATION = GR-QC 9512047;%%


%\cite{Low:2000pq}
\bibitem{Low:2000pq}
I.~Low and A.~Zee,
%``Naked singularity and Gauss-Bonnet term in brane world scenarios,''
Nucl.\ Phys.\ B {\bf 585}, 395 (2000)
[arXiv:hep-th/0004124];
%%CITATION = HEP-TH 0004124;%%
%\cite{Mavromatos:2000az}
N.~E.~Mavromatos and J.~Rizos,
%``String inspired higher-curvature terms and the Randall-Sundrum  scenario,''
Phys.\ Rev.\ D {\bf 62}, 124004 (2000)
[arXiv:hep-th/0008074];
%%CITATION = HEP-TH 0008074;%%
%\cite{Neupane:2000wt}
I.~P.~Neupane,
%``Consistency of higher derivative gravity in the brane background,''
JHEP {\bf 0009}, 040 (2000)
[arXiv:hep-th/0008190];
%%CITATION = HEP-TH 0008190;%%
%\cite{Meissner:2000dy}
K.~A.~Meissner and M.~Olechowski,
%``Domain walls without cosmological constant in higher order gravity,''
Phys.\ Rev.\ Lett.\  {\bf 86}, 3708 (2001)
[arXiv:hep-th/0009122];
%%CITATION = HEP-TH 0009122;%%
%\cite{Cho:2001su}
Y.~M.~Cho, I.~P.~Neupane and P.~S.~Wesson,
%``No ghost state of Gauss-Bonnet interaction in warped background,''
Nucl.\ Phys.\ B {\bf 621}, 388 (2002)
[arXiv:hep-th/0104227];
%%CITATION = HEP-TH 0104227;%%
%\cite{Mavromatos:2002vt}
N.~E.~Mavromatos and J.~Rizos,
%``Exact solutions and the cosmological constant problem in dilatonic  domain wall higher-curvature string gravity,''
[arXiv:hep-th/0205299];
%%CITATION = HEP-TH 0205299;%%
%\cite{Binetruy:2002ck}
P.~Binetruy, C.~Charmousis, S.~C.~Davis and J.~F.~Dufaux,
%``Avoidance of naked singularities in dilatonic brane world scenarios  with a Gauss-Bonnet term,''
[arXiv:hep-th/0206089].
%%CITATION = HEP-TH 0206089;%%


%\cite{Deruelle:2000ge}
\bibitem{Deruelle:2000ge}
N.~Deruelle and T.~Dolezel,
%``Brane versus shell cosmologies in Einstein and Einstein-Gauss-Bonnet  theories,''
Phys.\ Rev.\ D {\bf 62}, 103502 (2000)
[arXiv:gr-qc/0004021].
%%CITATION = GR-QC 0004021;%%


%\cite{Kim:2000pz}
\bibitem{Kim:2000pz}
J.~E.~Kim, B.~Kyae and H.~M.~Lee,
%``Various modified solutions of the Randall-Sundrum model with the  Gauss-Bonnet interaction,''
Nucl.\ Phys.\ B {\bf 582}, 296 (2000)
[Erratum-ibid.\ B {\bf 591}, 587 (2000)]
[arXiv:hep-th/0004005];
%%CITATION = HEP-TH 0004005;%%
%\cite{Abdesselam:2001ff}
B.~Abdesselam and N.~Mohammedi,
%``Brane world cosmology with Gauss-Bonnet interaction,''
Phys.\ Rev.\ D {\bf 65}, 084018 (2002)
[arXiv:hep-th/0110143];
%%CITATION = HEP-TH 0110143;%%
%\cite{Germani:2002pt}
C.~Germani and C.~F.~Sopuerta,
%``String inspired braneworld cosmology,''
Phys.\ Rev.\ Lett.\  {\bf 88}, 231101 (2002)
[arXiv:hep-th/0202060].
%%CITATION = HEP-TH 0202060;%%



%\cite{Davis:2002gn}
\bibitem{Davis:2002gn}
S.~C.~Davis,
%``Generalised Israel junction conditions for a Gauss-Bonnet brane world,''
arXiv:hep-th/0208205.
%%CITATION = HEP-TH 0208205;%%



%\cite{Kanti:1995vq}
\bibitem{Kanti:1995vq}
P.~Kanti, N.~E.~Mavromatos, J.~Rizos, K.~Tamvakis and E.~Winstanley,
%``Dilatonic Black Holes in Higher Curvature String Gravity,''
Phys.\ Rev.\ D {\bf 54}, 5049 (1996)
[arXiv:hep-th/9511071];
%%CITATION = HEP-TH 9511071;%%
%\cite{Alexeev:1996vs}
S.~O.~Alexeev and M.~V.~Pomazanov,
%``Black hole solutions with dilatonic hair in higher curvature gravity,''
Phys.\ Rev.\ D {\bf 55}, 2110 (1997)
[arXiv:hep-th/9605106];
%%CITATION = HEP-TH 9605106;%%
%\cite{Torii:1996yi}
T.~Torii, H.~Yajima and K.~i.~Maeda,
%``Dilatonic black holes with Gauss-Bonnet term,''
Phys.\ Rev.\ D {\bf 55}, 739 (1997)
[arXiv:gr-qc/9606034];
%%CITATION = GR-QC 9606034;%%
%\cite{Alekseev:cg}
S.~O.~Alekseev and M.~V.~Sazhin,
%``Four-Dimensional Dilatonic Black Holes In Gauss-Bonnet Extended String  Gravity,''
Gen.\ Rel.\ Grav.\  {\bf 30}, 1187 (1998).
%%CITATION = GRGVA,30,1187;%%


%\cite{Lidsey:2002zw}
\bibitem{Lidsey:2002zw}
J.~E.~Lidsey, S.~Nojiri and S.~D.~Odintsov,
%``Braneworld cosmology in (anti)-de Sitter Einstein-Gauss-Bonnet-Maxwell  gravity,''
JHEP {\bf 0206}, 026 (2002)
[arXiv:hep-th/0202198].
%%CITATION = HEP-TH 0202198;%%
%\cite{Cvetic:2001bk}
M.~Cvetic, S.~Nojiri and S.~D.~Odintsov,
%``Black hole thermodynamics and negative entropy in deSitter and  anti-deSitter Einstein-Gauss-Bonnet gravity,''
Nucl.\ Phys.\ B {\bf 628}, 295 (2002)
[arXiv:hep-th/0112045].
%%CITATION = HEP-TH 0112045;%%



\end{thebibliography} 

\end{document} 

%\cite{Metsaev:1987zx}
\bibitem{Metsaev:1987zx}
R.~R.~Metsaev and A.~A.~Tseytlin,
%``Order Alpha-Prime (Two Loop) Equivalence Of The String Equations Of Motion And The Sigma Model Weyl Invariance Conditions: Dependence On The
Dilaton And The Antisymmetric Tensor,''
Nucl.\ Phys.\ B {\bf 293}, 385 (1987).
%%CITATION = NUPHA,B293,385;%%




The reality condition for the Hubble parameter now reads 
\begin{equation} \label{real}
\bar{h}^2 \rho^2-128 \alpha^2  \bar{h} \left( \bar{h} \frac{f(a)}{a^2}-\frac{1}{4 \alpha} \right)^3 \geq 0
\end{equation}
where $\bar{h}=h+3a \, dh/da$ and we restrict ourselves to the flat case $k=0$. Note
that $\bar{h}$ is not positive in general. 






