%%%%%%%%%%%%%%%%%%%%%%%%%%%%%%%%%%%%%%%%%%%%%%%%%%%%%%%%%%%%%%%%%%%
%
%          Towards Noncommutative Integrable Systems
%
%%%%%%%%%%%%%%%%%%%%%%%%%%%%%%%%%%%%%%%%%%%%%%%%%%%%%%%%%%%%%%%%%%%
\documentclass[a4paper,12pt]{article}\setlength{\topmargin}{-1cm}
\setlength{\oddsidemargin}{0cm}\setlength{\evensidemargin}{0cm}
\setlength{\textwidth}{16cm}
\setlength{\textheight}{22cm}

\makeatletter
\@addtoreset{equation}{section}
\def\theequation{\thesection .\arabic{equation}}
\makeatother

\begin{document}

\begin{titlepage}
\null
\begin{flushright}
UT-02-54
\\
hep-th/0211148
\\
November, 2002
\end{flushright}

\vskip 1.5cm
\begin{center}

  {\LARGE Towards Noncommutative Integrable Systems}

\lineskip .75em
\vskip 2cm
\normalsize

  {\large Masashi Hamanaka\footnote{
e-mail: hamanaka@hep-th.phys.s.u-tokyo.ac.jp} and 
Kouichi Toda\footnote{e-mail: kouichi@yukawa.kyoto-u.ac.jp} }

\vskip 2cm

  {\it Department of Physics, University of Tokyo,\\
               Tokyo 113-0033, Japan}

\vskip 0.5cm

  {\it Department of Mathematical Physics,
Toyama Prefectural University,\\
%Kurokawa 5180, Kosugi, Imizu,
Toyama, 939-0398, Japan}

\vskip 1.5cm

{\bf Abstract}

\end{center}

We present a strong method to generate various equations
which have the Lax representations 
on noncommutative $(1+1)$ and $(2+1)$-dimensional spaces. 
The generated equations contain noncommutative integrable
equations obtained by using the bicomplex method
and by reductions of the noncommutative
(anti-)self-dual Yang-Mills equation.
This suggests that
the noncommutative Lax equations would be integrable
and be derived from reductions of the noncommutative 
(anti-)self-dual Yang-Mills equations,
which implies the noncommutative version of Richard Ward conjecture.

\end{titlepage}

\clearpage

\baselineskip 6mm

\section{Introduction}

Non-Commutative (NC) gauge theory has been studied intensively 
for the last several years 
and succeeded in revealing various aspects of gauge theories
in the presence of background magnetic fields \cite{DoNe}. 
Especially,
NC solitons play crucial roles in the study of D-brane
dynamics, such as tachyon condensation \cite{Harvey}.
 
NC spaces are characterized by the noncommutativity of
the coordinates:
\begin{eqnarray}
\label{nc_coord}
[x^i,x^j]=i\theta^{ij},
\end{eqnarray}
where $\theta^{ij}$ are real constants.  
This relation looks like the canonical commutation
relation in quantum mechanics
and leads to ``space-space uncertainty relation.''
Hence
the singularity which exists on commutative spaces could resolve
on NC spaces.
This is one of the distinguished features of NC theories
and gives rise to various new physical objects.
For example, even when the gauge group is $U(1)$,
instanton solutions still exist \cite{NeSc}
because of the resolution of the small instanton singularities 
of the complete instanton moduli space \cite{Nakajima}.

NC gauge theories are naively realized from
ordinary commutative theories just by replacing
all products of the fields with star-products.
In this context, NC theories are
considered to be deformed theories from commutative ones
and look very close to the commutative ones.

Under the deformation,
the (anti-)self-dual (ASD) Yang-Mills equations
could be considered to preserve the integrability
in the same sense as in commutative cases \cite{KKO, Nekrasov}.
On the other hand, with regard to typical integrable equations
such as the Kadomtsev-Petviasfvili (KP) equation,
the naive NC extension generally destroys the integrability.
There is known to be a method, the {\it bicomplex method}, to yield
NC integrable equations which have many conserved quantities
\cite{Bicomplex, GrPe}.

In this paper, we discuss
NC extensions of wider class of integrable equations
which are expected to preserve the integrability.
First, we present a strong method 
to give rise to NC Lax pairs 
and construct various NC Lax equations.
Then we discuss the relationship between
the generated equations and the NC integrable equations 
obtained from the bicomplex method
and from reductions of the NC ASD Yang-Mills equations.
All the results are consistent and we can expect that 
the NC Lax equations would be integrable. 
Hence it is natural to
propose the following conjecture which contains 
the NC version of Ward conjecture:
{\it many of NC Lax equations would be integrable
and be obtained from reductions of the NC ASD Yang-Mills equations}.

\section{Noncommutative Lax Equations}

\subsection{The Lax-Pair Generating Technique}

In commutative cases, 
Lax representations are
common in many of known integrable equations
and fit well to the discussion of reductions 
of the ASD Yang-Mills equations.
Here we look for the Lax representations on NC spaces.
First we introduce how to find Lax representations on commutative spaces.

An integrable equation which has the Lax representation
can be rewritten as the following equation:
\begin{eqnarray}
[L,T+\partial_t]=0,
\label{lax}
\end{eqnarray}
where $\partial_t:=\partial/\partial t$.
This equation and the pair of operators $(L,T)$ are 
called the {\it Lax equation} and the {\it Lax pair}, respectively.

The NC version of the Lax equation (\ref{lax}),
the {\it NC Lax equation}, is easily defined
just by replacing the product of $L$ and $T$ with the star product.
The star product is defined by
\begin{eqnarray}
f\star g(x):=
\exp{\left(\frac{i}{2}\theta^{ij}\partial^{(x^\prime)}_i
\partial^{(x^{\prime\prime})}_j\right)}
f(x^\prime)g(x^{\prime\prime})\Big{\vert}_{x^{\prime}
=x^{\prime\prime}=x.}%\nonumber\\
%&=&f(x)g(x)+\frac{i}{2}\theta^{ij}\partial_if(x)\partial_jg(x)
%+{\cal O}(\theta^2).
\end{eqnarray}
The star product has associativity: $f\star (g\star h)=(f\star g)\star h$,
and reduces to the ordinary product in the limit $\theta^{ij}\rightarrow 0$.
The modification of the product  makes the ordinary
coordinate ``noncommutative,'' 
which means : $[x^i,x^j]_\star :=x^i\star x^j-x^j\star x^i=i\theta^{ij}$.

In this paper, we look for the NC Lax equation
whose operator $L$ is a differential operator.
In order to make this study systematic,
we set up the following problem :

\vspace{3mm}
\noindent
{\bf Problem :}
For a given operator $L$, 
find the corresponding operator $T$
which satisfies the Lax equation (\ref{lax}).
\vspace{3mm}

This is in general very difficult to solve.
However if we put an ansatz on the operator $T$,
then we can get the answer 
for wide class of Lax pairs
including NC case.
The ansatz for the operator $T$ is of the following type:

\vspace{3mm}
\noindent
{\bf Ansatz for the operator $T$ :}
\begin{eqnarray}
\label{ansatz}
T=\partial_i^n L+T^\prime.
\end{eqnarray}
\noindent
Then the problem for $T$ reduces to that for $T^\prime$.
This ansatz is very simple, however, very strong
to determine the unknown operator $T^\prime$, 
that is, the Lax pair $(L,T)$,
which is called, in this paper, the
{\it Lax-pair generating technique}.

In order to explain it more concretely,
let us consider the Korteweg-de-Vries (KdV) 
equation on commutative $(1+1)$-dimensional space
where the operator $L$ is given 
by $L_{\scriptsize\mbox{KdV}}:=\partial_x^2+u(t,x)$.

The ansatz for the operator $T$ is given by
\begin{eqnarray}
T=\partial_x L_{\scriptsize\mbox{KdV}} +T^\prime,
\label{ansatz_KdV}
\end{eqnarray}
which corresponds to $n=1$ and $\partial_i=\partial_x$ in
the general ansatz (\ref{ansatz}).
This factorization 
was first used to find more wider class of
Lax pairs in higher dimensional case \cite{ToYu}.

The Lax equation (\ref{lax}) leads to the equation for 
the unknown operator $T^\prime$:
\begin{eqnarray}
[\partial_x^2+u,T^\prime]=u_x\partial_x^2+u_t+uu_x,
\label{Tpr}
\end{eqnarray}
where $u_x:=\partial u/\partial x$ and so on.
Here we want to delete the term $u_x\partial_x^2$ in the RHS of (\ref{Tpr})
so that this equation
finally reduces to a differential equation.
Therefore the operator $T^\prime$ could be taken as
\begin{eqnarray}
T^\prime=A \partial_x+B,
\end{eqnarray}
where $A, B$ are polynomials of $u, u_x, u_t, u_{xx},$ etc.
Then the Lax equation becomes $f\partial_x^2+g\partial_x+h=0$.
{}From $f=0, g=0$, we get\footnote{Exactly speaking, 
an integral constant should appear in $A$ as $A=u/2+\alpha$.
This constant $\alpha$ is unphysical
and can be absorbed by the scale transformation $u\rightarrow u+2\alpha/3$. 
Hence we can take $\alpha=0$ without loss of generality.}
\begin{eqnarray}
A=\frac{u}{2},~~~B=-\frac{1}{4}u_x+\beta,
\end{eqnarray}
that is,
\begin{eqnarray}
T=\partial_x^3+\frac{3}{4}u_x+\frac{3}{2}u\partial_x.
\end{eqnarray}
Finally $h=0$ yields a Lax equation, the KdV equation:
\begin{eqnarray}
u_t+\frac{3}{2}uu_x+\frac{1}{4}u_{xxx}=0.
\end{eqnarray}

In this way, we can generate wide class of Lax equations 
including higher dimensional integrable equations \cite{ToYu}.
For example, $L_{\scriptsize\mbox{mKdV}}:=\partial_x^2+v(t,x)\partial_x$ and 
$L_{\scriptsize\mbox{KP}}:=\partial_x^2+u(t,x,y)+\partial_y$ give rise to 
the modified KdV equation and the KP equation, respectively 
by the same ansatz (\ref{ansatz_KdV}) for $T$.
If we take $L_{\scriptsize\mbox{BCS}}:=\partial_x^2+u(t,x,y)$ and 
the modified ansatz $T=\partial_y L_{\scriptsize\mbox{BCS}}+T^\prime$,
then we get the Bogoyavlenskii-Calogero-Schiff (BCS) 
equation \cite{BoSc}.\footnote{The multi-soliton solution 
is found in \cite{YTSF}.}

Good news here is that this technique is also applicable to
NC cases.

\subsection{Some Results}

We present some results by using the Lax-pair generating technique.
First we focus on NC $(2+1)$-dimensional
Lax equations.
Let us suppose that the noncommutativity is basically
introduced in the space directions.  
\begin{itemize}

\item The NC KP equation 
%(the noncommutativity: $[x,y]=i\theta$.) 
\cite{Paniak} :

The Lax operator is given by
\begin{eqnarray}
L_{\scriptsize\mbox{KP}}=\partial_x^2+u(t,x,y)+\partial_y
:=L_{\scriptsize\mbox{KP}}^\prime+\partial_y.
\end{eqnarray}
The ansatz for the operator $T$ is the same as commutative case:
\begin{eqnarray}
T=\partial_x L_{\scriptsize\mbox{KP}}^\prime+T^\prime.
\end{eqnarray}
Then we find
\begin{eqnarray}
T^\prime=\frac{1}{2}u\partial_x-\frac{1}{4}u_x-\frac{3}{4}\partial_x^{-1}u_y,
\end{eqnarray}
and the NC KP equation:
\begin{eqnarray}
\label{ncKP}
u_t+\frac{1}{4}u_{xxx}+\frac{3}{4}(u_x\star u+u\star u_x)
+\frac{3}{4}\partial_x^{-1}u_{yy}
+\frac{3}{4}[u,\partial_x^{-1} u_y]_\star
=0,
\end{eqnarray}
%The Lax pair:
%\begin{eqnarray}
%T&=&\partial_x(L)+T^\prime,\\
%T^\prime&=&\frac{1}{2}u\partial_x
%-\frac{1}{4}u_x+\frac{3}{4}\partial_x^{-1}u_y.
%\end{eqnarray}
where $\partial_x^{-1}f(x):=\int^x dx^\prime f(x^\prime),
~u_{xxx}=\partial^3 u/\partial x^3$ and so on.
This coincides with that in \cite{Paniak}.
There is seen to be a nontrivial deformed term $[u,\partial_x^{-1} u_y]_\star$
in the equation (\ref{ncKP}) which vanishes in the commutative limit.
In \cite{Paniak}, the multi-soliton solution is found
by the first order to small $\theta$ expansion,
which suggests that this equation would be considered as an 
integrable equation.

The modified version of the NC KP equation is also found
in the similar way. This is new.

\item The NC modified KP equation 
%(the noncommutativity: $[x,y]=i\theta$.)
:

For a given operator $L_{\scriptsize\mbox{mKP}}
=\partial_x^2+v(t,x,y)\partial_x+\partial_y 
:=L_{\scriptsize\mbox{mKP}}^\prime+\partial_y$,
we can take the ansatz $T=\partial_x L_{\scriptsize\mbox{mKP}}^\prime+T^\prime$,
which yields the NC modified KP equation:
\begin{eqnarray}
&&v_t+\frac{1}{4}v_{xxx}-\frac{3}{8}v \star v_x\star v
+\frac{3}{8}([v,v_x]_\star)_x
-\frac{3}{4}(\partial_x^{-1}v_y)\star v_x\nonumber\\
&&+\frac{3}{4}\partial_x^{-1}v_{yy}
+\frac{3}{8}[v,v_{xx}+v_y]_\star=0,
\label{ncmKP}
\end{eqnarray}
where the operator $T^\prime$ are determined as
\begin{eqnarray}
T^\prime = \frac{1}{2}v\partial_x^2 
+ \left(-\frac{1}{4}v_x + \frac{3}{8}v\star v - \frac{3}{4}\partial_x^{-1}v_y
\right) \partial_x. 
\end{eqnarray}
%\begin{eqnarray}
%T^\prime = (1/2)v\partial_x^2 
%+ \left(-(1/4)v_x + (3/8)(v\star v) - (3/4)\partial_x^{-1}v_y
%\right) \partial_x. 
%\end{eqnarray}
Nontrivial terms are also seen in the equation (\ref{ncmKP}).

%The Lax pair:
%\begin{eqnarray}
%L&=&\partial_x^2+v\partial_x+\partial_y,\nonumber\\
%T&=&\partial_x(L)+T^\prime,\\
%T^\prime&=&
%\frac{1}{2}v\partial_x^2
%+\left(-\frac{1}{4}v_x+\frac{3}{8}(v\star v)-\frac{3}{4}\partial_x^{-1}v_y
%\right)\partial_x.
%\end{eqnarray}

\item The NC BCS equation 
%(the noncommutativity: $[x,y]=i\theta$.)
:

This is obtained by the same steps as in commutative case.
The new equation is 
\begin{eqnarray}
&&u_t+\frac{1}{4}u_{xxy}+\frac{1}{2}(u_y\star u+u\star u_y)
+\frac{1}{4}u_x\star (\partial_x^{-1} u_y)\nonumber\\
&&+\frac{1}{4}(\partial_x^{-1}u_{y})\star u_x
+\frac{1}{4}[u,\partial_x^{-1}[u,\partial_x^{-1}u_{y}]_\star]_\star=0,
\end{eqnarray}
whose Lax pair and the ansatz are
\begin{eqnarray}
L_{\scriptsize\mbox{BCS}}&=&\partial_x^2+u(t,x,y),\nonumber\\
T&=&\partial_y L_{\scriptsize\mbox{BCS}}+T^\prime,\\
T^\prime&=&\frac{1}{2}(\partial_x^{-1}u_y)\partial_x
-\frac{1}{4}u_y-\frac{1}{4}\partial_x^{-1}[u,\partial_x^{-1}u_y]_\star.
\end{eqnarray}
This time, a non-trivial term is found even in the operator $T$.

\end{itemize}

We can generate many other NC Lax equations in the same way.
Moreover if we introduce the noncommutativity into time coordinate
as $[t,x]=i\theta$,
we can construct NC $(1+1)$-dimensional integrable equations.

For example, the NC KdV equation is
\begin{eqnarray}
u_t+\frac{3}{4}(u_x\star u +u\star u_x)+\frac{1}{4}u_{xxx}=0,
\end{eqnarray}
which coincides with that derived by using the bicomplex method \cite{DiMH}
and by the reduction from NC KP equation (\ref{ncKP})
setting the fields $y$-independent: $\partial_y u =0$
and reintroducing the noncommutativity as $[t,x]=i\theta$, 
this time.\footnote{We note that this reduction is formal
and the noncommutativity here contains subtle points in
the derivation from the $(2+2)$-dimensional
NC ASD Yang-Mills equation by reduction
because the coordinates $(t,x,y)$
originate partially from the parameters 
in the gauge group of the NC ASD Yang-Mills theory \cite{Conj}.
We are grateful to T.~Ivanova for pointing out this point to us.}
(This is true of the NC modified KdV equation.)
We also find the NC KdV hierarchy \cite{Toda}.

As one of the new Lax equations, the NC Burgers equation is obtained:
\begin{eqnarray}
u_t-\alpha u_{xx}+(1-\alpha-\beta)u\star u_x+(1+\alpha-\beta)u_x\star u=0.
\end{eqnarray}
We succeed in linearizing it by the NC 
Cole-Hopf transformation \cite{HaTo}.

All of the NC integrable equations derived from the bicomplex method
are also obtained by our method.
The bicomplex method guarantees the existence of the 
many conserved topological quantities.
These results suggest that NC Lax equations would possess 
the integrability. 

Here we comment on the multi-soliton solutions.
First we note that 
if the field is holomorphic, that is, $f=f(x-vt)=f(z)$,
then the star product reduces to the ordinary product:
\begin{eqnarray}
f(x-vt)\star g(x-vt)=f(x-vt)g(x-vt).
\end{eqnarray}
Hence the commutative
multi-soliton solutions where all the solitons
move at the same velocity always satisfy
the NC version of the equations.
Of course, this does not mean that
the equations possess the integrability.

The comprehensive list and the more detailed discussion 
are reported later soon.

\section{Comments on the Noncommutative Ward Conjecture}

In commutative case, it is well known that 
many of integrable equations could be
derived from symmetry reductions of the four-dimensional
ASD Yang-Mills equation \cite{Conj},
which is first conjectured by R.Ward \cite{Ward}.

Even in NC case,
the corresponding discussions would be possible and be interesting.
The NC ASD Yang-Mills equations also have the Yang's 
forms \cite{Nekrasov2, Nekrasov} and many other similar
properties to commutative ones \cite{KKO}.
The simple reduction to three dimension
yields the NC Bogomol'nyi equation
which has the exact monopole solutions and can be rewritten 
as the non-Abelian Toda lattice equation \cite{Nekrasov, GrNe}.
It is interesting that a discrete structure appears.
%\footnote{
%At the solution level, the Fourier-transformed configuration
%of a periodic instanton satisfies the NC Bogomol'nyi equation and
%coincides with the fluxon \cite{Fluxon} 
%in the zero-period limit \cite{Hamanaka}.}
Moreover M.Legar\'e \cite{Legare} succeeded in some reductions
of the $(2+2)$-dimensional NC ASD Yang-Mills equations which coincide
with our results and those by using the bicomplex method,
which strongly suggests that the noncommutative deformation
would be unique and integrable
and the Ward conjecture would still hold on NC spaces.

In four-dimensional Yang-Mills theory,
the NC deformation resolves the small instanton 
singularity of the (complete) instanton moduli space
and gives rise to a new physical object, the $U(1)$ instanton.
Hence the NC Ward conjecture would imply that
the NC deformations of lower-dimensional integrable equations
might contain new physical objects
because of the deformations of the solution spaces in some case.

\section{Conclusion and Discussion}

In the present paper, 
we found a powerful method to
find NC Lax equations which
is expected to be integrable.
The simple, but mysterious, ansatz (\ref{ansatz}) 
plays an important role
and actually gives rise to various new NC Lax equations.
Finally we pointed out that some reductions of the NC
ASD Yang-Mills equations give rise to
NC integrable equations including our results.

Now there would be mainly three methods to yield NC integrable
equations:
\begin{itemize}
\item Lax-pair generating technique
\item Bicomplex method
\item Reduction of the ASD Yang-Mills equation
\end{itemize}
The interesting point is that 
all the results are consistent
at least with the known NC Lax equations, 
which suggests the existence and 
the uniqueness of the NC deformations of integrable equations
which preserve the integrability.

Though we can get many new NC Lax equations,
there need to be more discussions so that
such study should be fruitful as integrable systems.
First, we have to clarify whether the NC Lax equations
are really good equations in the sense of integrability, that is,
the existence of many conserved quantities or
of multi-soliton solutions, and so on.
All of the previous studies including our works
strongly suggest that this would be true.
Second, we have to reveal the physical meaning of such equations.
If such integrable theories can be embedded in string
theories, there would be fruitful interactions
between the both theories,
just as between the (NC) ASD Yang-Mills equation and
D0-D4 brane system (in the background of NS-NS $B$ field).
%For this purpose, D-brane picture would be useful because
%ASD Yang-Mills equation describes D0-D4 brane system and
%Ward conjecture suggests that there are correspondences
%between integrable equations and some conditions of 
%the reduced D-brane systems.
%The co-supplement study would be very interesting.
%and $N=2$ string \cite{OoVa} would be worth studying.

The systematic and co-supplement studies of them would
pioneer a new area of integrable systems and 
perhaps string theories.

\vskip5mm\noindent
{\bf Note added}
\vskip2mm

\noindent
%After submitting the first version of the present paper 
%to the e-Print archive, 
We were informed by O. Lechtenfeld 
that the $(2+2)$-dimensional NC ASD Yang-Mills equation and
some reductions of it can be embedded \cite{LPS, LePo}
in $N=2$ string theory \cite{OoVa},
which guarantees that such directions would have a physical
meaning and might be helpful to understand new aspects of 
the corresponding string theory.

\vskip7mm\noindent
{\bf Acknowledgments}
\vskip2mm

\noindent
We would like to thank the YITP at Kyoto University 
for the hospitality during the YITP
workshop YITP-W-02-04 on ``QFT2002''
and our stay as atom-type visitors.
M.H. is also grateful to M.~Asano and I.~Kishimoto for useful comments.
The work of M.H. was supported in part 
by the Japan Securities Scholarship Foundation (\#12-3-0403).


\begin{thebibliography}{99}

%\baselineskip 6mm

\bibitem{DoNe}
M.~R.~Douglas and N.~A.~Nekrasov,
%``Noncommutative field theory,''
Rev.\ Mod.\ Phys.\  {\bf 73} (2002) 977
[hep-th/0106048]
%%CITATION = HEP-TH 0106048;%%
and references therein.

\bibitem{Harvey}
J.~A.~Harvey,
``Komaba lectures on noncommutative solitons and D-branes,''
hep-th/0102076
%%CITATION = HEP-TH 0102076;%%
and references therein.

\bibitem{NeSc}
N.~Nekrasov and A.~Schwarz,
%``Instantons on noncommutative ${\bf R}^4$, 
%and (2,0) superconformal six dimensional theory,''
Commun.\ Math.\ Phys.\ {\bf 198} (1998) 689
[hep-th/9802068].
%%CITATION = HEP-TH 9802068;%%

\bibitem{Nakajima}
H.~Nakajima,
``Resolutions of moduli spaces of ideal instantons on ${\bf R}^4$,''
in {\it Topology, Geometry and Field Theory}
(World Sci., 1994) 129
[ISBN/981-02-1817-6].

\bibitem{KKO}
A.~Kapustin, A.~Kuznetsov and D.~Orlov,
%``Noncommutative instantons and twistor transform,''
Commun.\ Math.\ Phys.\  {\bf 221} (2001) 385
[hep-th/0002193];
%%CITATION = HEP-TH 0002193;%%
%\bibitem{Takasaki}
K.~Takasaki,
%``Anti-self-dual Yang-Mills equations on noncommutative spacetime,''
J.\ Geom.\ Phys.\ {\bf 37} (2001) 291
[hep-th/0005194];
%%CITATION = HEP-TH 0005194;%%
%\bibitem{Hannabuss}
K.~C.~Hannabuss,
%``Non-commutative twistor space,''
Lett.\ Math.\ Phys.\  {\bf 58} (2001) 153
[hep-th/0108228];
%%CITATION = HEP-TH 0108228;%%
%\bibitem{LePo}
O.~Lechtenfeld and A.~D.~Popov,
%``Noncommutative 't Hooft instantons,''
JHEP {\bf 0203} (2002) 040
[hep-th/0109209];
%%CITATION = HEP-TH 0109209;%%
%\bibitem{HLW}
Z.~Horv\'ath, O.~Lechtenfeld and M.~Wolf
``Noncommutative instantons via dressing and splitting approaches,''
hep-th/0211041.
%%CITATION = HEP-TH 02011041;%%

\bibitem{Nekrasov}
N.~A.~Nekrasov,
``Trieste lectures on solitons in noncommutative gauge theories,''
hep-th/0011095
%%CITATION = HEP-TH 0011095;%%
and references therein.

\bibitem{Bicomplex}
A.~Dimakis and F.~Muller-Hoissen,
%``Bi-differential calculi and integrable models,''
J.\ Phys.\ A {\bf 33} (2000) 957
[math-ph/9908015];
%%CITATION = MATH-PH 9908015;%%
%\bibitem{Dimakis:1999nq}
%A.~Dimakis and F.~Muller-Hoissen,
%``Bicomplexes, integrable models, and noncommutative geometry,''
Int.\ J.\ Mod.\ Phys.\ B {\bf 14} (2000) 2455
[hep-th/0006005];
%%CITATION = HEP-TH 0006005;%%
%\bibitem{Dimakis:2000tm}
%A.~Dimakis and F.~Muller-Hoissen,
%``Bicomplexes and integrable models,''
J.\ Phys.\ A {\bf 33} (2000) 6579
[nlin.si/0006029];
%%CITATION = NLIN-SI 0006029;%%
%\bibitem{Dimakis:2000mu}
%A.~Dimakis and F.~Muller-Hoissen,
``A noncommutative version of the nonlinear Schroedinger equation,''
hep-th/0007015;
%%CITATION = HEP-TH 0007015;%%
%``Bicomplexes and Backlund transformations,''
J.\ Phys.\ A {\bf 34} (2001) 9163
[nlin.si/0104071].
%%CITATION = NLIN-SI 0104071;%%

\bibitem{GrPe}
M.~T.~Grisaru and S.~Penati,
``The noncommutative sine-Gordon system,''
hep-th/0112246.
%%CITATION = HEP-TH 0112246;%%

\bibitem{ToYu}
K.~Toda and S-J.~Yu,
%``he investigation into the Schwarz Korteweg-de Vries equation and
% the Schwarz derivative in ($2 + 1$) dimensions,''
J.\ Math.\ Phys.\ {\bf 41} (2000) 4747;
%``A study of the construction of equations in (2+1) dimensions,''
J.\ Nonlinear Math.\ Phys.\ Suppl. {\bf 8} (2001) 272;
Inverse \ Problems \ {\bf 17} (2001) 1053 and references therein.

\bibitem{BoSc}
O.~I.~Bogoyavlenskii,
Math.\ USSR-Izv.\ {\bf 34} (1990) 245;
F.~Calogero,
%``A Method To Generate Solvable Nonlinear Evolution Equations,''
Lett.\ Nuovo Cim.\  {\bf 14} (1975) 443;
%%CITATION = NCLTA,14,443;%%
J.~Schiff,
``Integrability of Chern-Simons-Higgs vortex equations and a reduction 
of the selfdual Yang-Mills equations to three-dimensions,''
%CU-TP-493
{\it Presented at NATO Adv. Res. Workshop on Painleve Transcendents, Their 
Asymptotics and Physical Applications, Ste. Adele, Canada, Sep 1990}.
NATO ASI Ser.\ B {\bf 278} (Plenum, 1992) 393.

\bibitem{YTSF}
S-J.~Yu, K.~Toda, N.~Sasa and T.~Fukuyama,
J.\ Phys.\ A {\bf 31} (1998) 3337;
%\bibitem{YTF}
S-J.~Yu, K.~Toda and T.~Fukuyama,
J.\ Phys.\ A {\bf 31} (1998) 10181.

\bibitem{Paniak}
L.~D.~Paniak,
``Exact noncommutative KP and KdV multi-solitons,''
hep-th/0105185.
%%CITATION = HEP-TH 0105185;%%

\bibitem{DiMH}
A.~Dimakis and F.~Muller-Hoissen,
``Noncommutative Korteweg-de-Vries equation,''
hep-th/0007074.
%%CITATION = HEP-TH 0007074;%%

\bibitem{Conj}
M.~J.~Ablowitz and P.~A.~Clarkson,
{\it Solitons, Nonlinear Evolution Equations and Inverse Scattering},
(Cambridge UP, 1991) [ISBN/0-521-38730-2];
%(London Mathematical
% Society lecture note series, 149)}.
L.~J.~Mason and N.~M.~Woodhouse,
{\it Integrability, Self-Duality, and Twistor Theory}
(Oxford UP, 1996)
%(London Mathematical Society monographs, new series: 15)},
[ISBN/0-19-853498-1], and references therein.

\bibitem{Toda}
K.~Toda,
``Extensions of soliton equations to non-commutative $(2+1)$ dimensions,''
to appear in JHEP proceedings of workshop on 
Integrable Theories, Solitons and Duality, Sao Paulo, Brazil, 1-6 July 2002. 

\bibitem{HaTo}
M.~Hamanaka and K.~Toda,
``Noncommutative Burgers equation,''
to appear.

\bibitem{Ward}
R.~S.~Ward,
%``Integrable And Solvable Systems, And Relations Among Them,''
Phil.\ Trans.\ Roy.\ Soc.\ Lond.\ A {\bf 315} (1985) 451;
%%CITATION = PTRSA,A315,451;%%
%\bibitem{Ward2}
%R.~S.~Ward,
``Multidimensional integrable systems,''
Lect.\ Notes.\ Phys.\ {\bf 280} (Springer, 1986) 106;
%{\sf [ISBN/0-387-17925-9]}.
%\bibitem{Ward3}
%R.~S.~Ward,
``Integrable systems in twistor theory,''
in {\it Twistors in Mathematics and Physics}
(Cambridge UP, 1990) 246.
%{\sf [ISBN/0-521-39783-9]}.

\bibitem{Nekrasov2}
N.~A.~Nekrasov,
``Noncommutative instantons revisited,''
hep-th/0010017.
%%CITATION = HEP-TH 0010017;%%

\bibitem{GrNe}
D.~J.~Gross and N.~A.~Nekrasov,
%``Monopoles and strings in noncommutative gauge theory,''
JHEP {\bf 0007} (2000) 034
[hep-th/0005204].
%%CITATION = HEP-TH 0005204;%%

\bibitem{Legare}
M.~Legare,
``Noncommutative generalized NS and super matrix KdV systems from a  
noncommutative version of (anti-)self-dual Yang-Mills equations,''
hep-th/0012077.%;
%%CITATION = HEP-TH 0012077;%%

\bibitem{LPS}
O.~Lechtenfeld, A.~D.~Popov and B.~Spendig,
%``Open N = 2 strings in a B-field background and noncommutative self-dual 
Yang-Mills,''
Phys.\ Lett.\ B {\bf 507} (2001) 317
[hep-th/0012200];
%%CITATION = HEP-TH 0012200;%%
%``Noncommutative solitons in open N = 2 string theory,''
JHEP {\bf 0106} (2001) 011
[hep-th/0103196].
%%CITATION = HEP-TH 0103196;%%

\bibitem{LePo}
O.~Lechtenfeld and A.~D.~Popov,
%``Noncommutative multi-solitons in 2+1 dimensions,''
JHEP {\bf 0111} (2001) 040
[hep-th/0106213];
%%CITATION = HEP-TH 0106213;%%
%``Scattering of noncommutative solitons in 2+1 dimensions,''
Phys.\ Lett.\ B {\bf 523} (2001) 178
[hep-th/0108118].
%%CITATION = HEP-TH 0108118;%%

\bibitem{OoVa}
H.~Ooguri and C.~Vafa,
%``Selfduality And N=2 String Magic,''
Mod.\ Phys.\ Lett.\ A {\bf 5} (1990) 1389;
%%CITATION = MPLAE,A5,1389;%%
%``Geometry of N=2 strings,''
Nucl.\ Phys.\ B {\bf 361} (1991) 469;
%%CITATION = NUPHA,B361,469;%%
%``N=2 heterotic strings,''
Nucl.\ Phys.\ B {\bf 367} (1991) 83.
%%CITATION = NUPHA,B367,83;%%

\end{thebibliography}

\end{document}

