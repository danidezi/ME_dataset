
\documentclass[a4paper,12pt]{article}
%
\usepackage{a4,amsmath}
%
\oddsidemargin -10 pt    
\evensidemargin 10 pt   
\marginparwidth 1 in     
\oddsidemargin 0 in      
\evensidemargin 0 in
\marginparwidth 0.75 in
\textwidth 6.375 true in 
%
\renewcommand{\baselinestretch}{1.1}
\textheight 45\baselineskip
\headsep 0pt
%
\providecommand{\Section}[1]{Section~\ref{#1}}
\providecommand{\Table}[1]{Table~\ref{#1}}
\providecommand{\Figure}[1]{Fig.\ref{#1}}
\providecommand{\eqn}[1]{Eq.(\ref{#1})}
\providecommand{\nl}{\notag\\}
%
\providecommand{\vhi}{\varphi}
\providecommand{\la}{\lambda}
\providecommand{\ve}{\varepsilon}
\providecommand{\srac}[2]{{\textstyle\frac{#1}{#2}}}
\providecommand{\PP}{\mathcal{P}}
\providecommand{\TT}{\mathcal{T}}
\providecommand{\CC}{\mathcal{C}}
%
\usepackage{useful_macros}
\begin{document}
\begin{center}
%
%%%%%%%%%%%%%%%%%%%%%%%%%%%%%%%%% title %%%%%%%%%%%%%%%%%%%%%%%%%%%%%%%%%%%%%%%
{\Huge Comment on}\\\vspace{0.5\baselineskip}
{\Huge `Must a Hamiltonian be hermitian'}

\vspace{2\baselineskip}%%%%%%%%% authors %%%%%%%%%%%%%%%%%%%%%%%%%%%%%%%%%%%%%%
{\Large Andr\'e van Hameren%
        %\footnote{E-mail: {\tt andrevh@inp.demokritos.gr}} 
}

\vspace{0.25\baselineskip}%%%%%% adress %%%%%%%%%%%%%%%%%%%%%%%%%%%%%%%%%%%%%%%
{\it\large Institute of Nuclear Physics, NCSR Demokritos, 15310 Athens, Greece}

\vspace{0.25\baselineskip}%%%%%% Email %%%%%%%%%%%%%%%%%%%%%%%%%%%%%%%%%%%%%%%%
{\tt\large andrevh@inp.demokritos.gr}

\vspace{0.25\baselineskip}%%%%%%% date %%%%%%%%%%%%%%%%%%%%%%%%%%%%%%%%%%%%%%%%
{\large\today}

\renewcommand{\baselinestretch}{1}
\vspace{2\baselineskip}%%%%%%%% abstract %%%%%%%%%%%%%%%%%%%%%%%%%%%%%%%%%%%%%%
{\bf Abstract}\\\vspace{0.5\baselineskip}
\parbox{0.8\linewidth}{\small\hspace{15pt}%
A small comment on the paper with the mentioned title by Carl M.\ Bender, Dorje C.\ Brody and Hugh F.\ Jones.
%
}
\end{center}
\vspace{\baselineskip}

%%%%%%%%%%%%%%%%%%%%%%%%%%%%%%%%%%%%%%%%%%%%%%%%%%%%%%%%%%%%%%%%%%%%%%%%%%%%%%%
%%%%%%%%%%%%%%%%%%%%%%%%%%%%%%%%%%%%%%%%%%%%%%%%%%%%%%%%%%%%%%%%%%%%%%%%%%%%%%%
As argued in \cite{paper}, the eigenfunctions \myHighlight{$\phi_n$}\coordHE{} of the Sturm-Liouville
eigenvalue problem \myHighlight{$(8)$}\coordHE{} are fixed up to a constant phase factor.  
Instead of \myHighlight{$\phi_n$}\coordHE{}, one can also choose
\begin{displaymath}\coord{}\boxMath{
   \psi_n(x) = i^n\phi_n(x)
\;\;,
}{corchetes}{0pt}\coordE{}\end{displaymath}
to be the eigenfunctions under considerations, which satisfy 
\begin{displaymath}\coord{}\boxMath{
   \PP\TT\psi_n(x) = \psi_n^*(-x) = (-i)^n\phi_n^*(-x) 
                   = (-i)^n\phi_n(x) = (-)^n\psi_n(x)
\;\;,
}{corchetes}{0pt}\coordE{}\end{displaymath}
and 
\begin{displaymath}\coord{}\boxMath{
   \delta(x-y) = \sum_n(-)^n\phi_n(x)\phi_n(y) = \sum_n\psi_n(x)\psi_n(y)
\;\;.
}{corchetes}{0pt}\coordE{}\end{displaymath}
This formula, then, looks more familiar than \myHighlight{$(6)$}\coordHE{}. Only if the functions
\myHighlight{$\psi_n$}\coordHE{} are non-real, one usually has a different formula:
\myHighlight{$\sum_n\psi_n^*(x)\psi_n(y) = \delta(x-y)$}\coordHE{}.
The operator \myHighlight{$\CC$}\coordHE{} can be represented by
\begin{displaymath}\coord{}\boxMath{
   \CC(x,y) = \sum_n(-)^n\psi_n(x)\psi_n(y)
\;\;.
}{corchetes}{0pt}\coordE{}\end{displaymath}
It acts on the functions \myHighlight{$\psi_n$}\coordHE{} as \myHighlight{$\CC\psi_n=(-)^n\psi_n$}\coordHE{}, so that they
are invariant under \myHighlight{$\CC\PP\TT$}\coordHE{}:
\begin{displaymath}\coord{}\boxMath{
   \CC\PP\TT\psi_n = \psi_n
\;\;,
}{corchetes}{0pt}\coordE{}\end{displaymath}
and are orthonormal under
\begin{displaymath}\coord{}\boxMath{
   \langle \psi_n|\psi_m \rangle
   = \int[\CC\PP\TT\psi_n(x)]\,\psi_m(x)\,dx
   = \int\psi_n(x)\,\psi_m(x)\,dx
\;\;,
}{corchetes}{0pt}\coordE{}\end{displaymath}
which looks like a {\em real\/} inner product%
\footnote{{\it i.e.\/} \myHighlight{$\langle f|g\rangle=\langle g|f\rangle$}\coordHE{} instead of 
\myHighlight{$\langle f|g\rangle=\langle g|f\rangle^*$}\coordHE{}}. 
So we see that \myHighlight{$\CC\PP\TT$}\coordHE{}-invariance of a Hamiltonian, as introduced in \cite{paper},
is equivalent with the
existence of a set of eigenfunctions that are complete and orthonormal under a
{\em real\/} inner product.  The (complex) inner product in the whole Hilbert
space can be defined by
\begin{displaymath}\coord{}\boxMath{
   \langle f|g\rangle = \sum_nf_n^*g_n
   \quad\textrm{with}\quad
   f_n = \int\psi_n(x)\,f(x)\,dx
   \;\;,\;\;
   g_n = \int\psi_n(x)\,g(x)\,dx
\;\;.
}{corchetes}{0pt}\coordE{}\end{displaymath}

%%%%%%%%%%%%%%%%%%%%%%%%%%%%%%%%%%%%%%%%%%%%%%%%%%%%%%%%%%%%%%%%%%%%%%%%%%%%%%%
\begin{thebibliography}{}
\bibitem{paper} Carl M.\ Bender, Dorje C.\ Brody and Hugh F.\ Jones,
          {\it Must a Hamiltonian be hermitian\/},
          {\tt hep-th/0303005}.
\end{thebibliography} 
%
%%%%%%%%%%%%%%%%%%%%%%%%%%%%%%%%%%%%%%%%%%%%%%%%%%%%%%%%%%%%%%%%%%%%%%%%%%%%%%%

%
\end{document}
\bye
