
\documentclass[a4paper,12pt]{article}
\topmargin 2cm
\sloppy
\textwidth 170mm%185 mm
\textheight 225mm%235mm
\hoffset=-2cm
\voffset=-3cm
\title{
\vspace{-3mm}
\rightline{\small IFUP-TH 2002/41}
\vspace{8mm}
\bf Finite-temperature properties of the supersymmetric generalization of
3D compact QED}
\author{
Dmitri Antonov \thanks{
E-mail: {\tt antonov@df.unipi.it}}
\thanks{Permanent address:
ITEP, B. Cheremushkinskaya 25, RU-117 218 Moscow, Russia.}\\
{\it INFN-Sezione di Pisa, Universit\'a degli studi di Pisa,
Dipartimento di Fisica,}\\
{\it Via Buonarroti, 2 - Ed. B -
I-56127 Pisa, Italy}}

\date{}
\usepackage{useful_macros}
\begin{document}


\maketitle
\vspace{1mm}
\centerline{\bf {Abstract}}
\vspace{3mm}
\noindent
The finite-temperature properties of supersymmetric version of (2+1)D compact QED are explored.
Integration over the dual photino yields an action of the dual photon in the
form of a certain series. Only the first term of this series is of the potential form,
while all the others are some unimportant corrections to the kinetic term.
Disregarding these, we arrive at the action where besides the purely bosonic sector with
monopoles of charge 2, monopoles of charge 1 and 3 are also present yielding
certain corrections to the Debye mass of the dual photon. The finite-temperature analysis
of this action is performed, and the RG trajectories of all the fugacities are found
in the vicinity of the Berezinsky-Kosterlitz-Thouless (BKT) critical point.
It is found that in this critical region, the fugacities \myHighlight{$\xi_n$}\coordHE{}'s of monopoles of charge
\myHighlight{$n=1,2,3$}\coordHE{} are scaled with the correlation radius, \myHighlight{$a$}\coordHE{}, as \myHighlight{$\xi_n\sim a^{-2n^2}$}\coordHE{}.




\vspace{5mm}
\noindent
PACS: 11.10.Wx, 14.80.Hv, 11.10.Kk


\newpage
\section{Introduction. The model.}
Finite-temperature behavior of (2+1)D compact QED has for the first time been analysed in ref.~\cite{nk}.
In that paper, it has been demonstrated that the monopole plasma in the respective dimensionally-reduced theory
undergoes the BKT phase transition~\cite{bkt} to the molecular phase, where the confining properties of the
model, discovered in ref.~\cite{1}, disappear.
It is also interesting to study the finite-temperature behavior of the more general theory, namely of the
(2+1)D Georgi-Glashow model. This theory becomes reduced to compact QED
in the limit of infinitely large Higgs-boson mass, but in addition contains W-bosons which play an important role for the
dynamics of the phase transition~\cite{2}. From the finite-temperature analysis of the (2+1)D Georgi-Glashow model
(reviewed in ref.~\cite{revGG}), it in particular
becomes clear that a lot of new effects open up when one includes into the theory some matter fields~\cite{mat}.
This is the main motivation for the present research. It is devoted to the investigation
of finite-temperature properties of the supersymmetric generalization of (2+1)D compact QED~\cite{ahw} which
contains besides the dual photon also its superpartner, photino. This theory is described by the supersymmetric
sine-Gordon model, whose action in the superfield notation reads~\footnote{For simplicity, we consider the
case of vanishing \myHighlight{$\Theta$}\coordHE{}-parameter, although the inclusion of a nonvanishing \myHighlight{$\Theta$}\coordHE{} is straightforward.}

\begin{equation}\coord{}\boxEquation{
\label{0}
S=\int d^3xd\bar\theta d\theta\left[\frac12(\bar D_\alpha\Phi)(D_\alpha\Phi)-\zeta\cos\left(g_m\Phi\right)\right].
}{
S=\int d^3xd\bar\theta d\theta\left[\frac12(\bar D_\alpha\Phi)(D_\alpha\Phi)-\zeta\cos\left(g_m\Phi\right)\right].
}{ecuacion}\coordE{}\end{equation}
In this equation, the scalar supermultiplet and
the supercovariant derivative have the form~\footnote{We use the convention \myHighlight{$\int d\bar\theta d\theta\bar\theta\theta=1$}\coordHE{}.}

$$\coord{}\boxMath{\Phi({\bf x},\theta)=\chi+2\bar\theta\lambda+2\bar\theta\theta F,~~
D_\alpha=\frac12\frac{\partial}{\partial\bar\theta^\alpha}+\left(\gamma_i\theta\right)_\alpha\partial_i.}{dollar}{0pt}\coordE{}$$
Here, \myHighlight{$\chi$}\coordHE{} is the dual-photon field (real scalar), \myHighlight{$\lambda$}\coordHE{} is the photino field,
which is the two-component real (Majorana) spinor (\myHighlight{$\bar\lambda=i\lambda^T\sigma_2$}\coordHE{}),
\myHighlight{$F$}\coordHE{} is an auxiliary scalar field, and the Euclidean \myHighlight{$\gamma$}\coordHE{}-matrices are
expressed in terms of the Pauli matrices as \myHighlight{$\gamma_1=i\sigma_3$}\coordHE{}, \myHighlight{$\gamma_2=i\sigma_1$}\coordHE{}, \myHighlight{$\gamma_3=-i\sigma_2$}\coordHE{}.
Next, \myHighlight{$g_m$}\coordHE{} is the magnetic coupling of dimensionality \myHighlight{$[{\rm mass}]^{-1/2}$}\coordHE{},
and \myHighlight{$\zeta$}\coordHE{} is the monopole fugacity of dimensionality \myHighlight{$[{\rm mass}]^{2}$}\coordHE{} which is
exponentially small with respect to \myHighlight{$g_m^{-4}$}\coordHE{}~\cite{1, ahw}.

Performing the Taylor expansion of \myHighlight{$\cos\left(g_m\Phi\right)$}\coordHE{} in the Grassmann variables and integrating them out, one
readily rewrites the action~(\ref{0}) in the component notations as

$$\coord{}\boxMath{S=\int d^3x\left[\frac12(\partial_i\chi)^2+\frac{i}{2}\bar\lambda\hat\partial\lambda-\frac12F^2+2g_m\zeta F\sin(g_m\chi)+
V({\bf x})\bar\lambda\lambda\right]\stackrel{\int{\cal D}F}{\longrightarrow}}{dollar}{0pt}\coordE{}$$

$$\coord{}\boxMath{
\stackrel{\int{\cal D}F}{\longrightarrow}
\int d^3x\left[\frac12(\partial_i\chi)^2+\frac{i}{2}\bar\lambda\hat\partial\lambda+V({\bf x})\bar\lambda\lambda-
(g_m\zeta)^2\cos(2g_m\chi)\right],
}{dollar}{0pt}\coordE{}$$
where \myHighlight{$\hat\partial\equiv\gamma_i\partial_i$}\coordHE{} and \myHighlight{$V({\bf x})\equiv -g_m^2\zeta\cos(g_m\chi)$}\coordHE{}. The obtained action
coincides with the one presented in ref.~\cite{ahw}. Integrating further over photino we obtain

$$\coord{}\boxMath{\int {\cal D}\bar\lambda{\cal D}\lambda\exp
\left\{-\int d^3x\left[\frac{i}{2}\bar\lambda\hat\partial\lambda+V({\bf x})\bar\lambda\lambda
\right]\right\}=\exp\left\{\frac12{\rm tr}{\,}\ln\left[\left(\frac{i}{2}\hat\partial+V\right)\left(-\frac{i}{2}\hat\partial+V\right)
\right]\right\}\equiv{\rm e}^{-\frac12{\cal S}},}{dollar}{0pt}\coordE{}$$
where
\myHighlight{${\cal S}={\rm tr}{\,}\int d^3x\int\limits_{0}^{\infty}\frac{d\tau}{\tau}\left<{\bf x}\right|{\rm e}^{-{\cal E}\tau\left(
-\frac14\partial^2+\frac{\sigma}{4}\right)}\left|{\bf x}\right>$}\coordHE{}. In this formula,
\myHighlight{$\frac{\sigma}{4}\equiv\frac{i}{2}\hat\partial V+V^2$}\coordHE{}, and \myHighlight{${\cal E}$}\coordHE{} (the so-called einbein) is an arbitrary constant
which we shall set for convenience to be equal 4. Next, to evaluate \myHighlight{${\cal S}$}\coordHE{}, it is natural to apply the heat-kernel method
(see e.g. ref.~\cite{book}) that leads to the following expression:

$$\coord{}\boxMath{{\cal S}={\rm tr}{\,}\int d^3x\int\limits_{0}^{\infty}\frac{d\tau}{\tau}\int\frac{d^3p}{(2\pi)^3}{\rm e}^{-\tau p^2}
{\rm e}^{\tau\left(\partial^2+2ip_i\partial_i\right)}{\rm e}^{-\tau\sigma}.}{dollar}{0pt}\coordE{}$$
The second exponent in this formula results into the factor 1, since all its other terms yield full derivatives. We obtain

$$\coord{}\boxMath{{\cal S}={\rm tr}{\,}\int d^3x\int\limits_{0}^{\infty}\frac{d\tau}{\tau}\frac{{\rm e}^{-4\tau V^2}}{\tau^{3/2}}
{\rm e}^{-2i\tau\hat\partial V}=8{\,}{\rm tr}{\,}\int d^3x\sum\limits_{n=0}^{\infty}\left(-\frac{i}{2}\right)^n
\frac{\Gamma\left(n-\frac32\right)}{n!}V^{3-2n}\left(\hat\partial V\right)^n}{dollar}{0pt}\coordE{}$$
with ``\myHighlight{$\Gamma$}\coordHE{}'' standing for the Gamma-function. The parameter of this expansion can be estimated as
\myHighlight{$\frac{\left|\partial_iV\right|}{V^2}\sim\tan^2\left(g_m\chi\right)$}\coordHE{}, i.e., the expansion is well convergent in the
weak-field limit \myHighlight{$g_m|\chi|\ll 1$}\coordHE{}. However, obviously only the zeroth term of the expansion does not contain the
derivatives of \myHighlight{$\chi$}\coordHE{}. Other terms, containing
such derivatives, are proportional to some powers of \myHighlight{$g_m^4\zeta$}\coordHE{}.~\footnote{For instance,
the first nonvanishing term of this kind reads

$$\coord{}\boxMath{-\sqrt{\pi}{\,}{\rm tr}{\,}\int d^3x\frac{(\hat\partial V)^2}{V}=
2\sqrt{\pi}\int d^3x\frac{\left(\partial_i V\right)^2}{V}=
2\sqrt{\pi}g_m^4\zeta\int d^3x\tan(g_m\chi)\sin(g_m\chi)(\partial_i\chi)^2.}{dollar}{0pt}\coordE{}$$} Owing to the above-mentioned fact that \myHighlight{$g_m^4\zeta$}\coordHE{}
is an exponentially small quantity, all these terms can be disregarded with respect to the leading kinetic term of the
field \myHighlight{$\chi$}\coordHE{}. Within this approximation, the action of the model takes the form

\begin{equation}\coord{}\boxEquation{
\label{2}
S\simeq\int d^3x\left[\frac12(\partial_i\chi)^2-(g_m\zeta)^2\cos(2g_m\chi)-\frac{32\sqrt{\pi}}{3}(g_m^2\zeta)^3\cos^3(g_m\chi)\right].
}{
S\simeq\int d^3x\left[\frac12(\partial_i\chi)^2-(g_m\zeta)^2\cos(2g_m\chi)-\frac{32\sqrt{\pi}}{3}(g_m^2\zeta)^3\cos^3(g_m\chi)\right].
}{ecuacion}\coordE{}\end{equation}
The Debye mass of the dual photon stemming from this action reads \myHighlight{$m_D\simeq2g_m^2\zeta\left(1+4\sqrt{\pi}g_m^4\zeta\right)$}\coordHE{}. Here
the second term in the brackets, which is due to the \myHighlight{$\cos^3$}\coordHE{}-term in the action, represents the
leading correction to the purely bosonic expression \myHighlight{$m_D\simeq2g_m^2\zeta$}\coordHE{}. Besides that, since \myHighlight{$\cos^3(g_m\chi)=\frac14\left[\cos(3g_m\chi)+
3\cos(g_m\chi)\right]$}\coordHE{}, the appearance of the \myHighlight{$\cos^3$}\coordHE{}-term in the action means that monopoles of charge 1 and 3 (in the units of \myHighlight{$g_m$}\coordHE{})
become present in the plasma of monopoles of charge 2.
However, due to the extra factor \myHighlight{$\zeta$}\coordHE{} at the \myHighlight{$\cos^3$}\coordHE{}-term,
the densities of these monopoles are exponentially small with respect to the
density of monopoles of charge 2. Clearly, this situation is different from the
standard case of (2+1)D compact QED,
where the monopoles of charge 1 have the highest density among monopoles of all possible charges.

The rest of the paper will be devoted to the finite-temperature analysis of the obtained action~(\ref{2}).





\section{RG analysis at finite temperature.}
At finite temperature \myHighlight{$T\equiv1/\beta$}\coordHE{}, one should supply the field \myHighlight{$\chi$}\coordHE{}
with the periodic boundary conditions in the temporal direction, with the period equal to
\myHighlight{$\beta$}\coordHE{}. Because of that, the lines of magnetic field emitted by a monopole cannot cross
the boundary of the one-period region and, at the distances larger than \myHighlight{$\beta$}\coordHE{},
should approach this boundary, going almost parallel to it.
Therefore, monopoles separated by such distances
interact via the 2D Coulomb potential, rather than the 3D one. The minimal average distance
between monopoles in the plasma is that of the monopoles of charge 2. It is of the order of \myHighlight{$(g_m\zeta)^{-2/3}$}\coordHE{},
and therefore at \myHighlight{$T>(g_m\zeta)^{2/3}$}\coordHE{},
the monopole ensemble becomes two-dimensional. On the other hand, the critical temperature of the BKT phase transition
reads \myHighlight{$T_c=8\pi/g_m^2$}\coordHE{},~\footnote{In the (2+1)D Georgi-Glashow model, \myHighlight{$T_c$}\coordHE{} can be rewritten in terms of the
electric coupling, \myHighlight{$g$}\coordHE{}, as \myHighlight{$T_c=\frac{g^2}{2\pi}$}\coordHE{}~\cite{nk}, where \myHighlight{$g$}\coordHE{} is related to \myHighlight{$g_m$}\coordHE{} as \myHighlight{$gg_m=4\pi$}\coordHE{}.}
which is exponentially larger than \myHighlight{$(g_m\zeta)^{2/3}$}\coordHE{}.
Therefore, the idea of dimensional reduction is perfectly applicable
at the temperatures of the order of \myHighlight{$T_c$}\coordHE{}.

The factor \myHighlight{$\beta$}\coordHE{}, appearing in front of the
action~(\ref{2}) upon the dimensional reduction,
can be removed [and the action can be cast to the original form~(\ref{2}) with the substitution \myHighlight{$d^3x\to d^2x$}\coordHE{}]
by the obvious rescaling \myHighlight{$\chi_{\rm new}=\sqrt{\beta}\chi$}\coordHE{}. We thus arrive at the following dimensionally-reduced theory:
\myHighlight{$S_{\rm d.-r.}[\chi]=\int d^2x\left[\frac12(\partial_i\chi)^2-\sum\limits_{n=1}^{3}\xi_n\cos\left(n\sqrt{K}\chi\right)\right]$}\coordHE{},
where \myHighlight{$K\equiv g_m^2T$}\coordHE{}, \myHighlight{$\xi_1\equiv8\sqrt{\pi}\beta(g_m^2\zeta)^3$}\coordHE{}, \myHighlight{$\xi_2\equiv\beta(g_m\zeta)^2$}\coordHE{}, and
\myHighlight{$\xi_3\equiv\frac{8\sqrt{\pi}}{3}\beta(g_m^2\zeta)^3$}\coordHE{}. Note once more that the BKT critical temperature, \myHighlight{$T_c$}\coordHE{}, is determined
by the term \myHighlight{$\cos\left(\sqrt{K}\chi\right)$}\coordHE{}. This temperature, at which monopoles of charge 1 bind into molecules,
is larger than such temperatures for monopoles of charges \myHighlight{$n=2,3$}\coordHE{}, equal to \myHighlight{$8\pi/(ng_m)^2$}\coordHE{}.~\footnote{
Note that if our theory were considered as being originated from some extension of the (2+1)D Georgi-Glashow model,
possessing charged matter fields, the true phase transition would be determined by the equality of density of these
fields to that of monopoles~\cite{2}.
Since \myHighlight{$\xi_1$}\coordHE{} is exponentially smaller than \myHighlight{$\xi_2$}\coordHE{} (i.e., the density of monopoles of charge 2
is the highest one), such a phase transition would then be determined by the
monopoles of charge 2, rather than 1.}

In what follows, we shall adapt the usual RG strategy~\cite{kogut} based on the integration over the
high-frequency modes.
Splitting the momenta into two ranges, \myHighlight{$0<p<\Lambda'$}\coordHE{} and \myHighlight{$\Lambda'<p<\Lambda$}\coordHE{}, one can then define these
modes as \myHighlight{$h=\chi_\Lambda-\chi_{\Lambda'}$}\coordHE{}, where \myHighlight{$\chi_{\Lambda'}({\bf x})=
\int\limits_{0<p<\Lambda'}^{}\frac{d^2p}{(2\pi)^2}{\rm e}^{i{\bf p}{\bf x}}\chi({\bf p})$}\coordHE{} and
consequently, \myHighlight{$h({\bf x})=\int\limits_{\Lambda'<p<\Lambda}^{}\frac{d^2p}{(2\pi)^2}{\rm e}^{i{\bf p}{\bf x}}\chi({\bf p})$}\coordHE{}.
The partition function,
\myHighlight{${\cal Z}_\Lambda=\int\limits_{0<p<\Lambda}^{}{\cal D}\chi({\bf p})
{\rm e}^{-S_{\rm d.-r.}\left[\chi_\Lambda\right]}$}\coordHE{},
can be rewritten as
\myHighlight{${\cal Z}_\Lambda=\int\limits_{0<p<\Lambda'}^{}{\cal D}\chi({\bf p})
{\rm e}^{-\frac12\int d^2x\left(\partial_i\chi_{\Lambda'}\right)^2}{\cal Z}'$}\coordHE{}, where
\myHighlight{${\cal Z}'=\left<\exp\left\{
\int d^2x\left[\sum\limits_{n=1}^{3}\xi_n
\cos\left(n\sqrt{K}\chi_\Lambda\right)\right]\right\}\right>_h$}\coordHE{}
and \myHighlight{$\left<{\cal O}\right>_h\equiv\frac{\int\limits_{\Lambda'<p<\Lambda}^{}{\cal D}\chi({\bf p})
\exp\left[-\frac12\int d^2x(\partial_ih)^2\right]{\cal O}}{\int\limits_{\Lambda'<p<\Lambda}^{}{\cal D}\chi({\bf p})
\exp\left[-\frac12\int d^2x(\partial_ih)^2\right]}$}\coordHE{}. Next, because of the \myHighlight{$\cos\left(3\sqrt{K}\chi_\Lambda\right)$}\coordHE{}-term,
in the course of average over \myHighlight{$h$}\coordHE{}, one should keep the third irreducible average (also called cumulant) of
\myHighlight{$\cos\left(\sqrt{K}\chi_\Lambda\right)$}\coordHE{}. It is defined by the formula

\begin{equation}\coord{}\boxEquation{
\label{fff}
\left<\left<f_1f_2f_3\right>\right>=\left<f_1f_2f_3\right>-\left<\left<f_1f_2\right>\right>\left<f_3\right>-
\left<\left<f_1f_3\right>\right>\left<f_2\right>-\left<\left<f_2f_3\right>\right>\left<f_1\right>-
\left<f_1\right>\left<f_2\right>\left<f_3\right>,
}{
\left<\left<f_1f_2f_3\right>\right>=\left<f_1f_2f_3\right>-\left<\left<f_1f_2\right>\right>\left<f_3\right>-
\left<\left<f_1f_3\right>\right>\left<f_2\right>-\left<\left<f_2f_3\right>\right>\left<f_1\right>-
\left<f_1\right>\left<f_2\right>\left<f_3\right>,
}{ecuacion}\coordE{}\end{equation}
where \myHighlight{$\left<\left<f_1f_2\right>\right>=\left<f_1f_2\right>-\left<f_1\right>\left<f_2\right>$}\coordHE{}
and in our case \myHighlight{$f_i\equiv\cos\left(\sqrt{K}\chi_\Lambda({\bf x}_i)\right)$}\coordHE{}, \myHighlight{$\left<\ldots\right>\equiv\left<\ldots\right>_h$}\coordHE{}.
Denoting for brevity \myHighlight{$\sqrt{K}\chi_{\Lambda'}({\bf x})$}\coordHE{} by \myHighlight{$\phi_{\bf x}$}\coordHE{} we first obtain for a single average and for the second cumulant
(cf. ref.~\cite{kogut}):

\begin{equation}\coord{}\boxEquation{
\label{cos}
\left<\cos\left(\sqrt{K}\chi_\Lambda({\bf x})\right)\right>=D(0)\cos\phi_{\bf x},
}{
\left<\cos\left(\sqrt{K}\chi_\Lambda({\bf x})\right)\right>=D(0)\cos\phi_{\bf x},
}{ecuacion}\coordE{}\end{equation}

$$\coord{}\boxMath{
\left<\left<\cos\left(\sqrt{K}\chi_\Lambda({\bf x})\right)\cos\left(\sqrt{K}\chi_\Lambda({\bf y})\right)\right>\right>=
}{dollar}{0pt}\coordE{}$$

\begin{equation}\coord{}\boxEquation{
\label{cc}
=\frac{D^2(0)}{2}\left\{\left[D^2({\bf x}-{\bf y})-1\right]\cos\left(\phi_{\bf x}+\phi_{\bf y}\right)+
\left[D^{-2}({\bf x}-{\bf y})-1\right]\cos\left(\phi_{\bf x}-\phi_{\bf y}\right)\right\},
}{
=\frac{D^2(0)}{2}\left\{\left[D^2({\bf x}-{\bf y})-1\right]\cos\left(\phi_{\bf x}+\phi_{\bf y}\right)+
\left[D^{-2}({\bf x}-{\bf y})-1\right]\cos\left(\phi_{\bf x}-\phi_{\bf y}\right)\right\},
}{ecuacion}\coordE{}\end{equation}
where \myHighlight{$D({\bf x})\equiv\exp\left[-\frac{K}{2}\int\limits_{\Lambda'<p<\Lambda}^{}\frac{d^2p}{(2\pi)^2}
\frac{{\rm e}^{i{\bf p}{\bf x}}}{p^2}\right]$}\coordHE{}. In the same way, we obtain the following expression for the
three-local average:


$$\coord{}\boxMath{
\left<\cos\left(\sqrt{K}\chi_\Lambda({\bf x})\right)\cos\left(\sqrt{K}\chi_\Lambda({\bf y})\right)
\cos\left(\sqrt{K}\chi_\Lambda({\bf z})\right)\right>=\frac{D^3(0)}{4}\Bigl[D^2({\bf x}-{\bf y})
D^2({\bf x}-{\bf z})D^2({\bf y}-{\bf z})\times}{dollar}{0pt}\coordE{}$$

$$\coord{}\boxMath{\times\cos\left(\phi_{\bf x}+\phi_{\bf y}+\phi_{\bf z}\right)+D^{-2}({\bf x}-{\bf y})
D^{-2}({\bf x}-{\bf z})D^2({\bf y}-{\bf z})\cos\left(\phi_{\bf x}-\phi_{\bf y}-\phi_{\bf z}\right)+
D^{-2}({\bf x}-{\bf y})
D^2({\bf x}-{\bf z})\times}{dollar}{0pt}\coordE{}$$


\begin{equation}\coord{}\boxEquation{
\label{ccc}
\times D^{-2}({\bf y}-{\bf z})\cos\left(\phi_{\bf x}-\phi_{\bf y}+\phi_{\bf z}\right)+D^2({\bf x}-{\bf y})
D^{-2}({\bf x}-{\bf z})D^{-2}({\bf y}-{\bf z})\cos\left(\phi_{\bf x}+\phi_{\bf y}-\phi_{\bf z}\right)\Bigr].
}{
\times D^{-2}({\bf y}-{\bf z})\cos\left(\phi_{\bf x}-\phi_{\bf y}+\phi_{\bf z}\right)+D^2({\bf x}-{\bf y})
D^{-2}({\bf x}-{\bf z})D^{-2}({\bf y}-{\bf z})\cos\left(\phi_{\bf x}+\phi_{\bf y}-\phi_{\bf z}\right)\Bigr].
}{ecuacion}\coordE{}\end{equation}
The third cumulant can now be readily evaluated by virtue of eqs.~(\ref{fff})-(\ref{ccc}). The expression for it is, however,
rather lengthy and for the sake of shortness
we shall not present it here. It is only important that similarly to eq.~(\ref{cc}), the third cumulant
vanishes when the distance between any two points in it grows. This fact enables one to approximate, for example,
\myHighlight{$\cos\left(\phi_{\bf x}+\phi_{\bf y}-\phi_{\bf z}\right)$}\coordHE{} by \myHighlight{$\cos\phi_{\bf u}$}\coordHE{}~\footnote{The next term, \myHighlight{$-\frac12\left[
({\bf x}-{\bf z})_i\partial_i\phi_{\bf u}\right]^2\cos\phi_{\bf u}$}\coordHE{}, may be omitted, since it results into the exponentially small
correction to the kinetic term, analogous to the one discussed in footnote~3.} and
\myHighlight{$\cos\left(\phi_{\bf x}+\phi_{\bf y}+\phi_{\bf z}\right)$}\coordHE{} by \myHighlight{$\cos\left(3\phi_{\bf u}\right)$}\coordHE{}, where
\myHighlight{${\bf u}$}\coordHE{} is the center of mass of the triangle with the vertices \myHighlight{${\bf x}$}\coordHE{}, \myHighlight{${\bf y}$}\coordHE{}, and \myHighlight{${\bf z}$}\coordHE{}.
The partition function \myHighlight{${\cal Z}'$}\coordHE{} then takes the form

$$\coord{}\boxMath{
{\cal Z}'\simeq\exp\Biggl\{-\int d^2x\Biggl[\xi_1D(0)\left(1+\frac{\xi_1^2\alpha_1}{24}D^2(0)\right)\cos\phi_{\bf x}+D^2(0)
\left(\xi_2D^2(0)-\frac{\xi_1^2\alpha_2}{4}\right)\cos(2\phi_{\bf x})+}{dollar}{0pt}\coordE{}$$

$$\coord{}\boxMath{+D^3(0)\left(\xi_3D^6(0)+\frac{\xi_1^3\alpha_3}{24}\right)\cos(3\phi_{\bf x})-
\int d^2y\left(D^{-2}({\bf y})-1\right)\left[1-\frac{K^2}{2}\left(y_i\partial_i\phi_{\bf x}\right)^2\right]\Biggr]\Biggr\},}{dollar}{0pt}\coordE{}$$
where

$$\coord{}\boxMath{\alpha_1=\int d^2yd^2z\Bigl\{D^2({\bf y})\Bigl[D^{-2}({\bf z})D^{-2}({\bf y}-{\bf z})-1\Bigr]+
D^{-2}({\bf y})\Bigl[D^2({\bf z})D^{-2}({\bf y}-{\bf z})+}{dollar}{0pt}\coordE{}$$

$$\coord{}\boxMath{+D^{-2}({\bf z})D^2({\bf y}-{\bf z})-2\Bigr]+7-
D^2({\bf z})-2D^{-2}({\bf z})-2D^2({\bf y}-{\bf z})-2D^{-2}({\bf y}-{\bf z})\Bigr\},}{dollar}{0pt}\coordE{}$$

$$\coord{}\boxMath{
\alpha_2=\int d^2y\left[D^2({\bf y})-1\right],~~
\alpha_3=\int d^2yd^2z\Bigl\{D^2({\bf y})\left[D^2({\bf z})D^2({\bf y}-{\bf z})-1\right]+1-D^2({\bf z})\Bigr\}.}{dollar}{0pt}\coordE{}$$
The constants \myHighlight{$\alpha_{1,3}$}\coordHE{} here stem from the third cumulant. Taking \myHighlight{$\Lambda'=\Lambda-d\Lambda$}\coordHE{}, we readily obtain
\myHighlight{$\alpha_1=-\alpha_3={\cal A}K\frac{d\Lambda}{\Lambda^3}\left[\alpha+{\cal O}\left(\frac{d\Lambda}{\Lambda}\right)\right]$}\coordHE{}. Here,
\myHighlight{${\cal A}$}\coordHE{} is the (fixed) area of the system, and \myHighlight{$\alpha$}\coordHE{}
stands for some momentum-space-slicing dependent positive constant,
whose concrete value will turn out to be unimportant for the final expressions describing the
RG flow. In the same way, we have \myHighlight{$\alpha_2=-\alpha Kd\Lambda/\Lambda^3$}\coordHE{}. Taking further into account that
to the leading order in \myHighlight{$d\Lambda/\Lambda$}\coordHE{}, \myHighlight{$D(0)\simeq 1-\frac{K}{4\pi}\frac{d\Lambda}{\Lambda}$}\coordHE{}, we obtain the
following renormalizations of fugacities:

$$\coord{}\boxMath{
d\xi_1=\frac{\xi_1K}{4}\left(-\frac{1}{\pi}+\frac{\alpha{\cal A}\xi_1^2}{6\Lambda^2}\right)\frac{d\Lambda}{\Lambda},~~
d\xi_2=K\left(-\frac{\xi_2}{\pi}+\frac{\alpha \xi_1^2}{4\Lambda^2}\right)\frac{d\Lambda}{\Lambda},~~
d\xi_3=-\frac{K}{4}\left(\frac{9\xi_3}{\pi}+\frac{\alpha{\cal A}\xi_1^3}{6\Lambda^2}\right)\frac{d\Lambda}{\Lambda},}{dollar}{0pt}\coordE{}$$
where the terms higher in \myHighlight{$1/\Lambda$}\coordHE{} than those presented in the brackets have been disregarded.
The field itself becomes renormalized as~\footnote{The respective RG equation obviously reads \myHighlight{$d\chi=\frac{\beta}{2}
(K\xi_1)^2\chi d\Lambda/\Lambda^5$}\coordHE{}.}

$$\coord{}\boxMath{\chi_{\Lambda'}^{\rm new}=\sqrt{1+\beta\left(K\xi_1D(0)\right)^2\frac{d\Lambda}{\Lambda^5}}\chi_{\Lambda'}\simeq
\left(1+\frac{\beta}{2}(K\xi_1)^2\frac{d\Lambda}{\Lambda^5}\right)\chi_{\Lambda'},}{dollar}{0pt}\coordE{}$$
where \myHighlight{$\beta$}\coordHE{} is another (analogous to \myHighlight{$\alpha$}\coordHE{})
momentum-space-slicing dependent positive constant. Finally, the renormalizations of the coupling \myHighlight{$K$}\coordHE{} and of the free-energy density
\myHighlight{$F\equiv-\frac{\ln{\cal Z}'}{{\cal A}}$}\coordHE{} have the form:

$$\coord{}\boxMath{
dK=-\beta K^3\xi_1^2\frac{d\Lambda}{\Lambda^5}\left[1+{\cal O}\left(\frac{d\Lambda}{\Lambda}\right)\right],~~
dF=-\frac{\alpha}{4}K\xi_1^2\frac{d\Lambda}{\Lambda^3}\left[1+{\cal O}\left(\frac{d\Lambda}{\Lambda}\right)\right],
}{dollar}{0pt}\coordE{}$$
where the last terms in the brackets will further be disregarded.

Following the notations of ref.~\cite{kogut}, we shall further make the change of variables from the momentum scale
to the real-space one:
\myHighlight{$\Lambda\to a\equiv1/\Lambda$}\coordHE{}, \myHighlight{$d\Lambda\to-d\Lambda$}\coordHE{}, that obviously modifies the above RG equations as

$$\coord{}\boxMath{
d\xi_1=-\frac{\xi_1K}{4\pi}\frac{da}{a}\left(1-\frac{\pi\alpha{\cal A}(a\xi_1)^2}{6}\right),~~
d\xi_2=-\frac{K}{\pi}\frac{da}{a}\left(\xi_2-\frac{\pi\alpha (a\xi_1)^2}{4}\right),}{dollar}{0pt}\coordE{}$$

$$\coord{}\boxMath{d\xi_3=-\frac{9K}{4\pi}\frac{da}{a}\left(\xi_3+\frac{\pi\alpha{\cal A}a^2\xi_1^3}{54}\right),~~
dK=-\beta \xi_1^2(aK)^3da,~~ dF=\frac{\alpha}{4}K\xi_1^2ada.}{dollar}{0pt}\coordE{}$$

Our aim below is to obtain the RG flow in the vicinity of the BKT critical point~\cite{bkt, kogut, nk} \myHighlight{$K_c=8\pi$}\coordHE{}, \myHighlight{$z_{1{\,}c}=0$}\coordHE{}, where
\myHighlight{$z_1\equiv \xi_1a^2$}\coordHE{}. [The value of \myHighlight{$K_c$}\coordHE{} here stems, in particular, from the
value of \myHighlight{$T_c$}\coordHE{}, at which monopoles of charge 1 bind into molecules (cf. the first two paragraphs of this Section).]
Substituting this value of \myHighlight{$K_c$}\coordHE{} into the equation for \myHighlight{$\xi_1$}\coordHE{}, we see that in the critical region this fugacity scales
as \myHighlight{$\xi_1\sim a^{-2}$}\coordHE{},~\footnote{In particular, substituting the
behavior \myHighlight{$\xi_1\sim a^{-2}$}\coordHE{} into the equation for the free-energy density, we see that the latter scales
in the vicinity of the critical point as \myHighlight{$F\sim a^{-2}$}\coordHE{}, i.e, remains continuous.}
and the term \myHighlight{$\sim(a\xi_1)^2$}\coordHE{} in the brackets is of the order of \myHighlight{$a^{-2}$}\coordHE{}. At the same time,
in the molecular phase, the correlation
radius \myHighlight{$a$}\coordHE{} becomes infinite due to the short-rangeness of the molecular fields, and it diverges at \myHighlight{$T\to T_c-0$}\coordHE{}~\footnote{The
functional form of this divergence will be discussed below.}. Therefore, the above-mentioned term \myHighlight{$\sim(a\xi_1)^2$}\coordHE{} in the
RG equation for \myHighlight{$\xi_1$}\coordHE{} may be disregarded in the critical region, as well as the analogous term in the equation for \myHighlight{$\xi_2$}\coordHE{}
and the term \myHighlight{$\sim a^2\xi_1^3$}\coordHE{} in the equation for \myHighlight{$\xi_3$}\coordHE{}. In what follows, we shall adapt this approximation, that is
equivalent to the neglection of the above-described contribution of the third cumulant to \myHighlight{$\xi_n$}\coordHE{}'s.
The scaling behaviors of \myHighlight{$\xi_2$}\coordHE{} and \myHighlight{$\xi_3$}\coordHE{} in the critical region then read \myHighlight{$\xi_2\sim a^{-8}$}\coordHE{} and
\myHighlight{$\xi_3\sim a^{-18}$}\coordHE{}. One can write down the following general RG equation \myHighlight{$d\ln\xi_n=-n^2
\frac{K}{4\pi}d\ln a$}\coordHE{}, that in the vicinity of the critical point yields the scaling law \myHighlight{$\xi_n\sim a^{-2n^2}$}\coordHE{}.
We thus see that the absolute value of power of the correlation radius, with which the
fugacity of monopoles of a certain charge vanishes towards the critical point, grows as the squared charge of respective monopoles.
This is a formal expression for a heuristic fact that
at a given temperature, the fugacity of monopoles of charge \myHighlight{$n$}\coordHE{} is smaller for a larger \myHighlight{$n$}\coordHE{},
since the phase transitions occur in the
sequence \myHighlight{$n=3$}\coordHE{}, \myHighlight{$n=2$}\coordHE{}, and \myHighlight{$n=1$}\coordHE{}.


It is further convenient to introduce instead of \myHighlight{$K$}\coordHE{} the new coupling \myHighlight{$x=2-\frac{K}{4\pi}$}\coordHE{}, which is positive at \myHighlight{$T<T_c$}\coordHE{} and
vanishes at \myHighlight{$K=K_c$}\coordHE{}. Performing then rescalings \myHighlight{$a_{\rm new}=2(8\pi^2\beta)^{1/4}a$}\coordHE{}, \myHighlight{$z_{1{\,}\rm new}=\xi_1a_{\rm new}^2$}\coordHE{}, we obtain
equations \myHighlight{$dx=z_1^2da/a$}\coordHE{}, \myHighlight{$dz_1^2=2z_1^2xda/a$}\coordHE{}. They yield the BKT-type RG flow \myHighlight{$z_1^2-x^2=\tau$}\coordHE{}, where
\myHighlight{$\tau\propto(T_c-T)/T_c$}\coordHE{} is some constant.
In particular, \myHighlight{$x\simeq z_1$}\coordHE{} at \myHighlight{$T\to T_c-0$}\coordHE{}. Owing to the above equation for \myHighlight{$z_1^2$}\coordHE{}, this relation yields
\myHighlight{$z_{1{\,}(\rm in)}^{-1}-z_1^{-1}=\ln\left(a/a_{(\rm in)}\right)$}\coordHE{}, where the subscript ``\myHighlight{${\,}{}_{(\rm in)}{\,}$}\coordHE{}''
means the initial value.
Taking into account that \myHighlight{$z_{1{\,}(\rm in)}$}\coordHE{} is exponentially small, while
\myHighlight{$z_1\sim 1$}\coordHE{} [the value at which the growth of \myHighlight{$z_1(a)$}\coordHE{} stops], we obtain in the case \myHighlight{$x_{(\rm in)}\le\sqrt{\tau}$}\coordHE{}:
\myHighlight{$\ln\left(a/a_{(\rm in)}\right)\sim z_{1{\,}(\rm in)}^{-1}\sim\tau^{-1/2}$}\coordHE{}. According to this relation,
at \myHighlight{$T\to T_c-0$}\coordHE{}, the correlation radius diverges with an essential singularity as
\myHighlight{$a(\tau)\sim\exp\left({\rm const}/\sqrt{\tau}\right)$}\coordHE{} with \myHighlight{${\rm const}{\,}>0$}\coordHE{}.

In order to describe the RG flows of \myHighlight{$\xi_n$}\coordHE{}', \myHighlight{$n=2,3$}\coordHE{}, it is natural to introduce the dimensionless
quantities analogous to \myHighlight{$z_1$}\coordHE{}, \myHighlight{$z_n\equiv\xi_na^2$}\coordHE{}, which obey the equations
\myHighlight{$d\ln z_n^2=4(1-n^2)d\ln a$}\coordHE{}. Taking into account that \myHighlight{$dx\simeq\tau d\ln a$}\coordHE{} at \myHighlight{$x\ll 1$}\coordHE{},
we obtain in this region the following equation: \myHighlight{$\frac{dz_n^2}{dx}=\frac{4(1-n^2)}{\tau}z_n^2$}\coordHE{}.
The respective RG flow has the form \myHighlight{$z_n=C_n{\rm e}^{-\frac{2(n^2-1)}{\tau}x}$}\coordHE{}
with \myHighlight{$C_n$}\coordHE{}'s being the positive integration constants. We see that
in any of the planes \myHighlight{$(x, z_2)$}\coordHE{} and
\myHighlight{$(x, z_3)$}\coordHE{}, the line \myHighlight{$\tau=0$}\coordHE{} plays the same role as it plays in the \myHighlight{$(x, z_1)$}\coordHE{}-plane. Namely, this line [coinciding in the
\myHighlight{$(x, z_{2,3})$}\coordHE{}-planes with the
line \myHighlight{$x=0$}\coordHE{}] is the separatrix
between two distinct classes of RG trajectories. First of them are
those with \myHighlight{$\tau<0$}\coordHE{}, which decrease towards \myHighlight{$C_n$}\coordHE{} with the
decrease of \myHighlight{$x$}\coordHE{}, while the second ones, corresponding to \myHighlight{$\tau>0$}\coordHE{}, increase from zero to
\myHighlight{$C_n$}\coordHE{}. At \myHighlight{$x=0$}\coordHE{}, both \myHighlight{$z_{2,3}$}\coordHE{} have nonvanishing values \myHighlight{$C_{2,3}$}\coordHE{}, that is similar to \myHighlight{$z_1$}\coordHE{},
\myHighlight{$z_1(x=0)=\sqrt{\tau}$}\coordHE{}. As it has been discussed, these values vanish
at \myHighlight{$K\to K_c$}\coordHE{} according to the law \myHighlight{$C_n\sim a^{2(1-n^2)}$}\coordHE{}.

In conclusion of this letter, the dual photino present in the supersymmetric version of (2+1)D compact QED
leads to the increase of the Debye mass of the dual photon by means of the appearance of
the admixture of monopoles of charge 1 and 3 in the plasma of monopoles of charge 2 (the latter correspond
to the purely bosonic sector of the model). At finite temperature, the BKT phase transitions
of monopoles of charge \myHighlight{$n$}\coordHE{} occur in the sequence \myHighlight{$n=3$}\coordHE{}, \myHighlight{$n=2$}\coordHE{}, \myHighlight{$n=1$}\coordHE{}. In the critical region [corresponding
to the (\myHighlight{$n=1$}\coordHE{})-phase transition], the fugacities \myHighlight{$\xi_n$}\coordHE{}'s are found to vanish as
\myHighlight{$\xi_n\sim a^{-2n^2}$}\coordHE{}, where the correlation radius \myHighlight{$a$}\coordHE{} diverges with an essential singularity,
\myHighlight{$a(\tau)\sim\exp\left({\rm const}/\sqrt{\tau}\right)$}\coordHE{}. This scaling of \myHighlight{$\xi_n$}\coordHE{}'s is in accordance with
the above sequence of the phase transitions.
When the parameter \myHighlight{$x$}\coordHE{} (measuring the distance to the
critical point) is much smaller than unity, the dimensionless fugacity \myHighlight{$z_1=\xi_1a^2$}\coordHE{} scales according to the standard BKT RG flow
as \myHighlight{$z_1^2-x^2=\tau$}\coordHE{}. At the same time, the analogous fugacities for monopoles of higher charges, \myHighlight{$z_n=\xi_na^2$}\coordHE{}, \myHighlight{$n=2,3$}\coordHE{},
behave as \myHighlight{$z_n=C_n{\rm e}^{-\frac{2(n^2-1)}{\tau}x}$}\coordHE{} with \myHighlight{$C_n\sim a^{2(1-n^2)}$}\coordHE{}. For these monopoles (similarly to the case \myHighlight{$n=1$}\coordHE{}),
the line \myHighlight{$\tau=0$}\coordHE{} is the separatrix
between the RG trajectories corresponding to \myHighlight{$\tau<0$}\coordHE{} (which decrease towards \myHighlight{$C_n$}\coordHE{} with the
decrease of \myHighlight{$x$}\coordHE{}) and those corresponding to \myHighlight{$\tau>0$}\coordHE{} (which increase from zero to
\myHighlight{$C_n$}\coordHE{} with the decrease of \myHighlight{$x$}\coordHE{}).



\section{Acknowledgments.}
The author is grateful
to Prof. A.~Di~Giacomo for useful discussions and to Dr. A.~Kovner for informing him about the existence of
ref.~\cite{ahw}.
He is also grateful to
Prof. A.~Di~Giacomo and to the whole staff of the Physics Department of the
University of Pisa for cordial hospitality.
The work has been supported by INFN and partially by
the INTAS grant Open Call 2000, Project No. 110.




\begin{thebibliography}{100}
\bibitem{nk}
N.O. Agasyan and K. Zarembo, Phys. Rev. {\bf D 57} (1998) 2475.
%%CITATION = PHRVA,D57,2475;%%

\bibitem{bkt}
V.L. Berezinsky, Sov. Phys.- JETP {\bf 32} (1971) 493;
J.M. Kosterlitz and D.J. Thouless, J. Phys. {\bf C 6}
(1973) 1181; J.M. Kosterlitz, J. Phys. {\bf C 7} (1974) 1046.
%%CITATION = JTPHE,32,493;%%
%%CITATION = JPCBA,6,1181;%%
%%CITATION = JPCBA,7,1046;%%

\bibitem{1}A.M. Polyakov, Nucl. Phys. {\bf B 120} (1977) 429.
%%CITATION = NUPHA,B120,429;%%

\bibitem{2}
G. Dunne, I.I. Kogan, A. Kovner, and B. Tekin, JHEP {\bf 01} (2001) 032.
%%CITATION = JHEPA,0101,032;%%

\bibitem{revGG}
I.I. Kogan and A. Kovner, hep-th/0205026; D. Antonov, hep-th/0207224.
%%CITATION = HEP-TH 0205026;%%
%%CITATION = HEP-TH 0207224;%%

\bibitem{mat}
N.O. Agasian and D. Antonov, Phys. Lett. {\bf B 530} (2002) 153;
G.V. Dunne, A. Kovner, and Sh.M. Nishigaki, Phys. Lett. {\bf B 544} (2002) 215.
%%CITATION = HEP-TH 0109189;%%
%%CITATION = HEP-TH 0207049;%%

\bibitem{ahw}
I. Affleck, J. Harvey, and E. Witten, Nucl. Phys. {\bf B 206} (1982) 413.
%%CITATION = NUPHA,B206,413;%%

\bibitem{book}
J.F. Donoghue, E. Golowich, and B.R. Holstein, {\it Dynamics of the Standard Model}
(Cambridge Univ. Press, Cambridge, 1992).


\bibitem{kogut}
J.B. Kogut, Rev. Mod. Phys. {\bf 51} (1979) 659;
B. Svetitsky and L.G. Yaffe, Nucl. Phys. {\bf B 210 [FS6]} (1982) 423.
%%CITATION = RMPHA,51,659;%%
%%CITATION = NUPHA,B210,423;%%




\end{thebibliography}
\end{document}

























\bye
