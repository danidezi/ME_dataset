
\documentclass[a4paper,12pt]{article}
\hoffset -1.5cm 
\voffset -2cm 
\textheight 23cm 
\textwidth 15.5cm
\begin{document}
\title{\huge \bf Observable Algebra}
\author{{\bf Merab Gogberashvili} \\
Andronikashvili Institute of Physics \\
6 Tamarashvili Str.,Tbilisi 380077, Georgia \\
{\sl E-mail: gogber@hotmail.com }}
\maketitle
\begin{abstract}
A physical applicability of normed split-algebras, such as hyperbolic numbers, 
split-quaternions and split-octonions is considered. We argue that the observable 
geometry can be described by the algebra of split-octonions, which is naturally 
equipped by zero divisors (the elements of split-algebras corresponding to zero 
norm). In such a picture physical phenomena are described by the ordinary elements 
of chosen algebra, while the zero divisors give raise the coordinatization and 
two fundamental constants, namely velocity of light and Planck constant. It turns 
to be possible that uncertainty principle appears from the condition of positively 
defined norm, and has the same geometrical meaning as the existence of the maximal 
value of speed. The property of non-associativity of octonions could correspond to 
the appearance of fundamental probabilities in physics. Grassmann elements and a 
non-commutativity of space coordinates, which are widely used in various physical 
theories, appear naturally in our approach. 
\end{abstract}
\vskip 0.3cm 
{\sl PACS numbers: 01.55.+b; 02.10.De; 11.10.Kk}
\vskip 0.5cm
%%%%%%%%%%%%%%%%%%%%%%%%%%%%%%%%%%%%%%%%%%%%%%%%%%%%%%%%%%%%%

\section{Introduction}

Real physical phenomena always expose themselves experimentally as a set of 
measured numbers. The theory of representation of algebra serves as a tool, which 
gives a possibility to interpret abstract mathematical quantities from the point of 
view of the observable reality. The geometry of space-time being the main 
characteristic of the reality can be also described in the language of some 
algebras and symmetries. Usually in physics the geometry is thought to be 
objective without any connection to the way it gets observed. In other words, 
it is assumed the existence of some massive objects, with respect to which 
physical events can be numerated. However, geometry cannot be introduced 
independently. Results of any observation depend not only upon the phenomena 
we study but also on the way we make the observation and on the algebra one 
uses for a coordinatization. 

Let us start from the assumption that physical quantities must correspond to 
elements of some algebra, so they can be summed and multiplied. Since all 
observable quantities we know are real it is possible to restrict our self with 
a real algebra. To have a transition from a manifold consisted of elements being 
the results of measurements to geometry one must be able to introduce a distance 
between two objects. Introduction of a distance always means some comparison of 
two objects using one of them as an etalon. Thus we need algebra with a unit 
element. 

In the algebraic language all above requirements mean that to describe the geometry 
of the real world we need the normed composition algebra with unit elements over 
the real numbers. Besides of the usual algebra of real numbers according to the 
Hurwitz theorem there are three extraordinary and unique algebras satisfied the 
required conditions, namely the algebra of complex numbers, the algebra of 
quaternions and the octonions algebra \cite{Sc}. 

Essential for all these algebras is the existence of the real unit element $e_0$ and 
different number of adjoined hyper-complex units $e_n$. For the case of complex number 
$n = 1$, for quaternions $n = 3$ and for octonions $n = 7$. Square of unit element 
$e_0$ is always positive and square of hyper-complex units $e_n$ can be positive or 
negative as well
\begin{equation} \label{e-pm}
e_n^2 = \pm e_0 ~.
\end{equation}
In applications of division algebras mainly the negative sign in (\ref{e-pm}) is used. 
In this case their norms are positively defined.

We note that in real world any comparison of two objects can be done only by exchange 
of some signals between them. The new physical requirement on the algebra we would 
like to apply is that it must contain special elements corresponding to unit signals, 
which can be used for distance measurements. One-way signals can correspond to any kind 
of number. The important is to have the real norm, which is actually the multiplication 
of direct and backward signals, being the result of one measurement. To have a 
possibility to introduction some dynamics we wont to choose positive sign in (\ref{e-pm}), 
or
\begin{equation} \label{e-}
e_n^2 = - e_0 ~.
\end{equation}
This leads to so-called split algebras with several negative terms in definition of 
there norms. Adjoining to real unit element (which could be connected with time) any 
invertible hyper-complex units with negative norms could be understood as an appearance 
of dynamics. Critical elements of split-algebras, so-called zero divisors, corresponding 
to zero norms can be used as unit signals to measure distances between physical objects 
(which are described by usual elements of the algebra).

We believe that dimension of proper algebra and properties of its unit elements also can 
be extracted from the physical considerations. We know that physical quantities are divided 
into three main classes, scalars, vectors and spinors. We require having a unique way to 
describe these objects in appropriate algebra.

Scalar quantities have magnitude only, and do not involve directions. The complete 
specification of a scalar quantity requires a unit of the same kind and a number counting 
how many times the unit is contained in the quantity. Scalar quantities are manipulated by 
applying the rules of ordinary algebra. 

Vast range of physical phenomena finds their most natural description as vector quantities, 
having magnitude and direction. A vector quantity requires for its specification a real 
number, which gives the magnitude to the quantity in terms of the unit (as for scalars) 
and an assignment of direction. By revising the process of addition from standard vector 
algebra it was found that any vector could be decomposed into components 
\begin{equation} \label{a}
{\bf a} = a_ne_n ~,
\end{equation}
using unit orthogonal basis vectors $e_n$. The ordinary idea of a product in scalar algebra 
cannot apply to vectors because of their directional properties. Since vectors have their 
origin in physical problems, definitions of the products of vectors historically was 
obtained with the way in which such products occur in physical applications. It was 
introduced two types of multiplication of vectors - the scalar product 
\begin{equation} \label{scalar}
({\bf ab}) = ab\cos{\theta} ~,
\end{equation}
and the vector product 
\begin{equation} \label{vector}
 [{\bf ab}] = ab\sin{\theta} ~.
\end{equation}
These formulae result the following properties of unit basis vectors
\begin{equation} \label{e-vector}
(e_ne_m) = 0 ~, ~~~~~ e_ne_m = - e_me_n ~, ~~~~~(e_ne_n) = e_n^2 = 1 ~, 
~~~~~[e_ne_m] = \varepsilon_{nmk}e_k ~,
\end{equation}
where $\varepsilon_{nmk}$ is fully anti-symmetric tensor. Or vice versa, we can say that 
postulated algebra (\ref{e-vector}) for basis elements one receive the observed 
multiplication laws (\ref{scalar}) and (\ref{vector}) of physical vectors. 

Since non-zero are only vector product of different basis vectors and scalar product of 
a basis vector on itself, we don't need to write different kind of brackets in 
expressions of their multiplications. 

Because of ambiguity of choosing of direction of basis vectors one must introduce conjugate 
elements $e_n^*$, which differ from $e_n$ by the sign $e_n^* = - e_n$, what can be 
understood as a reflection of the vector ort. 

The algebra of the vectors basis elements (\ref{e-vector}) is exactly the same we have 
for basis units of split algebras (with the positive sign (\ref{e-})), which we had choose 
from some general requirements. 

Next restriction for appropriate algebra can be imposed from the properties of spinors, 
another important class of observed physical quantities. Spinors also can be characterized 
by some magnitude and direction. In difference with vectors, spinors changes the sign after 
full rotation. Using multi-dimensional picture this feature can be understood as when we 
transform ordinary space coordinates spinors rotate also in some extra space. It is well 
known that there exists an equivalence of vectors and spinors in eight dimensions 
\cite{CaGa}, what renders the 8-dimensional theory (connected with octonions) a very 
special case. In eight dimensions a vector and a left and right spinor all have an equal 
number of components (namely, eight) and their invariant forms all look the same. If one 
considers the vectors rotation as the primary rotation that induces the transformations 
of two kinds of spinors, it is equivalent if one starts from spinor rotation, which induce 
corresponding rotation of vector and second kind of spinor. This property is referred in 
the literature as "principle of triality". The physical result of this principle can be 
the observed fact that interaction in particle physics exhibits by the vertex containing 
left spinor, right spinor and vector \cite{Ba}. 

To resume our general consideration we conclude that the proper algebra needed to describe 
the geometry of real world must be the 8-dimensional algebra of split-octonions over the 
real numbers. 

In the next section the properties of zero divisors, which appear in split-algebras is 
considered. The following three sections is devoted to geometrical applications of 
hyper-numbers, split-quaternions and split-octonions respectively.

%%%%%%%%%%%%%%%%%%%%%%%%%%%%%%%%%%%%%%%%%%%%%%%%%%%%%%%%%%%%%%%

\section{Zero Divisors}

From the different combinations of basis elements of split-algebra special objects, called 
zero divisors can be constructed \cite{Sc, SoLo}. We assume that these objects (we shall 
call critical elements of the algebra) as the unit signals for characterization of the 
physical events could be used. Sommerfeld specially notes physical fitness of zero divisors 
of split algebras in his book \cite{So}. 

The first objects we wont to study are projective operators with the property 
\begin{equation} \label{D}
D^2 = D ~,
\end{equation}
for any non-zero $D$. In division algebras only projective operator is the identity $1 = e_o$.

Most general form of projective operator in split-algebra is 
\begin{equation} \label{anen}
D = 1/2 + a_n e_n ~,
\end{equation}
with $a_na_n = 1/4$, where $n$ runs over the number of the orthogonal unit elements $e_n$. 
Commutating projective operator can be constructed by decomposition of the identity element 
\begin{equation} \label{1}
1 = D^+_n + D^-_n ~,
\end{equation}
with a particular choice 
\begin{equation} \label{D=pm}
D^+_n = \frac{1}{2}(1 + e_n)~, ~~~~~D^-_n = \frac{1}{2}(1 - e_n) ~.
\end{equation}
The elements $D^+_n$ and $D^-_n$ are commutating projective operators, since 
\begin{equation} \label{commuting}
D^+_nD^-_n = D^-_nD^+_n = 0 ~.
\end{equation}

Another type of zero divisors in split-algebras are Grassmann numbers defined as the set of 
anti-commuting numbers ${G_1, G_2, ...,G_n }$ with the properties 
\begin{equation} \label{G}
G^2_n = 0~, ~~~~~{G_n G_m} = 0~. 
\end{equation}
Since hyper-complex basis units satisfying the relations:
\begin{equation} \label{ee=eta}
e_n e_m = 2\eta_{nm}~, 
\end{equation}
the Grassmann numbers can be constructed by coupling of two basis elements (except of unity) 
with the opposite choice of signs in (\ref{e-pm}). For example, Grassmann numbers are the 
sums 
\begin{equation} \label{G-pm}
G^\pm = \frac{1}{2}(e_1 \pm e_2)
\end{equation}
of the two basis elements $e_1$ and $e_2$ with the properties
\begin{equation} \label{e12}
e_1^2 = 1 ~, ~~~~~ e_2^2 = -1 ~, ~~~~~ e_1^* = - e_1 ~, ~~~~~ e_2^* = - e_2~.
\end{equation}

The Grassmann numbers not contains the unit element $1$ as one of the terms (structure with 
the unit element forms projective operator) and it can be represented as
\begin{equation} \label{G=D}
G = e_1 D ~,
\end{equation}
where $D$ is projective operator and $e_1$ is the vector-like basis element of algebra with 
the positive square $e_1^2 = 1$. 

It can be checked that Grassmann numbers $G^\pm $ and projective operators $D^\pm$ obey the 
following algebra:
\begin{eqnarray} \label{GD}
D^\pm D^\mp = 0 ~, ~~~~~ D^\pm D^\pm = D^\pm ~, \nonumber \\
G^\pm G^\pm = 0 ~, ~~~~~ G^\pm G^\mp = D^\mp ~, \nonumber \\
D^\pm G^\pm = 0 ~, ~~~~~ D^\pm G^\mp = G^\mp ~, \\
G^\pm D^\mp = 0 ~, ~~~~~ G^\pm D^\pm = G^\pm ~. \nonumber 
\end{eqnarray}
From this relations we see that separately quantities $G^+$ and $G^-$ are Grassmann numbers, 
but they not commute with each other (in difference with the projective operators $D^+$ and 
$D^-$) and thus don't obey to full Gtassmann algebra (\ref{G}). Instead quantities $G^+$ and 
$G^-$ are the elements of so-called algebra of Fermi operators with the anti-commutator 
\begin{equation} \label{fermi}
G^+G^- + G^-G^+ = 1~. 
\end{equation}
Algebra of Fermi operators is some syntheses of Grassmann and Clifford algebras.

Mutually commuting zero divisors of algebra corresponds to simultaneously measurable signals 
and can serve as the geometrical basis for the physical events, which correspond to usual 
elements of the algebra. 

%%%%%%%%%%%%%%%%%%%%%%%%%%%%%%%%%%%%%%%%%%%%%%%%%%%%%%%%%%%%%%%%%%%%%%

\section{Hyper-Numbers}

There are certain important types of physical phenomenon, which are intrinsically 2-dimensional 
in nature. Many physical functions occur naturally in pairs and was expressed by 2-dimensional 
real numbers that satisfies a different type of algebra \cite{StNe}. Most common 2-dimensional 
algebra is the algebra of complex numbers, however, there is no special reason to prefer one 
algebra to the other \cite{Sa}. 

Historically, the motivation to introduce complex numbers was mathematical rather than physical. 
This numbers made sense of the solution of algebraic equations, of the convergence of series, 
of formulae for trigonometric functions, differential equations, and many other things. This 
was quite unlike the initial motivation for using real numbers, which come about as an 
idealization of the kind of quantity that directly arose from physical measurements. 

Essential for the complex numbers 
\begin{equation} \label{z}
z= x + iy ~,
\end{equation}
where $x$ and $y$ are some real numbers, is existence of the unit element and one imaginary 
element $i$, with the property $i^2 = -1$. Later it was found that if we introduce conjugation 
of complex numbers 
\begin{equation} \label{z*}
z^* = x - iy ~,
\end{equation}
there algebra is division. A regular complex number can be represented geometrically by the 
amplitude and by the polar angle
\begin{equation} \label{Nz}
\rho^2 = zz^* = x^2 + y^2 ~, ~~~~~ \theta = \arctan{\frac{y}{x}} ~. 
\end{equation}
The amplitude $\rho $ is multiplicative and the polar angle $\theta$ is additive upon the 
multiplication of complex numbers. 

Complex number had found many uses in physics, but these were considered as a "mathematical 
tricks", like the employment of complex numbers in 2-dimensional hydrodynamics, electrical 
circuit theory, or the theory of vibrations. 

In quantum mechanics complex numbers enter at the foundation of the theory from probability 
amplitudes and superposition principle. To pass from a quantum information link, which does 
not respect the rules of ordinary classical causality (as in the case of Einstein-Podolsky-Rosen 
phenomenon) to a classical information link (which necessarily propagates causally into the 
future) and to obtain real number corresponding to probability, one must multiply the complex 
amplitude by its complex conjugate. In fact, the complex conjured amplitude may be thought of 
as applying to the quantum-information link in the reverse direction in time \cite{Pe}. 

Up to now it is admitted without justification that two-valuedness of physical quantities 
naturally appearing in quantum mechanics is to be described in terms of complex numbers. 
Complex numbers achieve a particular representation of quantum mechanics in term of which the 
fundamental equations take their simplest form. Other choices for the representation of the 
two-valuedness also possible, but would give to the Schrodinger equation a more complicated 
form (involving additional terms), although its physical meaning would be unchanged \cite{CeNo}. 

The main reason why complex numbers are popular is Euler's formula. As it was mentioned by 
Feynman \cite{Fe}, "the most remarkable formula in mathematics is:
\begin{equation} 
e^{i\theta} = \cos{\theta} + i\sin{\theta} ~. \nonumber
\end{equation}
This is our jewel. We may relate the geometry to the algebra by representing 
complex numbers in a plane
\begin{equation} \label{euler}
x + iy = \rho e^{i\theta} ~. 
\end{equation}
This is the unification of algebra and geometry."

Using (\ref{euler}) the rule for complex multiplication looks almost obvious as a 
consequence of the behavior of rotations in plane. We know that a rotation of $\alpha$-angle 
around the $z$-axis, can be represented 
by 
\begin{equation} 
e^{i\alpha}(x + iy) = \rho e^{i(\theta + \alpha )} ~. \nonumber
\end{equation} 

Another possible 2-dimensional normed algebra - algebra of hyper-numbers 
\begin{equation} \label{zh}
z = t + hx 
\end{equation} 
has also a long history but is rarely used in physics \cite{OlAnFj}. In (\ref{zh}) the 
quantities $t, x$ are the real numbers and hyper-unit $h$ has the properties similar to 
ordinary unit vector 
\begin{equation} \label{h2}
h^2 = 1~. 
\end{equation} 

Conjugation of $h$ means space-reflection when $h$ changes its direction on opposite 
$h^* = -h$. In this case in difference with ordinary complex numbers norm of hyper-number 
is full invariant of rotation with reflection. 

Hyper-numbers constitute a commutative ring, but not a field, since norm of a nonzero
 hyper-number 
\begin{equation} \label{Nzh}
N = zz^* = t^2 - x^2 ~, 
\end{equation}
can be zero. 

Hyper-number does not have a inverse when its norm is zero, i. e., when $x = \pm t$, 
or, alternatively, when $z = t(1 \pm h)$. These two lines whose elements have no 
inverses, play the same role as the point $z = 0$ does for the complexes, and provide 
the essential property of the light cone, which makes the hyper-numbers relevant for 
representation of relativistic coordinate transformations. Lorentz boosts can be 
succinctly expressed as 
\begin{equation} \label{lorentz}
(t + hx) = (t' + hx')e^{h\theta}~,
\end{equation}
where $\tanh{\theta} = v$.

Any hyper-number (\ref{zh}) can be expressed in terms of light cone coordinates
\begin{equation} \label{zhDec}
z = t +hx = (t + x)D^+ + (t - x)D^- ~,
\end{equation}
where we introduced projective operators
\begin{equation} \label{Dzh}
D^\pm = \frac{1}{2}(1 \pm h) 
\end{equation}
corresponding to the critical signals.

So we can say that hyper-numbers can be used to describe the dynamic in 1-dimensional 
space. Unit element of algebra corresponding to positive term in the expression of the 
norm (\ref{Nzh}) is connected with time. Orthogonal to time hyper-element giving the 
negative term in (\ref{Nzh}) can be consider as describing one space dimension. Unit 
of algebra needed to measure distances is connected with the critical signals and is 
universal for all physical quantities, which described by usual elements of the algebra. 
Thus it is just the hyper-unit that seems to be responsible for the signature of 
space-time metric and the velocity of light $c$ is related with it.

%%%%%%%%%%%%%%%%%%%%%%%%%%%%%%%%%%%%%%%%%%%%%%%%

\section{Quaternions}

William Hamilton's discovery of quaternions in 1843 was the first event in history when 
the concept of 2-dimensional numbers was successfully generalized. Many authors focused 
on the adoption of quaternions in physics \cite{Sa,Qua}. 

General element of quaternion algebra can be written by using of the only two basis 
elements $i$ and $j$ in the form
\begin{equation} \label{q}
q = a + bi + (c + di)j ~,
\end{equation}
where $a, b, c$ and $d$ are some real numbers. The third basis element $(ij)$ can be 
described by conjugation of the first two.

Quaternion reverse to (\ref{q}), or conjugated quaternion $q^*$ can be received using 
the properties of basis units under the conjugation (reflection)
\begin{equation} \label{*}
i^* = - i~, ~~~~~j^* = - j~, ~~~~~ (ij)^* = - (ij) ~,
\end{equation}
Physically observable quantity is the norm of quaternion $N = qq^*$ corresponding to 
the multiplication of the direct and reverse signals.

When the basis elements $i$ and $j$ are imaginary $ i^2 = j^2 = -1 $ as ordinary complex 
unit we have Hamilton's quaternion with the positively defined norm 
\begin{equation} \label{Nq+}
N = qq^* = q^*q = a^2 + b^2 + c^2 +d^2 ~.
\end{equation} 
In this case the third orthogonal basis unit $(ij)$ has the analogous to $i$ and $j$ 
properties. 

For the positive squares $i^2= j^2 = 1$ we have the algebra of split-quaternions and 
the unit elements $i, j$ have the properties of ordinary unit vectors. Norm of 
split-quaternions
\begin{equation} \label{Nq-}
N = qq^* = q^*q = a^2 - b^2 - c^2 + d^2 ~, 
\end{equation}
has $(2+2)$ signature and in general is not positive defined. 

We see that the third unit element $(ij)$ of split-quaternions is the vector product of 
unit vectors $i$ and $j$ and thus is pseudo-vector. It is important to realize that a 
pseudo-vector is not a geometrical object in the usual sense. In particular, it should 
not be considered as a real physical arrow in space. The norm of quaternion is defined 
geometrically without reference to the arbitrary system of coordinates used, but in 
determining of direction of $(ij)$ the notion of right-handedness is needed. Vector 
multiplication of two vectors is ordinary vector only if we fix a right-handed Cartesian 
coordinate system and use right-handed rule. But coordinate system can be defined only 
if we have the second quaternion for comparison. In our approach we cannot do this, there 
is no independent geometry and coordinate systems as itself, we ca define distances only 
by critical signals. The pseudo-vector $(ij)$ differs from $i$ and $j$ and behaves like 
pure imaginary object, since
\begin{equation} \label{ij}
(ij)^* = - (ij) ~, ~~~~~ (ij)(ij) = - i^2j^2 = -1 ~.
\end{equation} 
In this fashion one can justify the origins of complex numbers without introducing them 
ad-hoc. The imaginary unit is $(ij)$ and the number $x + (ij)y$ has the properties of 
ordinary complex numbers. In this sense we can say that split-quaternions are more rich 
structure in compare to Hamilton's quaternios and complex and hyper-numbers are there 
particular case.

In the algebra of split quaternions can be introduced two classes (totally four) 
projective operators 
\begin{equation} \label{Dq}
D^\pm_i = \frac{1}{2}(1 \pm i) ~, ~~~~~ D^\pm_j = \frac{1}{2}(1 \pm j) ~, 
\end{equation}
which don't commute with each other. Commutated projective operators with the properties
\begin{equation} \label{D12}
[D^+D^-] = 0 ~, ~~~~~ D^+D^+ = D^+~, ~~~~~ D^-D^- = D^-~, 
\end{equation}
can be only two of them $D^\pm_i$, or $ D^\pm_j$. Operators $D^+$ and $D^-$ differs from 
the each other by reflecting of basis element and thus correspond to the direct and 
reverse critical signals along the one from the two real directions. 

In algebra we have also two classes of Grassmann numbers 
\begin{equation} \label{Gq}
G^\pm_i = \frac{1}{2}(1 \pm i)j ~, ~~~~~ G^\pm_j = \frac{1}{2}(1 \pm j)i ~. 
\end{equation}

From the properties of quaternion zero divisors 
\begin{eqnarray} \label{DGq}
D^\pm_i G^\pm_i = G^\pm_i ~, ~~~~~ D^\pm_j G^\pm_j = G^\pm_j ~, \nonumber\\
D^\pm_i G^\mp_i = 0 ~, ~~~~~~~~~~ D^\pm_j G^\mp_j = 0 ~, 
\end{eqnarray}
we see that Grassmann numbers (\ref{Gq}) have pair wise commute relations with the 
projective operators (\ref{Dq})
\begin{equation} \label{[dg]}
[D^\pm_i G^\mp_i] = 0 ~, ~~~~~ [D^\pm_j G^\mp_j] = 0 ~.
\end{equation}

Using commutating zero divisors any quaternion can be written in the form
\begin{equation} \label{qDec}
q = a +ib +(c +id)j = D^+[a+b + (c + d)G^+] + D^-[a-b +(c - d)G^-] ~,
\end{equation}
where $D^\pm$ and $G^\pm$ are projective operators and Grassmann elements, which belong 
to the one of the classes from (\ref{Dq}) and (\ref{Gq}) labeled by $i$, or $j$.

The quaternion algebra is associative and therefore matrices can represent them. We get 
the simplest non-trivial representation of the split-quaternion basis units if we choose 
the real Pauli matrices accompanied by the unit matrix. Note that for the real matrix 
representation of Hamilton's quaternions one needs 4-dimensional matrices. 

The independent unit elements of split quaternions $i$ and $j$ have the following matrix 
representation 
\[ i = \left( \begin{array}{cc} 1 & 0 \\  
                                0 & -1   
      \end{array} \right) ~, ~~~~~
j = \left( \begin{array}{cc} 0 & 1 \\  
                             1 & 0   
            \end{array} \right) ~. \]
The third unit element of quaternion algebra $(ij)$ is formed by multiplication of $i$ 
and $j$ and has the representation
\[ ij = \left( \begin{array}{cc} 0 & 1 \\  
                                -1 & 0   
      \end{array} \right) ~.\]

It is easy to noticed that in difference with complex case three real Pauli matrices 
have different properties, since their square gives the unit matrix 
\[ (1) = \left( \begin{array}{cc} 1 & 0 \\  
                                  0 & 1   
      \end{array} \right) \]
with the different signs
\begin{equation} \label{square}
i^2 = j^2 = (1) ~, ~~~~~(ij)^2 = -(1) ~.
\end{equation}

Conjugation of unit vectors $i$ and $j$ means changing of signs of matrices $i$, $j$ 
and $(ij)$ and
\begin{equation} \label{i-norm}
ii^* = jj^* = -(1) ~, ~~~~~(ij)(ij)^* = (1) ~.
\end{equation}

Matrix representation of independent projective operators and Grassmann elements from 
(\ref{Dq}) and (\ref{Gq}) labeled by $i$ are
\[ D^+_i = \frac{1}{2}(1+i) = \left( \begin{array}{cc} 1 & 0 \\  
                                               0 & 0   
      \end{array} \right) ~, 
 ~~~~~
D^-_i =\frac{1}{2}(1-i)  = \left( \begin{array}{cc} 0 & 0 \\  
                                             0 & 1   
            \end{array} \right) ~, \]
\[  G^+_i =\frac{1}{2}(j+ij) = \left( \begin{array}{cc} 0 & 1 \\  
                                               0 & 0   
      \end{array} \right) ~, ~~~~~
G^-_i =\frac{1}{2}(j-ij)  = \left( \begin{array}{cc} 0 & 0 \\  
                                              1 & 0   
            \end{array} \right) ~. \]
It is easy to find also matrix representation of zero divisors labeled by $j$.

Decomposition of the quaternion (\ref{qDec}) now can be written in the form
\[ q = 
 \left( \begin{array}{cc} (a+b) & (c+d) \\  
                          (c-d) & (a-b)    
      \end{array} \right)  ~.\]
Norm of the quaternion (\ref{Nq-}) is now the determinant of this matrix 
\begin{equation} \label{q-norm}
{\it det} q = (a^2 - b^2) - (c^2 - d^2) ~.
\end{equation}

If we treat unit element of algebra as a time then split-quaternions could be used 
for describing of rotation in 2-dimensional space, similar, as hyper-complex numbers 
are useful to study dynamic in (1+1)-space. We have now two rotations in $t-i$ and 
$t-j$ planes described by $D^\pm$, which introduces fundamental constant $c$. 

But what is geometrical meaning of Grassmanns numbers and fourth orthogonal element 
in quaternion algebra? Similar elements was absent in algebra of hyper-numbers 
containing only one fundamental constant $c$. 

It can be shown, that pseudo-vector $(ij)$ connected with rotation in $(i-j)$ plane. 
Indeed operators 
\begin{equation} \label{Rq}
R^\pm = \frac{1}{\sqrt{2}}[1 \pm (ij)] ~,
\end{equation} 
with the property $ R^+R^- = 1$, represent rotation on the $90^0$ in $(i-j)$ plane 
and can transform $i$ to $j$
\begin{equation} \label{i-j}
R^+iR^- = - j ~, ~~~~~ R^+jR^- = i~.
\end{equation} 
So extra pseudo-direction $(ij)$ is similar to angular momentum and can be connected 
with the spin. 

There is the symmetry with choosing of independent projective operator $D_i^\pm$ or 
$D_j^\pm$ for decomposition of a quaternion (\ref{qDec}). This means that we have two 
independent space directions $i$ and $j$ where we can in principle exchange signals. 
However, we can simultaneously measure the signals only from one direction, since 
operators $D_i^\pm$ and $D_j^\pm$ not commute. This property is equivalent to 
non-commutativity of space coordinates \cite{Non}.

The constant $c$ (which is universal since connected with critical signals $D^\pm$) 
for the convenience can be extracted explicitly from the unit element of any usual 
quaternion 
\begin{equation} \label{dual}
q = ct + xi + yj + \lambda (ij) ~,
\end{equation} 
where $\lambda$ is some quantity with the dimension of length. 

Grassmann numbers correspond to critical rotations in $(i-ij)$ and $(j-ij)$ planes, 
similar as projective operators does in $(1-i)$ and $(1-j)$ planes. Existence of 
maximal velocity following from the requirement to have positive norm for the components 
of quaternion (\ref{dual}) means
\begin{equation} \label{delta-x}
\frac{\Delta x}{c\Delta t} < 1 ~, ~~~~~ \frac{\Delta y}{c\Delta t} < 1 ~.
\end{equation}
Now we can write similar extra relations 
\begin{equation} \label{delta-l}
\frac{\Delta x}{\Delta \lambda} < 1 ~, ~~~~~
\frac{\Delta y}{\Delta \lambda} < 1 ~, 
\end{equation}
connected with the Grassmann numbers $G^\pm$. So there must exists second fundamental 
constant (which can be extracted from $\lambda$) characterizing this critical property 
of algebra. It is natural to identify it with the Plank's constant $\hbar$ and write 
$\lambda$ in the form
\begin{equation} \label{P}
\lambda = \frac{\hbar}{P} ~, 
\end{equation}
where quantity $P$ has the dimension of the momentum. Inserting (\ref{P}) into 
(\ref{delta-l}) we found that probably uncertainty principle of quantum mechanics follows 
from the condition of the positively defined of the quaternion norm and has the geometrical
meaning similar as existence of the maximal velocity. 

To resume split quaternion (\ref{dual}) with the norm
\begin{equation} \label{qN}
N = c^2t^2 + \frac{\hbar^2}{P^2} - x^2 - y^2 ~,
\end{equation}
could be used to describe dynamics of particle with spin in 2-dimensional space. Two 
fundamental constants $c$ and $\hbar$ have the geometrical origin and correspond to two 
kinds of critical signals in (2+2)-dimensions with one time coordinate. The Lorentz factor 
\begin{equation} \label{qLorentz}
\gamma = \sqrt{1 - \frac{v^2}{c^2} + \frac{\hbar^2}{c^2}~ \frac{F^2}{P^2} }~,
\end{equation}
corresponding to general bust in the space (\ref{qN}) except of velocity $v^2$ contains 
extra positive term, where $F$ is some kind of force. So dispersion relation in (2+2)-space 
(\ref{qN}) will have form similar to the double-special relativity models \cite{double}. 

%%%%%%%%%%%%%%%%%%%%%%%%%%%%%%%%%%%%%%%%%%%%%%%%%%%%%%%%%%%%%%%%%%%%%%%%%%%%

\section{Octonions}

Despite the fascination of octonions for over a century (in 1844-1845 by Graves and Cayley) 
it is fair to say that they still await universal acceptance. However, this is not to say 
that there have not been various attempts to find appropriate uses for them in physics 
(see, for example \cite{SoLo,Qua,Oct,Ku}). 

As we had seen hyper-numbers are useful to describe dynamics in 1-dimensional space, 
wile split-quaternion - in 2-dimensional space. According to the Hurwitz theorem there 
is yet only one composite division algebra, the octonion algebra \cite{Sc}. We wont to 
show that split-octonions can describe dynamics in usually 3-dimensional space.

For constructing of the octonion algebra with a general element
\begin{equation} \label{O}
O = a_0 e_0 + a_n e_n ~, ~~~~~n =1,2, ..., 7
\end{equation}
where $e_0$ is the unit element and $a_0, a_n$ are the real numbers, multiplication law 
of its eight basis elements $e_0, e_n$ usually is given. For the case of ordinary octonions 
\begin{equation} \label{en}
e_0^2 = e_0~,~~~~~ e_n^2 = - e_0~, ~~~~~ e_0^* = e_0 ~, ~~~~~ e_n^* = - e_n ~,
\end{equation}
with the positively defined norm
\begin{equation} \label{NO+}
N = OO^* = O^*O = a_0^2 + a_1^2 + a_2^2 + a_3^2 + a_4^2 + a_5^2 + a_6^2 + a_7^2 ~,
\end{equation}
multiplication table can be written in the form
\begin{equation} \label{ee}
(e_n e_m) = - \delta_{nm}e_0 + \epsilon_{nmk}\epsilon_k ~,
\end{equation}
where $ \delta_{nm}$ is the Kronecker symbol and $\epsilon_{nmk}$ is the fully 
anti-symmetric tensor with the postulated value $\epsilon_{nmk} = + 1$ for the following 
values of indices 
$$
nmk = 123, 145, 176, 246, 257, 347, 365 ~.
$$
This definition of structure constants used by Cayley in his original paper but is not 
unique. There are $16$ different possibilities of definition of $\epsilon_{nmk}$ leading 
to the equivalent algebras. Sometimes for visualization of multiplication of octonion 
basis units except of tables the geometrical picture of Fano plane (a little gadget with 
7 points and 7 lines) is used \cite{Ba}. Unfortunately both of these methods are almost 
unenlightening. We wont to consider more obvious picture using properties of usual vectors 
products. 

For the case of quaternions we had shown that not all basis elements of the algebra are 
independent. We easily recover full algebra from two basis elements without considering 
of the multiplication tables or graphics. The same can be done with the split-octonions. 
Now beside of unit element $1$ we have three fundamental basis elements $i$, $j$ and $l$ 
with the properties of ordinary unit vectors
\begin{equation} \label{i2j2l2}
i^2 = j^2 = l^2 = 1~.
\end{equation}
Other four basis elements of octonion algebra $(ij), (il), (jl), ((ij)l)$ can be 
constructed by vector multiplications of $i$, $j$ and $l$. All brackets here denote 
vector product and they show only the order of multiplication. Here we have some ambiguity 
of placing brackets for the eights element $((ij)l)$. However, we can choose any convenient 
form of placing of brackets, all the other possibilities are defined by ordinary laws of 
triple vector multiplication. As it was mentioned in the introduction for the orthogonal 
basis vectors we don't need to use different kind of brackets to denote there scalar and 
vector multiplication, since the product of the different basis vectors is always vector 
product, and the product of two same basis vectors is always scalar product. 

Using the properties of standard scalar and vector products of orthogonal vectors $i$, $j$ 
and $l$ we find the properties of the other four, not fundamental basis elements of octonion 
algebra 
\begin{eqnarray} \label{ijl}
(i(jl)) = - ((ij)l) = - ((li)j)~, \nonumber\\
ij = -ji~, ~~~~~ il = -li ~, ~~~~~ jl = -lj ~, \nonumber \\
(ij)^2 = (il)^2 = (jl)^2 = -1~, ~~~~~((ij)l)^2 = 1 ~, \\
(ij)(ij)^* = (il)(il)^* = (jl)(jl)^* = 1~, ~~~~~((ij)l)((ij)l)^* = -1 ~. \nonumber 
\end{eqnarray}
Conjugation as for the quaternion case means reflection of the basis vectors, or changing 
of the signs of basis elements except of unit element. These obvious properties of the 
scalar and vector products of three fundamental basis vectors are hidden if we write abstract 
octonion algebra (\ref{ee}).

In the algebra of split-octonions there is possible to introduced the four classes of 
projective operators 
\begin{equation} \label{DO}
D^\pm_i = \frac{1}{2}(1 \pm i) ~, ~~~~~ D^\pm_j = \frac{1}{2}(1 \pm j) ~,
~~~~~ D^\pm_l = \frac{1}{2}(1 \pm l) ~, ~~~~~ D^\pm_{(ij)l} = \frac{1}{2}(1 \pm (ij)l) ~,  
\end{equation}
corresponding to the critical unit signal along the $i,j,l$ and $(ij)l$ directions. This 
four classes of projective operators don't commute with each other. Independent projective 
operators can be only one class from them, i.e. any pair $D^\pm$ from (\ref{DO}) with the 
same label. 

In the algebra we have also the four classes of Grasman numbers 
\begin{eqnarray} \label{GGO}
G^\pm_{i1} = \frac{1}{2}(1 \pm i)j ~, ~~~~~~~~ G^\pm_{i2} = \frac{1}{2}(1 \pm i)l ~,
~~~~~~~~  G^\pm_{i3} = \frac{1}{2}(1 \pm i)(jl) ~, \nonumber\\
G^\pm_{j1} = \frac{1}{2}(1 \pm j)i ~, ~~~~~~~~ G^\pm_{j2} = \frac{1}{2}(1 \pm j)l ~,
~~~~~~~~ G^\pm_{j3} = \frac{1}{2}(1 \pm j)(il) ~, \\
G^\pm_{l1} = \frac{1}{2}(1 \pm l)i ~, ~~~~~~~~ G^\pm_{l2} = \frac{1}{2}(1 \pm l)j ~,
~~~~~~~~  G^\pm_{l3} = \frac{1}{2}(1 \pm l)(ij) ~, \nonumber\\
G^\pm_{(ij)l1} = \frac{1}{2}(1 \pm (ij)l )i ~, ~~~ G^\pm_{(ij)l 2} = 
\frac{1}{2}(1 \pm (ij)l)j ~,~~~ G^\pm_{(ij)l 3} = \frac{1}{2}(1 \pm (ij)l)l ~. \nonumber
\end{eqnarray}
Mutually commute only the numbers $G_1, G_2$ and $G_3$ from each class with the same sign. 
So we have $8$ independent real Grassmann algebras with the elements $G^\pm_{n1}, 
G^\pm_{n2}$ and $ G^\pm_{n3}$, where $n$ runs over $i, j, l$ and $((ij)l)$. 

In difference with quaternions, octonions cannot be represented by matrices with the usual 
multiplication lows \cite{Sc}. The reason is non-associativity leading to different rules 
for left and right multiplications.

Using commuting projective operators and Grassmann numbers any split octonion 
\begin{equation} \label{Os}
O = A + Bi + Cj + D(ij) + El + F(il) + G(jl) + H(i(jl)) ~,
\end{equation}
where $A, B, C, D, E, F, G$ and $H$ are some real numbers, could be written in the form 
\begin{eqnarray} \label{O1}
O = D^+\left[ (A + B) + (C + D)G^+_{1} + (E + F) G^+_{2}  + (G + H) G^+_{3}\right] + \nonumber \\
+ D^-\left[ (A - B) + (C - D)G^-_{1} + (E - F) G^-_{2}  + (G - H) G^-_{3}\right] ~.
\end{eqnarray}
Because of symmetry between $i, j$ and $l$ using (\ref{DO}) and (\ref{GGO}) we can find 
$3$ different representation of (\ref{O1}). 

Similar to quaternion case here we also have two fundamental constants $c$ and $\hbar$ 
corresponding to the two types of critical signals $D^\pm$ and $G^\pm$ and any octonion 
can be written in the form
\begin{equation} \label{Op}
O = ct + xi + yj + zl + \frac{\hbar}{P_z}(ij) + \frac{\hbar}{P_y} (il) + \frac{\hbar}{P_x} (jl) 
+ \omega (i(jl)) ~,
\end{equation}
where $\omega$ is some quantity with the dimension of length. Here we had used similar to 
(\ref{P}) relations to extract Planck's constants form the components of (\ref{Os}) and 
thus introduced the quantities $ P_x, P_y$ and $ P_z $ with the dimensions of momentums. 

Two terms in formula (\ref{O1}) with different signs of zero divisors correspond to direct 
and backward signals in the direction of one of the axis $i, j$, or $l$. However, as for 
the quaternions different class of projective operators $D_i, D_j$ and $D_l$ are not 
commute, what is the analog of non-commutivity of coordinates in the models \cite{Non}.

Specific for the representation (\ref{O1}) also is the existence of three zero divisors 
forming full Grassmann algebras with three elements for any direction. So for octonions 
we have four type of commuting critical signals. One is rotation in $t-x$ planes and 
corresponds to the constant $c$. Two are different rotations in $x-P$ planes and gives 
uncertainty principles similar to (\ref{delta-l}) containing $\hbar$ and the last one is 
rotation in $P-\omega$ plane. 

Requiring to have the positively defined norm for octonion (\ref{Op}), in addition to 
(\ref{delta-x}) and (\ref{delta-l}), we can write the extra relations 
\begin{equation} \label{delta-e}
\frac{\Delta\omega}{c\Delta t} < 1 ~, ~~~~~ \frac{\Delta\omega\Delta P}{\hbar} < 1 ~.
\end{equation}
If we extract two fundamental constants $c$ and $\hbar$ from $\omega$ have the quantity 
with the dimension of energy
\begin{equation}\label{E}
E = \frac{c\hbar}{\omega} ~.
\end{equation}
Then norm of the octonion (\ref{Op}) 
\begin{equation} \label{ON}
N = OO^* = c^2t^2 + \frac{\hbar^2}{P_x^2}+ \frac{\hbar^2}{P_y^2} + \frac{\hbar^2}{P_z^2} 
- x^2 - y^2 - z^2 - \frac{c^2\hbar^2}{E^2} ~,
\end{equation}
will be similar to the distance in some kind of phase space.

In spite that in the split-octonion algebra there are four real basis vectors $i, j, l$ 
and $((ij)l)$ we can consider to have only the three space dimensions, since we have only 
the three independent rotation operators in $(i-j)$, $(i-l)$ and $(j-l)$ planes respectively
\begin{equation} \label{RO}
R^\pm_{ij} = \frac{1}{\sqrt{2}}[1 \pm (ij)] ~, ~~~~~ R^\pm_{il} = 
\frac{1}{\sqrt{2}}[1 \pm (il)]~, ~~~~~ R^\pm_{jl} = \frac{1}{\sqrt{2}}[1 \pm (jl)]~,
\end{equation} 
with the property $R^+R^- = 1$. Another fact is that in formulae (\ref{DO}) and 
(\ref{GGO}) there is no symmetry between zero divisors along the directions $i, j, l$ 
and along $((ij)l)$. The reason is non-associativity of octonions resulting not unique 
answer of the product of $D^\pm_{(ij)l} $ with $G^\pm_{(ij)l} $. It means that commuting 
this operators or not is depends on the order of multiplication of basis elements. As the 
result we can't write decomposition of a octonion (\ref{O1}) using $D^\pm_{(ij)l} $ and 
$G^\pm_{(ij)l} $ in the unique way. 

The basis element $((ij)l)$ all time appears when we study rotations of $i, j$ and $l$. 
For example, the rotation of basis element $i$ around itself (in $(j-l)$ plane) on the 
$90^0$ by the operator (\ref{RO}) gives the vector $((jl)i)$. After the rotation on $\pi $ 
we receive $-i$ and only after the full rotation on $2\pi$ we come back to $i$, since
\begin{equation} \label{Ri}
R^+_{jl} i R^-_{jl} = -i(jl) = (ij)l ~, ~~~~~ R^+_{jl} ((ij)l) R^-_{jl} = -i ~.
\end{equation} 
Similar situation is for the rotations around $j$ and $l$. 

So fourth vector direction $((ij)l)$ corresponds to the multi-valuence of the components 
of octonion and non-associativity, which corresponds of non-visibility of fourth 
direction, introduces fundamental uncertainty in physics. 

Non-associativity of octonions follows from the property of triple vector products and 
appears when we consider multiplication of three different basis elements of octonion 
necessarily including the eighth basis element $((ij)l)$ (in our representation) as one 
of the term. For example, we have
\begin{eqnarray} \label{nonasso}
j \left( l ~((ij)l)\right) = j \left(-(ij)\right) = i ~,\\
(jl)~((ij)l) = (jl)((jl)i) = -i ~.
\end{eqnarray}

Non-associativity of octonions is difficult to understand only studing the properties 
of basis elements. For example, if we introduce so-called open multiplication laws for 
the basis elements \cite{Ku}, i.e. we don't place the brackets, and formally using only 
the anti-commute properties of the basis elements we found 
\begin{equation} \label{ijl1}
ijl = lij ~. 
\end{equation}
But anti-commuting comes from the property of vector products and its impossible to use 
separately. If we put the brackets from the properties of vector products we shall 
receive the another result 
\begin{equation} \label{(ij)i}
((ij)l) = - (l(ij))~, 
\end{equation}
since this expression is the vector product of the vectors $l$ and $(ij)$.

Non-associativity follows from the fact that the basis elements of octonions $(ij), (il)$ 
and $(jl)$ are pseudo-vectors and have properties of imaginary unit, i.e. there square 
(scalar product) is negative. Even in three dimensions ordinary vector multiplication is 
non-associative. For example, if ${\bf a}$ and ${\bf b}$ are two real 3-vectors, there 
triple vector product is not associative 
\begin{equation} \label{non-ass}
[[{\bf aa}]{\bf b}] = 0 ~, ~~~~~ [{\bf a}[{\bf ab}]]=-{\bf b} ~.
\end{equation}
In 3-space triple vector product 
\begin{equation} \label{abc}
[{\bf a}[{\bf bc}]] = ({\bf ac}){\bf b} - ({\bf ab}){\bf c}
\end{equation}
of three orthogonal vectors (describing by the determinant of three-on-three matrix) is 
zero. Considering vector products in more than three dimensions there possible to 
construct extra vectors, normal to all three (using matrix with the more than three 
columns and rows). Then triple vector product of three orthogonal vectors is not zero 
and we shall automatically receive their non-associative properties. In particular, when 
we have product of three basis element, which contains the eights basis unit $((ij)l)$, 
for some order of multiplication of the basis elements there necessarily arises scalar 
product (square) of two pseudo-vectors $(ij), (il)$, or $(jl)$ giving the opposite sign 
in the result. 

Octonions have a weak form of associativity called alternativity, what in the language 
of basis elements means that in the expressions when only two fundamental basis elements 
exists the placing of brackets is arbitrary, for example 
\begin{equation} \label{ii}
(ii)l = i(il)~, ~~~~~ (li)i = l(ii) ~.
\end{equation}
This property, following from multidimensional generalization of (\ref{abc}), allows us 
to simplify expressions containing more than three octonionic unit elements. Unit 
vectors, which after commutating appears to be the neighbors, can be take out from the 
brackets and using scalar product to identify there square with the unit of algebra. 

At the end we want to note, that in general relations (\ref{delta-x}), (\ref{delta-l}) 
and (\ref{delta-e}) are not to be satisfied separately. Positively definition of the 
norm (\ref{ON}) gives only one relation containing all components of octonion. The 
Lorentz factor 
\begin{equation} \label{OLorentz}
\gamma = \sqrt{1 - \frac{v^2}{c^2}  - \hbar^2 ~\left(\frac{A^2}{E^2} - \frac{1}{c^2}~ 
\frac{F^2}{P^2}\right) }~,
\end{equation}
corresponding to general bust in (4+4)-space. Here the quantities $A$ and $F$ have 
the dimensions of work and force respectively. The formula (\ref{OLorentz}) contains 
the extra terms, which change the 4-dimensional dispersion relation similar as it is 
for double-special relativity theory \cite{double}. 

%%%%%%%%%%%%%%%%%%%%%%%%%%%%%%%%%%%%%%%%%%%%%%%%%%%%%%%%%%%%%%%%%%

\section{Conclusion}

A physical applicability of normed split-algebras, such as hyperbolic numbers, 
split-quaternions and split-octonions was considered. We argue that the observable 
geometry can be described by the algebra of split-octonions, which is naturally 
equipped by zero divisors (the elements of split-algebras corresponding to zero norm). 
In such a picture physical phenomena are described by the ordinary elements of chosen 
algebra, while the zero divisors give raise the coordinatization and two fundamental 
constants, namely velocity of light and Planck constant. It turns to be possible that 
uncertainty principle appears from the condition of positively defined norm, and has 
the same geometrical meaning as the existence of the maximal value of speed. The 
property of non-associativity of octonions could correspond to the appearance of 
fundamental probabilities in physics. Grassmann elements and a non-commutativity of 
space coordinates, which are widely used in various physical theories, appear naturally 
in our approach. 

We do not yet introduce any physical fields or equations. We had only shown that the 
proper algebra could describe geometry and introduce some necessary characteristics 
of future particle physics models. 

%%%%%%%%%%%%%%%%%%%%%%%%%%%%%%%%%%%%%%%%%%%%%%%%%%%%%%%%%%%%%%%%%%%%%%%%

\vskip 1cm

{\bf Acknowledgements:} Author would like to acknowledge the hospitality
extended during his visits at Theoretical Division of CERN where this work was done.

%%%%%%%%%%%%%%%%%%%%%%%%%%%%%%%%%%%%%%%%%%%%%%%%%%%%%%%%%%%%%%%%%%%%%%%%

\begin{thebibliography}{99}

\bibitem{Sc} R. Schafer,
            {\it Introduction to Non-Associative Algebras},
            (Dover, New York, 1995).

\bibitem{CaGa} E. Cartan,
             {\it Lecons sur la Theorie des Spineurs},
             (Hermann, Paris, 1938); \\
              A. Gamba,
             {\it Peculiarities of the Eight-Dimensional Space},
             J. of Math. Phys., {\bf 8}, 775 (1967).

\bibitem{Ba} J. Baez,
             {\it The Octonions},
             math.RA/0105155.

\bibitem{SoLo} L. Sorgsepp and J. Lohmus,
               {\it About Nonassociativity in Physics and Cayley-Graves' 
                Octonions},
               (Preprint F-7, Academy of Sci. of Estonia, 1978).

\bibitem{So} A. Sommerfeld,
            {\it Atombau und Spektrallinien, II Band} 
            (Vieweg, Braunschweig, 1953).

\bibitem{StNe} J. Stillwell,
             {\it Mathematics and its History}
             (Springer, New York 1989); \\
             T. Needham,
             {\it Visual Complex Analysis}
             (Oxford Univ. Press, Oxford 2002). 

\bibitem{Sa} M. Sachs,
            {\it General Relativity and Matter: a Spinor Field Theory from 
            Fermis to Light-Years} (Reidel Publ., Dordrecht, 1982).

\bibitem{Pe} R. Penrose,
            in {\it Mathematics: Frontiers and Perspectives}
            (editors: V. I. Arnol'd at all., NY AMS, New York, 1999).

\bibitem{CeNo} M.Celerier and L. Nottale,
              {\it Dirac Equation in Scale Relativity},
              hep-th/0112213 

\bibitem{Fe} R. P. Feynman,
            {\it The Feynman Lectures on Physics, vol. I - part 1} 
            (Inter European Edition, Amsterdam, 1975).

\bibitem{OlAnFj} S. Olariu,
             {\it Hyperboloc Complex Numbers in Two Dimensions},
             math.CV/0008119; \\
          F. Antonuccio,
             {\it Semi-Complex Analysis and Mathematical Physics}, 
             gr-qc/9311032; \\
             P. Fjelstad,
             {\it Extending Special Relativity via the Perplex Numbers},
             Am. J. Phys., {\bf 54}, 416 (1986).

\bibitem{Qua} S. Altmann,
            {\it Rotations, Quaternions, and Double Groups}
            (Claredon, Oxford, 1986); \\
            S. Adler,
            {\it Quaternion Quantum Mechanics and Quantum Fields}
            (Oxford Univ. Press, Oxford, 1995).

\bibitem{Non} M. Douglas and N. Nekrasov'
             {\it Noncommutative Field Theory},
             hep-th/0106048; Rev. Mod. Phys., {\bf 73}, 977 (2001); \\
             R. Szabo,
            {\it Quantum Field Theory on Noncommutative Spaces}, 
            hep-th/0109162.

\bibitem{Oct} D. Finkelstein, 
              {\it Quantum Relativity: A Synthesis of the Ideas of Einstein and 
               Heisenberg}(Springer, Berlin, 1996);\\
               G. Emch, 
              {\it Algebraic Methods in Statistical Mechanics and Quantum Field 
              Theory}
              (Wiley, New York, 1972); \\
               F. Gursey and C. Tze,
               {\it On the Role of Division, Jordan and Related Algebras in 
               Particle Physics} (World Scientific, Singapore, 1996).
\bibitem{Ku} D. Kurdgelaidze,
             {\it The Foundations of Nonassociative Classical Field Theoty},
             Acta Phys. Hung., {\bf 57}, 79 (1985).

\bibitem{double} G. Amelino-Camelia,
             {\it Doubly-Special Relativity: First Results and Key Open 
             Problems}, gr-qc/0210063; \\
             J. Mabuejo and L. Smolin,
             {\it Lorentz Invariance with an Invariant Energy Scale},
             hep-th/0112090, Phys. Rev. Lett. {\bf 88}, 190403 (2002).

\end{thebibliography}

\end{document}

