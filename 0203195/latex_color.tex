
\documentclass[a4paper,12pt]{article}

\usepackage{useful_macros}
\begin{document}

\bibliographystyle{unsrt}

\begin{titlepage}

\title{\bf An algebraic method for solving the SU(3) Gauss law}
\author{Antti Salmela\footnote{Email: Antti.Salmela@Helsinki.Fi} \\
\it Theoretical Physics Division \\ \it Department of Physical Sciences \\
\it P.O. Box 64, 00014 University of Helsinki, Finland}
\date{}

\maketitle

\begin{abstract} 

A generalisation of existing SU(2) results is obtained. In particular, the source-free Gauss law for SU(3)-valued gauge fields is solved using a non-Abelian analogue of the Poincar\'e lemma. When sources are present, the colour-electric field is divided into two parts in a way similar to the Hodge decomposition. Singularities due to coinciding eigenvalues of the colour-magnetic field are also analysed.

\end{abstract}

{\noindent PACS numbers: 11.15.-q, 12.38.-t, 02.20.Sv}

\end{titlepage}

\pagebreak

\section{Introduction}

Incorporating the Gauss law into a Hamiltonian formulation of Yang-Mills theory
is a nontrivial and important problem, which is usually solved by ignoring the
Gauss law first and then reintroducing it at the quantum level as a condition on the physical states. Yet in order to have a classical Hamiltonian Yang-Mills theory with Gauss's law in force we need a different approach. Majumdar and Sharatchandra have proposed a
parametrisation of the SU(2) colour-electric field \myHighlight{$E_k$}\coordHE{} which satisfies identically the
source-free Gauss law \cite{ms1}
\begin{eqnarray}\coord{}\boxAlignEqnarray{
&&\leftCoord{} \sum_{\rightCoord{}k=1}^{\leftCoord{}3} \nabla_k(A) E_k = 0, \rightCoord{}\label{gsf} \\\leftCoord{}*
&&\leftCoord{} \nabla_k (A) = \partial_k + ig [A_k (x),\rightCoord{}\, \cdot \rightCoord{}\, ].\nonumber\rightCoord{}
\rightCoord{}}{0mm}{4}{7}{
&& \sum_{k=1}^{3} \nabla_k(A) E_k = 0, \\*
&& \nabla_k (A) = \partial_k + ig [A_k (x),\, \cdot \, ].}{1}\coordE{}\end{eqnarray}
They express \myHighlight{$E_k$}\coordHE{} as a sum of a covariant curl and a
gradient thus obtaining an SU(2) generalisation of the Poincar\'e lemma. The
fields appearing in this decomposition then describe the physical degrees of
freedom of SU(2) Yang-Mills theory. In order to make use of the decomposition
in QCD we need to generalise the results of ref. \cite{ms1} to SU(3). The purpose of this 
paper is to provide such an extension. Besides Gauss's law, the decomposition might
also be useful in parametrising the non-Abelian generalisation of the Coulomb gauge
$$\coord{}\boxMath{ \sum_{k=1}^3 \nabla_k(A) \dot{A}_k = 0 }{dollar}{0pt}\coordE{}$$
proposed by Cronstr\"om \cite{cc}.

\section{SU(3) algebra}

I write every element of the SU(3) algebra in the form
$$\coord{}\boxMath{A = \frac{1}{2} A^a \lambda_a,}{dollar}{0pt}\coordE{}$$
where the \myHighlight{$\lambda_a$}\coordHE{}'s stand for the Gell-Mann matrices
$$\coord{}\boxMath{\lambda_a \lambda_b = \frac{2}{3} \delta_{ab} \, {\bf 1}_{3 \times 3} + \left( {d_{ab}}^c + i {f_{ab}}^c
\right) \lambda_c.}{dollar}{0pt}\coordE{}$$
Summation over repeated indices is implied. An inner product between two
algebra elements is given by the Killing form
\begin{eqnarray*}\coord{}\boxAlignEqnarray{\leftCoord{}
\leftCoord{}(A,B) &=& h_{ab} A^a B^b = 6 \rightCoord{}\, {\rm Tr} (AB), \\\leftCoord{}*
h_{ab} &=& - {f_{ac}}^d {f_{bd}}^c = \rightCoord{}\, 3 \rightCoord{}\, \delta_{ab}.\rightCoord{}
\rightCoord{}}{0mm}{3}{6}{
(A,B) &=& h_{ab} A^a B^b = 6 \, {\rm Tr} (AB), \\*
h_{ab} &=& - {f_{ac}}^d {f_{bd}}^c = \, 3 \, \delta_{ab}.
}{1}\coordE{}\end{eqnarray*}
I have chosen the convention where the inner product is positive definite.
This inner product defines a norm, which will be denoted by \myHighlight{$|\cdot|$}\coordHE{}. The \myHighlight{$d$}\coordHE{}
tensor can be used to define a matrix-valued product
\begin{eqnarray*}\coord{}\boxAlignEqnarray{\leftCoord{}
A*B &=& \frac{\leftCoord{}1}{\rightCoord{}2} {d_{ab}}^c A^a B^b \lambda_c \\\leftCoord{}*
&\leftCoord{}=& \{A,B\} - \frac{\leftCoord{}1}{\rightCoord{}9} (A,B) \rightCoord{}\, {\bf 1}_{3 \times 3}.\rightCoord{}
\rightCoord{}}{0mm}{5}{6}{
A*B &=& \frac{1}{2} {d_{ab}}^c A^a B^b \lambda_c \\*
&=& \{A,B\} - \frac{1}{9} (A,B) \, {\bf 1}_{3 \times 3}.
}{1}\coordE{}\end{eqnarray*}
In addition to the Jacobi identity there exist several other identities involving the
structure constants of the algebra. They were worked out in ref. \cite{msw}:
\begin{eqnarray}\coord{}\boxAlignEqnarray{
&&\leftCoord{}{f_{ea}}^d {d_{bc}}^e + {f_{eb}}^d {d_{ca}}^e + {f_{ec}}^d {d_{ab}}^e = 0
 \label{id1} \rightCoord{}\\
&&\leftCoord{}{f_{ea}}^b {f_{cd}}^e = \frac{\leftCoord{}2}{\rightCoord{}3} \left( \delta_{ac} {\delta^b}_d - \delta_{ad}
 {\rightCoord{}\leftCoord{}\delta^b}_c \right) + {d_{ac}}^e {d_{ed}}^b - {d_{ec}}^b {d_{ad}}^e \rightCoord{}\\
&&\leftCoord{}{d_{ad}}^e {d_{eb}}^c + {d_{bd}}^e {d_{ea}}^c + {d_{ed}}^c {d_{ab}}^e = \frac{\leftCoord{}1}{\rightCoord{}3} \left( \delta_{ab} {\delta^c}_d + {\delta_b}^c \delta_{ad} + {\delta_a}^c \delta_{bd}  \right) \rightCoord{}\\
&&\leftCoord{} 3 \rightCoord{}\, {d_{ea}}^b {d_{cd}}^e = \delta_{ac} {\delta^b}_d + \delta_{ad} {\delta^b}_c
\leftCoord{}- {\delta_a}^b \delta_{cd} + {f_{ac}}^e {f_{de}}^b + {f_{ad}}^e {f_{ce}}^b.\rightCoord{}
\rightCoord{}}{0mm}{8}{10}{
&&{f_{ea}}^d {d_{bc}}^e + {f_{eb}}^d {d_{ca}}^e + {f_{ec}}^d {d_{ab}}^e = 0
 \\
&&{f_{ea}}^b {f_{cd}}^e = \frac{2}{3} \left( \delta_{ac} {\delta^b}_d - \delta_{ad}
 {\delta^b}_c \right) + {d_{ac}}^e {d_{ed}}^b - {d_{ec}}^b {d_{ad}}^e \\
&&{d_{ad}}^e {d_{eb}}^c + {d_{bd}}^e {d_{ea}}^c + {d_{ed}}^c {d_{ab}}^e = \frac{1}{3} \left( \delta_{ab} {\delta^c}_d + {\delta_b}^c \delta_{ad} + {\delta_a}^c \delta_{bd}  \right) \\
&& 3 \, {d_{ea}}^b {d_{cd}}^e = \delta_{ac} {\delta^b}_d + \delta_{ad} {\delta^b}_c
- {\delta_a}^b \delta_{cd} + {f_{ac}}^e {f_{de}}^b + {f_{ad}}^e {f_{ce}}^b.
}{1}\coordE{}\end{eqnarray}
These relations correspond to the matrix identities
\begin{eqnarray}\coord{}\boxAlignEqnarray{
&&\leftCoord{}[A*B,C] + [B*C,A] + [C*A,B] = 0 \rightCoord{}\label{ego1} \rightCoord{}\\
&&\leftCoord{}A*[B,C] + B*[A,C] + [C,A*B] = 0 \rightCoord{}\label{ego2} \rightCoord{}\\
&&\leftCoord{}{[A,[B,C]]} = \frac{\leftCoord{}2}{\rightCoord{}9} (A,B) C - \frac{\leftCoord{}2}{\rightCoord{}9} (A,C) B + C*(A*B) - B*(A*C) \rightCoord{}\label{ego3} \rightCoord{}\\
&&\leftCoord{}A*(B*C) + B*(C*A) + C*(A*B) = \frac{\leftCoord{}1}{\rightCoord{}9} (A,B) C + \frac{\leftCoord{}1}{\rightCoord{}9} (B,C) A \nonumber\rightCoord{} \\\leftCoord{}* \rightCoord{}
&&\leftCoord{} \hspace{18.9em} + \frac{\leftCoord{}1}{\rightCoord{}9} (A,C) B \rightCoord{}\label{ego4} \rightCoord{}\\
&&\leftCoord{} 3 \rightCoord{}\, A*(B*C) = \frac{\leftCoord{}1}{\rightCoord{}3} (A,C) B + \frac{\leftCoord{}1}{\rightCoord{}3} (A,B) C - \frac{\leftCoord{}1}{\rightCoord{}3} (B,C) A + [[A,C],B] \nonumber\rightCoord{} \\\leftCoord{}* \rightCoord{}
&&\leftCoord{} \hspace{7em} + [[A,B],C], \rightCoord{}\label{ego5}
\rightCoord{}}{0mm}{17}{24}{
&&[A*B,C] + [B*C,A] + [C*A,B] = 0 \\
&&A*[B,C] + B*[A,C] + [C,A*B] = 0 \\
&&{[A,[B,C]]} = \frac{2}{9} (A,B) C - \frac{2}{9} (A,C) B + C*(A*B) - B*(A*C) \\
&&A*(B*C) + B*(C*A) + C*(A*B) = \frac{1}{9} (A,B) C + \frac{1}{9} (B,C) A \\* 
&& \hspace{18.9em} + \frac{1}{9} (A,C) B \\
&& 3 \, A*(B*C) = \frac{1}{3} (A,C) B + \frac{1}{3} (A,B) C - \frac{1}{3} (B,C) A + [[A,C],B] \\* 
&& \hspace{7em} + [[A,B],C], }{1}\coordE{}\end{eqnarray}
equation (\ref{id1}) giving rise to both of the relations
(\ref{ego1})-(\ref{ego2}). Modifying the conventions of ref. \cite{msw} by some
numerical factors I define two invariants of the algebra
\begin{eqnarray}\coord{}\boxAlignEqnarray{\leftCoord{} 
I_2(A) &=& |A|^2 \rightCoord{}\label{inv2} \\\leftCoord{}*
I_3(A) &=& (A,A*A) = 36 \det A. \rightCoord{}\label{inv3}
\rightCoord{}}{0mm}{2}{4}{ 
I_2(A) &=& |A|^2 \\*
I_3(A) &=& (A,A*A) = 36 \det A. }{1}\coordE{}\end{eqnarray}
They remain unchanged under the adjoint action of the group
\begin{equation}\coord{}\boxEquation{
A \rightarrow \Omega^{\dagger} A \Omega, \qquad \Omega \in {\rm SU(3)}.
\label{trans}
}{
A \rightarrow \Omega^{\dagger} A \Omega, \qquad \Omega \in {\rm SU(3)}.
}{ecuacion}\coordE{}\end{equation}
Given a matrix \myHighlight{$A$}\coordHE{} one can define, following ref. \cite{msw}, another matrix \myHighlight{$\widehat{A}$}\coordHE{}
\begin{equation}\coord{}\boxEquation{ \label{hat}
\widehat{A} = I_3(A) A - I_2(A) A*A
}{ \widehat{A} = I_3(A) A - I_2(A) A*A
}{ecuacion}\coordE{}\end{equation}
with the properties
$$\coord{}\boxMath{[A,\widehat{A}] = 0, \qquad (A,\widehat{A}) = 0.}{dollar}{0pt}\coordE{}$$
This suggests that we should define a third invariant by
\begin{equation}\coord{}\boxEquation{ \label{inv+}
I_8(A) = |\widehat{A}|^2 = I_2(A) \left( \frac{1}{9} I_2(A)^3 - I_3(A)^2 \right).
}{ I_8(A) = |\widehat{A}|^2 = I_2(A) \left( \frac{1}{9} I_2(A)^3 - I_3(A)^2 \right).
}{ecuacion}\coordE{}\end{equation}
Diagonalising \myHighlight{$A$}\coordHE{} with a transformation of the form (\ref{trans}) one can see
that \myHighlight{$I_8$}\coordHE{} vanishes if and only if \myHighlight{$A$}\coordHE{} has two coinciding eigenvalues. In the
generic case \myHighlight{$I_8$}\coordHE{} is strictly positive, though.

\section{Outline of solution}

In order to solve the Gauss law with sources
\begin{equation}\coord{}\boxEquation{\label{gauss}
\sum_{k=1}^3 \nabla_k(A) E_k = J_0
}{\sum_{k=1}^3 \nabla_k(A) E_k = J_0
}{ecuacion}\coordE{}\end{equation}
I take an ansatz of the form
\begin{equation}\coord{}\boxEquation{\label{ans}
E_k = \sum_{l,m=1}^3 \varepsilon_{klm} \nabla_l(A) C_m + \nabla_k(A) \phi
}{E_k = \sum_{l,m=1}^3 \varepsilon_{klm} \nabla_l(A) C_m + \nabla_k(A) \phi
}{ecuacion}\coordE{}\end{equation}
with the covariant derivative \myHighlight{$\nabla_k(A)$}\coordHE{} defined in equation (\ref{gsf}). Analogously with the ordinary Hodge decomposition I define \myHighlight{$\phi$}\coordHE{} as a solution to the covariant Poisson equation
\begin{equation}\coord{}\boxEquation{ \label{pois}
\sum_{k=1}^3 \nabla_k^2(A) \phi = J_0. 
}{ \sum_{k=1}^3 \nabla_k^2(A) \phi = J_0. 
}{ecuacion}\coordE{}\end{equation}
I will skip the analysis of this equation and assume that it admits solutions under some fairly general conditions. Majumdar and Sharatchandra also included a covariant gradient term in their ansatz for the source-free Gauss law \cite{ms1}, but their subsequent calculations \cite{ms2} indicate that the gradient degrees of freedom are generically redundant. In the Appendix I will discuss the question whether the ansatz (\ref{ans}) contains enough degrees of freedom to cover the space of colour-electric fields, but for the moment I take the ansatz (\ref{ans}) for granted. Combining now Gauss's law with the covariant divergence of equation (\ref{ans}) yields
\begin{equation}\coord{}\boxEquation{
\sum_{k=1}^3 ig [B_k,C_k] = 0, \label{ksum}
}{
\sum_{k=1}^3 ig [B_k,C_k] = 0, }{ecuacion}\coordE{}\end{equation}
where \myHighlight{$B_k$}\coordHE{} is the colour-magnetic field
$$\coord{}\boxMath{ B_k = \sum_{l,m=1}^3 \varepsilon_{klm} \left( \partial_l A_m + \frac{1}{2} ig
[A_l,A_m] \right). }{dollar}{0pt}\coordE{}$$
Equation (\ref{ksum}) could be solved by converting it into a system of real-valued equations and applying standard tools of linear algebra such as the Gauss elimination method. However, the  elimination procedure would give very little insight into the algebraic nature of equation (\ref{ksum}) and the solution obtained in this way would be complicated and formal. For this reason I prefer a less straightforward method which gives simpler solutions and makes their algebraic features more transparent. To begin with, let us parametrise the images of the commutators appearing in equation (\ref{ksum}). More precisely, each commutator takes a matrix value
\begin{equation}\coord{}\boxEquation{\label{keq}
ig [B_k,C_k] = F_k,
}{ig [B_k,C_k] = F_k,
}{ecuacion}\coordE{}\end{equation}
where \myHighlight{$F_k$}\coordHE{} must satisfy certain consistency conditions so that equation
(\ref{keq}) can be solved for \myHighlight{$C_k$}\coordHE{}. Making use of the following property of
the inner product
$$\coord{}\boxMath{ (X,i[B_k,C_k]) = -(i[B_k,X],C_k) }{dollar}{0pt}\coordE{}$$
we see that \myHighlight{$F_k$}\coordHE{} must be orthogonal to all matrices which commute with \myHighlight{$B_k$}\coordHE{}. I
will solve equation (\ref{keq}) properly in the next chapter, and it will
turn out that in the generic case the solvability conditions read
\begin{equation}\coord{}\boxEquation{\label{solv}
(F_k,B_k)=0, \qquad (F_k,\widehat{B}_k)=0,
}{(F_k,B_k)=0, \qquad (F_k,\widehat{B}_k)=0,
}{ecuacion}\coordE{}\end{equation}
where \myHighlight{$\widehat{B}_k$}\coordHE{} is defined according to equation (\ref{hat}). The geometric content 
of equations (\ref{ksum}) - (\ref{solv}) becomes clearer if we start regarding each matrix 
of the SU(3) algebra as an octet vector. The problem of parametrising the solutions 
of equation (\ref{ksum}) is then reduced to parametrising all possible sets of 
three vectors \myHighlight{$F_k$}\coordHE{} which satisfy the equation
\begin{equation}\coord{}\boxEquation{\label{fsum}
\sum_{k=1}^3 F_k = 0
}{\sum_{k=1}^3 F_k = 0
}{ecuacion}\coordE{}\end{equation}
and the orthogonality conditions (\ref{solv}). This task is simplified by a suitable 
choice of a basis for the SU(3) algebra. Generically, the following set of vectors 
will serve as a basis:
\begin{equation}\coord{}\boxEquation{\label{basis}
\left\{ \begin{array}{ll}
 i[B_k,B_l], & k < l \\
 i[B_k,\widehat{B}_l] + i[\widehat{B}_k,B_l], & k < l \\
 \chi_1, \chi_2. & \end{array} \right.
}{\left\{ \begin{array}{ll}
 i[B_k,B_l], & k < l \\
 i[B_k,\widehat{B}_l] + i[\widehat{B}_k,B_l], & k < l \\
 \chi_1, \chi_2. & \end{array} \right.
}{ecuacion}\coordE{}\end{equation}
Here \myHighlight{$\chi_1$}\coordHE{} and \myHighlight{$\chi_2$}\coordHE{} are some vectors which are orthogonal to all of the six vectors \myHighlight{$B_k$}\coordHE{} and \myHighlight{$\widehat{B}_k$}\coordHE{}. We can define them as determinants
$$\coord{}\boxMath{ \chi_j = \frac{1}{2} {\varepsilon_{a_1 \cdots a_6 b}}^c B_1^{a_1} \widehat{B}_1^{a_2} B_2^{a_3} \widehat{B}_2^{a_4} B_3^{a_5} \widehat{B}_3^{a_6} \eta_j^b \lambda_c, \quad j=1,2 }{dollar}{0pt}\coordE{}$$
where the \myHighlight{$\eta_j$}\coordHE{}'s are some constant octet vectors. Taking \myHighlight{$\eta_j$}\coordHE{} parallel to some Gell-Mann matrix \myHighlight{$\lambda_a$}\coordHE{} would reduce \myHighlight{$\chi_j$}\coordHE{} to a \myHighlight{$7 \times 7$}\coordHE{} determinant. To see the linear independence of the set (\ref{basis}) let us consider the equation
\begin{equation}\coord{}\boxEquation{\label{indep}
i\sum_{k<l} a_{kl} [B_k,B_l] + i\sum_{k<l} \widehat{a}_{kl} ( [B_k,\widehat{B}_l] + [\widehat{B}_k,B_l] ) + b_1 \chi_1 + b_2 \chi_2 = 0. 
}{i\sum_{k<l} a_{kl} [B_k,B_l] + i\sum_{k<l} \widehat{a}_{kl} ( [B_k,\widehat{B}_l] + [\widehat{B}_k,B_l] ) + b_1 \chi_1 + b_2 \chi_2 = 0. 
}{ecuacion}\coordE{}\end{equation}
Taking the inner product with respect to \myHighlight{$B_m$}\coordHE{} and \myHighlight{$\widehat{B}_m$}\coordHE{} leads to the following pair of equations
$$\coord{}\boxMath{ \left\{ \begin{array}{l} 
 \! a_{kl} (B_m,i[B_k,B_l]) + \widehat{a}_{kl} (B_m,i[B_k,\widehat{B}_l] + i[\widehat{B}_k,B_l]) = 0 \\
 \! a_{kl} (\widehat{B}_m,i[B_k,B_l]) + \widehat{a}_{kl} (\widehat{B}_m,i[B_k,\widehat{B}_l] + i[\widehat{B}_k,B_l]) = 0
 \end{array} \right. }{dollar}{0pt}\coordE{}$$
with \myHighlight{$m \neq k \neq l.$}\coordHE{} These equations have no non-trivial solutions if
\begin{eqnarray}\coord{}\boxAlignEqnarray{\leftCoord{} \rightCoord{}\label{notr}
&&\leftCoord{} \Bigl( (B_m,i[B_k,B_l]) (\widehat{B}_m,i[B_k,\widehat{B}_l] + i[\widehat{B}_k,B_l]) \\\leftCoord{}*
&&\leftCoord{}\hspace{1em} - (\widehat{B}_m,i[B_k,B_l]) (B_m,i[B_k,\widehat{B}_l] + i[\widehat{B}_k,B_l])  \Bigr) \neq 0. \rightCoord{}
\nonumber\rightCoord{} 
\rightCoord{}}{0mm}{4}{5}{ && \Bigl( (B_m,i[B_k,B_l]) (\widehat{B}_m,i[B_k,\widehat{B}_l] + i[\widehat{B}_k,B_l]) \\*
&&\hspace{1em} - (\widehat{B}_m,i[B_k,B_l]) (B_m,i[B_k,\widehat{B}_l] + i[\widehat{B}_k,B_l])  \Bigr) \neq 0. 
}{1}\coordE{}\end{eqnarray}
Generically this is true, since none of the identities (\ref{ego1}) - (\ref{ego5}) leads to the vanishing of the l.h.s. of condition (\ref{notr}). It is also possible to verify numerically, that is by assigning some test values to the vectors \myHighlight{$B_k$}\coordHE{}, that this expression does not vanish identically.
In the same way we see that the remaining coefficients \myHighlight{$b_1$}\coordHE{} and \myHighlight{$b_2$}\coordHE{} in equation (\ref{indep}) vanish if
$$\coord{}\boxMath{ (\chi_1,\chi_1) (\chi_2,\chi_2) - \left[ (\chi_1,\chi_2) \right]^2 \neq 0. }{dollar}{0pt}\coordE{}$$
As before, this is generically satisfied because the l.h.s. does not vanish identically. The linear independence of the set (\ref{basis}) thus proven, we use it as a basis for the vectors 
\myHighlight{$F_k$}\coordHE{}  
\begin{equation}\coord{}\boxEquation{ \label{fexp}
F_k = i\sum_{l=1 \atop l \neq k}^3 \left( \alpha_{kl} [B_k,B_l] + \widehat{\alpha}_{kl} ([B_k,\widehat{B}_l] + [\widehat{B}_k,B_l]) \right) + \beta_{1,k} \chi_1 + \beta_{2,k} \chi_2.
}{ F_k = i\sum_{l=1 \atop l \neq k}^3 \left( \alpha_{kl} [B_k,B_l] + \widehat{\alpha}_{kl} ([B_k,\widehat{B}_l] + [\widehat{B}_k,B_l]) \right) + \beta_{1,k} \chi_1 + \beta_{2,k} \chi_2.
}{ecuacion}\coordE{}\end{equation}
It should be noted that only six basis vectors are needed due to the orthogonality conditions (\ref{solv}). Substituting now these expansions into equation (\ref{fsum}) gives the following relations
\begin{eqnarray}\coord{}\boxAlignEqnarray{\leftCoord{} \rightCoord{}\label{rela}
&&\leftCoord{}\alpha_{kl} - \alpha_{lk} = 0, \qquad \widehat{\alpha}_{kl} - \widehat{\alpha}_{lk} = 0, \nonumber\rightCoord{} \\\leftCoord{}*
&&\leftCoord{}\hspace{0.9em} \sum_{\rightCoord{}k=1}^{\leftCoord{}3} \beta_{1,k} = 0, \qquad \sum_{\rightCoord{}k=1}^{\leftCoord{}3} \beta_{2,k} = 0.\rightCoord{}
\rightCoord{}}{0mm}{6}{7}{ &&\alpha_{kl} - \alpha_{lk} = 0, \qquad \widehat{\alpha}_{kl} - \widehat{\alpha}_{lk} = 0, \\*
&&\hspace{0.9em} \sum_{k=1}^{3} \beta_{1,k} = 0, \qquad \sum_{k=1}^{3} \beta_{2,k} = 0.
}{1}\coordE{}\end{eqnarray}
Let us finally count the number of degrees of freedom. Equation (\ref{rela}) states that the matrices \myHighlight{$\alpha$}\coordHE{} and \myHighlight{$\widehat{\alpha}$}\coordHE{} are symmetric. Since four of the six coefficients \myHighlight{$\beta_{i,k}$}\coordHE{} are independent, the total number of free variables is \myHighlight{$2 \times 3 + 2 \times 2 = 10$}\coordHE{}. This is the number of coordinates needed to parametrise three 6-dimensional vectors \myHighlight{$F_k$}\coordHE{} satisfying the 8-component equation (\ref{fsum}). We have thus found all solutions to equations (\ref{solv}) - (\ref{fsum}) in the generic case, expressed in the form of expansion (\ref{fexp}) satisfying relations (\ref{rela}). There are naturally exceptional cases where the set (\ref{basis}) fails to be a basis for the SU(3) algebra. Anyway, the method of solving equation (\ref{fsum}) remains the same, only the basis needs to be modified.

\section{Inverse of the commutator}

The vectors \myHighlight{$F_k$}\coordHE{} now known, it remains to solve equation (\ref{keq}) for the \myHighlight{$C_k$}\coordHE{}'s. Since the indices \myHighlight{$k$}\coordHE{} are fixed at this stage, I will omit them for a moment and consider the equation
\begin{equation}\coord{}\boxEquation{\label{nok}
ig [B,C] = F.
}{ig [B,C] = F.
}{ecuacion}\coordE{}\end{equation}
To obtain the solvability conditions for \myHighlight{$F$}\coordHE{} we must determine the zero modes of the commutator. For that purpose, let us express the l.h.s. of equation (\ref{nok}) using octet vector notation
\begin{equation}\coord{}\boxEquation{\label{adj}
[B,C]^a = i {M^a}_c \, C^c,
}{[B,C]^a = i {M^a}_c \, C^c,
}{ecuacion}\coordE{}\end{equation}
where
\begin{equation}\coord{}\boxEquation{\label{mdef}
{M^a}_c = {f_{bc}}^a B^b.
}{{M^a}_c = {f_{bc}}^a B^b.
}{ecuacion}\coordE{}\end{equation}
The characteristic polynomial of \myHighlight{$M$}\coordHE{} becomes simpler to evaluate if we diagonalise \myHighlight{$B$}\coordHE{} by a transformation of the form (\ref{trans})
\begin{equation}\coord{}\boxEquation{\label{diag}
\Omega^{\dagger} B \Omega = \frac{1}{2} b^3 \lambda_3 + \frac{1}{2} b^8 \lambda_8.
}{\Omega^{\dagger} B \Omega = \frac{1}{2} b^3 \lambda_3 + \frac{1}{2} b^8 \lambda_8.
}{ecuacion}\coordE{}\end{equation}
On the other hand, this transformation can equivalently be implemented by an orthogonal \myHighlight{$8 \times 8$}\coordHE{} matrix \myHighlight{$O$}\coordHE{}
\begin{equation}\coord{}\boxEquation{\label{uuu}
\left( \Omega^{\dagger} B \Omega \right)^a = {O^a}_b B^b.
}{\left( \Omega^{\dagger} B \Omega \right)^a = {O^a}_b B^b.
}{ecuacion}\coordE{}\end{equation}
Since the structure constants \myHighlight{${f_{ab}}^c$}\coordHE{} transform as a tensor, we have
\begin{eqnarray*}\coord{}\boxAlignEqnarray{\leftCoord{}
{\rightCoord{}\leftCoord{}M^a}_c &=& {O_b}^a {O^d}_c {O^e}_f {f_{ed}}^b \left( {O_3}^f \rightCoord{}\, b^3 + {O_8}^f \rightCoord{}\, b^8 
 \right) \\\leftCoord{}* \rightCoord{}
&\leftCoord{}=& {O_b}^a {\widetilde{M}^b}_{\rightCoord{}\,\rightCoord{}\,\rightCoord{}\,\rightCoord{}\, d} {O^d}_c, \rightCoord{}\\\leftCoord{}
{\rightCoord{}\leftCoord{}\widetilde{M}^b}_{\rightCoord{}\,\rightCoord{}\,\rightCoord{}\,\rightCoord{}\, d} &=& b^3 {f_{3d}}^b + b^8 {f_{8d}}^b. \rightCoord{}
\rightCoord{}}{0mm}{6}{17}{
{M^a}_c &=& {O_b}^a {O^d}_c {O^e}_f {f_{ed}}^b \left( {O_3}^f \, b^3 + {O_8}^f \, b^8 
 \right) \\* 
&=& {O_b}^a {\widetilde{M}^b}_{\,\,\,\, d} {O^d}_c, \\
{\widetilde{M}^b}_{\,\,\,\, d} &=& b^3 {f_{3d}}^b + b^8 {f_{8d}}^b. 
}{1}\coordE{}\end{eqnarray*}
A straightforward calculation now gives
\begin{eqnarray*}\coord{}\boxAlignEqnarray{
&&\leftCoord{} \det \left( M - x \rightCoord{}\, {\bf 1}_{8 \times 8} \right) =  \det \left( \widetilde{M} - x \rightCoord{}\, {\bf 1}_{8 \times 8} \right) \\\leftCoord{}*
&&\leftCoord{} \hspace{1.4em} = \rightCoord{}\, x^2 \Bigl\{ x^6 + \frac{\leftCoord{}3}{\rightCoord{}2} \left[ (b^3)^2 + (b^8)^2 \right] x^4 + \frac{\leftCoord{}9}{\rightCoord{}16} \left[ (b^3)^2 + (b^8)^2 \right]^2 x^2 \rightCoord{}\\
&&\leftCoord{} \hspace{4.1em} + \frac{\leftCoord{}1}{\rightCoord{}16} (b^3)^2 \left[ (b^3)^2 - 3 (b^8)^2 \right]^2 \Bigr\}. \rightCoord{}
\rightCoord{}}{0mm}{7}{10}{
&& \det \left( M - x \, {\bf 1}_{8 \times 8} \right) =  \det \left( \widetilde{M} - x \, {\bf 1}_{8 \times 8} \right) \\*
&& \hspace{1.4em} = \, x^2 \Bigl\{ x^6 + \frac{3}{2} \left[ (b^3)^2 + (b^8)^2 \right] x^4 + \frac{9}{16} \left[ (b^3)^2 + (b^8)^2 \right]^2 x^2 \\
&& \hspace{4.1em} + \frac{1}{16} (b^3)^2 \left[ (b^3)^2 - 3 (b^8)^2 \right]^2 \Bigr\}. 
}{1}\coordE{}\end{eqnarray*}
With the help of the invariants (\ref{inv2}) and (\ref{inv+}),
\begin{eqnarray*}\coord{}\boxAlignEqnarray{\leftCoord{}
I_2(B) &=& 3 \left[ (b^3)^2 + (b^8)^2 \right] \\\leftCoord{}*
I_8(B) &=& 9 (b^3)^2 \left[ (b^3)^2 + (b^8)^2 \right] \left[ (b^3)^2 - 3 (b^8)^2 \right]^2,
\rightCoord{}}{0mm}{2}{2}{
I_2(B) &=& 3 \left[ (b^3)^2 + (b^8)^2 \right] \\*
I_8(B) &=& 9 (b^3)^2 \left[ (b^3)^2 + (b^8)^2 \right] \left[ (b^3)^2 - 3 (b^8)^2 \right]^2,
}{1}\coordE{}\end{eqnarray*}
the characteristic polynomial can be written as 
$$\coord{}\boxMath{ \det \left( M - x \, {\bf 1}_{8 \times 8} \right) = x^2 \left[ x^2 \left( x^2 + \frac{1}{4} I_2 \right)^2  + 
 \frac{1}{48} \frac{I_8}{I_2} \right]. }{dollar}{0pt}\coordE{}$$
The commutators thus fall into three classes according to the number of zero modes:
\begin{enumerate}
\item
\myHighlight{$I_2 > 0$}\coordHE{}, \myHighlight{$I_8 > 0$}\coordHE{} \hfill \break
This is the generic case, when all the eigenvalues of \myHighlight{$B$}\coordHE{} are unequal. The zero modes are given by \myHighlight{$B$}\coordHE{} itself and the vector \myHighlight{$\widehat{B}$}\coordHE{} defined in equation (\ref{hat}).
\item
\myHighlight{$I_2 > 0$}\coordHE{}, \myHighlight{$I_8 = 0$}\coordHE{} \hfill \break
In this case \myHighlight{$B$}\coordHE{} is nonvanishing but has two coinciding eigenvalues. One can conjugate \myHighlight{$B$}\coordHE{} into the direction of \myHighlight{$\lambda_8$}\coordHE{}:
\begin{equation}\coord{}\boxEquation{\label{bform}
B = \Omega \left( \frac{1}{2} b^8 \lambda_8 \right) \Omega^{\dagger}.
}{B = \Omega \left( \frac{1}{2} b^8 \lambda_8 \right) \Omega^{\dagger}.
}{ecuacion}\coordE{}\end{equation}
There are four zero modes, given by the matrices
$$\coord{}\boxMath{ \Omega \left( \frac{1}{2} \lambda_a \right)  \Omega^{\dagger}, \qquad a=1,2,3,8.}{dollar}{0pt}\coordE{}$$
\item
This case is trivial, because \myHighlight{$B$}\coordHE{} vanishes.
\end{enumerate}
Let us now solve equation (\ref{nok}) in the generic case. There are solutions only if \myHighlight{$F$}\coordHE{} is orthogonal to the zero modes of the commutator
\begin{equation}\coord{}\boxEquation{\label{cond}
(F,B)=0, \qquad (F,\widehat{B})=0.
}{(F,B)=0, \qquad (F,\widehat{B})=0.
}{ecuacion}\coordE{}\end{equation}
Introducing a projection operator
\begin{equation}\coord{}\boxEquation{\label{projo}
\Pi(F) = F - \frac{1}{I_2(B)} (B,F) B - \frac{1}{I_8(B)} (\widehat{B},F) \widehat{B},
}{\Pi(F) = F - \frac{1}{I_2(B)} (B,F) B - \frac{1}{I_8(B)} (\widehat{B},F) \widehat{B},
}{ecuacion}\coordE{}\end{equation}
the conditions on \myHighlight{$F$}\coordHE{} can also be expressed as a single equation
$$\coord{}\boxMath{ F = \Pi(F). }{dollar}{0pt}\coordE{}$$
Equation (\ref{nok}) can thus be replaced by
\begin{equation}\coord{}\boxEquation{\label{preq}
ig [B,C] = \Pi(F).
}{ig [B,C] = \Pi(F).
}{ecuacion}\coordE{}\end{equation}
The general form of the solution \myHighlight{$C$}\coordHE{} is
\begin{equation}\coord{}\boxEquation{\label{form}
C^a = {X^a}_b F^b,
}{C^a = {X^a}_b F^b,
}{ecuacion}\coordE{}\end{equation}
where the tensor \myHighlight{$X$}\coordHE{} depends only on \myHighlight{$B$}\coordHE{}, because \myHighlight{$C$}\coordHE{} must be linear in \myHighlight{$F$}\coordHE{}. The basis for all such tensors was given in ref. \cite{msw}, and substituting it into equation (\ref{form}) yields the following ansatz
\begin{eqnarray}\coord{}\boxAlignEqnarray{\leftCoord{}\rightCoord{}\label{ansatz}
C &=& a_1 F + a_2 B*F + a_3 \widehat{B}*F + a_4 [B,F] + a_5 [\widehat{B},F] + a_6 B*[\widehat{B},F] \nonumber\rightCoord{} \\\leftCoord{}*
&&\leftCoord{} + a_7 B + a_8 \widehat{B}.\rightCoord{}
\rightCoord{}}{0mm}{3}{5}{C &=& a_1 F + a_2 B*F + a_3 \widehat{B}*F + a_4 [B,F] + a_5 [\widehat{B},F] + a_6 B*[\widehat{B},F] \\*
&& + a_7 B + a_8 \widehat{B}.
}{1}\coordE{}\end{eqnarray}
Using identities (\ref{ego1}) - (\ref{ego5}) the commutator of this expression becomes
\begin{eqnarray}\coord{}\boxAlignEqnarray{\leftCoord{}\rightCoord{}\label{kommu}
\leftCoord{}[B,C] &=& \left( \frac{\leftCoord{}I_2}{\rightCoord{}6} a_4 - \frac{\leftCoord{}I_8}{\rightCoord{}12 \rightCoord{}\, I_2} a_6 \right) F + \left( \frac{\leftCoord{}3 \rightCoord{}\, I_3}{\rightCoord{}2 \rightCoord{}\, I_2} a_4 -
\frac{\leftCoord{}3 \rightCoord{}\, I_8}{\rightCoord{}2 \rightCoord{}\, I_2^2} a_5 \right) B*F \nonumber \rightCoord{}\\
&&\leftCoord{} - \left( \frac{\leftCoord{}3}{\rightCoord{}2 \rightCoord{}\, I_2} a_4 + \frac{\leftCoord{}3 \rightCoord{}\, I_3}{\rightCoord{}2 \rightCoord{}\, I_2} a_5 + \frac{\leftCoord{}I_2}{\rightCoord{}12} a_6 \right) \widehat{B}*F \nonumber \rightCoord{}\\
&&\leftCoord{} + \left( a_1 + \frac{\leftCoord{}I_3}{\rightCoord{}2 \rightCoord{}\, I_2} a_2 - \frac{\leftCoord{}I_8}{\rightCoord{}I_2^2} a_3 \right) [B,F] \rightCoord{}\\
&&\leftCoord{} -  \left( \frac{\leftCoord{}1}{\rightCoord{}2 \rightCoord{}\, I_2} a_2 + \frac{\leftCoord{}I_3}{\rightCoord{}I_2} a_3 \right) [\widehat{B},F] - a_3 \rightCoord{}\, B*[\widehat{B},F] \nonumber \rightCoord{}\\
&&\leftCoord{} - \frac{\leftCoord{}1}{\rightCoord{}3} a_4 \rightCoord{}\, (B,F) B + \frac{\leftCoord{}1}{\rightCoord{}6 \rightCoord{}\, I_2} a_6 \rightCoord{}\, (\widehat{B},F) \widehat{B} \nonumber \rightCoord{}\\
&&\leftCoord{} - \left( \frac{\leftCoord{}1}{\rightCoord{}6} a_5 + \frac{\leftCoord{}I_3}{\rightCoord{}12 \rightCoord{}\, I_2} a_6 \right) \left( (\widehat{B},F) B + (B,F) \widehat{B} \right). \nonumber\rightCoord{}
\rightCoord{}}{0mm}{22}{39}{[B,C] &=& \left( \frac{I_2}{6} a_4 - \frac{I_8}{12 \, I_2} a_6 \right) F + \left( \frac{3 \, I_3}{2 \, I_2} a_4 -
\frac{3 \, I_8}{2 \, I_2^2} a_5 \right) B*F \\
&& - \left( \frac{3}{2 \, I_2} a_4 + \frac{3 \, I_3}{2 \, I_2} a_5 + \frac{I_2}{12} a_6 \right) \widehat{B}*F \\
&& + \left( a_1 + \frac{I_3}{2 \, I_2} a_2 - \frac{I_8}{I_2^2} a_3 \right) [B,F] \\
&& -  \left( \frac{1}{2 \, I_2} a_2 + \frac{I_3}{I_2} a_3 \right) [\widehat{B},F] - a_3 \, B*[\widehat{B},F] \\
&& - \frac{1}{3} a_4 \, (B,F) B + \frac{1}{6 \, I_2} a_6 \, (\widehat{B},F) \widehat{B} \\
&& - \left( \frac{1}{6} a_5 + \frac{I_3}{12 \, I_2} a_6 \right) \left( (\widehat{B},F) B + (B,F) \widehat{B} \right). }{1}\coordE{}\end{eqnarray}
Inserting expansions (\ref{kommu}) and (\ref{projo}) into equation (\ref{preq}) and equating terms of
the same form determines six of the coefficients \myHighlight{$a_i$}\coordHE{}
\begin{eqnarray*}\coord{}\boxAlignEqnarray{
&&\leftCoord{} a_1 = 0, \qquad a_2 = 0, \qquad a_3 = 0, \\\leftCoord{}*
&&\leftCoord{} a_4 = - \frac{\leftCoord{}i}{\rightCoord{}g} \frac{\leftCoord{}3}{\rightCoord{}I_2}, \qquad a_5 = - \frac{\leftCoord{}i}{\rightCoord{}g} \frac{\leftCoord{}3 \rightCoord{}\, I_3}{\rightCoord{} I_8}, \qquad a_6 = \frac{\leftCoord{}i}{\rightCoord{}g} \frac{\leftCoord{}6 \rightCoord{}\, I_2}{\rightCoord{}I_8}.\rightCoord{}
\rightCoord{}}{0mm}{9}{11}{
&& a_1 = 0, \qquad a_2 = 0, \qquad a_3 = 0, \\*
&& a_4 = - \frac{i}{g} \frac{3}{I_2}, \qquad a_5 = - \frac{i}{g} \frac{3 \, I_3}{ I_8}, \qquad a_6 = \frac{i}{g} \frac{6 \, I_2}{I_8}.
}{1}\coordE{}\end{eqnarray*}
Hence, the solution to equation (\ref{preq}) is
\begin{equation}\coord{}\boxEquation{\label{solu}
C = - \frac{3i}{g} \left( \frac{1}{I_2} [B,F] + \frac{I_3}{I_8} [\widehat{B},F] - \frac{2 \, I_2}{I_8} B*[\widehat{B},F] \right)
+ a_7 B + a_8 \widehat{B}.
}{C = - \frac{3i}{g} \left( \frac{1}{I_2} [B,F] + \frac{I_3}{I_8} [\widehat{B},F] - \frac{2 \, I_2}{I_8} B*[\widehat{B},F] \right)
+ a_7 B + a_8 \widehat{B}.
}{ecuacion}\coordE{}\end{equation}
This formula becomes singular when \myHighlight{$I_8$}\coordHE{} tends to zero. In this limit \myHighlight{$B$}\coordHE{} can be written in the form (\ref{bform}). The orthogonality conditions on \myHighlight{$F$}\coordHE{} are
\begin{equation}\coord{}\boxEquation{ \label{excon}
\left( \Omega \left( \frac{1}{2} \lambda_a \right)  \Omega^{\dagger},F \right) = 0, \qquad a = 1,2,3,8. 
}{ \left( \Omega \left( \frac{1}{2} \lambda_a \right)  \Omega^{\dagger},F \right) = 0, \qquad a = 1,2,3,8. 
}{ecuacion}\coordE{}\end{equation}
When these requirements are fullfilled, it is straightforward to see that the following expression satisfies equation (\ref{nok})
\begin{equation}\coord{}\boxEquation{\label{except}
C = - \frac{4i}{g} \frac{1}{I_2} [B,F]  + \Omega \Bigl( \frac{1}{2} a^1 \lambda_1 + \frac{1}{2} a^2 \lambda_2  + \frac{1}{2} a^3 \lambda_3 + \frac{1}{2} a^8 \lambda_8 \Bigr) \Omega^{\dagger}.
}{C = - \frac{4i}{g} \frac{1}{I_2} [B,F]  + \Omega \Bigl( \frac{1}{2} a^1 \lambda_1 + \frac{1}{2} a^2 \lambda_2  + \frac{1}{2} a^3 \lambda_3 + \frac{1}{2} a^8 \lambda_8 \Bigr) \Omega^{\dagger}.
}{ecuacion}\coordE{}\end{equation}
Formulas (\ref{solu}) and (\ref{except}) thus solve the commutator equation (\ref{nok}) in the two nontrivial cases.

\section{Results}

We are now ready to write down the general solution to equation (\ref{ksum}). In the generic case the expansions (\ref{fexp}) parametrise all possible values for the commutators (\ref{keq}). Substituting these expansions into equation (\ref{solu}) and simplifying the result with identities (\ref{ego1}) - (\ref{ego5}) yields the solution 
\begin{eqnarray}\coord{}\boxAlignEqnarray{\leftCoord{}\rightCoord{}\label{res}
C_k &=& \frac{\leftCoord{}1}{\rightCoord{}g} \sum_{\rightCoord{}l=1 \atop l \neq k}^{\leftCoord{}3} \Bigl[ (\alpha_{kl} + I_3^{(k)} \widehat{\alpha}_{kl}) B_l + \widehat{\alpha}_{kl} \left( \widehat{B}_l - 2 I_2^{(k)} B_k * B_l \right) \Bigr] \nonumber\rightCoord{} \\\leftCoord{}*
&&\leftCoord{} + \frac{\leftCoord{}1}{\rightCoord{}g} \sum_{\rightCoord{}j=1}^{\leftCoord{}2} \beta_{j,k} \psi_{j,k} + \gamma_k B_k + \widehat{\gamma}_k \widehat{B}_k,
\rightCoord{}}{0mm}{7}{8}{C_k &=& \frac{1}{g} \sum_{l=1 \atop l \neq k}^{3} \Bigl[ (\alpha_{kl} + I_3^{(k)} \widehat{\alpha}_{kl}) B_l + \widehat{\alpha}_{kl} \left( \widehat{B}_l - 2 I_2^{(k)} B_k * B_l \right) \Bigr] \\*
&& + \frac{1}{g} \sum_{j=1}^{2} \beta_{j,k} \psi_{j,k} + \gamma_k B_k + \widehat{\gamma}_k \widehat{B}_k,
}{1}\coordE{}\end{eqnarray}
where
\begin{eqnarray}\coord{}\boxAlignEqnarray{\leftCoord{} \rightCoord{}\label{psi}
\Pi_k (\psi_{j,k}) &=& -3i \Bigl( \frac{\leftCoord{}1}{\rightCoord{}I_2^{(k)}} [B_k, \chi_j] + \frac{\leftCoord{}I_3^{(k)}}{I_8^{(k)}} [\widehat{B}_k, \chi_j] - \frac{\leftCoord{}2 \rightCoord{}\, I_2^{(k)}}{\rightCoord{}I_8^{(k)}} B_k *[\widehat{B}_k,\chi_j] \Bigr), \nonumber\rightCoord{} \\\leftCoord{}*
I_i^{(k)} &\equiv& I_i(B_k) 
\rightCoord{}}{0mm}{5}{7}{ \Pi_k (\psi_{j,k}) &=& -3i \Bigl( \frac{1}{I_2^{(k)}} [B_k, \chi_j] + \frac{I_3^{(k)}}{I_8^{(k)}} [\widehat{B}_k, \chi_j] - \frac{2 \, I_2^{(k)}}{I_8^{(k)}} B_k *[\widehat{B}_k,\chi_j] \Bigr), \\*
I_i^{(k)} &\equiv& I_i(B_k) 
}{1}\coordE{}\end{eqnarray}
and \myHighlight{$\Pi_k$}\coordHE{} stands for the projection operator of equation (\ref{projo}). The symmetry relations (\ref{rela}) must hold, while the zero mode coefficients \myHighlight{$\gamma_k$}\coordHE{} and \myHighlight{$\widehat{\gamma}_k$}\coordHE{} are arbitrary. I have left unsimplified those parts of the solution which correspond to the two vectors \myHighlight{$\chi_j$}\coordHE{}. Of course, simplifications can be performed using the results of ref. \cite{ammp}. In particular, it is shown there how the eighth rank permutation symbol can be expressed in a form involving only the structure constants \myHighlight{${f_{ab}}^c$}\coordHE{} and \myHighlight{${d_{ab}}^c$}\coordHE{}. Constructing all possible sixth rank tensors which are antisymmetric in five indices and contracting them with the vectors \myHighlight{$B_l$}\coordHE{}, \myHighlight{$\widehat{B}_l$}\coordHE{} (\myHighlight{$l \neq k$}\coordHE{}) and \myHighlight{$\eta_j$}\coordHE{} would give us the vectors needed to reduce the expression (\ref{psi}). Unfortunately there is such a large number of these tensors that the resulting formula would be unduly long. The unsimplified formula (\ref{psi}) is the shortest expression I have been able to obtain. Still there is a relatively simple formula at hand if we diagonalise \myHighlight{$B_k$}\coordHE{} with a transformation of the form (\ref{diag}). Making use of the fact that the eighth rank permutation symbol transforms as a tensor under the adjoint action (\ref{trans}) leads to
\begin{equation}\coord{}\boxEquation{ \label{sipsi}
\Pi_k (\psi_{j,k}) = \frac{1}{\sqrt{3}} I_2^{(k)} c^{ab} \Delta_{38b} \, \Omega \! \left( \frac{1}{2} \lambda_a \right) \! \Omega^{\dagger},
}{ \Pi_k (\psi_{j,k}) = \frac{1}{\sqrt{3}} I_2^{(k)} c^{ab} \Delta_{38b} \, \Omega \! \left( \frac{1}{2} \lambda_a \right) \! \Omega^{\dagger},
}{ecuacion}\coordE{}\end{equation}
where
\begin{eqnarray*}\coord{}\boxAlignEqnarray{\leftCoord{}
\Delta_{abc} &=& \varepsilon_{ab d_1 \cdots d_5 c} \left( \Omega^{\dagger} B_l \Omega \right)^{d_1} \left( \Omega^{\dagger} \widehat{B}_l \Omega \right)^{d_2} \left( \Omega^{\dagger} B_m \Omega \right)^{d_3} \\\leftCoord{}*
&&\leftCoord{} \hspace{2.7em} \times \left( \Omega^{\dagger} \widehat{B}_m \Omega \right)^{d_4} \left( \Omega^{\dagger} \eta_j \Omega \right)^{d_5}, \qquad l \neq m \neq k, \rightCoord{}\\\leftCoord{}
c^{12} &=& - c^{21} = \frac{\leftCoord{}1}{\rightCoord{}3} [(b_k^3)^2 - 3(b_k^8)^2] \rightCoord{}\\\leftCoord{} 
c^{45} &=& - c^{54} = \frac{\leftCoord{}2}{\rightCoord{}3} b_k^3 (b_k^3 - \sqrt{3} \rightCoord{}\, b_k^8) \rightCoord{}\\\leftCoord{}
c^{67} &=& - c^{76} = -\frac{\leftCoord{}2}{\rightCoord{}3} b_k^3 (b_k^3 + \sqrt{3} \rightCoord{}\, b_k^8) 
\rightCoord{}}{0mm}{9}{10}{
\Delta_{abc} &=& \varepsilon_{ab d_1 \cdots d_5 c} \left( \Omega^{\dagger} B_l \Omega \right)^{d_1} \left( \Omega^{\dagger} \widehat{B}_l \Omega \right)^{d_2} \left( \Omega^{\dagger} B_m \Omega \right)^{d_3} \\*
&& \hspace{2.7em} \times \left( \Omega^{\dagger} \widehat{B}_m \Omega \right)^{d_4} \left( \Omega^{\dagger} \eta_j \Omega \right)^{d_5}, \qquad l \neq m \neq k, \\
c^{12} &=& - c^{21} = \frac{1}{3} [(b_k^3)^2 - 3(b_k^8)^2] \\ 
c^{45} &=& - c^{54} = \frac{2}{3} b_k^3 (b_k^3 - \sqrt{3} \, b_k^8) \\
c^{67} &=& - c^{76} = -\frac{2}{3} b_k^3 (b_k^3 + \sqrt{3} \, b_k^8) 
}{1}\coordE{}\end{eqnarray*}
and all the other components of the matrix \myHighlight{$c$}\coordHE{} vanish. I have suppressed the indices \myHighlight{$k$}\coordHE{} and \myHighlight{$j$}\coordHE{} on the r.h.s. of equation (\ref{sipsi}) so as to keep the notation readable. 

In order to avoid singularities in the limit when two eigenvalues of \myHighlight{$B_k$}\coordHE{} coincide we must find a way to regularise the vectors \myHighlight{$\psi_{j,k}$}\coordHE{}. This singularity is present in equation (\ref{solu}), but it does not mean that the solution (\ref{res}) would have to be singular. Actually, even the first six vectors of the basis (\ref{basis}), when inserted into equation (\ref{solu}), produced singular terms, but these terms were proportional to the zero modes \myHighlight{$B_k$}\coordHE{} and \myHighlight{$\widehat{B}_k$}\coordHE{}. The singularities could then be removed by redefining the zero mode coefficients \myHighlight{$\gamma_k$}\coordHE{} and \myHighlight{$\widehat{\gamma}_k$}\coordHE{}, and the same procedure can also be applied to the vectors \myHighlight{$\psi_{j,k}$}\coordHE{}. Specifically, let us define
\begin{equation}\coord{}\boxEquation{ \label{repsi}
\psi_{j,k} = - \frac{1}{2 \sqrt{3}} \, I_2^{(k)} \tilde{c}^a \, \Omega \! \left( \frac{1}{2} \lambda_a \right) \! \Omega^{\dagger},
}{ \psi_{j,k} = - \frac{1}{2 \sqrt{3}} \, I_2^{(k)} \tilde{c}^a \, \Omega \! \left( \frac{1}{2} \lambda_a \right) \! \Omega^{\dagger},
}{ecuacion}\coordE{}\end{equation}
with
\begin{eqnarray*}\coord{}\boxAlignEqnarray{\leftCoord{}
\tilde{c}^a &=& \frac{\leftCoord{}1}{\rightCoord{}3} [(b_k^3)^2 - 3(b_k^8)^2] \rightCoord{}\, \varepsilon_{123}^{abc} \Delta_{b8c} \rightCoord{}\\
&&\leftCoord{} -\frac{\leftCoord{}2}{\rightCoord{}3} b_k^3 (b_k^3 - \sqrt{3} \rightCoord{}\, b_k^8) \left( \frac{\leftCoord{}1}{\rightCoord{}2} \varepsilon_{453}^{abc} + \frac{\leftCoord{}\sqrt{3}}{2} \varepsilon_{458}^{abc} \right) \left( \frac{\leftCoord{}\sqrt{3}}{2} \Delta_{b3c} - \frac{\leftCoord{}1}{\rightCoord{}2} \Delta_{b8c} \right) \rightCoord{}\\
&&\leftCoord{} -\frac{\leftCoord{}2}{\rightCoord{}3} b_k^3 (b_k^3 + \sqrt{3} \rightCoord{}\, b_k^8) \left( -\frac{\leftCoord{}1}{\rightCoord{}2} \varepsilon_{673}^{abc} + \frac{\leftCoord{}\sqrt{3}}{2} \varepsilon_{678}^{abc} \right) \left( -\frac{\leftCoord{}\sqrt{3}}{2} \Delta_{b3c} - \frac{\leftCoord{}1}{\rightCoord{}2} \Delta_{b8c} \right),
\rightCoord{}}{0mm}{14}{14}{
\tilde{c}^a &=& \frac{1}{3} [(b_k^3)^2 - 3(b_k^8)^2] \, \varepsilon_{123}^{abc} \Delta_{b8c} \\
&& -\frac{2}{3} b_k^3 (b_k^3 - \sqrt{3} \, b_k^8) \left( \frac{1}{2} \varepsilon_{453}^{abc} + \frac{\sqrt{3}}{2} \varepsilon_{458}^{abc} \right) \left( \frac{\sqrt{3}}{2} \Delta_{b3c} - \frac{1}{2} \Delta_{b8c} \right) \\
&& -\frac{2}{3} b_k^3 (b_k^3 + \sqrt{3} \, b_k^8) \left( -\frac{1}{2} \varepsilon_{673}^{abc} + \frac{\sqrt{3}}{2} \varepsilon_{678}^{abc} \right) \left( -\frac{\sqrt{3}}{2} \Delta_{b3c} - \frac{1}{2} \Delta_{b8c} \right),
}{1}\coordE{}\end{eqnarray*}
where \myHighlight{$\varepsilon_{ijk}^{abc}$}\coordHE{} stands for the three-dimensional permutation symbol with indices taking the values \myHighlight{$i$}\coordHE{}, \myHighlight{$j$}\coordHE{} and \myHighlight{$k$}\coordHE{}. Equations (\ref{repsi}) and (\ref{sipsi}) are equivalent apart from terms which are proportional to the zero modes. We can now pass to the limit when two eigenvalues of \myHighlight{$B_k$}\coordHE{} coincide. Let us assume that the eigenvalues are ordered so that
$$\coord{}\boxMath{ b_k^3 + \frac{1}{\sqrt{3}} \, b_k^8 \geq -b_k^3 + \frac{1}{\sqrt{3}} \, b_k^8 \geq -\frac{2}{\sqrt{3}} \, b_k^8, }{dollar}{0pt}\coordE{}$$
which means that the largest eigenvalues coincide in the limit \myHighlight{$b_k^3 \rightarrow 0$}\coordHE{}. In this limit the vectors \myHighlight{$\psi_{j,k}$}\coordHE{} are reduced to
\begin{equation}\coord{}\boxEquation{ \label{su2}
\psi_{j,k} \rightarrow \frac{1}{6 \sqrt{3}} \left( I_2^{(k)} \right)^2 \varepsilon_{123}^{abc} \Delta_{b8c} \, \Omega \! \left( \frac{1}{2} \lambda_a \right) \! \Omega^{\dagger}.
}{ \psi_{j,k} \rightarrow \frac{1}{6 \sqrt{3}} \left( I_2^{(k)} \right)^2 \varepsilon_{123}^{abc} \Delta_{b8c} \, \Omega \! \left( \frac{1}{2} \lambda_a \right) \! \Omega^{\dagger}.
}{ecuacion}\coordE{}\end{equation}
In order to show that this expression is single-valued we must verify that it is invariant under transformations which also leave \myHighlight{$B_k$}\coordHE{} invariant. As \myHighlight{$B_k$}\coordHE{} now takes the form (\ref{bform}), the transformations in question are the SU(2)\myHighlight{$\times$}\coordHE{}U(1) reparametrisations of the matrix \myHighlight{$\Omega$}\coordHE{} defined by
\begin{eqnarray}\coord{}\boxAlignEqnarray{\leftCoord{} \rightCoord{}\label{repa}
&&\leftCoord{} \Omega \rightarrow \Omega \omega, \rightCoord{}\\ 
&&\leftCoord{} \omega = \exp \left[ \frac{\leftCoord{}i}{\rightCoord{}2} ( \theta^1 \lambda_1 + \theta^2 \lambda_2 + \theta^3 \lambda_3 + \theta^8 \lambda_8 ) \right]. \nonumber\rightCoord{} 
\rightCoord{}}{0mm}{4}{6}{ && \Omega \rightarrow \Omega \omega, \\ 
&& \omega = \exp \left[ \frac{i}{2} ( \theta^1 \lambda_1 + \theta^2 \lambda_2 + \theta^3 \lambda_3 + \theta^8 \lambda_8 ) \right]. }{1}\coordE{}\end{eqnarray}
These transformations take
\begin{eqnarray*}\coord{}\boxAlignEqnarray{
&&\leftCoord{}\lambda_a \rightarrow {P_a}^{a'} \rightCoord{}\, \lambda_{a'} \rightCoord{}\\
&&\leftCoord{}\Delta_{b8c} \rightarrow {P_b}^{b'} {P_c}^{c'} \Delta_{b'8c'}, \rightCoord{}\\
&&\leftCoord{}{P_a}^{a'} = \left[ \omega \left( \frac{\leftCoord{}1}{\rightCoord{}2} \lambda_a \right) \omega^{\dagger} \right]^{a'},
\rightCoord{}}{0mm}{4}{6}{
&&\lambda_a \rightarrow {P_a}^{a'} \, \lambda_{a'} \\
&&\Delta_{b8c} \rightarrow {P_b}^{b'} {P_c}^{c'} \Delta_{b'8c'}, \\
&&{P_a}^{a'} = \left[ \omega \left( \frac{1}{2} \lambda_a \right) \omega^{\dagger} \right]^{a'},
}{1}\coordE{}\end{eqnarray*}
and as a result
\begin{eqnarray*}\coord{}\boxAlignEqnarray{
&&\leftCoord{} \psi_{j,k} \rightarrow \left( \det_{3 \times 3} P \right) \rightCoord{}\, \psi_{j,k}, \rightCoord{}\\
&&\leftCoord{} \det_{3 \times 3} P = \varepsilon_{123}^{abc} {P_a}^{1} {P_b}^{2} {P_c}^{3}.\rightCoord{}
\rightCoord{}}{0mm}{2}{5}{
&& \psi_{j,k} \rightarrow \left( \det_{3 \times 3} P \right) \, \psi_{j,k}, \\
&& \det_{3 \times 3} P = \varepsilon_{123}^{abc} {P_a}^{1} {P_b}^{2} {P_c}^{3}.
}{1}\coordE{}\end{eqnarray*}
Since \myHighlight{$\det_{3 \times 3} P = 1$}\coordHE{} for transformations of the form (\ref{repa}), the solution (\ref{su2}) is invariant. Although this result was derived in the case when the largest eigenvalues of \myHighlight{$B_k$}\coordHE{} coincide, the form of the solution (\ref{repsi}) makes it evident that \myHighlight{$\psi_{j,k}$}\coordHE{} is really single-valued regardless of which SU(2) subgroup survives in the limit of coinciding eigenvalues. So there will be no singularities in the fields \myHighlight{$C_k$}\coordHE{} in equation (\ref{res}).

So far we have found out that equation (\ref{ksum}) possesses solutions which are regular everywhere in space. Yet it is possible that there might be some physically relevant degrees of freedom residing at points where two eigenvalues of \myHighlight{$B_k$}\coordHE{} coincide and that we should search for singular solutions to equation (\ref{ksum}) in order to detect these degrees of freedom. In fact, it is widely believed that there are singularities with local monopole-like behaviour in the vicinity of points where two eigenvalues coincide. Usually these singularities arise as a result of gauge fixing \cite{th}, but here they could emerge in connection with special solutions to equation (\ref{ksum}). To determine such solutions we need to modify the basis (\ref{basis}) slightly. I will consider the case when one component of the colour-magnetic field, say \myHighlight{$B_3$}\coordHE{}, has coinciding eigenvalues. \myHighlight{$B_3$}\coordHE{} takes the form (\ref{bform}) and in particular \myHighlight{$\widehat{B}_3 = 0$}\coordHE{}. (I am not assuming that the eigenvalues should be ordered this time.) Yet the first six vectors of the set (\ref{basis}) remain generically indepent while \myHighlight{$\chi_1$}\coordHE{} and \myHighlight{$\chi_2$}\coordHE{} vanish. Since \myHighlight{$F_3$}\coordHE{} in equation (\ref{keq}) now has to satisfy four orthogonality conditions according to equation (\ref{excon}), we see that the vectors \myHighlight{$\chi_j$}\coordHE{} should be replaced by two vectors \myHighlight{$\widetilde{\chi}_j$}\coordHE{} which are orthogonal to the space spanned by the set \myHighlight{$\{ B_1, \widehat{B}_1, B_2, \widehat{B}_2 \}$}\coordHE{}. We can take
$$\coord{}\boxMath{ \left\{ \begin{array}{l} 
\widetilde{\chi}_1 = i[\widehat{B}_1,\widehat{B}_2] \\
\widetilde{\chi}_2 = [B_1,B_2]*[\widehat{B}_1,\widehat{B}_2] - [B_1,\widehat{B}_2]*[\widehat{B}_1,B_2].
\end{array} \right. }{dollar}{0pt}\coordE{}$$
The expansion for \myHighlight{$F_k$}\coordHE{} now takes the form of equation (\ref{fexp}) satisfying the relations (\ref{rela}) with the obvious substitutions 
$$\coord{}\boxMath{ \left\{ \begin{array}{ll} \chi_j \rightarrow \widetilde{\chi}_j &\\
\beta_{j,k} \rightarrow \widetilde{\beta}_{j,k}, & k=1,2 \\
\beta_{j,3} \rightarrow 0. & \end{array} \right. }{dollar}{0pt}\coordE{}$$ 
Inserting these expansions into equations (\ref{solu}) and (\ref{except}) leads to the solution 
\begin{eqnarray}\coord{}\boxAlignEqnarray{\leftCoord{}\rightCoord{}\label{excres}
C_k &=& \frac{\leftCoord{}1}{\rightCoord{}g} \sum_{\rightCoord{}l=1 \atop l \neq k}^{\leftCoord{}3} \Bigl[ (\alpha_{kl} + I_3^{(k)} \widehat{\alpha}_{kl}) B_l + \widehat{\alpha}_{kl} \left( \widehat{B}_l - 2 I_2^{(k)} B_k * B_l \right) \Bigr] \nonumber\rightCoord{} \\\leftCoord{}*
&&\leftCoord{} + \frac{\leftCoord{}1}{\rightCoord{}g} \sum_{\rightCoord{}j=1}^{\leftCoord{}2} \widetilde{\beta}_{j,k} \widetilde{\psi}_{j,k} + \gamma_k B_k + \widehat{\gamma}_k \widehat{B}_k, \qquad k = 1,2 \nonumber \rightCoord{}\\\leftCoord{}
C_3 &=& \frac{\leftCoord{}1}{\rightCoord{}g} \sum_{\rightCoord{}l=1}^{\leftCoord{}2}  \left( \alpha_{3l} B_l + \widehat{\alpha}_{3l} \widehat{B}_l \right) \\\leftCoord{}*
&&\leftCoord{} + \Omega \Bigl( \frac{\leftCoord{}1}{\rightCoord{}2} \widetilde{\gamma}_3^1 \lambda_1 + \frac{\leftCoord{}1}{\rightCoord{}2} \widetilde{\gamma}_3^2 \lambda_2 + \frac{\leftCoord{}1}{\rightCoord{}2} \widetilde{\gamma}_3^3 \lambda_3 + \frac{\leftCoord{}1}{\rightCoord{}2} \widetilde{\gamma}_3^8 \lambda_8 \Bigr) \Omega^{\dagger}. \nonumber\rightCoord{}
\rightCoord{}}{0mm}{16}{16}{C_k &=& \frac{1}{g} \sum_{l=1 \atop l \neq k}^{3} \Bigl[ (\alpha_{kl} + I_3^{(k)} \widehat{\alpha}_{kl}) B_l + \widehat{\alpha}_{kl} \left( \widehat{B}_l - 2 I_2^{(k)} B_k * B_l \right) \Bigr] \\*
&& + \frac{1}{g} \sum_{j=1}^{2} \widetilde{\beta}_{j,k} \widetilde{\psi}_{j,k} + \gamma_k B_k + \widehat{\gamma}_k \widehat{B}_k, \qquad k = 1,2 \\
C_3 &=& \frac{1}{g} \sum_{l=1}^{2}  \left( \alpha_{3l} B_l + \widehat{\alpha}_{3l} \widehat{B}_l \right) \\*
&& + \Omega \Bigl( \frac{1}{2} \widetilde{\gamma}_3^1 \lambda_1 + \frac{1}{2} \widetilde{\gamma}_3^2 \lambda_2 + \frac{1}{2} \widetilde{\gamma}_3^3 \lambda_3 + \frac{1}{2} \widetilde{\gamma}_3^8 \lambda_8 \Bigr) \Omega^{\dagger}. }{1}\coordE{}\end{eqnarray}
Here the coefficients \myHighlight{$\widetilde{\gamma}_3^a$}\coordHE{} are arbitrary and \myHighlight{$\Omega$}\coordHE{} is a matrix which diagonalises \myHighlight{$B_3$}\coordHE{}. The vectors \myHighlight{$\widetilde{\psi}_{j,k}$}\coordHE{} are defined as in equation (\ref{psi}), replacing only \myHighlight{$\chi_j \rightarrow \widetilde{\chi}_j$}\coordHE{}. Let us now compare the two solutions (\ref{res}) and (\ref{excres}) at points where two eigenvalues of \myHighlight{$B_3$}\coordHE{} coincide. As 't Hooft mentioned in ref. \cite{th}, this takes place at isolated points in three-dimensional space for generic magnetic fields. At such points the vectors \myHighlight{$\psi_{j,1}$}\coordHE{} and \myHighlight{$\psi_{j,2}$}\coordHE{} vanish, which corresponds to setting \myHighlight{$\widetilde{\beta}_{j,k} = 0$}\coordHE{} in equation (\ref{excres}). Equating the \myHighlight{$C_3$}\coordHE{}-components of formulas (\ref{res}) and (\ref{excres}) with the help of equation (\ref{su2}) determines the coefficients \myHighlight{$\widetilde{\gamma}_3^a$}\coordHE{} as functions of \myHighlight{$\beta_{j,3}$}\coordHE{}, \myHighlight{$\gamma_3$}\coordHE{} and \myHighlight{$\widehat{\alpha}_{3l}$}\coordHE{}. However, since \myHighlight{$\widehat{\alpha}_{3l}$}\coordHE{} already appears in the first term of equation (\ref{excres}), there are effectively only three arbitrary parameters determining four unknown coefficients and accordingly, there is one more degree of freedom in the solution (\ref{excres}) which is not present in formula (\ref{res}). In all, there are thus three degrees of freedom in the exceptional solution (\ref{excres}) which cannot be obtained by taking the limit of equation (\ref{res}). This leaves the door open for accepting singular solutions to equation (\ref{ksum}). In that case, though, equation (\ref{pois}) should be replaced by
$$\coord{}\boxMath{ \sum_{k=1}^3 \nabla_k^2(A) \phi = J_0 \, - \!\! \sum_{k,l,m=1}^3 \! \varepsilon_{klm} \partial_k \partial_l C_m }{dollar}{0pt}\coordE{}$$
to compensate for the possibility that the second weak derivatives of \myHighlight{$C_m$}\coordHE{} do not commute.
 
\section{Conclusions}

I have presented here a method by which the Gauss law (\ref{gauss}) can be solved in the case of the SU(3) algebra using the ansatz (\ref{ans}). The fact that the l.h.s. of the consistency equation (\ref{ksum}) depends on the commutator properties of the colour-magnetic field divides the solutions into different classes. I have written down the source-free part of the solution explicitly in the generic case when the set (\ref{basis}) is linearly independent and in the case when one component of the colour-magnetic field has coinciding eigenvalues. Although the SU(2) solution of ref. \cite{ms1} was simple, its SU(3) generalisation (\ref{res}) is much more complicated. The vectors \myHighlight{$\chi_j$}\coordHE{} of the basis (\ref{basis}) are mostly responsible for the complexity, and unfortunately I see no way out of this problem. I could replace the \myHighlight{$\chi_j$}\coordHE{}'s by vectors which would be easier to invert with the formula (\ref{solu}),  e.g. by
$$\coord{}\boxMath{ i[\widehat{B}_1,\widehat{B}_2], i[\widehat{B}_1,\widehat{B}_3], }{dollar}{0pt}\coordE{}$$
but then the orthogonality conditions (\ref{solv}) for \myHighlight{$F_2$}\coordHE{} and \myHighlight{$F_3$}\coordHE{} would lead to complicated relations between the expansion coefficients. So there seems to be some kind of 'conservation of trouble' inherent in this problem. Anyway, it is interesting that the fields \myHighlight{$C_k$}\coordHE{} may have singularities at points where one component of the colour-magnetic field possesses two coinciding eigenvalues. No explicit gauge fixing is needed to detect this singularity as it becomes apparent whenever one tries to solve equation (\ref{ksum}). The method of solving this equation could also be generalised to higher dimensional SU(N) algebras in a straightforward manner, but the results would undoubtedly be even more complicated.

\appendix

\section*{Appendix: Motivation for the generalised \\ Hodge decomposition}
In order to prove that the space of colour-electric fields can be parametrised with the ansatz (\ref{ans}) it is sufficient to show that the equation
\begin{eqnarray}\coord{}\boxAlignEqnarray{\leftCoord{} \rightCoord{}\label{lapla}
E_k &=& \sum_{\rightCoord{}l=1}^{\leftCoord{}3} \Bigl( \nabla_l^2(A) Z_k - ig [G_{kl},Z_l] \Bigr), \rightCoord{}\\\leftCoord{}
G_{kl} &=& \partial_l A_k - \partial_k A_l - ig [A_k,A_l] \nonumber\rightCoord{} \\\leftCoord{}*
&\leftCoord{}=& - \sum_{\rightCoord{}m=1}^{\leftCoord{}3} \varepsilon_{klm} B_m \nonumber\rightCoord{}
\rightCoord{}}{0mm}{6}{8}{ E_k &=& \sum_{l=1}^{3} \Bigl( \nabla_l^2(A) Z_k - ig [G_{kl},Z_l] \Bigr), \\
G_{kl} &=& \partial_l A_k - \partial_k A_l - ig [A_k,A_l] \\*
&=& - \sum_{m=1}^{3} \varepsilon_{klm} B_m }{1}\coordE{}\end{eqnarray}
can be solved for the field \myHighlight{$Z_k$}\coordHE{}. Making use of the identity
\begin{eqnarray*}\coord{}\boxAlignEqnarray{\leftCoord{}
\sum_{\rightCoord{}l=1}^{\leftCoord{}3} \Bigl( \nabla_l^2(A) Z_k - ig [G_{kl},Z_l] \Bigr) &=& - \sum_{\rightCoord{}l,m=1}^{\leftCoord{}3} \varepsilon_{klm} \nabla_l(A) \sum_{\rightCoord{}p,q=1}^{\leftCoord{}3} \varepsilon_{mpq} \nabla_p(A) Z_q \\\leftCoord{}*
&&\leftCoord{} + \nabla_k(A) \sum_{\rightCoord{}l=1}^{\leftCoord{}3} \nabla_l(A) Z_l
\rightCoord{}}{0mm}{7}{6}{
\sum_{l=1}^{3} \Bigl( \nabla_l^2(A) Z_k - ig [G_{kl},Z_l] \Bigr) &=& - \sum_{l,m=1}^{3} \varepsilon_{klm} \nabla_l(A) \sum_{p,q=1}^{3} \varepsilon_{mpq} \nabla_p(A) Z_q \\*
&& + \nabla_k(A) \sum_{l=1}^{3} \nabla_l(A) Z_l
}{1}\coordE{}\end{eqnarray*}
we see that equation (\ref{lapla}) then takes the form of equation (\ref{ans}) with
\begin{eqnarray}\coord{}\boxAlignEqnarray{\leftCoord{} 
C_m &=& - \sum_{\rightCoord{}p,q=1}^{\leftCoord{}3} \varepsilon_{mpq} \nabla_p(A) Z_q, \rightCoord{}\label{konst1} \\\leftCoord{}*
\phi &=& \sum_{\rightCoord{}l=1}^{\leftCoord{}3} \nabla_l(A) Z_l. \rightCoord{}\label{konst2}
\rightCoord{}}{0mm}{4}{6}{ 
C_m &=& - \sum_{p,q=1}^{3} \varepsilon_{mpq} \nabla_p(A) Z_q, \\*
\phi &=& \sum_{l=1}^{3} \nabla_l(A) Z_l. }{1}\coordE{}\end{eqnarray}
I am not trying to solve equation (\ref{lapla}) here, but it seems fairly obvious that a solution exists. In a finite volume this equation can be converted into an integral equation after choosing suitable boundary conditions so that the ordinary Laplacian 
$$\coord{}\boxMath{\Delta = \sum_{l=1}^3 \partial_l^2}{dollar}{0pt}\coordE{}$$
has a unique inverse. The resulting integral equation can then be put into the form of a Fredholm equation and solved, at least formally, using the well-known Fredholm formulas. In  infinite space this procedure would require that the fields \myHighlight{$E_k$}\coordHE{} and the gauge potentials \myHighlight{$A_k$}\coordHE{} decay sufficiently rapidly at infinity. 

Equation (\ref{lapla}) shows that the ansatz (\ref{ans}) contains one redundant SU(3) algebra -valued field, because in general there are no relations like equations (\ref{konst1}) - (\ref{konst2}) among \myHighlight{$C_k$}\coordHE{} and \myHighlight{$\phi$}\coordHE{}. This gives rise to a heuristic argument in favour of the choice (\ref{pois}) for the field \myHighlight{$\phi$}\coordHE{}, since we seem to be free to fix one field component at will. In order to be more exact we should investigate whether the space of colour-electric fields with vanishing covariant divergence can be parametrised with the covariant curl ansatz 
\begin{equation}\coord{}\boxEquation{\label{frans}
\widetilde{E}_k = \sum_{l,m=1}^3 \varepsilon_{klm} \nabla_l(A) C_m,
}{\widetilde{E}_k = \sum_{l,m=1}^3 \varepsilon_{klm} \nabla_l(A) C_m,
}{ecuacion}\coordE{}\end{equation}
where
\begin{eqnarray*}\coord{}\boxAlignEqnarray{
&&\leftCoord{} \sum_{\rightCoord{}k=1}^{\leftCoord{}3} \nabla_k(A) \widetilde{E}_k = 0, \rightCoord{}\\
&&\leftCoord{} \widetilde{E}_k = E_k - \nabla_k(A) \phi. \rightCoord{}
\rightCoord{}}{0mm}{3}{5}{
&& \sum_{k=1}^{3} \nabla_k(A) \widetilde{E}_k = 0, \\
&& \widetilde{E}_k = E_k - \nabla_k(A) \phi. 
}{1}\coordE{}\end{eqnarray*}
In ref. \cite{ms2} Majumdar and Sharatchandra considered an SU(2) equation of the form (\ref{frans}) with \myHighlight{$\widetilde{E}_k = 0$}\coordHE{} and presented a method for obtaining a formal solution. Using the consistency condition (\ref{ksum}) they eliminated \myHighlight{$C_3$}\coordHE{} and converted the remaining equations into the form of a Cauchy problem with initial data given on the plane \myHighlight{$x_3=0$}\coordHE{}. They showed that a formal solution to the Cauchy problem can be constructed in a certain generic case as a power series near the initial plane \myHighlight{$x_3=0$}\coordHE{}. Unfortunately there is an error in their reasoning concerning the convergence of the power series. Namely, they try to apply the Cauchy-Kovalevskaya theorem to equations of the form
$$\coord{}\boxMath{ \partial_3 C_j = G[C_1, C_2, \{A_k\}], \qquad j=1,2 }{dollar}{0pt}\coordE{}$$ 
where the functional \myHighlight{$G$}\coordHE{} depends on second order derivatives of \myHighlight{$C_1$}\coordHE{} and \myHighlight{$C_2$}\coordHE{} with respect to \myHighlight{$x_1$}\coordHE{} and \myHighlight{$x_2$}\coordHE{}. In this case the Cauchy-Kovalevskaya theorem is not valid and the formal solution does not necessarily converge. The method of ref. \cite{ms2} would be easy to generalise to the case of equation (\ref{frans}), but the solution might be mathematically meaningless. Anyway, this method hints that equation (\ref{frans}) can generically be solved for the \myHighlight{$C_k$}\coordHE{}'s and accordingly, that the space of sufficiently regular colour-electric fields with vanishing covariant divergence can be parametrised with the ansatz (\ref{frans}). From the mathematical point of view this problem is still open, though.

\section*{Acknowledgements}

I would like to thank professor C. Cronstr\"om for his guidance during the
preparation of this paper. This research was supported by the Academy of Finland.


\begin{thebibliography}{9}

\bibitem{ms1} P. Majumdar and H. S. Sharatchandra, Phys. Rev. D {\bf 58},
067702 (1998), hep-th/9804128
\bibitem{cc} C. Cronstr\"om, Acta Physica Slovaca {\bf 50}, 369-379 (2000), hep-th/9906184
\bibitem{msw} A. J. Macfarlane, A. Sudbery and P. H. Weisz, Commun. Math. Phys.
{\bf 11}, 77-90 (1968)
\bibitem{ms2} P. Majumdar and H. S. Sharatchandra, Phys. Rev. D {\bf 63},
067701 (2001), hep-th/9804091
\bibitem{ammp} J. A. de Azc\'arraga, A. J. Macfarlane, A. J. Mountain and J. C. P\'erez Bueno, Nucl. Phys. B {\bf 510}, 657-687 (1998), physics/9706006
\bibitem{th} G. 't Hooft, Nucl. Phys. B {\bf 190}, 455-478 (1981)

\end{thebibliography}

\end{document}
\bye
