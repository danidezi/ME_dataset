
\documentclass[a4paper,a4paper]{article}
\usepackage{useful_macros}
\begin{document}
\begin{center}
y-DEFORMED BPS Dp- BRANES ON A SURFACE  IN A CALABI-YAU THREEFOLD\\ [.25in]
by Juan F. Ospina G.
\end{center}
\begin{center}
ABSTRACT \\ [.25in]
Using y-deformed algebraic geometric techniques the y-deformed Mukay vector of RR-charges of the y-deformed BPS Dp-branes localized on a sufarce in a Calabi-Yau threefold. The formules that are obtained here are generalizations of the formulaes of the fourth section of the preprint hep-th/0007243
\end{center}

\section{Introduction:y-deformed BPS Dp-branes on a Calabi-Yau threefold}
\setlength{\baselineskip}{20pt}


A BPS D-brane on a Calabi-Yau threefold X can be represented using a  coherent \myHighlight{$O_{X}$}\coordHE{}-module G. The RR charge of G is given by the Mukai vector[1]:
\begin{center}
{ \mathversion{bold} \myHighlight{$ v_{X}(G)=ch(G)\sqrt{Todd(T_{X})}{\in}H_{2*}(X;Q):={\oplus}_{i=0}^3H_{2i}(X;Q)  $}\coordHE{} }
\end{center}

where \myHighlight{$ ch(G)= \sum_{i=0}^{3}ch_{i}(G)$}\coordHE{} is the Chern character with \myHighlight{$ ch_{i}(G){\in}H_{6-2i}(X;Q)$}\coordHE{}, which can be computed by the homology-cohomology duality[1]:  always one can to have a resolution of G by locally free sheaves  \myHighlight{$(V_{*}) $}\coordHE{},in such way that one can to set that
\myHighlight{$ ch(G):= \sum_{i=0}^{3}(-1)^ich(V_i)$}\coordHE{},and these result does not depend on the choise of the resolution. Finally  \myHighlight{$Todd(T_{X})=[X]+\frac{\rm c_{1}[X]}{\rm 2}+\frac{\rm c_{2}[X]+c_{1}[X]^2}{\rm 12}+\frac{\rm c_{2}[X]c_{1}[X]}{\rm 24} $}\coordHE{}. Now when X is a Calabi-Yau threefold one has \myHighlight{$c_{1}[X]=0
   $}\coordHE{} and then one obtains: \myHighlight{$Todd(T_{X})=[X]+\frac{\rm c_{2}[X]}{\rm 12}$}\coordHE{}.
From these the effect of the square root of the Todd Class on the RR charges,is to say the geometric version of the Witten effect is given by:
\begin{center}
{ \mathversion{bold} \myHighlight{$\sqrt{Todd(T_{X})} =[X]+\frac{\rm c_{2}[X]}{\rm 24} $}\coordHE{} }
\end{center}

For the investigation of the topological aspects of D-branes is of the great importance to obtain several basic invariants of BPS D-Branes. One of these invariants is the RR charge of the D-brane. Other invariant is the intersection form on D-branes on X [1]. This invariant for intersections of 
two Dp-branes is obtained by multiplication of the Mukay vectors of RR charges corresponding to the intersecting Dp-branes and is given by: [1]
 
\begin{center}
\setlength{\baselineskip}{30pt}
{ \mathversion{bold} \myHighlight{$ I_{X}(G_1,G_2)=[v_{X}(G_1)^v.v_{X}(G_2)]_X=[(ch(G_1)\sqrt{Todd(T_{X})})^v.ch(G_2)\sqrt{Todd(T_{X})}]_X=[ch(G_1)^v.ch(G_2)Todd(T_{X})]_X $}\coordHE{} }
\end{center}
where \myHighlight{$ [...]_X $}\coordHE{}  evaluates the degree of \myHighlight{$ H_{0}(X;Q){\cong}Q$}\coordHE{} component, and \myHighlight{$v^{\vee}$}\coordHE{} flips the sign of \myHighlight{$H_{0}(X){\oplus}H_{4}(X)$}\coordHE{} part of the Mukay vector v . In particular, if G itself is locally free, then \myHighlight{$ch(G)^{\vee}=ch(G^{\vee})$}\coordHE{}, where  \myHighlight{$ G^{\vee}=Hom_{X}(G,0_X)$}\coordHE{} is the dual sheaf. Finally is easy to check that: \myHighlight{$I_{X}(G_1,G_2) =-I_{X}(G_2,G_1)$}\coordHE{}.
On other hand the invariant of intersection between D-branes is an application of the Hirzebruch-Riemann-Roch and for then you can write[1]
\begin{center}

{ \mathversion{bold} \myHighlight{$ I_{X}(G_1,G_2)=\sum_{i=0}^{3}(-1)^idimExt^i_X(G_1,G_2) $}\coordHE{} }
\end{center}
For this reason the skew-symmetric property \myHighlight{$I_{X}(G_1,G_2) =-I_{X}(G_2,G_1)$}\coordHE{} of the intersection form \myHighlight{$I_X$}\coordHE{}  for the intersection of two Dp-branes may be attributed to the Serre duality: \myHighlight{$Ext^i_X(G_1,G_2){\cong}Ext^{3-i}_X(G_1,G_2)^{\vee}$}\coordHE{} [1]. Another interesting comentary is that from the integrality theorems for diferential and complex manifolds the formula H.R.R. is an integer and this assures that
\myHighlight{$I_X$}\coordHE{} takes values in Z. [1],[2].
 
Now the result that this work presents is about the y-deformed Dp-branes on a Calabi Yau threefold. A y-deformed BPS Dp-brane on a Calabi-yau X can be represented by a y-deformed  \myHighlight{$O_{X}-modulo G$}\coordHE{}.  The y-deformed RR charge of G is given by the y-deformed Mukai vector:
\begin{center}
{ \mathversion{bold} \myHighlight{$ v_{X,y}(G)=ch_y(G)\sqrt{{\chi}_y(T_{X})}{\in}(H_{2*}(X;Q){\otimes}Q[y]):={\oplus}_{i=0}^3(H_{2i}(X;Q){\otimes}Q[y])  $}\coordHE{} }
\end{center}
where \myHighlight{${\chi}_y$}\coordHE{} is the y-chi-genus which is a generalization of the Todd class [2,3] and \myHighlight{$ch_y(G)$}\coordHE{} is the y-deformed Chern Character. the total Chern Class for \myHighlight{$T_{X}$}\coordHE{}  has the following sumarization:
\begin{center}
{ \mathversion{bold} \myHighlight{$ c(T_{X}) = \sum_{j=0}^{3}c_j(T_{X}) $}\coordHE{} }
\end{center} 
also, the total Chern Class for the such bundle has the following factorization:

\begin{center}
{ \mathversion{bold} \myHighlight{$ c(T_{X}) = \prod_{i=1}^{3}(1+x_i)$}\coordHE{} }
\end{center}
The  CHI-y- genus for \myHighlight{$T_{X}$}\coordHE{} has the following formal factorisation:
\begin{center}
{ \mathversion{bold} \myHighlight{$ {\chi}_y(T_{X}) = \prod_{i=1}^3\frac{\rm(1+yexp(-(y+1)x_i))x_i }{\rm 1-exp(-(y+1)x_i)}$}\coordHE{} }
\end{center}

The CHI-y- genus for \myHighlight{$T_{X}$}\coordHE{}  has the following formal sumarisation in terms of the y-deformed Todd polynomials which are formed from the corresponding Chern classes and from the polynomials on y :
\begin{center}
{ \mathversion{bold} \myHighlight{${\chi}_y(T_{X})  = \sum_{j=0}^{\infty}T_j(c_1(T_{X}),...,c_j(T_{X}),y) $}\coordHE{} }
\end{center}

The y-Todd  polynomials are given by:
\begin{center}
{ \mathversion{bold} \myHighlight{$ T_0(c_0(T_{X}),y) =T _0(1,y)=1 $}\coordHE{} }
\end{center}
\begin{center}
{ \mathversion{bold} \myHighlight{$ T_1(c_1(T_{X}),y) = \frac{\rm (1-y)c_1(T_{X})}{\rm 2} $}\coordHE{} }
\end{center}
\begin{center}
{ \mathversion{bold} \myHighlight{$ T_2(c_1(T_{X}),c_2(T_{X}),y) = \frac{\rm (y+1)^2c_1(T_{X})^2+(y^2-10y+1)c_2(T_{X})}{\rm 12} $}\coordHE{} }
\end{center}
\begin{center}
{ \mathversion{bold} \myHighlight{$ T_3(c_1(T_{X}),c_2(T_{X}),c_3(T_{X}),y) = \frac{\rm -(y+1)^2(y-1)c_1(T_{X})c_2(T_{X})+12y(y-1)c_3(T_{X})}{\rm 24} $}\coordHE{} }
\end{center}.

Then one has:

\begin{center}
{ \mathversion{bold} \myHighlight{$ {\chi}_y(T_{X}) = 1+\frac{\rm (1-y)c_1(T_{X})}{\rm 2}+\frac{\rm (y+1)^2c_1(T_{X})^2+(y^2-10y+1)c_2(T_{X})}{\rm 12}+\frac{\rm -(y+1)^2(y-1)c_1(T_{X})c_2(T_{X})+12y(y-1)c_3(T_{X})}{\rm 24} $}\coordHE{} }
\end{center}.

When X is a Calabi-Yau threefold then the chi-y-genus is given by


\begin{center}
{ \mathversion{bold} \myHighlight{$ {\chi}_y(T_{X}) = 1+\frac{\rm (y^2-10y+1)c_2(T_{X})}{\rm 12}+\frac{\rm 12y(y-1)c_3(T_{X})}{\rm 24} $}\coordHE{} }
\end{center}.

From this one can to write the following formula for the y-deformed geometric version of the Witten effect:

\begin{center}
{ \mathversion{bold} \myHighlight{$\sqrt{{\chi}_y(T_{X})} =[X]+\frac{\rm (y^2-10y+1)c_{2}[X]}{\rm 24}+\frac{\rm y(y-1)c_{3}[X]}{\rm 4} $}\coordHE{} }
\end{center}
when y=0 one obtains the usual Witten effect:

\begin{center}
{ \mathversion{bold} \myHighlight{$\sqrt{{\chi}_0(T_{X})} =[X]+\frac{\rm (0^2-0+1)c_{2}[X]}{\rm 24}+\frac{\rm 0(0-1)c_{3}[X]}{\rm 4}=[X]+\frac{\rm c_{2}[X]}{\rm 24} $}\coordHE{} }
\end{center}

For the other hand  the y-deformed Chern Character \myHighlight{$ch_y(G)$}\coordHE{}  is given by:
\myHighlight{$ ch_y(G)= \sum_{i=0}^{3}ch_{i,y}(G)$}\coordHE{} with \myHighlight{$ ch_{i,y}(G){\in}(H_{6-2i}(X;Q){\otimes}Q[y])$}\coordHE{}, which can be computed using y-deformed homology-cohomology duality: always one can to have a y-deformed resolution of G by y-deformed locally free sheaves  \myHighlight{$(V_{*}) $}\coordHE{},in such way that one can to set that \myHighlight{$ ch_y(G):= \sum_{i=0}^{3}(-1)^ich_y(V_i)$}\coordHE{},and these result does not depend on the choise of the y-deformed resolution.
The total Chern Class for G  has the following sumarization:
\begin{center}
{ \mathversion{bold} \myHighlight{$ c(G) = \sum_{j=0}^{q}c_j(G) $}\coordHE{} }
\end{center} 
also, the total Chern Class for G has the following factorization:

\begin{center}
{ \mathversion{bold} \myHighlight{$ c(G) = \prod_{i=1}^{q}(1+z_i)$}\coordHE{} }
\end{center}
The total Chern character of G is defined by:
\begin{center}
{ \mathversion{bold} \myHighlight{$ ch(G) = \sum_{j=1}^{q}e^{z_i} $}\coordHE{} }
\end{center} 

The total y-deformed Chern character for G  has the following sumarization:

\begin{center}
{ \mathversion{bold} \myHighlight{$ ch_y(G) = \sum_{j=1}^{q}e^{(1+y)z_i} $}\coordHE{} }
\end{center}
The total y-deformed Chern character for G has the following expantion in terms of the Chern class of G and polynomials for y:

\begin{center}
{ \mathversion{bold} \myHighlight{$ ch_y(G) =rk(G)+(y+1)c_{1}(G)+(y+1)^2({\frac{\rm c_{1}(G)^{2}-c_2(G)}{\rm 2}})+(y+1)^3({\frac{\rm c_{1}(G)^{3}-3c_1(G)c_2(G)+3c_3(G)}{\rm 6}})
 $}\coordHE{} }
\end{center}

It is easy to see that when y=0, one obtains the usual expantion for the usual Chern character.
For the investigation of the topological aspects of the y-deformed  D-branes is of the great importance to obtain several basic y-deformed invariants of y-deformed BPS D-Branes. One of these y-deformed invariants is the y-deformed RR charge of the y-deformed D-brane. Other y-deformed invariant is the y-deformed  intersection form on y-deformed D-branes on X . This y-deformed invariant for intersections of 
two y-deformed Dp-branes is obtained by multiplication of the y-deformed Mukay vectors of the y-deformed RR charges corresponding to the intersecting y-deformed Dp-branes and is given by: 
 
\begin{center}
\setlength{\baselineskip}{30pt}
{ \mathversion{bold} \myHighlight{$ I_{X,y}(G_1,G_2)=[v_{X,y}(G_1)^v.v_{X,y}(G_2)]_X=[(ch(G_1)\sqrt{{\chi}_y(T_{X})})^v.ch(G_2)\sqrt{{\chi}_y(T_{X})}]_X=[ch(G_1)^v.ch(G_2){\chi}_y(T_{X})]_X $}\coordHE{} }
\end{center}

where \myHighlight{$ [...]_{X,y} $}\coordHE{}  evaluates the degree of \myHighlight{$ (H_{0}(X;Q){\otimes}Q[y]){\cong}(Q{\otimes}Q[y])$}\coordHE{} component, and \myHighlight{$v^{\vee}$}\coordHE{} flips the sign of \myHighlight{$(H_{0}(X){\otimes}Q[y]){\oplus}(H_{4}(X){\otimes}Q[y]))$}\coordHE{} y-deformed part of the y-deformed  Mukay vector v . In particular, if G itself is locally free, then \myHighlight{$ch_{y}(G)^{\vee}=ch_y(G^{\vee})$}\coordHE{}, where  \myHighlight{$ G^{\vee}=Hom_{X}(G,0_X)$}\coordHE{} is the y-deformed dual sheaf. Finally is easy to check that: \myHighlight{$I_{X,y}(G_1,G_2) =-I_{X,y}(G_2,G_1)$}\coordHE{}.

On other hand the y-deformed invariant of intersection between y-deformed D-branes is an application of the y-deformed Hirzebruch-Riemann-Roch and for then you can write:

 
\begin{center}

{ \mathversion{bold} \myHighlight{$ I_{X,y}(G_1,G_2)=\sum_{i=0}^{3}(-1)^idimExt^i_{X,y}(G_1,G_2) $}\coordHE{} }
\end{center}
For this reason the skew-symmetric property \myHighlight{$I_{X,y}(G_1,G_2) =-I_{X,y}(G_2,G_1)$}\coordHE{} of the intersection form \myHighlight{$I_{X,y}$}\coordHE{}  for the intersection of two y-deformed Dp-branes may be attributed to the y-deformed Serre duality: \myHighlight{$Ext^i_{X,y}(G_1,G_2){\cong}Ext^{3-i}_{X,y}(G_1,G_2)^{\vee}$}\coordHE{}. Another interesting comentary is that from the y-deformed integrality theorems for diferential and complex manifolds the y-deformed formula H.R.R. is an polynomial on y  and this assures that \myHighlight{$I_{X,y}$}\coordHE{} takes values in Q[y].
 

Now let \myHighlight{$J_{X,y}{\in}(H_{4}(X;R){\otimes}R[y]) $}\coordHE{}  be a y-deformed Kahler form on X, whis is here identified with an y-deformed R-extended ample divisor. The y-deformed classical expression of the y-deformed central charge of the y-deformed D-brane G is then given by [1]:
\begin{center}
{ \mathversion{bold} \myHighlight{$ Z_{J_{X,y}}^d(G)=-[e^{-J_{X,y}}.v_{X,y}(G)]_{X}=-\sum_{k=0}^{3}{\frac{\rm (-1)^{k}}{\rm k!}[J_{X,y}^k.v_{X,y,k}(G)]_{X}} $}\coordHE{} }
\end{center}

where \myHighlight{$v_{X,y,k} $}\coordHE{} is the \myHighlight{$H_{2k}(X){\otimes}Q[y] $}\coordHE{} component of \myHighlight{$v_{X,y}{\in}(H_{2*}(X;Q){\otimes}Q[y]) $}\coordHE{}.

In such way we obtain the three y-deformed invariants: y-deformed RR charge,
y-deformed central charge and y-deformed intersections pairings of two y-deformed BPS Dp-branas. With this aid of some algebraic geometry-topology techniques we can to begin the study of topological aspects of y-deformed BPS Dp-branes bounded on a proyective algebraic surface in a Calabi-Yau threefold X. 

\section{y-deformed BPS Dp-branes localized on a surface in a Calabi-Yau
threefold}

Let f be an embedding of a proyective algebraic surface S in a Calabi-Yau threefold X. In the limit of infinite elliptic fiber, the y-deformed BPS Dp-branes  for which the y-deformed central charge remains  finite are those y-deformed BPS Dp-branes which are confined to the algebraic surface S. The physical and topological propertis of the y-deformed BPS D-p-branes  localized on the algebraic surface S then dependen not on the details of the
global model X, but only on the intrinsic y-deformed  geometry of S ans its y-deformed normal bundle \myHighlight{$N_{S,y}=N_{S|X,y}$}\coordHE{} which is isomorphic to the y-deformed canonical line bundle \myHighlight{$K_{S,y}$}\coordHE{}. In particular, this means that we can compute the y-deformed central charges of y-deformed BPS D-p-branes using y-deformed local mirror symmetry principle on S.

In a elementary physical configuration you have a y-deformed BPS Dp-brane sticking to S.  Such y-deformed  D-brane sticking to S can be described mathematically by a y-deformed \myHighlight{$O_{S}-module E$}\coordHE{}. For this configuration an important y-deformed topological invariant is the y-deformed Euler number of E (the Euler y-polynomial for E) which is defined by  \myHighlight{${\chi}_{y}(E)=\sum_{j=0}^{2}(-1)^ih^i(S,E,y)$}\coordHE{}, where \myHighlight{$h^i(S,E,y)=dim(H^i(S,E))_y$}\coordHE{}.
For to obtain the y-deformed Euler number of E or the Euler  polynomial of E
the first thing that one needs is the y-deformed Todd class of S or \myHighlight{${\chi}_y  $}\coordHE{} class of S:

\begin{center}
{ \mathversion{bold} \myHighlight{$ {\chi}_y(T_S) = [S]+\frac{\rm (1-y)c_1(S)}{\rm 2}+\frac{\rm (y+1)^2c_1(S)^2+(y^2-10y+1)c_2(S)}{\rm 12} $}\coordHE{} }
\end{center}

this expansion can be writen as:

\begin{center}
{ \mathversion{bold} \myHighlight{$ {\chi}_y(T_S) = [S]+\frac{\rm (1-y)c_1(S)}{\rm 2}+{\chi}_{y}(O_S)[pt] $}\coordHE{} }
\end{center}
where:

\begin{center}
{ \mathversion{bold} \myHighlight{$ {\chi}_{y}(O_S) = [\frac{\rm (y+1)^2c_1(S)^2+(y^2-10y+1)c_2(S)}{\rm 12} ]_S$}\coordHE{} }

\end{center}
 The second thing for to do is to apply the y-deformed H.R.R formula, and then one get:

\begin{center}
\setlength{\baselineskip}{40pt}
{ \mathversion{bold} \myHighlight{$ {\chi}_{y}(E) = [ch_{y}(E){\chi}_y(T_S)]_S=[ch_{y}(E)([S]+\frac{\rm (1-y)c_1(S)}{\rm 2}+{\chi}_{y}(O_S)]_S=[(rk(E)+(y+1)c_{1}(E)+(y+1)^2({\frac{\rm c_{1}(E)^{2}-c_2(E)}{\rm 2}}))([S]+\frac{\rm (1-y)c_1(S)}{\rm 2}+{\chi}_{y}(O_S)]_S=rk(E){\chi}_{y}(O_S)+[(y+1)^2({\frac{\rm c_{1}(E)^{2}-c_2(E)}{\rm 2}}))+\frac{\rm (y+1)(1-y)c_1(S).c_{1}(E)}{\rm 2}]_S$}\coordHE{} }

\end{center}

From the other side, there is y-deformed canonical push-forward homomorphism
\myHighlight{$f_*$}\coordHE{} from \myHighlight{$H_{2*}(S;Q){\otimes}Q[y]$}\coordHE{} to \myHighlight{$H_{2*}(X;Q){\otimes}Q[y]$}\coordHE{}, which
maps a y-deformed cycle on S that on X. Also, on can define the y-deformed coherent sheaf \myHighlight{$f_{!}E$}\coordHE{} on X by extending E by zero to X/S. Now using the y-deformation of the celebrated Grothendieck-Riemman-Roch formula for the embeding f od S in X, one can to relate the y-deformed chern characters of E and \myHighlight{$f_{!}E$}\coordHE{} as follows:

\begin{center}
{ \mathversion{bold} \myHighlight{$ ch_y(f_{!}E) = f_{*}(ch_{y}(E)\frac{\rm 1}{\rm chi_{y}(N_S)}) $}\coordHE{} }
\end{center}

Multiplying the boht sides of the y-deformed GRR formula by \myHighlight{$\sqrt{{\chi}_{y}(T_X)} $}\coordHE{} , one has:

\begin{center}
{ \mathversion{bold} \myHighlight{$ ch_y(f_{!}E)\sqrt{{\chi}_{y}(T_X)} = f_{*}(ch_{y}(E)\sqrt{\frac{\rm chi_{y}(T_S))}{\rm chi_{y}(N_S)}
}) $}\coordHE{} }
\end{center}






where we have used the y-deformed proyection formula:

 \begin{center}
{ \mathversion{bold} \myHighlight{$ f_{*}(a.f^*b) = f_{*}a.b $}\coordHE{} }

\end{center}

with  \myHighlight{$a{\in}(H_{2*}(S;Q){\otimes}Q[y]),  b{\in}(H_{2*}(X;Q){\otimes}Q[y]) $}\coordHE{}

and \myHighlight{$f^*{chi_{y}}(T_X)={chi_{y}}(T_S).{chi_{y}}(N_S)$}\coordHE{}  , which follows from the y-deformed short exact sequence of bundles on S: \myHighlight{$0 ------> T_S ---->f^*T_X ---->  N_S ----> 0$}\coordHE{}, combined with the multiplicative property of the chi-y-genus.

Now the y-deformed BPS Dp-brane on a Calabi-Yau threefold X  is represented by G and y-deformed BPS Dp-brane sticking to S can be described by E then one has \myHighlight{$G=f_{!}E$}\coordHE{} and following formula for the y-deformed Mukai vector of the y-deformed RR charges of \myHighlight{$G=f_{!}E$}\coordHE{}

\begin{center}
{ \mathversion{bold} \myHighlight{$ v_{X,y}(f_{!}E)=ch_{y}(f_{!}E)\sqrt{{\chi}_{y}(T_{X})}{\in}(H_{2*}(X;Q){\otimes}Q[y]):={\oplus}_{i=0}^3(H_{2i}(X;Q){\otimes}Q[y])  $}\coordHE{} }
\end{center}

The you have:

\begin{center}
{ \mathversion{bold} \myHighlight{$ v_{X,y}(f_{!}E)= f_{*}(ch_{y}(E)\sqrt{\frac{\rm chi_{y}(T_S))}{\rm chi_{y}(N_S)}
})=  f_{*}(v_{S,y}(E))$}\coordHE{} }
\end{center}

In such way the y-deformed RR charge of the y-deformed BPS Dp-brane represented by E on S regarded as a y-deformed BPS Dp-brane on  X can written in the following intrinsic description (of the y-deformed RR charge on S):

\begin{center}
{ \mathversion{bold} \myHighlight{$ v_{S,y}(E)= ch_{y}(E)\sqrt{\frac{\rm chi_{y}(T_S))}{\rm chi_{y}(N_S)}
}=ch_{y}(E)\sqrt{\frac{\rm chi_{y}(T_S))}{\rm chi_{y}(K_S)}
}$}\coordHE{} }
\end{center}

The y-deformed gravitational correction factor for S admits the following expansion:

\begin{center}
\setlength{\baselineskip}{30pt}
{ \mathversion{bold} \myHighlight{$\sqrt{\frac{\rm chi_{y}(T_S))}{\rm chi_{y}(K_S)}
} = [S]+\frac{\rm (1-y)c_1(S)}{\rm 2}+\frac{\rm (-10y+1+y^2)c_2(S)+3(y-1)^2c_1(S)^2}{\rm 24}  {\in} (H_{2*}(S;Q){\otimes}Q[y])$}\coordHE{} }
\end{center}

As a simple exercise one can to compute the y-deformed RR charge of a y-deformed sheaf on S. For this let i: C---> S be an embedding of a smooth genus g algebraic curve in S with the normal bundle \myHighlight{$N_{C}=N_{C\S}$}\coordHE{}. Then from a lin bundle \myHighlight{$L_{C}$}\coordHE{} on C, one obtains a y-deformed torsion sheaf \myHighlight{$i_{!}L_{C}$}\coordHE{} on S and \myHighlight{$ch_{y}(i_{!}L_{C})$}\coordHE{} can be computed from the y-deformed G.R.R. formula:


\begin{center}
\setlength{\baselineskip}{40pt}   
{ \mathversion{bold} \myHighlight{$ ch_y(i_{!}L_{C}) = i_{*}(ch_{y}(L_{C})\frac{\rm 1}{\rm chi_{y}(N_C)})=i_{*}((rk(L_{C})+(y+1)c_{1}(L_{C})(1+\frac{\rm (y-1)c_{1}(N_{C})}{\rm 2}))=i_{*}[C]+((y+1)c_{1}(L_{C})+\frac{\rm (y-1)c_{1}(N_{C})}{\rm 2})[pt]=i_{*}[C]+((y+1)deg(L_{C})+\frac{\rm (y-1)deg(N_{C})}{\rm 2})[pt]$}\coordHE{} }
\end{center}

where \myHighlight{$deg(L):=[c_{1}(L)]_C$}\coordHE{} for a line bundle on C.  Then y-deformed RR charge of the y-deformed BPS Dp-brane bounded on S represented by the y-deformed \myHighlight{$O_{S}-module$}\coordHE{}   \myHighlight{$i_{!}L_{C}$}\coordHE{} can be computed  as follows:

\begin{center}
\setlength{\baselineskip}{40pt}
{ \mathversion{bold} \myHighlight{$ v_{S,y}(i_{!}L_{C})= ch_{y}(i_{!}L_{C})\sqrt{\frac{\rm chi_{y}(T_C))}{\rm chi_{y}(K_C)}
}=(i_{*}[C]+((y+1)deg(L_{C})+\frac{\rm (y-1)deg(N_{C})}{\rm 2})[pt])([C]+\frac{\rm (1-y)c_1(C)}{\rm 2})=(i_{*}[C]+((y+1)deg(L_{C})+(1-y)c_1(C))[pt]){\in}{\oplus}(H_{0}(S){\otimes}Q[y])) $}\coordHE{} }
\end{center}

I now turn again to intersection pairings of the y-deformed BPS Dp-branes one has the question about the what is the most appropiate intersection   for on y-deformed D-branes on S. Here we will describe only y-deformed candite. 

The  y-deformed candidate uses the intrinsic y-deformed Mukay vector  \myHighlight{$v_{S,y} $}\coordHE{} and defines a y-deformed symmetric form:

\begin{center}
{ \mathversion{bold} \myHighlight{$ I_{S,y}(E_{1},E_{1})= -[v_{S,y}(E_{1})^{v}.v_{S,y}(E_{2})]_S=\frac{\rm r_{1}r_{2}(y^2-10y+1)chi(S)}{\rm 12})+[r_{1}ch_{2}(E_{2})+r_{2}ch_{2}(E_{1})-c_{1}(E_{1}).c_{1}(E_{2})]_{S}$}\coordHE{} }
\end{center}

where  \myHighlight{$ch(E)=r[S]+c_{1}(E)+ch_{2}(E) $}\coordHE{},  \myHighlight{${\chi}(S)=[c_{2}(S)]_S $}\coordHE{} Is THE euler number, and  \myHighlight{$v_{y}^{\vee}=-v_{0,y}+v_{1,y}-v_{2,y} $}\coordHE{} with \myHighlight{$v_{i,y}$}\coordHE{}
being the y-deformed \myHighlight{$(H_{2i}(S){\otimes}Q[y]) $}\coordHE{} componente of the y-deformed vector \myHighlight{$v_{y}$}\coordHE{}.

In constrast with \myHighlight{$I_{X}$}\coordHE{} that have values in  \myHighlight{$Q[y] $}\coordHE{} and when y=0 then takes values in Z, now \myHighlight{$I_{S} $}\coordHE{} also have values in  \myHighlight{$Q[y] $}\coordHE{} but in this case when y=0 \myHighlight{$I_{S} $}\coordHE{} is not Z-valued in general.
























 





 


\section{References}

\setlength{\baselineskip}{20pt}
[1]   hep-th/0007243

[2]   F. Hirzebruch, Topological Methods in Algebraic Geometry, 1978







\setlength{\baselineskip}{50pt}   
\end{document}





\bye
