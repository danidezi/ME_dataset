
\documentclass[a4paper,a4paper]{article}
\setlength{\topmargin}{-30pt}
\setlength{\oddsidemargin}{0cm}
\setlength{\evensidemargin}{0cm}
\setlength{\textheight}{22cm}
\setlength{\textwidth}{16cm}

\makeatletter
\@addtoreset{equation}{section}
\def\theequation{\thesection.\arabic{equation}}
\makeatother

\usepackage{useful_macros}
\begin{document}


\begin{flushright}
Feb 2003

KEK-TH-867
\end{flushright}

\begin{center}

\vspace{5cm}

{\Large Target Space Approach to Closed String Tachyons}

\vspace{2cm}

Takao Suyama \footnote{e-mail address : tsuyama@post.kek.jp}

\vspace{1cm}

{\it Theory Group, KEK}

{\it Tsukuba, Ibaraki 305-0801, Japan}

\vspace{4cm}

{\bf Abstract} 

\end{center}

We use the low energy effective theory of string theory to investigate condensations of closed string tachyons 
propagating in the bulk. 
The c-function is related to the total energy of the system via the effective action. 
A possible modification of the c-theorem is discussed. 
We also deduce endpoints of the decays by investigating scalar potential of gauged supergravities. 
A string theory in the flat spacetime would be a possible endpoint. 

\newpage





















\vspace{1cm}

\section{Introduction}

\vspace{5mm}

The decay of an unstable vacuum is a dynamical process. 
To study this, one has to consider non-supersymmetric theories. 
Therefore, investigations of this kind of phenomena are difficult, in general. 

In this paper, our interest is on condensations of tachyons which can propagate in the bulk. 
What we have to clarify would be the following two questions. 
The first one is into what theory the original tachyonic theory decays, and the second one is how to analyze the 
decay. 
The answer to the latter is, in principle, clear; one can use a closed string field theory \cite{closedSFT}, or 
matrix 
formulations of string theories and M-theory \cite{BFSS}\cite{IKKT}\cite{DVV}. 
However, such an analysis will be, in practice, difficult to perform. 
The formar question is more difficult to answer. 
There are two naive expectations for endpoints of the decay. 
One possibility is to decay into a non-critical string theory, since the Zamolodchikov c-theorem 
\cite{Zamolodchikov}  states that 
the central charge of a worldsheet theory decreases via a relevant deformation. 
The other is discussed in our previous paper \cite{suyama}. 
We argued that there may be a process which connects two critical string theories via a tachyon condensation, 
when the 
corresponding low energy effective theory has a scalar potential, two of whose critical points correspond to these 
critical string theories. 
We also argued that since the theory flows into a strong coupling theory, the worldsheet analysis, i.e. arguments 
based on the c-theorem, might be irrelevant for the dynamics of string theory. 

In this paper, we will employ the low energy effective theory as a tool, and try to answer the two questions 
mentioned above. 
The results of our investigations are summarized as follows. 
The Zamolodchikov c-function of a sigma model is equal to the action functional of the corresponding low energy 
effective theory \cite{Tseytlin}. 
This expression of the c-function was used recently \cite{GHMS} to show that worldsheet renormalization group 
flows of a kind correspond to processes which are favored energetically. 
The relation between the c-function and the effective action can be naturally generalized, and as a result, 
the decrease of the c-function is related to the decrease of the total energy of the spacetime theory, when the 
dilaton is trivial. 
Otherwise, the relation between the c-function and the energy does not necessarily hold. 
Therefore, we would like to claim that the c-theorem cannot always be a guide to deduce endpoints of the 
decay. 
We further study the final states of the decay by using the low energy effective theory which can be regarded as 
the zeroth-order approximation of the closed string field theory. 
In some cases, one can relate an evolution of the tachyon vev to a deformation of the geometry of a manifold on 
which string theory or M-theory are compactified. 
An interesting suggestion of this analysis is that the flat spacetime could be the background of a theory 
corresponding to the endpoint of the decay. 

As we will show in this paper, the use of the effective theory, in particular a gauged supergravity, would be very 
fruitful for the investigations of bulk tachyon condensations. 
We can obtain some explicit results since some of the effective theories have been studied exhaustively. 
More detailed information of other gauged supergravities, in particular a relation to string theory and M-theory, 
will be valuable for the study of tachyon condensations. 

This paper is organized as follows. 
In section \ref{Ddep}, we review the results of \cite{suyama}. 
In section \ref{Dindep} we discuss a tachyon condensation different from those in \cite{suyama}, in which the 
worldsheet analysis would not break down. 
We suggest here that there is a subtlety on the comparison of central charges when the background changes during 
the process. 
In section \ref{center}, we generalize the expression of \cite{Tseytlin}. 
We then argue how it is related to the energy and how the decrease of the 
c-function can deviate from the decrease of the spacetime energy. 
In section \ref{potential}, we deduce the possible endpoints of bulk tachyon condensations based on the analysis 
of the scalar potentials of the low energy effective theories. 
Section \ref{discussion} is devoted to discussion. 





















\vspace{1cm}

\section{Worldsheet RG}  \label{Ddep}

\vspace{5mm}

The tree level dynamics of strings in a non-trivial background is described by a non-linear sigma 
model. 
The background field configuration is encoded in coupling functions of the sigma model. 
For the sigma model to be conformal, beta-functionals of the coupling functions must vanish, and this 
condition is equivalent to the on-shell condition of the background fields. 
Since one imposes the conformal invariance, the corresponding string theory is critical, that is, its 
central charge is zero after adding contributions from the ghosts. 

Assume that a string theory in a background contains a tachyon propagating in the bulk. 
The presence of the tachyon would be a signal of an instability of the background, and the theory 
would be deformed so as to cure such the instability. 
This process would be realized via a condensation of the tachyon, as in the case of open string tachyons 
\cite{open} and of localized closed string tachyons \cite{localized}. 
(See also \cite{70s} for ealier study on this phenomena.)

During the condensation of localized closed string tachyons, it is claimed that the central charge 
does not decrease, due to the fact that the relevant space for the tachyons has an infinite volume 
\cite{HKMM}. 
This argument does not apply to situations in which tachyons come from a compact space. 
Also in the case of bulk tachyons, one cannot conclude that the central charge does not need to 
decrease. 
Therefore, according to the above discussions, one might expect that a condensation of a bulk tachyon 
would lead the theroy to the one away from conformal theories, and the result of the condensation would be 
a complicated, maybe unknown, theory. 

\vspace{3mm}

However, it does not seem to be so simple \cite{suyama}. 
Consider a non-linear sigma model whose low energy effective action is 
\begin{eqnarray}\coord{}\boxAlignEqnarray{\leftCoord{}
S &=& \frac1{2\kappa^2}\int d^Dx\sqrt{-g}e^{-2\Phi}\left[ R+4\partial_\mu\Phi\partial^\mu\Phi
     \leftCoord{}-\frac1{12}H_{\mu\nu\rho}H^{\mu\nu\rho}\right. \nonumber \rightCoord{}\\
&&\leftCoord{} \hspace*{2cm}\left.\frac{\leftCoord{} }{\rightCoord{} } \rightCoord{}
     \leftCoord{}-f_{ab}(\phi)F^a_{\mu\nu}F^{b\mu\nu}-g_{IJ}(\phi)D_\mu\phi^I D^\mu\phi^J-V(\phi)\right]. \rightCoord{}
          \label{gaugedSUGRA}
\rightCoord{}}{0mm}{5}{6}{
S &=& \frac1{2\kappa^2}\int d^Dx\sqrt{-g}e^{-2\Phi}\left[ R+4\partial_\mu\Phi\partial^\mu\Phi
     -\frac1{12}H_{\mu\nu\rho}H^{\mu\nu\rho}\right. \\
&& \hspace*{2cm}\left.\frac{ }{ } 
     -f_{ab}(\phi)F^a_{\mu\nu}F^{b\mu\nu}-g_{IJ}(\phi)D_\mu\phi^I D^\mu\phi^J-V(\phi)\right]. 
          }{1}\coordE{}\end{eqnarray}
The action of this form appears, for example, in the heterotic string theory compactified on a torus with 
flux \cite{hetero}. 

Suppose that the scalar potential \myHighlight{$V(\phi)$}\coordHE{} has two critical points, at \myHighlight{$\phi^I=\phi^I_1$}\coordHE{} and 
\myHighlight{$\phi^I=\phi^I_2$}\coordHE{}, and \myHighlight{$V(\phi_1)>V(\phi_2)$}\coordHE{}. 
Then each classical solution corresponding to each critical point will correspond to a background of 
a {\it critical} string theory. 
If \myHighlight{$\phi^I=\phi^I_1$}\coordHE{} and \myHighlight{$\phi^I=\phi^I_2$}\coordHE{} are connected to each other in the space of field configurations, 
then it would be natural to expect 
that the background corresponding to \myHighlight{$\phi^I_1$}\coordHE{} could decay into the other background corresponding to 
\myHighlight{$\phi^I_2$}\coordHE{}. 
This decay cannot be described by a marginal deformation of the sigma model, and thus during this decay the 
central charge would decrease, contradicting the fact that both backgrounds correspond to critical string 
theories. 

\vspace{3mm}

A possible resolution of this puzzle is proposed in \cite{suyama}. 
It is summarized as follows. 
The decay discussed above indeed decrease the central charge, which can be shown in the range of validity of the 
tree level approximation, but whole trajectory of the RG flow in the theory space does not lie within the range. 
Let us explain it briefly. 

The central charge of a sigma model can be defined for an off-shell background. 
The existence of the Virasoro algebra is ensured by requiring that all beta-functionals of the sigma model, 
except for that \myHighlight{$\beta_{\Phi}$}\coordHE{} of the dilaton, vanish. 
Then the central charge is equal to \myHighlight{$\beta_{\Phi}$}\coordHE{}. 
Note that \myHighlight{$\beta_{\Phi}$}\coordHE{} is constant provided that the other beta-functionals vanish. 
(The central charge for a more general situation is discussed in \cite{Tseytlin}.) 

Consider a background in which the dilaton is constant so that the tree level approximation is reliable. 
We set the two-form field and all the gauge fields to be zero and the scalars to be constant. 
Then the remaining equations to ensure the existence of the Virasoro algebra are 
\begin{equation}\coord{}\boxEquation{
R_{\mu\nu}=0, \hspace{5mm} \frac{\partial V}{\partial \phi^I}=0.
}{
R_{\mu\nu}=0, \hspace{5mm} \frac{\partial V}{\partial \phi^I}=0.
}{ecuacion}\coordE{}\end{equation}
Thus we can choose the flat background and \myHighlight{$\phi^I=\phi^I_1$}\coordHE{} or \myHighlight{$\phi^I_2$}\coordHE{}. 
Now \myHighlight{$\beta_{\Phi}$}\coordHE{} is proportional to the equation of motion of \myHighlight{$\Phi$}\coordHE{}, 
\begin{eqnarray}\coord{}\boxAlignEqnarray{\leftCoord{}
\beta_{\Phi} &\propto& 
\leftCoord{}-R+4\partial_\mu\Phi\partial^\mu\Phi-4\nabla_\mu\partial^\mu\Phi
\leftCoord{}+\frac1{12}H_{\mu\nu\rho}H^{\mu\nu\rho}
        \nonumber \rightCoord{}\\
&&\leftCoord{}+f_{ab}(\phi)F_{\mu\nu}^aF^{b\mu\nu}+g_{IJ}(\phi)D_\mu\phi^ID^\mu\phi^J+V(\phi) \nonumber \rightCoord{}\\
&\leftCoord{}=& V(\phi), \rightCoord{}
\rightCoord{}}{0mm}{5}{5}{
\beta_{\Phi} &\propto& 
-R+4\partial_\mu\Phi\partial^\mu\Phi-4\nabla_\mu\partial^\mu\Phi
+\frac1{12}H_{\mu\nu\rho}H^{\mu\nu\rho}
        \\
&&+f_{ab}(\phi)F_{\mu\nu}^aF^{b\mu\nu}+g_{IJ}(\phi)D_\mu\phi^ID^\mu\phi^J+V(\phi) \\
&=& V(\phi), 
}{1}\coordE{}\end{eqnarray}
with a positive proportionality constant. 
This indicates that the central charge for each critical point \myHighlight{$\phi^I=\phi^I_c$}\coordHE{} of the potential is 
proportional to \myHighlight{$V(\phi_c)$}\coordHE{}, and therefore the rolling down the potential decreases the central 
charge of the sigma model. 

We have not analyzed all the process of the decay. 
Since \myHighlight{$\beta_{\Phi}\ne 0$}\coordHE{}, the sigma model further flows. 
If \myHighlight{$V(\phi_2)\ne0$}\coordHE{}, then the classical solution corresponding to \myHighlight{$\phi^I_2$}\coordHE{} is typically a linear 
dilaton background, and thus this does not correspond to a weakly-coupled theory. 
Therefore the final stage of the flow is out of reach of the tree level approximation. 
When \myHighlight{$V(\phi_2)=0$}\coordHE{}, then by our assumption \myHighlight{$V(\phi_1)>0$}\coordHE{}, and thus we cannot analyze the initial stage of 
the flow. 




















\vspace{1cm}

\section{Rolling down with constant dilaton}  \label{Dindep}

\vspace{5mm}

As we briefly commented in \cite{suyama}, when the scalar potential does not depend on the dilaton, transitions 
between critical points of the potential would occur without going into a strongly-coupled string theory. 
Then the same puzzle discussed in the previous section will appear in this case.  
Thus we have to investigate such a case and find what is going on. 

We would like to consider a string theory whose low energy effective theory contains a non-trivial scalar 
potential which is a function independent of the dilaton \myHighlight{$\Phi$}\coordHE{}. 
There is a supergravity theory of this kind \cite{5dim1}\cite{5dim2} which is familiar in the study of AdS/CFT 
correspondence \cite{AdS/CFT}. 
This is a five-dimensional theory with maximal supersymmetry, which is believed to be a consistent truncation of 
Type IIB supergravity compactified on \myHighlight{$S^5$}\coordHE{}. 
It contains 42 scalars which is identified with the coordinates of the coset \myHighlight{$E_{6(6)}/USp(4)$}\coordHE{}, and the scalar 
potential is a function on the coset. 
The potential is invariant under \myHighlight{$SL(2,{\bf R})\subset E_{6(6)}$}\coordHE{}, and this corresponds to the 
\myHighlight{$SL(2,{\bf R})$}\coordHE{} invariance of Type IIB supergravity. 
Thus the potential is independent of the dilaton. 
Note that we have discussed in the Einstein frame. 
In the string frame, the scalar potential is multiplied by the exponential of \myHighlight{$\Phi$}\coordHE{}. 

The scalar potential has many critical points \cite{5dim1}\cite{critical1}\cite{critical2}\cite{critical3} 
and they are interpreted, in AdS/CFT correspondence, IR fixed 
points of the \myHighlight{${\cal N}$}\coordHE{}=4 super Yang-Mills theory in four dimensions perturbed by relevant deformations. 
Then, according to the correspondence, transitions between the critical points would also be relevant 
deformations of the corresponding Type IIB string theory. 
Therefore AdS/CFT correspondence would imply that there may be such a process which connects two critical string 
theories corresponding to the critical points, although along the whole process the tree level approximation 
seems to be reliable. 

\vspace{3mm}

We will show in this section that the situation considered above is qualitatively different from that 
discussed before. 
One difference is that one cannot obtain a nonvanishing \myHighlight{$\beta_{\Phi}$}\coordHE{} with a constant \myHighlight{$\Phi$}\coordHE{}. 
Another is that one has to compare between central charges defined for theories with different backgrounds. 

Let us explain in details. 
Consider an action
\begin{equation}\coord{}\boxEquation{
S = \frac1{2\kappa^2}\int d^dx\sqrt{-g}\ \left[ \ R-\frac12\partial_\mu\Phi\partial^\mu\Phi
   -\frac12g_{IJ}(\phi)\partial_\mu\phi^I\partial^\mu\phi^J-V(\phi) \ \right],
}{
S = \frac1{2\kappa^2}\int d^dx\sqrt{-g}\ \left[ \ R-\frac12\partial_\mu\Phi\partial^\mu\Phi
   -\frac12g_{IJ}(\phi)\partial_\mu\phi^I\partial^\mu\phi^J-V(\phi) \ \right],
}{ecuacion}\coordE{}\end{equation}
and suppose that \myHighlight{$\partial V/\partial \Phi=0$}\coordHE{}. 
This can be understood as the action of the gauged supergravity mentioned above with all other fields set to be 
zero. 
Obviously, \myHighlight{$\Phi$}\coordHE{}=const. satisfies the equation of motion of the dilaton. 
Thus in this case, the conditions for the existence of the Virasoro algebra with \myHighlight{$\Phi$}\coordHE{}=const. is equivalent 
to the conditions for the conformal invariance. 
That is, one cannot obtain the Virasoro algebra with a non-trivial central charge. 
This result might indicate that the transition is not described by a relevant deformation, since the initial and 
the final theory have the same value of the central charge. 
However, I would like to claim that there is a subtle issue and one has to take into account in what background 
the theory lives. 

The beta-functional \myHighlight{$\beta_G$}\coordHE{} of the metric is a linear combination of the equations of motion of the metric and 
the dilaton, and thus when \myHighlight{$\Phi$}\coordHE{}=const. the condition \myHighlight{$\beta_G=0$}\coordHE{} is equivalent to the equation of motion 
of the metric. 
Suppose that the scalars are constant and at a critical point of \myHighlight{$V(\phi)$}\coordHE{}, then the equation of motion of the 
metric is 
\begin{equation}\coord{}\boxEquation{
R_{\mu\nu}-\frac12g_{\mu\nu}R+\frac12g_{\mu\nu}V(\phi_c) = 0.
}{
R_{\mu\nu}-\frac12g_{\mu\nu}R+\frac12g_{\mu\nu}V(\phi_c) = 0.
}{ecuacion}\coordE{}\end{equation}
Therefore, the background metric is different for each critical point of \myHighlight{$V(\phi)$}\coordHE{}, which is in contrast with the 
previous case where one can take the flat metric for every critical points. 
In general, central charge measures the degrees of freedom of the system, which would correspond to physical 
states in the theory. 
However, the notion of physical states would depend on the asymptotic nature of the background. 
Thus the naive comparison of such a quantity in different background would possibly be a subtle issue. 
This comment would become to sound more natural when the central charge is related to a physical quantity in the 
target space theory, namely, the total energy of the system. 




















\vspace{1cm}

\section{Central charge as spacetime energy} \label{center}

\vspace{5mm}

We have discussed the central charge of a sigma model in a special situation in which all but one beta-functional 
for the background fields vanish. 
There is an expression of the central charge, or c-function, for a more general situation \cite{Tseytlin}. 

The Zamolodchikov c-function \myHighlight{$C(r)$}\coordHE{} is defined in terms of correlation functions of a two-dimensional field theory, 
\begin{equation}\coord{}\boxEquation{
C(r) = 2F(r)-G(r)-\frac38H(r),
}{
C(r) = 2F(r)-G(r)-\frac38H(r),
}{ecuacion}\coordE{}\end{equation}
where \myHighlight{$r^2=z\bar{z}$}\coordHE{} and 
\begin{eqnarray}\coord{}\boxAlignEqnarray{\leftCoord{}
F(r) &=& z^4\langle T_{zz}(z,\bar{z})T_{zz}(0,0)\rangle, \nonumber \rightCoord{}\\\leftCoord{}
G(r) &=& 4z^3\bar{z}\langle T_{zz}(z,\bar{z})T_{z\bar{z}}(0,0)\rangle \nonumber \rightCoord{}\\
&\leftCoord{}=& 4z^3\bar{z}\langle T_{z\bar{z}}(z,\bar{z})T_{zz}(0,0)\rangle, 
          \label{c_fn}\rightCoord{}\\\leftCoord{}
H(r) &=& 16z^2\bar{z}^2\langle T_{z\bar{z}}(z,\bar{z})T_{z\bar{z}}(0,0)\rangle\ge 0. \nonumber\rightCoord{} 
\rightCoord{}}{0mm}{4}{6}{
F(r) &=& z^4\langle T_{zz}(z,\bar{z})T_{zz}(0,0)\rangle, \\
G(r) &=& 4z^3\bar{z}\langle T_{zz}(z,\bar{z})T_{z\bar{z}}(0,0)\rangle \\
&=& 4z^3\bar{z}\langle T_{z\bar{z}}(z,\bar{z})T_{zz}(0,0)\rangle, 
          \\
H(r) &=& 16z^2\bar{z}^2\langle T_{z\bar{z}}(z,\bar{z})T_{z\bar{z}}(0,0)\rangle\ge 0. }{1}\coordE{}\end{eqnarray}
We have assumed the rotational invariance. 
Then the derivative of \myHighlight{$C(r)$}\coordHE{} is non-positive
\begin{equation}\coord{}\boxEquation{
r\frac d{dr}C(r) = -\frac34H(r),
}{
r\frac d{dr}C(r) = -\frac34H(r),
}{ecuacion}\coordE{}\end{equation}
due to the conservation of the energy-momentum tensor. 

The functions (\ref{c_fn}) can be calculated perturbatively for a sigma model, and thus one can obtain the 
c-function in terms of background fields. 
In the following calculations, only \myHighlight{$G_{ij},\ B_{ij},\ \Phi$}\coordHE{} are turned on. 
The result \cite{Tseytlin}\cite{GHMS} is that the c-function is an integral transformation of 
\begin{equation}\coord{}\boxEquation{
c(\mu) = -\frac{3\alpha'}2\frac{S(\mu)}{V(\mu)},
    \label{c=action}
}{
c(\mu) = -\frac{3\alpha'}2\frac{S(\mu)}{V(\mu)},
    }{ecuacion}\coordE{}\end{equation}
where \myHighlight{$\mu$}\coordHE{} is the renormalization scale, and 
\begin{eqnarray}\coord{}\boxAlignEqnarray{\leftCoord{}
S(\mu) &=& \frac1{2\kappa^2\alpha'}\int d^dx\sqrt{G}e^{-2\Phi}(G^{ij}\beta_{G,ij}-4\beta_\Phi) \nonumber \rightCoord{}\\
&\leftCoord{}=& \frac1{2\kappa^2}\int d^dx\sqrt{G}e^{-2\Phi}\left(R-\frac1{12}H^2-4(\nabla\Phi)^2+4\nabla^2\Phi\right) \rightCoord{}
          \leftCoord{}+O(\alpha'), \rightCoord{} 
   \label{action} \rightCoord{}\\\leftCoord{}
V(\mu) &=& \frac1{2\kappa^2}\int d^dx\sqrt{G}e^{-2\Phi}. \rightCoord{} 
\rightCoord{}}{0mm}{4}{7}{
S(\mu) &=& \frac1{2\kappa^2\alpha'}\int d^dx\sqrt{G}e^{-2\Phi}(G^{ij}\beta_{G,ij}-4\beta_\Phi) \\
&=& \frac1{2\kappa^2}\int d^dx\sqrt{G}e^{-2\Phi}\left(R-\frac1{12}H^2-4(\nabla\Phi)^2+4\nabla^2\Phi\right) 
          +O(\alpha'),  
   \\
V(\mu) &=& \frac1{2\kappa^2}\int d^dx\sqrt{G}e^{-2\Phi}.  
}{1}\coordE{}\end{eqnarray}
The scale dependence comes from that of each background field. 
Note that we have considered the target space with Euclidean signature, or in other words, a spatial part of the 
whole spacetime. 

The quantity (\ref{action}) is nothing but a part of the low energy effective action. 
The identification of the c-function with the effective action is plausible since this is consistent with the 
fact that the criticality 
of the c-function (conformal invariance) corresponds to the criticality of the effective action (on-shell 
condition). 
In addition, the expression (\ref{c=action}) is a natural generalization of the fact that \myHighlight{$c\propto\beta_\Phi$}\coordHE{} 
when \myHighlight{$\beta_G=0$}\coordHE{}, since then \myHighlight{$\beta_\Phi$}\coordHE{} is constant. 
This expression (\ref{c=action}) is used in \cite{GHMS} to show that the total energy decreases after a 
condensation of a localized tachyon. 
One can see from (\ref{c=action}) that if the tachyon condensation is a phenomenon local in the target space, 
the effect of this to the central charge is suppressed by \myHighlight{$1/V$}\coordHE{}. 

\vspace{3mm}

A natural generalization of the c-function (\ref{c=action}) to that of the whole theory would be obtained by 
substituting \myHighlight{$S$}\coordHE{} with the low energy effective action itself and \myHighlight{$V$}\coordHE{} with the spacetime volume,
\begin{equation}\coord{}\boxEquation{
c(\mu) = -\frac{3\alpha'}2\frac{S_{total}(\mu)}{V_{total}(\mu)}.
   \label{totalC}
}{
c(\mu) = -\frac{3\alpha'}2\frac{S_{total}(\mu)}{V_{total}(\mu)}.
   }{ecuacion}\coordE{}\end{equation}
When the sigma model is separable, that is, a product of two sigma models, and in particular, the one sigma model 
is chosen to correspond to the flat spacetime, then \myHighlight{$V_{total}$}\coordHE{} is the product of volumes of these two models, 
and (\ref{totalC}) reduces to (\ref{c=action}) of the other non-trivial sigma model. 

The definition (\ref{totalC}) is also plausible in another reason. 
It is known that the spacetime action in the Einstein frame is equal to the time integral of the Hamiltonian 
(times \myHighlight{$-1$}\coordHE{}) for a stationary solution \cite{GR}. 
Therefore, the decrease of the c-function seems to be related to the decrease of the total energy, similar to the 
discussion in \cite{GHMS}. 
In general relativity, the physical energy is defined as the difference with a reference background, since the 
straightforward evaluation of the energy provides a divergence in general. 
If one choose a vacuum solution as the reference, then its energy is zero by definition. 
This suggests that it is the effective action with a suitable subtraction that corresponds to the central charge, 
since the central charge for a classical solution should be zero by definition. 
The subtraction depends on what background one chooses as the reference, and thus the definition of the 
central charge seems to be different for different background in the spacetime point of view . 
Therefore, the comparison of central charges for different backgrounds would be a subtle issue. 

\vspace{3mm}

There is a quantity which can be defined for arbitrary background and expected naturally to decrease in physical 
processes. 
It is the quantity (\ref{totalC}) itself, not a difference with some reference value. 
The action diverges due to the infinite volume, and thus (\ref{totalC}) can be finite with a suitable 
regularization procedure. 
As explained above, this reduces to the ordinary c-function when the sigma model is a product of the flat 
spacetime CFT and something, and the dilaton is trivial. 

An important point to define (\ref{totalC}) might be that the spacetime action should be written in the Einstein 
frame to relate it to the spacetime energy. 
This would suggest that physical processes could be different from those anticipated from the c-theorem when the 
dilaton becomes non-trivial. 
However, difference of energies for different vacua is ambiguous, and thus there could be a surface term which 
is added to the Einstein frame action to provide a modified c-function. 

Interestingly enough, an action of this familiar form 
\begin{equation}\coord{}\boxEquation{
S = \frac1{2\kappa^2}\int d^Dx\sqrt{-g}e^{-2\Phi}(R+4(\nabla\Phi)^2-V(\phi_c))
    \label{SFaction}
}{
S = \frac1{2\kappa^2}\int d^Dx\sqrt{-g}e^{-2\Phi}(R+4(\nabla\Phi)^2-V(\phi_c))
    }{ecuacion}\coordE{}\end{equation}
has the following nice property. 
This is different from the Einstein frame action but the difference is only a surface term which does not vanish in 
general classical solutions. 
The value of the action (\ref{SFaction}) for each classical solution is 
\begin{equation}\coord{}\boxEquation{
S = \frac1{2\kappa^2}\int d^Dx\sqrt{-g}e^{-2\Phi}(-2V(\phi_0)),
}{
S = \frac1{2\kappa^2}\int d^Dx\sqrt{-g}e^{-2\Phi}(-2V(\phi_0)),
}{ecuacion}\coordE{}\end{equation}
and thus, by substituting this into (\ref{totalC}), one obtains 
\begin{equation}\coord{}\boxEquation{
c = 3\alpha'V(\Phi_0).
}{
c = 3\alpha'V(\Phi_0).
}{ecuacion}\coordE{}\end{equation}
This quantity decreases as the scalars roll down the potential, as expected. 
It should be noted that the surface term added is not a unique choice to provide a decreasing function. 
Therefore it is desired that there would be a criterion to fix this ambiguity. 




















\vspace{1cm}

\section{Internal space geometry from scalar vevs}  \label{potential}

\vspace{5mm}

In the previous section, we have discussed the relation between the c-function of a sigma model and the action of 
the low energy effective theory, which can be related to the total spacetime energy of the system. 
This relation would provide a clear understanding of the c-theorem; the energetically-favored process occurs 
physically. 
Moreover, a slight difference between the c-function obtained in \cite{Tseytlin} and the total energy would 
suggest that there might be some modification of the c-theorem when the dilaton becomes non-trivial, that is, 
the tree level approximation for string theory is not reliable. 

In the rest of this paper, we will show another use of the spacetime theory to investigate a condensation of a 
closed string tachyon propagating in the bulk. 
That is, we deduce a final state of the tachyon condensation. 

\vspace{3mm}

We consider in this section a gauged supergravity which is a consistent truncation of a string theory or 
M-theory compactified on a non-trivial manifold. 
The theory has a scalar potential, which has many critical points. 
Some of them correspond to stable vacua of the theory, and the others to unstable vacua. 
When one defines the theory on one of the unstable vacua, it would be deformed via a condensation of an unstable 
mode, until the theory will reach another vacuum which is then stable. 
Therefore, one can obtain the knowledge of the condensation by examining the shape of the scalar potential. 

Recall that the scalars in the gauged supergravity in lower dimensions originally come from fields of a theory in 
higher dimensions, 
for example, metric of the internal manifold. 
Thus the vevs of the scalars in a lower dimensional theory should have information of the target space geometry in 
a higher dimensional theory. 
This is indeed the case, and some of the critical points of the scalar potential are related to some internal 
manifolds \cite{AdS/CFT}. 
Moreover, in a particular case, there is a general formula \cite{embedding} to obtain the metric of the internal 
manifold from the vevs of the scalars. 
It is clear that bulk tachyon condensations inevitably deform background geometries, since they cannot be 
decoupled from the bulk gravity. 
Therefore it is important to know how the background geometry is deformed via the tachyon condensation. 
The relation between the scalar vevs and the internal geometry will enable one to investigate it. 

The most interesting final state of the condensation would be the most stable vacuum. 
In ordinary supersymmetric field theory without gravity, the most stable vacua is the states with zero energy. 
This is also the case for open string tachyon condensations in superstring \cite{open}. 
However, in supergravity theories the situation is different since its scalar potential is not necessarily 
non-negative. 
Moreover, in many cases the potential is not even bounded below. 
Naively, in such cases, the system would roll down the potential eternally toward the ``bottom". 
Thus it would be interesting to ask to what kind of theory it is deformed by such an evolution of the scalar vevs. 
We discuss this issue below with explicit examples. 




















\vspace{5mm}

\subsection{M-theory on \myHighlight{$S^7$}\coordHE{}}

\vspace{5mm}

In this subsection, we discuss an \myHighlight{${\cal N}=8$}\coordHE{} gauged supergravity in four dimensions \cite{CJ}, which is a 
consistent truncation of the eleven-dimensional supergravity compactified on \myHighlight{$S^7$}\coordHE{} \cite{truncation}. 
This theory has been well-studied and several critical points of its scalar potential have been found 
\cite{extrema}. 

\myHighlight{$S^7$}\coordHE{} can be regarded as an \myHighlight{$S^1$}\coordHE{}-bundle over \myHighlight{$CP^3$}\coordHE{}. 
Therefore, M-theory compactified on \myHighlight{$S^7$}\coordHE{} can be regarded as Type IIA string theory compactified on \myHighlight{$CP^3$}\coordHE{} with 
R-R fluxes. 
From this, we can consider instabilities which appear in the gauged supergravity as unstable modes of the Type IIA 
strings. 
Since the eleven-dimensional metric of the classical solution is a direct product \myHighlight{$AdS_4\times S^7$}\coordHE{}, the R-R 
fluxes do not depend on the non-compact (\myHighlight{$AdS$}\coordHE{}) coordinates. 
If there is an unstable mode coming from scalars, it is a tachyon in the bulk and thus its condensation would 
decrease the effective c-function discussed in section \ref{center}. 
The situation considered is similar to the Melvin background in M-theory \cite{Melvin}, except for the 
compactness of the relevant geometry. 

\vspace{3mm}

The gauged supergravity contains 70 scalars which can be identified with the coordinates of the coset 
\myHighlight{$E_{7(7)}/SU(8)$}\coordHE{}. 
It is convenient to represent them in terms of a group element of the \myHighlight{$E_{7(7)}$}\coordHE{} which acts on the {\bf 56} 
representation, 
\begin{equation}\coord{}\boxEquation{
{\cal V} = \left(
\begin{array}{cc}
{u_{ij}}^{IJ} & v_{ijKL} \\ \bar{v}^{klIJ} & {\bar{u}^{kl}}_{KL}
    \label{vielbein}
\end{array}
\right).
}{
{\cal V} = \left(
\begin{array}{cc}
{u_{ij}}^{IJ} & v_{ijKL} \\ \bar{v}^{klIJ} & {\bar{u}^{kl}}_{KL}
    \end{array}
\right).
}{ecuacion}\coordE{}\end{equation}
We follow the notations and conventions of \cite{notation}. 
The indices take values from 1 to 8, and \myHighlight{$ij$}\coordHE{}, \myHighlight{$KL$}\coordHE{} etc. are anti-symmetrized. 
Note that an \myHighlight{$SU(8)$}\coordHE{} part of \myHighlight{${\cal V}$}\coordHE{} is not physical degrees of freedom. 

The scalar potential is given as 
\begin{equation}\coord{}\boxEquation{
{\cal P(V)} = \frac1{24}|A_2|^2-\frac34|A_1|^2,
   \label{P(V)}
}{
{\cal P(V)} = \frac1{24}|A_2|^2-\frac34|A_1|^2,
   }{ecuacion}\coordE{}\end{equation}
where the tensors \myHighlight{${A_1}^{ij}$}\coordHE{} and \myHighlight{${{A_2}_i}^{jkl}$}\coordHE{} are complicated functions of \myHighlight{$u$}\coordHE{} and \myHighlight{$v$}\coordHE{} in 
(\ref{vielbein}). 
One can see that the potential has a negative-definite term, and this potential is indeed unbounded below. 

The directions along which the \myHighlight{${\cal P(V)}$}\coordHE{} diverges in a limit correspond to the non-compact directions of 
\myHighlight{$E_{7(7)}$}\coordHE{}, or more conveniently, the non-compact directions of \myHighlight{$SL(8,{\bf R})\subset E_{7(7)}$}\coordHE{}. 
Explicitly, we consider one-parameter subgroups of \myHighlight{$E_{7(7)}$}\coordHE{} corresponding to 
\begin{equation}\coord{}\boxEquation{
X = \left(
\begin{array}{cccc}
e^{a_1s} & & & \\ & e^{a_2s} & & \\ & & \ddots & \\ & & & e^{a_8s}
   \label{SL(8)}
\end{array}
\right) \hspace{5mm} \in \ SL(8,{\bf R}), 
}{
X = \left(
\begin{array}{cccc}
e^{a_1s} & & & \\ & e^{a_2s} & & \\ & & \ddots & \\ & & & e^{a_8s}
   \end{array}
\right) \hspace{5mm} \in \ SL(8,{\bf R}), 
}{ecuacion}\coordE{}\end{equation}
with \myHighlight{$\sum a_k=0$}\coordHE{}. 
The explicit form of the potential (\ref{P(V)}) has been calculated in special cases \cite{explicitP}, 
by using as \myHighlight{${\cal V}$}\coordHE{} the element of the one-parameter subgroup corresponding to (\ref{SL(8)}), 
\begin{eqnarray}\coord{}\boxAlignEqnarray{\leftCoord{}
{\rightCoord{}\leftCoord{}\cal P}_{7,1} &=& \frac18(-35e^{2s}-14e^{-6s}+e^{-14s}), \rightCoord{}\\\leftCoord{}
{\rightCoord{}\leftCoord{}\cal P}_{6,2} &=& -3(e^{2s}+e^{-2s}), 
      \label{62}  \rightCoord{}\\\leftCoord{}
{\rightCoord{}\leftCoord{}\cal P}_{5,3} &=& -\frac38(5e^{2s}+10e^{-2s/3}+e^{-10s/3}), \rightCoord{}\\\leftCoord{}
{\rightCoord{}\leftCoord{}\cal P}_{4,4} &=& -(e^{2s}+4+e^{-2s}), 
\rightCoord{}}{0mm}{8}{9}{
{\cal P}_{7,1} &=& \frac18(-35e^{2s}-14e^{-6s}+e^{-14s}), \\
{\cal P}_{6,2} &=& -3(e^{2s}+e^{-2s}), 
      \\
{\cal P}_{5,3} &=& -\frac38(5e^{2s}+10e^{-2s/3}+e^{-10s/3}), \\
{\cal P}_{4,4} &=& -(e^{2s}+4+e^{-2s}), 
}{1}\coordE{}\end{eqnarray}
where \myHighlight{${\cal P}_{p,8-p}$}\coordHE{} is the potential (\ref{P(V)}) for the element (\ref{SL(8)}) with 
\begin{equation}\coord{}\boxEquation{
a_1=\cdots=a_p, \ a_{p+1}=\cdots=a_8,
}{
a_1=\cdots=a_p, \ a_{p+1}=\cdots=a_8,
}{ecuacion}\coordE{}\end{equation}
and their values are suitably chosen. 
This corresponds to the direction which preserves \myHighlight{$SO(p)\times SO(8-p)$}\coordHE{} symmetry. 
Note that \myHighlight{${\cal P}_{p,q}(s)={\cal P}_{q,p}(-ps/q)$}\coordHE{}. 
A point \myHighlight{$s=0$}\coordHE{} is always a critical point which corresponds to the round \myHighlight{$S^7$}\coordHE{}. 
One can see that there are directions along which \myHighlight{${\cal P}\to-\infty$}\coordHE{} as \myHighlight{$|s|\to\infty$}\coordHE{}. 

\vspace{3mm}

As mentioned above, the vevs of the scalars contain information of the deformation of \myHighlight{$S^7$}\coordHE{}. 
How \myHighlight{$S^7$}\coordHE{} is deformed for \myHighlight{$|s|\to\infty$}\coordHE{} ?
To investigate it, one has to reconstruct the metric of the \myHighlight{$S^7$}\coordHE{} from the scalar vevs. 
This was achieved in \cite{embedding}, and in particular, the simple formula for the metric for the vevs 
corresponding to (\ref{SL(8)}) was obtained. 
The \myHighlight{$S^7$}\coordHE{} is deformed to an ellipsoid defined by 
\begin{equation}\coord{}\boxEquation{
x_mP_{mn}x_n = \mbox{const.}, \hspace{5mm} (x_1,\cdots,x_8) \in {\bf R}^8,
}{
x_mP_{mn}x_n = \mbox{const.}, \hspace{5mm} (x_1,\cdots,x_8) \in {\bf R}^8,
}{ecuacion}\coordE{}\end{equation}
where \myHighlight{$P=X^2$}\coordHE{}, and the metric on this ellipsoid is the induced metric with a scale factor \myHighlight{$\mu$}\coordHE{},
\begin{eqnarray}\coord{}\boxAlignEqnarray{\leftCoord{}
g_{mn} &=& 2^{2/9}\mu^{-2/3}(\delta_{mn}-\hat{n}_m\hat{n}_n), \rightCoord{}\\\leftCoord{}
\mu^2 &=& x_mP_{ml}P_{ln}x_n,
\rightCoord{}}{0mm}{2}{3}{
g_{mn} &=& 2^{2/9}\mu^{-2/3}(\delta_{mn}-\hat{n}_m\hat{n}_n), \\
\mu^2 &=& x_mP_{ml}P_{ln}x_n,
}{1}\coordE{}\end{eqnarray}
where \myHighlight{$\hat{n}_m$}\coordHE{} is the unit normal vector of the ellipsoid. 
From this result, one can see that the \myHighlight{$x_k$}\coordHE{}-direction shrinks (extends) when \myHighlight{$a_k>0$}\coordHE{} (\myHighlight{$a_k<0$}\coordHE{}) as 
\myHighlight{$s\to +\infty$}\coordHE{}, up to the overall factor. 
Since \myHighlight{$X$}\coordHE{} is an element of \myHighlight{$SL(8,{\bf R})$}\coordHE{}, not all the directions simultaniously shrink or extend. 
In other words, there must be a direction which extends as the scalar vevs grow. 

Among the directions for \myHighlight{${\cal P}\to -\infty$}\coordHE{}, an interesting case would be the direction which preserves 
\myHighlight{$SO(2)\times SO(6)$}\coordHE{} symmetry. 
In this case, the deformed \myHighlight{$S^7$}\coordHE{} is given as 
\begin{equation}\coord{}\boxEquation{
e^{3t}((x_1)^2+(x_2)^2)+e^{-t}((x_3)^2+\cdots+(x_8)^2)=\rho^2
}{
e^{3t}((x_1)^2+(x_2)^2)+e^{-t}((x_3)^2+\cdots+(x_8)^2)=\rho^2
}{ecuacion}\coordE{}\end{equation}
in \myHighlight{${\bf R}^8$}\coordHE{}, and \myHighlight{$t$}\coordHE{} is proportional to \myHighlight{$s$}\coordHE{} in (\ref{SL(8)}). 
The metric on this ellipsoid which is manifestly \myHighlight{$SO(2)\times SO(6)$}\coordHE{} invariant is 
\begin{equation}\coord{}\boxEquation{
ds^2 = \rho^2\left[ \left(e^t+e^{-3t}\frac{r^2}{1-r^2}\right)dr^2+e^tr^2d\Omega_5^2+e^{-3t}(1-r^2)d\theta^2
       \right],
}{
ds^2 = \rho^2\left[ \left(e^t+e^{-3t}\frac{r^2}{1-r^2}\right)dr^2+e^tr^2d\Omega_5^2+e^{-3t}(1-r^2)d\theta^2
       \right],
}{ecuacion}\coordE{}\end{equation}
where \myHighlight{$0\le r\le1$}\coordHE{}, so that this is a product of a six-dimensional ball and a circle whose radius depends on \myHighlight{$r$}\coordHE{}. 
When \myHighlight{$t$}\coordHE{} is positive and large, then 
\begin{equation}\coord{}\boxEquation{
ds^2 \sim \rho^2\left[ d\tilde{r}^2+\tilde{r}^2d\Omega_5^2+e^{-3t}(1-e^{-t}\tilde{r}^2)d\theta^2\right],
}{
ds^2 \sim \rho^2\left[ d\tilde{r}^2+\tilde{r}^2d\Omega_5^2+e^{-3t}(1-e^{-t}\tilde{r}^2)d\theta^2\right],
}{ecuacion}\coordE{}\end{equation}
where \myHighlight{$\tilde{r}=e^{t/2}r$}\coordHE{}. 
Therefore, in the large \myHighlight{$t$}\coordHE{} limit, the ellipsoid is approximately the direct product of the six-dimensional flat 
space and the small \myHighlight{$S^1$}\coordHE{} with a constant radius, as long as \myHighlight{$\tilde{r}<<e^{t/2}$}\coordHE{}. 
By choosing this \myHighlight{$S^1$}\coordHE{} as the M-theory circle, this limiting theory would be regarded as a weak coupling 
Type IIA string theory. 
Thus this might suggest that a tachyon condensation of this theory would have a final state which is a 
weakly-coupled Type IIA string theory in the flat spacetime. 
Note that no R-R 1-form appears via the dimensional reduction. 
Since the \myHighlight{$S^1$}\coordHE{} chosen above is not the same circle with the one, reducing along which provides Type IIA string 
theory compactified on \myHighlight{$CP^3$}\coordHE{}, the relation between the initial and the final states would be non-trivial. 

However, the argument given above is too naive since we did not take into account the overall factor \myHighlight{$\mu$}\coordHE{}. 
In fact, the proper metric is, up to a trivial numerical factor,
\begin{eqnarray}\coord{}\boxAlignEqnarray{\leftCoord{}
ds^2_M &=& \rho^2\mu^{-2/3}
    \left[ \left(e^t+e^{-3t}\frac{\leftCoord{}r^2}{\rightCoord{}1-r^2}\right)dr^2+e^tr^2d\Omega_5^2+e^{-3t}(1-r^2)d\theta^2\right], \rightCoord{}\\\leftCoord{}
\mu^2 &=& e^{3t}(1-r^2)+e^{-t}r^2. \rightCoord{}
\rightCoord{}}{0mm}{3}{5}{
ds^2_M &=& \rho^2\mu^{-2/3}
    \left[ \left(e^t+e^{-3t}\frac{r^2}{1-r^2}\right)dr^2+e^tr^2d\Omega_5^2+e^{-3t}(1-r^2)d\theta^2\right], \\
\mu^2 &=& e^{3t}(1-r^2)+e^{-t}r^2. 
}{1}\coordE{}\end{eqnarray}
Since the radius of the \myHighlight{$S^1$}\coordHE{} becomes small as \myHighlight{$t\to+\infty$}\coordHE{}, this limit indeed 
corresponds to a weak coupling Type IIA 
string theory on a manifold whose metric is 
\begin{equation}\coord{}\boxEquation{
ds^2_{IIA} = \rho^2e^{-3t/2}\mu^{-1}\sqrt{1-r^2}
        \left[ \left(e^t+e^{-3t}\frac{r^2}{1-r^2}\right)dr^2+e^tr^2d\Omega_5^2 \right].
}{
ds^2_{IIA} = \rho^2e^{-3t/2}\mu^{-1}\sqrt{1-r^2}
        \left[ \left(e^t+e^{-3t}\frac{r^2}{1-r^2}\right)dr^2+e^tr^2d\Omega_5^2 \right].
}{ecuacion}\coordE{}\end{equation}
One can calculate the size of this space and show that 
\begin{equation}\coord{}\boxEquation{
\int_0^1dr\sqrt{g_{rr}} \to 0\hspace{5mm} (t\to +\infty).
}{
\int_0^1dr\sqrt{g_{rr}} \to 0\hspace{5mm} (t\to +\infty).
}{ecuacion}\coordE{}\end{equation}
Therefore, the internal space also becomes small as the scalar vevs grow. 
This result would be consistent with the expectation based on the RG argument which is meaningful since the string 
coupling is small. 
Therefore this would indicate that the analysis above is appropriate for investigations of tachyon 
condensations. 

Remember that the scalar potential (\ref{62}) is invariant under \myHighlight{$s\to-s$}\coordHE{}, so that the opposite limit 
\myHighlight{$t\to-\infty$}\coordHE{} should 
also be relevant. 
In this limit, the radius of the \myHighlight{$S^1$}\coordHE{} is large. 
Thus in the point of view of the worldsheet analysis, this limit corresponds to a strong coupling limit. 
On the other hand, the M-theory picture is still valid, since the size of the ball is also large. 
Therefore the \myHighlight{$t\to-\infty$}\coordHE{} limit would correspond to a decompactification limit of the M-theory. 

\vspace{3mm}

A similar phenomenon will occur in M-theory compactified on \myHighlight{$S^4$}\coordHE{}, although the relation to string theory would 
be unclear. 
The consistent truncation of the eleven-dimensional supergravity is a seven-dimensional supergravity \cite{7dim}. 
The explicit form of the scalar potential along \myHighlight{$SO(2)\times SO(3)$}\coordHE{}-invariant direction was obtained in 
\cite{7dimpotential}, and 
the internal metric was constructed \cite{embed7dim} in terms of the seven-dimensional fields. 




















\vspace{1cm}

\subsection{Type IIB string theory on \myHighlight{$S^5$}\coordHE{}}

\vspace{5mm}

Since the analysis in this case is parallel to that in the previous subsection, we will discuss briefly. 

The truncated theory is believed to be the five-dimensional gauged supergravity mentioned before, 
and critical points of its 
scalar potential is 
well-studied. 
In fact, some of them are shown to be unstable \cite{critical1}\cite{critical3}. 
Such an unstable vacuum would decay via a tachyon condensation, and as a result, the scalar vev would roll 
down the potential toward its ``bottom''. 

As before, we focus on the direction which is invariant under \myHighlight{$SO(2)\times SO(4)$}\coordHE{}. 
The explicit expression of the potential restricted to this direction parametrized be \myHighlight{$\lambda$}\coordHE{} is \cite{5dim1} 
\begin{equation}\coord{}\boxEquation{
{\cal P}(\lambda) = -\frac14(e^{2\lambda}+2e^{-\lambda}). 
}{
{\cal P}(\lambda) = -\frac14(e^{2\lambda}+2e^{-\lambda}). 
}{ecuacion}\coordE{}\end{equation}
Note that the critical point \myHighlight{$\lambda=0$}\coordHE{} is the maximally supersymmetric vacuum, so that it cannot decay into 
\myHighlight{$|\lambda|\to\infty$}\coordHE{}. 
However, some other unstable vacua might decay into a direction toward \myHighlight{$|\lambda|\to\infty$}\coordHE{}. 

The formula for the metric of the internal \myHighlight{$S^5$}\coordHE{} in terms of the five-dimensional fields was conjectured in 
\cite{critical3}, 
and in particular, along the \myHighlight{$SO(2)\times SO(4)$}\coordHE{}-invariant direction the \myHighlight{$S^5$}\coordHE{} is expected to be deformed 
to an ellipsoid
\begin{equation}\coord{}\boxEquation{
e^{2t}((x_1)^2+(x_2)^2)+e^{-t}((x_3)^2+\cdots+(x_6)^2)=\rho^2.
}{
e^{2t}((x_1)^2+(x_2)^2)+e^{-t}((x_3)^2+\cdots+(x_6)^2)=\rho^2.
}{ecuacion}\coordE{}\end{equation}
If deformations of the \myHighlight{$S^5$}\coordHE{} occur in the same way as that of the \myHighlight{$S^7$}\coordHE{}, then there are two limits; one is a 
Type IIB theory on a small space, and the other is a decompactification limit of the Type IIB theory. 
To verify this conjecture, it is necessary to obtain a nonlinear embedding of the gauged supergravity into the 
Type IIB supergravity. 





















\vspace{1cm}

\section{Discussion} \label{discussion}

\vspace{5mm}

We have investigated the condensations of the bulk closed string tachyons by using the spacetime effective 
theory. 
It would be the most suitable situation in which the effective theory is a gauged supergravity, and then the 
tachyons appear when one choose an unstable vacuum. 
The spacetime supersymmetry strongly restrict the dynamics, so that it should be expected that one could control 
part of quantum corrections, although one would like to analyze a non-supersymmetric background. 
One interesting, maybe important, indication of our results is that there might be a process of tachyon 
condensation whose endpoint is a string theory in the flat spacetime. 
It is desired to confirm that such a process does exist. 

\vspace{3mm}

We have considered only the cases in which the gauge group is compact. 
This is because one can relate the vevs of the scalars to the geometry of the internal manifold. 
There are many other gauged supergravities with non-compact gauge groups, and critical points of the scalar 
potentials were found \cite{explicitP}\cite{5dim1}. 
However, their geometric origins do not seem to be understood well until now. 
It is expected that relations between such gauged supergravities and higher dimensional theories would provide us 
more examples of bulk tachyon condensations, and our understanding of the condensations would become deeper. 
A proposed correspondence \cite{DW/QFT} will be relevant in this direction of research. 

\vspace{3mm}

Another generalization can be considered. 
Since we have focused on the maximally supersymmetric theories, the scalar potentials are completely determined 
by the supersymmetry. 
It is plausible since then we can obtain explicit formulae. 
But if we would like to have a more general potential to consider a more general situation, we have to reduce 
the number of supersymmetry. 
There is a nice characterization of a kind of \myHighlight{${\cal N}=2$}\coordHE{} gauged supergravities in four dimensions. 
They can be realized as the low energy effective theory of Type II string theory compactified on a manifold with 
the SU(3) structure (see e.g. \cite{SU(3)str} and references theirin). 
A six-dimensional manifold \myHighlight{$M$}\coordHE{} with the SU(3) structure admits a single SU(3) invariant spinor on \myHighlight{$M$}\coordHE{} which is 
not required to be covariantly constant. 
For this reason, the four-dimensional Minkowski spacetime 
is not, in general, a classical solution of the effective 
theory, that is, there is a non-trivial scalar potential. 
This realization of gauged supergravities will provide various examples of theories with unstable vacua, and 
one could extract a general feature of the decays of them. 
The investigation of a meaning of the SU(3) structure in terms of a two-dimensional theory is also interesting, 
and would be important to understand the mirror symmetry in the presence of flux. 

\vspace{3mm}

We have argued that the sigma model approach would not apply to whole process of the bulk tachyon condensations. 
However, it would be still useful to study properties of each vacuum. 
It is well-known that the notion of stability depends on what background the theory is defined \cite{BFbound}. 
To understand them in terms of the sigma model would be interesting. 

\vspace{3mm}

Our interest has been mainly on the bulk tachyon condensations. 
The usefulness of the spacetime action might, however, continue to hold for the localized tachyon condensations 
\cite{localized}. 
Localized tachyons usually come from twisted sectors, and thus it would be realized by just adding the 
corresponding matter fields in the effective action. 
Then it would be possible to do the similar analysis as we have done in this paper. 

\vspace{2cm}

{\bf {\Large Acknowledgements}}

\vspace{5mm}

We would like to thank T. Kimura and Y. Yasui for valuable discussions. 


















\newpage


\begin{thebibliography}{99}

\bibitem{closedSFT}
See, e.g. 
B. Zwiebach, 
{\it Closed String Field Theory: Quantum Action and the BV Master Equation}, 
Nucl.Phys. {\bf B390} (1993) 33, hep-th/9206084. 

\bibitem{BFSS}
T. Banks, W. Fischler, S.H. Shenker, L. Susskind, 
{\it M Theory As A Matrix Model: A Conjecture}, 
Phys.Rev. {\bf D55} (1997) 5112, hep-th/9610043. 

\bibitem{IKKT}
N. Ishibashi, H. Kawai, Y. Kitazawa, A. Tsuchiya, 
{\it A Large-N Reduced Model as Superstring}, 
Nucl.Phys. {\bf B498} (1997) 467, hep-th/9612115. 

\bibitem{DVV}
R. Dijkgraaf, E. Verlinde, H. Verlinde, 
{\it Matrix String Theory}, 
Nucl.Phys. {\bf B500} (1997) 43, hep-th/9703030. 

\bibitem{Zamolodchikov}
A.B. Zamolodchikov, 
{\it 'Irreversibility' of the flux of the renormalization group in a 2-D field theory}, 
JETP Lett. {\bf 43} (1986) 730 [Pisma Zh. Eksp. Teor. Fiz. {\bf 43} (1986) 565]. 

\bibitem{suyama}
T. Suyama, 
{\it Closed String Tachyons and RG Flows}, 
JHEP {\bf 0210} (2002) 051, hep-th/0210054. 

\bibitem{Tseytlin}
A.A. Tseytlin, 
{\it Conditions of Weyl Invariance of the Two-Dimensional Sigma Model from Equations of Stationarity of the 
``Central Charge" Action}, 
Phys. Lett. {\bf B194} (1987) 63. 

\bibitem{GHMS}
M. Gutperle, M. Headrick, S. Minwalla, V. Schomerus, 
{\it Spacetime Energy Decreases under World-sheet RG Flow}, 
hep-th/0211063. 

\bibitem{open}
A. Sen, 
{\it Non-BPS States and Branes in String Theory}, 
hep-th/9904207. 

\bibitem{localized}
A. Adams, J. Polchinski, E. Silverstein, 
{\it Don't Panic! Closed String Tachyons in ALE Spacetimes}, 
JHEP {\bf 0110} (2001) 029, hep-th/0108075. 

\bibitem{70s}
K. Bardakci, 
{\it Dual Models and Spontaneous Symmetry Breaking}, 
Nucl.Phys. {\bf B68} (1974) 331.

K. Bardakci, M.B. Halpern, 
{\it Explicit Spontaneous Breakdown in a Dual Model}, 
Phys.Rev. {\bf D10} (1974) 4230; 
{\it Explicit Spontaneous Breakdown in a Dual Model II: N Point Functions}, 
Nucl.Phys. {\bf B96} (1975) 285.

K. Bardakci, 
{\it Spontaneous Symmetry Breakdown in the Standard Dual String Model}, 
Nucl.Phys. {\bf B133} (1978) 297. 

\bibitem{HKMM}
J.A. Harvey, D. Kutasov, E.J. Martinec, G. Moore, 
{\it Localized Tachyons and RG Flows}, 
hep-th/0111154. 

\bibitem{hetero}
N.Kaloper, R.C. Myers, 
{\it The O(dd) Story of Massive Supergravity}, 
JHEP {\bf 05} (1999) 010, hep-th/9901045. 

\bibitem{5dim1}
M. G\"unaydin, L.J. Romans, N.P.Warner, 
{\it Gauged N=8 Supergravity in Five Dimensions}, 
Phys.Lett. {\bf B154} (1985) 286\ ;\ 
{\it Compact and Non-Compact Gauged Supergravity Theories in Five Dimensions}, 
Nucl.Phys. {\bf B272} (1986) 598.

\bibitem{5dim2}
M. Pernici, K. Pilch, P. van Nieuwenhuizen, 
{\it Gauged N=8 D=5 Supergravity}, 
Nucl.Phys. {\bf B259} (1985) 460.

\bibitem{AdS/CFT}
O. Aharony, S.S. Gubser, J. Maldacena, H. Ooguri, Y. Oz, 
{\it Large N Field Theories, String Theory and Gravity}, 
Phys.Rept. {\bf 323} (2000) 183-386, hep-th/9905111. 

\bibitem{critical1}
J. Distler, F. Zamora, 
{\it Non-Supersymmetric Conformal Field Theories from Stable Anti-de Sitter Spaces}, 
Adv.Theor.Math.Phys. {\bf 2} (1999) 1405, hep-th/9810206. 

\bibitem{critical2}
L. Girardello, M. Petrini, M. Porrati, A. Zaffaroni, 
{\it Novel Local CFT and Exact Results on Perturbations of N=4 Super Yang Mills from AdS Dynamics}, 
JHEP {\bf 9812} (1998) 022, hep-th/9810126. 

\bibitem{critical3}
A. Khavaev, K. Pilch, N.P. Warner, 
{\it New Vacua of Gauged N=8 Supergravity}, 
Phys.Lett. {\bf B487} (2000) 14, hep-th/9812035. 

\bibitem{GR}
S.W. Hawking, G.T. Horowitz, 
{\it The Gravitational Hamiltonian, Action, Entropy, and Surface Terms}, 
Class.Quant.Grav. {\bf 13} (1996) 1487, gr-qc/9501014. 

\bibitem{embedding}
B. de Wit, H. Nicolai, N.P. Warner, 
{\it The Embedding of Gauged N=8 Supergravity into d=11 Supergravity}, 
Nucl.Phys. {\bf B255} (1985) 29. 

\bibitem{CJ}
E. Cremmer, B. Julia, 
{\it The SO(8) Supergravity}, 
Nucl.Phys. {\bf B159} (1979) 141. 

\bibitem{truncation}
B. de Wit, H. Nicolai, 
{\it The Consistency of the \myHighlight{$S^7$}\coordHE{} Truncation in d=11 Supergraviity}, 
Nucl.Phys. {\bf B281} (1987) 211.

\bibitem{extrema}
N.P. Warner, 
{\it Some New Extrema of the Scalar Potential of Gauged N=8 Supergravity}, 
Phys.Lett. {\bf B128} (1983) 169\ ;\ 
{\it Some Properties of the Scalar Potential in Gauged Supergravity Theories}, 
Nucl.Phys. {\bf B231} (1984) 250.

\bibitem{Melvin}
M.S. Costa, M. Gutperle, 
{\it The Kaluza-Klein Melvin Solution in M-theory}, 
JHEP {\bf 0103} (2001) 027, hep-th/0012072. 

\bibitem{notation}
B. de Wit, H. Nicolai, 
{\it N=8 Supergravity}, 
Nucl.Phys. {\bf B208} (1982) 323. 

\bibitem{explicitP}
C.M. Hull, 
{\it The Minimal Couplings and scalar Potentials of the Gauged N=8 Supergravities}, 
Class.Quantum Grav. {\bf 2} (1985) 343. 

\bibitem{7dim}
M. Pernici, K. Pilch, P. van Nieuwenhuizen, 
{\it Gauged Maximally Extended Supergravity in Seven Dimensions}, 
Phys.Lett. {\bf B143} (1984) 103. 

\bibitem{7dimpotential}
M. Percini, K. Pilch, P. van Nieuwenhuizen, N.P. Warner, 
{\it Noncompact Gauging and Critical Points of Maximal Supergravity in Seven Dimensions}, 
Nucl.Phys. {\bf B249} (1985) 381. 

\bibitem{embed7dim}
H. Nastase, D. Vaman, P. van Nieuwenhuizen, 
{\it Consistent nonlinear KK reduction of 11d supergravity on \myHighlight{$AdS_7\times S^7$}\coordHE{} and self-duality in odd 
dimensions}, 
Phys.Lett. {\bf B469} (1999) 96, hep-th/9905075. 

\bibitem{DW/QFT}
H.J. Boonstra, K. Skenderis, P.K. Townsend, 
{\it The Domain Wall/QFT Correspondence}, 
JHEP {\bf 9901} (1999) 003, hep-th/9807137. 

\bibitem{SU(3)str}
S. Gurrieri, J. Louis, A. Micu, D. Waldram, 
{\it Mirror Symmetry in Generalized Calabi-Yau Compactifications}, 
hep-th/0211102. 

\bibitem{BFbound}
P. Breitenlohner, D. Freedman, 
{\it Positive Energy in Anti-de Sitter Backgrounds and Gauged Extended Supergravity}, 
Phys.Lett. {\bf B115} (1982) 197. 


\end{thebibliography}




\end{document}
\bye
