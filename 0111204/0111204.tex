\documentclass[12pt,a4paper]{article}
\usepackage{graphicx}
\usepackage{amsmath}
\usepackage{amsfonts}
\usepackage{amssymb}
\newtheorem{theorem}{Theorem}
\newtheorem{acknowledgement}[theorem]{Acknowledgement}
\newtheorem{algorithm}[theorem]{Algorithm}
\newtheorem{axiom}[theorem]{Axiom}
\newtheorem{case}[theorem]{Case}
\newtheorem{claim}[theorem]{Claim}
\newtheorem{conclusion}[theorem]{Conclusion}
\newtheorem{condition}[theorem]{Condition}
\newtheorem{conjecture}[theorem]{Conjecture}
\newtheorem{corollary}[theorem]{Corollary}
\newtheorem{criterion}[theorem]{Criterion}
\newtheorem{definition}[theorem]{Definition}
\newtheorem{example}[theorem]{Example}
\newtheorem{exercise}[theorem]{Exercise}
\newtheorem{lemma}[theorem]{Lemma}
\newtheorem{notation}[theorem]{Notation}
\newtheorem{problem}[theorem]{Problem}
\newtheorem{proposition}[theorem]{Proposition}
\newtheorem{remark}[theorem]{Remark}
\newtheorem{solution}[theorem]{Solution}
\newtheorem{summary}[theorem]{Summary}
\newenvironment{proof}[1][Proof]{\textbf{#1.} }{\ \rule{0.5em}{0.5em}}



\begin{document}


\title{On interaction in extended particle model on $\left(  M^{4}\times
M^{4}\right)  \otimes Z_{4}$}
\author{A. Smida, M. Hachemane, A.H. Hamici et R. Djelid.\\Facult\'{e} des Sciences (Physiques),\\USTHB, B.P. 32 El-Alia Bab-Ezzouar 16111,\\Algiers Algeria.}
\maketitle
\begin{abstract}
A $\left(  M^{4}\times M^{4}\right)  \otimes Z_{4}$ model, describing an
extended particle composed of two local modes and represented by a field
$\psi\left(  x,\xi;z\right)  $, is formulated in its most general form
($\left(  x,\xi;z\right)  \in\left(  M^{4}\times M^{4}\right)  \otimes Z_{4}%
$). The $z$ argument specifies whether the particle is observable,
unobservable, or partially observable (the latter case appears in two forms).
In this four-sheeted structure, each sheet posses its own symmetry localized
with respect to both space-times $M^{4}$ inducing thereby connections in the
continuous directions. Connections in the discrete direction describe
transitions between observable, unobservable, and partially observable states.
Curvatures and propagators are determined.
\end{abstract}



\section{Introduction}

The present work is a reconsideration of a previous one,\cite{Smida 2000}
aiming at a description of conversion of external modes into internal ones,
and vice-versa, in a geometro-differential conception of extended particles.

Initially,\cite{Smida 1995} this conception was based on a Hilbert bundle
structure $E^{D}(M,H^{D},U^{D}(G^{\prime}))$. The base space $M$ is ordinary
space-time which may have a $G$ symmetry or be curved. The typical fiber
$H^{D}$ is a Hilbert space carrying an induced representation\cite{Mensky
1976} $U^{D}\left(  G^{\prime}\right)  $ of the internal symmetry group
$G^{\prime}$. The particle extension stems from the fact that the latter is no
more represented by a point $x\in M$, but by a function $\Psi\in H^{D}$ which
depends on another spatio-temporal variable $\xi$ with a $G^{\prime}$
symmetry. The function $\Psi$ is not the probabilistic wave function but
describes the intrinsic properties of the particle (of which $x$ is a partial
representation).\cite{Destouches 1956} The probability amplitude role, played
by the quantum mechanics wave function $\psi\left(  x\right)  $, is guaranteed
by a functional\cite{Destouches 1956} $X[\Psi,t]$ in this conception. In a
realistic model, we assumed that $\Psi_{x}\left(  \xi\right)  $ describes a
quantum mode localized at $\xi$ for a particle localized at $x$ (from a
partial standpoint). The particle is composed then of two modes.\cite{Smida 1995}

The probabilistic functional has been chosen as a bilocal function
$X[\Psi](x,\xi)=\psi(x,\xi)$ representing an external quantum mode localized
at $x$ and an internal one localized at $\xi$. This function was quantized by
applying the induced representation method to both the external and internal
symmetries. When interaction is absent, the induced representation method
leads to a propagator which is a product of the propagators of each local
mode. The external mode propagation is determined by a transition from
external configuration induced representation to the external momentum one and
back to the external configuration representation. Internal mode propagation
is realized in the internal spaces. If an interaction is represented by a
gauge field (a connection in the fiber), the semigroup induced representations
lead to a path integral propagator.\cite{Mensky 1983, Smida 1998}

The assumption that the external mode may transit via internal momentum space
and the generalization of this idea to the possibility of transitions between
external and internal representations, called mode conversions, led to new
physical interpretations and ideas. However, mathematical expressions were
deduced by analogy with induced representation results leading thereby to some
inconsistencies.\cite{Smida 2000}

The purpose of the present work is to overcome the latter inconsistencies by
abandoning the fiber bundle structure in favor of an $\left(  M^{4}\times
M^{4}\right)  \otimes Z_{4}$, where $M^{4}$ is Minkowski space and $Z_{4}$ is
a discrete space with four elements. The inducing method is applied then
between symmetries of the same type only, and connection in the continuous
directions is taken into account. Transitions between symmetries of different
(external or internal) types is realized by means of connections in the
discrete direction.

The idea of discrete structure is drawn from works of Konisi\cite{Konisi 1996}
and Kubo\cite{Kubo 1998} who associated connections in the discrete directions
with Higgs fields without recourse to noncommutative geometry (NCG). For the
sake of definiteness, we mention that our work is neither to be compared with
that of Konisi and Kubo nor with those based on NCG. We have just used the
discrete structure to provide our idea of conversion with mathematical consistency.

In Sec. 2, the state spaces structure, the connections, and the physical
interpretations are presented. In Sec 3, curvatures are calculated following
Kubo's work.\cite{Kubo 1998} In Sec. 4, propagators containing both types of
connections are deduced and Sec. 5 is devoted to the conclusion.

\section{The structure and connections}

To describe an extended particle composed of two local modes let us consider a
$\left(  M^{4}\times M^{4}\right)  \otimes Z_{4}$ structure, where $M^{4}$ is
Minkowski space-time and $Z_{4}$ is the discret space with four elemnts.
States $\Psi^{z}$ of the particle belong to Hilbert spaces $H^{z}$ and are
considered as physical wave functions in the sens of providing all the
physical properties of the particle but not probabilties. The latter are
provided by functionals $X^{z}[\Psi^{z}]\left(  x,\xi\right)  =\psi\left(
x,\xi;z\right)  $.\cite{Destouches 1956} The variables $x$ and $\xi$ belong,
respectively, to the first and second space-time and the variable $z$ is an
element of $Z_{4}$ taking values $p$ for pure, $c$ for crossed, $e$ for
external, and $i$ for internal. Each case corresponds to a certain type of
localizability of the extended particle composed of a first mode localized at
$x$ and a second mode localized at $\xi$. In the pure case, the first mode is
localized in the external space-time and the second mode in the internal
space-time. The crossed case is the reverse of the former. The external and
internal cases mean that both modes are localized in external or internal
space-time, respectively. In other words, the localizability type $z$
attributes a fixed physical meaning to each space in the product $\left(
M^{4}\times M^{4}\right)  $ as being external or internal space-time. It
endows thereby the extended particle with the property of being completely
observable as a bilocal object in external space-time (external case),
partially observable as a local object in each space-time (pure and crossed
cases), or unobservable (internal case).

In the present work, we associate to each type of localizability $z$ a
symmetry group $G\left(  z\right)  $ with elements
\begin{equation}
U=\exp iT\left(  z\right)  .\theta\left(  x,\xi;z\right)
\end{equation}
We can assume that
\begin{equation}
T\left(  z\right)  .\theta\left(  x,\xi;z\right)  =T_{a}\left(  z\right)
\theta^{a}\left(  x,\xi;z\right)  +T_{\alpha}\left(  z\right)  \theta^{\alpha
}\left(  x,\xi;z\right)
\end{equation}
where $T_{a}$ and $\theta^{a}$ are, respectively, generators and parameters of
transformations related with the first space-time. In the same way,
$T_{\alpha}$ and $\theta^{\alpha}$ are generators and parameters of
transformations related with the second space-time.

These gauge transformations can be either spatio-temporal or unitary and
induce a connection corresponding to parallel transport in the continuous
directions, i.e. in one or both space-times. In fact, in the covariant
derivative\cite{Kubo 1998}
\begin{equation}
\nabla^{z}\psi\left(  x,\xi;z\right)  =\psi\left(  x+\delta x,\xi+\delta
\xi;z\right)  -\psi_{||}\left(  x+\delta x,\xi+\delta\xi;z\right)
\end{equation}
the parallel transported field $\psi_{||}\left(  x+\delta x,\xi+\delta
\xi;z\right)  $ from $\left(  x,\xi\right)  $ to $\left(  x+\delta
x,\xi+\delta\xi\right)  $ can be written
\begin{align}
\psi_{||}\left(  x+\delta x,\xi+\delta\xi;z\right)   & =H\left(  x+\delta
x,\xi+\delta\xi;z\right)  \psi\left(  x,\xi;z\right) \\
H\left(  x+\delta x,\xi+\delta\xi;z\right)   & =1-i\omega_{i}\left(
x,\xi;z\right)  \delta x^{i}-i\omega_{\mu}\left(  x,\xi;z\right)  \delta
\xi^{\mu}\label{HCI}%
\end{align}
The Lie algebra valued connection one-forms corresponding to gauge
transformations in each space-time are written in terms of gauge potentials
$A\left(  x,\xi;z\right)  $:
\begin{align}
\omega_{i}\left(  x,\xi;z\right)   & =T_{a}\left(  z\right)  A_{i}^{a}\left(
x,\xi;z\right) \\
\omega_{\mu}\left(  x,\xi;z\right)   & =T_{\alpha}\left(  z\right)  A_{\mu
}^{\alpha}\left(  x,\xi;z\right)
\end{align}
The covariant derivative takes then the following form
\begin{align}
\nabla^{z}  & =\delta x^{i}\nabla_{i}^{z}+\delta\xi^{\mu}\nabla_{\mu}^{z}\\
\nabla_{i}^{z}  & =\partial_{i}+i\omega_{i}\left(  x,\xi;z\right)
=\partial_{i}+iT_{a}\left(  z\right)  A_{i}^{a}\left(  x,\xi;z\right) \\
\nabla_{\mu}^{z}  & =\partial_{\mu}+i\omega_{\mu}\left(  x,\xi;z\right)
=\partial_{\mu}+iT_{\alpha}\left(  z\right)  A_{\mu}^{\alpha}\left(
x,\xi;z\right)
\end{align}
In the same manner, a covariant difference can be defined in the discrete
direction. Parallel transport is a transition from one type of localizability
(say $z^{\prime}$) to another ($z$)
\begin{equation}
\psi_{||}\left(  x,\xi;z\right)  =H\left(  x,\xi;z,z^{\prime}\right)
\psi\left(  x,\xi;z^{\prime}\right)
\end{equation}
Covariant difference is then written as follows
\begin{align}
\delta_{z^{\prime}}\psi\left(  x,\xi;z\right)   & =\psi\left(  x,\xi;z\right)
-\psi_{||}\left(  x,\xi;z\right) \\
& =\psi\left(  x,\xi;z\right)  -H\left(  x,\xi;z,z^{\prime}\right)
\psi\left(  x,\xi;z^{\prime}\right) \nonumber
\end{align}
but the $H\left(  x,\xi;z,z^{\prime}\right)  $ field has not a conventional
expression in terms of one forms and gauge potentials. It can be interpreted
as a transition operator from one type of localizability to another. It
corresponds then to a conversion of internal modes to external ones and
vice-versa. This conversion can be viewed as a creation of the particle when
it passes from an unobservable state to a partially or completely observable
one. It is viewed as an annihilation in the inverse transitions. It is then
natural to define the conjugate of such a conversion by the following
relation
\begin{equation}
H^{\dagger}\left(  x,\xi;z,z^{\prime}\right)  =H\left(  x,\xi;z^{\prime
},z\right)
\end{equation}

\section{Curvatures}

Now, we define and calculate different types of curvatures stemming from the
structure considered in this work.

The first type of curvature corresponds to parallel transport along closed
paths in the continuous direction and is given by the well known strength
field $F_{AB}\left(  x,\xi;z\right)  =-i[\nabla_{A}^{z},\nabla_{B}^{z}]$
components where the indices $A$ and $B$ take the values $i$ or $\mu$
\begin{equation}
F_{AB}\left(  x,\xi;z\right)  =\partial_{\lbrack A}\omega_{B]}\left(
x,\xi;z\right)  +i[\omega_{A}(x,\xi;z),\omega_{B}(x,\xi;z)]
\end{equation}
Subscript brackets $[,]$ indicates an antisymmetrization over the indices and
ordinary ones are commutator of connection forms. If parallel transport takes
place in the first space-time, the curvature takes the following form
\begin{equation}
F_{ij}\left(  x,\xi;z\right)  =\partial_{\lbrack i}\omega_{j]}\left(
x,\xi;z\right)  +i[\omega_{i}(x,\xi;z),\omega_{j}(x,\xi;z)]
\end{equation}
For a path in the second space-time, the curvature is
\begin{equation}
F_{\mu\nu}\left(  x,\xi;z\right)  =\partial_{\lbrack\mu}\omega_{\nu]}\left(
x,\xi;z\right)  +i[\omega_{\mu}(x,\xi;z),\omega_{\nu}(x,\xi;z)]
\end{equation}
and a closed path lying in the two spaces corresponds to the following
curvature
\begin{equation}
F_{i,\mu}\left(  x,\xi;z\right)  =\partial_{\lbrack i}\omega_{\mu]}\left(
x,\xi;z\right)  +i[\omega_{i}(x,\xi;z),\omega_{\mu}(x,\xi;z)]
\end{equation}
If the symmetry groups of each space-time commute, the latter curvature
becomes
\begin{equation}
F_{i,\mu}\left(  x,\xi;z\right)  =\partial_{\lbrack i}\omega_{\mu]}\left(
x,\xi;z\right)
\end{equation}
and, if in addition each connection depends only on its corresponding
space-time variable, this curvature vanishes identically.

The second type of curvature is concerned with a combination of a parallel
transport in the continuous direction with a parallel transport in the
discrete direction, Fig.(1).%

\begin{picture} (200,140)
\put(40,110) {\small{$(C_2)$}}
\put(40,25) {\small{$(C_1)$}}
\put(105,70) {\small{Fig. (1)}}
\put(-10,10) {\small{$(x,\xi;z)$}}
\put(-10,125) {\small{$(x,\xi;z')$}}
\put(80,125) {\small{$(x + \delta x,\xi+ \delta\xi;z')$}}
\put(80,10) {\small{$(x + \delta x,\xi+ \delta\xi;z)$}}
\put(0,20) {\vector(1,0){50}}
\put(0,20) {\vector(0,1){50}}
\put(0,120) {\vector(1,0){50}}
\put(100,20) {\vector(0,1){50}}
\put(50,20) {\line(1,0){50}}
\put(0,70) {\line(0,1){50}}
\put(50,120) {\line(1,0){50}}
\put(100,70) {\line(0,1){50}}
\end{picture}
\begin{picture} (150,140)
\put(15,10) {\small{$(x,\xi;z)$}}
\put(15,125) {\small{$(x,\xi;z')$}}
\put(35,70) {\small{Fig. (2)}}
\put(24,20) {\vector(0,1){50}}
\put(26,120) {\vector(0,-1){50}}
\put(0,20) {\line(1,0){50}}
\put(24,70) {\line(0,1){50}}
\put(0,120) {\line(1,0){50}}
\put(26,20) {\line(0,1){50}}
\end{picture}

The curvature is defined as a difference between two paths $C_{1}$ and $C_{2}
$ where
\begin{align}
C_{1}  & =H\left(  x+\delta x,\xi+\delta\xi;z^{\prime},z\right)  H\left(
x+\delta x,\xi+\delta\xi;z\right)  \psi\left(  x,\xi;z\right) \\
C_{2}  & =H\left(  x+\delta x,\xi+\delta\xi;z^{\prime}\right)  H\left(
x,\xi;z^{\prime},z\right)  \psi\left(  x,\xi;z\right)
\end{align}
We have
\begin{equation}
C_{1}-C_{2}=\{\delta x^{i}F_{iz^{\prime}}^{H}+\delta\xi^{\mu}F_{\mu z^{\prime
}}^{H}\}\psi\left(  x,\xi;z\right)
\end{equation}
where
\begin{align}
F_{iz^{\prime}}^{H}\left(  x,\xi;z\right)   & =\partial_{i}H\left(
x,\xi;z^{\prime},z\right) \\
& -iH\left(  x,\xi;z^{\prime},z\right)  \omega_{i}(x,\xi;z)+i\omega_{i}%
(x,\xi;z^{\prime})H\left(  x,\xi;z^{\prime},z\right) \nonumber\\
F_{\mu z^{\prime}}^{H}\left(  x,\xi;z\right)   & =\partial_{\mu}H\left(
x,\xi;z^{\prime},z\right) \\
& -iH\left(  x,\xi;z^{\prime},z\right)  \omega_{\mu}(x,\xi;z)+i\omega_{\mu
}(x,\xi;z^{\prime})H\left(  x,\xi;z^{\prime},z\right) \nonumber
\end{align}
It is clear that if parallel transport in the continuous direction concerns
one space-time, only the corresponding curvature has to be considered.

Parallel transport of the third type curvature links two points in the
discrete direction only, Fig. (2). Then
\begin{equation}
F_{z^{\prime}(z)}\left(  x,\xi;z\right)  =1-H\left(  x,\xi;z,z^{\prime
}\right)  H\left(  x,\xi;z^{\prime},z\right) \label{f2z}%
\end{equation}
For purely discrete curvatures, we adopt the following notation. The initial
point $z$ in the diagram is considered as an argument, the intermediate point
$z^{\prime}$ as an index, and the end point $z$ as an index between parenthesis.

There are also parallel transports linking three and four points in the
discrete directions depicted in Figs. (3) and (4), respectively.%

\begin{picture} (150,140)
\put(-10,10) {\small{$(x,\xi;z)$}}
\put(35,125) {\small{$(x,\xi;z')$}}
\put(85,10) {\small{$(x,\xi;z'')$}}
\put(80,70) {\small{Fig. (3)}}
\put(25,70) {\line(1,2) {25}}
\put(75,70) {\line(1,-2) {25}}
\put(100,20) {\line(-1,0) {50}}
\put(0,20) {\vector(1,2) {25}}
\put(50,120) {\vector(1,-2) {25}}
\put(0,20) {\vector(1,0) {50}}
\end{picture}%
\begin{picture} (150,140)
\put(-10,10) {\small{$(x,\xi;z)$}}
\put(-10,125) {\small{$(x,\xi;z'')$}}
\put(80,125) {\small{$(x,\xi;z''')$}}
\put(80,10) {\small{$(x,\xi;z')$}}
\put(105,70) {\small{Fig. (4)}}
\put(0,20) {\vector(1,0){50}}
\put(0,20) {\vector(0,1){50}}
\put(0,120) {\vector(1,0){50}}
\put(100,20) {\vector(0,1){50}}
\put(50,20) {\line(1,0){50}}
\put(0,70) {\line(0,1){50}}
\put(50,120) {\line(1,0){50}}
\put(100,70) {\line(0,1){50}}
\end{picture}

They give the curvature of the third type which has the following form
\begin{equation}
F_{z^{\prime}(z^{\prime\prime})}\left(  x,\xi;z\right)  =H\left(
x,\xi;z^{\prime\prime},z\right)  -H\left(  x,\xi;z^{\prime\prime},z^{\prime
}\right)  H\left(  x,\xi;z^{\prime},z\right) \label{f3z}%
\end{equation}
and the curvature of the fourth type which has an analogous expression
\begin{equation}
F_{z^{\prime}z^{\prime\prime}(z^{\prime\prime\prime})}\left(  x,\xi;z\right)
=H\left(  x,\xi;z^{\prime\prime\prime},z^{\prime\prime}\right)  H\left(
x,\xi;z^{\prime\prime},z\right)  -H\left(  x,\xi;z^{\prime\prime\prime
},z^{\prime}\right)  H\left(  x,\xi;z^{\prime},z\right) \label{f4z}%
\end{equation}
Note that curvature (\ref{f2z}) is compatible with (\ref{f3z}) since $H\left(
x,\xi;z,z\right)  =1$ and that (\ref{f4z}) can be derived from (\ref{f3z})
\begin{equation}
F_{z^{\prime}z^{\prime\prime}(z^{\prime\prime\prime})}\left(  x,\xi;z\right)
=F_{z^{\prime\prime}(z^{\prime\prime\prime})}(z)-F_{z^{\prime}(z^{\prime
\prime\prime})}(z)
\end{equation}
The latter relation shows antisymmetry with respect to $z^{\prime}$ and
$z^{\prime\prime}$. Moreover, it is easy to show that
\begin{equation}
F_{\bullet(z^{\prime})}\left(  x,\xi;z\right)  =F_{\bullet(z)}^{\dagger
}\left(  x,\xi;z^{\prime}\right)
\end{equation}
where the dot ($\bullet$) is to be replaced by the adequate arguments of the
purely discrete curvatures. Consequently, curvature (\ref{f2z}) is hermitic.

All our curvatures are analogous to those of local theories\cite{Konisi
1996,Kubo 1998} except the continuous curvature which contains extra terms du
to extension.

\section{Propagators}

In order to determine propagators of extended particles described by the
$\left(  M^{4}\times M^{4}\right)  \otimes Z_{4}$ structure, we first consider
the case of one sheet with fixed $z$. The propagator has already been deduced
for this case (which is equivalent to $\left(  M^{4}\times M^{4}\right)  $) by
means of trajectory semigroups induced representations in each
space-time.\cite{Smida 1998} Use of trajectory semigroups is imposed as far as
the continuous connection is to be taken into account in the inducing method
of quantization.

A trajectory in $M^{4}$ is a class of parallel curves $x(l)$. It is
represented by an element
\begin{equation}
\lbrack u]_{t}=\{u(l)\in\mathbf{R}^{4}/\,0\leq l\leq
t\},\;\;\;\;\;\;t=(s-s^{\prime}),\;\;\;\;\;\;u(l)=(\frac{dx}{dl})_{s^{\prime
}+l}%
\end{equation}
where $x^{\prime}=x(s^{\prime})$ and $x=x(s)$ are the initial and final points
of the curve $x\left(  l\right)  $, respectively. Right action of trajectories
on space-time points is defined in the following way
\begin{equation}
x[u]_{t}=x^{\prime}=x-\int_{0}^{t}u(l)\,dl\label{ex}%
\end{equation}
Trajectories $[\gamma]_{\tau}$ are also defined for curves $\xi\left(
\lambda\right)  $ of the second Minkowski space-time in $\left(  M^{4}\times
M^{4}\right)  \otimes Z_{4}$. To each pair of trajectories is associated a
translation operator $U([u]_{t},[\gamma]_{\tau})$
\begin{equation}
\lbrack U([u]_{t},[\gamma]_{\tau})\psi](x,\xi;z)=\psi(x[u]_{t},\xi
\lbrack\gamma]_{\tau};z)
\end{equation}
and a parallel transport operator $H_{([u]_{t},[\gamma]_{\tau};z)}$, acting
jointly with $U$ and taking account of the continuous gauge fields effect
\begin{equation}
\lbrack H_{([u]_{t},[\gamma]_{\tau};z)}U([u]_{t},[\gamma]_{\tau})\psi
](x,\xi;z)=H_{([u]_{t},[\gamma]_{\tau})}(x,\xi;z)\psi(x[u]_{t},\xi
\lbrack\gamma]_{\tau};z)
\end{equation}
The connection $H_{([u]_{t},[\gamma]_{\tau})}(x,\xi;z)$ corresponds to a
parallel transport from $(x[u]_{t},\xi\lbrack\gamma]_{\tau};z)$ to $(x,\xi;z)$
along two curves belonging to trajectories $[u]_{t}$ and $[\gamma]_{\tau}$.
Its infinitesimal form is given by relation (\ref{HCI}) and its finite form is
an ordered path integral over both curves
\begin{equation}
H(x[u]_{t},\xi\lbrack\gamma]_{\tau};z)=P\left[  \exp-\int_{(x[u]_{t}%
,\xi\lbrack\gamma]_{\tau};z)}^{(x,\xi;z)}\left(  i\omega_{i}\left(
x,\xi;z\right)  dx^{i}+i\omega_{\mu}\left(  x,\xi;z\right)  d\xi^{\mu}\right)
\right]
\end{equation}
In our previous works, $i\omega_{i}\left(  x,\xi;z\right)  dx^{i}$ has been
denoted $\Gamma_{i}\left(  x\right)  dx^{i}$ whereas $i\omega_{\mu}\left(
x,\xi;z\right)  d\xi^{\mu}$ was not considered since we used a gauging of the
internal symmetry with respect to the external space-time only. The situation
is quite different now, we have a symmetry of two space-times gauged with
respect to both.

The one sheet propagation operator of functions $\psi\left(  x,\xi,z\right)  $
is a path integral expression\cite{Smida 1998} ($\theta$ is the step
function)
\begin{align}
\Pi_{z}^{c}  & =\int dt\,\theta(t)\,\exp(-im_{z}^{2}t)\int d\tau\,\theta
(\tau)\,\exp(-im_{z}^{\prime2}\tau)\int d[u]_{t}\int d[\gamma]_{\tau}\\
& \ \ \ \exp(\frac{-i}{4}\int_{0}^{t}dl\,u^{2}(l))\,\,\exp(\frac{-i}{4}%
\int_{0}^{\tau}d\lambda\,\gamma^{2}(\lambda))H_{([u]_{t},[\gamma]_{\tau}%
;z)}\,U([u]_{t},[\gamma]_{\tau})\nonumber
\end{align}
with measures
\begin{equation}
d[u]_{t}=\prod_{l=0}^{t}du(l)\,\,\,\,\,\,\,\,\,\,\,\,\,\,\,\,\,\,\,d[\gamma
]_{\tau}=\prod_{\lambda=0}^{\tau}d\gamma(\lambda)
\end{equation}
Masses $\left(  m_{z},m_{z}^{\prime}\right)  $ correspond to the first and
second modes, respectively. The subscript $z$ is a reminder that momentum
representations may be different for different sheets. So for masses and spins
since in applying the method of induced representations for ordinary symmetry
groups, irreducible momentum representations labeled by the mass $m$ and the
spin $j$ of each mode must be used. Propagation is determined by a transition
from configuration representation to momentum representation (intertwining)
and then to configuration representation. The propagator has then to be
labeled by the masses $\left(  m_{z},m_{z}^{\prime}\right)  $ and spins
$\left(  j_{z},j_{z}^{\prime}\right)  $, of the first and second mode
respectively, corresponding to the sheet where intertwining is realized. In
contrast, trajectory (semigroup) momentum representation does not specify
masses which are introduced through integration over $\left(  s,\lambda
\right)  $,\cite{Smida 1998} but they should naturally be labeled by $z$.

For a pure sheet $z=p$ and $\left(  m_{p},m_{p}^{\prime}\right)  =\left(
m,\mu\right)  $, the first mode is external with mass $m$ and the second mode
is internal with mass $\mu$. For $z=c$ and $\left(  m_{c},m_{c}^{\prime
}\right)  =\left(  \mu,m\right)  $, the first mode is internal while the
second is external. For the remaining cases $z=e$ and $z=i$, the respective
masses are $\left(  m,m\right)  $ and $\left(  \mu,\mu\right)  $.

Now, we come to the implementation of our idea of conversion and proceed by
comparison with the case where the continuous connection is ignored, that is
when groups are used instead of semigroups. A general propagation amounts to a
transition from $\psi\left(  x^{\prime},\xi^{\prime};z^{\prime}\right)  $ to
$\psi\left(  x,\xi;z\right)  $ through momentum representation in a
$z^{\prime\prime}$-sheet. \cite{Smida 2001} The function $\psi\left(
x^{\prime},\xi^{\prime};z^{\prime}\right)  $ is transformed by means of
$H\left(  x^{\prime},\xi^{\prime};z^{\prime\prime},z^{\prime}\right)  $ and
propagation is realized from $\left(  x^{\prime},\xi^{\prime};z^{\prime\prime
}\right)  $ to $\left(  x,\xi;z^{\prime\prime}\right)  $ in the $z^{\prime
\prime}$-sheet. Then the result is transformed by $H\left(  x,\xi
;z,z^{\prime\prime}\right)  $. When the continuous connection is considered
(trajectory case), the two $H$ fields representing the discrete connection
must be included in the propagator and masses have to be labeled with the
intertwining sheet parameter $z^{\prime\prime}$.

Hence, the most general expression of the propagator operator is
\begin{align}
\Pi_{z^{\prime\prime}}^{c}\left(  z,z^{\prime}\right)   & =\int dt\,\theta
(t)\,\exp(-im_{z^{\prime\prime}}^{2}t)\int d\tau\,\theta(\tau)\,\exp
(-im_{z^{\prime\prime}}^{\prime2}\tau)\\
& \int d[u]_{t}\int d[\gamma]_{\tau}\exp(\frac{-i}{4}\int_{0}^{t}%
dl\,u^{2}(l))\,\,\exp(\frac{-i}{4}\int_{0}^{\tau}d\lambda\,\gamma^{2}%
(\lambda))\nonumber\\
& H_{([u]_{t},[\gamma]_{\tau};z^{\prime\prime})}\left(  z,z^{\prime}\right)
\,U([u]_{t},[\gamma]_{\tau})\nonumber
\end{align}
The new operators $H_{([u]_{t},[\gamma]_{\tau};z^{\prime\prime})}\left(
z,z^{\prime}\right)  $ defined by
\begin{align}
& \left[  H_{([u]_{t},[\gamma]_{\tau};z^{\prime\prime})}\left(  z,z^{\prime
}\right)  U([u]_{t},[\gamma]_{\tau}\psi\right]  (x,\xi;z^{\prime
})\,\nonumber\\
& =H\left(  x,\xi;z,z^{\prime\prime}\right)  H_{([u]_{t},[\gamma]_{\tau}%
)}(x,\xi;z^{\prime\prime})\times\\
& H(x[u]_{t},\xi\lbrack\gamma]_{\tau};z^{\prime\prime},z^{\prime}%
)\psi(x[u]_{t},\xi\lbrack\gamma]_{\tau};z^{\prime})\nonumber
\end{align}
contain continuous and discrete connections. We note that the general
propagation operator is compatible with the one sheet propagation operator
since $H(x,\xi;z,z)=1$. In total, we have sixty-four propagations differing by
the values of $\left(  z,z^{\prime};z^{\prime\prime}\right)  $. Propagation of
fields can be written in the following way
\begin{align}
\psi(x,\xi;z)  & =[\Pi_{z^{\prime\prime}}^{c}\left(  z,z^{\prime}\right)
\psi](x,\xi;z^{\prime})\label{PGa}\\
& =\int dx^{\prime}d\xi^{\prime}\Pi_{z^{\prime\prime}}^{c}\left(
x,\xi,z;x^{\prime},\xi^{\prime},z^{\prime}\right)  \psi(x^{\prime},\xi
^{\prime};z^{\prime})\label{PGb}%
\end{align}
The kernel $\Pi_{z^{\prime\prime}}^{c}\left(  x,\xi,z;x^{\prime},\xi^{\prime
},z^{\prime}\right)  $ in relation (\ref{PGb}) is to be determined after
calculation of (\ref{PGa}). Such a kernel is interpreted as a spatio-temporal
evolution of two modes from $\left(  x^{\prime},\xi^{\prime}\right)  $ to
$\left(  x,\xi\right)  $, which may be accompanied by conversions (if $z$ is
different from $z^{\prime}$ or $z^{\prime\prime}$).

\section{Conclusion}

The $\left(  M^{4}\times M^{4}\right)  \otimes Z_{n}$ structure seemed
interesting in interpreting the geometrical origin of Higgs fields without
recourse to noncommutative geometry.\cite{Konisi 1996,Kubo 1998}

The present work reveals another aspect of this structure, which is not
concerned with the Higgs phenomenon. It opens the way to the construction of a
theory of extended particles interacting with gauge fields and reaches the
determination of a path integral form of the propagators. This work explains
in a simple manner the creation of particles by admitting their prior
existence in unobservable states and the possibility of their transition to
observable ones. The $\left(  M^{4}\times M^{4}\right)  \otimes Z_{4}$
interest is that it provides the mathematical objects representing these
transitions, namely the discrete connections $H\left(  x,\xi;z,z^{\prime
}\right)  $. Moreover, the theory allows consideration of gauge fields
corresponding to a symmetry localized not only with respect to one space-time
(generally, external space-time) but in both space-times.

Symmetries, connections, curvatures, and propagators have been presented in
their most general form. Propagators incorporate conversions and effects of
gauge fields in the continuous directions. Hence, the particle evolves in
space-time and this evolution may be accompanied by its annihilation, its
creation, or a transformation of those of its properties which are coupled to
the continuous gauge field.

The following step is the adoption of specific physical models and the
derivation of equation of motion for $\psi\left(  x,\xi;z\right)  $, $A\left(
x,\xi;z\right)  $, and $H\left(  x,\xi;z,z^{\prime}\right)  $. The study of
this question has already been initiated by the determination of curvatures
which may lead to a Lagrangian formulation. However, this is to be carefully
analyzed since the validity of a bilocal Lagrangian theory is not well established.

\textbf{Acknowledgment :} A. S., M. H., and A.-H. H. are supported by research
Project N${{}^{\circ}}$ : D 1602/07/2000.

\begin{thebibliography}{99}
\bibitem{Smida 2000}A. Smida, M. Hachemane, R. Djelid and A.-H. Hamici, Found.
Phys. \textbf{30}, 287 (2000).

\bibitem {Smida 1995}A. Smida, M. Hachemane and M. Fellah, Found. Phys.
\textbf{25}, 1769 (1995).

\bibitem {Mensky 1976}M.B. Mensky,\textit{\ The Method of Induced
Representations: Space-Time and the Concept of Particles} [in Russian] (Nauka,
Moscow, 1976).

\bibitem {Destouches 1956}J.L. Destouches, \textit{La Quantification en
Th\'{e}orie Fonctionnelle des Corpuscules} (Gauthier- Villars, Paris, 1956).

\bibitem {Mensky 1983}M.B. Mensky, Teor. Mat. Fiz. \textbf{57}, 217 (1983).

\bibitem {Smida 1998}A. Smida, M. Hachemane and A.-H. Hamici, Found. Phys.
\textbf{28}, 1367 (1998).

\bibitem {Konisi 1996}G. Konisi and T. Saito, Prog. Theor. Phys. \textbf{95},
657 (1996).

\bibitem {Kubo 1998}M. Kubo \textit{et al}, Prog. Theor. Phys. \textbf{100}
(1998) No. 1., \textit{hep-th/9804161}.
\end{thebibliography}
\end{document}

