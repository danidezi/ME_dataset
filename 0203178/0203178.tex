\documentclass[a4paper,12pt]{article}
\usepackage{graphicx}
\usepackage{latexsym}
\usepackage{amsfonts}
%\linespread{1.6}
\begin{document}
\title{\textbf{EFFECTIVE ``GLUON'' DYNAMICS IN A STOCHASTIC VACUUM}\thanks{to be submitted for publication}}
\author{Jose A. Magpantay \thanks{email: jamag@nip.upd.edu.ph}\\National Institute of Physics\\University of the Philippines\\Diliman Quezon City, 1101\\Philippines}
\maketitle
\begin{abstract}
Using the new scalar and vector degrees of freedom derived from the non-linear gauge condition $(\partial\cdot D)(\partial\cdot A)=0$, we show that the effective dynamics of the vector fields (identified as ``gluons'') in the stochastic vacuum defined by the scalars result in the vector fields acquiring a mass.  We also find the vector fields losing their self-interactions.
\end{abstract}
\clearpage
\section{Introduction}

The non-linear gauge condition	
\begin{equation}\label{1}
(\partial\cdot D)^{ab}(\partial\cdot A^{b})=(D\cdot\partial)^{ab}(\partial\cdot A^{b})=\frac{1}{2}[(D\cdot\partial)+(\partial\cdot D)]^{ab}(\partial\cdot A^{b})=0,
\end{equation}
was proposed by the author\cite{gauge} as a generalization to the Coulomb gauge because the non-linear sector of (1), i.e., those configurations that satisfy $\partial\cdot A^{a}=f^{a}(x)\neq 0$, cannot be gauge transformed to the Coulomb gauge\cite{grav}. Subsequent papers established that:\\

(a) The gauge condition interpolates between the Coulomb gauge (with transverse fields in the short-distance regime) and the quadratic condition (with the new scalar and vector fields) in the large-distance regime\cite{low}.\\

(b) The gauge condition can no longer be extended to that with higher orders of the Fadeev-Popov operator, i.e., $(\partial\cdot D)^{n}(\partial\cdot A)=0$ for $n=2$,... is not consistent\cite{low}.\\  

(c) The effective action of the scalar degrees of freedom is infinitely non-linear and a stochastic treatment of a class of classical configurations, those that are spherically symmetric in 4D Euclidean space-time, leads to the linear potential and the area law behaviour of the Wilson loop\cite{mass}.\\

(d) The full quantum treatment of the scalar degree of freedom shows that its dynamics is equivalent to that of a 2D O(1,3) non-linear sigma model\cite{free}.\\

In this paper, the dynamics of the vector field, which we will identify as the ``gluons'', in the stochastic background defined by the spherically symmetric scalar will be considered.  We will show that the ``gluons'' will acquire a mass and become non self-interacting.\\

This paper is then arranged as follows:  In Section 2, we give the basic equations that govern the new degrees of freedom derived from the non-linear gauge.  Section 3 focuses on the effective dynamics of the ``gluons'' and derives an expression for the mass.  Section 4 concludes with a comparison of the ideas presented here and those found in the literature.

\section{The Dynamics of the New Degrees of Freedom}

Consider $SU(2)$ Yang-Mills in 4D Euclidean space-time.  Field configurations that satisfy equation (1) with $\partial\cdot A^{a}=\frac{1}{g\ell^{2}}f^{a}$ ($\ell$ introduced for dimensional reason) can be decomposed as:
\begin{equation}\label{2}
A^{a}_{\mu}(x)=\frac{1}{(1+\vec{f}\cdot\vec{f})}(\delta^{ab}+\epsilon^{abc}f^{c}+f^{a}f^{b})(\frac{1}{g}\partial_{\mu}f^{b}+t^{b}_{\mu}),
\end{equation}
where $f^{a}$ and $t^{a}_{\mu}$ are the new scalar (dimensionless) and vector (dimension 1) degrees of freedom.  These new degrees of freedom satisfy
\begin{eqnarray}\label{3}
\partial_{\mu}t^{a}_{\mu}&=&\frac{1}{g\ell^{2}}f^{a},\\
\rho^{a}&=&\frac{1}{(1+\vec{f}\cdot\vec{f})^{2}}[\epsilon^{abc}+\epsilon^{abd}f^{d}f^{c}-\epsilon^{acd}f^{d}f^{b}+f^{a}f^{d}\epsilon^{dbc}\nonumber\\
&-&f^{a}(1+\vec{f}\cdot\vec{f})\delta^{bc}-f^{c}(1+\vec{f}\cdot\vec{f})\delta^{ab}]\partial_{\mu}f^{b}t^{c}_{\mu}=0,
\end{eqnarray}
which make the number of degrees of freedom tally with those of $A^{a}_{\mu}$.\\

Substituting equation (2) in $F^{a}_{\mu\nu}$, we find that
\begin{equation}\label{4}
F^{a}_{\mu\nu}=\frac{1}{g}Z^{a}_{\mu\nu}(f)+L^{a}_{\mu\nu}(f;t)+g Q^{a}_{\mu\nu}(f;t),
\end{equation}
where Z, L and Q are all functions of f and zeroth order, linear and quadratic in $t^{a}_{\mu}$, respectively.  Explicitly, the functions are:
\begin{eqnarray}\label{5}
Z^{a}_{\mu\nu}&=&X^{abc}\partial_{\mu}f^{b}\partial_{\nu}f^{c},\\
L^{a}_{\mu\nu}&=&R^{ab}(f)(\partial_{\mu}t^{b}_{\nu}-\partial_{\nu}t^{b}_{\mu})+Y^{abc}(\partial_{\mu}f^{b}t^{c}_{\nu}-\partial_{\nu}f^{b}t^{c}_{\mu}),\\
Q^{a}_{\mu\nu}&=&T^{abc}(f)t^{b}_{\mu}t^{c}_{\nu},
\end{eqnarray}
and the coefficient functions are
\begin{eqnarray}\label{6}
X^{abc}&=&\frac{1}{(1+\vec{f}\cdot\vec{f})^{2}}[-(1+2\vec{f}\cdot\vec{f})\epsilon^{abc}+2\delta^{ab}f^{c}-2\delta^{ac}f^{b}\nonumber\\
&+&3\epsilon^{abd}f^{d}f^{c}-3\epsilon^{acd}f^{d}f^{b}+\epsilon^{bcd}f^{a}f^{d}],\\
R^{ab}(f)&=&\frac{1}{(1+\vec{f}\cdot\vec{f})}(\delta^{ab}+\epsilon^{abc}f^{c}+f^{a}f^{b}),\\
Y^{abc}&=&\frac{1}{(1+\vec{f}\cdot\vec{f})^{2}}[-(\vec{f}\cdot\vec{f})\epsilon^{abc}+(1+\vec{f}\cdot\vec{f})f^{a}\delta^{bc}-(1-\vec{f}\cdot\vec{f})\delta^{ac}f^{b}\nonumber\\
&+&3\epsilon^{cad}f^{d}f^{b}-2f^{a}f^{b}f^{c}+\epsilon^{abd}f^{d}f^{c}+f^{a}\epsilon^{bcd}f^{d}],\\
T^{abc}&=&\frac{1}{(1+\vec{f}\cdot\vec{f})^{2}}[\epsilon^{abc}+(1+\vec{f}\cdot\vec{f})f^{b}\delta^{ac}-(1+\vec{f}\cdot\vec{f})f^{c}\delta^{ab}\nonumber\\
&+&\epsilon^{abd}f^{d}f^{c}+f^{a}\epsilon^{bcd}f^{d}+\epsilon^{adc}f^{d}f^{b}].
\end{eqnarray}
Note that in the Coulomb gauge limit $(f^{a}=0)$, $t^{a}_{\mu}=A^{a}_{\mu}$ and equations (2) to (12) consistently reduce to the Coulomb gauge results.\\

The pure f dynamics suggests non-perturbative physics because the kinetic term is $(\partial f)^{4}$ and its action is $\sim\frac{1}{g^{2}}$ and infinitely non-linear.  This intuition was verified in reference (4) where it was shown that a stochastic treatment of a class of classical configurations, i.e., those that are spherically symmetric in 4D Euclidean space-time leads to the linear potential and the area law behaviour of the Wilson loop.  Full quantum treatment of the scalar fields, on the other hand, leads to dimensional reduction as the scalar dynamics was shown to be equivalent to a 2D O(1,3) non-linear sigma model via the Parisi-Sourlas mechanism.\\

Before we proceed in the next section with the discussion on the effective dynamics of the ``gluon'' field (the quotes allude to the fact that we deal with SU(2) and refer to $t^{a}_{\mu}$), let us point out some new perspectives on the pure f dynamics.\\

First, the class of classical configurations, the spherically symmetric configurations $\tilde{f}^{a}(x)$, not only satisfies
\begin{equation}\label{7}
\frac{\delta S_{f}}{\delta f^{a}(x)}\vert_{\tilde{f}}=0,
\end{equation}
but also
\begin{equation}\label{8}
Z^{a}_{\mu\nu}(\tilde{f}) = 0,
\end{equation}
where
\begin{equation}\label{9}
S_{f} = \int d^{4}x\frac{1}{4}Z^{a}_{\mu\nu} Z^{a}_{\mu\nu}.
\end{equation}
Equations (l3,14) follow from the fact that for spherically symmetric fields $\partial_{\mu}\tilde{f}^{a} = \frac{x_{\mu}}{x}\frac{d\tilde{f}^{a}}{dx}$ and $X^{abc}$ is anti-symmetric with respect to b and c.  This means that $S_{f}$ has a broad minimum in the function space of the scalars.  It also means that all the spherically symmetric scalars are elements of the classical vacuum.  It is not apparent if the spherically symmetric functions exhaust all classical configurations or the classical vacuum.\\

What is intriguing about the result of reference (4) is that the linear potential follows from a stochastic treatment of vacuum configurations with zero field strength $Z^{a}_{\mu\nu}$.  Is there a simple way to understand this result?  One argument is that the stochastic vacuum configurations  reduce the dimension of space-time from 4 to 2 resulting in a linear potential.  This was confirmed in reference (6) where the full quantum dynamics of $f^{a}$ was shown to be equivalent to a 2D O(1,3) non-linear sigma model.\\

The second comment has to do with the 2D O(1,3) non-linear sigma model referred to in the previous section.  Its action is given by
\begin{equation}\label{10}
S_{f} = \int d^{2}x [\frac{1}{2}\partial_{\mu}\phi\partial_{\mu}\phi-\frac{1}{2}\partial_{\mu}f^{a}\partial_{\mu}f^{a}],
\end{equation}
with
\begin{equation}\label{11}
\phi^{2} - \vec{f}\cdot\vec{f} = 1.
\end{equation}

This hints of string physics, albeit only a $3+1$ string $(\phi \sim x_{o}, \vec{f}\sim\vec{x})$ with unit length.  Also, this string is automatically Lorentz invariant because of O(1,3) symmetry.  In essence, the full $f^{a}$ dynamics suggests a Lorentz invariant $3+1$ string theory.

\section{The ``gluon'' in the vacuum defined by scalars}

In equation (2), we will consider the ``gluon'' $t^{a}_{\mu}$ in the vacuum defined by the spherically symmetric scalars $\tilde{f}^{a}(x)$, where $x=(x\cdot x)^{\frac{1}{2}}$.  The resulting expansion is not the usual background decomposition given by
\begin{equation}\label{12}
A^{a}_{\mu} = \tilde{A}^{a}_{\mu}[\tilde{f}]+t^{a}_{\mu},
\end{equation}
as can be seen from equation (2) because $t^{a}_{\mu}$ is coupled in a non-trivial way with $\tilde{f}^{a}(x)$.  Equation (3) was used in reference (5) to derive the linear potential by assuming a white-noise distribution for the spherically symmetric $\tilde{f}^{a}(x)$.  Equation (4) on the other hand, will lead to the gauge condition
\begin{equation}\label{13}
x_{\mu}t^{a}_{\mu} = 0,
\end{equation}
which is known in the literature as the radial gauge\cite{linear},\cite{string}.

Using equations (5) to (12), we find that the effective dynamics of $t^{a}_{\mu}$ in the vacuum defined by $\tilde{f}^{a}(x)$ is given by the action
\begin{eqnarray}\label{13}
S_{eff}(t^{a}_{\mu}, \tilde{f}^{a}) &=&\frac{1}{4}\int d^{4}x\{\mathbb{R}^{ab}(\tilde{f})(\partial_{\mu}t^{a}_{\nu}-\partial_{\nu}t^{a}_{\mu})(\partial_{\mu}t^{b}_{\nu}-\partial_{\nu}t^{a}_{\mu})\nonumber\\
&+& \mathbb{S}^{ab}(\tilde{f})(\partial_{\mu}t^{a}_{\nu}-\partial_{\nu}t^{a}_{\mu})(\frac{x_{\mu}}{x}t^{b}_{\nu}-\frac{x_{\nu}}{x}t^{b}_{\mu})\nonumber\\
&+&2\mathbb{Y}^{ab}(\tilde{f})t^{a}_{\mu}t^{b}_{\mu}+\mathcal{\mathbb{U}}^{abc}(\partial_{\mu}t^{a}_{\nu}-\partial_{\nu}t^{a}_{\mu})t^{b}_{\mu}t^{c}_{\nu}\nonumber\\
&+&\mathbb{T}^{abcd}(\tilde{f})t^{a}_{\mu}t^{b}_{\nu}t^{c}_{\mu}t^{d}_{\nu}\},
\end{eqnarray}
where we used equation (19) and the coefficients are given by
\begin{eqnarray}\label{14}
\mathbb{R}^{ab}(\tilde{f}) &=& R^{ca}(\tilde{f})R^{cb}(\tilde{f}),\\
\mathbb{S}^{ab}(\tilde{f}) &=& R^{ca}Y^{cdb}\frac{d\tilde{f}^{d}}{dx},\\
\mathbb{Y}^{ab}(\tilde{f}) &=& Y^{cda}(\tilde{f})Y^{ceb}(\tilde{f})\frac{df^{d}}{dx}\frac{df^{e}}{dx},\\
\mathbb{U}^{abc}(\tilde{f}) &=& R^{da}(\tilde{f})T^{dbc}(\tilde{f}),\\
\mathbb{T}^{abcd} &=& T^{eab}(\tilde{f})T^{ecd}(\tilde{f}).
\end{eqnarray}

From equation (20), the equation of motion for $t^{a}_{\mu}$ is:
\begin{eqnarray}\label{15}
\frac{\delta S_{eff}}{\delta t^{a}_{\mu}(x)}&=& R^{ab}(\tilde{f})\Box^{2}t^{b}_{\mu}+\mathbb{M}^{ab}(\tilde{f})t^{b}_{\mu}+\mathbb{N}^{ab}_{\mu\nu}(\tilde{f},x,\partial)t^{b}_{\nu}+\mathbb{P}^{a}_{\mu}(\tilde{f},x)\nonumber\\
&+&\mathbb{U}^{abc}(\tilde{f})\partial_{\nu}t^{b}_{\mu}t^{c}_{\nu}-\frac{1}{2}\mathbb{U}^{abc}(\tilde{f})(\partial_{\mu}t^{b}_{\nu})t^{c}_{\nu}\nonumber\\
&+&\frac{1}{2}\mathbb{T}^{abcd}t^{b}_{\nu}(t^{c}_{\mu}t^{d}_{\nu}-t^{c}_{\nu}t^{d}_{\mu}) = 0,
\end{eqnarray}
where
\begin{eqnarray}\label{16}
\mathbb{M}^{ab}(\tilde{f}) &=& \mathbb{Y}^{ab}+\frac{1}{2}\mathbb{U}^{abc}(\frac{1}{g\ell^{2}})\tilde{f}^{c}-\frac{1}{2}\frac{\delta \mathbb{S}^{ab}}{\delta\tilde{f}^{d}}\frac{d\tilde{f}^{d}}{dx},\\
\mathbb{N}^{ab}_{\mu\nu}(\tilde{f},x,\partial) &=& -\frac{\delta\mathbb{R}^{ab}}{\delta\tilde{f}^{c}}\frac{d\tilde{f}^{c}}{dx}(\frac{1}{x})x_{\nu}\partial_{\mu}+\frac{1}{2}(\mathbb{S}^{ba}-\mathbb{S}^{ab})(\frac{1}{x})\delta_{\mu\nu}(x\cdot\partial)\nonumber\\
&+&(\frac{1}{2}\mathbb{S}^{ba}-\frac{3}{2}\mathbb{S}^{ab})(\frac{1}{x})\delta_{\mu\nu}+\frac{1}{2}\mathbb{S}^{ab}(\frac{1}{x^{2}})\delta_{\mu\nu},\\
\mathbb{P}^{a}_{\mu}(\tilde{f}) &=& -\frac{1}{g\ell^{2}}\mathbb{R}^{ab}\frac{d\tilde{f}^{b}}{dx}(\frac{x_{\mu}}{x})+\frac{1}{2}(\frac{1}{g\ell^{2}})\mathbb{S}^{ab}\tilde{f}^{b}(\frac{x_{\mu}}{x}).
\end{eqnarray}
To derive equations (26) to (29), integration by parts and equation (3) were used.  The last three terms are interaction terms of the vector fields in the presence of the $\tilde{f}$.  The first three terms give the propagation of the ``gluon'' in the vacuum defined by $\tilde{f}$.  The fourth is $t^{a}_{\mu}$ independent and acts like a source term.\\

At this point, the dynamics seem unwieldy.  However, since all spherically symmetric $\tilde{f}^{a}(x)$ are vacuum solutions, we should average over all these configurations.  As in reference (4), we use the white-noise distribution and compute
\begin{equation}\label{17}
\mathcal{N}^{-1}\int(d\tilde{f}^{a}(x))\frac{\delta S_{eff}}{\delta t^{a}_{\mu}(x)}e^{-\frac{1}{\ell}\int^{\infty}_{0}ds\tilde{f}^{a}(s)\tilde{f}^{a}(s)} = 0.
\end{equation}
where $\ell$ was introduced for dimensional reasons.\\

We will discretize the integral in the exponent by
\begin{equation}\label{18}
\int^{\infty}_{0}ds\tilde{f}^{a}(s)\tilde{f}^{a}(s) = \sum_{s}\tilde{f}^{a}(s)\tilde{f}^{a}(s)\Delta s.
\end{equation}
For each s, the integration in $\tilde{f}(s)$ becomes ordinary integrals involving
\begin{eqnarray}\label{19}
K^{(4)} &=& N^{-1}\int^{\infty}_{0}r^{4}e^{-\sigma r^{2}}=\frac{3}{2}(\pi\sigma)^{-1},\\
I^{(n,m)} &=& N^{-1}\int^{\infty}_{0}\frac{(r^{2})^{m}}{(1+r^{2})^{n}}e^{-\sigma r^{2}}dr,
\end{eqnarray}
where m and n are integers, $\sigma = \frac{\Delta s}{\ell}$ and $N = \pi^{\frac{3}{2}}\sigma^{-\frac{3}{2}}$ is the normalization factor, the products of which for all s values give the overall normalization $\mathcal{N}$ found in equation (30).  Equation (33) is evaluated using integrals derived from (by differentiating w.r.t. $\beta$ or $\sigma$)
\begin{equation}\label{20}
\int^{\infty}_{0}\frac{e^{-\sigma r^{2}}}{(r^{2}+\beta^{2})}dr = [1-\Phi(\sigma^{\frac{1}{2}}\beta)\frac{\pi}{2\beta}e^{\beta^{2}\sigma}],
\end{equation}
where $Re\beta > 0$ and arg $\sigma<\frac{\pi}{4}$ (see reference\cite{integral}). The function $\Phi$ is the error function given by
\begin{eqnarray}\label{21}
\Phi(\sigma^{\frac{1}{2}}\beta) &=& \frac{2}{\sqrt{\pi}}\int^{\beta\sigma\frac{1}{2}}_{0}e^{-t^{2}}dt,\nonumber\\
&=&\frac{2}{\sqrt{\pi}}[\beta\sigma^{\frac{1}{2}}-\frac{1}{3}(\beta\sigma^{\frac{1}{2}})^{3}+\frac{1}{10}(\beta\sigma^{\frac{1}{2}})^{5}-\frac{1}{42}(\beta\sigma^{\frac{1}{2}})^{7}+...].
\end{eqnarray}
The evaluation of the averages of the terms given by equations (21) to (29) is  simplified by assuming that $\tilde{f}^{a}(x)$ has continuous derivatives at least up to the second order (because $Z^{a}_{\mu\nu}\sim\frac{d^{2}\tilde{f}^{a}}{dx^{2}}$).  This enables us to write
\begin{equation}\label{22}
\frac{d\tilde{f}^{a}}{dx}=\lim_{\Delta x\to 0}\frac{1}{2}\left\{\frac{[\tilde{f}^{a}(x+\Delta x)-\tilde{f}^{a}(x)]+[\tilde{f}^{a}(x)-\tilde{f}^{a}(x-\Delta x)]}{\Delta x}\right\}
\end{equation}

After a tedious computation, we find
\begin{eqnarray}\label{23}
\langle\mathbb{P}^{a}_{\mu}(\tilde{f})\rangle_{\tilde{f}}&=&\langle\mathbb{N}^{ab}_{\mu\nu}(\tilde{f})\rangle_{\tilde{f}}=\langle\mathbb{U}^{abc}(\tilde{f})\rangle_{\tilde{f}}=\langle\mathbb{T}^{abcd}(\tilde{f})\rangle_{\tilde{f}}=0,\\
\langle\mathbb{R}^{ab}\rangle_{\tilde{f}}&=&\frac{\pi^{\frac{1}{2}}}{6}\delta^{ab},\\
\langle\mathbb{M}^{ab}\rangle_{\tilde{f}}&=&(\frac{5}{6})\frac{1}{\ell^{2}\sigma^{2}}\delta^{ab}.
\end{eqnarray}
These results follow in the limit $\sigma\to 0$.  Equation (37) says that the ``gluons'' are no longer self-interacting and are not driven by an external force.  Equations (38) and (39) say that the ``gluons'' develop a mass
\begin{equation}\label{24}
m_{g}=(\frac{5}{\pi\frac{1}{2}})^{\frac{1}{2}}(\frac{1}{\ell\sigma}).
\end{equation}
As it stands, this mass is divergent as $\sigma\to 0$.

To make sense of this result, note that the divergence as $\sigma\to 0$ is a continuum effect, i.e., we are considering the action of an infinite number of classical configurations $\tilde{f}^{a}(x)$ on the ``gluons''.  To see how to get a finite result, let us look at the mass term.  Carrying out the averaging on $\mathbb{M}^{ab}$ yields
\begin{eqnarray}\label{25}
\langle\mathbb{M}^{ab}\rangle_{\tilde{f}} 
& = &\delta^{ab}
   \left[
     \frac{ \pi^{ -\frac{1}{2} } }{ 3\ell^{2} }
     \sigma^{ -\frac{3}{2} }
   \right]
   \left\{ 5\int^{\infty}_{0}e^{-\sigma r^{2} }dr-                                   \frac{406}{9}\int^{\infty}_{0}\frac{ e^{-\sigma r^{2}} }{(1+r^2)}dr
   \right. \nonumber\\
& + &
   \left.
     \frac{1094}{9}\int^{\infty}_{0}\frac{e^{-\sigma r^2}}{(1+r^2)^2}dr -                    \frac{779}{9}\int^{\infty}_{0} \frac{e^{-\sigma r^2}}{(1+r^2)^3}dr +                                         \frac{46}{9}\int^{\infty}_{0}  \frac{e^{-\sigma r^2}}{(1+r^2)^4}dr
   \right\} \nonumber\\
& = &\frac{\delta^{ab}}{\ell^2\sigma^2}(\frac{5}{6})
   \left\{1+\frac{1758}{720}\pi^{\frac{1}{2}}
     \sigma^{\frac{1}{2}}+\frac{812}{45}\sigma-          \frac{1101}{36}\pi^{\frac{1}{2}}\sigma^{\frac{3}{2}}+\frac{892}{15}\sigma^{2}
   \right\}.
\end{eqnarray}
Note that $r^{2}$ above actually represents $\tilde{f}^{a}(x)\tilde{f}^{a}(x)$ for each space-time point $x_{\mu}$ with $x=(x_{\mu}x_{\mu})^{\frac{1}{2}}$.  Since we are accounting for an infinite number of classical configurations, we can only get a finite value for the mass by limiting the r integral to finite value $\Lambda$.


Note that $\Lambda$ is dimensionless like $\sigma$ and limiting $r=(\tilde{f}\cdot\tilde{f})^{\frac{1}{2}}$ to a sphere of radius $\Lambda$ limits the vacuum configurations that influence the ``gluon'' dynamics.  Aside from changing the upper limit of the integrals in equation (41) from $\infty$ to $\Lambda$, we also change $N(\Lambda) =\frac{4}{3}\pi\Lambda^{3}$ and $K^{(4)}=\frac{3}{5}\Lambda^{2}$.  The resulting integrals are:
\begin{eqnarray}\label{26}
\mathbb{R}^{ab}
&=& \delta^{ab}(\frac{3}{\Lambda^{3}})
 [\frac{2}{3}\Lambda-\frac{2}{3}\arctan\Lambda +\frac{1}{9}\Lambda^{3}]
 +O(\sigma)...\nonumber\\
\mathbb{M}^{ab}
&=& \delta^{ab}(\frac{1}{\ell^{2}\sigma^{2}})(\frac{9}{5})(\frac{1}{\Lambda})
  \left\{
    5\int^{\Lambda}_{0}e^{-\sigma r^{2}}dr
    -\frac{406}{9}\int^{\Lambda}_{0}\frac{e^{-\sigma r^{2}}}{(1+r^2)}dr
  \right.\nonumber\\
&+& \frac{1094}{9}\int^{\Lambda}_{0}\frac{e^{-\sigma r^{2}}}{(1+r^{2})^{2}}dr-  \frac{779}{9}\int^{\Lambda}_{0}\frac{e^{-\sigma r^{2}}}{(1+r^{2})^{3}}dr
\nonumber\\
&+&
 \left.
   \frac{46}{9}\int^{\Lambda}_{0}\frac{e^{-\sigma r^{2}}}{(1+r^{2})^{4}}dr
 \right\}.
\end{eqnarray}
The $\frac{1}{\ell^{2}\sigma^{2}}$ factor in equation (42) comes from the fact that $\mathbb{M}^{ab}\sim(\frac{d\tilde{f}}{ds})^{2}$, which when discretized gives $(\frac{1}{\Delta s})^{2}$.  Extracting the most divergent term $(\sim\sigma^{-2})$, we find that the ``gluon'' mass is given by
\begin{equation}\label{27}
m_{g}
 =(\frac{3}{5})^{\frac{1}{2}}(\frac{1}{\ell\sigma})\frac{\Lambda a(\Lambda)}{b(\Lambda)},
\end{equation}
where
\begin{eqnarray}\label{28}
a(\Lambda)
&=&
   \left\{
     - \frac{1094}{72}(tan^{-1}\Lambda-n\pi)+5\Lambda 
     + \frac{361}{18}\frac{\Lambda}{1+\Lambda^{2}}
   \right.\nonumber\\
&-&
  \left.
     \frac{710}{72}(\frac{\Lambda}{1+\Lambda^{2}})(\frac{1-\Lambda^{2}}{1+\Lambda^{2}}) 
     -\frac{46}{56}\frac{\Lambda^{3}}{(1+\Lambda^{2})^{3}}
  \right\} ^{\frac{1}{2}},\\
b(\Lambda)
&=&
  \left[
     \frac{2}{3}\Lambda-\frac{2}{3}(tan^{-1}
     \Lambda-n\pi)+\frac{1}{9}\Lambda^{3}
  \right] ^{\frac{1}{2}},
\end{eqnarray}
and n = 0, 1, 2....  Taking $\ell$ to be of the order of the confinement length ($\sim$ fermi) and using the ``experimental'' value of $m_{g}$, we can solve equation (44) for $\Lambda=\Lambda(\sigma)$.  The lowest non-trivial value of $\Lambda(0)$ will be determined from $a(\Lambda)$ using either n = 0 or n = 1.  The other n values represent the other states of the vector field and the spacings of these states easily follow from equations (44, 45, 46).\\

Finally, we note that even in the finite function space of $\tilde{f}^{a}(x)$ (restricted to the sphere $\Vert\tilde{f}\Vert = \Lambda^{\frac{1}{2}})$, the ``gluons'' still lose their self-interactions.  This is because $\mathbb{U}^{ab}$ and $\mathbb{T}^{abcd}$ are functions only of $\tilde{f}^{a}(x)$ and not $\frac{d\tilde{f}^{a}}{dx}$.  The resulting averages will not have a $\sigma^{-2}$ factor.  Since the integrals are of the same terms as those found in $\mathbb{M}^{ab}$, once we fix the $m^{2}_{g}$ value, the $\langle\mathbb{U}^{abc}\rangle_{\tilde{f}}$ and $\langle\mathbb{T}^{abcd}\rangle_{\tilde{f}}$ go to zero by factor of $\sigma^{2}$.

\section{Conclusion}

In this paper, we argued that the class of spherically symmetric $\tilde{f}^{a}(x)$ (in $R^{4}$) form vacuum configurations with zero action.  Then it was shown that by treating $\tilde {f}^{a}(x)$ as stochastic, the effective dynamics of the ``gluons'' in this background result in the vanishing of the ``gluon'' self-interaction and generation of its mass.\\

It is apparent that the class of spherically-symmetric zero field strength $(Z^{a}_{\mu\nu}=0)$ vacuum configurations that produced the linear potential and ``gluon'' mass is in disagreement with current ideas on the nature of the QCD vacuum.  In the 1970's and early 80's, Pagels and Tomboulis\cite{Pagels}, and Savvidy\cite{Sav} showed that a non-zero field strength configuration has lower energy than the zero field strength configuration.  Unfortunately, the effective Hamiltonian has an imaginary part which signals instability and eventual decay to the trivial vacuum.\\

About the same time the concept of condensates\cite{Shifman}, (in particular, the $\langle 0\mid F^{2}\mid 0\rangle\neq 0)$, supports the idea of a vacuum with non-zero field strength.  Obviously, the ideas presented here and the condensate result are contradictory.  Or are they?\\

Note that the decomposition of the original $A^{a}_{\mu}$ given in equation (2) shows the scalar field $f^{a}$ inextricably linked to the $t^{a}_{\mu}$ and not in the usual background decomposition $A^{a}_{\mu} = \tilde {A}^{a}_{\mu} + t^{a}_{\mu}$.  When we expand the field strength, we get equation (5), which shows $t^{a}_{\mu}$ interacting with f in a non-linear manner.  The vacuum configurations defined by $f^{a}$, which are the class of spherically symmetric configurations $\tilde{f}^{a}(x)$, have vanishing $Z^{a}_{\mu\nu}$.  The non-zero condensate $\langle 0\vert F^{2}\vert 0\rangle$ involves $L^{a}_{\mu\nu}$ and $Q^{a}_{\mu\nu}$, i.e., it also reflects the $t^{a}_{\mu}$ dynamics.  Thus, the ideas presented here and that of the non-vanishing condensate are not necessarily contradictory.\\

Finally, we make the following observation about the non-linear gauge.  The non-linear gauge condition, as already stated, interpolates between the Coulomb gauge (with weakly interacting transverse gluons at the short-distance regime) and the quadratic regime (with $f^{a}$ and $t^{a}_{\mu}$ as degrees of freedom, which accounts for confinement and the mass for the vector field in the long-distance regime).  This assumes that the coupling runs, implying that the non-linear gauge is essentially a quantum gauge condition.  This is an unusual way to fix the gauge.  Normally, the gauge condition represents a fixed submanifold in configuration space regardless of the distance scale of the physics.  The non-linear gauge, on the other hand, defines a quadratic submanifold, which, depending on the coupling, will highlight either the transverse degrees of freedom of the vector field or the scalar $f^{a}$ and the vector field $t^{a}_{\mu}$.  At this stage, this is put in by hand.  It would be interesting to derive this consistently along with the running of the coupling.

\begin{thebibliography}{99}
\bibitem{gauge} Jose A. Magpantay, {Progress of Theoretical Physics} \textbf{91}, 573 (1994).
\bibitem{grav} J.A. Magpantay and E. Cuansing, Jr., {Modern Physics Letters} \textbf{A11}, 87 (1996).
\bibitem{low} Jose A. Magpantay, in ``Mathematical Methods of Quantum Physics:  Essays in Honor of Hiroshi Ezawa'', C.C. Bernido, et. al. Eds., Gordon and Breach Science Publishers, 241 (1999).
\bibitem{mass}	Jose A. Magpantay, {Modern Physics Letters} \textbf{A14}, 442 (1999).
\bibitem{free}	Jose A. Magpantay, {International Journal of Modern Physics} \textbf{A15}, 1613 (2000).
\bibitem{linear}	V.A. Fock, {Sov. Phys.} \textbf{12}, 404 (1937).
\bibitem{string} J. Schwinger, {Phys. Rev.} \textbf{182}, 684 (1952).
\bibitem{integral} I.S. Gradsteyn and I.M. Ryzhik, Tables of Integrals, Series and Products, Academic Press, p. 338 (1965).
\bibitem{Pagels} H. Pagels and E. Tomboulis, {Nuclear Physics} \textbf{B143}, 485 (1978).
\bibitem{Sav} G.K. Savvidy, {Phys. Letters} \textbf{B71}, 133 (1977).
\bibitem{Shifman} M. Shifman, A. Vainshtein and V. Zakharov, {Nuclear Physics} \textbf{B147}, 385, 448 (1979).
\end{thebibliography}
\end{document}


